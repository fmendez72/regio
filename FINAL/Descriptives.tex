\documentclass[]{article}
\usepackage{lmodern}
\usepackage{amssymb,amsmath}
\usepackage{ifxetex,ifluatex}
\usepackage{fixltx2e} % provides \textsubscript
\ifnum 0\ifxetex 1\fi\ifluatex 1\fi=0 % if pdftex
  \usepackage[T1]{fontenc}
  \usepackage[utf8]{inputenc}
\else % if luatex or xelatex
  \ifxetex
    \usepackage{mathspec}
  \else
    \usepackage{fontspec}
  \fi
  \defaultfontfeatures{Ligatures=TeX,Scale=MatchLowercase}
\fi
% use upquote if available, for straight quotes in verbatim environments
\IfFileExists{upquote.sty}{\usepackage{upquote}}{}
% use microtype if available
\IfFileExists{microtype.sty}{%
\usepackage{microtype}
\UseMicrotypeSet[protrusion]{basicmath} % disable protrusion for tt fonts
}{}
\usepackage[margin=1in]{geometry}
\usepackage{hyperref}
\hypersetup{unicode=true,
            pdftitle={Appendix 5: List of sources per country},
            pdfborder={0 0 0},
            breaklinks=true}
\urlstyle{same}  % don't use monospace font for urls
\usepackage{graphicx,grffile}
\makeatletter
\def\maxwidth{\ifdim\Gin@nat@width>\linewidth\linewidth\else\Gin@nat@width\fi}
\def\maxheight{\ifdim\Gin@nat@height>\textheight\textheight\else\Gin@nat@height\fi}
\makeatother
% Scale images if necessary, so that they will not overflow the page
% margins by default, and it is still possible to overwrite the defaults
% using explicit options in \includegraphics[width, height, ...]{}
\setkeys{Gin}{width=\maxwidth,height=\maxheight,keepaspectratio}
\IfFileExists{parskip.sty}{%
\usepackage{parskip}
}{% else
\setlength{\parindent}{0pt}
\setlength{\parskip}{6pt plus 2pt minus 1pt}
}
\setlength{\emergencystretch}{3em}  % prevent overfull lines
\providecommand{\tightlist}{%
  \setlength{\itemsep}{0pt}\setlength{\parskip}{0pt}}
\setcounter{secnumdepth}{0}
% Redefines (sub)paragraphs to behave more like sections
\ifx\paragraph\undefined\else
\let\oldparagraph\paragraph
\renewcommand{\paragraph}[1]{\oldparagraph{#1}\mbox{}}
\fi
\ifx\subparagraph\undefined\else
\let\oldsubparagraph\subparagraph
\renewcommand{\subparagraph}[1]{\oldsubparagraph{#1}\mbox{}}
\fi

%%% Use protect on footnotes to avoid problems with footnotes in titles
\let\rmarkdownfootnote\footnote%
\def\footnote{\protect\rmarkdownfootnote}

%%% Change title format to be more compact
\usepackage{titling}

% Create subtitle command for use in maketitle
\newcommand{\subtitle}[1]{
  \posttitle{
    \begin{center}\large#1\end{center}
    }
}

\setlength{\droptitle}{-2em}

  \title{Appendix 5: List of sources per country}
    \pretitle{\vspace{\droptitle}\centering\huge}
  \posttitle{\par}
    \author{}
    \preauthor{}\postauthor{}
    \date{}
    \predate{}\postdate{}
  
\usepackage{booktabs}
\usepackage{longtable}
\usepackage{array}
\usepackage{multirow}
\usepackage{wrapfig}
\usepackage{float}
\usepackage{colortbl}
\usepackage{pdflscape}
\usepackage{tabu}
\usepackage{threeparttable}
\usepackage{threeparttablex}
\usepackage[normalem]{ulem}
\usepackage{makecell}
\usepackage{xcolor}

\begin{document}
\maketitle

\subsubsection{News sites per country}\label{news-sites-per-country}

\begin{tabular}{lllllllllllllllllllllllllrlllllllrllllrrrrrrrrllllll}
\toprule
country & url & id & site & sourceType & sourceFormat & sourceLevel & CPvisibility & subFrame1 & tone1 & actor1 & myth1 & subFrame2 & tone2 & actor2 & myth2 & subFrame3 & tone3 & actor3 & myth3 & Country & title & date & searchTerm & body & wordN & vis & vis2 & f1 & f2 & f3 & date2 & year & year2 & dim & vis4 & level & words & net\_bal\_pc & cp\_pc & equi17 & bad\_thing17 & selfrule & sharedrule & RAI & goodMu & Net & Actor1 & Actor2 & Actor3 & dimActor & Tone\\
\midrule
Poland & https://expressbydgoski.pl/2-miliardy-przepadna-bez-porozumienia-bydgoszczy-i-torunia-pieniedzy-z-zit-nie-bedzie/ar/c3-10882944 & a25 & https://expressbydgoski.pl/ & Private/Non-Public & Online and Offline & Regional/Local & medium = CP is important part of story & Poor communication of funding/rules & Negative & Subnational & No myth & NA & NA & NA & NA & NA & NA & NA & NA & Poland & 2 miliardy przepadną. bez porozumienia bydgoszczy i torunia pieniędzy z zit nie będzie & 2014-03-21 & fundusze strukturalne & -oba ciała pochwaliły sposób przygotowania regionalnego programu operacyjnego kujawsko-pomorskiego. jednak ostatecznie program otrzymał negatywną ocenę. jak pisze marszałek całbecki: „ze względu na brak porozumienia w zakresie strategii rozwoju i sposobu jej realizacji w ramach zintegrowanych inwestycji terytorialnych dla bydgosko-toruńskiego obszaru metropolitalnego”.piotr całbecki w swoim oświadczeniu podkreśla, że brak porozumienia może spowodować utratę 153 milionów euro, przeznaczonych na zit, ale też uniemożliwi skorzystanie ze znacznie większych pieniędzy w ramach rządowych programów operacyjnych. ponadto może to spowodować zamrożenie dalszych prac nad naszym rpo i realną groźbę utraty całej kwoty naszego rpo
-jednocześnie apelujemy o rozsądek, do tych, którzy dziś blokują możliwość wykorzystania szansy na pozyskanie ogromnych środków finansowych w ramach nowego budżetu i nowej perspektywy finansowej unii europejskiej”
-rada miasta torunia apeluje do prezydenta miasta bydgoszczy oraz do rady miasta bydgoszczy o niezwłoczne zaprzestanie działań hamujących proces uzgodnienia partnerskich warunków funkcjonowania wspólnego zit wojewódzkiego” & 145 & medium & Medium & Governance & NA & NA & 2014-03-21 & 2014 & 1 & POL
Frame & low-medium & Regional & <500 & 1.0448136 & 1.8253861 & -0.2867631 & -1.2581601 & -1.2725643 & 0.0 & -1.1973642 & 0.9927066 & Recipient & Domestic & Domestic & Domestic & Domestic|POL & Negative\\
Poland & https://nto.pl/powiat-nyski-ile-zyskal-na-czlonkostwie-w-unii-europejskiej/ar/4615583 & a49 & https://nto.pl & Private/Non-Public & Online and Offline & Regional/Local & high = CP is most important issue in story (can also cover other issues) & Economic development & Positive & Subnational & No myth & NA & NA & NA & NA & NA & NA & NA & NA & Poland & powiat nyski - ile zyskał na członkostwie w unii europejskiej & 2014-05-18 & fundusze strukturalne & -150 mln złotych pozyskanych z funduszy strukturalnych, najwięcej zrealizowanych zadań na pograniczu polsko-czeskim i wyróżnienie za projekt siedem gwiazd - to efekty aktywności powiatu przez 10 lat obecności w ue.
-przez te dziesięć lat naprawdę sporo udało się zrobić dzięki unijnemu wsparciu i aż strach pomyśleć, jak wyglądałyby nasze wioski i miasta, gdyby nie akces do europy - mówi edyta bednarska-kolbiarz, rzecznik nyskiego starostwa.
-najważniejsze, że efekty widać gołym okiem. przykładem jest nyski szpital, który na rozbudowę i remonty pozyskał aż 16 mln zł. nie wspominając o zakupionym za unijne dotacje sprzęcie i wyposażeniu.
-bez europejskich pieniędzy kiepsko wyglądałyby dziś również wszystkie budynki szkolne i drogi. z unijnej kasy udało się wyremontować i przebudować m.in.: skrzyżowanie przy ulicy mickiewicza i rodziewiczówny w nysie. na oświatę i drogi unia dołożyła przeszło 20 mln zł. & 134 & high & High & Socio-Economic & NA & NA & 2014-05-18 & 2014 & 1 & ECO
Frame & high-very high & Regional & <500 & 1.0448136 & 1.8253861 & -0.2867631 & -1.2581601 & -1.2725643 & 0.0 & -1.1973642 & 0.9927066 & Recipient & Domestic & Domestic & Domestic & Domestic|ECO & Positive\\
Poland & http://kurierzurominski.pl/203785,Powiat-10-lat-w-Unii-dalo-projekty-za-22-miliony.html & a42 & http://kurierzurominski.pl & Private/Non-Public & Online and Offline & Regional/Local & low = CP mentioned more times but NOT important part of story (mainly about others issues) & Jobs & Factual & Subnational & No myth & NA & NA & NA & NA & NA & NA & NA & NA & Poland & powiat. 10 lat w unii dało projekty za 22 miliony & 2014-05-19 & europejski fundusz społeczny & -18 projektów, 2 693 zaktywizowanych bezrobotnych, 22 miliony złotych to bilans korzyści, jakie członkostwa w unii europejskiej mają bezrobotni z powiatu żuromińskiego. liczby te, to efekt projektów realizowanych przez powiatowy urząd pracy w żurominie z funduszy europejskich.
-środki unijne dają szanse na pomoc bezrobotnym, która w takim wymiarze nie byłaby możliwa z środków z funduszu pracy – mówi dyrektor powiatowego urzędu pracy w żurominie joanna hajdas – realizacja takich projektów nie jest jednak łatwą sprawą, ale jest ciekawym doświadczeniem – dodaje. & 78 & low & Low & Socio-Economic & NA & NA & 2014-05-19 & 2014 & 1 & ECO
Frame & low-medium & Regional & <500 & 1.0448136 & 1.8253861 & -0.2867631 & -1.2581601 & -1.2725643 & 0.0 & -1.1973642 & 0.9927066 & Recipient & Domestic & Domestic & Domestic & Domestic|ECO & Neutral\\
Poland & https://fakty.interia.pl/raporty/raport-wybory-samorzadowe-2018/aktualnosci/news-sasin-o-wyborach-samorzadowych-duze-miasta-nie-sa-najwazniej,nId,2645509 & 288 & fakty.interia.pl & Private/Non-Public & Online only & National & very low = CP mentioned once & Institutional bargaining over funding & Balanced & National & No myth & NA & NA & NA & NA & NA & NA & NA & NA & Poland & sasin o wyborach samorządowych: duże miasta nie są najważniejsze & 2018-10-17 & polityka regionalna & "duże miasta nie są najważniejsze w wyborach samorządowych. najważniejsza jest walka o sejmiki wojewódzkie" - powiedział w środę szef komitetu stałego rady ministrów jacek sasin. zaznaczył, że w 15 na 16 sejmików rządzą koalicje po-psl czasami wspierane przez lewicę. jego zdaniem "trzeba to zmienić". jacek sasin w programie kwadrans polityczny w tvp ocenił, że najważniejsza w nadchodzących wyborach samorządowych jest walka o sejmiki wojewódzkie. "duże miasta nie są w tych wyborach najważniejsze. w tych wyborach najważniejsza jest walka o sejmiki wojewódzkie, o to kto będzie rządził w sejmikach wojewódzkich" - wskazał sasin. reklama zwrócił uwagę, że w to sejmiki wojewódzkie decydują m.in. o podziale funduszy europejskich. "jeśli chodzi o realną władzę w polsce, o to jak będzie wyglądała polityka regionalna, jak będą rozdysponowane ogromne pieniądze, które mają sejmiki, fundusze europejskie - to się decyduje w sejmikach. dzisiaj w 15 na 16 sejmików rządzą koalicje po-psl, czasami wspierane przez lewicę. to trzeba zmienić" - zaznaczył minister. sasin był też pytany, w ilu z największych miast w polsce wygra zjednoczona prawica. "nie chcę budować daleko idących oczekiwań. duże miasta to nie jest pole, gdzie kandydaci pis są faworytami. nie są faworytami" - zaznaczył. sasin wyraził też nadzieję, że "we wszystkich największych miastach dojdzie do drugiej tury i że w drugiej turze rzeczywiście trzeba będzie wybierać pomiędzy kontynuującą a zmianą". minister dodał, że dzisiaj ten wybór jest zaburzony. "jest on zaburzony poprzez to, że jest wielu różnych innych kandydatów, którzy składają różne propozycje, ale zasadniczo pewnie nie mają dużej szansy, żeby osiągnąć sukces w tych wyborach" - dodał. wybory samorządowe odbędą się 21 października. dwa tygodnie później, 4 listopada przeprowadzona zostanie druga tura głosowania tam, gdzie w pierwszej nie zostanie wybrany wójt, burmistrz lub prezydent miasta. do drugiej tury przechodzi dwóch kandydatów, którzy otrzymali najlepszy wynik. wybieranych będzie blisko 47 tys. radnych gmin, powiatów i sejmików wojewódzkich oraz blisko 2,5 tys. wójtów, burmistrzów i prezydentów miast. & 311 & very low & Low & Power & NA & NA & 2018-10-17 & 2018 & 3 & POL
Frame & v.low & National & <500 & 1.0448136 & 1.8253861 & -0.2867631 & -1.2581601 & -1.2725643 & 0.0 & -1.1973642 & 0.9927066 & Recipient & Domestic & Domestic & Domestic & Domestic|POL & Neutral\\
Poland & http://forsal.pl/artykuly/994440,poznan-modertrans-zaprezentowal-nowoczesny-niskopodlowy-tramwaj.html & 425 & forsal.pl & Private/Non-Public & Online only & National & very low = CP mentioned once & Infrastructure & Factual & Subnational & No myth & NA & NA & NA & NA & NA & NA & NA & NA & Poland & poznań: modertrans zaprezentował nowoczesny, niskopodłowy tramwaj & 2016-11-18 & europejski fundusz rozwoju regionalnego & nowoczesny, w pełni niskopodłogowy tramwaj moderus gamma, mogący pomieścić 244 pasażerów, zaprezentowano w piątek w podpoznańskich biskupicach. pojazd wyprodukowany przez firmę modertrans już w przyszłym miesiącu ma odbyć jazdy testowe po poznańskich torowiskach. moderus gamma to następca modeli - alfa i beta, które obecnie wożą pasażerów w gdańsku, poznaniu, elblągu, szczecinie i aglomeracji śląskiej. zaprezentowany w piątek nowy tramwaj jest całkowicie niskopodłogowym, co wyróżnia go od poprzedników."model alfa posiadał wyłącznie wysoką podłogę, beta była pojazdem częściowo niskopodłogowym. gamma to z kolei 100 procent niskiej podłogi, przez całą długość nie ma żadnych stopni, przechyleń, można swobodnie się poruszać. gabarytowo nowy tramwaj jest bardzo podobny do tych wcześniej oferowanych, ale posiada całkiem inny design" - powiedział wiceprezes modertrans jarosław bakinowski.pojazd ma 32 metry długości i dysponuje 244 miejscami w tym 66 siedzącymi. posiada szereg udogodnień dla pasażerów - ergonomiczne poręcze, intercomy do kontaktu z motorniczym, rozbudowany system informacji, ładowarki usb oraz jest w pełni klimatyzowany. posiada także rampy i specjalne stanowiska dla osób niepełnosprawnych. nowoczesna i innowacyjna jest także kabina motorniczego. znajdują się w niej panele dotykowe, nastawnik jazdy zintegrowany z fotelem i przyciski w podłokietniku, które podnoszą komfort pracy motorniczego. prototyp powstał przy dużej pomocy zakładu pojazdów szynowych politechniki poznańskiej. dzięki symulacjom przeprowadzonym przez pracowników uczelni, ruch poszczególnych członów oraz całej bryły tramwaju, a także jego masa, zostały zoptymalizowane. jak dodał bakinowski, pojazd jest praktycznie gotowy do użytkowania i gotowy do jazd próbnych. "w przeciągu dwóch-trzech tygodni tramwaj powinien jeździć po poznaniu. oczywiście, na początku jazdy testowe będą odbywały się nocą. trwają już finalne procedury w zakresie uzyskania homologacji i wkrótce będzie on dopuszczony także do ruchu z pasażerami" - podkreślił. pojazd powstał w ramach projektu "innowacyjny tramwaj miejski" i kosztował ponad 14 mln zł. był on współfinansowany przez narodowe centrum badań i rozwoju oraz przez europejski fundusz rozwoju regionalnego. wartość dofinansowania wyniosła 5,6 mln zł. & 305 & very low & Low & Socio-Economic & NA & NA & 2016-11-18 & 2016 & 2 & ECO
Frame & v.low & National & <500 & 1.0448136 & 1.8253861 & -0.2867631 & -1.2581601 & -1.2725643 & 0.0 & -1.1973642 & 0.9927066 & Recipient & Domestic & Domestic & Domestic & Domestic|ECO & Neutral\\
\addlinespace
Poland & http://www.pap.pl/aktualnosci/news,1433748,kwiecinski-znaczace-ciecia-w-polityce-spojnosci-i-wspolnej-polityce-rolnej.html & 840 & pap.pl & Public & Online only & National & low = CP mentioned more times but NOT important part of story (mainly about others issues) & Institutional bargaining over funding & Factual & National & No myth & NA & NA & NA & NA & NA & NA & NA & NA & Poland & kwieciński: znaczące cięcia w polityce spójności i wspólnej polityce rolnej & 2018-05-29 & polityka spójności & na konferencji prasowej minister inwestycji i rozwoju jerzy kwieciński powiedział, że koperta na politykę spójności jest o 10 proc. niższa w cenach realnych niż w obecnej perspektywie finansowej, a przypadku wspólnej polityki rolnej - według niego - są to cięcia nawet o 15 proc. dodał też, że przesunięcia te dotyczą wszystkich krajów, jednak na ograniczeniu polityki spójności najwięcej (20-25 proc.) straci osiem krajów m.in. polska, węgry, słowacja, kraje bałtyckie, ale także niemcy. z propozycji przedstawionej we wtorek przez komisję europejską wynika, że polska ma otrzymać 64,4 mld euro w ramach polityki spójności w nowym wieloletnim budżecie ue. cięcia wynoszące ponad 23 proc. oznaczają, że do polski ma trafić 19,5 mld euro mniej niż obecnie. całkowity budżet na politykę spójności dla całej ue na lata 2021-2027 ma wynieść w zobowiązaniach i cenach bieżących (uwzględniających inflację) 373 mld euro. polsce, która jest największym nowym krajem unijnym ma przypaść największa część tych środków 64,4 mld euro w cenach z 2018 roku lub 72,7 mld euro w cenach stałych (uwzględniających przyszłą inflację). drugie pod tym względem włochy mają otrzymać 38,5 mld euro (w cenach z 2018 roku), a trzecia hiszpania 34 mld euro. "skutkuje to przesunięciem finansowania europejskiego z krajów europy środkowo-wschodniej do krajów południa" - podsumował minister. zauważył, że to przesunięcie jest m.in. efektem zmiany metodologii podziału pieniędzy. ke - jak mówił - mniej wagi przywiązuje do czynników, które są ważne dla polski, takich jak niszy niż w ue poziom zatrudnienia, niższa innowacyjność czy problemy ze smogiem. "niebezpieczne jest to, że w ten sposób wprowadza się pewne podziały, a przecież polityka spójności powinna być polityką, która zasypuje dziury w poziomie rozwoju pomiędzy krajami czy regionami" - powiedział. jerzy kwieciński zapewnił, że rozmowy stron będą trwały, ale najprawdopodobniej nie zakończą się prędzej niż w drugiej połowie 2021 roku. zapytany, jak ewentualne zmniejszenie środków unijnych wpłynie na wzrost gospodarczy polski, powiedział, że największymi stratnymi mogą być samorządy czy organizacje samorządowe, które w wyniku zmniejszenia intensywności wsparcia, będą miały problemy z realizacją projektów, ze względu na niewystarczający wkład własny.(pap) & 338 & low & Low & Power & NA & NA & 2018-05-29 & 2018 & 3 & POL
Frame & low-medium & National & <500 & 1.0448136 & 1.8253861 & -0.2867631 & -1.2581601 & -1.2725643 & 0.0 & -1.1973642 & 0.9927066 & Recipient & Domestic & Domestic & Domestic & Domestic|POL & Neutral\\
Poland & http://fakty.interia.pl/deutsche-welle/news-sz-polska-i-wegry-zaplaca-za-odmowe-solidarnosci,nId,2505803 & 508 & fakty.interia.pl & Private/Non-Public & Online only & National & medium = CP is important part of story & Institutional bargaining over funding & Negative & Other country & No myth & NA & NA & NA & NA & NA & NA & NA & NA & Poland & "sz": polska i węgry zapłacą za odmowę solidarności & 2018-01-09 & fundusz spójności & w berlinie i innych stolicach europejskich nie zapomniano, które państwa odmówiły solidarności w czasie kryzysu uchodźczego. węgrom i polsce wystawiony zostanie za to rachunek podczas negocjacji o przyszłym budżecie ue - pisze we wtorek "sueddeutsche zeitung". autorzy materiału "cena solidarności" - daniel broessler i alexander muehlauer - przypominają na wstępie, że z powodu brexitu unii zabraknie od 2021 r. od 10 do 13 mld euro rocznie. "to ciężki cios dla europejskiej maszyny rozdzielającej fundusze" - piszą dziennikarze. szefowie państw i rządów są ponadto zdecydowani, by przeznaczyć więcej środków na ochronę granic, walkę z terroryzmem i obronność. reklama nawiązując do poniedziałkowej konferencji w brukseli, w której uczestniczyli szef komisji europejskiej jean-claude juncker i komisarz ds. budżetu guenther oettinger, autorzy wysuwają przypuszczenie, że redukcje dotkną przede wszystkim fundusz spójności, stanowiący obecnie 30 proc. całego unijnego budżetu. kluczowym pytaniem w rozmowach o ramach finansowych po 2021 r. będzie problem wzajemnej solidarności - uważają broessler i muehlauer. autorzy zwracają uwagę na wystąpienie szefa msz niemiec sigmara gabriela, który opowiedział się co prawda za "solidarną i ambitną polityką europejską o szerokich horyzontach, zamiast wątpliwości i czerwonych linii", lecz równocześnie przypomniał, że konieczna jest "zgoda co do wolności, demokracji i praworządności". wyraźny sygnał "to wyraźny sygnał w kierunku polski, wobec której komisja europejska z obawy o praworządność wdrożyła bezprecedensowe postępowanie" - czytamy w gazecie, która przypomina, że polska należy do największych beneficjentów unijnego budżetu. niemcy, francja, szwecja, austria i holandia chcą zredukować transfery do europy wschodniej - zaznaczają autorzy. z ekonomicznego punktu widzenia jest to zrozumiałe, gdyż krajom tego regionu powodzi się lepiej niż w czasie negocjacji o poprzednim budżecie. "istnieje jednak także argument polityczny: w berlinie i w innych stolicach nie zapomniano, kto podczas walki z kryzysem uchodźczym odmówił solidarności. węgry i polska otrzymają zapewne za to podczas negocjacji budżetowych rachunek do zapłacenia" - czytamy w "sz". wśród możliwych rozwiązań autorzy wymieniają wprowadzenie zasady, że kraj korzystający ze środków unijnych będzie musiał na realizację projektu wyłożyć z własnej kieszeni jedną trzecią kosztów. obecnie udział własny wynosi 10 proc. taki zabieg byłby gwarancją, że kraje członkowskie będą przedkładać tylko rozsądne projekty. berlin zabiega ponadto o to, by zwiększyć środki dla krajów przeprowadzających ambitne reformy. opr. jacek lepiarz & 354 & medium & Medium & Power & NA & NA & 2018-01-09 & 2018 & 3 & POL
Frame & low-medium & National & <500 & 1.0448136 & 1.8253861 & -0.2867631 & -1.2581601 & -1.2725643 & 0.0 & -1.1973642 & 0.9927066 & Recipient & European & European & European & European|POL & Negative\\
Poland & http://www.pap.pl/aktualnosci/news,1289870,kwiecinski-podstawy-prawne-powiazania-polityki-spojnosci-z-praworzadnoscia-sa-watle.html & 283 & pap.pl & Public & Online only & National & very high = CP is most important issue + CP is mentioned in title/headline & Institutional bargaining over funding & Negative & National & No myth & NA & NA & NA & NA & NA & NA & NA & NA & Poland & kwieciński: podstawy prawne powiązania polityki spójności z praworządnością są wątłe & 2018-02-15 & polityka spójności & minister przyznał, że powiązanie polityki spójności z praworządnością jest dyskutowane w aspekcie obecnej perspektywy finansowej, ale głównie nowej. "jeżeli mówimy o obecnej perspektywie finansowej, to prawdopodobieństwo powiązania (zasady praworządności z wypłatami - pap) jest znikome. nie ma ani podstaw traktatowych, ani innych podstaw prawnych, żeby takiego powiązania dokonać" - ocenił minister. "jeżeli chodzi natomiast o nową perspektywę finansową, to oczywiście takie propozycje pojawiają się ze strony - na razie w postaci bardziej rozmów, dezyderatów, niż konkretnych sformułowań - płatników netto i również ze strony komisji europejskiej" - wyjaśnił. "my się mocno tym propozycjom sprzeciwiamy, widzimy również, że bardzo wiele krajów, szczególnie krajów naszego regionu też się przeciwstawia tego typu rozwiązaniom. wydaje nam się, że jest bardzo wątła podstawa prawna, szczególnie traktatowa dla tego typu rozwiązań. szczególnie, jeżeli chodzi o politykę spójności, to możemy powiedzieć, że w tych rozmowach jest za dużo polityki, a za mało spójności" - podkreślił. jego zdaniem obecnie za dużo się mówi się o kwestiach politycznych, a za mało o spójności, która jest potrzebna całej ue. minister wskazał, że tego typu rozwiązania będą musiały być przyjęte przez wszystkie kraje członkowskie, bowiem wszystkie będą musiały zatwierdzić nowy budżet ue. "my tego typu rozwiązaniom będziemy się bardzo mocno przeciwstawiać" - podkreślił. według kwiecińskiego musimy być jednak "absolutnie" przygotowani, że w przyszłej polityce spójności po 2020 r. będzie mniej pieniędzy niż obecnie. jak wytłumaczył, wynika z trzech czynników. "pierwszy to taki, że po prostu bogacimy się. jako kraj mamy coraz wyższy poziom rozwoju, również nasze regiony mają coraz wyższy poziom i kolejne regiony wypadają z grupy tych regionów najbiedniejszych" - powiedział. wyjaśnił, że limit, który upoważnia do korzystania ze znaczącego wsparcia z polityki spójności jest poziom 75 proc. pkb. mazowsze w starych granicach (jako region - pap) miało poziom 109 proc. pkb i weszło do grupy bogatszych regionów w unii. od stycznia tego roku mazowsze podzielono na dwa nowe regiony - centrum z aglomeracją warszawską ma pkb powyżej 150 proc. średniej unijnej. "gdyby z aglomeracji warszawskiej utworzyć nowy kraj (...), to byłby to najbogatszy kraj w ue po luksemburgu" - zauważył kwieciński. dodał, że po podziale pozostała część mazowsza ma średnią ok. 70 proc. wyjaśnił ponadto, że poziom 75 proc. przekroczyły niedawno dolny śląsk i wielkopolska, a kolejni kandydaci to śląsk i województwo pomorskie. drugi powód to brexit, wyjście z ue drugiego najbogatszego kraju wspólnoty, który przekazywał do budżetu ue co roku ok. 14 mld euro, 10 proc. średniorocznych wydatków ue. "z racji tego powstaje dziura (...) tych pieniędzy będzie mniej" - mówił. trzeci czynnik to nowe priorytety polityk ue, które będą wymagały finansowania: m.in. polityka migracyjna czy bezpieczeństwo zewnętrzne i wewnętrzne wspólnoty. zwrócił uwagę, że ue będzie chciała przeznaczać więcej środków na różnego rodzaju wydatki związane z obronnością, co polska popiera. kwieciński odniósł się też do propozycji komisarza ue ds. budżetu guenthera oettingera, który chce by w przyszłej perspektywie finansowej kraje członkowskie znacząco zwiększyły swój wkład do budżetu, że powinien on wynosić 1,1x, co oznacza, że może to być nawet 1,2 dochodu narodowego brutto. w tej chwili składka wynosi 1 proc. "to by oznaczało zwiększenie wkładu do budżetu unijnego krajów członkowskich o prawie jedną piątą, więc mamy z jednej strony tę presję na zmniejszenie wydatków w ramach polityki spójności i we wspólnej polityce rolnej związane z brexitem i nowymi zadaniami, ale z drugiej strony propozycje ke (...), która zapewne będzie proponować zwiększenie składki członkowskiej" - mówił minister. "rozmowy są w toku, musimy być w nich obecni, akcentować potrzebę polityki spójności, jako największej polityki inwestycyjnej ue, jako tej która odegrała bardzo ważną role w mitygowaniu kryzysu gospodarczego" - poinformował kwieciński. jego zdaniem nie ma obecnie debaty, czy polityka spójności jest potrzeba, czy nie - co miało miejsce dwa lata temu - ale rozmowy dotyczące priorytetów i zmian w tej polityce. komisja europejska zaprezentowała w środę dokument przedstawiający różne warianty nowego wieloletniego budżetu ue, który ma umożliwić realizację priorytetów po 2020 r. jedną z przedstawionych pod rozwagę opcji jest możliwość większego uzależnienia finansowania z ue od poszanowania podstawowych wartości unii. "każdy taki mechanizm musiałby być przejrzysty, proporcjonalny i prawnie niepodważalny" - podkreślono w dokumencie komisji. urzędnicy w brukseli wskazali, że co do zasady mógłby być on stosowany do wszystkich polityk unijnych, ale jakakolwiek warunkowość finansowa musiałaby być precyzyjna i propozycjonalna. "w trakcie debaty potrzebne będzie również rozważenie wpływu możliwego łamania podstawowych wartości czy praworządności na poziomie krajowym na indywidulanych beneficjentów funduszy z ue, takich jak studentów erasmusa, badaczy czy organizacji społeczeństwa obywatelskiego" - zaznaczono. uzależnienie funduszy od stanu praworządności byłoby niekorzystne dla polski, która jest w sporze w tej sprawie z ke. ze słów szefa ke jean-claude'a junckera wynika, że jest jednak szansa na kompromis. "kanały dialogu są otwarte na stałe i sądzę, że jest bardzo duża szansa, że polskie stanowisko ewoluuje nieco bliżej naszego, a nasze stanowisko przesunie się powoli w kierunku polskiego" - oświadczył szef ke, przypominając niedawne rozmowy o praworządności między polskimi władzami a komisją. choć w dokumencie komisji warunkowość dostępu do funduszy jest opcją (nie decyzją), komisarz ds. budżetu oettinger poinformował, że już rozpoczęły się prace legislacyjne nad odpowiednim tekstem prawnym. rozmówcy pap w brukseli tłumaczą, że chodzi o przygotowanie się na wszystkie scenariusze. komisja chce, żeby w sprawie praworządności, ale też innych dylematów dotyczących budżetu wypowiedzieli się szefowie państw i rządów. "jeśli tylko dwa kraje poprą to rozwiązanie, nie będziemy go prezentować. potrzebujemy jasnego sygnału od przywódców" - podkreśliło źródło. nieformalny szczyt 27 przywódców ue, na którym będą omawiane wieloletnie ramy finansowe, zaplanowany jest na 23 lutego. (pap) & 884 & very high & High & Power & NA & NA & 2018-02-15 & 2018 & 3 & POL
Frame & high-very high & National & 500-1000 & 1.0448136 & 1.8253861 & -0.2867631 & -1.2581601 & -1.2725643 & 0.0 & -1.1973642 & 0.9927066 & Recipient & Domestic & Domestic & Domestic & Domestic|POL & Negative\\
Poland & https://www.wprost.pl/kraj/10171857/premier-po-spotkaniu-przyjaciol-polityki-spojnosci-musimy-wynegocjowac-jak-najlepsza-pule-srodkow.html & 676 & WPROST.pl & Private/Non-Public & Online and Offline & National & medium = CP is important part of story & Institutional bargaining over funding & Positive & National & No myth & Economic development & Positive & National & No myth & NA & NA & NA & NA & Poland & premier po spotkaniu "przyjaciół polityki spójności": musimy wynegocjować jak najlepszą pulę środków & 2018-11-29 & polityka spójności & - polityka spójności na lata 2020-2027 jest kluczową polityką i musimy wynegocjować jak najlepszą pulę środków właśnie przeznaczoną na spójność - oświadczył premier mateusz morawiecki po spotkaniu "grupy przyjaciół polityki spójności" w bratysławie. jak mówił szef polskiego rządu, pieniądze z polityki spójności są wydatkowane na inwestycje infrastrukturalne, środowiskowe, energetyczne. - te inwestycje pomagają krajom, które mają niższy poziom rozwoju doganiać te, które miały więcej szczęścia po ii wojnie światowej i rozwijały się w bardziej efektywnym systemie - mówił premier morawiecki. zdaniem premiera, polska odgrywa jeśli chodzi o integrację i budowę jednolitego, jednoznacznego stanowiska krajów-przyjaciół spójności w odniesieniu do przyszłego budżetu ue. jak dodał, odnosząc się do przebiegu szczytu w bratysławie, kraje te jednoznacznie wyznaczyły . jak poinformowano, grupie zależy na utrzymaniu finansowania polityki spójności i wspólnej polityki rolnej na poziomie z wieloletnich ram finansowych z lat 2014-2020. w szczycie wzięli udział premierzy ośmiu państw: słowacji, polski, czech, estonii, chorwacji, węgier, malty i słowenii, a także przedstawiciele władz: bułgarii, cypru, litwy, łotwy, rumunii, włoch, portugalii i grecji. w szczycie uczestniczyli też unijny komisarz ds. unii energetycznej marosz szefczovicz oraz komisarz ds. budżetu guenther oettinger, a także wiceprezes europejskiego banku inwestycyjnego vazil hudak. & 189 & medium & Medium & Power & Socio-Economic & NA & 2018-11-29 & 2018 & 3 & POL
Frame & low-medium & National & <500 & 1.0448136 & 1.8253861 & -0.2867631 & -1.2581601 & -1.2725643 & 0.0 & -1.1973642 & 0.9927066 & Recipient & Domestic & Domestic & Domestic & Domestic|POL & Positive\\
Poland & http://www.gazetaprawna.pl/artykuly/1043276,komitet-regionow-ue-po-brexicie-udzial-polityki-spojnosci-bez-zmian.html & 63 & gazetaprawna.pl & Private/Non-Public & Online and Offline & National & very high = CP is most important issue + CP is mentioned in title/headline & Institutional bargaining over funding & Positive & EU + Subnational & No myth & NA & NA & NA & NA & NA & NA & NA & NA & Poland & komitet regionów ue: po brexicie udział polityki spójności bez zmian & 2017-05-16 & polityka spójności & nawet po brexicie udział procentowy polityki spójności w budżecie ue musi pozostać bez zmian - podkreślił komitet regionów ue, w którym zasiadają samorządowcy z 28 państw unii europejskiej. komitet regionów ue, jako pierwsza instytucja unijna, przyjął oficjalne stanowisko w sprawie polityki spójności po 2020 r., czyli po obecnym okresie programowania 2014-2020. w ramach tej polityki państwa ue, w tym polska, otrzymują środki unijne. jak podkreślono, komitet regionów "sprzeciwia się wszelkim przekierowaniom funduszy i domaga się, żeby nawet po brexicie, udział procentowy polityki spójności w budżecie ue pozostał bez zmian". "w czasach rosnącego populizmu i nacjonalizmu, polityka spójności może pomóc zmniejszyć dystans między instytucjami ue a obywatelami i urzeczywistnić unijną solidarność w każdym zakątku ue" - napisano w oświadczeniu. według unijnych samorządowców przy kształtowaniu przyszłej polityki spójności należy pamiętać, że dzięki setkom tysięcy projektów wspieranych przez fundusze europejskie w ostatnim dziesięcioleciu udało się stworzyć nowe miejsca pracy, zmodernizować infrastrukturę i poprawić jakość życia europejczyków. komitet regionów w swoim stanowisku sceptycznie odniósł się do pomysłów zwiększenia udziału pożyczek kosztem dotacji w następnym okresie programowania. decyzje w sprawie wykorzystania instrumentów finansowych i dotacji powinny być, w opinii samorządowców, podejmowane w regionie. regionalny pkb, będący głównym wskaźnikiem przydziału środków unijnych, powinien zostać uzupełniony o kryteria związane z rynkiem pracy i "szczególnymi wyzwaniami" (społecznymi, środowiskowymi, geograficznymi, naturalnymi i demograficznymi). jak podkreślono, należy ograniczyć biurokrację w planowaniu i zarządzaniu funduszami europejskimi. "koncentracja tematyczna powinna zostać zachowana, (...) ale nie powinno to wykluczać wsparcia finansowego na projekty infrastrukturalne w obszarach priorytetowych. powinna istnieć możliwość dostosowania przyszłych programów inwestycyjnych poprzez uproszczenie procedur, aby reagować na sytuacje kryzysowe i nieprzewidziane zdarzenia (takie jak intensywne przepływy uchodźców lub klęski żywiołowe)" - głosi dokument. komitet regionów wezwał ponadto do "radykalnego uproszczenia procedur" w zakresie zarządzania i kontroli w odniesieniu do funduszy ue. europejskie regiony i miasta domagają się również zapewnienia "prawdziwej współpracy" w ramach planowania i zarządzania środkami unijnymi między instytucjami ue, władzami krajowymi, regionalnymi i lokalnymi. przedstawiciele unijnego komitetu zwracają uwagę na sukces polskiej delegacji, która zabiegała, by w dokumencie znalazły się zapisy o konieczności zapewnienia polityce spójności odpowiednich środków finansowych. polskim samorządowcom udało się również umieścić w oficjalnym stanowisku postulaty odnośnie tzw. regionów przejściowych czy potrzeby większej elastyczności regionalnych programów operacyjnych. szef polskiej delegacji w komitecie regionów, marszałek województwa wielkopolskiego marek woźniak podkreślił, że stanowisko (które trafi teraz do komisji europejskiej, parlamentu europejskiego i rady unii europejskiej) jest kluczowe i zostało przyjęta we właściwym czasie, aby wywrzeć wpływ na trwającą debatę na temat przyszłości polityki spójności. "cieszy mnie fakt, że polskie regiony aktywnie uczestniczyły w tym procesie, a nasze postulaty znalazły odzwierciedlenie w dokumencie. jesteśmy przekonani, że przyszła polityka spójności musi dysponować wystarczającymi środkami, jej zasadnicza rola i podstawowe cele muszą zostać zachowane, a zasady zarządzania europejskimi funduszami strukturalnymi i inwestycyjnymi ujednolicone. potrzebujemy silnej i skutecznej polityki spójności gwarantującej przezwyciężenie kryzysów" - podkreślił woźniak. (pap) & 462 & very high & High & Power & NA & NA & 2017-05-16 & 2017 & 2 & POL
Frame & high-very high & National & <500 & 1.0448136 & 1.8253861 & -0.2867631 & -1.2581601 & -1.2725643 & 0.0 & -1.1973642 & 0.9927066 & Recipient & Domestic & European & Mixed & Domestic|POL & Positive\\
\addlinespace
Poland & http://forsal.pl/artykuly/975925,borys-w-tym-roku-ok-100-tys-msp-skorzystalo-juz-z-instrumentow-wsparcia.html & 344 & forsal.pl & Private/Non-Public & Online only & National & very low = CP mentioned once & Economic development & Positive & National & No myth & NA & NA & NA & NA & NA & NA & NA & NA & Poland & borys: w tym roku ok. 100 tys. mśp skorzystało już z instrumentów wsparcia & 2016-09-14 & polityka spójności & tylko w tym roku ok. 100 tysięcy małych i średnich przedsiębiorstw skorzystało z różnych instrumentów wsparcia, a gwarancje dla nich wyniosły 22 mld zł, co przekłada się na 30 mld w kredytach w sektorze bankowym - powiedział prezes polskiego funduszu rozwoju paweł borys w środę w warszawie. szef pfr wziął udział w forum dla małych i średnich przedsiębiorstw "twój biznes nasze wsparcie", organizowanym przez polską agencję rozwoju przedsiębiorczości (parp), a objętym patronatem ministerstwa rozwoju. borys wypowiadał się w kontekście pomocy dla małych i średnich firm w konsultowanej obecnie przez mr strategii na rzecz odpowiedzialnego rozwoju (sor). jak podkreślił, sektor mśp jest priorytetowy dla polskiego funduszu rozwoju. "jesteśmy w gospodarce, w której małe i średnie firmy tworzą 70 proc. zatrudnienia i dwie trzecie produktu krajowego brutto. zgadzam się też, że jest pewne optimum między rozdrobnieniem a skalą działalności - polska gospodarka jest bardzo mocno rozdrobniona, powinniśmy więc dążyć do tego, żeby firmy w niektórych branżach się konsolidowały, a przez to były bardziej konkurencyjne na rynkach międzynarodowych" - mówił. "w ramach tego co my robimy, tylko w tym roku podsumowaliśmy, że około 100 tysięcy małych i średnich firm skorzystało z różnego typu wsparcia. głównie dotyczy to programów gwarancji de minimis dla małych i średnich przedsiębiorstw, dystrybuowanych za pośrednictwem banków, po to, aby ten koszt finansowania był trochę niższy, a finansowanie było bardziej dostępne, ale bez eliminowania mechanizmu oceny ryzyka kredytowego" - podkreślił. "to jest w tej chwili 22 mld takich gwarancji udzielonych, co przekłada się na 30 mld kredytów w sektorze bankowym" - mówił. jak tłumaczył, pomagają one przedsiębiorcom w rozwinięciu biznesu i działaniu w większej skali. "bank gospodarstwa krajowego udziela w tej chwili takich kredytów, czy gwarancji na innowacje, trochę pod kątem ukierunkowania tych inwestycji w nowe technologie. także parp ma szereg programów z tym związanych" - wskazał szef pfr. przypomniał też o ogłoszonym w krynicy programie wsparcia rozwoju na rynkach międzynarodowych. "w tym nowym systemie przedstawicielstw międzynarodowych chcemy, by służyły one małym i średnim (firmom) w tym, aby na przykład nasze biuro w brazylii czy w chinach mogło służyć jako przedstawicielstwo zagraniczne takiej małej i średniej firmy, gdzie może mieć dostarczone usługi księgowe, adres, bo wiadomo, że takiej firmy nie stać by było na własne przedstawicielstwo" - wyjaśnił. uczestnicząca w debacie wiceminister rozwoju jadwiga emilewicz określiła wsparcie państwa dla mśp mianem "inteligentnego pieniądza", który nie stanowi środka łatwego do pozyskania. jednocześnie - wskazała - sprzyja celowi sor, a więc rozwinięciu się małych przedsiębiorstw w średnie, a średnich w duże. prezentując założenia strategii na rzecz odpowiedzialnego rozwoju, wiceszefowa mr wskazywała na potrzebę selekcji branż priorytetowych dla polskiej gospodarki. wśród nich wymieniła m.in. elektromobilność, przemysł lotniczy i kosmiczny, systemy wydobywcze, żywność wysokiej jakości, odzysk materiału surowcowego, czy ekobudownictwo, realizowane poprzez projektowany program rządowy mieszkanie plus. "to też wynik analizy, gdzie jesteśmy dzisiaj i gdzie możemy być w horyzoncie trzech, pięciu, dziesięciu lat" - mówiła. dr jacek adamski z lewiatan business angels zauważył, że unia europejska zakazuje polityki sektorowej, czyli wspierania pewnych branż kosztem innych. ale - zauważył - polityka spójności pozwala na pewne wyjątki i wyznaczenie tzw. inteligentnych specjalizacji. "to stwarza pokusę, by wybrać przyszłych liderów i ich wspomagać" - mówił. "to będzie trwało do roku 2023. to jest okres historyczny, w którym można zrealizować politykę sektorową. zróbmy tak, by była ona mądra i skuteczna, a nie tak, byśmy omyłkowo wybrali pewne dziedziny gospodarki, czy branże przemysłowe, które okażą się porażką i te środki bezpowrotnie będą zmarnowane" - podkreślił. według niego ryzyko jest "olbrzymie" i dobór branż powinien być poprzedzony ścisłymi i niespiesznymi konsultacjami z kołami gospodarczymi. wskazał, że w sor zabrakło mu np. branży meblarskiej, która świetnie się rozwija, oferuje dużą wartość dodaną, mimo że nie jest źródłem innowacji. "wartość dodana tkwi w materiałach, brandzie i we wzornictwie. to nie jest high-tech, innowacyjności za dużo tam nie ma" - przyznał, postulując zarazem, by design traktować tak samo jak innowacyjność. odnosząc się do kwestii przemysłu rolno-spożywczego zwrócił natomiast uwagę, że wartość dodana tego sektora nie zależy tylko od samego dobrego produktu, ale sposobu pakowania, ekspozycji i wyglądu. "pieniądze przeznaczone na towaroznawstwo (...) będą o wiele lepiej wydane, niż na inwestycje" - przekonywał. dr hab. jerzy cieślik z akademii leona koźmińskiego zwrócił natomiast uwagę, że nie ma bezpośredniej zależności między innowacyjnością a rozwojem; według niego motorem rozwoju jest "ambicja rozwojowa" przedsiębiorcy. przypomniał przy tym, że w polsce istnieje 1,8 mln mikroprzedsiębiorstw, z czego 1,2 mln to osoby samozatrudnione. spośród nich należy wspierać tylko te, które chcą zatrudniać pracowników, tu efekt rozwojowy byłby największy - wskazał. & 729 & very low & Low & Socio-Economic & NA & NA & 2016-09-14 & 2016 & 2 & ECO
Frame & v.low & National & 500-1000 & 1.0448136 & 1.8253861 & -0.2867631 & -1.2581601 & -1.2725643 & 0.0 & -1.1973642 & 0.9927066 & Recipient & Domestic & Domestic & Domestic & Domestic|ECO & Positive\\
Poland & https://www.euractiv.pl/section/polityka-regionalna/news/komitet-regionow-ue-broni-polityki-spojnosci-jako-sily-napedowej-innowacji/ & 793 & EurActiv.pl & Private/Non-Public & Online only & National & high = CP is most important issue in story (can also cover other issues) & Research \& innovation & Positive & EU & No myth & NA & NA & NA & NA & NA & NA & NA & NA & Poland & komitet regionów ue broni polityki spójności jako siły napędowej innowacji & 2018-12-04 & polityka spójności & wykorzystując środki z funduszy polityki spójności, ue powinna wzmocnić regionalne inwestycje w innowacje, aby móc konkurować z usa i chinami - podkreślili przywódcy regionalni podczas europejskiego szczytu innowacji w brukseli. niedawny raport opublikowany przez europejski bank inwestycyjny alarmował, że "inwestycje w walkę ze zmianami klimatycznymi pozostają nie stoją na zadowalającym poziomie", a "firmy ue nadal nie przeznaczają wystarczających środków na badania i rozwój", aby zachować konkurencyjność na skalę światową. "jeśli europa chce być globalnie konkurencyjna w zakresie cyfryzacji, rozwijać się w sposób bardziej zrównoważony, i tworzyć jednocześnie inkluzyjne społeczeństwo, musi zainwestować. i to mądrze" - podkreślił wiceprzewodniczący ebi andrew mcdowell. polityka spójności ue jest głównym źródłem inwestycji tak w obecnym okresie finansowym (2014-2020),jak i w następnym (2021-2027). przywódcy regionalni bronili więc roli władz lokalnych w stymulowaniu wzrostu i rozwoju w europie. "inwestycje w europie w zasadzie mają się dobrze" - powiedział mcdowell. ostrzegł jednak, że "istnieją luki inwestycyjne w sektorach, których rozwój jest niezbędny, aby stawiać czoła wyzwaniom dnia dzisiejszego i przyszłości". globalizacja, automatyzacja, dekarbonizacja, transformacja cyfrowa - te globalne zjawiska mają istotny wpływ na gospodarkę, ponieważ zmuszają przedsiębiorstwa i pracowników do zmian. wpływ tych przekształceń jest różny w poszczególnych regionach i miastach. lokalne problemy mają już wymiar globalny, zaś globalne problemy stały się lokalne. w nawiązaniu do tych wyzwań, przewodniczący europejskiego komitetu regionów karl-heinz lambertz podkreślił, że podczas gdy ludzie mówią o polityce spójności jako o "tradycyjnej" polityce, tak naprawdę jest dokładnie odwrotnie. "polityka spójności jest jednym z najważniejszych środków, którymi dysponują regiony, aby dostosować się do globalizacji i pomóc przedsiębiorstwom i obywatelom zrobić to samo," powiedział lambertz. "to najważniejsza unijna polityka inwestycyjna, część dna unii europejskiej" - podkreślił. jedną z nowości kolejnego długofalowego budżetu ue jest synergia między polityką spójności a narzędziami badawczymi i innowacyjnymi. ich zadaniem jest bowiem ułatwianie rozdzielania funduszy oraz pomoc regionom we wdrażaniu innowacji. "każdy region potrzebuje polityki spójności, aby rozwijać swoje możliwości innowacyjne" - powiedział przewodniczący komitetu regionów i podkreślił znaczenie utrzymania trzech kategorii regionów - mniej rozwiniętych, rozwijających się i wysoko rozwiniętych. ma to zapewnić, że fundusze będą dostosowane do różnych szerokiego spektrum potrzeb . chociaż regiony europejskie znacznie różnią się między sobą, to charakter wyzwań, przed którymi obecnie stoją, jest taki sam. dlatego też gubernator dolnej austrii, johanna mikl-leitner, podkreśliła wagę włączenia wszystkich regionów do kolejnej unijnej polityki spójności na lata 2021-2027. nadal jednak potrzeba w tym kierunku wieli starań, ponieważ lepiej rozwinięte regiony zazwyczaj lepiej wykorzystują fundusze ue na wspieranie innowacji. w obliczu coraz bardziej protekcjonistycznej polityki chin i donalda trumpa, mikl-leitner mocno wierzy, że "wciąż potrzebujemy silnej i zjednoczonej europy. jeśli nie będziemy trzymać się razem, przegramy". w kontekście rewolucji cyfrowej istotną rolę odgrywają zmiany klimatu, zmiany demograficzne i fundusze innowacyjne. jednak fundusze te mogą być zagrożone, ponieważ komisja zaproponowała zmniejszenie kolejnego długofalowego budżetu ue. "decyzje, które podejmie europa, będą kształtować przyszłość. potrzebujemy jasnej strategii, aby zrealizować nasze przyszłe wyzwania, ale też, żeby wiedzieć na czym stoimy." - powiedziała gubernator. "nadal potrzebujemy i przez długi czas będziemy potrzebować silnej polityki spójności dla wszystkich regionów" - podkreślił. według danych ebi inwestycje ue w badania i innowacje pozostają obecnie na poziomie 2 proc. pkb, podobnie jak w chinach. w usa jest to natomiast 2,8 proc. "jeśli ue chce osiągnąć swój cel 3 proc. do 2020 roku, będzie musiała zainwestować dodatkowe 140 miliardów euro rocznie" - zastrzegł bank. komisja przedstawiła już swój wniosek w sprawie polityki spójności na lata 2021-2027. jego głównymi celami są wspieranie innowacji, małych przedsiębiorstw, technologii cyfrowych i modernizacji przemysłu, a także promowanie przejścia na gospodarkę o niskiej emisji dwutlenku węgla. państwa członkowskie rozpoczną dyskusje na temat kolejnego długofalowego budżetu ue w ciągu kolejnych dwóch tygodni podczas zimowego posiedzenia rady europejskiej. chociaż parlament europejski i komisja zwróciły się do europejskich przywódców o wypracowanie porozumienia w sprawie planu finansowego na lata 2021-2027 przed wyborami do parlamentu europejskiego w maju, perspektywa ta nie wydaje się realna do osiągnięcia. & 636 & high & High & Socio-Economic & NA & NA & 2018-12-04 & 2018 & 3 & ECO
Frame & high-very high & National & 500-1000 & 1.0448136 & 1.8253861 & -0.2867631 & -1.2581601 & -1.2725643 & 0.0 & -1.1973642 & 0.9927066 & Recipient & European & European & European & European|ECO & Positive\\
Poland & http://www.gazetaprawna.pl/artykuly/1121211,projekt-nowego-budzetu-unijnego-oczekiwania-polski.html & 242 & gazetaprawna.pl & Private/Non-Public & Online and Offline & National & medium = CP is important part of story & Institutional bargaining over funding & Balanced & National & No myth & Political capital/interests & Positive & National & No myth & NA & NA & NA & NA & Poland & szymański: w projekcie nowego budżetu ue spełniono niektóre oczekiwania polski & 2018-05-02 & polityka spójności & w propozycji wieloletnich ram finansowych na lata 2021-2027 zostały spełnione niektóre oczekiwania polski i europy środkowej; droga do pełnego kompromisu jest jeszcze daleka - podkreślił wiceszef msz konrad szymański. ocenił, że obecne negocjacje budżetu należą do najtrudniejszych w historii ue. komisja europejska przyjęła w środę propozycję wieloletniego budżetu ue na lata 2021-2027. całkowita suma zobowiązań ma wynieść w tym okresie 1,279 biliona euro, natomiast płatności 1,246 biliona euro. podziału na koperty narodowe na razie nie ma. w projekcie budżetu ue na lata 2021-2027 ke zaproponowała uzależnienie wypłaty funduszy od przestrzegania praworządności. wniosek w sprawie zawieszenia lub ograniczenia funduszy dla danego kraju będzie składany przez ke i przyjmowany przez kraje członkowskie. w zaproponowanym przez ke projekcie unijnych ram finansowych na l. 2021-2027 cięcia w polityce spójności mają wynieść około 7 proc., a we wspólnej polityce rolnej - około 5 proc. "proces negocjowania wieloletnich ram finansowych się rozpoczyna, proces ten będzie zapewne trudny, długi, miejscami dość nerwowy i niepozbawiony zwrotów akcji" - powiedział szymański na środowym briefingu. dodał, że tempo tego procesu będzie zależało od zdolności do kompromisu zarówno państw członkowskich, jak i instytucji unijnych. podkreślił, że polska docenia fakt, iż w projekcie nowego budżetu zostały uwzględnione "niektóre kierunkowe - polskie i europy środkowej, oczekiwania, jeśli chodzi o kształt i architekturę budżetu wieloletniego". jego zdaniem droga do "pełnego kompromisu" ws. budżetu jest jeszcze "bardzo daleka". "doceniamy fakt, że udało się, z walnym udziałem premiera morawieckiego, wyhamować najbardziej radykalne propozycje cięć, najbardziej radykalne propozycje zmian, propozycje prawdziwego zrewolucjonizowania budżetu unijnego" - dodał wiceszef msz. ocenił, że przedstawiony w środę dokument "jest jeszcze bardzo niedoskonały". jak mówił, wyznacza on "perspektywę porozumienia", do którego "jest jeszcze daleko". wiceszef msz był pytany na środowym briefingu prasowym, czy cięcia w polityce spójności wpłyną na sytuację biedniejszych regionów polski. "cały czas podkreślamy, że polityka spójności, polityka wsparcia regionalnego, to jest polityka, która służy całej unii. to nie jest polityka pomocowa, jakichś bezzwrotnych pożyczek, to jest polityka inwestycji w całość ue. z całą pewnością nie zgodzimy się na nieproporcjonalne cięcia w tym obszarze" - zapewnił szymański. podkreślił, że polska zgadza się z argumentem, który - w jego ocenie - w debacie nad budżetem będzie "na pewno wielokrotnie podnoszony" o tym, że polska gospodarka w ostatnich latach "urosła bardzo poważnie" w porównaniu z innymi europejskimi gospodarkami, dotkniętymi przez kryzys. "przyjmujemy do wiadomości ten argument, natomiast chcemy, żeby był on potraktowany proporcjonalnie" - zaznaczył szymański. dodał, że polska jest "dumna" z tego, iż dzięki rozwojowi gospodarczemu, unijny budżet będzie odgrywał mniejszą rolę. szymański zauważył też, że polska będzie największym beneficjentem polityki spójności. jego zdaniem wynika to ze wszystkich "realistycznych propozycji" zmian w unijnym budżecie. "polska naturalnie, z uwagi na swój status w ue, nie może nie być najpoważniejszym odbiorcą pomocy strukturalnej" - ocenił. wiceminister odniósł się również do propozycji powiązania wypłaty środków unijnych od przestrzegania praworządności. "w ke nikt na poważnie nie bierze pod uwagę instrumentów, które by na polityczne życzenie, czy presję mogły ograniczać prawa państw członkowskich, także prawa w zakresie budżetu, na który wszyscy razem się składamy. polska składka do tego budżetu jest składką niebagatelną" - podkreślił. dodał, że niepokoją go propozycje, które - jak powiedział "mogłyby sugerować polityczną uznaniowość" stojącą, w ocenie szymańskiego, w sprzeczności z zasadą praworządności. "każda norma prawna musi być określona, nie może taka norma prawna być uległa politycznym emocjom, czy politycznym zmianom w samej ue" - zaznaczył wiceszef msz. jak zaznaczył, polska, bez względu na losy negocjacji, będzie największym beneficjentem polityki spójności i bardzo poważnym beneficjentem polityki rolnej. zobacz także:polska prawdopodobnie pozostanie największym beneficjentem nowego budżetu unijnego " szymański ocenił jednocześnie, że obecne negocjacje unijnego budżetu należą do najtrudniejszych w historii ue. jak mówił, "z jednej strony mamy rosnącą, zaognioną można powiedzieć presję na rzecz oszczędności w krajach północy". "mam tu na myśli przede wszystkim partie protestu, które eksploatują bardzo silnie wątek ograniczenia zaufania do ue i ograniczenia przekazywania pieniędzy na rzecz ue" - mówił wiceminister. z drugiej strony - wskazał - w krajach południa "mamy wciąż do czynienia z niezakończonym poważnym kryzysem strukturalnym, w szczególności w zakresie bezrobocia". według wiceministra - na tym tle europa środkowa jest "oazą stabilności politycznej i wzrostu gospodarczego". podkreślił, że "proces konwergencji ekonomicznej" nie jest jeszcze zakończony. "w związku z czym oczekujemy zrównoważenia, oczekujemy równowagi nowych i starych celów unii europejskiej. tej równowagi na razie nie mamy, ale na pewno wykonano pewne kroki w kierunku tej równowagi. widzieliśmy znaczeni gorsze propozycje ułożenia wieloletniego budżetu ue w najbliższym czasie" - podkreślił szymański. & 721 & medium & Medium & Power & Power & NA & 2018-05-02 & 2018 & 3 & POL
Frame & low-medium & National & 500-1000 & 1.0448136 & 1.8253861 & -0.2867631 & -1.2581601 & -1.2725643 & 0.0 & -1.1973642 & 0.9927066 & Recipient & Domestic & Domestic & Domestic & Domestic|POL & Neutral\\
Poland & https://www.wprost.pl/kraj/10121831/kwiecinski-o-budzecie-ue-czekaja-nas-nielatwe-rozmowy.html & 663 & WPROST.pl & Private/Non-Public & Online and Offline & National & very low = CP mentioned once & Institutional bargaining over funding & Factual & National & No myth & NA & NA & NA & NA & NA & NA & NA & NA & Poland & kwieciński o budżecie ue: czekają nas niełatwe rozmowy & 2018-05-02 & polityka spójności & minister inwestycji i rozwoju na specjalnie zwołanej konferencji mówił o propozycji budżetu unijnego na lata 2021-2027, którą przedstawiła dziś komisja europejska. - widać, że czekają nas niełatwe rozmowy w najbliższych miesiącach - stwierdził jerzy kwieciński. jerzy kwieciński wyjaśnił, że na razie nie są znane szczegóły dotyczące tego, jak będzie wyglądał podział środków na poszczególne kraje. mamy dowiedzieć się tego pod koniec maja, kiedy pojawią się propozycje rozporządzeń unijnych. - co dla polski jest ważne - po pierwsze, udało nam się to, że polityka spójności i wspólna polityka rolna mają silne odzwierciedlenie w tym budżecie. również udało nam się zażegnać pewne zagrożenia, które się wcześniej pojawiały, a mianowicie dot. znaczących cięć w tych najważniejszych politykach kosztem finansowania tych nowych priorytetów. mamy propozycje zwiększenia wkładu krajów członkowskich do budżetu. mamy jednocześnie deklarację ze strony pe, który chciałby widzieć jeszcze większe kontrybucje z krajów członkowskich, sięgające nawet 1,3 proc. dochodu narodowego brutto - mówił. dodał przy tym jednak, że czekają nas "niełatwe rozmowy". - kto wie, czy to nie będą w ogóle najtrudniejsze rozmowy. jednocześnie debata, dyskusja, która będzie miała duże znaczenie nt. przyszłości ue - mówił. odnosząc się do propozycji dotyczących praworządności minister zaznaczył, ze nie dotyczy polityki spójności ani żadnego konkretnego kraju. - my jesteśmy za jak najbardziej transparentną polityką finansową ue. w związku z tym jeżeli wprowadzamy zasady praworządności, to też muszą być określone bardzo konkretne kryteria, które będą stosowane do tego określenia, czy dany kraj spełnia czy nie kryteria praworządności - stwierdził. & 236 & very low & Low & Power & NA & NA & 2018-05-02 & 2018 & 3 & POL
Frame & v.low & National & <500 & 1.0448136 & 1.8253861 & -0.2867631 & -1.2581601 & -1.2725643 & 0.0 & -1.1973642 & 0.9927066 & Recipient & Domestic & Domestic & Domestic & Domestic|POL & Neutral\\
Poland & http://fakty.interia.pl/raporty/raport-unia-europejska/polska-w-ue/news-projekt-budzetu-ue-kwoty-zalezne-od-przestrzegania-praworzad,nId,2576473 & 156 & fakty.interia.pl & Private/Non-Public & Online only & National & medium = CP is important part of story & Political leverage & Factual & EU & No myth & NA & NA & NA & NA & NA & NA & NA & NA & Poland & projekt budżetu ue. kwoty zależne od przestrzegania praworządności? & 2018-05-02 & polityka spójności & ke przedstawiła projekt budżetu ue na lata 2021-2027. całkowita suma zobowiązań ma wynieść w tym okresie 1,279 biliona euro. ke zaproponowała także uzależnienie wypłaty funduszy od przestrzegania praworządności. obecny siedmioletni budżet wynosi ok. 1,1 biliona euro. w liczbach bezwzględnych kolejna siedmiolatka będzie zatem większa i to mimo brexitu, ale porównania takie nie biorą pod uwagę inflacji. reklama z informacji podanych w środę przez ke wynika, że w wieloletnich ramach finansowych na lata 2021-2027 przewidziano wynoszące około 7 proc. cięcia w polityce spójności i około 5 proc. - cięcia we wspólnej polityce rolnej (wpr). ponadto polityka spójności ma odgrywać ważniejszą rolę we wspieraniu reform strukturalnych i długoterminowej integracji migrantów. dział dotyczący wydatków na wyrównywanie różnic między regionami ue został zatytułowany "spójność i wartości". przewidziano w nim 442 mld euro, z czego na gospodarczą, społeczną i terytorialną spójność ma iść 373 mld euro (wszystko w cenach bieżących). w obecnej perspektywie budżetowej na cały okres 2014-2020 samej tylko polsce przyznano 82 mld euro w ramach polityki spójności oraz około 31 mld euro w ramach wpr. praworządność a finanse ke zaproponowała też uzależnienie wypłaty funduszy od przestrzegania praworządności. w komunikacie służb prasowych nazwano to "najważniejszą zmianą" w ramach proponowanego budżetu. nowy instrument, do przyjęcia którego nie będzie potrzebna jednomyślność, ma chronić unijną kasę przed "ryzykiem finansowym związanym z uogólnionymi brakami w zakresie praworządności w państwach członkowskich". narzędzie to ma umożliwić zawieszanie i zmniejszanie finansowania ze środków ue lub ograniczenie dostępu do nich "w sposób proporcjonalny do charakteru, wagi i skali braków w zakresie praworządności". wniosek w sprawie takiej decyzji będzie składany przez komisję i przyjmowany przez państwa członkowskie, przy czym jednomyślność nie będzie konieczna. propozycja wieloletniego budżetu zakłada zwiększenie składek narodowych. w przypadku zobowiązań mają one wynieść w sumie 1,11 proc. dochodu narodowego brutto (dnb) państw ue, a w przypadku płatności - 1,08 proc. dnb. jednak projekt przewiduje też sumy, które nie są wliczane do limitu wieloletnich ram finansowych, m.in. na rezerwę kryzysową czy fundusz solidarnościowy. łącznie z tymi środkami wydatki ue mają w latach 2021-2027 sięgać 1,14 proc. dnb państw ue. dla porównania w obecnej perspektywie unijny budżet stanowi ok. 1 proc. dnb. z wyliczeń ke wynika, że zobowiązania budżetowe ue w cenach z 2018 r. (czyli przy uwzględnieniu inflacji) w latach 2021-2027 mają wynieść 1 135 mld euro. środki na płatności przewidziano na poziomie 1 105 mld euro. płatności to sumy, które mają być realnie wydane w danym roku, a zobowiązania to kwoty, które ue planuje pokryć w przyszłości. zgodnie z zapowiedziami ke zaproponowała osobną część w wieloletnich ramach finansowych na migrację i zarządzanie granicami, a także bezpieczeństwo i obronę. na ten pierwszy cel zarezerwowane ma być ponad 34 mld euro, a na drugi 27,5 mld euro (w cenach bieżących, czyli bez inflacji). po wydatkach na politykę spójności drugim największym działem unijnego budżetu będzie polityka rolna. na płatności bezpośrednie i inne powiązane z rynkiem wydatki ma pójść 286 mld euro. propozycja komisji przewiduje też zapewnienie większej elastyczności wewnątrzprogramowej i międzyprogramowej, wzmocnienie instrumentów zarządzania kryzysowego i utworzenie nowej "unijnej rezerwy" na wypadek nieprzewidzianych wydarzeń, aby ue mogła podejmować działania w sytuacjach kryzysowych w dziedzinach takich jak bezpieczeństwo i migracja. zgodnie z wcześniejszymi zapowiedziami ke proponuje też ukierunkowany głównie na strefę euro nowy program wsparcia reform z budżetem w wysokości 25 mld euro. ponadto kraje, które przymierzają się do wejścia do obszaru wspólnej waluty, będą miały możliwość skorzystania z instrumentu wsparcia konwergencji. nowe zasoby własne innym instrumentem pomyślanym pod kątem eurolandu ma być europejski instrument stabilizacji inwestycji, który ma pomagać w utrzymaniu inwestycji w razie kryzysów. "jego funkcjonowanie rozpocznie się w formie pożyczek wzajemnych w ramach budżetu ue, do wysokości 30 mld euro, w połączeniu z pomocą finansową dla państw członkowskich na pokrycie kosztów odsetek" - wyjaśnia ke. propozycja budżetu na lata 2021-2027 przewiduje też nowe zasoby własne ue. 20 proc. dochodów z systemu handlu uprawnieniami do emisji, które obecnie trafiają do państw członkowskich, miałoby być kierowane do unijnej kasy. ke chce też, by państwa ue płaciły 0,80 euro za kilogram niepoddawanych recyklingowi odpadów opakowaniowych z tworzyw sztucznych. nowym źródłem dochodu ma być też 3-procentowa stawka poboru mająca zastosowanie do nowej wspólnej skonsolidowanej podstawy opodatkowania osób prawnych (ma być dopiero stopniowo wprowadzana, bo na razie nie ma przepisów). z wyliczeń ke wynika, że te nowe zasoby własne będą stanowić ok. 12 proc. łącznego budżetu ue i mogą wnieść nawet 22 mld euro rocznie na potrzeby finansowania nowych priorytetów. komisja zaproponowała też zlikwidowanie wszystkich rabatów oraz zmniejszenie z 20 do 10 proc. odsetka dochodów z ceł (jednego z zasobów własnych), który państwa członkowskie zatrzymują podczas ich poboru dla budżetu ue. w najbliższych tygodniach komisja przedstawi szczegółowe propozycje dotyczące przyszłych programów finansowych dla poszczególnych sektorów. do przyjęcia wieloletnich ram finansowych potrzebna jest jednomyślność wszystkich państw ue, jednak do towarzyszących im poszczególnych rozporządzeń już nie. & 800 & medium & Medium & Power & NA & NA & 2018-05-02 & 2018 & 3 & POL
Frame & low-medium & National & 500-1000 & 1.0448136 & 1.8253861 & -0.2867631 & -1.2581601 & -1.2725643 & 0.0 & -1.1973642 & 0.9927066 & Recipient & European & European & European & European|POL & Neutral\\
\addlinespace
Poland & https://plus.dziennikzachodni.pl/wiadomosci/a/polityka-spojnosci-metropolii-biedniejsze-gminy-zasluguja-na-wiecej-wiec-dostana-60-mln-zl,12867896 & a11 & plus.dziennikzachodni.pl & Private/Non-Public & Online and Offline & Regional/Local & high = CP is most important issue in story (can also cover other issues) & Economic development & Positive & Subnational & No myth & Solidarity to poor countries/regions & Positive & Subnational & No myth & NA & NA & NA & NA & Poland & polityka spójności metropolii: biedniejsze gminy zasługują na więcej więc dostaną 60 mln zł & 2018-01-22 & polityka spójności & 60 mln zł dla miast z funduszu solidarności górnośląsko-zagłębiowskiej metropolii. co najmniej 60 mln zł zainwestują nasze miasta w przedsięwzięcia o znaczeniu metropolitalnym. trwają uzgodnienia dotyczące podziału pieniędzy z funduszu solidarności - pierwszego projektu budowania polityki spójności w aglomeracji. polityka spójności, czyli, najprościej: systemowe wspieranie słabszych, mniejszych i biedniejszych, w zamyśle miała być jednym z wyzwań górnośląsko-zagłębiowskiej metropolii. w samym jej centrum są bowiem obszary o ogromnych dysproporcjach rozwojowych - bytom i gliwice, mysłowice i tychy, świętochłowice i katowice. w pierwszym budżecie g-zm zapisano 100 mln zł na fundusz solidarności. ostatecznie, w tym roku samorządy będą mogły wydać co najmniej 60 mln zł. pozostałe 40 mln jest traktowane jako rezerwa, jeśli podczas planowania pojawi się więcej projektów priorytetowych. pieniądze zostały po równo, po 12 mln zł, rozdysponowane pomiędzy 5 podregionów metropolii i na tym poziomie będą podzielone na poszczególne gminy. - to wsparcie dla biedniejszych i słabszych gmin, by się mogły zbliżać standardem usług oferowanych mieszkańcom do swoich bogatszych sąsiadów - mówi kazimierz karolczak, przewodniczący zarządu metropolii. bogatsi sąsiedzi zrezygnowali solidarnie z udziału w podziale pieniędzy. taką decyzję podjęły m.in. katowice, gliwice, zabrze i tychy. - po to utworzyliśmy ten fundusz, by korzystały z niego wyłącznie gminy, które mają gorszą sytuację finansową i duże potrzeby rozwojowe. chodzi o wyrównywanie szans w całej metropolii - mówi marcin krupa, prezydent katowic. - siłą metropolii jest współpraca - dlatego mam nadzieję, że inne gminy dobrze i wspólnie wykorzystają pieniądze. & 232 & high & High & Socio-Economic & Values & NA & 2018-01-22 & 2018 & 3 & ECO
Frame & high-very high & Regional & <500 & 1.0448136 & 1.8253861 & -0.2867631 & -1.2581601 & -1.2725643 & 0.0 & -1.1973642 & 0.9927066 & Recipient & Domestic & Domestic & Domestic & Domestic|ECO & Positive\\
Poland & https://echodnia.eu/podkarpackie/sa-pieniadze-dla-domow-kultury-bibliotek-i-zabytkow/ar/c3-10666826 & a27 & https://echodnia.eu/podkarpackie/ & Private/Non-Public & Online and Offline & Regional/Local & high = CP is most important issue in story (can also cover other issues) & Cultural heritage & Positive & Subnational & No myth & Bureaucracy and/or delays & Balanced & Subnational & No myth & NA & NA & NA & NA & Poland & są pieniądze dla domów kultury, bibliotek i zabytków & 2016-09-25 & fundusze europejskie & -wszystkie te placówki, jak również obiekty dotąd z kulturą niezwiązane, za unijne pieniądze mogą zostać rozbudowane, dostosowane do nowych funkcji i wyposażone w sprzęt pozwalający uatrakcyjnić ich ofertę. ważne, by tworzyły w kulturze nową jakość, na przykład poprzez organizowanie nowych wydarzeń i imprez, nowego sposobu prezentacji wystaw oraz elementów dedykowanych nowym grupom odbiorców. tak, by jak najwięcej mieszkańców chciało z dóbr kultury korzystać.
-ochrona dziedzic¬twa narodowego, a także edukacja kulturalna młodego pokolenia to dwa najważniejsze zadania, które stoją dziś przed „ludźmi kultury”. ich realizacja wymaga nie tylko żmudnej pracy, ale też ogromnych nakładów finansowych. w tym ostatnim z pomocą przychodzą pieniądze unijne. & 103 & high & High & Socio-Economic & Governance & NA & 2016-09-25 & 2016 & 2 & ECO
Frame & high-very high & Regional & <500 & 1.0448136 & 1.8253861 & -0.2867631 & -1.2581601 & -1.2725643 & 0.0 & -1.1973642 & 0.9927066 & Recipient & Domestic & Domestic & Domestic & Domestic|ECO & Positive\\
Poland & http://www.money.pl/gospodarka/wiadomosci/artykul/die-zeit-uzaleznic-fundusze-unijne-od\%2C168\%2C0\%2C1964712.html & 558 & WP money & Private/Non-Public & Online only & National & medium = CP is important part of story & Institutional bargaining over funding & Negative & Other country & No myth & NA & NA & NA & NA & NA & NA & NA & NA & Poland & "die zeit": uzależnić fundusze unijne od postawy wobec uchodźców & 2015-11-26 & fundusze strukturalne & tygodnik "die zeit" apeluje do niemieckiego rządu, aby uzależnił fundusze strukturalne dla krajów europy środkowej i wschodniej od ich gotowości do kompromisu w kwestii uchodźców. w razie konieczności "trzeba warszawę zmusić" - pisze mariam lau. komentatorka lewicowo-liberalnego tygodnika pisze, że pomysł ustanowienia europejskich kontyngentów dla uchodźców imigrujących do europy jest dobrym rozwiązaniem. realizacja tego projektu zależy jednak od europejskiej solidarności - zastrzega lau. problem uchodźców nie jest tylko problemem niemieckim, lecz europejskim - podkreśla autorka komentarza zatytułowanego "no to trzeba warszawę zmusić". "najwyższy czas, aby niemiecki rząd pozbył się obaw przed połączeniem polityki finansowej z polityką migracyjną" - pisze lau, dodając, że "wszystko ze wszystkim się łączy". "to absurd, że kraje wschodnioeuropejskie traktują 'molocha brukselę' z szyderstwem i nienawiścią, a równocześnie na wielką skalę biorą środki strukturalne na rozwój" - zaznacza niemiecka dziennikarka. "to jasne, że przyznanie się do finansowej zależności nie pasuje do wrogiej wobec obcych postawy macho, demonstrowanej na węgrzech i ostatnio także w polsce. nadszedł być może czas, aby przypomnieć o tym (węgrom i polakom) z zewnątrz" - czytamy w "die zeit". "polska, węgry, kraje bałtyckie i słowacja są od 10 lat członkami unii europejskiej. w tych latach ue płaciła im miliardy na poprawę ich niskiego poziomu życia, na unowocześnienie ich fatalnej infrastruktury i na sprawienie, by ich gospodarki stały się bardziej konkurencyjne. kraje te oczekują też słusznie europejskiej solidarności, w tym także wsparcia wojskowego przeciwko rosji" - pisze lau. komentatorka zwraca uwagę, że zapewnienie schronienia uchodźcom należy do fundamentów ue. przypomina, że to właśnie dysydenci z europy wschodniej korzystali w czasach zimnej wojny z tego prawa. jak dodaje, ludzie pokroju lecha wałęsy ciągle o tym pamiętają. "nowa szefowa rządu w jego (wałęsy) kraju, beata szydło, usunęła natomiast teraz flagę europejską z pomieszczenia, gdzie organizuje swoje cotygodniowe konferencje prasowe. pozostała tam tylko flaga polska. pomoc z brukseli jest jednak nadal chętnie brana - proszę tylko przekazywać w sposób nierzucający się w oczy, nie chcemy, by łączono te dwie rzeczy" - czytamy w "die zeit". "coraz bardziej obco brzmią odwołania się do chrześcijaństwa, które najwidoczniej rozumiane są przede wszystkim jako nienawiść do gejów i do zdemoralizowanej europy" - pisze lau. w kryzysie finansowym wszyscy domagali się solidarności od najsilniejszej gospodarki europy - niemiec. "nie ma w tym nic zdrożnego, jeżeli nie zostawia się niemców samym sobie w chwili, gdy chodzi o ludzi, a nie o drachmy" - podsumowuje mariam lau w tygodniku "die zeit". wydawany w hamburgu "die zeit" należy do czołówki niemieckich tygodników. & 395 & medium & Medium & Power & NA & NA & 2015-11-26 & 2015 & 1 & POL
Frame & low-medium & National & <500 & 1.0448136 & 1.8253861 & -0.2867631 & -1.2581601 & -1.2725643 & 0.0 & -1.1973642 & 0.9927066 & Recipient & European & European & European & European|POL & Negative\\
Poland & https://plus.dziennikzachodni.pl/wiadomosci/a/debata-pod-patronatem-dz-silne-katowice-to-silna-metropolia,12879232 & a12 & plus.dziennikzachodni.pl & Private/Non-Public & Online and Offline & Regional/Local & high = CP is most important issue in story (can also cover other issues) & Economic development & Positive & Subnational & No myth & NA & NA & NA & NA & NA & NA & NA & NA & Poland & debata pod patronatem dz: silne katowice to silna metropolia & 2018-01-25 & polityka spójności & kto komu jest bardziej potrzebny - silne katowice metropolii czy odwrotnie? - dyskutowali wczoraj przedstawiciele władz górnośląsko-zagłębiowskiej metropolii i stolicy aglomeracji oraz obecni na spotkaniu mieszkańcy. środowe spotkanie w sali konferencyjnej centrum informacji naukowej i bibliotece akademickiej w katowicach było pierwszą z cyklu debat "tu metropolia" poświęconych przyszłości zgromadzenia. - metropolia, katowice, jak i inne jej gminy, są to naczynia połączone. nie po to staraliśmy się o utworzenie metropolii na obszarze górnego śląska, aby dzisiaj mówić kto ma być ważniejszy. z punktu widzenia biznesowego mają ugrać i metropolia, i katowice - mówił dr marcin krupa, prezydent stolicy woj. śląskiego. podkreślał też, że wspólnota współpracujących ze sobą miast i gmin, jest bardziej znaczącym podmiotem w rozmowach z inwestorami. wszak to już miejsce, w którym mieszka ponad dwa miliony osób, a nie jak w katowicach - "zaledwie" 300 tys. kazimierz karolczak, przewodniczący zarządu metropolii, podkreślał, że jest ona wartością dodaną dla wszystkich miast i gmin ją tworzących. - katowice bardzo mocno czerpią ze swojego otoczenia, a metropolia jest m.in. po to, aby to jeszcze bardziej ułatwiać - mówił. jednym z głównych wyzwań metropolii ma być polityka spójności, której gwarancją będzie m.in. fundusz solidarnościowy metropolii - mający wspierać słabsze, mniejsze i biedniejsze gminy, a który lada moment zostanie uchwalony. i na ten aspekt zwrócił szczególną uwagę prof. tomasz pietrzykowski, prorektor uniwersytetu śląskiego i członek władz obserwatorium procesów miejskich i metropolitalnych. - taka świadoma polityka spójności jest niezwykle potrzebna. dziś bowiem metropolia ma dwa cele - przyspieszenie rozwoju, ale i pilnowanie tego, aby nikt nie został za bardzo z tyłu - podkreślał. nie zabrakło też głosów ze strony publiczności. przybyli na spotkanie mieszkańcy zadawali pytania chętnie i sypali pomysłami jak z rękawa. jednym z ciekawszych była propozycja wprowadzenia wspólnej taryfy biletowej komunikacji miejskiej dla bezrobotnych mieszkańców gzm. "tu metropolia" - przyjdź, posłuchaj, zabierz głos - to hasło przewodnie cyklu debat o sprawach związanych z górnośląsko-zagłębiowską metropolią. spotkania, na które wstęp jest wolny, będą się odbywać co dwa tygodnie w różnych miastach tworzących metropolię. zgodnie z hasłem - będzie można na nich nie tylko posłuchać, co mówią eksperci, politycy i zaproszeni goście, ale również samemu zabrać głos w nurtujących nas kwestiach. & 346 & high & High & Socio-Economic & NA & NA & 2018-01-25 & 2018 & 3 & ECO
Frame & high-very high & Regional & <500 & 1.0448136 & 1.8253861 & -0.2867631 & -1.2581601 & -1.2725643 & 0.0 & -1.1973642 & 0.9927066 & Recipient & Domestic & Domestic & Domestic & Domestic|ECO & Positive\\
Poland & http://fakty.interia.pl/raporty/raport-unia-europejska/polska-w-ue/news-nowy-pomysl-ke-szef-msz-polska-nie-zgodzi-sie-na-taki-warian,nId,2514458 & 336 & fakty.interia.pl & Private/Non-Public & Online only & National & high = CP is most important issue in story (can also cover other issues) & Political leverage & Negative & National & No myth & NA & NA & NA & NA & NA & NA & NA & NA & Poland & nowy pomysł ke. szef msz: polska nie zgodzi się na taki wariant & 2018-01-26 & fundusze strukturalne & polska na pewno nie zgodzi się na uzależnienie wypłacania funduszy unijnych od kryteriów praworządności w państwach członkowskich ue - powiedział w piątek szef msz jacek czaputowicz. ocenił, że ue wykorzystuje środki strukturalne jako "mechanizm nacisku". czaputowicz został spytany w piątek w radiu wnet o doniesienia, według których komisja europejska pracuje nad "uzależnieniem wypłacania środków ue od kryteriów praworządności". "na taki wariant polska na pewno się nie zgodzi. nie sądzę, że jest to zgodnie z zasadami ue. fundusze strukturalne to nie jest jakaś dobroczynność; to metoda wyrównywania startu, wypłacana według określonych zasad dochodów w danym państwie" - powiedział czaputowicz. reklama unijna komisarz ds. sprawiedliwości vera jourova potwierdziła w środę w brukseli, że komisja europejska pracuje nad sposobem uzależnienia wszystkich środków ue od funkcjonowania skutecznego wymiaru sprawiedliwości oraz praworządności w krajach członkowskich. "to logiczne, że jeśli wykorzystujemy pieniądze zebrane od europejskich podatników, musi być gwarancja, że mamy niezależny wymiar sprawiedliwości i praworządność" - powiedziała jourova na konferencji prasowej w brukseli. szef msz zwrócił uwagę, że po akcesji do ue, polska otworzyła swój rynek dla firm zagranicznych, które - jak mówił - "czerpią (z tego) znaczne zyski". "za to polska otrzymała prawo do korzystania z pewnej rekompensaty w postaci funduszy strukturalnych" - dodał. czaputowicz ocenił, że z uwagi na szybkie tempo rozwoju polski, unijne wsparcie będzie w kolejnych latach coraz mniejsze. czaputowicz przypomniał ponadto, że polska nie tylko otrzymuje środki z budżetu ue, ale też wpłaca do niego składki. "to są miliardy złotych. to nie jest żadna działalność charytatywna, czy łaska, że otrzymujemy te fundusze" - oświadczył. "martwi mnie, że ue traktuje to jako pewien mechanizm nacisku, czy wręcz szantażu" - dodał. zdaniem czaputowicza, uzależnienie wypłacania funduszy unijnych od stanu praworządności "jest niemożliwe", bo - jak mówił - "nie zależy od tego, jak państwa członkowskie się reformują, tylko od obiektywnych kryteriów". "nie można zabrać żadnemu państwu tych funduszy" - ocenił. podkreślił też, że polska "nie podda się narracji, że będzie można wywierać w tym zakresie presję"; przypomniał, że pod kierunkiem premiera mateusza morawieckiego powstaje "biała księga", w której rząd ustosunkowuje się do zarzutów komisji europejskiej. według czaputowicza, zarzuty ke wobec polski są "niesłychanie niesprawiedliwie". komisja europejska w grudniu ub.r. zdecydowała o uruchomieniu wobec polski procedury z art. 7 traktatu unijnego. wiceszef komisji frans timmermans, który poinformował o tej decyzji, podkreślił, że ke daje polsce trzy miesiące na wprowadzenie rekomendacji dotyczących praworządności. & 378 & high & High & Power & NA & NA & 2018-01-26 & 2018 & 3 & POL
Frame & high-very high & National & <500 & 1.0448136 & 1.8253861 & -0.2867631 & -1.2581601 & -1.2725643 & 0.0 & -1.1973642 & 0.9927066 & Recipient & Domestic & Domestic & Domestic & Domestic|POL & Negative\\
\addlinespace
Poland & https://gazetawroclawska.pl/dotacje-unijne-z-przeszkodami/ar/c3-10192574 & a33 & https://gazetawroclawska.pl/ & Private/Non-Public & Online and Offline & Regional/Local & very high = CP is most important issue + CP is mentioned in title/headline & Bureaucracy and/or delays & Negative & EU + Subnational & No myth & NA & NA & NA & NA & NA & NA & NA & NA & Poland & dotacje unijne z przeszkodami & 2014-10-30 & fundusz rozwoju regionalnego & -jeśli chodzi o dofinansowanie z programu operacyjnego kapitał ludzki 2007-2013, to we wrocławiu zrealizowano 69 umów na łączną kwotę ponad 103 mln zł, z czego dofinansowanie wyniosło ponad 96, 8 mln zł. na jednego mieszkańca przypadło 153,16 złotych dotacji.
w pozyskiwaniu funduszy europejskich występują jednak prze\_x001f\_szkody. w trakcie wdrażania po kl zaobserwowano, że wnioskodawcy postrzegają realizację projektów z pomocą publiczną jako trudną. -zniechęca ich to do realizacji takich przedsięwzięć, nawet jeśli istnieje realna potrzeba ich przeprowadzenia. & 78 & very high & High & Governance & NA & NA & 2014-10-30 & 2014 & 1 & POL
Frame & high-very high & Regional & <500 & 1.0448136 & 1.8253861 & -0.2867631 & -1.2581601 & -1.2725643 & 0.0 & -1.1973642 & 0.9927066 & Recipient & Domestic & European & Mixed & Domestic|POL & Negative\\
Poland & http://www.pap.pl/aktualnosci/news,1262984,premierzy-v4-o-przyszlosci-ue-silna-europa-silnych-panstw.html & 408 & pap.pl & Public & Online only & National & very low = CP mentioned once & Territorial cooperation & Positive & National + Other country & No myth & NA & NA & NA & NA & NA & NA & NA & NA & Poland & premierzy v4 o przyszłości ue: silna europa silnych państw & 2018-01-26 & polityka spójności & silną europę mogą tworzyć tylko silne państwa członkowskie, a zastanawiając się nad przyszłością ue należy zachować już osiągnięte rezultaty - ocenili premierzy państw grupy wyszehradzkiej w oświadczeniu przyjętym w piątek w budapeszcie. "silna i efektywna ue leży w naszym interesie. powinniśmy zachować i umocnić jedność unii, respektując przy tym nasze wspólne europejskie wartości, tożsamość i specyfikę państw członkowskich. silną europę mogą tworzyć tylko silne państwa członkowskie, wspierane przez skuteczne instytucje ue wykonujące swe zadania w oparciu o swe kompetencje zdefiniowane w traktatach" - podkreślili premierzy v4, zaznaczyli przy tym, że instytucje ue powinny traktować wszystkie państwa członkowskie jednakowo i należy respektować ich prawo do przeprowadzania wewnętrznych reform zgodnie ich kompetencjami. premierzy podkreślili, że myśląc nad przyszłością ue, trzeba zachować już osiągnięte rezultaty, zdobyte wysiłkami poprzednich pokoleń. "podstawowe osiągnięcia należy utrzymać. musimy przywrócić odpowiednie funkcjonowanie schengen, jak również odzyskać pełną kontrolę nad granicami zewnętrznymi. musimy też ochronić i dalej rozwijać wspólny rynek oparty na podstawowych wolnościach, w tym wolności przemieszczania się pracowników i usług" - ocenili premierzy. jak zaznaczyli, należy też utrzymać możliwość poszerzenia unii o bałkany zachodnie i zwiększyć zaangażowanie ue w pomaganie wschodnim sąsiadom na drodze do osiągnięcia europejskich standardów. w opinii premierów v4, zastanawiając się nad kolejnymi posunięciami dotyczącymi przyszłości ue, należy wychodzić od poszanowania istniejących ram prawnych. jak ocenili szefowie rządów, najważniejszą podstawą dobrze funkcjonującego, demokratycznego i zgodnego z prawem projektu europejskiego jest "międzyinstytucjonalna równowaga zapisana w traktatach". podkreślili, że komisja europejska powinna nadal pełnić kluczową rolę w określaniu ogólnych kierunków politycznych i priorytetów, w tym co do przyszłości ue, ale "decyzji podejmowanych przez szefów państw i rządów nie wolno lekceważyć na niższych szczeblach procesu podejmowania decyzji". "jedność ue będzie naszym głównym celem. (...) powinniśmy zmniejszyć dystans między europejskimi obywatelami a instytucjami w brukseli i jesteśmy gotowi do szerokiej dyskusji publicznej na temat naszej europejskiej przyszłości zgodnie z praktykami narodowymi" - napisali premierzy. podkreślili też, że należy wzmocnić konkurencyjność ue w wymiarze zarówno wewnętrznym, jak i globalnym. "w naszej opinii polityka spójności oraz konkurencja na wspólnym rynku przyczyniają się do wzmocnienia tak pożądanej społecznej i gospodarczej konwergencji między państwami członkowskimi, co jest korzystne dla ue jako całości" - ocenili. opowiedzieli się wreszcie za kompleksowym podejściem do polityki migracyjnej. "nasze doświadczenia pokazały, że tylko te rozwiązania, które zostały przyjęte w drodze konsensusu, przynoszą najlepsze wyniki w praktyce i są w stanie skutecznie zaradzić kryzysowi" - napisali. podkreślili przy tym, że rozwiązanie tego problemu powinno mieć na celu "nie dystrybucję, tylko zapobieganie presji migracyjnej w europie". oświadczenie podpisali premierzy: węgier - viktor orban, polski - mateusz morawiecki, słowacji - robert fico i czech - andrej babisz. & 416 & very low & Low & Socio-Economic & NA & NA & 2018-01-26 & 2018 & 3 & ECO
Frame & v.low & National & <500 & 1.0448136 & 1.8253861 & -0.2867631 & -1.2581601 & -1.2725643 & 0.0 & -1.1973642 & 0.9927066 & Recipient & Domestic & European & Mixed & Domestic|ECO & Positive\\
Poland & http://www.pb.pl/3926784\%2C22491\%2Cprawie-7-mld-eur-nieprawidlowych-wydatkow-z-budzetu-ue-w-2013-r?utm\_source=tag\_\&utm\_medium=rss & 214 & pb.pl & Private/Non-Public & Online and Offline & National & medium = CP is important part of story & Mismanagement & Negative & EU + National + Subnational & No myth & NA & NA & NA & NA & NA & NA & NA & NA & Poland & prawie 7 mld eur nieprawidłowych wydatków z budżetu ue w 2013 r. & 2014-11-05 & polityka regionalna & kontrolujący wydatki unijne europejski trybunał obrachunkowy wyliczył, że z ubiegłorocznego budżetu ue wydano nieprawidłowo prawie 7 mld euro. gdyby nie interwencje ke i państw członkowskich nieprawidłowych wydatków byłoby o ponad 2 mld euro więcej. w środę europejski trybunał obrachunkowy (eto) opublikował raport na temat budżetu ue w 2013 r. kontrolerzy zaaprobowali sprawozdanie finansowe podkreślając jednocześnie, że zarządzanie wydatkami wymaga poprawy zarówno na poziomie ue, jak i w państwach członkowskich. w 2013 r. wydatki z budżetu ue wyniosły 148,5 mld euro, z czego niezgodnie z przepisami rozdysponowano 4,7 proc. gdyby nie podjęto działań korygujących, mających na celu odzyskanie środków, poziom błędu w wydatkach wynosiłby 6,3 proc., czyli 9,3 mld euro. audytorzy unijni wskazali, że w poprzedniej perspektywie finansowej obejmującej lata 2007-2013 więcej uwagi poświęcano wydatkowaniu środków na zasadzie "wykorzystać, bo przepadną", niż uzyskiwaniu dobrych wyników z realizacji programów. ich zdaniem przy wyborze projektów, które miały zostać objęte dofinansowaniem ue, w pierwszej kolejności skupiano się na wydatkowaniu dostępnych środków unijnych, następnie - na zgodności z przepisami, a dopiero potem, i to w ograniczonym zakresie - na rezultatach i oddziaływaniu. "komisja europejska i państwa członkowskie muszą zwrócić większą uwagę na to, jak wydatkują pieniądze podatników" - oświadczył prezes trybunału vitor caldeira. największy poziom błędów w zeszłorocznych wydatkach wystąpił w obszarach: "polityka regionalna, transport i energia" (szacowany poziom błędu 6,9 proc.) oraz "rozwój obszarów wiejskich, środowisko naturalne, rybołówstwo i zdrowie" (6,7 proc.). jednym z przykładów nieprawidłowych wydatków jest składanie przez rolników wniosków o dopłaty na grunty, które się do tego nie kwalifikują. taka sytuacja miała miejsce m.in. w niemczech, irlandii, grecji, rumunii, francji, polsce i rumunii. rolnicy deklarowali (i brali za to pieniądze z ue) jako trwałe użytki zielone grunty, które w rzeczywistości pokryte były gęstymi krzewami lub drzewami. tam, gdzie państwa unijne dzieliły zarządzanie wydatkami z komisją europejską, poziom błędu wyniósł 5,2 proc., tymczasem w programach zarządzanych przez samą ke poziom ten wyniósł 3,7 proc. w raporcie wskazano m.in., że 9 spośród 40 objętych kontrolą polskich grup producenckich, skupiających wytwórców owoców i warzyw, nie spełniło warunków, by otrzymać dofinansowanie. "ke poinformowała, że zgłosiła zastrzeżenie o wartości odpowiadającej 25 proc. ogółu wydatków w ramach obarczonego ryzykiem działania w polsce" - czytamy w raporcie. za właściwe wydatkowanie środków budżetowych odpowiada głównie komisja. w przypadku około 80 proc. wydatków - głównie na rolnictwo i spójność - komisja dzieli to zadanie z państwami członkowskimi ue. kontrolerzy ue stwierdzili, że w przypadku znacznego odsetka wykrytych błędów organy krajowe dysponowały wystarczającymi informacjami, aby skorygować nieprawidłowości przed wystąpieniem do komisji o zwrot kosztów. w raporcie wskazano, że takie działania mogłyby przyczynić się do znacznego obniżenia poziomu błędu, np. z 6,7 proc. do 2,0 proc. w obszarze "rozwój obszarów wiejskich, środowisko naturalne, rybołówstwo i zdrowie". szacowany przez trybunał poziom błędu nie jest miarą nadużyć, braku wydajności czy marnotrawstwa. jest to oszacowanie kwot, które nie powinny były zostać wypłacone z budżetu ue, ponieważ nie zostały wykorzystane zgodnie z przepisami unijnymi. typowe błędy polegają na przykład na wypłacie środków na rzecz podmiotu zaklasyfikowanego jako małe lub średnie przedsiębiorstwo (mśp), które w rzeczywistości stanowi własność dużej firmy, czy na poszerzeniu zakresu udzielonego już zamówienia publicznego bez umożliwienia innym oferentom przedstawienia swoich ofert. europejski trybunał obrachunkowy jest niezależną instytucją kontrolującą finanse ue. został utworzony w 1975 roku i ma siedzibę w luksemburgu. & 544 & medium & Medium & Governance & NA & NA & 2014-11-05 & 2014 & 1 & POL
Frame & low-medium & National & 500-1000 & 1.0448136 & 1.8253861 & -0.2867631 & -1.2581601 & -1.2725643 & 0.0 & -1.1973642 & 0.9927066 & Recipient & Domestic & European & Mixed & Domestic|POL & Negative\\
Poland & http://www.pap.pl/aktualnosci/news,1271845,kwiecinski-opowiadamy-sie-za-utrzymaniem-obecnego-poziomu-polityki-spojnosci.html & 750 & pap.pl & Public & Online only & National & medium = CP is important part of story & Institutional bargaining over funding & Factual & National & No myth & NA & NA & NA & NA & NA & NA & NA & NA & Poland & kwieciński: opowiadamy się za utrzymaniem obecnego poziomu polityki spójności & 2018-02-02 & europejski fundusz społeczny & kwieciński reprezentuje polskę podczas spotkania ministrów ds. polityki spójności krajów v4+4 (grupa wyszehradzka oraz bułgaria, chorwacja, rumunia i słowenia) w budapeszcie. "dla nas najważniejsze jest, aby prezentować wspólne stanowisko tej części ue, grupę wyszehradzką, ale z udziałem bułgarii, rumunii, chorwacji i słowenii. to kraje, które są dużymi beneficjentami polityki spójności i chcielibyśmy, żeby rola polityki spójności była utrzymana w kolejnej perspektywie finansowej" - powiedział kwieciński. zaznaczył, że polityka spójności pozwala na lepszą konwergencję naszych krajów, ale także naszych regionów, do średniego poziomu rozwoju w ue. minister zwrócił uwagę, że nasze państwa leżą na peryferiach ue, a to oznacza, że mają mniej sprzyjające warunki do rozwoju gospodarczego, niż kraje, które są w centrum unii. według niego zapóźniania możemy niwelować, właśnie dzięki wykorzystaniu unijnej polityki spójności. według szefa resortu inwestycji i rozwoju polityka spójności dotyczy wszystkich krajów członkowskich ue. z analiz ministerstwa wynika, że korzystają z niej także kraje, które są tzw. płatnikami netto. "bardzo wiele zleceń jest realizowanych przez firmy z tych krajów. sporo do tych krajów wraca również w postaci zamówień eksportowych, więc cała unia europejska korzysta z tej polityki" - wyjaśnił. kwieciński zwrócił też uwagę, że polityka spójności była bardzo skutecznym narzędziem w okresie spowolnienia gospodarczego, działała jako bufor na napięcia gospodarcze. pytany, o przyszłość polityki spójności po 2020 r. i zagrożenia dla niej, przypomniał, że dwa-trzy lata temu, kiedy rozpoczęła się debata na ten temat, pojawiały się liczne głosy, że ta polityka nie będzie już dłużej potrzebna ue, że trzeba ją zlikwidować. "w tej chwili te obawy nie spełniają się, co nas oczywiście cieszy. nadal są pewne głosy mówiące o większej nacjonalizacji tej polityki, żeby poszczególne kraje członkowskie więcej do niej dokładały niż dotychczas. my w stanowisku, które wypracowujemy w ramach grupy wyszehradzkiej opowiadamy się za utrzymaniem podobnego poziomu dofinansowania" - podkreślił kwieciński. "dla nas najważniejszy jest fakt, że kraje członkowskie ue uznały rolę tej polityki i nie ma już pytań o to, czy będzie ona kontynuowana, ale jak będzie wyglądała po roku 2020. zdajemy sobie sprawę, że ona nie będzie wyglądała tak samo jak do tej pory" - przyznał minister. poinformował, że jest szereg elementów, które powinny zostać zmienione. to stanowisko, które reprezentuje wiele krajów. "przede wszystkim chcemy, żeby ta polityka była znacznie prostsza i znacznie bardziej elastyczna niż do tej pory, żeby bardziej odpowiadała potrzebom poszczególnych krajów, aby łatwiej było ją wdrażać" - wyjaśnił. wskazał, że polska apeluje też, aby zmniejszać różnice między instrumentami wykorzystywanymi w ramach polityki spójności a tymi, które są wykorzystywane na poziomie unijnym. kwieciński powiedział, że znacznie prościej jest bowiem realizować duże projekty infrastrukturalne np. z instrumentu "łącząc europę", niż korzystać ze wsparcia na duże projekty infrastrukturalne w ramach polityki spójności. "zasady zarządzania projektami na poziomie unijnym są prostsze niż te w polityce spójności. będziemy chcieli wyrównania sposobów ich realizacji. to bardzo ważne" - dodał. według kwiecińskiego nowa polityka spójności będzie w większym stopniu stawiała na kwestie konkurencyjności, w szczególności zaś na podniesienie innowacyjności gospodarki ue przez wzrost innowacyjności gospodarek na poziomie krajowym czy regionalnym. "myślę, że wzrośnie rola instrumentów zwrotnych. stopniowo będzie się zwiększała pula środków przeznaczanych na takie instrumenty, choć my się wyraźnie opowiadamy za tym, aby w polityce spójności, również po 2020 r., dominowały dotacje, jako forma, która była stosowana do tej pory" - zaznaczył. kwieciński zapewnił, że do tych postulatów nie musi przekonywać naszych partnerów z v4+4. "mówimy jednym głosem - w zeszłym roku, kiedy przygotowaliśmy wspólne stanowisko, podobnie było w 2016 r. nieprzypadkowo to stanowisko przyjmujemy teraz, najbliższe miesiące będą bowiem okresem najbardziej intensywnych prac w komisji europejskiej nad propozycją budżetu i nowej perspektywy finansowej. ke w maju ma przedstawić swoją propozycje" - poinformował minister. jego zdaniem wagę spotkania w budapeszcie podkreśla fakt, że odbywa się ono z udziałem komisarza guenthera oettingera, który odpowiada za przygotowanie budżetu unijnego po roku 2020 i komisarz marianne thyssen, która odpowiada za instrumenty społeczne w szczególności za europejski fundusz społeczny. kwieciński reprezentuje polskę podczas spotkania ministrów ds. polityki spójności krajów v4+4 (grupa wyszehradzka oraz bułgaria, chorwacja, rumunia i słowenia) w budapeszcie. mają w nim wziąć udział także komisarz ue ds. budżetu guenther oettinger oraz komisarz ds. zatrudnienia, spraw społecznych, umiejętności i mobilności pracowników marianne thyssen. rozmowy w budapeszcie dotyczą przyszłości europejskiej polityki spójności po roku 2020. zakończą się w piątek przyjęciem w tym zakresie wspólnego oświadczenia grupy wyszehradzkiej oraz chorwacji, słowenii i rumunii. w tym gronie nie ma bułgarii, sprawuje bowiem prezydencję w radzie unii europejskiej. (pap) & 721 & medium & Medium & Power & NA & NA & 2018-02-02 & 2018 & 3 & POL
Frame & low-medium & National & 500-1000 & 1.0448136 & 1.8253861 & -0.2867631 & -1.2581601 & -1.2725643 & 0.0 & -1.1973642 & 0.9927066 & Recipient & Domestic & Domestic & Domestic & Domestic|POL & Neutral\\
Poland & http://wgospodarce.pl/informacje/45884-kwiecinski-wiecej-spojnosci-w-polityce-spojnosci-ue?utm\_source=feedburner\&utm\_medium=feed\&utm\_campaign=Feed\%3A+wGospodarce+\%28wGospodarce.pl\%29\&utm\_content=FeedBurner & 403 & wgospodarce.pl & Private/Non-Public & Online only & National & very high = CP is most important issue + CP is mentioned in title/headline & Territorial cooperation & Positive & National + Other country & No myth & NA & NA & NA & NA & NA & NA & NA & NA & Poland & kwieciński: więcej spójności w polityce spójności ue & 2018-02-02 & europejski fundusz społeczny & tagi: biznes budapeszt dotacje grupa wyszehradzka kwieciński polityka polityka spójności spotkanie v4 visegrad wyszehrad w ramach grupy wyszehradzkiej (v4) opowiadamy się za utrzymaniem obecnego poziomu dofinansowania w polityce spójności, choć jej kształt na pewno się zmieni po 2020 r. - powiedział przebywający w budapeszcie minister inwestycji i rozwoju jerzy kwieciński kwieciński reprezentuje polskę podczas spotkania ministrów ds. polityki spójności krajów v4+4 (grupa wyszehradzka oraz bułgaria, chorwacja, rumunia i słowenia) w budapeszcie. dla nas najważniejsze jest, aby prezentować wspólne stanowisko tej części ue, grupę wyszehradzką, ale z udziałem bułgarii, rumunii, chorwacji i słowenii. to kraje, które są dużymi beneficjentami polityki spójności i chcielibyśmy, żeby rola polityki spójności była utrzymana w kolejnej perspektywie finansowej - powiedział kwieciński. zaznaczył, że polityka spójności pozwala na lepszą konwergencję naszych krajów, ale także naszych regionów, do średniego poziomu rozwoju w ue. minister zwrócił uwagę, że nasze państwa leżą na peryferiach ue, a to oznacza, że mają mniej sprzyjające warunki do rozwoju gospodarczego, niż kraje, które są w centrum unii. według niego zapóźniania możemy niwelować, właśnie dzięki wykorzystaniu unijnej polityki spójności. według szefa resortu inwestycji i rozwoju polityka spójności dotyczy wszystkich krajów członkowskich ue. z analiz ministerstwa wynika, że korzystają z niej także kraje, które są tzw. płatnikami netto. bardzo wiele zleceń jest realizowanych przez firmy z tych krajów. sporo do tych krajów wraca również w postaci zamówień eksportowych, więc cała unia europejska korzysta z tej polityki - wyjaśnił. kwieciński zwrócił też uwagę, że polityka spójności była bardzo skutecznym narzędziem w okresie spowolnienia gospodarczego, działała jako bufor na napięcia gospodarcze. pytany, o przyszłość polityki spójności po 2020 r. i zagrożenia dla niej, przypomniał, że dwa-trzy lata temu, kiedy rozpoczęła się debata na ten temat, pojawiały się liczne głosy, że ta polityka nie będzie już dłużej potrzebna ue, że trzeba ją zlikwidować. w tej chwili te obawy nie spełniają się, co nas oczywiście cieszy. nadal są pewne głosy mówiące o większej nacjonalizacji tej polityki, żeby poszczególne kraje członkowskie więcej do niej dokładały niż dotychczas. my w stanowisku, które wypracowujemy w ramach grupy wyszehradzkiej opowiadamy się za utrzymaniem podobnego poziomu dofinansowania - podkreślił kwieciński. - dla nas najważniejszy jest fakt, że kraje członkowskie ue uznały rolę tej polityki i nie ma już pytań o to, czy będzie ona kontynuowana, ale jak będzie wyglądała po roku 2020. zdajemy sobie sprawę, że ona nie będzie wyglądała tak samo jak do tej pory - przyznał minister. poinformował, że jest szereg elementów, które powinny zostać zmienione. to stanowisko, które reprezentuje wiele krajów. przede wszystkim chcemy, żeby ta polityka była znacznie prostsza i znacznie bardziej elastyczna niż do tej pory, żeby bardziej odpowiadała potrzebom poszczególnych krajów, aby łatwiej było ją wdrażać - wyjaśnił. wskazał, że polska apeluje też, aby zmniejszać różnice między instrumentami wykorzystywanymi w ramach polityki spójności a tymi, które są wykorzystywane na poziomie unijnym. kwieciński powiedział, że znacznie prościej jest bowiem realizować duże projekty infrastrukturalne np. z instrumentu "łącząc europę", niż korzystać ze wsparcia na duże projekty infrastrukturalne w ramach polityki spójności. zasady zarządzania projektami na poziomie unijnym są prostsze niż te w polityce spójności. będziemy chcieli wyrównania sposobów ich realizacji. to bardzo ważne - dodał. według kwiecińskiego nowa polityka spójności będzie w większym stopniu stawiała na kwestie konkurencyjności, w szczególności zaś na podniesienie innowacyjności gospodarki ue przez wzrost innowacyjności gospodarek na poziomie krajowym czy regionalnym. myślę, że wzrośnie rola instrumentów zwrotnych. stopniowo będzie się zwiększała pula środków przeznaczanych na takie instrumenty, choć my się wyraźnie opowiadamy za tym, aby w polityce spójności, również po 2020 r., dominowały dotacje, jako forma, która była stosowana do tej pory - zaznaczył. kwieciński zapewnił, że do tych postulatów nie musi przekonywać naszych partnerów z v4+4. mówimy jednym głosem - w zeszłym roku, kiedy przygotowaliśmy wspólne stanowisko, podobnie było w 2016 r. nieprzypadkowo to stanowisko przyjmujemy teraz, najbliższe miesiące będą bowiem okresem najbardziej intensywnych prac w komisji europejskiej nad propozycją budżetu i nowej perspektywy finansowej. ke w maju ma przedstawić swoją propozycje - poinformował minister. jego zdaniem wagę spotkania w budapeszcie podkreśla fakt, że odbywa się ono z udziałem komisarza guenthera oettingera, który odpowiada za przygotowanie budżetu unijnego po roku 2020 i komisarz marianne thyssen, która odpowiada za instrumenty społeczne w szczególności za europejski fundusz społeczny. kwieciński reprezentuje polskę podczas spotkania ministrów ds. polityki spójności krajów v4+4 (grupa wyszehradzka oraz bułgaria, chorwacja, rumunia i słowenia) w budapeszcie. mają w nim wziąć udział także komisarz ue ds. budżetu guenther oettinger oraz komisarz ds. zatrudnienia, spraw społecznych, umiejętności i mobilności pracowników marianne thyssen. rozmowy w budapeszcie dotyczą przyszłości europejskiej polityki spójności po roku 2020. zakończą się w piątek przyjęciem w tym zakresie wspólnego oświadczenia grupy wyszehradzkiej oraz chorwacji, słowenii i rumunii. w tym gronie nie ma bułgarii, sprawuje bowiem prezydencję w radzie unii europejskiej. & 769 & very high & High & Socio-Economic & NA & NA & 2018-02-02 & 2018 & 3 & ECO
Frame & high-very high & National & 500-1000 & 1.0448136 & 1.8253861 & -0.2867631 & -1.2581601 & -1.2725643 & 0.0 & -1.1973642 & 0.9927066 & Recipient & Domestic & European & Mixed & Domestic|ECO & Positive\\
\addlinespace
Poland & http://www.gazetaprawna.pl/artykuly/1057394,ue-wybrano-nowego-przewodniczacego-komitetu-regionow.html & 101 & gazetaprawna.pl & Private/Non-Public & Online and Offline & National & medium = CP is important part of story & Institutional bargaining over funding & Positive & EU & No myth & NA & NA & NA & NA & NA & NA & NA & NA & Poland & ue:  wybrano nowego przewodniczącego komitetu regionów & 2017-07-12 & polityka spójności & przedstawiciel wspólnoty niemieckojęzycznej w belgijskim senacie karl-heinz lambertz został w środę wybrany na nowego przewodniczącego komitetu regionów ue, organu doradczego unii europejskiej reprezentującego samorządy. "spójność stanowi dna unii europejskiej, nie ma mowy o jej ograniczaniu, umniejszaniu czy też rozwadnianiu. to byłoby absolutnie nonsensem, byłoby nie do przyjęcia" - powiedział lambertz w pierwszym oficjalnym wystąpieniu, odnosząc się do przyszłości funduszy unijnych. jak podkreślił, polityka spójności musi być podtrzymana, ponieważ wspiera spójność społeczną, miejsca pracy czy infrastrukturę, których potrzebują europejczycy. "unia europejska bez polityki spójności to nie jest europa, której pragniemy (...) unia europejska nie ma sensu, jeżeli wszyscy jej członkowie nie będą szli w tym samym kierunku w interesie wszystkich europejczyków. (...) solidarność europejska jest niezbędna" - zaznaczył nowy przewodniczący kr. lambertz w drugiej połowie pięcioletniej kadencji komitetu regionów zastąpi na tym stanowisku fina markku markkulę. jest to wynik porozumienia zawartego przez dwie największe grupy polityczne w kr: partię europejskich socjalistów (pes), w której zasiada belg, i europejską partię ludową (epl). 65-letni lambertz, z wykształcenia prawnik, jest również wiceprzewodniczącym kongresu władz lokalnych i regionalnych rady europy. zanim został senatorem, pełnił funkcję przewodniczącego parlamentu wspólnoty niemieckojęzycznej w belgii. w komitecie regionów zasiada od 1999 r. lambertz jest autorem opinii kr w sprawie reformy unijnych zasad pomocy państwa w odniesieniu do usług świadczonych w ogólnym interesie gospodarczym, a także stanowiska ws. sytuacji pracowników transgranicznych w ue. unijne instytucje - komisja europejska, rada ue i parlament europejski - są zobowiązane do zasięgania opinii komitetu regionów w procesie tworzenia prawa w sprawach dotyczących samorządów, takich jak opieka zdrowotna, edukacja, zatrudnienie, polityka społeczna, transport czy zmiany klimatu. w kr zasiada 350 przedstawicieli władz samorządowych z 28 państw członkowskich wybranych w drodze wyborów; do tego dochodzi taka sama liczba ich zastępców. każdy kraj wyznacza swoich członków, którzy są mianowani na odnawialną pięcioletnią kadencję przez radę ue. liczba przedstawicieli poszczególnych państw zależy od liczby ludności danego kraju; polsce przypada 21 członków i tyle samo zastępców. w kr działa pięć grup politycznych: poza epl i pes są to porozumienie liberałów i demokratów na rzecz europy (alde), przymierze europejskie (ea) oraz europejscy konserwatyści i reformatorzy (ekr). z brukseli mateusz kicka (pap) & 347 & medium & Medium & Power & NA & NA & 2017-07-12 & 2017 & 2 & POL
Frame & low-medium & National & <500 & 1.0448136 & 1.8253861 & -0.2867631 & -1.2581601 & -1.2725643 & 0.0 & -1.1973642 & 0.9927066 & Recipient & European & European & European & European|POL & Positive\\
Poland & http://www.gazetaprawna.pl/artykuly/1051861,bochenek-procedura-przeciwko-polsce-przez-ke-to-zagranie-polityczne.html & 623 & gazetaprawna.pl & Private/Non-Public & Online and Offline & National & very low = CP mentioned once & Political leverage & Negative & National & No myth & NA & NA & NA & NA & NA & NA & NA & NA & Poland & bochenek: wszczęcie procedury przeciwko polsce przez ke to zagranie typowo polityczne & 2017-06-20 & fundusze strukturalne & wszczęcie procedury przeciwko polsce przez komisję europejską to zagranie typowo polityczne, ponieważ zobowiązań relokacyjnych nie wykonują również inne państwa europejskie i ke nie podejmuje żadnych działań przeciwko tym państwom - mówił we wtorek rzecznik rządu rafał bochenek. komisja europejska w minioną środę formalnie wszczęła procedurę w związku z niewywiązywaniem się przez polskę, czechy i węgry ze zobowiązań w sprawie relokacji uchodźców. kraje te mają miesiąc na wystosowanie odpowiedzi do ke. bochenek pytany o tę kwestię w telewizji wp, relacjonował, że rząd przygotowuje odpowiedź, w której odniesie się do zarzutów podnoszonych przez komisję europejską w stosunku do polski. "warto zwrócić uwagę, że jest to zagranie typowo polityczne, ponieważ zobowiązań relokacyjnych nie wykonują również inne państwa europejskie i ke nie podejmuje żadnych działań przeciwko tym państwom" - ocenił. zobacz także:węgry: relokacja migrantów tematem obrad szefów msw forum salzburskiego " "poziom realizacji decyzji relokacyjnej jest na poziomie, dosłownie, kilkunastu procent. do relokacji było przeznaczonych 60 tys. migrantów, relokowanych zostało 20 tys. to pokazuje, jakie jest realne, rzeczywiste zaangażowanie państw europy zachodniej w rozwiązywaniu tego problemu" - wskazał bochenek. rzecznik rządu był pytany również, czy rząd nie boi się konsekwencji finansowych w związku z niewywiązaniem się z zobowiązań relokacyjnych. "oczywiście, że się nie boimy, bo to nie ma żadnego związku" - podkreślił. "te dwie rzeczy należy zdecydowanie od siebie rozdzielić. z jednej strony są fundusze unijne, fundusze strukturalne, na które się godziliśmy i z tytułu których my również ponosimy określone zobowiązania do budżetu unijnego i tutaj z tego polska się również wywiązuje" - wskazał bochenek. zaznaczył, że polityka migracyjna, zgodnie z traktatami, leży w gestii państw członkowskich ue, a nie w gestii instytucji unijnych, które - zdaniem rzecznika rządu - próbują przejąć kompetencje państw narodowych. zobacz także:trójka: politycy o relokacji i konsekwencjach odmowy przyjmowania uchodźców " bochenek odniósł się także do zarzutu prowadzenia polityki zarządzania strachem obywateli przez rząd. "rząd polski nigdy nie utożsamiał uchodźców, migrantów z terrorystami" - odpowiedział. zdaniem bochenka mechanizmy weryfikacji osób napływających do europy nie funkcjonują. "nie wiemy, jakie są powiązania tych osób, które przybywają do bram europy" - wskazywał. "bardzo często te osoby, które przybywają do europy mają po kilka paszportów, albo dokumentów nie mają żadnych, albo dokumenty mają podrobione" - mówił bochenek. według niego, świadczy to o tym, że instytucje państwa, z których migranci napływają do europy nie funkcjonują. "to nie państwo polskie powinno sprawdzać takie rzeczy, tylko są odpowiednie agendy europejskie, które również nie są w stanie z tym poradzić" - podkreślił. pis zapytało komisję europejską w skierowanej do niej interpelacji, dlaczego wszczęła procedury za odmowę relokacji uchodźców tylko przeciw polsce, czechom i węgrom, skoro inne kraje unijne także "w wysokim stopniu" nie wykonują decyzji w tej sprawie. & 430 & very low & Low & Power & NA & NA & 2017-06-20 & 2017 & 2 & POL
Frame & v.low & National & <500 & 1.0448136 & 1.8253861 & -0.2867631 & -1.2581601 & -1.2725643 & 0.0 & -1.1973642 & 0.9927066 & Recipient & Domestic & Domestic & Domestic & Domestic|POL & Negative\\
Poland & http://www.gazetaprawna.pl/artykuly/1346534,pe-o-propozycji-ke-ws-budzetu-na-lata-2021-2027.html & 506 & gazetaprawna.pl & Private/Non-Public & Online and Offline & National & medium = CP is important part of story & Institutional bargaining over funding & Balanced & EU & No myth & NA & NA & NA & NA & NA & NA & NA & NA & Poland & w pe europosłowie krytykują propozycję ke ws. budżetu na lata 2021-2027 & 2018-11-13 & fundusze strukturalne & podczas wtorkowej debaty w pe wielu europosłów różnych ugrupowań krytykowało proponowany przez ke projekt budżetu na lata 2021-2027, sprzeciwiając się m.in cięciom w polityce spójności i rolnictwie. apelowali o przyjęcie propozycji pe. debata poprzedziła głosowanie nad propozycją pe ws. budżetu na lata 2021-2027, które zaplanowano na środę. jeśli propozycja ta zostanie przegłosowana, stanie się oficjalnym stanowiskiem pe ws. przyszłych ram finansowych i punktem wyjścia do negocjacji europarlamentu z krajami członkowskimi ws. ostatecznego kształtu budżetu. współsprawozdawcami projektu w pe są polscy europosłowie. jeden z nich, jan olbrycht (po), przekonywał we wtorek w pe, że propozycja komisji europejskiej jest nie do zaakceptowania, bo zaproponowana przez nią wartość budżetu nie pozwoli na realizowanie zadań w ue. przyznał jednak, że pe zgadza się w kilku elementach z propozycją ke, np. z tym, że przyszły budżet powinien uwzględniać nowe zadania, jak obronność, ochrona granic i system azylowy. pe - jak mówił - zgadza się również co do tego, że należy zwiększyć środki na badania naukowe, program erasmus oraz małe i średnie firmy. zobacz także:schetyna: z merkel rozmawialiśmy o budżecie ue, nord stream 2 i wspólnej polityce ue "ue: porozumienie w sprawie projektu unijnego budżetu na 2019 r. " zbigniew kuźmiuk (pis, ekr) przekonywał, że budżet unijny powinien odpowiadać ambicjom politycznym ue, a nie być ograniczany. jak mówił, jeśli godzimy się na nowe priorytety, jak np. kwestia bezpieczeństwa czy większe środki dla krajów południa europy, to nie może to się odbywać kosztem krajów europy środkowo-wschodniej. "propozycje cięć we wspólnej polityce rolnej i polityce spójności są nie do przyjęcia" - mówił, podkreślając, że cele realizowane przez obie polityki nie straciły na aktualności. przypomniał, że kraje "starej" unii ciągle uzyskują korzyści z dostępu do rynków krajów "nowej" unii i dlatego te polityki powinny być utrzymane. opowiedział się także za wyrównaniem stawek dopłat bezpośrednich w rolnictwie w całej ue. "jeśli tego nie będzie, można uznać, że rolnictwo krajów europy środkowej jest dyskryminowane" - zaznaczył. isabelle thomas (socjaliści) przekonywała, że ke w swojej propozycji przyszłego budżetu "rozmywa" ambicje. krytykowała ke, że chce przeznaczyć mniej środków na politykę rolną i spójności dla najbiedniejszych krajów. "ke chce dalej dokonywać cięć (w budżecie - pap). to jest niezgodne z interesem obywateli. (...). ten budżet to ostatnia szansa dla europy. obiecajmy sobie, że nie zgodzimy się na budżet, który podważy ambicje europy" - mówiła. zdaniem janusza lewandowskiego (po), zagrożeniem dla przyszłego budżetu wcale nie jest brexit, bo największy płatnik do budżetu, czyli niemcy i największy beneficjent, czyli polska deklarują chęć zwiększenia składki. apelował też o przejście do fazy uzgodnienia kształtu budżetu. ivo belet (epl) propozycję pe ws. budżetu nazwał realistyczną. "ci, którzy twierdzą, że budżet powinien zostać zmniejszony, mylą się. przed nami stoją poważne wyzwania. budżet nie powinien zostać zmniejszony" - mówił. derek vaughan (socjaliści) wezwał wszystkich europosłów, aby sprzeciwili się cięciom w polityce spójności. "wzywam, aby większość funduszy trafiła do regionów mniej rozwiniętych, aby wszystkie regiony otrzymywały fundusze strukturalne" - powiedział. zaapelował też o jak najszybsze porozumienie ws. wieloletnich ram finansowych. w podobnym duchu wypowiedział się jordi sole (zieloni). argumentował, że budżet musi być większy, by stawiać czoło wyzwaniom gospodarczym i klimatycznym. "pe proponuje silny budżet wieloletni, dostoswany do wyzwań, które przed nami stoją. propozycja pe jest lepsza niż propozycja ke" - zaznaczył. zobacz także:pe przeciwko zaproponowanym przez komisję cięciom w unijnym budżecie " 5 listopada komisja budżetowa parlamentu europejskiego zagłosowała za tym, by przyszły budżet unijny na lata 2021-27 był większy niż w projekcie komisji europejskiej. więcej pieniędzy ma być przeznaczonych m.in. na naukę, małe firmy i infrastrukturę. jednocześnie posłowie komisji sprzeciwili się cięciom w polityce spójności i rolnictwie. w środę ta propozycja będzie poddana pod głosowanie na sesji plenarnej. jeśli zostanie przyjęta, stanie się oficjalnym stanowiskiem europarlamentu do negocjacji z krajami członkowskimi (radą ue) nad ostatecznym kształtem przeszłego budżetu. posłowie oczekują, że porozumienie zostanie osiągnięte jeszcze przed wyborami do parlamentu europejskiego w 2019 r. długoterminowy budżet na lata 2021-2027 będzie pierwszym po brexicie, więc ue będzie już liczyć 27 państw członkowskich. z tego też powodu w brukseli i stolicach europejskich od miesięcy mówiło się o dziurze po brexicie i konieczności szukania oszczędności. plan przedstawiony 2 maja przez komisarza ds. budżetu guenthera oettingera przewiduje, że w niektórych obszarach wspólnota rzeczywiście wyda mniej, ale nie brak też takich, w których wydatki poszybują w górę. polityka spójności, której polska jest obecnie największym beneficjentem, a także wspólna polityka rolna dalej będą największymi częściami unijnej kasy. z oficjalnych wypowiedzi przedstawicieli komisji europejskiej wynika, że cięcie w polityce spójności ma wynieść 7 proc., natomiast w rolnictwie - około 5 proc. & 739 & medium & Medium & Power & NA & NA & 2018-11-13 & 2018 & 3 & POL
Frame & low-medium & National & 500-1000 & 1.0448136 & 1.8253861 & -0.2867631 & -1.2581601 & -1.2725643 & 0.0 & -1.1973642 & 0.9927066 & Recipient & European & European & European & European|POL & Neutral\\
Poland & http://wgospodarce.pl/informacje/40096-w-tygodniku-sieci-ten-szantaz-sie-nie-uda?utm\_source=feedburner\&utm\_medium=feed\&utm\_campaign=Feed\%3A+wGospodarce+\%28wGospodarce.pl\%29\&utm\_content=FeedBurner & 239 & wgospodarce.pl & Private/Non-Public & Online only & National & medium = CP is important part of story & Political leverage & Negative & National & No myth & NA & NA & NA & NA & NA & NA & NA & NA & Poland & w tygodniku "sieci": ten szantaż się nie uda & 2017-09-07 & polityka spójności & musimy stać w prawdzie. przymusowa relokacja to zły pomysł. dopóki ja jestem premierem, nie będzie na to polskiej zgody. nie złamią nas - o polityce imigracyjnej w rozmowie z jackiem i michałem karnowskimi na łamach tygodnika "sieci" mówi premier beata szydło. w nowym numerze "sieci" jacek i michał karnowscy rozmawiają z premier beatą szydło. oprócz tematu polityki imigracyjnej i unijnych nacisków na polskę, omawiają kwestię praworządności w polsce, programu rodzina 500+ i dekoncentracji mediów. premier wspomina również wydarzenia drugiej wojny światowej, której wybuchu rocznice obchodziliśmy w piątek. w wywiadzie premier beata szydło wyraźnie przeciwstawia się naciskom unijnym dotyczącym polityki imigracyjnej: - nie możemy być szantażowani tym, że obetnie nam się część środków unijnych za karę, bo nie godzimy się na przymusową relokację migrantów z afryki północnej i bliskiego wschodu - mówi premier. - fundusze unijne, polityka spójności jest takim samym filarem unii europejskiej jak swoboda przepływu towarów i usług. mamy do nich po prostu prawo. one nam się należą. dlatego domagamy się przestrzegania traktatów unijnych i nie akceptujemy dyktatu największych państw. mamy po swojej stronie silne argumenty. chcemy być w unii, cenimy ją i dlatego właśnie mamy prawo upominać się o przestrzeganie zasad, o prawdziwie wspólny rynek, o bezpieczeństwo, o rozwój - czytamy. premier szydło krytykuje również politykę poprzedniej premier: bardzo dużo złego wtedy się stało. premier kopacz złamała solidarność grupy wyszehradzkiej, zapewniała o gotowości przyjmowania nawet większej liczby. nie było na to zgody społecznej ani politycznej - wytyka premier. - my mówimy jasno: nie godzimy się na radykalne i długotrwałe zachwianie bezpieczeństwa polaków, tylko dlatego że niemieccy politycy wpadli na pomysł sprowadzenia do europy milionów imigrantów, zresztą bez żadnego planu, co im zaproponować, by mogli się tu odnaleźć. to o tyle szokujące, że zarówno wszystkie nasze dane, jak i moje bezpośrednie rozmowy z ludźmi, którzy pomagają w syrii, wskazują, że to, czego tamte społeczeństwa chcą najbardziej, to pomoc w odbudowaniu ich życia tam, na miejscu, w ich ojczyźnie. i to robimy i chcemy robić w jak największej skali - zapewnia premier szydło. polska premier wyraża również swoje zdanie na temat weta prezydenta i projektów reform sądownictwa: wyobrażam sobie, że była to dla niego trudna decyzja. i pewnie te oceny byłyby też inne, gdyby te weta dotyczyły innej sprawy. ale reforma sądownictwa, naszego sztandarowego elementu programu, ma charakter szczególny. to po prostu trzeba zrobić! przypomnę, że sam andrzej duda stworzył w kampanii "duda pomoc", bo spotykał się z morzem nieszczęść i niesprawiedliwości, jakie dotykają ludzi w sądach. mówił o tym dużo w kampanii. teraz prezydent bierze odpowiedzialność za reformę sądownictwa i wierzę, że przygotuje dobre ustawy. wierzę też, że nie pozwoli, by obóz dobrej zmiany się podzielił - zapewnia. - powinny zawierać to, co znalazło się w projektach pis, bo one były dobrze przemyślane. a zatem większa dostępność ludzi do wymiaru sprawiedliwości, odpowiedzialność dyscyplinarna sędziów, którzy uchybiają godności tego zawodu, większa demokratyzacja sądownictwa. sądy mają być dla ludzi, a nie dla sędziów - podkreśla stanowczo premier beata szydło. cały wywiad z premier beatą szydło w najnowszym numerze tygodnika "sieci", dostępnym w sprzedaży od 4 września, także w formie e-wydania na http://www.wsieciprawdy.pl/e-wydanie.html. zapraszamy też do subskrypcji tygodnika w sieci przyjaciół - www.siecprzyjaciol.pl i oglądania ciekawych audycji telewizji internetowej www.wpolsce.pl. & 527 & medium & Medium & Power & NA & NA & 2017-09-07 & 2017 & 2 & POL
Frame & low-medium & National & 500-1000 & 1.0448136 & 1.8253861 & -0.2867631 & -1.2581601 & -1.2725643 & 0.0 & -1.1973642 & 0.9927066 & Recipient & Domestic & Domestic & Domestic & Domestic|POL & Negative\\
Poland & http://www.dziennikpolski24.pl/artykul/3848561\%2Csamorzad-sie-sprawdzil-przyszlosc-europy-to-dobrze-rozwiniete-miasta-i-regiony-zdjecia\%2Cid\%2Ct.html?cookie=1 & a5 & dziennikpolski24.pl & Private/Non-Public & Online and Offline & Regional/Local & medium = CP is important part of story & Territorial cooperation & Positive & EU + Subnational & No myth & NA & NA & NA & NA & NA & NA & NA & NA & Poland & samorząd się sprawdził. przyszłość europy to dobrze rozwinięte miasta i regiony [zdjęcia] & 2015-05-05 & polityka regionalna & markku markkula, przewodniczący europejskiego komitetu regionów - co jest ważniejsze - region czy państwo? - oczywiście jedno i drugie jest równie ważne, samorząd regionalny czy gminny ma jednak zdecydowanie lepsze rozeznanie potrzeb lokalnej wspólnoty i lepiej wie, jak rozwiązywać jej problemy. to właśnie w regionach powstaje ponad 55 procent wszystkich inwestycji, a będzie ich jeszcze więcej, zwłaszcza dzięki nowej inicjatywie przewodniczącego komisji europejskiej jeana-claude'a junckera. powstanie bowiem europejski fundusz inwestycji strategicznych, zaś państwa członkowskie ue zobowiązano do tworzenia warunków sprzyjających powstaniu w całej unii inwestycji o łącznej wartości 315 mld euro. to jeden z istotnych instrumentów rozwoju. - trudno jednak o spójną politykę regionalną, skoro zarówno państwa, jak i regiony znajdują się na różnym poziomie rozwoju. - to prawda, ale wszyscy powinniśmy wykorzystać szanse i możliwości tkwiące w różnych źródłach finansowania. to nie są tylko unijne programy operacyjne - polska jest bardzo dobrym przykładem efektywnego wykorzystania funduszy europejskich, to są również środki publiczne - samorządowe i państwowe a także prywatne. znacznie łatwiej je pozyskać, jeśli dany region wykształci własną specjalizację i będzie szerzej współpracował z biznesem i nauką, często bowiem kapitał intelektualny jest cenniejszy od finansowego. - jakie znaczenie ma polityka regionalna i współpraca między przedstawicielami samorządu właśnie na tym szczeblu? olbrzymie, mocne regiony to nasza przyszłość. musimy działać, bo samo planowanie już niewiele zmieni. nie bójmy się eksperymentować, poszukiwać pionierskich rozwiązań. jeśli zaś chodzi o konkrety, to rozmawiałem z marszałkiem małopolski markiem sową o możliwości wsparcia strategicznego projektu "kraków - nowa huta przyszłości". rewitalizacja tej części miasta i wykorzystanie nowych technologii, w tym też chmury edukacyjnej, do dalszego rozwoju, to korzystne połączenie świata przemysłu i nauki dla dobra mieszkańców. możemy tu wykorzystać rozwiązania i doświadczenie ludzi realizujących podobne przedsięwzięcia na przykład w helsinkach i poczdamie, gdzie powstały m.in. parki technologiczne i specjalne strefy dla start-upów. małopolska może być zatem pionierskim regionem w tym zakresie. & 300 & medium & Medium & Socio-Economic & NA & NA & 2015-05-05 & 2015 & 1 & ECO
Frame & low-medium & Regional & <500 & 1.0448136 & 1.8253861 & -0.2867631 & -1.2581601 & -1.2725643 & 0.0 & -1.1973642 & 0.9927066 & Recipient & Domestic & European & Mixed & Domestic|ECO & Positive\\
\addlinespace
Poland & http://wiadomosci.onet.pl/kraj/nik-polska-najczesciej-kontrolowanym-krajem-przez-europejski-trybunal-obrachunkowy/z5wwc & 784 & Onet Wiadomości & Private/Non-Public & Online only & National & medium = CP is important part of story & Mismanagement & Balanced & National & No myth & NA & NA & NA & NA & NA & NA & NA & NA & Poland & nik nik: poland the most controlled country by the european court of auditors 11:09 & 2014-12-05 & polityka regionalna & polska jako największy beneficjent unijnych funduszy jest jednocześnie najczęściej kontrolowana przez audytorów z eto - podkreśla nik. na tle innych państw ue wypadamy dobrze; według nik, "to zasługa m.in. krajowego systemu nadzoru i kontroli". w informacji przesłanej w piątek pap najwyższa izba kontroli podkreśliła, że polska jest największym beneficjentem środków unijnych; w latach 2007-2013 polska "wchłonęła" blisko 80 mld euro z budżetu ue. "nawet odliczając składkę członkowską uzyskaliśmy najwięcej pieniędzy spośród krajów ue. ponadto znaleźliśmy się w czołówce państw, które otrzymały najwyższe kwoty w ramach wspólnej polityki rolnej i polityki spójności" - wskazuje nik. izba zwraca uwagę, że jednocześnie polska była krajem najczęściej odwiedzanym przez unijnych kontrolerów. "w latach 2009-2013 zbadali oni 383 zawarte w naszym kraju transakcje w ramach funduszy pomocowych: rolnictwo i spójność. różnego rodzaju błędy stwierdzili w 128 z nich. to 33,4 proc. przypadków zbadanych przez eto" - wylicza izba. jednocześnie nik przypomniała, że średnia europejska wynosi 45,2 procent, a mniej błędów od polaków w transakcjach mają procentowo tylko cztery kraje: łotwa, słowenia, belgia i estonia. w ocenie nik, "to zasługa m.in. sprawnie funkcjonującego w polsce systemu nadzoru i kontroli". "nieprzypadkowo eto wysoko oceniło działanie w naszym kraju systemu kontrolowania transakcji zawieranych w ramach polityki spójności. pozytywna ocena eto w tej dziedzinie wyróżnia nas w unii europejskiej" - zwraca uwagę izba. jak informuje nik, eto obliczył, że w 2013 r. 4,7 proc. budżetu unijnego wydatkowano w mniejszym lub większym stopniu niewłaściwie, a złożyły się na to nieprawidłowości dostrzeżone we wszystkich państwach członkowskich w ramach wszystkich polityk pomocowych. najgorzej pod tym względem było w 2007 roku, a najlepiej - w 2009. nik wyjaśnił, że błędy stwierdzane przez eto mają różną wagę, a tych najcięższych jest niewiele. w 2013 r. trybunał skierował do olaf-u (europejski urząd ds. zwalczania nadużyć finansowych) 14 spraw i były to przypadki, w których audytorzy podejrzewali oszustwo polegające na wyłudzeniu lub sprzeniewierzeniu unijnych pieniędzy. zdecydowana większość stwierdzanych nieprawidłowości - jak wyjaśnia nik - ma charakter formalny lub merytoryczny. audytorzy wskazują na przekroczenie jakiegoś przepisu, niespełnienie wymagania lub błędną interpretację zasad projektu. z kolei najwięcej błędów pojawia się tam, gdzie zarządzanie funduszami jest dzielone między komisję europejską i państwa członkowskie. w ten sposób zarządzane są dwie polityki z największą liczbą błędnych transakcji: polityka regionalna oraz rozwój obszarów wiejskich. to wynik nakładających się na siebie systemów zarządzania, mnogości przepisów, a przede wszystkim zasięgu obu polityk. jak wskazuje nik, wśród stwierdzonych w 2013 r. w całej ue błędów najwięcej (39 proc.), dotyczyło pojedynczych elementów projektu, które nie powinny być w jego ramach finansowane. 22 proc. nieprawidłowości polegało na przekazaniu pieniędzy podmiotom, które nie spełniały wszystkich wymagań, 21 proc. - miało związek z wadliwymi przetargami, a 14 proc. - polegało na nieprawidłowym zadeklarowaniu powierzchni rolnej. izba zwróciła uwagę, że polska, mimo wyróżniających się wyników, w dalszym ciągu będzie intensywnie kontrolowana przez eto, co wynika z podziału unijnych funduszy. "w nowej perspektywie budżetowej nasz kraj znów będzie w czołówce beneficjentów" - przypomniała izba. "dlatego trybunałowi szczególnie zależy na współpracy z polską najwyższą izbą kontroli. ocenia ją zresztą bardzo wysoko" - wskazuje nik. jak poinformowała izba, europejski trybunał obrachunkowy (eto) zaproponowało nik pełną współpracę przy audycie unijnego programu zwalczania chorób zwierzęcych jako pierwszemu spośród wszystkich europejskich najwyższych organów kontroli. "ta wyjątkowa propozycja jest wyrazem zaufania trybunału do naszej instytucji i naszego kraju" - ocenił prezes nik krzysztof kwiatkowski. (rz) & 544 & medium & Medium & Governance & NA & NA & 2014-12-05 & 2014 & 1 & POL
Frame & low-medium & National & 500-1000 & 1.0448136 & 1.8253861 & -0.2867631 & -1.2581601 & -1.2725643 & 0.0 & -1.1973642 & 0.9927066 & Recipient & Domestic & Domestic & Domestic & Domestic|POL & Neutral\\
Poland & http://biznes.interia.pl/news/mniej-pieniedzy-dla-polski-z-unijnego-budzetu-po-2020-r\%2C2525382?utm\_source=Biznes\&utm\_medium=RSS\&utm\_campaign=RSS & 489 & biznes.interia.pl & Private/Non-Public & Online only & National & medium = CP is important part of story & Institutional bargaining over funding & Factual & EU & No myth & NA & NA & NA & NA & NA & NA & NA & NA & Poland & mniej pieniędzy dla polski z unijnego budżetu po 2020 r. & 2017-06-26 & fundusze strukturalne & w dokumencie bruksela stwierdza, że w związku z brexitem i nowymi wyzwaniami będzie mniej pieniędzy w unijnym budżecie. daje do zrozumienia, że nie da się utrzymać na obecnym poziomie wydatków na fundusze strukturalne (z których najbardziej korzysta polska) i wspólną politykę rolną (także w dużym stopniu korzystają z tego nasi rolnicy). należy zadać pytanie, czy obecny poziom wydatków w dwóch głównych programach budżetu ue, tj. we wspólnej polityce rolnej (39 proc. wydatków w 2015 r.) i funduszach strukturalnych (36 proc.) powinny zostać utrzymane - napisano w dokumencie. komisja europejska chce wprowadzić także nowe priorytety budżetowe na kwestie związane z migracją czy obronnością, co oznacza, że jeżeli polska nie będzie uczestniczyć w tych politykach, otrzyma mniej pieniędzy. a polska na razie odrzuca unijne prawo w sprawie relokacji uchodźców i nie podjęła decyzji w sprawie przystąpienia do tworzącej się wspólnej, europejskiej obronności (pesco). bruksela zamierza także wprowadzić większą "elastyczności", czyli możliwość szybkiego przesuwania (odbierania?) funduszy na te nowe cele, jeżeli kraj ich nie wydaje. "elastyczność", w żargonie unijnym, może być różnie rozumiana. dla polski to przede wszystkim zagrożenie przesuwaniem środków, na przykład, miedzy funduszami strukturalnymi a innymi programami finansowanymi przez ue - na przykład funduszem obronnym czy programami edukacyjnymi. - poza wyzwaniem, jakim jest brexit, obecna struktura wieloletnich ram finansowych nie odzwierciedla nowych priorytetów, które pojawiły się w ostatnich latach takich, jak wewnętrzne i zewnętrzne bezpieczeństwo, obrona i migracja - czytamy w dokumencie - dlatego rekonfiguracja jest nieunikniona. komisja europejska proponuje także inną zmianę. zamiast 7-letniej perspektywy budżetowej, chce wprowadzić 5-letnią. - to będzie rewolucja budżetowa - mówi jeden z urzędników - dodatkowo ułatwi to zarówno przesuwanie, jak i odbieranie środków. w krótszym czasie można mniej zaplanować. "5-latce" trudniej będzie także przygotować i rozliczyć duże projekty na kilka miliardów euro (chodzi między innymi o projekty dotyczące dróg). w tej sytuacji łatwiej będzie zarzucić polsce na przykład nieudolność w wydatkowaniu, a potem przesunąć środki na przykład na migrację. to pierwszy dokument ke w sprawie przyszłego budżetu. został on zaprezentowany unijnym komisarzom w zeszłym tygodniu. w dokumencie nie ma żadnych odniesień, co do możliwości uszczuplania pieniędzy dla krajów, które nie przestrzegają praworządności, czy nie przyjmują uchodźców - taką groźbę formułował m.in. szef ke jean claude juncker). jednak - jak zauważa jeden z urzędników ke w rozmowie z dziennikarką rmf fm - to dopiero początek dyskusji i "już widać, że jest myślenie, jak wprowadzić mechanizm szybszego przesuwania czy odbierania środków, który można stosować także wobec krajów, które nie wywiązują się ze swoich zobowiązań". & 401 & medium & Medium & Power & NA & NA & 2017-06-26 & 2017 & 2 & POL
Frame & low-medium & National & <500 & 1.0448136 & 1.8253861 & -0.2867631 & -1.2581601 & -1.2725643 & 0.0 & -1.1973642 & 0.9927066 & Recipient & European & European & European & European|POL & Neutral\\
Poland & https://www.24kurier.pl/aktualnosci/wiadomosci/sld-o-napietnowaniu-polski-bublu-ke-i-teatrze-polskim/ & a16 & 24kurier.pl & Private/Non-Public & Online and Offline & Regional/Local & low = CP mentioned more times but NOT important part of story (mainly about others issues) & Political leverage & Negative & EU + National + Subnational & No myth & NA & NA & NA & NA & NA & NA & NA & NA & Poland & sld o napiętnowaniu polski, bublu ke i teatrze polskim & 2018-05-26 & fundusz spójności & dla własnego dobra polski rząd powinien wycofać się ze zmian w wymiarze sprawiedliwości w takim zakresie, jak tego oczekuje komisja europejska - uważają politycy zachodniopomorskiego sojuszu lewicy demokratycznej. jednocześnie krytykują ke za "rozminięcie się z prawdą" oraz zarząd naszego województwa za działania w sprawie rozbudowy teatru polskiego. - polski rząd wykonał kilka kroków wstecz, ale komisja europejska ocenia, że są to kroki niewystarczające. wedle ke skoro polska otrzymuje 106 miliardów euro, w przyszłej perspektywie ma otrzymać też znaczącą kwotę, to w przypadku zaistnienia sporów, powinny je rozstrzygać niezależne sądy - stwierdził europoseł bogusław liberadzki. - w naszym interesie leży, aby polska nie była napiętnowana, tak jak dotychczas. sprawy nie należy przeciągać. spór o praworządność może nam zaszkodzić. wypłata środków może być związana z oceną poziomu praworządności. komisja europejska dwa tygodnie temu zajęła jednomyślne i jednogłośne stanowisko - praworządność powinna być uwzględnianana w wypłacie środków z budżetu europejskiego po roku 2020. jednocześnie polityk zwrócił uwagę, że analiza wieloletnich ram finansowych, które przygotowała ke, dowodzi, że ta instytucja "rozminęła się z prawdą". - tak naprawdę redukcja środków na politykę rolną to nie zapowiedziane 5\% a co najmniej 10\% - powiedział. - zmniejszenie funduszu spójności to nie byłoby 7\% a co najmniej 15\%. fundusz spójności mógłby być mniejszy o niemal 50\%! mamy więc w trzech obszarach potencjalnie pogarszającą się sytuację. fundusz spójność to są te środki, które idą na drogi s3, s6, tunel w świnoujściu, nadodrzańską linię kolejową. komisja europejska chciała włożyć nam bubla. to, co przedłożyła nam ke, jest niekorzystne. zdaniem liberadzkiego powinniśmy zorganizować w polsce rzetelną debatę na temat naszego ewentualnego wejścia do strefy euro. - unia europejska po brexicie pójdzie w takim kierunku, że kto chce się w danej dziedzinie mocniej integrować, to zapraszamy, a kto nie chce, ten zostaje na poboczu - ocenił. - nie ma lepszego rozwiązania niż starać się wzmacniać więzy europejskie. euro jest swego rodzaju emanacją prawdziwych relacji. nawet bułgaria zadelarowała, że chce wstąpić do strefy euro. dlatego potrzebna jest debata o euro, o obawach, o ich słuszności. z kolei dariusz wieczorek, szef regionalnych struktur sld, mówił o rozbudowie teatru polskiego. zapewniał, że sojusz popiera tę inwestycję, ale nie ma zaufania do zarządu województwa. zwrócił uwagę, że na razie na rozbudowę nie ma żadnych pieniędzy unijnych ani ministerialnych. - w piśmie z 2013 roku koszt budowy teatru polskiego to były 54 miliony złotych, natomiast w 2017 roku ta kwota to już było 86 milionów - kontynuował. - to pokazuje pewną niekonsekwencję. wiemy, że ceny rosną, ale nie o 100 procent. budżet województwa to ok. 850 milionów złotych. więc jeżeli mówimy o 86 milionach, to jest 10 \% budżetu. jeśli mielibyśmy to robić tylko ze środków budżetowych, to nie ma szansy, żeby było na to stać województwo. na zdjęciu: - w naszym interesie leży, aby polska nie była napiętnowana, tak jak dotychczas - mówił europoseł bogusław liberadzki. & 453 & low & Low & Power & NA & NA & 2018-05-26 & 2018 & 3 & POL
Frame & low-medium & Regional & <500 & 1.0448136 & 1.8253861 & -0.2867631 & -1.2581601 & -1.2725643 & 0.0 & -1.1973642 & 0.9927066 & Recipient & Domestic & European & Mixed & Domestic|POL & Negative\\
Poland & http://wgospodarce.pl/informacje/56264-budzet-unijny-koscia-niezgody-w-pe?utm\_source=feedburner\&utm\_medium=feed\&utm\_campaign=Feed\%3A+wGospodarce+\%28wGospodarce.pl\%29\&utm\_content=FeedBurner & 412 & wgospodarce.pl & Private/Non-Public & Online only & National & medium = CP is important part of story & Institutional bargaining over funding & Factual & EU & No myth & NA & NA & NA & NA & NA & NA & NA & NA & Poland & budżet unijny kością niezgody w pe & 2018-11-17 & fundusze strukturalne & podczas wtorkowej debaty w pe wielu europosłów różnych ugrupowań krytykowało proponowany przez ke projekt budżetu na lata 2021-2027, sprzeciwiając się m.in cięciom w polityce spójności i rolnictwie. apelowali o przyjęcie propozycji pe. debata poprzedziła głosowanie nad propozycją pe ws. budżetu na lata 2021-2027, które zaplanowano na środę. jeśli propozycja ta zostanie przegłosowana, stanie się oficjalnym stanowiskiem pe ws. przyszłych ram finansowych i punktem wyjścia do negocjacji europarlamentu z krajami członkowskimi ws. ostatecznego kształtu budżetu. współsprawozdawcami projektu w pe są polscy europosłowie. jeden z nich, jan olbrycht (po), przekonywał we wtorek w pe, że propozycja komisji europejskiej jest nie do zaakceptowania, bo zaproponowana przez nią wartość budżetu nie pozwoli na realizowanie zadań w ue. przyznał jednak, że pe zgadza się w kilku elementach z propozycją ke, np. z tym, że przyszły budżet powinien uwzględniać nowe zadania, jak obronność, ochrona granic i system azylowy. pe - jak mówił - zgadza się również co do tego, że należy zwiększyć środki na badania naukowe, program erasmus oraz małe i średnie firmy. zbigniew kuźmiuk (pis, ekr) przekonywał, że budżet unijny powinien odpowiadać ambicjom politycznym ue, a nie być ograniczany. jak mówił, jeśli godzimy się na nowe priorytety, jak np. kwestia bezpieczeństwa czy większe środki dla krajów południa europy, to nie może to się odbywać kosztem krajów europy środkowo-wschodniej. propozycje cięć we wspólnej polityce rolnej i polityce spójności są nie do przyjęcia" - mówił, podkreślając, że cele realizowane przez obie polityki nie straciły na aktualności. przypomniał, że kraje "starej" unii ciągle uzyskują korzyści z dostępu do rynków krajów "nowej" unii i dlatego te polityki powinny być utrzymane. opowiedział się także za wyrównaniem stawek dopłat bezpośrednich w rolnictwie w całej ue. jeśli tego nie będzie, można uznać, że rolnictwo krajów europy środkowej jest dyskryminowane - zaznaczył. isabelle thomas (socjaliści) przekonywała, że ke w swojej propozycji przyszłego budżetu "rozmywa" ambicje. krytykowała ke, że chce przeznaczyć mniej środków na politykę rolną i spójności dla najbiedniejszych krajów. ke chce dalej dokonywać cięć (w budżecie ). to jest niezgodne z interesem obywateli. ten budżet to ostatnia szansa dla europy. obiecajmy sobie, że nie zgodzimy się na budżet, który podważy ambicje europy - mówiła. zdaniem janusza lewandowskiego (po), zagrożeniem dla przyszłego budżetu wcale nie jest brexit, bo największy płatnik do budżetu, czyli niemcy i największy beneficjent, czyli polska deklarują chęć zwiększenia składki. apelował też o przejście do fazy uzgodnienia kształtu budżetu. ivo belet (epl) propozycję pe ws. budżetu nazwał realistyczną. ci, którzy twierdzą, że budżet powinien zostać zmniejszony, mylą się. przed nami stoją poważne wyzwania. budżet nie powinien zostać zmniejszony - mówił. derek vaughan (socjaliści) wezwał wszystkich europosłów, aby sprzeciwili się cięciom w polityce spójności. wzywam, aby większość funduszy trafiła do regionów mniej rozwiniętych, aby wszystkie regiony otrzymywały fundusze strukturalne - powiedział. zaapelował też o jak najszybsze porozumienie ws. wieloletnich ram finansowych. w podobnym duchu wypowiedział się jordi sole (zieloni). argumentował, że budżet musi być większy, by stawiać czoło wyzwaniom gospodarczym i klimatycznym. & 477 & medium & Medium & Power & NA & NA & 2018-11-17 & 2018 & 3 & POL
Frame & low-medium & National & <500 & 1.0448136 & 1.8253861 & -0.2867631 & -1.2581601 & -1.2725643 & 0.0 & -1.1973642 & 0.9927066 & Recipient & European & European & European & European|POL & Neutral\\
Poland & http://www.pap.pl/aktualnosci/news,1430124,premier-polska-gospodarka-coraz-mniej-bedzie-zalezala-od-srodkow-z-ue.html & 625 & pap.pl & Public & Online only & National & medium = CP is important part of story & Institutional bargaining over funding & Balanced & National & No myth & NA & NA & NA & NA & NA & NA & NA & NA & Poland & premier: polska gospodarka coraz mniej będzie zależała od środków z ue & 2018-05-27 & fundusze strukturalne & premier odniósł się do pytania o ewentualnie mniejsze pieniądze dla polski z przyszłego budżetu ue. "mamy błędne wyobrażenie, że nasza gospodarka zależy od środków unijnych" - powiedział. ocenił, że polska gospodarka "w coraz mniejszym stopniu będzie zależała od środków unijnych". dodał, że na skutek ogromnych zapóźnień iii rp i prl brakuje nam infrastruktury. "i dlatego my będziemy bardzo twardo walczyć o fundusze strukturalne w unii europejskiej i o fundusze na wspólną politykę rolną" - zapowiedział. przypomniał, że w piątek spotkał się z premier rumunii vioricą dancilą. "podkreśliłem, że polska organizuje i koordynuje wszelkie działania związane z polityką spójności, czyli polityką związaną z funduszami unijnymi. na pewno twardo będziemy stawiać nasze interesy, bronić naszych interesów i wygramy tutaj" - podkreślił. na początku maja komisja europejska przedstawiła projekt wieloletniego budżetu ue na lata 2021-2027. całkowita suma zobowiązań ma wynieść w tym okresie 1,279 biliona euro. z informacji podanych przez ke wynika, że w ramach finansowych na lata 2021-2027 przewidziano wynoszące około 7 proc. cięcia w polityce spójności i około 5 proc. - cięcia we wspólnej polityce rolnej (wpr). ponadto polityka spójności ma odgrywać ważniejszą rolę we wspieraniu reform strukturalnych i długoterminowej integracji migrantów. z obliczeń resortu inwestycji i rozwoju wynika, że w przypadku polityki spójności polska może dostać nie o 7 proc., ale o 10 proc. mniej. natomiast jeśli chodzi o wpr to środki dla polski mogą zostać zmniejszone o 15 proc., a nie, o 5 proc.(pap) & 235 & medium & Medium & Power & NA & NA & 2018-05-27 & 2018 & 3 & POL
Frame & low-medium & National & <500 & 1.0448136 & 1.8253861 & -0.2867631 & -1.2581601 & -1.2725643 & 0.0 & -1.1973642 & 0.9927066 & Recipient & Domestic & Domestic & Domestic & Domestic|POL & Neutral\\
\addlinespace
Poland & http://supernowosci24.pl/czy-zwierzeta-beda-cierpiec-w-imie-nauki/ & a20 & SuperNowości24 & Private/Non-Public & Online and Offline & Regional/Local & very low = CP mentioned once & Research \& innovation & Factual & Subnational & No myth & NA & NA & NA & NA & NA & NA & NA & NA & Poland & czy zwierzęta będą cierpieć w imię nauki? & 2019-01-10 & europejski fundusz rozwoju regionalnego & rzeszów. uniwersytet rzeszowski zrealizuje kontrowersyjny projekt. jego głównym celem jest budowa obiektu centralnej podkarpackiej zwierzętarni doświadczalnej. w centrum badań mają być prowadzone eksperymenty na dużych zwierzętach, głównie na świniach, ponieważ są one odpowiednie do badań biotechnologicznych, ich wyniki zaś użyteczne dla medycyny. na zwierzętach mają "trenować" lekarze różnych specjalności, m.in. transplantolodzy, kardiochirurdzy czy anestezjolodzy. - naukowcom trzeba patrzeć na ręce! - mówi halina derwisz z rzeszowskiego stowarzyszenia ochrony zwierząt - ludzie są zdolni do strasznych rzeczy! 21 grudnia 2018 r. uniwersytet rzeszowski podpisał umowę o dofinansowanie projektu "interdyscyplinarne centrum badań przedklinicznych i klinicznych". projekt będzie współfinansowany przez europejski fundusz rozwoju regionalnego. uniwersytet otrzyma ponad 12 mln zł, a sam wyłoży około 5 mln zł. głównym celem projektu jest budowa obiektu centralnej podkarpackiej zwierzętarni doświadczalnej i zakup specjalistycznej aparatury badawczej. uniwersytet rzeszowski będzie mógł prowadzić badania biomedyczne - od badań molekularnych, biofizycznych po testy in vitro oraz in vivo. na stronie internetowej uczelni możemy przeczytać, że "nowoczesna infrastruktura badawcza pozwoli nawiązać i rozwijać współpracę z przedsiębiorstwami z branży biomedycznej, biotechnologicznej i farmaceutycznej". - centrum badan przedklinicznych i klinicznych uniwersytetu rzeszowskiego będzie jednostką naukową uniwersytetu rzeszowskiego powołaną do prowadzenia badań przedklinicznych i klinicznych przez pracowników ur - mówi dr maciej ulita, rzecznik prasowy. & 196 & very low & Low & Socio-Economic & NA & NA & 2019-01-10 & 2019 & 3 & ECO
Frame & v.low & Regional & <500 & 1.0448136 & 1.8253861 & -0.2867631 & -1.2581601 & -1.2725643 & 0.0 & -1.1973642 & 0.9927066 & Recipient & Domestic & Domestic & Domestic & Domestic|ECO & Neutral\\
Poland & https://gloswielkopolski.pl/fundusze-europejskie-opolskie-potrafi-najlepiej-wydawac-pieniadze-na-rozwoj/ar/13443569 & a37 & https://gloswielkopolski.pl/ & Private/Non-Public & Online and Offline & Regional/Local & very high = CP is most important issue + CP is mentioned in title/headline & Economic development & Positive & Subnational & No myth & Infrastructure & Positive & Subnational & No myth & NA & NA & NA & NA & Poland & fundusze europejskie. opolskie potrafi najlepiej wydawać pieniądze na rozwój & 2018-08-27 & fundusze europejskie & -opolskie, jak mało który region w polsce, docenia możliwości jakie dają fundusze europejskie przeznaczone na politykę spójności. środki te wydawane są na poprawę życia mieszkańców oraz budowanie konkurencyjnej miejscowej gospodarki w tempie, którego inne województwa mogą pozazdrościć.
-nasi mieszkańcy doceniają poprawę dostępu do infrastruktury, większych możliwości wykorzystania potencjału opolszczyzny, a co za tym idzie staje się ona konkurencyjna wśród regionów. ważne są inwestycje drogowe, poprawa komunikacji i przedsięwzięcia w poprawę jakości powietrza i środowiska. 
- & 74 & very high & High & Socio-Economic & Socio-Economic & NA & 2018-08-27 & 2018 & 3 & ECO
Frame & high-very high & Regional & <500 & 1.0448136 & 1.8253861 & -0.2867631 & -1.2581601 & -1.2725643 & 0.0 & -1.1973642 & 0.9927066 & Recipient & Domestic & Domestic & Domestic & Domestic|ECO & Positive\\
Poland & https://pomorska.pl/spacerkiem-po-dawnej-nasycalni-podkladow-w-solcu-kujawskim-zdjecia/ga/12077894/zd/23815310 & a50 & https://pomorska.pl/ & Private/Non-Public & Online and Offline & Regional/Local & low = CP mentioned more times but NOT important part of story (mainly about others issues) & Environment/green/low-carbon & Positive & Subnational & No myth & NA & NA & NA & NA & NA & NA & NA & NA & Poland & spacerkiem po dawnej nasycalni podkładów w solcu kujawskim & 2017-05-15 & fundusz rozwoju regionalnego & -gmina solec kujawski jako pierwsza w polsce przeprowadziła nowatorską metodą rekultywację terenów po nasycalni podkładów kolejowych. metodę biologicznego usuwania z gleby chemikaliów, bez konieczności wywożenia zanieczyszczonej gleby, opracował dr wojciech irmiński. rekultywację terenów po byłej nasycalni zakończono jesienią ub.r. dzięki niej w solcu udało się przywrócić do życia 16 ha terenów. w minionym tygodniu unikalną metodę rekultywacji zniszczonych chemicznie terenów poznali zagraniczni goście - partnerzy projektu realizowanego przez europejski fundusz rozwoju regionalnego (efrr). przyjechali m.in. z włoskiej wenecji, niemiec, słowenii i chorwacji. po terenie dawnej nasycalni oprowadzał gości dr wojciech irwiński. & 92 & low & Low & Socio-Economic & NA & NA & 2017-05-15 & 2017 & 2 & ECO
Frame & low-medium & Regional & <500 & 1.0448136 & 1.8253861 & -0.2867631 & -1.2581601 & -1.2725643 & 0.0 & -1.1973642 & 0.9927066 & Recipient & Domestic & Domestic & Domestic & Domestic|ECO & Positive\\
Poland & http://wgospodarce.pl/informacje/60223-w-ue-klincz-w-sprawie-nowego-budzetu?utm\_source=feedburner\&utm\_medium=feed\&utm\_campaign=Feed\%3A+wGospodarce+\%28wGospodarce.pl\%29\&utm\_content=FeedBurner & 810 & wgospodarce.pl & Private/Non-Public & Online only & National & medium = CP is important part of story & Institutional bargaining over funding & Factual & EU & NA & NA & NA & NA & NA & NA & NA & NA & NA & Poland & w ue klincz w sprawie nowego budżetu & 2019-02-22 & polityka spójności & choć od przedstawienia przez ke propozycji wieloletniego budżetu upłynęło 9 miesięcy, w ue nie ma zbliżenia stanowisk w tej sprawie. wtorkowa debata ministrów ds. europejskich potwierdziła, że część stolic chce powiązania dostępu do funduszy z praworządnością. dyskusja podczas posiedzenia rady ds. ogólnych na temat projektu wieloletnich ram finansowych na lata 2021-2027 kolejny raz uwypukliła podziały wśród państw członkowskich, dotyczące zwłaszcza poziomu środków, jakie ue powinna mieć do dyspozycji w kolejnej siedmiolatce. choć zdecydowana większość ministrów w ogóle nie poruszała tematu powiązania dostępu do środków unijnych z praworządnością, ci, którzy to robili, opowiadali się za wprowadzeniem tego rozwiązania. kwestia rządów prawa to temat dla nas szczególnie istotny. chodzi o ochronę środków budżetowych, a także o stworzenie podstawy do zaufania, które w ostatnich latach straciliśmy - mówił niemiecki minister ds. europejskich michael roth. wtórował mu szef greckiej dyplomacji jeorjos katrungalos wskazując, że "praworządność jest kluczową sprawą". to nie jest dylemat dotyczący wyłącznie kontynuacji europejskiego modelu społecznego, państwa opiekuńczego, ale chodzi także o otwarty charakter naszych społeczeństw, trzymanie się wartości i swobód europejskich. jesteśmy zwolennikami warunkowości, powiązania (dostępu do funduszy ) z praworządnością - oświadczył katrungalos. jego zdaniem wszystkie państwa ue powinny przechodzić przegląd w obszarze przestrzegania nie tylko zasad rządów prawa, ale też zachowywania odpowiednich praw socjalnych. podobne podejście zaprezentował wicepremier i minister spraw zagranicznych belgii didier reynders, który zaproponował, by propozycja komisji europejskiej została połączona z pomysłem zorganizowania wzajemnego przeglądu rządów prawa prowadzonego we wszystkich krajach ue. belgia już przed trzema laty przedstawiła ten pomysł, jednak do tej pory nie zyskał on odpowiedniego poparcia. do postępu w pracach w tej sprawie wzywał stały przedstawiciel luksemburga przy ue georges friden. jak zauważył, eksperci państw członkowskich zajęli się już propozycją ke dotyczącą powiązania dostępu do funduszy z praworządnością, ale nie ma na razie wszystkich dokumentów, żeby pójść do przodu. zachęcamy prezydencję do postępu w ramach tego dossier - zwracał się do reprezentanta rządu rumuńskiego. jedyną osobą, która negatywnie odniosła się do tej propozycji, był minister ds. europejskich węgier szabolcs takacs. apelował o wyjaśnienie wszystkich wątpliwości, m.in. obaw służb prawnych rady, dotyczących tego rozwiązania. sama propozycja powinna być zgodna z zasadą praworządności. nie może obchodzić traktatu. nie możemy przyjąć obecnej wersji wniosku - oświadczył węgier. polski ambasador przy ue andrzej sadoś (zastępował nieobecnego ministra ds. europejskich konrada szymańskiego), tak jak przedstawiciele wielu innych krajów, nie poruszał tego tematu w swoim wystąpieniu. więcej miejsca w dyskusji poświęcono kwestii, czy budżet i poszczególne jego działy, np. wspólna polityka rolna lub polityka spójności, powinny być mniejsze czy większe. tu nie ma jednak przełomu, bo państwa północy europy, z holandią na czele, opowiadają się wciąż za cięciami, natomiast kraje uzyskujące ze wspólnej kasy więcej, niż do niej wpłacają, opowiadają się za zwiększeniem budżetu. powinniśmy znaleźć oszczędności, żeby ograniczyć ogólne wydatki ue - przekonywał minister spraw zagranicznych holandii stef blok. cięcia dotyczące wspólnej polityki rolnej i polityki spójności nie są pomocne i nie będą dobrze odebrane przez naszych obywateli - odpowiadała na takie wezwania minister ds. europejskich irlandii helen mcentee. przysłuchujący się tym powtarzającym się od miesięcy głosom unijny komisarz ds. budżetowych guenther oettinger nie krył frustracji. jak zauważył, z jednej strony cześć państw członkowskich mówi, że nie może być żadnych cięć we wspólnej polityce rolnej i w polityce spójności, z drugiej inne państwa wskazują, że trzeba finansować nowe zadania, wzmocnić program erasmus+, wydawać więcej na badania naukowe itd. proszę państwa, to się nie sumuje. nie trzeba być genialnym matematykiem, żeby to dostrzec. jeśli chcemy sfinansować nowe zadania , musimy podejść do wszystkiego bardziej oszczędnie" - powiedział. przywódcy państw i rządów krajów ue chcą, żeby porozumienie dotyczące wieloletnich ram finansowych zostało wypracowane jesienią br. rumuńska prezydencja chce wrócić do tego tematu w marcu. & 597 & medium & Medium & Power & NA & NA & 2019-02-22 & 2019 & 3 & POL
Frame & low-medium & National & 500-1000 & 1.0448136 & 1.8253861 & -0.2867631 & -1.2581601 & -1.2725643 & 0.0 & -1.1973642 & 0.9927066 & Recipient & European & European & European & European|POL & Neutral\\
Poland & https://fakty.interia.pl/swiat/news-unijna-komisarz-w-ue-pracuje-239-mln-ludzi-to-historyczny-re,nId,2698250 & 247 & fakty.interia.pl & Private/Non-Public & Online only & National & medium = CP is important part of story & Jobs & Positive & EU & No myth & NA & NA & NA & NA & NA & NA & NA & NA & Poland & unijna komisarz: w ue pracuje 239 mln ludzi - to historyczny rekord & 2018-11-26 & europejski fundusz społeczny & obecnie w unii europejskiej pracuje 239 mln ludzi, więcej niż kiedykolwiek w historii - oświadczyła unijna komisarz ds. zatrudnienia i spraw społecznych marianne thyssen podczas poniedziałkowej debaty w europejskim komitecie regionów w brukseli. dyskusja dotyczyła europejskiego filaru praw socjalnych. to proklamowany przez unijnych przywódców zbiór 20 zasad, na których mają opierać się sprawiedliwe i sprawnie funkcjonujące rynki pracy i systemy opieki społecznej. jak podkreśliła belgijska komisarz, do europy "powraca nadzieja", której przejawem ma być sytuacja na europejskim rynku pracy. "około 12 mln miejsc pracy utworzono od początku kadencji tej komisji. obecnie w ue pracuje 239 mln ludzi, więcej niż kiedykolwiek w historii. bezrobocie wynosi 6,7 proc. - to najniższy poziom nie tylko od czasu kryzysu, ale od początku tego milenium" - wyliczała thyssen. dodała, że europejskie społeczeństwo staje się coraz bardziej aktywne zawodowo. więcej osób, w tym kobiet, pracuje i szuka pracy, a tego właśnie potrzebuje starzejące się społeczeństwo. według komisarz liczba ludzi zagrożonych biedą i wykluczeniem społecznym jest poniżej poziomu, o którym mowa w strategii 2020. "tylko w zeszłym roku 5 mln ludzi wyszło z biedy i wykluczenia społecznego" - zaznaczyła thyssen. jednocześnie zauważyła, że ponad 40 proc. ludzi pracujących w europie to osoby samozatrudnione albo zatrudnione "niestandardowo". zwróciła też uwagę na różnice w poziomie bezrobocia w krajach wspólnoty, które (według metodologii ue) wynosi ok. 3 proc. w czechach, w niemczech i w polsce, natomiast w grecji - ok. 20 proc. "występują też ogromne różnice regionalne, np. we włoszech" - podkreśliła. zdaniem belgijskiej polityk nadal jest wiele do zrobienia. jak zaznaczyła, dochody gospodarstw domowych nie nadążają za wzrostem pkb, natomiast ludzie muszą poczuć wzrost, "kiedy otrzymują wypłatę". w przeciwnym razie będzie to dla nich wyłącznie statystyka pokazywana w telewizji. wśród wyzwań dla rynku pracy wymieniła m.in. globalizację, robotyzację, zmiany klimatyczne oraz demograficzne. "dzisiaj mamy trzy osoby pracujące na jednego emeryta, w 2070 r. będzie to dwóch pracowników" - powiedziała thyssen. jak dodała, właśnie by sprostać tym wyzwaniom, ue stworzyła europejski filar praw socjalnych. przewodniczący europejskiego komitetu regionów karl-heinz lambertz powiedział, że ue musi zawsze dążyć do wzmacniania statusu swoich obywateli, tworząc godne miejsca pracy i chroniąc ich zdrowie; zapewnić, że nikt "nie pozostanie w tyle". "teraz bardziej niż kiedykolwiek potrzebujemy ambitnego budżetu ue z silną polityką spójności po roku 2020. regiony i miasta są gotowe na odnowienie europy, ale cięcia lub centralizacja funduszy ue - zwłaszcza europejskiego funduszu społecznego - powstrzymają nasze ambicje" - zaznaczył lambertz. z oficjalnych wypowiedzi przedstawicieli komisji europejskiej wynika, że redukcja polityki spójności w przyszłym wieloletnim budżecie ue ma być na poziomie 7 proc. zgodnie z propozycją ke polska jest wśród krajów, które mają dotknąć największe cięcia (23 proc.). proporcjonalnie najwięcej stracić mają węgrzy, czesi, litwini, estończycy i maltańczycy (po 24 proc.). europejski fundusz społeczny (efs), będący narzędziem polityki spójności, jest jednym z pięciu głównych funduszy unijnych. komisja europejska proponuje, aby w latach 2021-27 budżet jego następcy, europejskiego funduszu społecznego plus, wynosił 101,2 mld euro. efs+ będzie skoncentrowany na inwestycjach w ludzi i wspieraniu wdrażania europejskiego filaru praw socjalnych. "filar" został podpisany wspólnie przez parlament europejski, radę ue i komisję europejską w listopadzie ub.r. na szczycie społecznym w goeteborgu (szwecja). jego 20 zasad można uporządkować według trzech kategorii: równe szanse i dostęp do zatrudnienia, uczciwe warunki pracy oraz ochrona socjalna i integracja społeczna. & 534 & medium & Medium & Socio-Economic & NA & NA & 2018-11-26 & 2018 & 3 & ECO
Frame & low-medium & National & 500-1000 & 1.0448136 & 1.8253861 & -0.2867631 & -1.2581601 & -1.2725643 & 0.0 & -1.1973642 & 0.9927066 & Recipient & European & European & European & European|ECO & Positive\\
\addlinespace
Poland & https://fakty.interia.pl/polska/news-premier-wywalczymy-wlasciwy-budzet-ue-dla-polski,nId,2644599 & 466 & fakty.interia.pl & Private/Non-Public & Online only & National & low = CP mentioned more times but NOT important part of story (mainly about others issues) & Institutional bargaining over funding & Factual & National & No myth & NA & NA & NA & NA & NA & NA & NA & NA & Poland & premier: wywalczymy właściwy budżet ue dla polski & 2018-10-15 & fundusze strukturalne & wspólna polityka rolna i fundusze strukturalne muszą być utrzymane w interesie państw doganiających europę zachodnią; będziemy walczyć i wywalczymy właściwy budżet ue dla polski i europy środkowej - powiedział w poniedziałek premier mateusz morawiecki. na wspólnej konferencji prasowej po poniedziałkowych rozmowach z premierem czech andrejem babiszem szef polskiego rządu podkreślił, że polskę i czechy łączy wspólne stanowisko m.in. w odniesieniu do prac nad wieloletnim budżetem ue po roku 2020. reklama "jesteśmy właściwie tego samego zdania, jakich pozycji w tym budżecie należy bronić" - oświadczył morawiecki. jak zaznaczył, zarówno dla pragi, jak i dla warszawy kluczowa jest jakość budżetu, a "tempo jego przyjęcia jest ważne, ale drugorzędne względem jakości". "tutaj nasze interesy muszą być przede wszystkim uwzględniane i nasza pozycja w brukseli jest niezmienna: zarówno wspólna polityka rolna, fundusze dla rolnictwa, fundusze na rozwój dróg, fundusze na rozwój kolei i wszystkie fundusze strukturalne muszą być utrzymane w interesie państw doganiających europę zachodnią" - podkreślił polski premier. przekonywał, że również kraje europy zachodniej "czerpią pewne istotne korzyści" z integracji europejskiej. "częścią tego pakietu ue jest po prostu to, że w ramach długoletnich ram finansowych fundusze infrastrukturalne, na rozwój dróg, kolei i innych elementów infrastruktury muszą być dostępne. my na pewno będziemy walczyć i wywalczymy właściwy budżet dla polski, dla państw europy środkowej" - dodał morawiecki. & 210 & low & Low & Power & NA & NA & 2018-10-15 & 2018 & 3 & POL
Frame & low-medium & National & <500 & 1.0448136 & 1.8253861 & -0.2867631 & -1.2581601 & -1.2725643 & 0.0 & -1.1973642 & 0.9927066 & Recipient & Domestic & Domestic & Domestic & Domestic|POL & Neutral\\
Poland & https://tvn24bis.pl/article/url/887150 & 794 & TVN24 BiS & Private/Non-Public & Online and Offline & National & high = CP is most important issue in story (can also cover other issues) & Jobs & Positive & EU & No myth & NA & NA & NA & NA & NA & NA & NA & NA & Poland & do europy "powraca nadzieja". rekord na rynku pracy & 2018-11-27 & europejski fundusz społeczny & pracodawca z asem w rękawie. "z roku na rok ten benefit staje się coraz popularniejszy" "źródło: tvn24 bis" obecnie w unii europejskiej pracuje 239 milionów ludzi, więcej niż kiedykolwiek w historii - oświadczyła unijna komisarz do spraw zatrudnienia i spraw społecznych marianne thyssen podczas debaty w europejskim komitecie regionów w brukseli. poniedziałkowa dyskusja dotyczyła europejskiego filaru praw socjalnych. to proklamowany przez unijnych przywódców zbiór 20 zasad, na których mają opierać się sprawiedliwe i sprawnie funkcjonujące rynki pracy i systemy opieki społecznej. jak podkreśliła belgijska komisarz, do europy "powraca nadzieja", której przejawem ma być sytuacja na europejskim rynku pracy. - około 12 mln miejsc pracy utworzono od początku kadencji tej komisji. obecnie w ue pracuje 239 mln ludzi, więcej niż kiedykolwiek w historii. bezrobocie wynosi 6,7 proc. - to najniższy poziom nie tylko od czasu kryzysu, ale od początku tego milenium - wyliczała thyssen. dodała, że europejskie społeczeństwo staje się coraz bardziej aktywne zawodowo. więcej osób, w tym kobiet, pracuje i szuka pracy, a tego właśnie potrzebuje starzejące się społeczeństwo. według komisarz liczba ludzi zagrożonych biedą i wykluczeniem społecznym jest poniżej poziomu, o którym mowa w strategii 2020. - tylko w zeszłym roku 5 mln ludzi wyszło z biedy i wykluczenia społecznego - zaznaczyła thyssen. jednocześnie zauważyła, że ponad 40 proc. ludzi pracujących w europie to osoby samozatrudnione albo zatrudnione "niestandardowo". zwróciła też uwagę na różnice w poziomie bezrobocia w krajach wspólnoty, które (według metodologii ue) wynosi ok. 3 proc. w czechach, w niemczech i w polsce, natomiast w grecji - ok. 20 proc. - występują też ogromne różnice regionalne, np. we włoszech - podkreśliła. zdaniem belgijskiej polityk nadal jest wiele do zrobienia. jak zaznaczyła, dochody gospodarstw domowych nie nadążają za wzrostem pkb, natomiast ludzie muszą poczuć wzrost, "kiedy otrzymują wypłatę". w przeciwnym razie będzie to dla nich wyłącznie statystyka pokazywana w telewizji. wśród wyzwań dla rynku pracy wymieniła m.in. globalizację, robotyzację, zmiany klimatyczne oraz demograficzne. - dzisiaj mamy trzy osoby pracujące na jednego emeryta, w 2070 r. będzie to dwóch pracowników - powiedziała thyssen. jak dodała, właśnie by sprostać tym wyzwaniom, ue stworzyła europejski filar praw socjalnych. przewodniczący europejskiego komitetu regionów karl-heinz lambertz powiedział, że ue musi zawsze dążyć do wzmacniania statusu swoich obywateli, tworząc godne miejsca pracy i chroniąc ich zdrowie; zapewnić, że nikt "nie pozostanie w tyle". - teraz bardziej niż kiedykolwiek potrzebujemy ambitnego budżetu ue z silną polityką spójności po roku 2020. regiony i miasta są gotowe na odnowienie europy, ale cięcia lub centralizacja funduszy ue, zwłaszcza europejskiego funduszu społecznego, powstrzymają nasze ambicje - zaznaczył lambertz. z oficjalnych wypowiedzi przedstawicieli komisji europejskiej wynika, że redukcja polityki spójności w przyszłym wieloletnim budżecie ue ma być na poziomie 7 proc. zgodnie z propozycją ke polska jest wśród krajów, które mają dotknąć największe cięcia (23 proc.). proporcjonalnie najwięcej stracić mają węgrzy, czesi, litwini, estończycy i maltańczycy (po 24 proc.). europejski fundusz społeczny (efs), będący narzędziem polityki spójności, jest jednym z pięciu głównych funduszy unijnych. komisja europejska proponuje, aby w latach 2021-27 budżet jego następcy, europejskiego funduszu społecznego plus, wynosił 101,2 mld euro. efs plus będzie skoncentrowany na inwestycjach w ludzi i wspieraniu wdrażania europejskiego filaru praw socjalnych. "filar" został podpisany wspólnie przez parlament europejski, radę ue i komisję europejską w listopadzie ub.r. na szczycie społecznym w goeteborgu (szwecja). jego 20 zasad można uporządkować według trzech kategorii: równe szanse i dostęp do zatrudnienia, uczciwe warunki pracy oraz ochrona socjalna i integracja społeczna. & 554 & high & High & Socio-Economic & NA & NA & 2018-11-27 & 2018 & 3 & ECO
Frame & high-very high & National & 500-1000 & 1.0448136 & 1.8253861 & -0.2867631 & -1.2581601 & -1.2725643 & 0.0 & -1.1973642 & 0.9927066 & Recipient & European & European & European & European|ECO & Positive\\
Poland & https://plus.dziennikzachodni.pl/czestochowa-i-szkoly-europejskie-projekty-edukacyjne-odkryj-w-sobie-talent-i-stawiamy-na-rozwoj/ar/13469092 & a17 & plus.dziennikzachodni.pl & Private/Non-Public & Online and Offline & Regional/Local & medium = CP is important part of story & Jobs & Positive & Subnational & No myth & NA & NA & NA & NA & NA & NA & NA & NA & Poland & częstochowa i szkoły: europejskie projekty edukacyjne  "odkryj w sobie talent" i "stawiamy na rozwój" & 2018-09-05 & europejski fundusz społeczny & po wsparciu dla szkolnictwa zawodowego, miasto uruchamia nowe programy, z dofinansowaniem unijnym, przeznaczone dla szkół podstawowych i liceów. częstochowa uruchamia dwa nowe programy dla częstochowskich szkół. - po programach dla szkolnictwa zawodowego uruchamiany programy dla podstawówek i szkół średnich - mówi prezydent częstochowy, krzysztof matyjaszczyk. - liczymy, że dzięki nim częstochowskie szkoły podstawowe i licea będą lepiej wyposażone i przygotowane do zadań, które przed nimi stoją są to dwa europejskie projekty edukacyjne "odkryj w sobie talent" i "stawiamy na rozwój". ten pierwszy jest dla szkół podstawowych, drugi dla liceów. dotychczas częstochowa z pomocą środków unijnych wspierała szkoły zawodowe, a także przedszkola i żłobki. nowe projekty oparte są o zdobyte przez miasto europejskie fundusze. program "odkryj w sobie talent" jest wart prawie 4,7 mln zł, z czego dofinansowanie wynosi ponad 4,2 mln zł. - to największy tego typu projekt w województwie śląskim - podkreśla piotr grzybowski, naczelnik wydziału funduszy europejskich i rozwoju w urzędzie miasta w częstochowie. - obejmiemy nim wszystkie szkoły podstawowe, które się do nas zgłosiły. europejski fundusz społeczny z reguły wyrównuje szanse edukacyjne, ale nam udało się pozyskać także środki dla najzdolniejszych uczniów. projekt "odkryj w sobie talent" będzie realizowany w 35 szkołach podstawowych w klasach i-viii i obejmie ponad 1500 uczniów, w tym 145 niepełnosprawnych, w trzech grupach. pierwsza z nich, to ci którzy mają problemy w nauce. dzięki projektowi skorzystają ze specjalnie przygotowanych zajęć wyrównawczych. druga grupa to uczniowie szczególnie uzdolnieni, dla których przewidziano indywidualne wsparcie w obszarach, w których przejawiają szczególne zdolności. trzecia grupa to uczniowie z niepełnosprawnościami z klas i-viii i różnego typu specjalistyczne zajęcia terapeutyczne im dedykowane. projekt przewiduje również poszerzenie bazy dydaktycznej. specjalnie dla nich dedykowane będą zajęcia wyrównawcze, a placówki zostaną doposażone w niezbędny sprzęt. & 282 & medium & Medium & Socio-Economic & NA & NA & 2018-09-05 & 2018 & 3 & ECO
Frame & low-medium & Regional & <500 & 1.0448136 & 1.8253861 & -0.2867631 & -1.2581601 & -1.2725643 & 0.0 & -1.1973642 & 0.9927066 & Recipient & Domestic & Domestic & Domestic & Domestic|ECO & Positive\\
Poland & http://tvn24bis.pl/ze-swiata,75/ke-chce-laczenia-srodkow-z-planu-junckera-z-funduszami-unijnymi,693999.html & 570 & TVN24 BiS & Private/Non-Public & Online and Offline & National & high = CP is most important issue in story (can also cover other issues) & Poor communication of funding/rules & Balanced & EU & No myth & NA & NA & NA & NA & NA & NA & NA & NA & Poland & polska ma szansę na więcej pieniędzy z unii. nowy plan brukseli & 2016-11-22 & europejskie fundusze strukturalne i inwestycyjne & ke chciałaby, aby kraje unijne łączyły środki dostępne w ramach planu inwestycyjnego jean-claude'a junckera z europejskimi funduszami strukturalnymi i inwestycyjnymi. dla polski to szansa na większe korzyści z europejskiego funduszu inwestycji strategicznych. europejski fundusz inwestycji strategicznych (efis) to instrument, dzięki któremu do połowy 2018 r. w unijną gospodarkę ma być zainwestowane 315 mld euro. oparty jest na systemie gwarancji mających przyciągać inwestorów do wyłożenia środków na bardziej ryzykowne projekty. europejskie fundusze strukturalne i inwestycyjne to z kolei system bezzwrotnego wsparcia pochodzącego z budżetu ue. z zamówionego przez komisję europejską niezależnego audytu europejskiego funduszu inwestycji strategicznych wynika, że aż 91 proc. dostępnych z jego ramach środków trafiło do tzw. starych krajów ue (piętnastki, która tworzyła wspólnotę przed rozszerzeniem w 2004 r.) do pozostałej 13, czyli głównie krajów europy środkowo-wschodniej, trafiło zaledwie 9 proc. środków. zdaniem ey, które przygotowało raport w tej sprawie, powodem niższego wsparcia efis w naszej części kontynentu jest konkurencja ze strony europejskich funduszy strukturalnych i inwestycyjnych (polska jest największym odbiorcą środków z funduszu spójności w ue). do poprawy sytuacji może się przyczynić łączenie bezzwrotnych funduszy unijnych (europejskich funduszy strukturalnych i inwestycyjnych) ze środkami opartymi na pożyczkach zagwarantowanych przez europejski fundusz inwestycji strategicznych. ścieżką, która ma prowadzić do tego celu, ma być uproszczenie zasad i procedur pozwalających na korzystanie z różnego rodzaju wsparcia. od strony starających się o środki ważne będzie np., że wystarczać będzie jedna aplikacja, aby otrzymać środki z różnych źródeł. "europejskie fundusze regionalne oraz tak zwany plan junckera uzupełniają się. w szczególności w polsce istnieje dużo możliwości na inwestycje, które mobilizują zarówno sektor publiczny, jak i sektor prywatny" - powiedział rzecznik ke odpowiedzialny za fundusze regionalne jakub adamowicz. choć wytyczne ws. tego, jak łączyć europejskie fundusze strukturalne i inwestycyjne z efis, wydano w lutym, dotychczas nie widać efektów tego splatania. jednak - jak przekonują rozmówcy z ke - ma się to zmienić. w procesie zatwierdzania jest już jeden infrastrukturalny projekt z polski. o krok dalej jest francja, gdzie w jednym z departamentów małe i średnie firmy już korzystają i z pożyczek gwarantowanych przez efis i z funduszy strukturalnych. aby zwiększyć wykorzystanie efis, potrzebna jest też praca ze strony państw członkowskich. w ocenie rozmówcy kluczowe jest zwiększanie wiedzy na temat możliwości, jakie daje fundusz, zwłaszcza w połączeniu ze środkami z budżetu ue. w polsce funkcję koordynacyjno-informacyjną ws. dostępnych instrumentów w ramach planu inwestycyjnego dla europy pełni ministerstwo rozwoju. nasz kraj na lata 2014-2020 ma dostać do 86 mld euro z funduszy strukturalnych i inwestycyjnych ue. to największa suma dla pojedynczego kraju członkowskiego z wynoszącej w skali ue 454 mld euro puli. skuteczne łączenie tych środków z kredytami gwarantowanymi przez efis nie jest bez znaczenia w kontekście nowego wieloletniego budżetu ue. już w tej perspektywie dwukrotnie zwiększył się udział instrumentów finansowych (czyli gwarancji na konkretne przedsięwzięcia zamiast prostych grantów) w porównaniu z poprzednią siedmiolatką. jeśli ten trend się utrzyma, doświadczenie administracji i biznesu w splataniu funduszy z unijnego budżetu z planem junckera będzie procentowało lepszym wykorzystaniem mechanizmów finansowych po 2020 r. z danych komisji europejskiej wynika, że operacje zatwierdzone w ramach planu inwestycyjnego dla europy przez europejski bank inwestycyjny osiągnęły dotychczas wartość 24 mld euro. mają one uruchomić inwestycje o łącznej wartości 138,3 mld euro. propozycje projektów, które mogą być wspierane gwarancjami z efis, państwa członkowskie i inwestorzy prywatni powinni przesyłać do europejskiego banku inwestycyjnego. & 550 & high & High & Governance & NA & NA & 2016-11-22 & 2016 & 2 & POL
Frame & high-very high & National & 500-1000 & 1.0448136 & 1.8253861 & -0.2867631 & -1.2581601 & -1.2725643 & 0.0 & -1.1973642 & 0.9927066 & Recipient & European & European & European & European|POL & Neutral\\
Poland & http://www.gazetaprawna.pl/artykuly/1112902,oettinger-ciecia-w-polityce-spojnosci-na-poziomie-bliskim-5-proc.html & 903 & gazetaprawna.pl & Private/Non-Public & Online and Offline & National & high = CP is most important issue in story (can also cover other issues) & Institutional bargaining over funding & Balanced & EU + National & No myth & NA & NA & NA & NA & NA & NA & NA & NA & Poland & oettinger: cięcia w polityce spójności na poziomie bliskim 5 proc. & 2018-03-22 & polityka spójności & komisarz ue ds. budżetu guenther oettinger powiedział w czwartek w brukseli, że trzeba zaakceptować cięcia nieszkodliwe dla polityki spójności na poziomie bliskim 5 proc. podkreślił, że pozostawienie tej polityki na obecnym poziomie może się okazać niewykonalną misją. "z powodu brexitu i nowych zobowiązań nie mogę zagwarantować, że nie będzie cięć. musimy być realistami, musimy zaakceptować cięcia nieszkodliwe dla polityki spójności. nie będzie to minus 30 proc., minus 20 proc. czy minus 15 proc., lecz mniej niż 10 proc., bliżej 5 proc." - oświadczył oettinger. komisarz uczestniczył w konferencji zorganizowanej przez europejski komitet regionów oraz europejskie stowarzyszenia samorządów, podczas której wręczono mu petycję w sprawie konieczności utrzymania silnej pozycji polityki spójności w kolejnym budżecie ue. pod koniec zeszłego roku pap dotarła do dokumentów roboczych komisji europejskiej, w których rozważane były trzy scenariusze przyszłej polityki spójności. dwa z nich wiązały się z uszczupleniem jej budżetu o 15 proc. bądź 30 proc. w cenach bieżących (odpowiednio 26 proc. oraz 39 proc. w odniesieniu do cen z r. 2011). polska zasadniczo nie traciła w analizowanych wariantach, ponieważ zakładane cięcia obejmowały głównie regiony państw tzw. starej unii, natomiast ryzyko utraty środków dotyczyło relatywnie bogatych województw: mazowsza i dolnego śląska. (nie uwzględniano wtedy jeszcze statystycznego podziału województwa mazowieckiego, który miał miejsce 1 stycznia). tylko jeden scenariusz w dokumentach ke zakładał wzrost budżetu polityki spójności po 2020 r. ue miałaby wygospodarować o 15 proc. więcej niż obecnie, mimo to opcję tę nazwano "zamrożeniem" w związku z odwołaniem się do cen z 2011 r. w celu wyeliminowania wpływu inflacji. łącznie najbardziej pesymistyczne warianty przyjęte w dokumentach pochodzących z dwóch dyrekcji generalnych ke oznaczałyby uszczuplenie przyszłego budżetu polityki spójności znacznie przekraczające 100 mld euro. członek komisji budżetowej pe europoseł jan olbrycht powiedział pap, że wątpi w prawdopodobieństwo takiego rozwoju wydarzeń. według niego całkowite zakręcenie kurka finansowego byłoby nie do zaakceptowania przez gminy z francji czy hiszpanii. w jego ocenie mało realne jest też, by uszczuplenie budżetu polityki spójności nie dotknęło polski. "wersją akceptowalną jest polityka spójności dla wszystkich przy mniejszym budżecie, czyli wersja - tniemy wszystkim. wtedy każdy jest zainteresowany polityką spójności, jest to w dalszym ciągu polityka inwestycyjna, prorozwojowa" - powiedział europoseł. ekskluzywnemu podejściu do funduszy unijnych sprzeciwiają się same regiony. europejski komitet regionów, unijny organ doradczy, w którym zasiadają przedstawiciele samorządów z państw członkowskich ue, manifestuje potrzebę kontynuowania ogólnounijnej polityki spójności. szukanie oszczędności w funduszach unijnych wynika przede wszystkim z brexitu. wielka brytania rozpoczęła proces wyjścia z ue i powinna ją opuścić do końca marca 2019 roku, co prawdopodobnie uszczupli unijny budżet o składkę jednego z największych płatników netto. nie wiadomo jeszcze, ile dokładnie wyniesie to uszczuplenie. szacunki ke mówią o 10-14 mld euro rocznie. lukę finansową dodatkowo pogłębią szacowane na 10 mld wydatki na nowe cele związane m.in. z migracją, bezpieczeństwem czy obronnością. z brukseli mateusz kicka (pap) & 461 & high & High & Power & NA & NA & 2018-03-22 & 2018 & 3 & POL
Frame & high-very high & National & <500 & 1.0448136 & 1.8253861 & -0.2867631 & -1.2581601 & -1.2725643 & 0.0 & -1.1973642 & 0.9927066 & Recipient & Domestic & European & Mixed & Domestic|POL & Neutral\\
\addlinespace
Poland & http://forsal.pl/artykuly/1113579,oettinger-nie-bedziemy-proponowac-osobnego-budzetu-dla-strefy-euro.html & 839 & forsal.pl & Private/Non-Public & Online only & National & low = CP mentioned more times but NOT important part of story (mainly about others issues) & Institutional bargaining over funding & Factual & National & No myth & NA & NA & NA & NA & NA & NA & NA & NA & Poland & oettinger: nie będziemy proponować osobnego budżetu dla strefy euro & 2018-03-26 & polityka spójności & "nie będziemy proponować żadnego budżetu dla eurolandu"; "chcemy mieć budżet 27 krajów, w którym będą zawarte interesy wszystkich państw ue"; "musimy się integrować, nie chcemy europy różnych prędkości" - powiedział guenther oettinger na wspólnym posiedzeniu sejmowej komisji ds. ue oraz finansów publicznych. guenther oettinger przedstawia w poniedziałek na posiedzeniu komisji informację na temat planów dotyczących przyszłych wieloletnich ram finansowych. prezentowany jest programu prac komisji europejskiej na 2018 rok. polsko-unijny kompromis polska i komisja europejska mają podobną wizję budowania kompromisu ws. wieloletniego budżetu ue - powiedział pap wiceszef msz ds. europejskich konrad szymański po poniedziałkowym spotkaniu w warszawie z unijnym komisarzem ds. budżetu i zasobów ludzkich guentherem oettingerem. jak podkreślił wiceminister, zarówno komisja, jak i rząd w warszawie widzą "potrzebę reform i sfinansowania nowych celów np. w zakresie bezpieczeństwa, ale nie kosztem polityki spójności, czy polityki rolnej". "gdyby budżet zależał tylko od dobrego porozumienia brukseli i warszawy, o kompromis byłoby bardzo łatwo. sytuacja polityczna w niektórych państwach członkowskich jest jednak na tyle trudna, że zbliżające się negocjacje będą bardzo trudne" - dodał szymański. ocenił przy tym, że ograniczenie pieniędzy unijnych na skutek brexitu jest "tylko jednym z problemów", jakie stoją przed uczestnikami negocjacji ws. wieloletnich ram finansowych ue po roku 2020. komisarz oettinger od połowy 2017 r. spotyka się z premierami, ministrami finansów i ministrami spraw zagranicznych państw członkowskich, aby zapoznać się z ich stanowiskiem ws. przyszłych wieloletnich ram finansowych ue po 2020 r. dotychczas odwiedził 21 krajów. podczas poniedziałkowej wizyty w warszawie oettinger spotkał się z premierem mateuszem morawieckim, rozmawiał ponadto także z wiceministrami finansów piotrem nowakiem i tomaszem robaczyńskim. po południu rozmawiali m.in. o przyszłości finansów unii europejskiej, w tym wspólnotowego budżetu po 2020 r. jak poinformowała kancelaria premiera, podczas spotkania szefa rządu z unijnym komisarzem ds. budżetu i zasobów strona polska przedstawiła najważniejsze polskie postulaty dotyczące przyszłego budżetu wspólnoty. premier - poinformowano - "wskazał, że polska opowiada się za ambitnym budżetem ue, który będzie uwzględniał pozytywne efekty generowane przez politykę spójności i wspólną politykę rolną". również przedstawiciele ministerstwa finansów - jak poinformowano w komunikacie resortu - podkreślili na spotkaniu z komisarzem ue, że polska popiera przyjęcie ambitnego budżetu ue po roku 2020. "konieczne jest także znalezienie równowagi między finansowaniem nowych wyzwań a tradycyjnymi politykami ue, takimi jak polityka spójności i wspólna polityka rolna. zwrócili także uwagę na kwestie związane z zarządzaniem budżetem ue, w tym na konieczność zapewnienia odpowiedniego poziomu środków w kolejnych, rocznych budżetach ue" - poinformował resort finansów. w poniedziałek po południu komisarz oettinger wziął udział w posiedzeniu połączonych sejmowych komisji: do spraw unii europejskiej oraz komisji finansów publicznych. marceli sommer, marcin musiał >>> polecamy: pis wygasza konflikty przed wyborami. także te z unią & 431 & low & Low & Power & NA & NA & 2018-03-26 & 2018 & 3 & POL
Frame & low-medium & National & <500 & 1.0448136 & 1.8253861 & -0.2867631 & -1.2581601 & -1.2725643 & 0.0 & -1.1973642 & 0.9927066 & Recipient & Domestic & Domestic & Domestic & Domestic|POL & Neutral\\
Poland & https://www.euractiv.pl/section/polityka-regionalna/news/komisja-europejska-historyczny-rekord-w-ue-pracuje-239-mln-ludzi/ & 475 & EurActiv.pl & Private/Non-Public & Online only & National & very low = CP mentioned once & Jobs & Positive & EU & No myth & NA & NA & NA & NA & NA & NA & NA & NA & Poland & komisja europejska: historyczny rekord - w ue pracuje 239 mln ludzi & 2018-11-29 & europejski fundusz społeczny & "obecnie w ue pracuje 239 mln ludzi, to więcej niż kiedykolwiek w historii" - oświadczyła unijna komisarz ds. zatrudnienia i spraw społecznych marianne thyssen podczas poniedziałkowej debaty w europejskim komitecie regionów. posiedzenie dotyczyło europejskiego filaru praw socjalnych. to proklamowany przez unijnych przywódców zbiór 20 zasad, na których mają opierać się sprawiedliwe i sprawnie funkcjonujące rynki pracy i systemy opieki społecznej. jak podkreśliła komisarz, do krajów ue "powraca nadzieja", której przejawem ma być sytuacja na rynku pracy. "około 12 mln miejsc pracy utworzono od początku kadencji tej komisji. obecnie w ue pracuje 239 mln ludzi, to więcej niż kiedykolwiek w historii. bezrobocie wynosi 6,7 proc. - to najniższy poziom nie tylko od czasu kryzysu, ale od początku tego millenium" - powiedziała marianne thyssen. zauważyła także, że ponad 40 proc. ludzi pracujących to osoby samozatrudnione. zwróciła uwagę na różnice w poziomie bezrobocia w krajach, których wskaźnik wynosi tylko 3 proc. (jak np. w niemczech, a takich gdzie jest to ok. 20 proc., jak w grecji. zdaniem belgijskiej komisarz nadal jest wiele do zrobienia. jak podkreśliła, dochody gospodarstw domowych nie nadążają za wzrostem pkb. wśród wyzwań dla rynku pracy wymieniła m.in. globalizację, robotyzację oraz wyzwania demograficzne. "dzisiaj mamy trzy osoby pracujące na jednego emeryta, w 2070 r. będzie to już tylko dwóch pracowników" - dodała thyssen. jak podkreśliła,właśnie aby sprostać tym wyzwaniom, ue stworzyła filar praw socjalnych. przewodniczący europejskiego komitetu regionów karl-heinz lambertz powiedział, że ue musi dążyć do poprawiania statusu materialnego swoich obywateli, tworząc godne miejsca pracy i chroniąc ich zdrowie. "teraz bardziej niż kiedykolwiek potrzebujemy ambitnego budżetu ue z silną polityką spójności. regiony i miasta są gotowe na odnowienie europy, ale cięcia lub centralizacja funduszy ue - zwłaszcza europejskiego funduszu społecznego - powstrzymają nasze plany i ambicje" - zaznaczył lambertz. z oficjalnych wypowiedzi przedstawicieli komisji europejskiej wynika, że redukcja polityki spójności w przyszłym wieloletnim budżecie ue będzie oscylować na poziomie 7 proc. zgodnie z propozycją ke polska jest wśród krajów, które mają dotknąć największe cięcia (ok. 23 proc.). proporcjonalnie najwięcej stracić mają zaś węgrzy, czesi, litwini, estończycy i maltańczycy (po ok. 24 proc.). europejski fundusz społeczny (efs), będący narzędziem polityki spójności, jest jednym z pięciu głównych funduszy unijnych. komisja europejska proponuje, aby w latach 2021 -2027 budżet jego następcy - europejskiego funduszu społecznego plus- wynosił 101,2 mld euro. & 374 & very low & Low & Socio-Economic & NA & NA & 2018-11-29 & 2018 & 3 & ECO
Frame & v.low & National & <500 & 1.0448136 & 1.8253861 & -0.2867631 & -1.2581601 & -1.2725643 & 0.0 & -1.1973642 & 0.9927066 & Recipient & European & European & European & European|ECO & Positive\\
Poland & http://www.wspolczesna.pl/apps/pbcs.dll/article?AID=\%2F20150225\%2FREG06\%2F150229846 & a4 & wspolczesna.pl & Private/Non-Public & Online and Offline & Regional/Local & very low = CP mentioned once & Jobs & Positive & Subnational & No myth & NA & NA & NA & NA & NA & NA & NA & NA & Poland & łapy. młodzi ludzie dzięki praktykom uczą się zawodu & 2015-02-25 & europejski fundusz społeczny & 31 kradzionych audi odzyskanych przez policję. złodzieje samochodów i paserzy zatrzymani (zdjęcia) w pracy handlowca ważne jest dbanie o siebie - mówią marlena jenda i edyta leśniewska z łap. wczoraj malowały paznokcie, a dzisiaj będą robiły makijaż. dziewczyny uczestniczą w projekcie "równi na rynku pracy" finansowanego przez europejski fundusz społeczny. realizuje go ochotniczy hufiec pracy w białymstoku. adresatami jest czternastu młodych osób z białegostoku i 15 z łap. młodzież poznaje m.in. tajniki zawodu handlowiec - sprzedawca. - zależy nam, by uczestnicy projektu zdobyli wiedzę teoretyczną oraz nabyli umiejętności przydatne w pracy - mówi aneta pełszyńska z centrum edukacji pracy młodzieży w białymstoku. projekt ohp potrwa do grudnia. w tym czasie młodzież zostanie skierowana na półroczne i płatne staże. uczestniczy liczą, że po zakończeniu praktyk znajdą pracę w firmach. & 125 & very low & Low & Socio-Economic & NA & NA & 2015-02-25 & 2015 & 1 & ECO
Frame & v.low & Regional & <500 & 1.0448136 & 1.8253861 & -0.2867631 & -1.2581601 & -1.2725643 & 0.0 & -1.1973642 & 0.9927066 & Recipient & Domestic & Domestic & Domestic & Domestic|ECO & Positive\\
Poland & https://plus.dziennikzachodni.pl/wiadomosci/a/fundusz-solidarnosci-w-metropolii-60-mln-zl-dla-mniejszych-gmin-w-gornoslaskozaglebiowskiej-metropolii,12776847 & a10 & plus.dziennikzachodni.pl & Private/Non-Public & Online and Offline & Regional/Local & high = CP is most important issue in story (can also cover other issues) & Economic development & Positive & Subnational & No myth & NA & NA & NA & NA & NA & NA & NA & NA & Poland & fundusz solidarności w metropolii: 60 mln zł dla mniejszych gmin w górnośląsko-zagłębiowskiej metropolii & 2017-12-18 & polityka spójności & fundusz solidarności w górnośląsko-zagłębiowskiej metropolii: w budżecie na 2018 rok, który jutro przyjmie metropolia, znajdzie się 60 mln zł na inwestycje w mniejszych i biedniejszych miastach. fundusz solidarności ma pomóc w wyrównywaniu szans metropolitalnych gmin polityka spójności, a więc najprościej mówiąc - systemowe wspieranie słabszych, mniejszych i biedniejszych, to w zamyśle miał być jeden z kluczowych celów górnośląsko-zagłębiowskiej metropolii. w samym jej centrum są bowiem obszary o ogromnych różnicach rozwojowych. w przyszłorocznym budżecie związku, który ma zostać uchwalony na jutrzejszej sesji jego zgromadzenia, na politykę spójności zapisano 60 mln zł. - to wsparcie dla biedniejszych i słabszych gmin, by się mogły zbliżać standardem usług oferowanych mieszkańcom do swoich bogatszych sąsiadów - mówi kazimierz karolczak, przewodniczący zarządu metropolii. pieniądze mają być przeznaczane na inwestycje o znaczeniu ponadlokalnym. zostaną rozdysponowane po równo pomiędzy 5 podregionów metropolii, które będą musiały zgodnie się nimi podzielić. jutro rozstrzygną się założenia funduszu solidarności, który mniejszym i biedniejszym gminom metropolii zapewni solidne wsparcie inwestycyjne. to już pewne, że fundusz zacznie działać od przyszłego roku. samorządowcy są zgodni co do idei, ale pozostaje kwestia uzgodnienia, jakie pieniądze na metropolitalną politykę spójności będą chcieli przeznaczyć: 30, 60 czy 100 milionów złotych. tak zwany fundusz solidarności był jedną z pierwszych programowych zapowiedzi kazimierza karolczaka, przewodniczącego zarządu metropolii. to niejako odpowiedź na potrzebę wyrażaną w wielu dyskusjach poprzedzających integrację. prezydenci najlepiej prosperujących miast aglomeracji sami wielokrotnie akcentowali potrzebę mechanizmu, dzięki któremu będzie możliwe coś, co określilibyśmy mianem "likwidowania barier rozwojowych" na całym obszarze metropolitalnym. przed ustawą metropolitalną nie było to prawnie wykonalne. & 250 & high & High & Socio-Economic & NA & NA & 2017-12-18 & 2017 & 2 & ECO
Frame & high-very high & Regional & <500 & 1.0448136 & 1.8253861 & -0.2867631 & -1.2581601 & -1.2725643 & 0.0 & -1.1973642 & 0.9927066 & Recipient & Domestic & Domestic & Domestic & Domestic|ECO & Positive\\
Poland & http://www.pb.pl/4650109,96092,ue-ma-budzet-na-2017-r & 538 & pb.pl & Private/Non-Public & Online and Offline & National & medium = CP is important part of story & Institutional bargaining over funding & Factual & EU & No myth & Jobs & Factual & EU & No myth & Economic development & Factual & EU & No myth & Poland & ue ma budżet na 2017 r. & 2016-12-01 & europejskie fundusze strukturalne i inwestycyjne & parlament europejski przegłosował w czwartek w brukseli przyszłoroczny budżet ue. przewiduje on zwiększone wydatki na bezpieczeństwo oraz radzenie sobie z kryzysem migracyjnym; priorytetem ma pozostać wspieranie konkurencyjności i wzrostu. zobowiązania ue w 2017 r. mają wynieść 157,8 mld euro, czyli o prawie 3 mld euro mniej niż chciał parlament europejski. środki na realne płatności mają być na poziomie 134,4 mld euro (o 2 mld euro mniej niż postulowali europosłowie). pe podkreślił w komunikacie po głosowaniu, że jego negocjatorzy wywalczyli większe wsparcie dla zatrudnienia ludzi młodych, kluczowych inicjatyw dla małych i średnich przedsiębiorstw, projektów dotyczących infrastruktury transportowej, badań oraz mobilności studentów dzięki programowi erasmus+. za przyjęciem budżetu w takim kształcie opowiedziało się 438 europosłów, przeciw był 194, a 7 wstrzymało się od głosu. "osiągnęliśmy nasze cele, ponieważ budżet na rok 2017 jest zgodny z naszymi priorytetami dotyczącymi zwiększenia wzrostu i tworzenia miejsc pracy, szczególnie dla ludzi młodych, oraz zarządzania kryzysem migracyjnym. dodatkowe 500 milionów euro na inicjatywę na rzecz zatrudnienia ludzi młodych jest jasnym sygnałem do działania w ramach ue. osiągnęliśmy również to, co było możliwe na polu walki z przyczynami migracji" - oświadczył odpowiedzialny za budżet niemiecki europoseł z frakcji socjalistycznej jens geier. aby zaradzić kryzysowi migracyjnemu i wzmocnić bezpieczeństwo, ue zaplanowała na 2017 r. wzrost zobowiązań na te cele o ponad 11 proc. w porównaniu z obecnym rokiem. "28" będzie miała do wydania w tym dziale 5,91 mld euro. środki te pójdą m.in. na wsparcie państw członkowskich w przyjmowaniu rozdzielanych uchodźców, tworzenie ośrodków przyjęć, wsparcie działań integracyjnych, ale także sfinansowanie powrotów tych, którzy nie spełniają wymogów azylowych. z drugiej strony ue ma wzmacniać ochronę granic, podejmować działania przeciw terroryzmowi oraz przestępstwom i chronić krytyczną infrastrukturę. prawie połowa przyszłorocznej unijnej kiesy (74,9 mld euro w zobowiązaniach) ma służyć stymulowaniu wzrostu gospodarczego, tworzeniu miejsc pracy oraz zwiększaniu konkurencyjności. dla przykładu: 21,3 mld euro pójdzie na program finansowania badań naukowych i innowacji horyzont 2020, erasmus+, program służący konkurencyjności przedsiębiorstw cosme czy inicjatywę infrastrukturalną łącząc europę. wzrost w tym dziale wynosi 12 proc. w porównaniu z 2016 r. 2,7 mld euro, czyli o 25 proc. więcej niż w tym roku, zabezpieczono dla europejskiego funduszu inwestycji strategicznych, z którego pochodzą gwarancje na kredyty mające pomóc wygenerować 315 mld euro inwestycji do połowy 2018 r. dodatkowe pieniądze - 500 mln euro - zostaną, tak jak chciał parlament europejski, skierowane na wsparcie inicjatywy mającej pomagać młodym odnaleźć się na rynku pracy. budżet programu wymiany studenckiej oraz praktyk erasmus+ został zwiększony o 19 proc. - do 2,1 mld euro. ogromna część unijnego budżetu, bo 53,59 mld euro, zostanie skierowana na wspieranie wyrównywania poziomu gospodarczego państw unijnych i regionów poprzez europejskie fundusze strukturalne i inwestycyjne. w części dotyczącej polityki spójności w ue przewiduje on w przyszłym roku tąpnięcie, jeśli chodzi o płatności. na ten cel zarezerwowano 37 mld euro, czyli o ponad 11 proc. mniej niż w obecnym roku. nie wynika to jednak ze zmniejszenia nakładów na ważną dla polski politykę spójności, ale z wolnego rozkręcania się projektów w obecnym wieloletnim budżecie ue i tym samym mniejszej liczby faktur do opłacenia. inny duży dział wydatków to środki dla rolników, dla których zarezerwowano w przyszłym roku 42,6 mld euro. uzgodniono też wynoszące 500 mln euro wsparcie dla borykających się z kryzysem producentów mleka oraz wieprzowiny. program ich wsparcia na tę kwotę został ogłoszony jeszcze w wakacje. & 553 & medium & Medium & Power & Socio-Economic & Socio-Economic & 2016-12-01 & 2016 & 2 & POL
Frame & low-medium & National & 500-1000 & 1.0448136 & 1.8253861 & -0.2867631 & -1.2581601 & -1.2725643 & 0.0 & -1.1973642 & 0.9927066 & Recipient & European & European & European & European|POL & Neutral\\
\addlinespace
Poland & http://wgospodarce.pl/informacje/49544-ekg2018-polska-chce-dobrego-unijnego-budzetu?utm\_source=feedburner\&utm\_medium=feed\&utm\_campaign=Feed\%3A+wGospodarce+\%28wGospodarce.pl\%29\&utm\_content=FeedBurner & 243 & wgospodarce.pl & Private/Non-Public & Online only & National & medium = CP is important part of story & Institutional bargaining over funding & Balanced & National & No myth & NA & NA & NA & NA & NA & NA & NA & NA & Poland & ekg2018: polska chce dobrego unijnego budżetu & 2018-05-15 & polityka spójności & tagi: budżet ekg finanse jerzy kwieciński katowice ministerstwo inwestycji i rozwoju ue przedstawione przez komisję europejską propozycje dotyczące budżetu unii europejskiej na nową siedmioletnią perspektywę - to jeden z najważniejszych tematów dyskutowanych podczas europejskiego kongresu gospodarczego w katowicach. minister inwestycji jerzy kwieciński przekonywał, że potrzeba ambitnego budżetu ue oraz silnej polityki spójności i to pomimo brexitu. polityka spójności jest kluczowym narzędziem stymulującym inwestycje, a w konsekwencji zapewniającym długofalowy wzrost unii europejskiej i dobrobyt jej mieszkańców. polska popiera budżet dopasowany do ambicji politycznych i postuluje finansowanie nowych wyzwań z nowych środków, sprzeciwiając się rozwiązaniom polegającym na prostych transferach z dotychczasowych polityk - mówił jerzy kwieciński. przedstawione przez ke propozycje w zakresie polityki spójności to redukcja o 7 proc., co jak tłumaczy ke, wynika z brexitu i nowych wyzwań ue - potrzeby zwiększenia środków na cyfryzację, innowacje, młodzież, zmiany klimatyczne, bezpieczeństwo i migrację. jak mówił j. kwieciński zsypywanie powstałej luki nie powinno mieć miejsca kosztem cięć polityk traktatowych. jeśli teraz ograniczymy środki na działania prorozwojowe, m.in. w ramach polityki spójności - czy to poprzez cięcia budżetowe czy wygórowane warunki, utrudniające dostęp beneficjentów do funduszy - to negatywne skutki tych działań będziemy doświadczać przez wiele kolejnych lat - stwierdził j. kwieciński. polska jest zainteresowana nie tylko wielkością budżetu, ale także jakością - na co środki będą przeznaczone i jak ostatecznie wpłyną na długofalowy rozwój społeczno-gospodarczy ue i krajów członkowskich. jest na razie zbyt wcześnie, by kompleksowo ocenić propozycję ke w zakresie wieloletnich ram finansowych po 2020 r. zaproponowany budżet na lata 2021-2027 ma zupełnie inną strukturę niż obecnie oraz, na tym etapie, nie zawiera jeszcze wielu szczegółowych rozstrzygnięć. dział, w którym znajduje się polityka spójności obejmuje wiele "kieszeni" i składa się z kopert na rozwój regionalny, pogłębianie unii gospodarczo-walutowej oraz spójność społeczną. odnosząc się do sposobu dzielenia środków szef resortu inwestycji i rozwoju stwierdził, że w przypadku polityki spójności jesteśmy przekonani do stosowania dotychczasowej tzw. metody berlińskiej, która polega na określaniu kwot dla poszczególnych regionów i krajów członkowskich w oparciu o produkt krajowy brutto na mieszkańca. oznacza to, że im region jest biedniejszy tym więcej otrzymuje środków . teraz pojawiają się głosy, żeby poza tym kryterium wykorzystać inne np. te związane z bezrobociem wśród młodych czy z polityka migracyjną. polska uważa, że nie jest to dobre podejście. polityka spójności jest najważniejszą polityką inwestycyjną ue, służy temu by tworzyć nowe miejsca pracy i generować wzrost gospodarczy a wydaje się, że do tego najlepszym wskaźnikiem jest pkb na mieszkańca. jest to zgodne z zapisami art. 174 traktatu o funkcjonowaniu ue, intensywność wsparcia ue powinna być różnicowana ze względu na poziom rozwoju społeczno-gospodarczego poszczególnych regionów - wyjaśnił j. kwieciński. europejski kongres gospodarczy to także doskonała okazja do wielu spotkań dwustronnych. o współpracy gospodarczej, sytuacji ukraińców na polskim rynku pracy oraz programach współpracy transgranicznej j. kwieciński rozmawiał z wicepremierem, ministrem rozwoju gospodarczego i handlu ukrainy stepanem kubiwem. w 2017 r. na liście polskich partnerów eksportowych, ukraina uplasowała się na 14. pozycji z udziałem 2,1 proc. w łącznym wywozie oraz na 22. miejscu na liście polskich partnerów importowych z udziałem 1,1 proc. w przywozie ogółem. obie strony są zainteresowane dalszym rozwojem współpracy. duży wpływ na rozwój relacji gospodarczych ma umowa o pogłębionej i całościowej strefie wolnego handlu między unią europejską a ukrainą (dcfta), która zniosła wiele barier we wzajemnej wymianie handlowej. znaczącym krokiem w relacjach gospodarczych polski i ukrainy jest polsko-ukraińska umowa o udzieleniu kredytu w ramach pomocy wiązanej, podpisana w 2015 r., zgodnie z którą rząd rp udzielił rządowi ukrainy preferencyjnego kredytu do wysokości 100 milionów euro przeznaczonego na finansowanie eksportu towarów i usług z polski na ukrainę, w szczególności w zakresie modernizacji infrastruktury drogowej granic i budowy ukraińsko-polskich przejść granicznych. polska jest zaangażowana niemal we wszystkie europejskie programy wielostronne wsparcia ukrainy, w tym przede wszystkim w ramach partnerstwa wschodniego. w relacjach polsko-ukraińskich ważnym tematem dla obu narodów jest historia i dialog historyczny należy prowadzić. trzeba też budować i wzmacniać relacje gospodarcze, wymianę handlową, przepływ osób. są to niezwykle istotne kwestie dla kształtowania dwustronnych stosunków - stwierdził j. kwieciński. rozmawiano o promowanej przez polskę nowej strategii makroregionalnej - strategii karpackiej. w zamierzeniu ma to być dokument wzmacniający dotychczasowe formy współpracy na obszarze karpat, gdzie istnieje co prawda wiele form współdziałania, jednak nie obejmują one wszystkich państw lub mają selektywny charakter tematyczny. doceniamy poparcie ukrainy dla projektu strategii karpackiej. dużą rolę odgrywa w niej współpraca gospodarcza w makroregionie. mam nadzieję na stworzenie trwałych sieci powiązań gospodarczych, które przyczynią się do rozwoju wszystkich państw położonych na tym obszarze - podkreślił j. kwieciński. europejski kongres gospodarczy w katowicach potrwa do środy. w spotkaniach uczestniczyć będą m.in. wiceministrowie inwestycji i rozwoju artur soboń i witold słowik. & 764 & medium & Medium & Power & NA & NA & 2018-05-15 & 2018 & 3 & POL
Frame & low-medium & National & 500-1000 & 1.0448136 & 1.8253861 & -0.2867631 & -1.2581601 & -1.2725643 & 0.0 & -1.1973642 & 0.9927066 & Recipient & Domestic & Domestic & Domestic & Domestic|POL & Neutral\\
Poland & http://www.pap.pl/aktualnosci/news,1364034,kwiecinski-szwecja-poprze-kontynuacje-polityki-spojnosci.html & 194 & pap.pl & Public & Online only & National & very high = CP is most important issue + CP is mentioned in title/headline & Institutional bargaining over funding & Factual & EU + National & No myth & NA & NA & NA & NA & NA & NA & NA & NA & Poland & kwieciński: szwecja poprze kontynuację polityki spójności & 2018-04-09 & polityka spójności & szwecja będzie popierać kontynuację polityki spójności - powiedział pap w poniedziałek minister inwestycji i rozwoju jerzy kwieciński, który w sztokholmie rozmawiał o budżecie ue, o polityce spójności po roku 2020 i współpracy gospodarczej polsko-szwedzkiej. wizyta kwiecińskiego w sztokholmie była kolejną turą rozmów z przedstawicielami rządów państw członkowskich w sprawie przyszłości polityki spójności oraz budżetu ue w następnej perspektywie finansowej. celem serii spotkań, które minister odbywa w stolicach europejskich jest prezentacja polskiego stanowiska w sprawie polityki spójności oraz nowego budżetu ue. "to dla nas ważny moment, ponieważ 2 maja komisja europejska ma przedstawić swoją propozycję nowego budżetu, jego kształtu. staramy się rozmawiać przede wszystkim z tymi krajami ue, o których wiemy, że są mniej sympatycznie nastawione do polityki spójności, czy też do zwiększenia roli budżetu europejskiego w nowej perspektywie finansowej" - powiedział pap kwieciński. wyjaśnił, że do takich krajów zaliczają się państwa skandynawskie, obok szwecji, dania i finlandia. generalnie są one przeciwne zwiększaniu budżetu ue i jego roli, jednocześnie są one dużymi płatnikami netto do budżetu wspólnoty. kwieciński powiedział, że podczas rozmów z partnerami ze szwecji przedstawił racje, które przemawiają za tym, aby kontynuować politykę spójności w nowym budżecie ue, wyjaśnił jej znaczenie dla polski, regionu i całej europy. minister spotkał się z szwedzkim ministrem ds. obszarów wiejskich w resorcie przedsiębiorstw i innowacji, svenem-erikiem buchtem, który odpowiada za politykę spójności w szwecji. "nie są to łatwe rozmowy, szczególnie ze szwecją. to kraj, który w niewielkim stopniu korzysta z polityki spójności" - zauważył. dodał, że "usłyszeliśmy, że szwecja będzie popierała kontynuację polityki spójności w nowej perspektywie". "co ważniejsze, minister powiedział, że podziela nasz pogląd, iż powinna to być polityka spójności dla całej ue, dla wszystkich jej regionów, a nie tylko dla tych regionów najbiedniejszych. to jeden z naszych podstawowych celów" - podkreślił kwieciński. "zgodziliśmy się wspólnie co do tego, że ta polityka powinna być w większym stopniu nastawiona na wzmacnianie konkurencyjności gospodarki ue, w szczególności na finansowanie badań, rozwoju i innowacji. to coś - uważamy - co powinno być silnie zaznaczone w nowej perspektywie finansowej, aby finansowanie tych działań było nie tylko z poziomu unijnego w programie horyzont 2020, jak to ma miejsce w tej chwili, ale żeby miało również swoje znaczące miejsce w polityce spójności" - powiedział. minister dodał, że zgoda dotyczy również tego, iż w nowej perspektywie finansowej ue powinno nastąpić znaczące uproszczenie całego systemu. inną kwestią, w której strony mówią jednym głosem, to konieczność większej koncentracji tematycznej - środki ue powinny być kierowane na te obszary, które są dla danego kraju najważniejsze a nie wszystkie. według kwiecińskiego oprócz punktów wspólnych w stanowiskach polski i szwecji, są też kwestie trudne w rozmowach: zwiększenie składek do budżetu ue, czy temat praworządności w polsce. stanowisko szwecji, jak wyjaśnił minister, w tej ostatniej sprawie nie zostało bezpośrednio wypowiedziane. "tematem była też kwestia finansowania polityki migracyjnej w nowej perspektywie finansowej" - dodał kwieciński. zauważył, że w krajach skandynawskich polityka migracyjna jest postrzegana głównie przez pryzmat migrantów i uchodźców z południa, a nie ze wschodu. jak mówił, na temat przyszłego budżetu ue planowane są jeszcze rozmowy z przedstawicielami holandii i austrii oraz najważniejszymi - patrząc z tego punktu widzenia - komisarzami ue, którzy mogą mieć wpływ na jego kształt. drugim obszarem rozmów w sztokholmie, jak wyjaśnił szef miir, były dwustronne stosunki gospodarcze polski i szwecji, m.in. temat innowacji i wspierania start-up'ów. zdaniem ministra można się spodziewać lepszej współpracy między naszymi firmami i instytucjami, które zajmują się tymi dziedzinami. "środowisko start-up'ów w sztokholmie jest bardzo żywe i sporo możemy się od szwedów nauczyć. to obszar, który chcemy rozwijać" - zapewnił kwieciński. minister zaznaczył, że widzi potencjał do zwiększenia wymiany gospodarczej polsko-szwedzkiej i zależy mu. podkreślił, że zależy mu, aby współpraca szła w dwóch kierunkach, by polskie firmy wykorzystywały swój potencjał do ekspansji na rynek szwedzki. zgodnie z informacją miir polska importuje ze szwecji przede wszystkim maszyny i urządzenia mechaniczne, elektryczne oraz części do nich, a także produkty przemysłu chemicznego. w drugą stronę, z polski do szwecji, najchętniej sprowadzane są - poza maszynami oraz urządzeniami mechanicznymi - pojazdy, samoloty, statki oraz inne środki transportu. wartość polskiego eksportu do szwecji to ponad 23 mld złotych, z kolei import wynosi niecałe 14 mld zł. polska jest dla szwecji 11. partnerem gospodarczym. szwecja dla polski - dziewiątym. (pap) & 691 & very high & High & Power & NA & NA & 2018-04-09 & 2018 & 3 & POL
Frame & high-very high & National & 500-1000 & 1.0448136 & 1.8253861 & -0.2867631 & -1.2581601 & -1.2725643 & 0.0 & -1.1973642 & 0.9927066 & Recipient & Domestic & European & Mixed & Domestic|POL & Neutral\\
Poland & https://konkret24.tvn24.pl/polska,108/15-lat-polski-w-ue-garsc-wielkich-liczb-ktore-warto-znac,931335.html & 396 & Konkret24 & Private/Non-Public & Online only & National & high = CP is most important issue in story (can also cover other issues) & Economic development & Positive & National & No myth & NA & NA & NA & NA & NA & NA & NA & NA & Poland & 15 lat polski w ue - garść wielkich liczb, które warto znać & 2019-05-01 & polityka spójności & najgorszy natomiast był rok 2016, kiedy do polski z budżetu ue trafiło niespełna 10 mld euro. wtedy też zapłaciliśmy prawie 4,5 mld euro składek. rok ubiegły pod tym względem przyniósł poprawę i 15,7 mld euro. w ustawie budżetowej na 2019 r. dochody ze środków europejskich przewidziano na ok. 69,7 mld złotych, czyli ok. 16 mld euro. w zestawieniach resortu finansów rozliczeń z ue, największą pozycję stanowią fundusze związane z realizacją unijnej polityki spójności. ma ona na celu wspieranie działań prowadzących do wyrównania warunków ekonomicznych i społecznych we wszystkich regionach unii europejskiej. w szczególności unia europejska zmierza do zmniejszenia różnic w poziomie rozwoju regionów oraz likwidacji zacofania najmniej uprzywilejowanych regionów i wysp, w tym obszarów wiejskich. jeśli pkb na jednego mieszkańca w danym regionie jest mniejsze niż 75 proc. średniej w unii europejskiej, wówczas taki region może liczyć na wsparcie z dotacji unijnych. jeśli więc wszystkie regiony danego kraju mają niskie pkb, wówczas cały kraj może liczyć na wsparcie finansowe (tak jest w przypadku polski). polityka spójności jest realizowana dzięki funduszom strukturalnym, tj. europejskiemu funduszowi rozwoju regionalnego (efrr), europejskiemu funduszowi społecznemu (efs) oraz funduszowi spójności (fs). w ramach tych funduszy polska otrzymała ponad 103,4 mld euro - 443 mld złotych. & 202 & high & High & Socio-Economic & NA & NA & 2019-05-01 & 2019 & 3 & ECO
Frame & high-very high & National & <500 & 1.0448136 & 1.8253861 & -0.2867631 & -1.2581601 & -1.2725643 & 0.0 & -1.1973642 & 0.9927066 & Recipient & Domestic & Domestic & Domestic & Domestic|ECO & Positive\\
Poland & https://fakty.interia.pl/swiat/news-spotkanie-kwiecinskiego-z-oettingerem-zastrzezenia-do-propoz,nId,2632587 & 198 & fakty.interia.pl & Private/Non-Public & Online only & National & high = CP is most important issue in story (can also cover other issues) & Institutional bargaining over funding & Positive & EU + National & NA & NA & NA & NA & NA & NA & NA & NA & NA & Poland & spotkanie kwiecińskiego z oettingerem. zastrzeżenia do propozycji ke & 2018-09-17 & polityka spójności & minister inwestycji i rozwoju jerzy kwieciński spotkał się w poniedziałek w cztery oczy z unijnym komisarzem ds. budżetu guentherem oettingerem w budapeszcie. jak powiedział pap, przekazał mu zastrzeżenia do propozycji ke w sprawie nowej perspektywy finansowej ue. "rozmawialiśmy dzisiaj z panem komisarzem. wymieniliśmy poglądy na temat tego, jak wygląda postęp w negocjacjach dotyczących przyszłej perspektywy finansowej, a szczególnie w odniesieniu do polityki spójności. ja przekazałem te zastrzeżenia, o których już od wielu miesięcy mówimy w odniesieniu do propozycji ke" - powiedział. reklama jak wyjaśnił, chodzi o to, że polska i szereg innych krajów zostało "dość niekorzystnie, a można nawet powiedzieć niesprawiedliwie potraktowanych" przez proponowaną metodologię ze strony komisji europejskiej", jak również o nowe reguły wprowadzone przez ke przy przygotowywaniu nowej polityki spójności, dotyczące m.in. warunków takich jak np. poziom dofinansowania proponowany przez komisję europejską dla wszystkich projektów. "to się przekłada bardzo konkretnie na końcową alokację dla naszego kraju i na sposób, w jaki będziemy korzystali z tych pieniędzy unijnych w nowej perspektywie finansowej" - zaznaczył minister. dodał, że rozmowa na ten temat w formacie wyszehradzkim jest planowana na drugą połowę października, ale polska prowadzi też rozmowy bilateralne, np. z węgrami i niemcami. nowy budżet do rozmowy kwiecińskiego z oettingerem doszło na marginesie spotkania przyjaciół polityki spójności, czyli państw ue, które przeciwstawiają się cięciom środków unijnych w tym zakresie. według słów polskiego ministra było ono poświęcone przede wszystkim temu, jak finansować innowacje w ue, a w szczególności jak finansować innowacje z wykorzystaniem środków pochodzących z polityki spójności. ogromne zainteresowanie wzbudziło też - jak zaznaczył - wystąpienie oettingera. "pan komisarz guenter oettinger wyraźnie podkreślił, że (...) im wcześniej będzie wynegocjowany ten nowy budżet, tym wcześniej będzie można rozpocząć inwestycje w nowej perspektywie finansowej, i podkreślił, że ke chciałaby zakończyć te negocjacje przed wyborami do pe na wiosnę przyszłego roku i że również ke jest otwarta na różnego rodzaju uwagi i na propozycje, a to oznacza, że będziemy mogli szukać pola do kompromisu" - powiedział kwieciński. kwestia brexitu oettinger powiedział na konferencji prasowej w budapeszcie, że państwa członkowskie muszą zrozumieć, że z powodu wystąpienia wielkiej brytanii będzie mniej środków ue i dlatego mniej pieniędzy będzie do dyspozycji na politykę spójności. wskazał, że w następnej perspektywie finansowej tych środków będzie mniej o około 80 mld euro, ale obniżkę środków należy przeprowadzić w sposób jak najmniej bolesny. wskazał, że celem jest też to, by ograniczyć różnice rozwojowe między krajami ue i różnice między poziomem pkb na jednego mieszkańca. "to, co absolutnie zostało tutaj podkreślone, to rola polityki spójności dla ue" - powiedział kwieciński polskim dziennikarzom. jak dodał, sam mówił o tym, że polityka spójności jest wprawdzie starą polityką, ale cały czas przechodzi zmiany i cały czas jest to polityka, która "jest z jednej strony najbliżej wszystkich mieszkańców ue, a z drugiej strony jest polityką, która ma olbrzymie znaczenie gospodarcze dla ue". jeżeli zaś chodzi o innowacje, to widać jego zdaniem, że we wszystkich krajach członkowskich "jest silna wola kontynuacji wsparcia na innowacje w nowej perspektywie finansowej". dodał że wiele krajów zwracało uwagę na zapewnienie uzupełniania się polityki spójności przez fundusze z instrumentów, które są na poziomie unijnym, szczególnie programu horyzont 2020. "to najważniejsze nieformalne spotkanie, które jest organizowane w ue" minister ocenił, że inicjatywa spotkań przyjaciół polityki spójności, której autorem była polska, nabrała rozpędu, o czym świadczy choćby obecność w budapeszcie oettingera i dyrektora generalnego do spraw polityki regionalnej i miejskiej (dg regio) komisji europejskiej marca lemaitre'a. "po dzisiejszym spotkaniu możemy śmiało powiedzieć, że ta impreza naprawdę jest bardzo ważna i wpisuje się doskonale w kalendarz wszystkich rozmów i spotkań, które dotyczą polityki spójności w ue, a nawet możemy powiedzieć, że jest to najważniejsze nieformalne spotkanie, które jest organizowane w ue" - powiedział. grupa przyjaciół polityki spójności, z polską jako nieformalnym liderem, w poprzednich negocjacjach nad budżetem pomagała hamować zapędy unijnych płatników netto do ostrego cięcia wydatków. przedstawiony przez ke projekt budżetu przewiduje w porównaniu z obecną "siedmiolatką" cięcia wydatków na politykę spójności w latach 2021-2027 o 7 proc. w przypadku niektórych państw, m.in. polski, cięcia są jednak znacznie głębsze. z budapesztu małgorzata wyrzykowska & 665 & high & High & Power & NA & NA & 2018-09-17 & 2018 & 3 & POL
Frame & high-very high & National & 500-1000 & 1.0448136 & 1.8253861 & -0.2867631 & -1.2581601 & -1.2725643 & 0.0 & -1.1973642 & 0.9927066 & Recipient & Domestic & European & Mixed & Domestic|POL & Positive\\
Poland & http://www.dziennikzachodni.pl/wiadomosci/a/w-wisle-trwa-spotkanie-premier-beaty-szydlo-z-premierem-czech-bohuslavem-sobotka,11568230/ & a8 & Dziennikzachodni.pl & Private/Non-Public & Online and Offline & Regional/Local & low = CP mentioned more times but NOT important part of story (mainly about others issues) & Territorial cooperation & Factual & National + Other country & No myth & NA & NA & NA & NA & NA & NA & NA & NA & Poland & w wiśle trwa spotkanie premier beaty szydło z premierem czech bohuslavem sobotką & 2016-12-12 & polityka regionalna & współpraca dwustronna, polityka regionalna, bieżąca agenda unii europejskiej, polityka bezpieczeństwa - o tym dzisiaj beata szydło będzie rozmawiała w wiśle z czeskim premierem bohuslavem sobotką. premier beata szydło spotyka się dzisiaj z premierem czech bohuslavem sobotką w zameczku prezydenckim w wiśle. spotkanie w wiśle jest okazją do dyskusji na temat infrastruktury drogowej i inwestycji. planowane jest również omówienie kwestii współpracy w ramach grupy wyszehradzkiej, bezpieczeństwa energetycznego, stosunków na linii unia europejska - rosja. & 71 & low & Low & Socio-Economic & NA & NA & 2016-12-12 & 2016 & 2 & ECO
Frame & low-medium & Regional & <500 & 1.0448136 & 1.8253861 & -0.2867631 & -1.2581601 & -1.2725643 & 0.0 & -1.1973642 & 0.9927066 & Recipient & Domestic & European & Mixed & Domestic|ECO & Neutral\\
\addlinespace
Poland & http://www.pap.pl/aktualnosci/news,1300942,szef-komitetu-regionow-w-liscie-do-donalda-tuska-o-ryzyku-rozpadu-ue.html & 419 & pap.pl & Public & Online only & National & medium = CP is important part of story & Institutional bargaining over funding & Factual & EU & No myth & NA & NA & NA & NA & NA & NA & NA & NA & Poland & szef komitetu regionów w liście do donalda tuska o ryzyku rozpadu ue & 2018-02-22 & europejskie fundusze strukturalne i inwestycyjne & w przededniu nowej dekady europa znajduje się na rozdrożu; potrzebujemy ambitnego unijnego budżetu - pisze w liście do przewodniczącego rady europejskiej szef komitetu regionów karl-heinz lambertz. w piątek szefowie państw i rządów będą dyskutować na temat priorytetów politycznych przyszłych wieloletnich ram finansowych na okres po 2020 roku, czyli przyszłego wieloletniego budżetu unijnego. z tego powodu przewodniczący komitetu regionów skierował list do szefa rady europejskiej donalda tuska, który będzie przewodniczył posiedzeniu. "członkowie rady europejskiej mogą wysłać sygnał, że myślą przyszłościowo i pragną wzmocnić projekt europejski, albo przyjąć na siebie ryzyko rozpadu unii europejskiej. w przededniu nowej dekady europa znajduje się na rozdrożu. najnowsza historia wykazała niebezpieczeństwo eurosceptycyzmu i populizmu" - zaznaczył na wstępie belg. "z tych wszystkich powodów zwracam się do rady europejskiej o uwzględnienie przedstawionej przez regiony i miasta wizji przyszłego budżetu ue. (...) jeśli chcemy europy na miarę naszych ambicji, to potrzebujemy także ambitnego budżetu" - dodał. jak zaznaczył lambertz, "europejski komitet regionów domaga się zwiększenia następnego budżetu ue do poziomu 1,3 proc. dochodu narodowego brutto ue-27, jako pułapu wydatków". wyjaśnił przy tym, że wzrost ten należy zapewnić w drodze wkładów krajowych i nowych zasobów własnych ue, co jest zgodne ze stanowiskiem, które ma przyjąć parlament europejski. belgijski polityk zauważył, że polityka spójności jest głównym narzędziem inwestycyjnym ue. "nie możemy finansować nowych wyzwań, takich jak bezpieczeństwo i obrona, kosztem solidarności i inwestowania w przyszłość. nowe zadania oznaczają nowe zasoby. należy dbać o te zasoby i powinny one stanowić ponad jedną trzecią kolejnych wieloletnich ram finansowych" - podkreślił. jego zdaniem, choć polityka spójności "nie jest być może powszechnie rozpoznawalna", stała się najskuteczniejszym środkiem, z pomocą którego ue stawia czoło obecnym wyzwaniom i zajmuje się działaniem w dziedzinie klimatu, migracją, zrównoważonym wzrostem oraz badaniami i innowacjami. powołał się przy tym na dokument komisji europejskiej sprzed tygodnia, w którym napisano, że "budżet ue, a zwłaszcza europejskie fundusze strukturalne i inwestycyjne, stał się głównym źródłem stabilnych inwestycji wspierających wzrost" podczas kryzysu gospodarczego i finansowego. według belga, europejska współpraca terytorialna jest też jednym z głównych zadań w ramach projektu integracji europejskiej, mającym na celu wzmacnianie transgranicznych powiązań oraz więzi osobistych. lambertz zastrzegł jednocześnie, że mając na uwadze stan finansów publicznych w większości państw członkowskich, komitet regionów nie domaga się, by w przyszłości przeznaczać na politykę spójności większe nakłady. będzie natomiast wzywał do przeznaczenia na nią tych samych środków. "członkowie europejskiego komitetu regionów przeprowadzili ponad 140 debat z obywatelami europejskimi (w tym również w polsce - pap). jako politycy ue działający na szczeblu rządów najbliższym obywateli, nasi członkowie wiedzą, że obywatele nie chcą zadowalać się ograniczeniem europy. pragną czegoś ambitniejszego - budżetu na rzecz obywateli z myślą o silnej europie w przyszłości" - zaznaczył w piśmie lambertz. "liczę, że zechce pan przekazać to przesłanie szefom państw i rządów, tak aby krótkowzroczna polityka lub egoizm narodowy nie podważały dziedzictwa, które otrzymaliśmy od ojców założycieli unii" - podsumował belg. jak wynika z danych ke, w polsce środki unijne z polityki spójności w latach 2015-2017 stanowiły 61,17 proc. inwestycji publicznych, co lokuje nasz kraj na czwartym miejscu w ue pod tym względem, po portugalii, chorwacji i litwie. w aktualnej perspektywie finansowej 2014-2020 ue przeznaczyła na ten cel 351,8 mld euro, co stanowi jedną trzecią unijnego budżetu. w tym czasie polska zainwestuje z tych środków 82,5 mld euro, najwięcej spośród państw członkowskich. w brukseli trwają obecnie prace nad przyszłym wieloletnim budżetem unii europejskiej, który będzie obowiązywał po 2020 r. są one szczególnie trudne w kontekście brexitu i pojawienia się wydatków na nowe cele związane m.in. z migracją czy obronnością. dlatego kluczowe w tej sprawie będzie stanowisko unijnego szczytu. europejski komitet regionów jest unijnym organem doradczym, w którym zasiadają przedstawiciele samorządów, w tym 21 z polski. & 606 & medium & Medium & Power & NA & NA & 2018-02-22 & 2018 & 3 & POL
Frame & low-medium & National & 500-1000 & 1.0448136 & 1.8253861 & -0.2867631 & -1.2581601 & -1.2725643 & 0.0 & -1.1973642 & 0.9927066 & Recipient & European & European & European & European|POL & Neutral\\
Poland & https://biznes.interia.pl/news/konrad-szymanski-bez-zgody-dla-projektu-budzetu-ke-w\%2C2585972 & 653 & biznes.interia.pl & Private/Non-Public & Online only & National & very low = CP mentioned once & Institutional bargaining over funding & Factual & National & No myth & NA & NA & NA & NA & NA & NA & NA & NA & Poland & konrad szymański: bez zgody dla projektu budżetu ke w obecnym kształcie & 2018-09-18 & polityka spójności & minister ds. europejskich konrad szymański oświadczył, że projekt budżetu na lata 2021-2027, który zaoponowała ke, "nie przejdzie". to dlatego, że propozycje komisji w sprawie realizacji nowych celów odbiegają od jej pierwotnych deklaracji. szymański przyleciał do brukseli na spotkanie ministrów ds. europejskich krajów ue. rozmowy mają dotyczyć procedury art. 7 prowadzonej wobec polski, przyszłego unijnego budżetu na lata 2021-2027 i brexitu. - polska zwraca uwagę, że propozycje komisji (dot. przyszłego budżetu - pap) odbiegają od pierwotnych deklaracji samej ke. komisja na początku roku, przy polskim poparciu, deklarowała, że nowe cele będą realizowane głównie przez poszukiwanie nowych pieniędzy. w 80 proc. miały być zrealizowane przez nowe pieniądze. tak się nie stało. to jest powód, dla którego mogę wszystkich zapewnić, że budżet wieloletni ue w tym kształcie nie przejdzie - powiedział. dodał, że polska nie jest jedynym krajem, który ma krytyczne stanowisko wobec tej propozycji. - z całą pewnością budżet ue na końcu tej drogi, bardzo trudnej - myślę, ze historycznie to są najtrudniejsze negocjacje - będzie wyglądał inaczej - zaznaczył. szymański poruszył też kwestę brexitu. jak powiedział, jest to sprawa krytyczna z uwagi na czas. - wchodzimy w najbardziej kluczowy moment negocjacji w sprawie brexitu. polska zwraca uwagę na jedno. z jednej strony musimy zabezpieczyć równe warunki konkurowania na wspólnym rynku. nie możemy doprowadzić do tego, że wielka brytania będzie miała jakiś uprzywilejowany status na wspólnym rynku. z drugiej strony musimy zrobić wszystko, aby porozumienie o wyjściu wielkiej brytanii mogło być poparte w brytyjskim parlamencie i w parlamencie europejskim - powiedział. dodał, że bez tego porozumienia wszystkie osiągniecia dotychczasowych negocjacji - w zakresie praw obywateli, jak i rozliczeń finansowych - mogą okazać się nieaktualne. - to byłaby klęska nie tylko wielkiej brytanii, ale również unii europejskiej. będziemy apelowali, aby szukać rozwiązań kreatywnych, szczególnie kreatywnych politycznie. biorąc pod uwagę cykl polityczny wielkiej brytanii, myślę, że ue jest w stanie znaleźć porozumienie w tej sprawie. takie porozumienie, by ten proces negocjacyjny zakończył się sukcesem również po brytyjskiej stronie - podsumował. długoterminowy budżet na lata 2021-2027 będzie pierwszym dla unii europejskiej po wyjściu wielkiej brytanii, a więc zmniejszonej do 27 państw. z tego też powodu w brukseli i stolicach europejskich od miesięcy mówiło się o dziurze po brexicie i konieczności szukania oszczędności. plan przedstawiony 2 maja przez komisarza ds. budżetu guenthera oettingera przewiduje, że w niektórych obszarach wspólnota rzeczywiście wyda mniej, ale nie brak też takich, w których wydatki poszybują w górę. polityka spójności, której polska jest obecnie największym beneficjentem, a także wspólna polityka rolna dalej będą największymi częściami unijnej kasy. wspólnie pochłoną około 60 proc. wydatków. z oficjalnych wypowiedzi przedstawicieli komisji europejskiej wynika, że cięcie w polityce spójności ma wynieść 7 proc., natomiast w rolnictwie około 5 proc. & 434 & very low & Low & Power & NA & NA & 2018-09-18 & 2018 & 3 & POL
Frame & v.low & National & <500 & 1.0448136 & 1.8253861 & -0.2867631 & -1.2581601 & -1.2725643 & 0.0 & -1.1973642 & 0.9927066 & Recipient & Domestic & Domestic & Domestic & Domestic|POL & Neutral\\
Poland & http://tvn24bis.pl/ze-swiata,75/polska-gospodarka-przyspieszy-o-4-proc-dzieki-funduszom-unijnym,633261.html & 882 & TVN24 BiS & Private/Non-Public & Online and Offline & National & high = CP is most important issue in story (can also cover other issues) & Economic development & Positive & EU & No myth & Jobs & Positive & EU & No myth & NA & NA & NA & NA & Poland & polska gospodarka przyspieszy o 4 proc. dzięki funduszom unijnym | ze świata & 2016-04-06 & polityka spójności & w 2020 r. polska będzie się rozwijać w tempie o ponad 4 proc. szybszym za sprawą funduszy unijnych z polityki spójności - wynika z szacunków ke, na które powołują się europejscy samorządowcy. w swojej analizie europejski komitet regionów (kr), unijny organ doradczy złożony z samorządów z państw członkowskich ue, przywołał dane z ostatniego raportu komisji europejskiej na temat spójności gospodarczej, społecznej i terytorialnej. w dokumencie ke napisano, że polityka spójności w latach 2007-2013 "miała duży wkład w budowę wzrostu gospodarczego i tworzenie nowych miejsc pracy". jak zaznaczono w raporcie, szacuje się, że polityka spójności przyczyniła się do zwiększenia poziomu pkb w polsce o średnio 1,7 proc. rocznie w stosunku do poziomu, który nasz kraj osiągnąłby, gdyby nie przedsięwzięcia finansowane z jej środków. dane ke pokazują również, że spowodowała zwiększenie poziomu zatrudnienia w polsce o 1 proc. rocznie. jak podkreślono w dokumencie, szacunki dotyczące skutków długoterminowych są jeszcze wyższe. ke ocenia, że pkb w polsce w 2020 r. wyniesie o 4 proc. więcej niż poziom, który osiągnięto by bez unijnych inwestycji. prezydent gdańska paweł adamowicz zaznaczył, że w interesie polski jest, by środki unijne w kolejnych dekadach nadal trafiały do obszarów, które w poziomie rozwoju gospodarczego nie osiągają średniej unijnej. adamowicz jest jednym z polskich przedstawicieli w komisji polityki spójności terytorialnej i budżetu ue komitetu regionów. - jeżeli chcemy, żeby podatnicy europejscy nas wspierali, to z drugiej strony musimy się czuć bardziej współodpowiedzialni za europę - powiedział. jak uznali uczestnicy debaty na temat przyszłości polityki spójności zorganizowanej przez komitet regionów, ma ona zasadnicze znaczenie dla ue, ale potrzebuje reform. w dyskusji wziął udział prezydent białegostoku tadeusz truskolaski, który w kr przewodzi grupie regionów słabiej rozwiniętych, których poziom pkb na mieszkańca jest poniżej 75 proc. średniego poziomu unijnego. jak zaznaczył truskolaski, w polsce dosyć powszechna jest opinia, że środki z polityki spójności z lat 2014-2020 to ostatnie duże pieniądze, tymczasem w trakcie debaty żadne tego typu stwierdzenie nie padło. - właściwie wszyscy mówili o kontunuowaniu polityki spójności, co jest bardzo pocieszające - powiedział. w ocenie prezydenta białegostoku kolejna perspektywa finansowa rzeczywiście może być nieco mniejsza, ale nadal będą to spore środki. sytuację może zrewolucjonizować - według truskolaskiego - wyjście z ue wielkiej brytanii. - wszytko to, o czym dzisiaj rozmawiamy, stałoby się praktycznie nieaktualne - zaznaczył. według szefa komitetu regionów markku markkuli polityka spójności to najważniejszy instrument, przy pomocy którego ue jest w stanie przezwyciężyć różnice w rozwoju regionów. - nie jest idealna, mimo to gra wiodącą rolę w rozwoju w unii europejskiej - uważa markkula. uczestnicząca w konferencji czeska minister rozwoju regionalnego karla slechtova nawiązała do podpisanego w styczniu w pradze wspólnego oświadczenia państw grupy wyszehradzkiej. w dokumencie podkreślono, że polityka spójności jest kluczową polityką inwestycyjną w ue, i że musi to uwzględniać również budżet unijny po 2020 r. w ocenie fabiana zuleega z think-tanku centrum polityki europejskiej polityka spójności potrzebuje radykalnych zmian. jak powiedział, spodziewa się presji, aby zmniejszyć budżet unijny po 2020 r. z kolei zdaniem szefa gabinetu unijnej komisarz ds. polityki regionalnej coriny cretu nicola de michelisa niedawne wydarzenia pokazały, że budżet unii europejskiej nie jest wystarczająco elastyczny, żeby reagować na nowe wyzwania. - mamy dwie opcje, albo jesteśmy w stanie stworzyć odpowiednią dozę elastyczności w polityce spójności, być może rezygnując z części przewidywalności na rzecz elastyczności, albo ta elastyczność zostanie wygenerowana na zewnątrz polityki spójności, ale kosztem jej budżetu - uznał de michelis. & 547 & high & High & Socio-Economic & Socio-Economic & NA & 2016-04-06 & 2016 & 2 & ECO
Frame & high-very high & National & 500-1000 & 1.0448136 & 1.8253861 & -0.2867631 & -1.2581601 & -1.2725643 & 0.0 & -1.1973642 & 0.9927066 & Recipient & European & European & European & European|ECO & Positive\\
Poland & http://forsal.pl/artykuly/933900,optymistyczne-prognozy-ke-polska-gospodarka-przyspieszy-o-4-proc-w-2020-roku.html & 782 & forsal.pl & Private/Non-Public & Online only & National & high = CP is most important issue in story (can also cover other issues) & Economic development & Positive & National & No myth & NA & NA & NA & NA & NA & NA & NA & NA & Poland & optymistyczne prognozy ke. polska gospodarka przyspieszy o 4 proc. w 2020 roku & 2016-04-06 & polityka spójności & w 2020 r. polska będzie się rozwijać w tempie o ponad 4 proc. szybszym za sprawą funduszy unijnych z polityki spójności - wynika z szacunków ke, na które powołują się europejscy samorządowcy. w swojej analizie europejski komitet regionów (kr), unijny organ doradczy złożony z samorządów z państw członkowskich ue, przywołał dane z ostatniego raportu komisji europejskiej na temat spójności gospodarczej, społecznej i terytorialnej. w dokumencie ke napisano, że polityka spójności w latach 2007-2013 "miała duży wkład w budowę wzrostu gospodarczego i tworzenie nowych miejsc pracy". jak zaznaczono w raporcie, szacuje się, że polityka spójności przyczyniła się do zwiększenia poziomu pkb w polsce o średnio 1,7 proc. rocznie w stosunku do poziomu, który nasz kraj osiągnąłby, gdyby nie przedsięwzięcia finansowane z jej środków. dane ke pokazują również, że spowodowała zwiększenie poziomu zatrudnienia w polsce o 1 proc. rocznie. jak podkreślono w dokumencie, szacunki dotyczące skutków długoterminowych są jeszcze wyższe. ke ocenia, że pkb w polsce w 2020 r. wyniesie o 4 proc. więcej niż poziom, który osiągnięto by bez unijnych inwestycji. >>> czytaj też: polska gospodarka zwolni czy przyśpieszy? oecd rozwiało wątpliwości prezydent gdańska paweł adamowicz zaznaczył w rozmowie z serwisem samorządowym pap, że w interesie polski jest, by środki unijne w kolejnych dekadach nadal trafiały do obszarów, które w poziomie rozwoju gospodarczego nie osiągają średniej unijnej. adamowicz jest jednym z polskich przedstawicieli w komisji polityki spójności terytorialnej i budżetu ue komitetu regionów. jak uznali uczestnicy debaty na temat przyszłości polityki spójności zorganizowanej przez komitet regionów, ma ona zasadnicze znaczenie dla ue, ale potrzebuje reform. w dyskusji wziął udział prezydent białegostoku tadeusz truskolaski, który w kr przewodzi grupie regionów słabiej rozwiniętych, których poziom pkb na mieszkańca jest poniżej 75 proc. średniego poziomu unijnego. jak zaznaczył truskolaski, w polsce dosyć powszechna jest opinia, że środki z polityki spójności z lat 2014-2020 to ostatnie duże pieniądze, tymczasem w trakcie debaty żadne tego typu stwierdzenie nie padło. - właściwie wszyscy mówili o kontunuowaniu polityki spójności, co jest bardzo pocieszające - powiedział. w ocenie prezydenta białegostoku kolejna perspektywa finansowa rzeczywiście może być nieco mniejsza, ale nadal będą to spore środki. sytuację może zrewolucjonizować - według truskolaskiego - wyjście z ue wielkiej brytanii. - wszystko to, o czym dzisiaj rozmawiamy, stałoby się praktycznie nieaktualne - zaznaczył. według szefa komitetu regionów markku markkuli polityka spójności to najważniejszy instrument, przy pomocy którego ue jest w stanie przezwyciężyć różnice w rozwoju regionów. - nie jest idealna, mimo to gra wiodącą rolę w rozwoju w unii europejskiej - uważa markkula. uczestnicząca w konferencji czeska minister rozwoju regionalnego karla slechtova nawiązała do podpisanego w styczniu w pradze wspólnego oświadczenia państw grupy wyszehradzkiej. w dokumencie podkreślono, że polityka spójności jest kluczową polityką inwestycyjną w ue, i że musi to uwzględniać również budżet unijny po 2020 r. w ocenie fabiana zuleega z think-tanku centrum polityki europejskiej polityka spójności potrzebuje radykalnych zmian. jak powiedział, spodziewa się presji, aby zmniejszyć budżet unijny po 2020 r. z kolei zdaniem szefa gabinetu unijnej komisarz ds. polityki regionalnej coriny cretu nicola de michelisa niedawne wydarzenia pokazały, że budżet unii europejskiej nie jest wystarczająco elastyczny, żeby reagować na nowe wyzwania. - mamy dwie opcje, albo jesteśmy w stanie stworzyć odpowiednią dozę elastyczności w polityce spójności, być może rezygnując z części przewidywalności na rzecz elastyczności, albo ta elastyczność zostanie wygenerowana na zewnątrz polityki spójności, ale kosztem jej budżetu - uznał de michelis. >>> czytaj też: jak będzie wyglądać polska gospodarka pod koniec roku? oto prognozy ekonomistów & 557 & high & High & Socio-Economic & NA & NA & 2016-04-06 & 2016 & 2 & ECO
Frame & high-very high & National & 500-1000 & 1.0448136 & 1.8253861 & -0.2867631 & -1.2581601 & -1.2725643 & 0.0 & -1.1973642 & 0.9927066 & Recipient & Domestic & Domestic & Domestic & Domestic|ECO & Positive\\
Poland & https://plus.gloswielkopolski.pl/budzet-unii-europejskiej-szansa-na-rozwoj-innowacji-i-ochrone-srodowiska-w-wielkopolsce/ar/13736798 & a18 & plus.gloswielkopolski.pl & Private/Non-Public & Online and Offline & Regional/Local & medium = CP is important part of story & Economic development & Positive & EU + National + Subnational & No myth & Environment/green/low-carbon & Positive & EU + National + Subnational & No myth & Research \& innovation & Positive & EU + National + Subnational & No myth & Poland & budżet unii europejskiej szansą na rozwój innowacji i ochronę środowiska w wielkopolsce & 2018-12-12 & polityka spójności & przybliżenie mieszkańcom wielkopolski tematyki związanej z kolejnym budżetem unii europejskiej, zwłaszcza w kontekście szans dla polski i wielkopolski - to cel debaty, która odbyła się w głosie wielkopolskim. podczas spotkania rozmawiano m. in. o priorytetach nowego budżetu ue, innowacjach w naszym regionie, a także rolnictwie i inwestycjach, na które nasze województwo będzie potrzebowało wsparcia ue. "budżet unii europejskiej - perspektywa 2021 - 2027. szanse dla wielkopolski: od projektów innowacyjnych po politykę rolną" - to temat debaty, która odbyła się we wtorek w redakcji głosu wielkopolskiego. uczestnikami dyskusji zorganizowanej przez polska press grupę i komisję europejską byli: prof. ida musiałkowska z uniwersytetu ekonomicznego w poznaniu, specjalizująca się, m. in. w procesach integracyjnych w unii europejskiej, stanisław kalemba, były minister rolnictwa i rozwoju wsi z psl, uczestnik negocjacji dotyczących poprzedniego budżetu ue, bartłomiej wróblewski, poseł pis, przewodniczący parlamentarnej grupy współpracy polsko-niemieckiej, grzegorz potrzebowski, dyrektor departamentu polityki regionalnej urzędu marszałkowskiego województwa wielkopolskiego, odpowiedzialny m. in. za podział unijnych funduszy w wielkopolsce oraz piotr świtalski z przedstawicielstwa komisji europejskiej w polsce. brexit, inwestycje infrastrukturalne głos wielkopolski: nasza rozmowa miała rozpocząć się od przyszłego budżetu unii europejskiej, ale tak naprawdę należy ją rozpocząć od brexitu, który na projekt tego budżetu wpłynął. przecież nadal nie wiadomo, jak wyjście wielkiej brytanii z ue będzie wyglądało. w ostatnich dniach theresa may przełożyła głosowanie nad projektem umowy dotyczącej brexitu, w czwartek zaś rozpocznie się unijny szczyt poświęcony tej tematyce. premier may zapowiedziała też próbę dalszych dyskusji ze wspólnotą. czy do brexitu w ogóle dojdzie? ida musiałkowska: myślę, że wszyscy chcieliby się tego dowiedzieć. natomiast ostatnie odwołanie głosowania w parlamencie brytyjskim wskazuje, że może dojść do drugiego referendum. gdyby faktycznie doszło do brexitu, na pewno odbije się to na budżecie ue, gdyż oznaczałoby to mniejsze kwoty, które do niego wpływają. głos wielkopolski: czy projektowany budżet na lata 2021-2017 uwzględnia ewentualny brexit? piotr świtalski: budżet zaproponowany w maju tego roku przez komisję europejską uwzględniał tzw. lukę brexitową. miała być ona uzupełniona z jednej strony poprzez zwiększenie wkładów państw członkowskich, a z drugiej strony poprzez pewne oszczędności. bardzo ciekawe, co wydarzy się podczas czwartkowego i piątkowego szczytu w brukseli, gdzie przywódcy państw członkowskich będą debatować nie tylko na temat brexitu, ale także przyszłego budżetu. zobaczymy, jak podejdą do tej sytuacji. stanisław kalemba: według wszystkich danych projekt budżetu jest wyższy niż w poprzedniej perspektywie, nawet mimo brexitu. i to trzeba podkreślić. powszechnie podaje się, że cięcia są przyczyną brexitu, ale jak patrzy się na wydatki, to ten projekt zakłada, że będą one wyższe niż w poprzedniej perspektywie. bartłomiej wróblewski: w tym przypadku dochodzimy nie tylko do kwestii tego czy wielka brytania będzie płatnikiem w kolejnej perspektywie, ale tego, jaka będzie struktura budżetu. to jest kwestia kluczowa. trzeba zakładać, że do brexitu dojdzie, co jest złą wiadomością. powinniśmy się jednak koncentrować przede wszystkim na strukturze wydatków, mniej na strukturze wpływów. p.ś.: propozycja komisji europejskiej zakłada zwiększenie budżetu w stosunku do pkb państw członkowskich, a europosłowie również się za tym opowiadają. jednak ostateczna decyzja należy do przywódców państw członkowskich. s.k.: brałem udział w negocjacjach dotyczących aktualnego budżetu ue i pamiętam, że sukcesem było to, że on w ogóle został uchwalony na czas. unii europejskiej groziło prowizorium budżetowe, a co to oznacza, to chyba dobrze wiemy... to byłoby najbardziej niebezpieczne dla takich państw jak polska. już wtedy były poważne sygnały dotyczące ewentualnego zmniejszenia wpływów do unijnego budżetu. p.ś.: odnosząc się jeszcze do poprzedniego budżetu pragnę przypomnieć, że o ile ilość pieniędzy na politykę spójności uległa wtedy zmniejszeniu, o tyle polska otrzymała największe dofinansowanie z tego tytułu, z czego możemy być dumni. w nowym budżecie ue wielkość środków w tym zakresie będzie mniejsza. b.w.: to nie jest jednak jeszcze ostateczna propozycja budżetu. bez wątpienia w interesie polski bardzo ważną rzeczą jest, by przekonywać państwa zachodnioeuropejskiego do kontynuowania polityki spójności i wsparcia infrastrukturalnego. od tego zależy np. droga s10, czy s11 w wielkopolsce, przyszłość poznańskiej kolei metropolitalnej, budowa trzeciego toru między stacjami poznań wschód - poznań główny czy uruchomienie nieczynnych linii kolejowych. to zależy od tego jak duży będzie budżet infrastrukturalny w skali europejskiej i ile pieniędzy trafi do polski. grzegorz potrzebowski: niestety, propozycja komisji europejskiej wskazuje, że na politykę spójności w nowej perspektywie jest zaplanowane ok. 10 procent mniej pieniędzy niż w budżecie na lata 2014-2020. polityka spójności, innowacyjność głos wielkopolski: co byłoby sukcesem w nowym unijnym budżecie dla naszego kraju? b.w.: na pewno utrzymanie albo zbliżenie się do tego poziomu wydatków na politykę spójności, jaki jest obecnie. a drugim bardzo ważnym dla nas elementem jest wsparcie polskiego rolnictwa. podobnie jak w przypadku polityki spójności mamy ten sam problem - kilkadziesiąt lat, przez które polskie rolnictwo było zamknięte, nie mogliśmy uczestniczyć w rynku europejskim, rozwijać się i unowocześniać gospodarstw. to spowodowało opóźnienie. potrzebujemy jeszcze czasu na nadgonienie tego okresu. dlatego istotne jest, by wsparcie dla polskiego rolnictwa, poprzez dopłaty, było duże. to nie oznacza jednak, że nie rozumiemy konieczności wspomagania innowacyjnej gospodarki. ale trzeba też szukać rozwiązań, które pogodzą konieczność wyrównywania poziomu rozwoju europy środkowo-wschodniej i europy zachodniej z oczekiwaniem zwiększenia innowacyjności gospodarki. g.p.: mimo że polska otrzymała w obecnym budżecie największą kwotę pieniędzy, musimy wziąć pod uwagę, że pod względem produktu krajowego brutto na mieszkańca należymy do najmniej zamożnych państw w ue. 19,5 mld euro mniej, które mamy otrzymać w przyszłej perspektywie, to też wskazówka, że duże pieniądze w perspektywie na lata 2021-2027 pójdą przede wszystkim do europy południowej jak np. grecja, hiszpania, portugalia. s.k.: odnośnie zadbania o największe środki z polityki spójności - musimy pamiętać, że chodzi o interes polski. b.w.: to bardzo ważne stwierdzenie. czuję się w obowiązku, aby wspomnieć o tym, żeby w momencie, kiedy negocjacje wkraczają w decydującą fazę, myśleć o interesie narodowym, niezależnie od tego, kto rządzi. mimo że na co dzień jest sporo emocji, jest to jeden z tych momentów, kiedy dobrze byłoby, aby we wszystkich głównych ugrupowaniach politycznych w polsce nastąpiła refleksja i wsparcie rządu. nie wiemy jak potoczą się wybory w kolejnych latach, ale wiemy, że tych pieniędzy polska potrzebuje. niezależnie od poglądów, każdy w wielkopolsce uważa, że np. s10 czy s11 powinny być zrobione. s.k.: ta refleksja musi mieć miejsce u wszystkich, ale wcześniej powinna mieć miejsce w rządzie. p.ś.: chciałbym tylko dodać, że w tej chwili mówimy o funduszach spójności, czyli tzw. bezzwrotnych pieniądzach. a trzeba też wspomnieć, że filozofia przyszłego budżetu przewiduje zwiększenie funduszy zwrotnych, czyli pieniędzy inwestycyjnych. chodzi o tzw. plan junckera. polska jest jednym z państw członkowskich, które najlepiej wykorzystują pieniądze w ramach niego. o ile pieniądze w polityce spójności zostaną zmniejszone, o tyle plan junckera to niesamowita możliwość dla naszego regionu, aby otrzymać fundusze zwrotne i z nich skorzystać. komisja europejska zwraca też uwagę na aspekty środowiskowe. podkreśla, że fundusze unijne powinny też mieć wpływ na ochronę środowiska. jeszcze innym elementem, na który warto zwrócić uwagę, jest współpraca międzyregionalna. wielkopolska z sąsiadującymi regionami, mogą zastanowić się nad projektami, które mogłyby być realizowane i współfinansowane. g.p.: wielkopolska, razem z czterema innymi województwami (zachodniopomorskie, lubuskie, dolnośląskie i opolskie), mają wspólną strategię polski zachodniej, która zakłada listę projektów kluczowych dla tej części kraju. jednak żaden z nich na razie nie znalazł możliwości finansowania ze środków ue i programów krajowych. b.w.: tak jak będziemy bronili naszych racji dotyczących polityki wspólności i polityki rolnej, tak też widzę szanse związane z kierunkiem myślenia innowacyjnego o gospodarce. to także szansa, niezależnie od tego jaki procent pieniędzy zostanie przeznaczony na wspieranie innowacyjności przedsiębiorstw. gdzie jak gdzie, ale w wielkopolsce powinniśmy włożyć szczególny w wysiłek w to, by jak najlepiej wykorzystać te środki. to może być trampoliną dla wielkopolskich przedsiębiorstw, by wejść do światowej pierwszej ligi lub umocnić się w rywalizacji międzynarodowej. nie chodzi jednak tylko o to, by były pieniądze na zakup technologii z zachodu. to też okazja, by promować naszą myśl techniczną. ważne jest jednak, żeby pieniądze były wydawane też na badania naukowe na uniwersytetach w poznaniu. p.ś.: pamiętajmy też, o ile polska jest prymusem w zakresie wykorzystywania pieniędzy w ramach polityki spójności, tak w przypadku innowacyjności jesteśmy na ostatnich miejscach w rankingach europejskich. z kolei budżet zaproponowany przez komisję europejską jest szansą na polepszenie naszej sytuacji. s.k.: to tutaj jest jeden z najmocniejszych regionów pod względem naukowym, kulturowym, badawczym, czy ekonomicznym. jednak większe wysiłki powinny być poświęcone badaniom nad efektem cieplarnianym czy smogiem w polsce. i.m.: rzeczywiście wskaźniki dotyczące innowacyjności regionu nie są wysokie i na to należy zwrócić uwagę w kontekście szansy dla wielkopolski. warto też pamiętać o tym, że komisja europejska zaproponowała elastyczne przenoszenie środków między różnymi programami. pamiętajmy też, że wielkopolska ma dobre doświadczenia w wykorzystywaniu pieniędzy z programów jeremie i jessica. byliśmy jednym z pierwszych regionów, które wprowadziły te rozwiązania. pieniądze z tych programów pozwoliły nam na utworzenie funduszu rozwoju, który z kolei może pozwolić na uzupełnienie środków obciętych na politykę spójności w kolejnej perspektywie. inteligentna i bezemisyjna europa głos wielkopolski: jaki będzie priorytet dla wielkopolski w najbliższym rozdaniu unijnym? g.p.: priorytety określa komisja europejska. ona daje pieniądze, ale pod pewnym warunkiem, gdyż wyznacza określone poziomy kwot, które trzeba przeznaczyć na konkretne obszary. nowa perspektywa nie różni się znacząco od obecnej pod względem wyznaczonych celów, lecz ich ważności. w przyszłej perspektywie ponad 35 procent ze środków unijnych musimy przeznaczyć na pierwszy cel, czyli tzw. bardziej inteligentną europę. to jest punkt zakładający innowacje, cyfryzacje, wsparcie małych i średnich przedsiębiorstw. kolejne 30 procent pieniędzy trzeba będzie przeznaczyć na tzw. bardziej przyjazną dla środowiska i bezemisyjną europę, czyli działania proekologiczne, inwestycje w odnawialne źródła energii czy transformację sektora energetycznego. subregiony w wielkopolsce, mobilność głos wielkopolski: poznań jest przygotowany na rozwój i innowacyjność. tutaj mamy uczelnie, centra badawcze, ale są też regiony nieprzygotowane do takich działań jak np. północ województwa. s.k.: panie redaktorze, wpadł pan w moją myśl. właśnie w tym temacie chciałem się wypowiedzieć. obracamy się w temacie poznania i wielkich aglomeracji, a ja uważam, że w najbliższej perspektywie należy bardziej przyjrzeć się terenom poza dużymi miastami: gminom i powiatom. mówimy o wielkich dokonaniach, które mają miejsce, ale jeszcze są takie enklawy, gdzie nie ma inwestycji i są problemy z pracą. powinniśmy przesunąć działania na tamte tereny, bo poznań, kalisz, konin, piła czy leszno radzą sobie całkiem dobrze. ale są obszary biedy i niedoinwestowania, mimo że w wielkopolsce generalnie jest zrównoważony rozwój. b.w.: to nie jest do końca tak, że na obrzeżach wielkopolski jest gorzej, a w centrum lepiej. z jednej strony mamy złotów, ale z drugiej kępno i powiat kępiński. są takie powiaty, którym powodzi się całkiem dobrze. z kolei w samym poznaniu mamy problem z depopulacją miasta i rozwoju powiatu, który za kilkanaście lat będzie większy od poznania. jeśli nie znajdziemy odpowiedniego rozwiązania infrastrukturalnego w kwestii transportu, to za kilkanaście lat zakorkuje nam się centrum wielkopolski. p.ś.: w trakcie tej debaty padło wiele haseł i pojawiły się różne tematy, co prowadzi mnie do jednej konkluzji, która jest też zawarta w propozycji komisji europejskiej. mam na myśli to, że przyszły budżet ma być elastyczny, czyli ma reagować na aktualne problemy, zagrożenia czy szanse. przy polityce rolnej założenie jest takie, że państwa mogą zmieniać nacisk na dopłaty lub inwestycje w rozwój obszarów wiejskich. przyszły budżet jest zdecydowanie bardziej elastyczny w porównaniu do obecnego. i.m.: któryś z panów wspomniał o rynku pracy, pojawiła się też kwestia depopulacji, nie zapominajmy, że zmagamy się też z brakiem rąk do pracy i to są problemy poznania. już w obecnej perspektywie finansowej możemy mieć problemy z terminowym zakończeniem projektów ze względu na brak pracowników. firmy będą potrzebowały rąk do pracy i to jest istotne w zakresie polityki komisji europejskiej i tego, jakie instrumenty można zapewnić. nie zapominajmy też, że w nowym budżecie są pieniądze na mobilność. jeśli są obszary, w których ludzie szukają pracy, a gdzieś ona jest, to trzeba im zapewnić odpowiednią komunikację, by chcieli ją podjąć. w wielkopolsce już teraz kilka firm zapewnia dojazd dla pracowników z innych regionów. p.ś.: warto też wspomnieć, że komisja europejska oraz przyszły budżet będą wspierały poszczególne państwa w przechodzeniu na gospodarkę niskoemisyjną. to jest jeden z priorytetów tego budżetu. b.w.: tam gdzie można, bez wątpienia trzeba inwestować w odnawialne źródła energii. ponadto środki europejskie byłyby szansą na to, żeby próbować rozwiązywać problemy, które narastają w naszym regionie np. dotyczące komunikacji. g.p.: wspominane odnawialne źródła energii nie są niczym nowym. w ramach wielkopolskiego regionalnego programu operacyjnego wsparliśmy bardzo wiele takich projektów. natomiast jest jeszcze jedna istotna kwestia dotycząca nowej perspektywy finansowej. mimo, że ten budżet też będzie siedmioletni, to będziemy mieli mniej czasu na zrealizowanie projektów. obecnie obowiązuje zasada, zgodnie z którą po zakończeniu perspektywy budżetowej są jeszcze trzy lata na realizację projektów. w przypadku przyszłego budżetu pojawiła się propozycja, by wszelkie projekty były zakończone dwa lata po zamknięciu perspektywy finansowej. to może stanowić problem z wdrażaniem dużych inwestycji. s.k.: dzisiaj widać, że to, co udało się wywalczyć na lata 2014-2020 było wielkim sukcesem. wtedy wydawało się to mało, a teraz mamy cel taki, żeby zbliżyć się do tamtej kwoty. i.m.: jeśli chodzi o negocjacje, musimy wykorzystać wszystkie kontakty, które mamy i przekazywać merytoryczne argumenty oparte o to, co udało nam się do tej pory wypracować. przyszły budżet jest bardziej ukierunkowany na ludzi, gdyż dotyka spraw im bliskich jak czyste powietrze, kształcenie czy wspólnoty lokalne. ten budżet skupi się na innych aspektach, które do tej pory nie były podkreślane. p.ś.: komisja europejska chciałaby, aby dyskusja o budżecie ue zakończyła się przed wyborami do parlamentu europejskiego - wiosną 2019 roku. ale taka propozycja padła w maju, a od tego czasu wydarzyły się już różne rzeczy. chcielibyśmy, żeby budżet został uchwalony w przyszłym roku i to jest data docelowa. & 2247 & medium & Medium & Socio-Economic & Socio-Economic & Socio-Economic & 2018-12-12 & 2018 & 3 & ECO
Frame & low-medium & Regional & +1000 & 1.0448136 & 1.8253861 & -0.2867631 & -1.2581601 & -1.2725643 & 0.0 & -1.1973642 & 0.9927066 & Recipient & Domestic & European & Mixed & Domestic|ECO & Positive\\
\addlinespace
Poland & http://www.euractiv.pl/section/60-lat-europy/news/60-lat-europy-jadwiga-wisniewska-europoslanka/ & 81 & EurActiv.pl & Private/Non-Public & Online only & National & medium = CP is important part of story & Political capital/interests & Negative & EU & No myth & NA & NA & NA & NA & NA & NA & NA & NA & Poland & 60 lat europy: jadwiga wiśniewska, europosłanka & 2017-03-24 & fundusze strukturalne & głębię kryzysu integracji najlepiej pokazał ostatni szczyt rady europejskiej, a zwłaszcza późniejsza wypowiedź prezydenta francji, który kierowane do polski fundusze strukturalne uznał za rodzaj jałmużny od bogatego zachodu - twierdzi posłanka do pe jadwiga wiśniewska. rozmawiamy z europosłanką w ramach naszego projektu na 60. rocznicę podpisania traktatów rzymskich. czytaj więcej o naszym projekcie rocznicowym tutaj. bez wątpienia zawdzięczamy jej dostęp do wspólnego rynku i jego czterech wolności: swobodnego przepływu towarów, osób, usług i kapitału, które gwarantują równe traktowanie polaków i polskich produktów na terenie całej ue. wspólny rynek jest fundamentem europejskiej współpracy i zapewnienie jego stabilnego rozwoju stanowi stały priorytet polskiego rządu. niestety, obsesja ue na punkcie równości już dawno temu doprowadziła do tego, że w imię niedyskryminacji ue zaprzeczyła chrześcijańskim wartościom, jakie wyznawali jej założyciele. jeśli dodamy do tego nieustanne zawłaszczanie kompetencji państw członkowskich przez unijne instytucje, widzimy, że ogromna część unijnych "zasług" to sianie prawnego i ideowego zamętu. czy ue wymaga przemian? a jeśli tak - dlaczego i jakich - a także, co owe zmiany mają nam przynieść? zdecydowanie tak. potrzebę zmian zauważa nawet komisja europejska. w opublikowanej na początku marca białej księdze wskazała, że źródłem problemów ue jest utrata społecznego zaufania do unijnych działań oraz brak demokratycznej legitymacji. głębię kryzysu integracji najlepiej pokazał ostatni szczyt rady europejskiej, a zwłaszcza późniejsza wypowiedź prezydenta francji, który kierowane do polski fundusze strukturalne uznał za rodzaj jałmużny od bogatego zachodu. ten skandaliczny stereotyp, który kraje spoza tzw. starej ue uznaje za peryferie, razem z propozycją stworzenia europy wielu prędkości to dziś podstawowy błąd w myśleniu o unijnych reformach. powinny one zmierzać to wzmocnienia pozycji wszystkich państw członkowskich, a nie wąskiego grona wybrańców. & 268 & medium & Medium & Power & NA & NA & 2017-03-24 & 2017 & 2 & POL
Frame & low-medium & National & <500 & 1.0448136 & 1.8253861 & -0.2867631 & -1.2581601 & -1.2725643 & 0.0 & -1.1973642 & 0.9927066 & Recipient & European & European & European & European|POL & Negative\\
Poland & http://forsal.pl/artykuly/1413593,polska-czechy-slowacja-i-wegry-zostana-polaczone-koleja-wysokich-predkosci.html & 122 & forsal.pl & Private/Non-Public & Online only & National & low = CP mentioned more times but NOT important part of story (mainly about others issues) & Territorial cooperation & Positive & National + Other country & No myth & NA & NA & NA & NA & NA & NA & NA & NA & Poland & polska, czechy, słowacja i węgry zostaną połączone koleją wysokich prędkości? & 2019-05-21 & polityka spójności & reprezentujący kraje grupy wyszehradzkiej ministrowie transportu bądź ich przedstawiciele podpisali we wtorek w bratysławie deklarację o połączeniu polski, czech, słowacji i węgier koleją wysokich prędkości. zapowiedzieli również walkę o utrzymanie dotychczasowego poziomu finansowania inwestycji drogowych z funduszy unijnych. minister transportu i budownictwa słowacji arpad ersek powiedział na konferencji prasowej, że trwają prace nad studiami opłacalności i technicznymi warunkami budowy nowego połączenia kolejowego z budapesztu przez bratysławę i brno do warszawy. "chodzi o trasę, po której pociągi będą jechały z prędkością do 300 kilometrów na godzinę, a do tego musi powstać nowy szybki tor kolejowy" - zaznaczył. jak poinformował ersek, celem jest stworzenie konkurencji dla sieci połączeń lotniczych. zwrócił uwagę, że podróże koleją rozpoczynają się w centach miast i nie traci się czasu na potrzebną w komunikacji powietrznej odprawę i odbiór bagaży. uczestniczący w spotkaniu minister infrastruktury andrzej adamczyk powiedział pap, że toczą się rozmowy na poziomie premierów krajów v4 i poszukiwane są źródła finansowania. dodał, że pracują zespoły robocze, a czas nie jest marnowany. podkreślił, że kolej wysokich prędkości pozwoli m.in. na połączenie portów bałtyckich z europą środkową i zapewni także komunikację z południem kontynentu. uczestniczący w spotkaniu minister innowacji węgier tamas schanda powiedział, że kraje v4 nie mogą dopuścić do obcięcia środków europejskich. "dla sektora gospodarczego i jego rozwoju niezbędny jest właśnie rozwój infrastruktury. jest rzeczą nadzwyczaj ważną, aby dotąd skuteczne narzędzia, jak polityka spójności, nadal funkcjonowały" - stwierdził. minister transportu czech vladimir kremlik zapowiedział, że dalsza realizacja szybkich połączeń kolejowych w ramach v4 będzie priorytetem czech, gdy 1 lipca przejmą przewodnictwo w grupie wyszehradzkiej. w bratysławie mówiono także o współpracy w transgranicznym odzyskiwaniu niezapłaconych przez użytkowników autostrad i dróg należności za wykorzystywanie infrastruktury drogowej. podkreślano, że problem ma charakter ogólnoeuropejski i nie dotyczy jedynie tak zwanego myta, które płacą kierowcy ciężarówek, ale także opłat, które powinni uiszczać kierowcy samochodów osobowych. kraje v4 zadeklarowały współpracę w stworzeniu systemów, które pozwolą na odzyskanie niezapłaconych kwot. "zrobimy wszystko, żeby goście zagraniczni, którzy nie regulują wszystkich płatności, musieli je uiszczać nawet jak wrócą do siebie do domu" - powiedział pap adamczyk. dodał, że polska chce przyspieszyć działanie dyrektywy unijnej przyjętej w kwietniu br., która mówi o tworzeniu punktów informacji i bazy danych pozwalających na windykację opłat na terenie całej ue, niezależnie od miejsca zamieszkania lub pobytu kierowcy. "wyjazd samochodu z polski bez uiszczenia opłaty nie oznacza, że nie będzie jej musiał uiścić później wraz karą" - powiedział adamczyk. w krajach ue funkcjonuje system transgranicznego egzekwowania mandatów za wykroczenia drogowe. nie ma natomiast wspólnego systemu pozwalającego na ściąganie należnych kwot za przejazd autostradami lub innymi płatnymi drogami. obowiązuje zasada, że zaległą opłatę i ewentualną karę można wyegzekwować jedynie podczas przeprowadzanej kontroli drogowej. poszczególne państwa ue pracują nad zmianą tego systemu i kraje v4 chcą współpracować w wymianie doświadczeń. & 452 & low & Low & Socio-Economic & NA & NA & 2019-05-21 & 2019 & 3 & ECO
Frame & low-medium & National & <500 & 1.0448136 & 1.8253861 & -0.2867631 & -1.2581601 & -1.2725643 & 0.0 & -1.1973642 & 0.9927066 & Recipient & Domestic & European & Mixed & Domestic|ECO & Positive\\
Poland & http://www.gazetaprawna.pl/artykuly/1122762,polska-i-czechy-beda-starac-sie-o-realizacje-polityki-spojnosci-ue.html & 484 & gazetaprawna.pl & Private/Non-Public & Online and Offline & National & medium = CP is important part of story & Institutional bargaining over funding & Factual & National + Other country & No myth & NA & NA & NA & NA & NA & NA & NA & NA & Poland & polska i czechy będą starać się o realizację polityki spójności ue & 2018-05-10 & polityka spójności & polska i czechy będą na forum unii europejskiej czyniły starania, aby polityka spójności mogła być dalej realizowana w jak największym stopniu; te elementy spójnych interesów - w działaniach co do unijnego budżetu - na pewno będą - oświadczył w czwartek prezydent andrzej duda. prezydenci polski i czech andrzej duda i milosz zeman byli pytani - podczas wspólnej konferencji w warszawie - czy oba kraje zjednoczą siły, aby doprowadzić do takiego kształtu unijnego budżetu, który byłby dla nich korzystny. zobacz także:prezydent: dzięki grupie wyszehradzkiej możemy się liczyć w ue " "oba nasze kraje, władze krajowe będą na forum unii europejskiej czyniły starania, aby ta polityka spójności mogła być dalej w jak największym stopniu realizowana. w tym zakresie my cały czas gonimy te bogate kraje zachodu, więc te elementy spójnych interesów - w tych działaniach co do unijnego budżetu - na pewno będą" - odpowiedział andrzej duda. "zgadzamy się z prezydentem (czech) niewątpliwie, co do jednego, że patrząc na przyszłe budżety europejskie i patrząc na przyszłe ramy finansowe - które zostaną przyjęte na kolejny okres - to niestety nie można abstrahować od brexitu i od tego, jakie będą obiektywne skutki brexitu w odniesieniu do tych kwestii" - wskazał duda. & 186 & medium & Medium & Power & NA & NA & 2018-05-10 & 2018 & 3 & POL
Frame & low-medium & National & <500 & 1.0448136 & 1.8253861 & -0.2867631 & -1.2581601 & -1.2725643 & 0.0 & -1.1973642 & 0.9927066 & Recipient & Domestic & European & Mixed & Domestic|POL & Neutral\\
Poland & https://dziennikzachodni.pl/minister-inwestycji-i-rozwoju-jerzy-kwiecinski-z-wizyta-na-slasku-trwaja-konsultacje-nad-projektem-rozwoju-regionu/ar/13818761\#strefa-biznesu & a21 & Dziennikzachodni.pl & Private/Non-Public & Online and Offline & Regional/Local & high = CP is most important issue in story (can also cover other issues) & Economic development & Factual & National + Subnational & No myth & NA & NA & NA & NA & NA & NA & NA & NA & Poland & minister inwestycji i rozwoju jerzy kwieciński z wizytą na śląsku. trwają konsultacje nad projektem rozwoju regionu & 2019-01-18 & polityka regionalna & jesteśmy na półmetku konsultacji projektu krajowej strategii rozwoju regionalnego 2030. to dokument, który ma być drogowskazem dla osób, które zajmują się rozwojem regionalnym, także w śląskiem - mówi jerzy kwieciński, minister inwestycji i rozwoju, który pojawił się dziś (18 stycznia) w katowicach, by rozmawiać z samorządowcami na temat rozwoju w miastach woj. śląskiego. dzisiaj, 18 stycznia, do katowic przyjechał minister rozwoju i inwestycji jerzy kwieciński, by spotkać się z samorządowcami i rozmawiać z nimi na temat rozwoju miast województwa śląskiego. a to wszystko w ramach projektu krajowej strategii rozwoju regionalnego 2030, którego trwające konsultacje - jak przekonywał minister - są już na półmetku. zobacz galerię - to dokumenty strategiczny, który ma być drogowskazem dla osób, które zajmują się rozwojem regionalnym, szczególnie dla samorządów wojewódzkich, ale i powiatowych - mówił minister kwieciński podczas dzisiejszego spotkania z dziennikarzami w urzędzie marszałkowskim w katowicach. - ta krajowa strategia jest jedną z dziewięciu strategii sektorowych, które wynikają bezpośrednio z rządowej strategii na rzecz odpowiedzialnego rozwoju - podkreślał. ksrr to dokument, w ramach którego rozpoczęto negocjacje nowej perspektywy finansowej. - zmieniamy w nim całkowicie podejście do rozwoju regionalnego, ponieważ to dotychczasowe okazało się nieskuteczne. rezygnujemy z niego na rzecz zrównoważonego rozwoju, który z jednej strony wzmacnia konkurencyjność naszych regionów, takich jak ślą, a z drugiej strony to rozwój, który dba o wszystkie obszary naszych działań. chcemy, żeby wszyscy polacy mieszkali w dobrych warunkach, żeby mieli wysokiej jakości życie, usługi społeczne, niezależnie od tego, czy mieszkają w dużym, czy małym mieście, czy na terenach wiejskich. ta nowa polityka regionalna musi być bardziej selektywna, skoncentrowana w swoich działaniach - mówił kwieciński. zdaniem ministra inwestycji i rozwoju, połowa średnich miast w polsce boryka się m.in. z problemami społeczno-gospodarczymi. takie miasta, jak wskazywał kwieciński, są również w śląskiem. - to na przykład sosnowiec, bytom czy jastrzębie-zdroju - wyliczał. - chcemy ustanowić nowe instrumenty, które będą wspierały rozwój miast. niektóre z nich już udało się zacząć realizować, czego przykładem jest program dla śląska. śląsk kiedyś był uważamy z 1-2 najlepszy region pod względem konkurencyjności, ten region kojarzył się od razu z przemysłem. dziś ta siła spada, a śląsk jest już na 4. miejscu. udział śląska w tworzeniu produktu krajowego brutto do tej pory spadała. chcemy ten proces zatrzymać - deklarował kwieciński. pytanie: jak to zrobić? - nowe instrumenty finansowe i instytucjonalne, jak chociażby porozumienie terytorialne, program sektorowy bardzo ważny i potrzebny dla śląska, czy to w obszarze energii, czy transportu. porozumienie terytorialne chcemy wdrażać w życie na bazie tego, co zrobiono w szczyru i całym regionie wokół niego. rewitalizacja terenów pod kątem ruchu turystycznego - tłumaczył minister. program dla śląska ma być przykładem, że takie działania są możliwe. - to nie jest program teoretyczny, już w tej chwili mamy w jego realizacji zaangażowane 25 miliardy złotych i bardzo konkretne przedsięwzięcia. w ramach tego programu liczymy jeszcze na zmobilizowanie środków na ponad 50 miliardów złotych - mówił jerzy kwieciński. - mamy przed sobą bardzo dużo pracy. nie ukrywam, że pokładam duże nadzieje w nowym zarządzie i panu marszałku jakubie chełstowskim. do tego pory region śląski ucierpiał na złym zarządzaniu przez ostatnie lata, co niestety przekładało się na problemy rozwojowe. chciałbym, żeby te negatywne trendy udało się przełamać, by śląsk był jasną perłą, świecącą nie tylko na polskę, ale i całą europę - dodał. z kwiecińskim zgodzili się obecni na spotkaniu samorządowcy, w tym wojewoda śląski jarosław wieczorek. - jesteśmy zdeterminowani, żeby region woj. śląskiego wyprowadzić z pewnej pułapki, która ma miejsce. to 4. miejsce, o którym wspomniał minister, to nie jest miejsce, które z naszym potencjałem śląsk powinien zajmować - mówił wojewoda. - sytuacja gospodarcza jest tutaj generalnie niezła, bezrobocie mamy na poziomie 4,3 proc., więc transformacja jest w toku, ale te środki i instrumenty, którymi dysponujemy, można spożytkować na znaczący wzrost i zwiększenie dynamiki naszego rozwoju - dodawał. na koniec spotkania marszałek woj. śląskiego jakub chełstowski zadeklarował, że jest gotowy do pracy. - musimy się wziąć do roboty, przepracować wszystkie sprawy, które są w toku, ułożyć je w dokumenty, zeznania, w projektu i skutecznie, w ścisłej współpracy z rządem i ministerstwem, realizować pomysły. inaczej nasze województwo wypadnie jeszcze dalej. populacja się zmniejsza, w 2050 roku szacuje się, że na śląsku będzie ponad 800 tys. mieszkańców mniej, a to przeniesie się na siłę roboczą i ekonomiczną, wszelkie sprawy życia społecznego i ekonomicznego. musimy podjąć to wzywanie. mamy zdiagnozowane problemy, bierzemy się ostro do pracy - podkreślał chełstowski. & 706 & high & High & Socio-Economic & NA & NA & 2019-01-18 & 2019 & 3 & ECO
Frame & high-very high & Regional & 500-1000 & 1.0448136 & 1.8253861 & -0.2867631 & -1.2581601 & -1.2725643 & 0.0 & -1.1973642 & 0.9927066 & Recipient & Domestic & Domestic & Domestic & Domestic|ECO & Neutral\\
Poland & http://biznes.gazetaprawna.pl/artykuly/1119539,budzet-na-polityke-spojnosci.html & 79 & biznes.gazetaprawna.pl & Private/Non-Public & Online and Offline & National & medium = CP is important part of story & Solidarity to poor countries/regions & Negative & EU & 1.Poor regions funded only & NA & NA & NA & NA & NA & NA & NA & NA & Poland & unijna kasa nie dla polski & 2018-04-24 & polityka spójności & zamiast państw europy środkowej pieniądze mają dostać grecja, włochy i hiszpania. - polityka spójności nie może być ofiarą kłopotów południa - mówi konrad szymański na nieco ponad tydzień przed zaprezentowaniem projektu unijnego budżetu z brukseli płyną niepokojące dla polski sygnały. do wcześniejszych zapowiedzi dotyczących powiązania unijnych funduszy z praworządnością i przyjmowaniem migrantów dochodzi teraz plan wsparcia krajów, które najbardziej ucierpiały podczas kryzysu finansowego. ma on być realizowany kosztem europy środkowej. pisał o tym wczoraj "financial times". brytyjski dziennik podał, że strumień środków z polityki spójności ma zostać przekierowany po 2020 r. z polski, węgier, czech czy krajów bałtyckich - dotychczas największych beneficjentów netto - do hiszpanii, grecji, włoch, a nawet niektórych regionów francji. oznaczałoby to rewolucję w polityce spójności. według wiceministra spraw zagranicznych konrada szymańskiego polska jest coraz bogatsza, więc rola budżetu ue będzie dla naszego kraju malała. - nie zgodzimy się na działania dyskryminacyjne i próby zrewolucjonizowania unijnego budżetu. polityka spójności dobrze służy całej europie i nie może być ofiarą politycznych kłopotów południa i wewnętrznych sprzeczności strefy euro - powiedział dgp wiceszef msz. & 168 & medium & Medium & Values & NA & NA & 2018-04-24 & 2018 & 3 & ECO
Frame & low-medium & National & <500 & 1.0448136 & 1.8253861 & -0.2867631 & -1.2581601 & -1.2725643 & 0.0 & -1.1973642 & 0.9927066 & Recipient & European & European & European & European|ECO & Negative\\
\addlinespace
Poland & http://forsal.pl/artykuly/1346537,europoslowie-w-pe-krytykuja-propozycje-ke-ws-budzetu-na-lata-2021-2027.html & 940 & forsal.pl & Private/Non-Public & Online only & National & medium = CP is important part of story & Institutional bargaining over funding & Balanced & EU & No myth & NA & NA & NA & NA & NA & NA & NA & NA & Poland & europosłowie w pe krytykują propozycję ke ws. budżetu na lata 2021-2027 & 2018-11-13 & fundusze strukturalne & podczas wtorkowej debaty w pe wielu europosłów różnych ugrupowań krytykowało proponowany przez ke projekt budżetu na lata 2021-2027, sprzeciwiając się m.in cięciom w polityce spójności i rolnictwie. apelowali o przyjęcie propozycji pe. debata poprzedziła głosowanie nad propozycją pe ws. budżetu na lata 2021-2027, które zaplanowano na środę. jeśli propozycja ta zostanie przegłosowana, stanie się oficjalnym stanowiskiem pe ws. przyszłych ram finansowych i punktem wyjścia do negocjacji europarlamentu z krajami członkowskimi ws. ostatecznego kształtu budżetu. współsprawozdawcami projektu w pe są polscy europosłowie. jeden z nich, jan olbrycht (po), przekonywał we wtorek w pe, że propozycja komisji europejskiej jest nie do zaakceptowania, bo zaproponowana przez nią wartość budżetu nie pozwoli na realizowanie zadań w ue. przyznał jednak, że pe zgadza się w kilku elementach z propozycją ke, np. z tym, że przyszły budżet powinien uwzględniać nowe zadania, jak obronność, ochrona granic i system azylowy. pe - jak mówił - zgadza się również co do tego, że należy zwiększyć środki na badania naukowe, program erasmus oraz małe i średnie firmy. zbigniew kuźmiuk (pis, ekr) przekonywał, że budżet unijny powinien odpowiadać ambicjom politycznym ue, a nie być ograniczany. jak mówił, jeśli godzimy się na nowe priorytety, jak np. kwestia bezpieczeństwa czy większe środki dla krajów południa europy, to nie może to się odbywać kosztem krajów europy środkowo-wschodniej. "propozycje cięć we wspólnej polityce rolnej i polityce spójności są nie do przyjęcia" - mówił, podkreślając, że cele realizowane przez obie polityki nie straciły na aktualności. przypomniał, że kraje "starej" unii ciągle uzyskują korzyści z dostępu do rynków krajów "nowej" unii i dlatego te polityki powinny być utrzymane. opowiedział się także za wyrównaniem stawek dopłat bezpośrednich w rolnictwie w całej ue. "jeśli tego nie będzie, można uznać, że rolnictwo krajów europy środkowej jest dyskryminowane" - zaznaczył. isabelle thomas (socjaliści) przekonywała, że ke w swojej propozycji przyszłego budżetu "rozmywa" ambicje. krytykowała ke, że chce przeznaczyć mniej środków na politykę rolną i spójności dla najbiedniejszych krajów. "ke chce dalej dokonywać cięć (w budżecie - pap). to jest niezgodne z interesem obywateli. (...). ten budżet to ostatnia szansa dla europy. obiecajmy sobie, że nie zgodzimy się na budżet, który podważy ambicje europy" - mówiła. zdaniem janusza lewandowskiego (po), zagrożeniem dla przyszłego budżetu wcale nie jest brexit, bo największy płatnik do budżetu, czyli niemcy i największy beneficjent, czyli polska deklarują chęć zwiększenia składki. apelował też o przejście do fazy uzgodnienia kształtu budżetu. ivo belet (epl) propozycję pe ws. budżetu nazwał realistyczną. "ci, którzy twierdzą, że budżet powinien zostać zmniejszony, mylą się. przed nami stoją poważne wyzwania. budżet nie powinien zostać zmniejszony" - mówił. derek vaughan (socjaliści) wezwał wszystkich europosłów, aby sprzeciwili się cięciom w polityce spójności. "wzywam, aby większość funduszy trafiła do regionów mniej rozwiniętych, aby wszystkie regiony otrzymywały fundusze strukturalne" - powiedział. zaapelował też o jak najszybsze porozumienie ws. wieloletnich ram finansowych. w podobnym duchu wypowiedział się jordi sole (zieloni). argumentował, że budżet musi być większy, by stawiać czoło wyzwaniom gospodarczym i klimatycznym. "pe proponuje silny budżet wieloletni, dostoswany do wyzwań, które przed nami stoją. propozycja pe jest lepsza niż propozycja ke" - zaznaczył. 5 listopada komisja budżetowa parlamentu europejskiego zagłosowała za tym, by przyszły budżet unijny na lata 2021-27 był większy niż w projekcie komisji europejskiej. więcej pieniędzy ma być przeznaczonych m.in. na naukę, małe firmy i infrastrukturę. jednocześnie posłowie komisji sprzeciwili się cięciom w polityce spójności i rolnictwie. w środę ta propozycja będzie poddana pod głosowanie na sesji plenarnej. jeśli zostanie przyjęta, stanie się oficjalnym stanowiskiem europarlamentu do negocjacji z krajami członkowskimi (radą ue) nad ostatecznym kształtem przeszłego budżetu. posłowie oczekują, że porozumienie zostanie osiągnięte jeszcze przed wyborami do parlamentu europejskiego w 2019 r. długoterminowy budżet na lata 2021-2027 będzie pierwszym po brexicie, więc ue będzie już liczyć 27 państw członkowskich. z tego też powodu w brukseli i stolicach europejskich od miesięcy mówiło się o dziurze po brexicie i konieczności szukania oszczędności. plan przedstawiony 2 maja przez komisarza ds. budżetu guenthera oettingera przewiduje, że w niektórych obszarach wspólnota rzeczywiście wyda mniej, ale nie brak też takich, w których wydatki poszybują w górę. polityka spójności, której polska jest obecnie największym beneficjentem, a także wspólna polityka rolna dalej będą największymi częściami unijnej kasy. z oficjalnych wypowiedzi przedstawicieli komisji europejskiej wynika, że cięcie w polityce spójności ma wynieść 7 proc., natomiast w rolnictwie - około 5 proc. & 702 & medium & Medium & Power & NA & NA & 2018-11-13 & 2018 & 3 & POL
Frame & low-medium & National & 500-1000 & 1.0448136 & 1.8253861 & -0.2867631 & -1.2581601 & -1.2725643 & 0.0 & -1.1973642 & 0.9927066 & Recipient & European & European & European & European|POL & Neutral\\
Poland & https://tvn24bis.pl/article/url/832397 & 909 & TVN24 BiS & Private/Non-Public & Online and Offline & National & medium = CP is important part of story & Institutional bargaining over funding & Balanced & National + Other country & No myth & Territorial cooperation & Positive & National + Other country & No myth & NA & NA & NA & NA & Poland & polska i słowacja chcą zasypać dziurę po brexicie. premierzy za podwyżkami & 2018-04-25 & polityka spójności & morawiecki: nie będzie osobnego budżetu strefy euro "źródło: tvn24 bis" premier mateusz morawiecki po środowym spotkaniu z premierem słowacji peterem pellegrinim podkreślił, że oba kraje myślą "praktycznie identycznie" w kwestii wieloletnich ram finansowych unii europejskiej. szef słowackiego rządu zaznaczył, że oba kraje deklarują chęć podniesienia wpłat do unijnego budżetu. nowy szef słowackiego rządu przebywał środę z wizytą w warszawie, na zaproszenie premiera mateusza morawieckiego. była to jedna z jego pierwszych wizyt zagranicznych w roli premiera, po tym jak został on mianowany na to stanowisko 22 marca. podczas wspólnej konferencji prasowej morawiecki podkreślił, że polska i słowacja to ważni partnerzy handlowi dla siebie. poinformował, że rozmowa z premierem słowacji dotyczyła między innymi wieloletnich rama finansowych, funduszy regionalnych, przyszłych kierunków inwestycji związanych z budżetem ue. morawiecki zaznaczył, że polska i słowacja myślą "praktycznie identycznie" o wieloletnich ramach finansowych ue. - my, kraje komunistyczne, nie mieliśmy tego szczęścia i dzisiaj uważamy, że w ramach kolejnej perspektywy budżetowej powinniśmy utrzymać jak najbardziej równy i solidarnościowy dostęp do środków - oświadczył polski premier. szef rządu poinformował, że z premierem słowacji "obiecali sobie" współpracę w zakresie podatkowym. - my chcemy współpracować po to, żeby te przedsiębiorstwa, które prowadzą biznes, prowadzą swoją działalność gospodarczą w jednym kraju, płaciły tam podatki, gdzie odnoszą korzyść finansową. jeśli to jest na słowacji - to na słowacji proporcjonalnie, jeśli w polsce - to w polsce - mówił morawiecki. - chcemy coraz bardziej mówić jednym głosem na forum unii europejskiej, żeby ten biznes był rzeczywiście uczciwy, żeby unikać rajów podatkowych najmocniej, jak tylko się da - dodał. słowacki premier odnosząc się do kwestii wspólnej walki z mafią watowską wskazał, że powinien powstać system wczesnego ostrzegania, by przyspieszyć przekazywanie informacji pomiędzy krajami o nieuczciwych firmach. według morawieckiego innym tematem rozmów była sytuacja małych i średnich firm w ue, a także swoboda świadczenia usług. - to jest ta swoboda traktatowa, pochodząca z traktatów rzymskich, która pozostawia bardzo wiele do życzenia. my chcemy zaznaczyć naszą mocną pozycję europy środkowej i wschodniej również w tym obszarze. chcemy, żeby nasze firmy transportowe, logistyczne, budowlane były traktowane w sposób uczciwy, jednakowy, tak jak my traktujemy towary i usługi z krajów zachodnich u nas - powiedział szef rządu. premierzy polski i słowacji zgodzili się także - podkreślił morawiecki - w temacie budowy gazociągu nord stream ii. - nie chcemy dalszej monopolizacji relacji gazowych. to jest bardzo ważny surowiec energetyczny, który musi być poddany ustawodawstwu unijnemu, to po pierwsze. po drugie, dywersyfikacja źródeł dostaw gazu jest ważnym, strategicznym tematem zarówno dla polski, jak i dla słowacji - mówił. pellegrini wyraził nadzieję, że rozmowy w ue dotyczące polityki spójności będą prowadzone przy uwzględnieniu argumentu, że "polityka spójności jest po to, by każdy europejczyk - czy jest z północy, południa, wschodu czy zachodu (europy) ma żyć w takim samym świecie, w takich samych warunkach życiowych". pellegrini zaznaczył, że zarówno polska jak i słowacja liczą się z tym, że będą musiały reagować na nowe wyzwania związane ze zmniejszeniem europejskiego budżetu po brexicie. - i stwierdziliśmy - ja w imieniu słowacji, a pan morawiecki w imieniu polski - że jesteśmy chętni do podniesienia naszych wpłat do budżetu, żebyśmy zapełnili tę dziurę po odejściu z ue wielkiej brytanii - zaznaczył słowacki premier. jak dodał, ma świadomość trudności dla ue związanych między innymi z problemem migracji. wyraził przekonanie, że w związku z tym unijny budżet "musi być realizowany rozważnie, a nie tylko wybierać sobie jedną ofiarę, w tym wypadku politykę spójności, z której środki miałyby być przesunięte gdzie indziej". relacjonował też, że podczas spotkania poinformował morawieckiego, iż w kontekście wieloletnich ramach finansowych i budżetu europejskiego słowacja w ramach grupy wyszehradzkiej chce zorganizować szczyt spójności, by wraz z partnerami omówić jak ma wyglądać przyszły budżet unijny. obaj premierzy podkreślili również znaczenie grupy wyszehradzkiej. morawiecki ocenił, że współpraca v4 na forum europejskim jest coraz lepsza i ma na celu "wzmocnienie ue, a nie jej osłabienie". pellegrini zaznaczył, że v4 nie może być alternatywnym formatem, lecz integralną częścią ue. komisja europejska debatowała w środę nad przyszłym budżetem unijnym po 2020 roku. wszyscy komisarze opowiedzieli się za wprowadzeniem mechanizmu uzależniania wypłaty środków ue od praworządności w państwach członkowskich. debata na ten temat trwa od miesięcy. w opublikowanym w lutym przez ke dokumencie o nowych wieloletnich ramach finansowych warunkowość, oznaczająca uzależnienie wypłaty funduszy od spełnienia określonych standardów w wymiarze sprawiedliwości, była wskazywana jako opcja. polska jest przeciwna powiązaniu funduszy unijnych z praworządnością. wiceszef msz ds. europejskich konrad szymański powiedział w poniedziałek, że polska nie zgodzi się na uznaniowe mechanizmy, które z zarządzania funduszami ue uczynią - jak to ujął - instrument politycznej presji na państwa członkowskie. budżet musi rządzić się prawem, a nie uznaniowością - dodał. propozycja nowego budżetu ma zostać przyjęta 2 maja. & 751 & medium & Medium & Power & Socio-Economic & NA & 2018-04-25 & 2018 & 3 & POL
Frame & low-medium & National & 500-1000 & 1.0448136 & 1.8253861 & -0.2867631 & -1.2581601 & -1.2725643 & 0.0 & -1.1973642 & 0.9927066 & Recipient & Domestic & European & Mixed & Domestic|POL & Neutral\\
Poland & http://wgospodarce.pl/informacje/42941-ke-nie-ma-zaufania-do-rzadow-panstw-ue?utm\_source=feedburner\&utm\_medium=feed\&utm\_campaign=Feed\%3A+wGospodarce+\%28wGospodarce.pl\%29\&utm\_content=FeedBurner & 147 & wgospodarce.pl & Private/Non-Public & Online only & National & high = CP is most important issue in story (can also cover other issues) & Institutional bargaining over funding & Balanced & EU + National & No myth & NA & NA & NA & NA & NA & NA & NA & NA & Poland & ke nie ma zaufania do rządów państw ue & 2017-11-23 & polityka spójności & tagi: bruksela finanse fundusze unijne komisja europejska pkb polityka polityka spójności ue pkb na mieszkańca powinno pozostać głównym kryterium decydującym o przyznaniu środków z polityki spójności po 2020 r. - powiedział we wtorek w brukseli wiceminister rozwoju paweł chorąży. dodał, że polityka ta powinna być skierowana do wszystkich regionów ue chorąży wziął udział we wtorek w brukseli w konferencji poświęconej przyszłości polityki spójności w ue, zorganizowanej przez stałe przedstawicielstwo rp przy ue w brukseli, dom polski wschodniej w brukseli, ministerstwo rozwoju i europosła krzysztofa hetmana (psl). wiceminister powiedział, że z polskiego punktu widzenia po roku 2020 pkb per capita powinno pozostać głównym kryterium przyznawania środków unijnych. jak powiedział klasyk, to może nie jest najlepsze rozwiązanie, tylko jak dotąd nie wymyślono lepszego. bardzo mocno postulujemy, aby polityka spójności była polityką skierowaną do wszystkich regionów unii europejskiej, uwzględniając ich różny kontekst. nawet londyn, który już nie skorzysta z polityki spójności po 2020 r., jest miejscem, w którym występuje szereg problemów o charakterze strukturalnym, problemów społecznych - wyjaśnił. chorąży podkreślił, że polityka spójności jest jednym z nielicznych obszarów w polskiej polityce, który nie jest tym, który dzieli. stanowisku polski, (...) przyjętemu w lipcu tego roku przez rząd, towarzyszył głos regionów, które w zdecydowanej większości mają inny kolor polityczny niż rząd w polsce. te stanowiska w 99 proc. są ze sobą zbieżne - powiedział. zaznaczył, że polska optuje też za radykalnym uproszczeniem funduszy spójności. to jest postulat, pod którym zapewne wszyscy się podpiszą. my natomiast zwracamy uwagę, że uproszczenie powinno mieć charakter ewolucyjny. nie stać nas na rewolucję uproszczeniową. (...) podstawy legislacyjne tej polityki powinny opierać się na tym, co mamy w tej perspektywie budżetowej - podkreślił. z kolei europoseł hetman podkreślał na konferencji, że warunkiem, aby polityka spójności była skuteczna, jest zacieśnienie współpracy z władzami lokalnymi i regionalnymi. zasada partnerstwa musi stać na pierwszym miejscu - zaznaczył. jego zdaniem w brukseli rozpocznie się dyskusja ws. wykorzystania w polityce spójności alternatywnych wskaźników wobec pkb na mieszkańca. raport europarlamentu wskazuje, by ke zwróciła uwagę na nowe wskaźniki, takie jak postęp społeczny lub wskaźniki demograficzne - powiedział. dodał, że pe w wielu raportach odwołuje się do pomysłów, by wskaźniki dotyczące demografii były jednymi z dodatkowych obok pkb. może to jest klucz do kolejnej edycji programu polska wschodnia - zaznaczył europoseł, dodając, że w 13 nowych państwach ue dużo ludzi wyjeżdża z regionów peryferyjnych, a zupełnie inna tendencja występuje w krajach "starej" ue, gdzie obserwuje się napływ do granicznych regionów. dodał, że w brukseli mówi się też o wzięciu pod uwagę wskaźników dotyczących bezrobocia. zdaniem hetmana jednym z największych problemów polityki spójności jest "brak zaufania". komisja europejska nie ma zaufania do rządów państw członkowskich, jeśli chodzi o realizowane programy, rządy nie mają zaufania do regionów i instytucji, które na poziomie regionalnym realizują poszczególne programy, instytucje wdrażające i zarządzające nie mają za grosz zaufania dla beneficjentów - powiedział. europoseł dodał, że brak zaufania działa też w drugą stronę. beneficjenci nie mają zaufania do instytucji wdrażających, bo często zmieniają zasady, instytucje wdrażające mają pretensje do swoich rządów, a ze strony rządów brak zaufania do ke. jeśli chcemy skutecznej polityki spójności, przede wszystkim musimy postawić na wzajemne zaufanie - powiedział. projekty w ramach polityki spójności są finansowane ze środków europejskiego funduszu rozwoju regionalnego (efrr), europejskiego funduszu społecznego (efs) oraz funduszu spójności - jego beneficjentami są państwa członkowskie, których pkb jest niższe niż 90 proc. średniego pkb w krajach ue-27, nie uwzględniając chorwacji. spójność gospodarcza i społeczna, zgodnie z definicją podaną w jednolitym akcie europejskim z 1986 r., ma za zadanie "zmniejszać dysproporcje między regionami i przeciwdziałać zacofaniu regionów znajdujących się w niekorzystnej sytuacji". ostatni traktat unijny, traktat lizboński, nadaje spójności dodatkowe znaczenie, odwołując się do "spójności gospodarczej, społecznej i terytorialnej". & 599 & high & High & Power & NA & NA & 2017-11-23 & 2017 & 2 & POL
Frame & high-very high & National & 500-1000 & 1.0448136 & 1.8253861 & -0.2867631 & -1.2581601 & -1.2725643 & 0.0 & -1.1973642 & 0.9927066 & Recipient & Domestic & European & Mixed & Domestic|POL & Neutral\\
Poland & http://biznes.interia.pl/news/rzad-szykuje-sie-na-bitwe-o-unijne-pieniadze\%2C2558072?utm\_source=Biznes\&utm\_medium=RSS\&utm\_campaign=RSS & 548 & biznes.interia.pl & Private/Non-Public & Online only & National & medium = CP is important part of story & Institutional bargaining over funding & Factual & EU & No myth & NA & NA & NA & NA & NA & NA & NA & NA & Poland & rząd szykuje się na bitwę o unijne pieniądze & 2018-02-24 & polityka spójności & stanowisko rządu jest jednoznaczne. to próba uniknięcia zagrożenia, jakim są zakusy na przesunięcia w budżecie ue po wyjściu wielkiej brytanii. dla polski najbardziej bolesne mogłyby być cięcia w polityce spójności, której jesteśmy największym beneficjentem na starcie negocjacji budżetowych głównym zagrożeniem dla polski są plany cięć w funduszach spójności i polityce rolnej. po wyjściu z ue wielkiej brytanii i pojawieniu się w związku z tym luki w unijnej kasie inne pozycje wydatkowe mają być wzmacniane kosztem tych funduszy. poznaliśmy polskie stanowisko na pierwszy etap rozmów o budżecie. - jeśli mają być nowe cele, potrzebne są nowe pieniądze. polska ma konstruktywne stanowisko: oferujemy wsparcie dla komisji europejskiej, jeśli chodzi o podniesienie budżetu europejskiego i tym samym podwyższenie wysokości składki. jesteśmy gotowi do rozmowy o tym, by płacić więcej, ale jest to uwarunkowane generalnym kompromisem w sprawie tego, na co mamy płacić - mówi konrad szymański, wiceszef msz odpowiedzialny za politykę unijną. w podobnym tonie wypowiada się wiceminister inwestycji i rozwoju paweł chorąży. - nasze otwarcie na dyskusję o podwyższeniu składki zależy od tego, co będzie w budżecie. jeśli nasze priorytety nie będą uszczuplone, jesteśmy w większym stopniu gotowi partycypować w budżecie - podkreśla wiceszef resoru rozwoju. stanowisko rządu jest jednoznaczne. to próba uniknięcia zagrożenia, jakim są zakusy na przesunięcia w budżecie ue po wyjściu wielkiej brytanii. dla polski najbardziej bolesne mogłyby być cięcia w polityce spójności, której jesteśmy największym beneficjentem. wewnętrzne dokumenty - opracowane przez dyrekcje generalne komisji europejskiej - wskazują, że brane są pod uwagę trzy scenariusze. dwa z nich oznaczają cięcia na poziomie 15 i 30 proc. plan zmniejszenia funduszy o 15 proc. według cen nominalnych (26 proc. według cen z 2011 r.) zakłada, że z dotacji mogłyby korzystać jedynie mniej rozwinięte regiony oraz kraje, których dnb (dochód narodowy brutto) na jednego mieszkańca wynosi poniżej 90 proc. unijnej średniej. takie rozwiązanie wykluczyłoby większość regionów we francji i niemczech. gdyby wprowadzono bardziej radykalne cięcie - o 30 proc. - z polityki spójności nie mogłyby korzystać także mniej rozwinięte regiony włoch i hiszpanii. w takim scenariuszu na fundusze spójności mogłyby liczyć wyłącznie kraje spełniające warunek dotyczący unijnej średniej, a więc kraje europy środkowej. trzeci scenariusz zakłada, że polityka spójności będzie dostępna - jak dotychczas - dla wszystkich regionów. fundusze zostałyby "zamrożone" według cen z 2011 r., co oznaczałoby de facto ich zwiększenie o 15 proc. w stosunku do obecnej perspektywy finansowej 2014-2020. dziś trudno przewidywać, w którą stronę potoczą się negocjacje. ważne są nie tylko sumy, lecz także kryteria przyznawania funduszy. więc szacunki, ile stracimy, są w nowym budżecie bardzo różne. nawet gdyby wielka brytania nadal była w ue i tak w kolejnym budżecie środki dla polski byłyby relatywnie mniejsze. to efekt rosnącego pkb i nadrabiania zaległości w rozwoju gospodarczym wobec innych krajów. jak zwraca uwagę wiceminister chorąży, w kolejnej perspektywie cztery lub nawet pięć polskich regionów przekroczy poziom 75 proc. unijnego pkb, co oznacza dla nich mniejszy udział w funduszach spójności. to warszawa, która została wyodrębniona jako oddzielny region statystyczny, ale także województwa: dolnośląskie, wielkopolskie, pomorskie i śląskie. jeśli do tego dojdą cięcia budżetu ue z powodu brexitu czy finansowania innych priorytetów, pieniądze dla polski mogą się okazać sporo mniejsze. - gdyby cięcia odnieść wprost do propozycji komisji, to przy obecnej wysokości naszych funduszy przy 10-proc. zmniejszeniu wydatków na spójność byłoby to ok. 8 mld euro dla nas mniej. gdyby cięto fundusze o 20 proc., wówczas ubytek wyniósłby nawet 16 mld euro mniej. żadnego scenariusza nie można wykluczyć - mówi europoseł po jan olbracht. te sumy zależą od kryteriów, jakie zostaną zastosowane do rozdziału funduszy. od jednego ze źródeł unijnych usłyszeliśmy, że polski budżet może być niższy w kolejnej perspektywie o 20-30 mld euro. to olbrzymia suma, biorąc pod uwagę, że polskie korzyści w polityce spójności i wspólnej polityce rolnej to ponad 100 mld euro na całe siedem lat. ubytek wyniósłby więc nawet 30 proc. ale takich szacunków nie potwierdzają inne źródła. w tym źródła rządowe. choć nie spotkaliśmy się jednocześnie z zaprzeczeniami. możliwe, że to element rozgrywki negocjacyjnej, pozycjonowania różnych graczy i krzyżujących się interesów. wydaje się jednak, że tak duże cięcia są wątpliwe i byłyby nie do zaakceptowania z powodów politycznych dla rządu. o wiele bardziej prawdopodobne jest, że cięcia będą mniejsze. jak wynika z raportu konferencji peryferyjnych regionów nadmorskich europy, jeśli zaakceptowane zostałyby obecne propozycje ke dotyczące cięć w polityce spójności po 2020 r., polska straciłaby zaledwie 5 proc. dotacji. konferencja zrzeszająca 160 regionów europejskich przedstawiła w raporcie symulacje, w jaki sposób rozważane przez ke warianty oszczędności wpłynęłyby na poszczególne kraje. wynika z niego, że na cięciach najmniej straciłyby kraje europy środkowej. zarówno w przypadku redukcji o 15 proc., jak i o 30 proc. dotacje dla polski zmniejszyłyby się zaledwie o 5 proc. podobnie wygląda sytuacja innych państw w regionie - czechy i słowacja straciłyby po 3 proc., węgry i rumunia - po 4 proc. to niedużo w porównaniu z niemcami (-94 proc. w obu wariantach cięć), francją (-76 proc. przy redukcji o 15 proc., -91 proc. - o 30 proc.), hiszpanią (-92 proc. przy redukcji o 15 proc., -98 proc. - o 30 proc.) czy belgią (-85 proc. w obu wariantach). wprowadzenie mniej radykalnego wariantu zakładającego zmniejszenie funduszy o 15 proc. wykluczałoby w praktyce z polityki spójności już większość regionów niemiec i hiszpanii oraz drastycznie ograniczyło możliwość korzystania ze wsparcia przez regiony we francji i belgii. niewykluczone, że kraje członkowskie postanowią podtrzymać finansowanie polityki spójności na dotychczasowym poziomie - co uwzględnia trzeci wariant proponowany przez komisję. ten optymistyczny scenariusz zakładałby de facto wzrost funduszy spójności o 15 proc. w stosunku do cen bieżących (bądź "zamrożenie" progów budżetowych według cen z 2011 r.). można się jednak spodziewać, że taki scenariusz oznaczałby zmianę celów i warunków przyznawania funduszy, jak zapowiadane powiązanie wypłaty z unijnej kasy z przestrzeganiem zasad państwa prawa. - to, jak polityka będzie wyglądać po 2020 r., zależeć będzie od procesu negocjacji i pozycji poszczególnych państw członkowskich. nasza w ostatnim czasie nie jest najmocniejsza - mówi dgp członek komitetu regionów i marszałek województwa zachodniopomorskiego olgierd geblewicz. & 973 & medium & Medium & Power & NA & NA & 2018-02-24 & 2018 & 3 & POL
Frame & low-medium & National & 500-1000 & 1.0448136 & 1.8253861 & -0.2867631 & -1.2581601 & -1.2725643 & 0.0 & -1.1973642 & 0.9927066 & Recipient & European & European & European & European|POL & Neutral\\
Poland & http://wroclaw.naszemiasto.pl/artykul/opera-wroclawska-wkrotce-zostanie-rozbudowana,4624645,artgal,t,id,tm.html & a15 & wroclaw.naszemiasto.pl & Private/Non-Public & Online and Offline & Regional/Local & medium = CP is important part of story & Cultural heritage & Positive & Subnational & No myth & NA & NA & NA & NA & NA & NA & NA & NA & Poland & opera wrocławska wkrótce zostanie rozbudowana & 2018-04-26 & europejski fundusz rozwoju regionalnego & opera wrocławska wkrótce się rozbuduje. będzie mieć magazyny kostiumów i rekwizytów teatralnych bibliotekę multimedialną. w nowych pomieszczeniach będzie stacja transformatorowa z rozdzielnią, pomieszczenia przyłączy, węzeł cieplny, wentylatorownia, pompownia i zbiornik na wodę tryskaczową. opera już ogłosiła przetarg na rozbudowę instytucji wraz z zakupem wyposażenia. nowe miejsce opera ma między swoim budynkiem a placem wolności. - na nasze potrzeby wystarczy zabudowanie poziomu -1, głębiej nie będziemy nic budować, ze względu na bliskość fosy - wyjaśnia marcin nałęcz-niesiołowski, dyrektor opery wrocławskiej. wartość projektu to ponad 16 mln 15 tys. zł, europejski fundusz rozwoju regionalnego dofinansował projekt prawie 11 mln zł, samorząd województwa dał prawie 4 mln 800 tys. zł.poziom -1 ma być rozbudowany do końca pierwszego kwartału 2020 roku. ale to nie koniec nowości. w ramach partnerstwa publiczno-prywatnego, na terenie parkingu między placem wolności a gmachem opery powstać ma budynek mieszczący duże sale prób dla orkiestry i chóru, a także mniejsze sale prób dla solistów. - partner prywatny będzie wyłoniony na drodze przetargu, będzie mógł prowadzić działalność zgodną z planem zagospodarowania przestrzennego - tłumaczy dyrektor opery. - rozmowy z potencjalnymi partnerami planujemy po wakacjach - dodaje. - partnerstwo publiczno-prywatne to formuła wymagająca czasu. zapraszamy potencjalnych partnerów do rozmowy, szukamy wspólnej wizji. postępowanie trwa 6-12 miesięcy - tłumaczy paulina kirschke, dyrektor ds. marketingu w poznańskiej firmie ingenis pomagającej samorządom i przedsiębiorstwom w pozyskiwaniu funduszy na dalszy rozwój. dobrym przykładem partnerstwa publiczno-prywatnego jest muzeum w dusseldorfie. - zdecydowało się na tę formułę, mieli problemy z remontem budynku, oddali partnerowi część miejsca za muzeum, partner wybudował biurowce, sponsoruje część działalności muzeum. dzięki temu muzeum jest utrzymywane przez partnera prywatnego - opowiada paulina kirschke. & 264 & medium & Medium & Socio-Economic & NA & NA & 2018-04-26 & 2018 & 3 & ECO
Frame & low-medium & Regional & <500 & 1.0448136 & 1.8253861 & -0.2867631 & -1.2581601 & -1.2725643 & 0.0 & -1.1973642 & 0.9927066 & Recipient & Domestic & Domestic & Domestic & Domestic|ECO & Positive\\
\addlinespace
Poland & https://plus.dziennikzachodni.pl/wiadomosci/a/mogliby-wczesniej-wykryc-raka-ale-nie-chca-sie-szkolic-choc-szkolenia-sa-bezplatne,12964565 & a13 & plus.dziennikzachodni.pl & Private/Non-Public & Online and Offline & Regional/Local & high = CP is most important issue in story (can also cover other issues) & Public services & Positive & Subnational & No myth & NA & NA & NA & NA & NA & NA & NA & NA & Poland & mogliby wcześniej wykryć raka,  ale nie chcą się szkolić. choć szkolenia są bezpłatne & 2018-02-27 & europejski fundusz społeczny & śląski uniwersytet medyczny prowadzi rekrutację dla lekarzy i pielęgniarek poz, by ich szkolić, aby wcześniej rozpoznawali nowotwory głowy i szyi. chętnych lekarzy podstawowej opieki zdrowotnej na szkolenia z onkologii jest niewielu. to zła wiadomość dla pacjentów. europejski fundusz społeczny na szkolenia dla lekarzy i pielęgniarek z trzech województw: śląskiego, małopolskiego i dolnośląskiego przeznaczył 850 tys. zł. aby projekt przyniósł efekt, powinno w nim uczestniczyć sto placówek (60 ze śląskiego i po 20 z pozostałych województw). ale choć termin rekrutacji mija ostatniego lutego, to do tej pory zgłosiły się do niego tylko 22 placówki. na czym polega projekt? celem szkolenia jest zwiększenie wykrywalności tych nowotworów, bo niestety należą do najpóźniej wykrywanych, co znacznie zmniejsza szanse pacjentów na wyleczenie. - szkolenie jest bezpłatne. obejmuje dwa kilkugodzinne spotkania w określone weekendy. za każdą przeprowadzoną konsultację medyczną placówka otrzyma potem 48 zł - wyjaśnia beata laubach z porozumienia zielonogórskiego, która zachęca do udziału w programie i pomaga sum w znalezieniu chętnych. każdy z uczestników ma oddelegować do projektu po dwóch pracowników: jednego lekarza i jedną pielęgniarkę. w swoich przychodniach będą po szkoleniu "wyłapywać" osoby z grupy ryzyka (40-65 lat), u których pojawią się objawy mogące świadczyć o możliwości zachorowania na jeden z nowotworów głowy i szyi. trzeba będzie wypełnić specjalną ankietę. - jest to niezwykle ważne przedsięwzięcie dla naszych pacjentów, bo wczesne wykrycie nowotworu głowy i szyi to także leczenie powodujące mniejsze okaleczenie - mówi prof. maciej misiołek, koordynator projektu, kierownik katedry i oddziału klinicznego otorynolaryngologii i onkologii laryngologicznej sum. od ponad dekady zachorowalność na nowotwory głowy i szyi rośnie. \#obecnie choruje o 25 proc. osób więcej niż w 2002 roku. najwięcej osób choruje na raka krtani. jak się dowiadujemy, prawdopodobnie w marcu sum ogłosi drugi etap rekrutacji, by dać szansę wahającym się placówkom. lekarz lub pielęgniarka, którzy wytypują pacjenta z grupy ryzyka, powinni mu m.in. wskazać adresy ośrodków, gdzie będzie mógł przeprowadzić podstawowe badania dotyczące nowotworów głowy i szyi, ale też odbyć z nim rozmowę edukacyjną. unia europejska nie bez powodu przeznaczyła z europejskiego funduszu społecznego niemal milion złotych, by lekarze i pielęgniarki znajdujący się najbliżej chorych, a więc zatrudnieni w podstawowej opiece zdrowotnej, z większą czujnością onkologiczną obserwowali pacjentów narażonych bardziej od innych na zachorowania na nowotwory głowy i szyi. zachorowalność właśnie na te nowotwory systematycznie rośnie, a chorzy zbyt późno trafiają do specjalistów, by wdrożyć skuteczne i mało inwazyjne leczenie. w konsekwencji pacjenci nie wracają do pracy, stają się społecznie wykluczeni i, niestety, coraz więcej z nich umiera. większa czujność w przypadków tych nowotworów jest więc koniecznością. & 414 & high & High & Socio-Economic & NA & NA & 2018-02-27 & 2018 & 3 & ECO
Frame & high-very high & Regional & <500 & 1.0448136 & 1.8253861 & -0.2867631 & -1.2581601 & -1.2725643 & 0.0 & -1.1973642 & 0.9927066 & Recipient & Domestic & Domestic & Domestic & Domestic|ECO & Positive\\
Poland & https://tvn24bis.pl/article/url/888053 & 14 & TVN24 BiS & Private/Non-Public & Online only & National & medium = CP is important part of story & Institutional bargaining over funding & Positive & National & No myth & NA & NA & NA & NA & NA & NA & NA & NA & Poland & premier: celem jest osiągnięcie takiego poziomu życia polaków jak w europie zachodniej & 2018-11-29 & polityka spójności & chcemy, żeby polacy czerpali jak najwięcej korzyści z rozwoju wspieranego środkami unijnymi; podstawowym celem jest osiągnięcie takiego poziomu życia polaków jak w europie zachodniej - mówił w bratysławie premier mateusz morawiecki po szczycie grupy przyjaciół polityki spójności. w przyjętej podczas czwartkowego szczytu w bratysławie deklaracji grupa przyjaciół polityki spójności zadeklarowała chęć utrzymania finansowania polityki spójności i wspólnej polityki rolnej na poziomie z wieloletnich ram finansowych z lat 2014-2020. - napawa mnie bardzo dużą radością, że są tutaj kraje z różnych części ue, a jednocześnie bardzo mocno podzielają nasze spojrzenie, nasz pogląd, że polityka spójności w ramach negocjowanego obecnie budżetu ue na następną perspektywę (...) jest kluczową polityką i że musimy wynegocjować jak najlepszą pulę środków właśnie przeznaczoną na spójność - oświadczył morawiecki. zdaniem premiera jest ważne, że polska odgrywa "jedną z wiodących ról", jeśli chodzi o integrację i budowę jednolitego, jednoznacznego stanowiska krajów-przyjaciół spójności w odniesieniu do przyszłego budżetu ue. jak dodał, odnosząc się do przebiegu szczytu w bratysławie, kraje te jednoznacznie wyznaczyły "pewne ramy, które są dla nas absolutnie kluczowe". premier podkreślił, że stanowisko grupy przyjaciół polityki spójności jest jednoznaczne. poinformował również, że rozmawiał o budżecie przed szczytem w bratysławie z unijnym komisarzem ds. budżetu guentherem oettingerem. morawiecki mówił również o znaczeniu wyjścia z ue wielkiej brytanii, która - jak dodał - była ważnym płatnikiem netto do budżetu, dlatego m.in. w unijnym kasie znajdzie się mniej pieniędzy. wśród celów ue wymienił m.in. innowacje i zapobieganie migracji. - to są dla nas ważne cele, które wspieramy i w ramach tych celów chcemy realizować naszą strategię rozwojową - przyspieszenia rozwoju w taki sposób, żeby polacy czerpali jak najwięcej korzyści z rozwoju gospodarczego wspieranego środkami unijnymi - powiedział szef rządu. jak podkreślił, "podstawowym celem jest osiągniecie takiego poziomu życia dla polaków, jakim cieszą się narody i państwa zachodniej europy". jednocześnie w rozmowie z pap minister inwestycji i rozwoju jerzy kwieciński zapewnił, iż polska jest gotowa wpłacać więcej pieniędzy do wspólnego budżetu unii aby utrzymać politykę spójności i wspólna politykę rolną. - polska wraz z 15 państwami ue są gotowe podnieść składkę członkowską do budżetu ue, by nie ograniczać w przyszłej perspektywie polityki spójności i wspólnej polityki rolnej -zapewnił minister. kwieciński przypomniał, że wyjście wielkiej brytanii z ue spowoduje uszczuplenie jej budżetu o ok. 10 proc. kwieciński pytany, kiedy realnie można spodziewać się przyjęcia budżetu na nową perspektywę, ocenił, że raczej nie stanie się to przed majowymi wyborami do parlamentu europejskiego. - optymizm, żeby zakończyć negocjacje przez wyborami wyraźnie osłabł i prawdopodobieństwo, że budżet zostanie uchwalony w tym czasie jest niewielki. w tej chwili znacznie częściej mówi się o drugiej połowie 2019 roku podczas prezydencji fińskiej. jeżeli tak się nie stanie budżet zapewne zostanie przyjęty dopiero w 2020 r. za prezydencji chorwackiej, albo nawet niemieckiej - powiedział. w szczycie wzięli udział premierzy ośmiu państw: słowacji, polski, czech, estonii, chorwacji, węgier, malty i słowenii, a także przedstawiciele władz: bułgarii, cypru, litwy, łotwy, rumunii, włoch, portugalii i grecji. polskę reprezentował premier mateusz morawiecki oraz minister inwestycji i rozwoju jerzy kwieciński. w szczycie wzięli też udział unijny komisarz ds. unii energetycznej marosz szefczovicz oraz komisarz ds. budżetu guenther oettinger, a także wiceprezes europejskiego banku inwestycyjnego vazil hudak. czwartkowy szczyt w bratysławie był pierwszym spotkaniem w tym formacie od sześciu lat. ostatni szczyt odbył się w październiku 2012 r. & 534 & medium & Medium & Power & NA & NA & 2018-11-29 & 2018 & 3 & POL
Frame & low-medium & National & 500-1000 & 1.0448136 & 1.8253861 & -0.2867631 & -1.2581601 & -1.2725643 & 0.0 & -1.1973642 & 0.9927066 & Recipient & Domestic & Domestic & Domestic & Domestic|POL & Positive\\
Poland & http://www.gazetaprawna.pl/artykuly/1113573,szymanski-o-kompromisie-w-spr-budzetu.html & 241 & gazetaprawna.pl & Private/Non-Public & Online and Offline & National & low = CP mentioned more times but NOT important part of story (mainly about others issues) & Institutional bargaining over funding & Factual & National & No myth & NA & NA & NA & NA & NA & NA & NA & NA & Poland & szymański: polska i ke z podobną wizją budowania kompromisu ws. wieloletniego budżetu ue & 2018-03-26 & polityka spójności & polska i komisja europejska mają podobną wizję budowania kompromisu ws. wieloletniego budżetu ue - powiedział pap wiceszef msz ds. europejskich konrad szymański po poniedziałkowym spotkaniu w warszawie z unijnym komisarzem ds. budżetu i zasobów ludzkich guentherem oettingerem. jak podkreślił wiceminister, zarówno komisja, jak i rząd w warszawie widzą "potrzebę reform i sfinansowania nowych celów np. w zakresie bezpieczeństwa, ale nie kosztem polityki spójności, czy polityki rolnej". "gdyby budżet zależał tylko od dobrego porozumienia brukseli i warszawy, o kompromis byłoby bardzo łatwo. sytuacja polityczna w niektórych państwach członkowskich jest jednak na tyle trudna, że zbliżające się negocjacje będą bardzo trudne" - dodał szymański. ocenił przy tym, że ograniczenie pieniędzy unijnych na skutek brexitu jest "tylko jednym z problemów", jakie stoją przed uczestnikami negocjacji ws. wieloletnich ram finansowych ue po roku 2020. komisarz oettinger od połowy 2017 r. spotyka się z premierami, ministrami finansów i ministrami spraw zagranicznych państw członkowskich, aby zapoznać się z ich stanowiskiem ws. przyszłych wieloletnich ram finansowych ue po 2020 r. dotychczas odwiedził 21 krajów. zobacz także:szymański o projektach pis: szansa na wyjście z impasu, który jest zły dla ue i niewygodny dla polski " podczas poniedziałkowej wizyty w warszawie oettinger spotkał się z premierem mateuszem morawieckim, rozmawiał ponadto także z wiceministrami finansów piotrem nowakiem i tomaszem robaczyńskim. po południu rozmawiali m.in. o przyszłości finansów unii europejskiej, w tym wspólnotowego budżetu po 2020 r. jak poinformowała kancelaria premiera, podczas spotkania szefa rządu z unijnym komisarzem ds. budżetu i zasobów strona polska przedstawiła najważniejsze polskie postulaty dotyczące przyszłego budżetu wspólnoty. premier - poinformowano - "wskazał, że polska opowiada się za ambitnym budżetem ue, który będzie uwzględniał pozytywne efekty generowane przez politykę spójności i wspólną politykę rolną". również przedstawiciele ministerstwa finansów - jak poinformowano w komunikacie resortu - podkreślili na spotkaniu z komisarzem ue, że polska popiera przyjęcie ambitnego budżetu ue po roku 2020. "konieczne jest także znalezienie równowagi między finansowaniem nowych wyzwań a tradycyjnymi politykami ue, takimi jak polityka spójności i wspólna polityka rolna. zwrócili także uwagę na kwestie związane z zarządzaniem budżetem ue, w tym na konieczność zapewnienia odpowiedniego poziomu środków w kolejnych, rocznych budżetach ue" - poinformował resort finansów. w poniedziałek po południu komisarz oettinger wziął udział w posiedzeniu połączonych sejmowych komisji: do spraw unii europejskiej oraz komisji finansów publicznych. & 367 & low & Low & Power & NA & NA & 2018-03-26 & 2018 & 3 & POL
Frame & low-medium & National & <500 & 1.0448136 & 1.8253861 & -0.2867631 & -1.2581601 & -1.2725643 & 0.0 & -1.1973642 & 0.9927066 & Recipient & Domestic & Domestic & Domestic & Domestic|POL & Neutral\\
Poland & https://biznes.trojmiasto.pl/Nowy-Dworzec-Drzewny-w-Porcie-Gdansk-n122150.html & a14 & trojmiasto.pl & Private/Non-Public & Online only & Regional/Local & high = CP is most important issue in story (can also cover other issues) & Infrastructure & Positive & Subnational & NA & NA & NA & NA & NA & NA & NA & NA & NA & Poland & nowy dworzec drzewny w porcie gdańsk & 2018-03-20 & fundusze strukturalne & nienawidzący polityki europejskiej rząd i narodek wyciąga obie rączki po pieniążki zaborców z ue do tego mijający się z prawdą. doprawdy trzeba być debilem, żeby płacić przymusową składkę na organizację, której się rzekomo nie lubi a w ramach protestu nie brać określonych w traktatach kasy (fundusze strukturalne nie są żadną jałmużną) pochodzącej z tej składki. po drugie to nienawidzenie polityki europejskiej według niektórych najwyraźniej polega na oponowaniu przeciwko niekorzystnym dla polski propozycjom. wedle tej logiki najbardziej polityki europejskiej nienawidzą niemcy i francuzi, którzy nawet brali się za renegocjacje traktatu i to nie raz. & 93 & high & High & Socio-Economic & NA & NA & 2018-03-20 & 2018 & 3 & ECO
Frame & high-very high & Regional & <500 & 1.0448136 & 1.8253861 & -0.2867631 & -1.2581601 & -1.2725643 & 0.0 & -1.1973642 & 0.9927066 & Recipient & Domestic & Domestic & Domestic & Domestic|ECO & Positive\\
Poland & http://gazetaolsztynska.pl/304794,Unijne-wsparcie-dla-Warmii-i-Mazur.html & a38 & http://gazetaolsztynska.pl & Private/Non-Public & Online and Offline & Regional/Local & high = CP is most important issue in story (can also cover other issues) & Environment/green/low-carbon & Positive & Subnational & No myth & NA & NA & NA & NA & NA & NA & NA & NA & Poland & unijne wsparcie dla warmii i mazur & 2015-10-07 & fundusz rozwoju regionalnego & -trzeba stwierdzić, że rzeczywiście wielkość środków przyznanych naszemu regionowi jest bardzo duża. program realizuje unijną strategię na rzecz inteligentnego, zrównoważonego wzrostu sprzyjającego włączeniu społecznemu oraz osiągnięciu spójności gospodarczej, społecznej i terytorialnej. jest programem dwufunduszowym finansowanym z dwóch źródeł: środków europejskiego funduszu rozwoju regionalnego i europejskiego funduszu społecznego.
-wsparcie z programu umożliwi realizację inwestycji z zakresu efektywności energetycznej i wykorzystania odnawialnych źródeł energii. przedsięwzięcia będą również realizowane z zakresu zrównoważonego transportu miejskiego, gospodarki odpadami, gospodarki wodno-ściekowej, ochrony różnorodności biologicznej oraz zapobiegania i zarządzania ryzykiem.
-działania rpo wim na lata 2007-2013 w zakresie ochrony środowiska pozwoliły na realizację projektów, które kompleksowo wpłynęły na poszczególne elementy środowiska. przyczyniły się do poprawy jakość wód, powietrza i ziemi w regionie. ochrona wód nastąpiła min. poprzez budowę sieci kanalizacyjnych, dzięki którym większa ilość ścieków wpływa do nowo wybudowanych, czy też zmodernizowanych oczyszczalni. natomiast zastosowane w projektach systemy selektywnej zbiórki odpadów niewątpliwie pozytywnie wpłynęły na walory przyrodnicze. & 152 & high & High & Socio-Economic & NA & NA & 2015-10-07 & 2015 & 1 & ECO
Frame & high-very high & Regional & <500 & 1.0448136 & 1.8253861 & -0.2867631 & -1.2581601 & -1.2725643 & 0.0 & -1.1973642 & 0.9927066 & Recipient & Domestic & Domestic & Domestic & Domestic|ECO & Positive\\
\addlinespace
Poland & http://wgospodarce.pl/informacje/53368-wdrazanie-programow-unijnych-jest-zbyt-kosztowne?utm\_source=feedburner\&utm\_medium=feed\&utm\_campaign=Feed\%3A+wGospodarce+\%28wGospodarce.pl\%29\&utm\_content=FeedBurner & 345 & wgospodarce.pl & Private/Non-Public & Online only & National & high = CP is most important issue in story (can also cover other issues) & Bureaucracy and/or delays & Negative & National & No myth & NA & NA & NA & NA & NA & NA & NA & NA & Poland & wdrażanie programów unijnych jest zbyt kosztowne & 2018-09-02 & polityka spójności & wdrażanie programów unijnych powinno być prostsze i mniej kosztowne - mówi minister inwestycji i rozwoju jerzy kwieciński. dodaje, że polityka spójności, bez względu na to, jaki będzie jest budżet po 2021 roku, powinna być bardziej dostosowana do potrzeb kraju czy regionu. unia europejska wyznaczyła 11 tzw. celów tematycznych, na które można przeznaczać środki z polityki spójności. odnoszą się one do wszystkich krajów członkowskich niezależnie od różnych realiów i różnych potrzeb poszczególnych państw - mówi minister inwestycji i rozwoju jerzy kwieciński. jako przykład wskazuje walkę z bezrobociem. owszem - wskazuje - kraje południa takie jak hiszpania, portugalia czy włochy mają duży odsetek osób bez pracy, zwłaszcza młodych i bardzo potrzebne są im pieniądze na walkę z bezrobociem, ale w polsce to nie jest dziś główny problem". uważa, że więcej pieniędzy polska powinna kierować na podnoszenie kwalifikacji i zwiększenie kompetencji zasobów ludzkich. ale nie bardzo możemy, ponieważ unijne przepisy bardzo ściśle określają, na co można przyznawać środki z programów dofinansowanych przez ue - mówi kwieciński. innym przykładem braku elastyczności unijnej polityki spójności jest likwidacja w obecnej perspektywie wsparcia na rozwój transportu lotniczego. było to - jak mówi- efektem pewnej niefrasobliwości inwestycyjnej niektórych beneficjentów, ale automatycznie cięcie nie jest dobrym rozwiązaniem. dziś - zauważa - żaden z krajów wspólnoty nie dostaje dofinansowania na budowę czy modernizację portów lotniczych, a jedynym wyjątkiem są inwestycje związane z zapewnieniem bezpieczeństwa na lotniskach. dla polski zniesienie możliwości takiego wsparcia to problem, bo liczba odprawianych pasażerów jest u nas kilka razy mniejsza niż w krajach zachodnich, a potrzeby społeczne są coraz większe - dodaje. minister zaznaczył, że od wielu miesięcy toczą się rozmowy dotyczące nowej perspektywy budżetowej na lata 2021-2027, także na temat zwiększenia elastyczności w polityce spójności. zarówno bilateralne, jak i w szerszym gronie państw członkowskich. ostatnie, polsko-węgierskie rozmowy miały miejsce w poniedziałek 27 sierpnia w budapeszcie, a kolejne odbędą się w dniach 16-17 września również na węgrzech. będzie to, jak powiedział kwieciński, trzecia już edycja spotkania tzw. "przyjaciół polityki spójności". w poprzednich uczestniczyli przedstawiciele niemal wszystkich krajów ue. na pytanie jakie mogą być efekty tego spotkania, minister powiedział, że będą tam poruszane m.in. kwestie istotne dla regionu europy środkowo-wschodniej. zaznaczył, że choć spotkania mają charakter nieformalny, to wpływają znacząco np. na postrzeganie i ocenę efektów polityki spójności. jeszcze w 2016 roku, w rozmowach o kolejnej perspektywie finansowej przedstawiciele niektórych państw zastanawiali się czy polityka spójności powinna być kontynuowana, że może trzeba ją znacząco ograniczyć - opowiada minister. i pokazuje jak wiele się od tego czasu zmieniało. w tym roku, gdy przedstawiany był nowy budżet ue na lata 2021-2017, nikt nie wątpił, że polityka spójności jest potrzebna. oczywiście trwają rozmowy o tym, co powinno być w jej ramach finansowane, ale absolutnie nikt nie wątpi, że przynosi ona korzyści całej ue - wyjaśnia. i to - jego zdaniem - jest właśnie efekt cyklicznych spotkań przyjaciół spójności. pytany o to, na ile realne są zmiany w mechanizmach działania ke i polityki spójności w kolejnych latach, twierdzi, że "na rewolucję nie liczymy, ale chcemy dokonać zmian, które uważamy za najbardziej potrzebne". zauważa m.in., że problemem jest biurokracja. i dodaje, że zasady powinny zostać uproszczone po to, by kraje mniej pieniędzy wydawały na obsługę wdrażanych programów. minister nie ukrywa, że przyszłoroczne wybory do parlamentu europejskiego mogą znacząco zmienić nastawienie komisji europejskiej wobec działania różnych mechanizmów unijnych. twierdzi też, że porównując różne składy komisji europejskiej od 1990 roku, nie widział aż tak "upolitycznionego składu". & 556 & high & High & Governance & NA & NA & 2018-09-02 & 2018 & 3 & POL
Frame & high-very high & National & 500-1000 & 1.0448136 & 1.8253861 & -0.2867631 & -1.2581601 & -1.2725643 & 0.0 & -1.1973642 & 0.9927066 & Recipient & Domestic & Domestic & Domestic & Domestic|POL & Negative\\
Poland & http://forsal.pl/artykuly/1164835,morawiecki-polska-i-czechy-sa-na-tej-samej-dlugosci-fali-m-in-o-funduszach-unijnych.html & 252 & forsal.pl & Private/Non-Public & Online only & National & low = CP mentioned more times but NOT important part of story (mainly about others issues) & Territorial cooperation & Positive & EU + Other country & No myth & NA & NA & NA & NA & NA & NA & NA & NA & Poland & morawiecki: polska i czechy są "na tej samej długości fali" m.in. o funduszach unijnych & 2018-07-06 & fundusz spójności & polska i czechy są "na tej samej długości fali" m.in. w sprawie funduszy unijnych i migracji - powiedział premier mateusz morawiecki po rozmowach z szefem czeskiego rządu andrejem babiszem. babisz podkreślił, że stosunki polsko-czeskie są "ponadstandardowe". szef polskiego rządu spotkał się w piątek z babiszem w karlowych warach na marginesie odbywającego się tam międzynarodowego festiwalu filmowego. morawiecki podkreślił na wspólnej konferencji prasowej z babiszem, że oba kraje mają coraz więcej wspólnego, a polska wkroczyła na arenę międzynarodową "pośrednio dzięki czechom". poinformował, że rozmowy z premierem czech dotyczyły m.in. kwestii gospodarczych, w tym infrastruktury, wymiany handlowej i współpracy kulturalnej. "republika czeska jest dla nas jednym z kluczowych partnerów gospodarczych" - powiedział morawiecki. podkreślił, że wymiana doświadczeń polski z czechami jest bardzo owocna także w kwestii podatku vat i "eliminowaniu karuzel, mafii vat-owskich". "my mamy jednolity plik kontrolny i to jest rozwiązanie, które istnieje również w czechach, wymieniamy się bazami, czarnymi listami +przedsiębiorców+, właściwie nie przedsiębiorców, a łobuzów czy bandytów podatkowych, i dzięki temu jesteśmy coraz bardziej skuteczni" - dodał. morawiecki zaznaczył, że rozmowy dotyczyły także energetyki oraz migracji i uchodźców. "na ostatniej radze europejskiej postawa pana premiera, bardzo konkretna, jednoznaczna pozwoliła wypracować takie a nie inne konkluzje, gdzie możemy być pewni, że od tego czasu nikt też nie będzie nas zmuszał do przyjmowania uchodźców" - mówił szef polskiego rządu. premier ocenił, że relacje polsko-czeskie na bardzo wielu polach są "znakomite". "przede wszystkim współpraca, nie tylko nasza handlowa, gospodarcza, ale także ta w europie: fundusz spójności, fundusze rolnicze, no i oczywiście fundusze infrastrukturalne i wielkie problemy typu uchodźcy; jesteśmy tutaj na tej samej długości fali" - podkreślił morawiecki. czeski premier mówił, że ostatnio bardzo często spotyka się z morawieckim. przypomniał o spotkaniu państw grupy wyszehradzkiej i późniejszym posiedzeniu rady europejskiej w brukseli. "zastanawiam się, czy w ciągu ostatnich dwóch tygodni nie widziałem premiera morawieckiego częściej niż własnej żony" - żartował czeski premier. stosunki z polską babisz nazwał "ponadstandardowymi". jak ocenił, europa zaczyna zmieniać podejście do migracji przejmując stopniowo poglądy v4. zdaniem czeskiego premiera polska i czechy mają wspólny pogląd na budżet ue. "rozmawialiśmy o funduszu spójności, rolnictwie i funduszach strukturalnych" - mówił babisz. zadeklarował chęć nadrobienia przez czechy opóźnień w budowaniu dróg i autostrad. podkreślił, że te inwestycje po polskiej stronie są bardziej zaawansowane. wśród dziedzin, w których polsko-czeska współpraca będzie się rozwijać, wymienił energetykę jądrową. powiedział też, że z morawieckim rozmawiał m.in. o polskiej produkcji dwóch typów śmigłowców. "to dla nas bardzo ważne, bo republika czeska musi kupić śmigłowce" - mówił czeski premier. zwrócił uwagę, że pod względem obrotów polska stała się drugim partnerem dla czech, wyprzedzając słowację. podkreślił, że czechy są siódmym inwestorem w polsce, a polskie firmy inwestują na czeskim rynku. po rozmowach premierów polski i czech oraz delegacji obu rządów odbyło się spotkanie morawieckiego i babisza z polskimi i czeskimi przedsiębiorcami. ze strony polskiej uczestniczyli w nim: prezes paih tomasz pisula, wiceprezes polskiego funduszu rozwoju piotr fill, prezes pko bp zbigniew jagiełło, członek zarządu pkn orlen zbigniew leszczyński, wiceprezes pko bp maks kraczkowski, ekonomista pko bp piotr bujak, prezes zarządu synthos s.a. zbigniew warmuz, prezes zarządu cedrob s.a. andrzej goździkowski, prezes zarządu ab s.a. andrzej przybyło, dyrektor generalna grupy adamed małgorzata adamkiewicz, wiceprezes fakro s.a. janusz komurkiewicz, pierwszy wiceprezes banku ochrony środowiska arkadiusz garbarczyk, wiceprezes izotechnik sp.z.o.o jakub rola, producent filmowy grzegorz olkowski, prezes impel grzegorz dzik, wiceprezes mlekovita stanisław duch. morawiecki i babisz, wraz z reżyserem barrym levinsonem, wezmą udział w pokazie wersji reżyserskiej filmu "rain man" zorganizowanym w ramach festiwalu w karlowych warach. międzynarodowy festiwal filmowy w karlowych warach to jedno z najważniejszych wydarzeń filmowych w europie środkowo-wschodniej. tegoroczna, 53. edycja imprezy odbywa się w dniach 29 czerwca - 7 lipca. >>> czytaj też: "kadencyjność to rzecz święta, ale każda konstytucyjna zasada ma swoje wyjątki" [wywiad] & 625 & low & Low & Socio-Economic & NA & NA & 2018-07-06 & 2018 & 3 & ECO
Frame & low-medium & National & 500-1000 & 1.0448136 & 1.8253861 & -0.2867631 & -1.2581601 & -1.2725643 & 0.0 & -1.1973642 & 0.9927066 & Recipient & European & European & European & European|ECO & Positive\\
Poland & https://www.tvn24.pl/r/840869 & 531 & TVN24.pl & Private/Non-Public & Online and Offline & National & low = CP mentioned more times but NOT important part of story (mainly about others issues) & Solidarity to poor countries/regions & Factual & National & No myth & NA & NA & NA & NA & NA & NA & NA & NA & Poland & "odwracanie się od europy ma konsekwencje w mniejszych pieniądzach" & 2018-05-29 & polityka spójności & rząd przede wszystkim powinien mieć refleksję, na ile postępowanie prawa i sprawiedliwości miało wpływ na tak duży spadek funduszy dla polski - powiedziała w dogrywce "jeden na jeden" na tvn24.pl joanna schmidt. była posłanka nowoczesnej skomentowała propozycje komisji europejskiej w sprawie podziału środków na politykę spójności. według nieoficjalnych informacji komisja europejska ma zaproponować nową metodologię podziału pieniędzy na politykę spójności, przez co do polski trafi ponad 23 procent mniej funduszy niż obecnie. - rząd przede wszystkim powinien mieć w tym momencie refleksję, na ile postępowanie prawa i sprawiedliwości miało wpływ na tak duży spadek, bo nominalnie to jest największy spadek - podkreślała w dogrywce "jeden na jeden" na tvn24.pl joanna schmidt. była posłanka nowoczesnej, a obecnie parlamentarzystka niezrzeszona podkreśliła, że na takie decyzje "wpływa przede wszystkim polityka, jaką prowadzi rząd". - prawda jest taka, że na to wpłynęła polityka spójności i priorytety. tak, jak jeszcze niedawno mieliśmy podział na kraje zachodnie i wschodnie - i te wschodnie były bardzo mocno wspierane - tak teraz południe będzie wspierane ze względu na politykę migracyjną - wskazywała. - prawo i sprawiedliwość musi wiedzieć, że odwracanie się od europy ma konsekwencje w mniejszych pieniądzach - oceniła i dodała, że jej zdaniem polityka, którą obecnie prowadzi prawo i sprawiedliwość "powoduje, że polska jest spychana na tory boczne". schmidt w internetowej dogrywce "jeden na jeden" odniosła się też do informacji "naszego dziennika", który napisał, że według stanowiska prokuratora generalnego aborcja wykonywana ze względów eugenicznych jest niezgodna z polską konstytucją. pytana, czy jej zdaniem to oznacza, że trybunał konstytucyjny zlikwiduje tę przesłankę, odparła: - taka jest droga prawa i sprawiedliwości od początku. (...) nie będą chcieli sami wprowadzać zmian w ustawie, tylko wysłużą się trybunałem konstytucyjnym, który jako pierwszy zlikwidowali - podkreśliła. nawiązała jednocześnie do irlandii, która w referendum opowiedziała się za liberalizacją przepisów aborcyjnych. - irlandia pokazała nam drogę zmian - stwierdziła schmidt. - przecież politycy prawa i sprawiedliwości jako przykład podawali irlandię, która miała prawo zbliżone do polskiego. okazało się, że irlandczycy są bardziej liberalni, bardziej otwarci na zmianę - wskazała. według posłanki "to jest dobra droga, żeby w polsce również takie referendum przeprowadzić". jak dodała, "widzi hipokryzję w działaniu w prawa i sprawiedliwości". - najpierw zmuszają do tego, żeby urodzić dzieci bardzo chore i niepełnosprawne, a potem pozostawiają je samych sobie, nie pomagają. a my mówimy, że w związku z tym, że państwo nie pomaga wystarczająco osobom niepełnosprawnym, to dajmy kobietom wybór, czy zdecydują się na urodzenie dziecka bardzo chorego - podsumowała schmidt w dogrywce "jeden na jeden" w tvn24.pl. & 401 & low & Low & Values & NA & NA & 2018-05-29 & 2018 & 3 & ECO
Frame & low-medium & National & <500 & 1.0448136 & 1.8253861 & -0.2867631 & -1.2581601 & -1.2725643 & 0.0 & -1.1973642 & 0.9927066 & Recipient & Domestic & Domestic & Domestic & Domestic|ECO & Neutral\\
Poland & http://forsal.pl/artykuly/1406860,bank-pekao-i-ebi-zaoferuja-finansowanie-msp-w-wojewodztwie-kujawsko-pomorskim.html & 177 & forsal.pl & Private/Non-Public & Online only & National & very low = CP mentioned once & Research \& innovation & Positive & EU + National & No myth & NA & NA & NA & NA & NA & NA & NA & NA & Poland & bank pekao i ebi zaoferują finansowanie mśp w województwie kujawsko-pomorskim & 2019-04-05 & europejskie fundusze strukturalne i inwestycyjne & bank pekao i ebi zaoferują finansowanie mśp w województwie kujawsko-pomorskim warszawa, 05.04.2019 (isbnews) - bank pekao oraz europejski bank inwestycyjny (ebi) podpisały umowę na linię finansową dla małych i średnich przedsiębiorstw w województwie kujawsko-pomorskim, podało pekao. środki w wysokości 150 mln zł mają pomóc w finansowaniu projektów związanych z podniesieniem efektywności energetycznej firm. "pekao jako jedyny bank będzie udzielać kredytów dla małych i średnich przedsiębiorstw w województwie kujawsko-pomorskim na finansowanie efektywności energetycznej we współpracy z ebi. zgodnie z umową, ebi udostępni bankowi pekao środki pochodzące z regionalnego programu operacyjnego województwa kujawsko-pomorskiego na lata 2014-2020. program finansowany jest przez europejskie fundusze strukturalne i inwestycyjne" - czytamy w komunikacie. bank pekao będzie udzielał preferencyjnych niskooprocentowanych kredytów inwestycyjnych dla mśp, m.in. na realizacje takich projektów jak: przebudowa linii produkcyjnych na bardziej efektywne energetycznie, zastosowanie technologii efektywnych energetycznie, technologii użytkowania energii, modernizację energetyczną obiektów i budynków w firmach, podano również. o wsparcie nawet do 5 mln zł starać się będą mogły projekty, które zwiększają efektywność energetyczną o co najmniej 25\%. pekao zakłada, że w ciągu 5 lat obowiązywania umowy może udzielić kredytów nawet dla 100 przedsiębiorstw. podpisana właśnie umowa jest już kolejną w ramach długotrwałej współpracy banku pekao i europejskiego banku inwestycyjnego. "efektywność energetyczna jest coraz istotniejsza w rozwoju polskich firm. przemyślane inwestycje w energooszczędne technologie to już nie potrzeba, ale konieczność. dzięki umowie z ebi już wkrótce przedsiębiorcy w kujawsko-pomorskim będą mogli łatwiej finansować projekty, umożliwiające np. przebudowę pod tym kątem parku maszynowego. dzięki nowej umowie możemy zaoferować klientom kolejne preferencyjne warunki finansowania. od września 2018 nasi klienci na śląsku z powodzeniem korzystają z taniej linii kredytowej silesia ze środków ue. z naszych rozmów z klientami wynika, że takie rozwiązania są znakomitą szansą na rozwój polskich mśp" - powiedziała wiceprezes banku pekao odpowiedzialna za pion mśp magdalena zmitrowicz, cytowana w materiale. bank pekao jest częścią grupy pzu - największej grupy finansowej w europie środkowo-wschodniej. od 1998 roku bank obecny jest na giełdzie papierów wartościowych w warszawie - notowany w ramach indeksu wig20, należy do grona pięciu największych spółek polskiej giełdy. (isbnews) & 343 & very low & Low & Socio-Economic & NA & NA & 2019-04-05 & 2019 & 3 & ECO
Frame & v.low & National & <500 & 1.0448136 & 1.8253861 & -0.2867631 & -1.2581601 & -1.2725643 & 0.0 & -1.1973642 & 0.9927066 & Recipient & Domestic & European & Mixed & Domestic|ECO & Positive\\
Poland & https://www.money.pl/gospodarka/wiadomosci/artykul/komisja-europejska-uwierzyla-w-polske,99,0,2404963.html & 730 & WP money & Private/Non-Public & Online only & National & very low = CP mentioned once & Economic development & Positive & National & No myth & NA & NA & NA & NA & NA & NA & NA & NA & Poland & komisja europejska uwierzyła w polskę. prognozy mocno w górę & 2018-05-04 & fundusze strukturalne & zmiana podejścia komisji europejskiej do polski nie na każdym polu jest tak pozytywna, jak w kwestii gospodarki. prognoza wzrostu poszła w górę o aż pół punktu procentowego. w europie są tylko cztery kraje, którym ke bardziej poprawiła ocenę. komisja europejska podwyższyła prognozę tegorocznego wzrostu pkb polski. jeszcze w listopadzie ub. roku szacowała wzrost zaledwie 3,8 proc. w wiosennej prognozie (european economic forecast. spring 2018) opublikowanej 3 maja daje nam już 4,3 proc. prognoza na 2019 r. została podwyższona do 3,7 proc. z 3,4 proc. - spodziewamy się, że gospodarka polski nadal będzie mocno rosnąć przy jedynie ograniczonym spowolnieniu w 2019 roku - podaje komisja europejska. - głównymi motorami wzrostu są konsumpcja i inwestycje, wspierane przez szybki wzrost płac, silne zaufanie konsumentów i fundusze strukturalne ue - dodano. "tłit". prof. zybertowicz: pis przechodzi przyspieszony kurs wzrost inwestycji jest tym, co najbardziej zaskoczyło w ostatnim czasie urzędników komisji. prognoza nakładów na środki trwałe w 2018 skoczyła z +7,9 proc. w listopadzie ubiegłego roku do +8,7 proc. obecnie. wyraźną poprawę widać też w szacunkach konsumpcji prywatnej i salda obrotów bieżących z zagranicą. poprawa widoczna jest też w ocenie budżetu państwa. zamiast 1,7 proc. pkb deficytu instytucji rządowych i samorządowych, jak prognozowano w listopadzie, obecnie szacuje się 1,4 proc. wyraźnie spaść ma też proporcja długu publicznego do pkb - z prognozowanych wcześniej 53 do 49,6 proc. w dużym skrócie - polska ma bardzo dobre notowania w ke, jeśli chodzi o gospodarkę. według prognoz niewiele jest krajów w europie, które będą mieć większy wzrost pkb. powyżej 5 proc. wzrostu przewidywane jest dla malty i irlandii, a wyższy od polski wzrost gospodarki osiągną jeszcze słowenia, turcja i rumunia. najwolniej w europie rosnąć mają włochy i wielka brytania (oba kraje po +1,5 proc. pkb). w porównaniu z listopadem 2017 r. w majowym raporcie ke najbardziej poprawiła prognozy wzrostu pkb irlandii (o 1,8 pkt. prod. - z 3,9 do ,7 proc.) oraz malty (o 0,9 pkt. proc.), słowenii i turcji (o 0,7 pkt. proc.). polska z poprawą o 0,5 pkt. proc. jest na piątym miejscu razem z estonią. pogorszyły się prognozy dla grecji z 2,5 proc. do 1,9 proc. wzrostu pkb. według ke unia europejska ma rosnąć w wolniejszym tempie niż usa, którym prognozuje 2,9 proc. dynamiki pkb w 2018, podczas gdy 28 krajów ue ma zyskać łącznie zaledwie 2,3 proc. & 395 & very low & Low & Socio-Economic & NA & NA & 2018-05-04 & 2018 & 3 & ECO
Frame & v.low & National & <500 & 1.0448136 & 1.8253861 & -0.2867631 & -1.2581601 & -1.2725643 & 0.0 & -1.1973642 & 0.9927066 & Recipient & Domestic & Domestic & Domestic & Domestic|ECO & Positive\\
\addlinespace
Poland & http://forsal.pl/artykuly/1366234,w-ue-pracuje-239-mln-ludzi-to-wiecej-niz-kiedykolwiek-w-historii.html & 815 & forsal.pl & Private/Non-Public & Online only & National & medium = CP is important part of story & Jobs & Positive & EU & No myth & NA & NA & NA & NA & NA & NA & NA & NA & Poland & w ue pracuje 239 mln ludzi. to więcej niż kiedykolwiek w historii & 2018-11-26 & europejski fundusz społeczny & dyskusja dotyczyła europejskiego filaru praw socjalnych. to proklamowany przez unijnych przywódców zbiór 20 zasad, na których mają opierać się sprawiedliwe i sprawnie funkcjonujące rynki pracy i systemy opieki społecznej. jak podkreśliła belgijska komisarz, do europy "powraca nadzieja", której przejawem ma być sytuacja na europejskim rynku pracy. "około 12 mln miejsc pracy utworzono od początku kadencji tej komisji. obecnie w ue pracuje 239 mln ludzi, więcej niż kiedykolwiek w historii. bezrobocie wynosi 6,7 proc. - to najniższy poziom nie tylko od czasu kryzysu, ale od początku tego milenium" - wyliczała thyssen. dodała, że europejskie społeczeństwo staje się coraz bardziej aktywne zawodowo. więcej osób, w tym kobiet, pracuje i szuka pracy, a tego właśnie potrzebuje starzejące się społeczeństwo. według komisarz liczba ludzi zagrożonych biedą i wykluczeniem społecznym jest poniżej poziomu, o którym mowa w strategii 2020. "tylko w zeszłym roku 5 mln ludzi wyszło z biedy i wykluczenia społecznego" - zaznaczyła thyssen. jednocześnie zauważyła, że ponad 40 proc. ludzi pracujących w europie to osoby samozatrudnione albo zatrudnione "niestandardowo". zwróciła też uwagę na różnice w poziomie bezrobocia w krajach wspólnoty, które (według metodologii ue) wynosi ok. 3 proc. w czechach, w niemczech i w polsce, natomiast w grecji - ok. 20 proc. "występują też ogromne różnice regionalne, np. we włoszech" - podkreśliła. zdaniem belgijskiej polityk nadal jest wiele do zrobienia. jak zaznaczyła, dochody gospodarstw domowych nie nadążają za wzrostem pkb, natomiast ludzie muszą poczuć wzrost, "kiedy otrzymują wypłatę". w przeciwnym razie będzie to dla nich wyłącznie statystyka pokazywana w telewizji. wśród wyzwań dla rynku pracy wymieniła m.in. globalizację, robotyzację, zmiany klimatyczne oraz demograficzne. "dzisiaj mamy trzy osoby pracujące na jednego emeryta, w 2070 r. będzie to dwóch pracowników" - powiedziała thyssen. jak dodała, właśnie by sprostać tym wyzwaniom, ue stworzyła europejski filar praw socjalnych. przewodniczący europejskiego komitetu regionów karl-heinz lambertz powiedział, że ue musi zawsze dążyć do wzmacniania statusu swoich obywateli, tworząc godne miejsca pracy i chroniąc ich zdrowie; zapewnić, że nikt "nie pozostanie w tyle". "teraz bardziej niż kiedykolwiek potrzebujemy ambitnego budżetu ue z silną polityką spójności po roku 2020. regiony i miasta są gotowe na odnowienie europy, ale cięcia lub centralizacja funduszy ue - zwłaszcza europejskiego funduszu społecznego - powstrzymają nasze ambicje" - zaznaczył lambertz. z oficjalnych wypowiedzi przedstawicieli komisji europejskiej wynika, że redukcja polityki spójności w przyszłym wieloletnim budżecie ue ma być na poziomie 7 proc. zgodnie z propozycją ke polska jest wśród krajów, które mają dotknąć największe cięcia (23 proc.). proporcjonalnie najwięcej stracić mają węgrzy, czesi, litwini, estończycy i maltańczycy (po 24 proc.). europejski fundusz społeczny (efs), będący narzędziem polityki spójności, jest jednym z pięciu głównych funduszy unijnych. komisja europejska proponuje, aby w latach 2021-27 budżet jego następcy, europejskiego funduszu społecznego plus, wynosił 101,2 mld euro. efs+ będzie skoncentrowany na inwestycjach w ludzi i wspieraniu wdrażania europejskiego filaru praw socjalnych. "filar" został podpisany wspólnie przez parlament europejski, radę ue i komisję europejską w listopadzie ub.r. na szczycie społecznym w goeteborgu (szwecja). jego 20 zasad można uporządkować według trzech kategorii: równe szanse i dostęp do zatrudnienia, uczciwe warunki pracy oraz ochrona socjalna i integracja społeczna. & 502 & medium & Medium & Socio-Economic & NA & NA & 2018-11-26 & 2018 & 3 & ECO
Frame & low-medium & National & 500-1000 & 1.0448136 & 1.8253861 & -0.2867631 & -1.2581601 & -1.2725643 & 0.0 & -1.1973642 & 0.9927066 & Recipient & European & European & European & European|ECO & Positive\\
Poland & https://echodnia.eu/fundusze-europejskie-lubuskie-nie-marnuje-czasu-i-wydaje-pieniadze-na-dynamiczny-rozwoj-regionu/ar/13498317 & a44 & https://echodnia.eu/swietokrzyskie/ & Private/Non-Public & Online and Offline & Regional/Local & medium = CP is important part of story & Economic development & Positive & Subnational & No myth & NA & NA & NA & NA & NA & NA & NA & NA & Poland & fundusze europejskie. lubuskie nie marnuje czasu i wydaje pieniądze na dynamiczny rozwój regionu & 2018-09-17 & fundusz rozwoju regionalnego & -w gronie siedmiu regionów, które nie mają problemów z wydatkowaniem środków zarezerwowanych przez unię europejską dla potrzeb polityki spójności jest województwo lubuskie.
-„od dziesięciu lat mamy przewagę eksportu nad importem. pkb wzrosło dwukrotnie z 28 mld zł do ponad 40 mld zł. bezrobocie spadło o połowę, przeciętne wynagrodzenie wzrosło dwukrotnie. nasze wskaźniki ekonomiczne szybują” powiedziała marszałek woj. lubuskiego elżbieta anna polak. regionalny program operacyjny - lubuskie 2020 stanowi narzędzie realizacji polityki spójności na obszarze województwa lubuskiego w perspektywie finansowej ue na lata 2014 - 2020. program realizuje cele województwa określone w zaktualizowanej strategii rozwoju województwa lubuskiego 2020 z dnia 19 listopada 2012 roku, zgodnie z kluczowymi kierunkami rozwoju regionu, poprzez wdrażanie projektów współfinansowanych z europejskiego funduszu rozwoju regionalnego oraz europejskiego funduszu społecznego. celem głównym programu jest długofalowy, inteligentny i zrównoważony rozwój oraz wzrost jakości życia mieszkańców województwa lubuskiego poprzez wykorzystanie i wzmocnienie potencjałów regionu i skoncentrowane niwelowanie barier rozwojowych. na jego realizację w latach 2014-2020 zarezerwowano prawie 907 mln euro. & 159 & medium & Medium & Socio-Economic & NA & NA & 2018-09-17 & 2018 & 3 & ECO
Frame & low-medium & Regional & <500 & 1.0448136 & 1.8253861 & -0.2867631 & -1.2581601 & -1.2725643 & 0.0 & -1.1973642 & 0.9927066 & Recipient & Domestic & Domestic & Domestic & Domestic|ECO & Positive\\
Poland & http://www.gazetaprawna.pl/artykuly/1026197,waszczykowski-msz-tusk-re.html & 597 & gazetaprawna.pl & Private/Non-Public & Online and Offline & National & very low = CP mentioned once & Political leverage & Negative & National & No myth & NA & NA & NA & NA & NA & NA & NA & NA & Poland & szef msz: przegraliśmy z tym, że zasady są zmieniane w trakcie gry & 2017-03-10 & fundusze strukturalne & myśmy przegrali z tym, że zasady są zmieniane w trakcie gry - tak do ponownego wyboru donalda tuska na szefa rady europejskiej odniósł się szef msz witold waszczykowski. podkreślił, że polska chce być w unii europejskiej, ale - zaznaczył - w unii z zasadami. tusk został w czwartek ponownie wybrany na przewodniczącego rady europejskiej. w trakcie szczytu ue w brukseli odbyło się głosowanie, w którym - jak poinformowały unijne źródła - przeciw wyborowi tuska była tylko premier beata szydło, pozostali przywódcy poparli go. waszczykowski mówił w radiowej trójce, że polska domaga się przestrzegania zasad w unii europejskiej. "te warunki zostały wczoraj na naszych oczach zmieniane i łamane. pytanie jest, dlaczego tak dużej grupie państw to nie przeszkadza" - mówił szef msz. zobacz także:sejm: spór w związku z wyborem tuska na szefa rady europejskiej " "zasady są zmieniane. to jest tak, jak byśmy porównali sytuację do meczu ma boisku i co pięć minut sędzia uznawałby, że trzeba coś zmienić, np.: +a teraz przez pięć minut nie liczy się spalony+, albo +teraz przez 15 minut nie będę odgwizdywał ręki na polu karnym+. myśmy przegrali z tym, że zasady są zmieniane w trakcie gry" - stwierdził waszczykowski. szef msz zapewnił, że polska chce być w unii europejskiej, ale - jak zastrzegł - w unii z zasadami. pytany, dlaczego polska zaczęła promować kandydaturę europosła jacka saryusz-wolskiego na szefa rady europejskiej dopiero przed kilkoma dniami, szef msz powiedział że wtedy saryusz-wolski się zdecydował. waszczykowski powiedział też, że "tusk nam nie pomagał i nie będzie pomagać". "tutaj z naszego punktu widzenia, dla polskiej dyplomacji żadnej zmiany nie ma, ani na niekorzyść, ani na korzyść" - stwierdził szef msz. zobacz także:szydło: donald tusk nie był kandydatem polskiego rządu " był też pytany o to, że - według relacji dyplomatów - prezydent francji francois hollande miał w czwartek powiedzieć do premier szydło: "wy macie zasady, my mamy fundusze strukturalne". "to bardzo niesłuszna wypowiedź, pokazująca jak dotacje, fundusze strukturalne są traktowane przez część polityków, jako jakaś jałmużna" - powiedział szef msz. jak dodał, "to duża złośliwość, arogancja i nieprawda po prostu". & 331 & very low & Low & Power & NA & NA & 2017-03-10 & 2017 & 2 & POL
Frame & v.low & National & <500 & 1.0448136 & 1.8253861 & -0.2867631 & -1.2581601 & -1.2725643 & 0.0 & -1.1973642 & 0.9927066 & Recipient & Domestic & Domestic & Domestic & Domestic|POL & Negative\\
Poland & https://tvn24bis.pl/article/url/883582 & 543 & TVN24 BiS & Private/Non-Public & Online and Offline & National & low = CP mentioned more times but NOT important part of story (mainly about others issues) & Institutional bargaining over funding & Factual & EU & No myth & NA & NA & NA & NA & NA & NA & NA & NA & Poland & europosłowie przeciw cięciom w budżecie & 2018-11-14 & polityka spójności & komisja europejska odrzuca projekt budżetu włoch "źródło: tvn24 bis, reuters" parlament europejski opowiedział się przeciw zaproponowanym przez ke cięciom środków na politykę spójności i rolnictwo w przyszłym budżecie unijnym na lata 2021-2027. chce też zwiększenia wydatków na naukę i walkę z bezrobociem. przyjęta przez pe rezolucja sprzeciwiająca się cięciom jest oficjalnym stanowiskiem europarlamentu do negocjacji ostatecznego kształtu przyszłego budżetu. w głosowaniu 429 europosłów było za, 207 przeciw, a 40 wstrzymało się. europosłowie uznali, że przedstawiona wcześniej przez komisję europejską propozycja przyszłego budżetu jest punktem wyjścia do negocjacji nad jego ostatecznym kształtem, ale proponowana przez nią wartość budżetu "nie pozwoli ue wywiązać się z zobowiązań politycznych i odpowiedzieć na ważne wyzwania". pe chce również, żeby w przyszłym budżecie znalazły się większe środki na naukę i badania, wsparcie dla małych i średnich firm oraz infrastrukturę transportową. europosłowie chcą m.in., aby budżet programu badawczego "horyzont europa" wyniósł 120 mld euro, a nie, jak proponuje komisja europejska, 83,5 mld euro. opowiadają się również za podwojeniem środków na walkę z bezrobociem wśród młodzieży i potrojenie środków na program erasmus+. parlament popiera również zniesienie wszystkich rabatów i innych mechanizmów korygujących w przyszłym budżecie. posłowie domagają się w przyjętym dokumencie ustanowienia m.in. nowego podatku od osób prawnych (w tym opodatkowania dużych przedsiębiorstw w sektorze cyfrowym), dochody z systemu handlu uprawnieniami do emisji oraz podatek od plastiku. parlamentarzyści skrytykowali też kraje członkowskie za brak wystarczających postępów w negocjacjach przyszłych ram finansowych ue. oczekują, że porozumienie zostanie osiągnięte jeszcze przed wyborami do parlamentu europejskiego w maju 2019 r. współsprawozdawcami projektu rezolucji byli europosłowie: jan olbrycht (po, epl), isabelle thomas (socjaliści), gerard deprez (liberałowie) i janusz lewandowski (po, epl). długoterminowy budżet na lata 2021-2027 będzie pierwszym po brexicie, więc ue będzie już liczyć 27 państw członkowskich. z tego też powodu w brukseli i stolicach europejskich od miesięcy mówiło się o dziurze po brexicie i konieczności szukania oszczędności. plan przedstawiony 2 maja przez komisję europejską przewiduje, że w niektórych obszarach wspólnota rzeczywiście wyda mniej, ale nie brak też takich, w których wydatki poszybują w górę. polityka spójności, której polska jest obecnie największym beneficjentem, a także wspólna polityka rolna dalej będą największymi częściami unijnej kasy. z oficjalnych wypowiedzi przedstawicieli komisji europejskiej wynika jednak, że cięcie w polityce spójności ma wynieść 7 proc., natomiast w rolnictwie - około 5 proc. około 94 proc. budżetu ue trafia do obywateli, regionów, miast, rolników i przedsiębiorstw. wydatki administracyjne ue stanowią około 6 proc. całkowitych wydatków. & 403 & low & Low & Power & NA & NA & 2018-11-14 & 2018 & 3 & POL
Frame & low-medium & National & <500 & 1.0448136 & 1.8253861 & -0.2867631 & -1.2581601 & -1.2725643 & 0.0 & -1.1973642 & 0.9927066 & Recipient & European & European & European & European|POL & Neutral\\
Poland & https://dzienniklodzki.pl/sprzet-ratujacy-zycie-dzieci-i-za-85-miliona-zlotych-dzieki-unijnej-dotacji/ar/c3-13700762 & a30 & https://dzienniklodzki.pl/ & Private/Non-Public & Online and Offline & Regional/Local & medium = CP is important part of story & Public services & Positive & EU + Subnational & No myth & NA & NA & NA & NA & NA & NA & NA & NA & Poland & sprzęt ratujący życie dzieci i za 8,5 miliona złotych dzięki unijnej dotacji & 2018-11-28 & fundusze strukturalne & -8,5 mln zł kosztował sprzęt, jaki trafił do oddziałów zajmujących się ratowaniem życia i zdrowia dzieci w wojewódzkim szpitalu zespolonym w kielcach. doposażenie placówki było możliwe dzięki unijnemu dofinansowaniu – podał portal onet
-- łączna wartość projektu wynosiła prawie 8,5 mln zł, a dofinansowanie unijne było to ponad 7,1 mln zł – mówi w onecie edyta warchałowska, kierownik działu funduszy strukturalnych w wojewódzkim szpitalu zespolonym w kielcach. zakup sprzętu wsparły też władze regionu, które wydały na ten cel ponad milion złotych.
-dzięki kompleksowości leczenia noworodków, wielospecjalistycznej opiece medycznej, z zaangażowaniem specjalistów z wielu dziedzin, jak również dzięki nowoczesnym urządzeniom medycznym, umieralność okołoporodowa noworodków w świętokrzyskiem zmalała do najniższej w kraju i utrzymuje się na tak niskim poziomie kolejny rok z rzędu – zaznacza dr grażyna pazera, kierownik kliniki neonatologii w wszz w kielcach. – w ostatnich dwóch latach, w klinice nie tylko odnowiono aparaturę, ale zakupiono też urządzenia, jakich dotychczas nie było w ogóle w żadnym szpitalu w regionie, jak np.: stanowisko do zastosowania hipotermii terapeutycznej – dodaje. & 165 & medium & Medium & Socio-Economic & NA & NA & 2018-11-28 & 2018 & 3 & ECO
Frame & low-medium & Regional & <500 & 1.0448136 & 1.8253861 & -0.2867631 & -1.2581601 & -1.2725643 & 0.0 & -1.1973642 & 0.9927066 & Recipient & Domestic & European & Mixed & Domestic|ECO & Positive\\
\addlinespace
Poland & http://wgospodarce.pl/informacje/49196-nie-bedzie-drastycznej-redukcji-budzetu-ue?utm\_source=feedburner\&utm\_medium=feed\&utm\_campaign=Feed\%3A+wGospodarce+\%28wGospodarce.pl\%29\&utm\_content=FeedBurner & 378 & wgospodarce.pl & Private/Non-Public & Online only & National & medium = CP is important part of story & Institutional bargaining over funding & Positive & National & No myth & NA & NA & NA & NA & NA & NA & NA & NA & Poland & nie będzie drastycznej redukcji budżetu ue & 2018-05-01 & polityka spójności & tagi: bruksela budżet komisja europejska negocjacje polityka polityka spójności ue wspólna polityka rolna na ostatniej prostej przed zaprezentowaniem projektu budżetu przez komisję europejską polsce udało się obronić znaczną część interesów regionu, jednak to dopiero początek trudnego procesu negocjowania budżetu wieloletniego - powiedział wiceszef msz konrad szymański. z dzisiejszych nieoficjalnych informacji polskiego radia wynika, że redukcje ogółu środków przeznaczonych na fundusze spójności i polityki rolnej - wynikające m.in. z brexitu - wyniosą około 5-10 procent, tj. ok. 10 mld euro i są mniejsze niż podawano wcześniej - wtedy miało chodzić o kilkanaście procent. z kolei według informacji z ubiegłego tygodnia, polska otrzyma 8 mld mniej w polityce spójności w latach 2021-2027. pytany o komentarz do doniesień pr wiceszef msz podkreślił, że jeszcze w lutym tego roku premier mateusz morawiecki zaprezentował "racjonalną wizję budżetu ue na lata po roku 2020". niektóre nowe cele ue, szczególnie w zakresie bezpieczeństwa, kontroli migracji i granic zewnętrznych wymagają wsparcia finansowego. nie może się to odbywać przez proste przesunięcia kosztem europy środkowej - powiedział szymański. - na ostatniej prostej przed zaprezentowaniem projektu budżetu przez komisję europejską polsce udało się obronić znaczną część interesów regionu. jednak pamiętajmy, że to dopiero początek bardzo trudnego i nerwowego procesu negocjowania budżetu wieloletniego w nowych okolicznościach politycznych w ue - dodał. według szymańskiego, kraje północy są pod jeszcze większą presją na rzecz oszczędności, często ze stron sił populistycznych i "otwarcie antyeuropejskich". kraje południa wciąż trwają w sytuacji kryzysowej jeśli chodzi o bezrobocie. europa na tej mapie jest przyczółkiem stabilizacji politycznej i wzrostu gospodarczego. z tych powodów będą to najtrudniejsze negocjacje budżetowe w historii - stwierdził wiceszef msz. szymański ocenił, że premier mateusz morawiecki należy do kluczowych graczy przy stole negocjacji budżetowych. "wszyscy w europie - także ke - mają świadomość, że bez porozumienia z polską nie można przyjąć budżetu wieloletniego" - powiedział wiceszef msz. wiceminister ds. europejskich stwierdził również, że polska w debacie budżetowej należy do najaktywniejszych państw ue. "razem z szeroką koalicją państw od ponad dwóch lat przedstawiamy dobrze uzasadnione postulaty we wszystkich obszarach budżetu ue" - dodał. & 328 & medium & Medium & Power & NA & NA & 2018-05-01 & 2018 & 3 & POL
Frame & low-medium & National & <500 & 1.0448136 & 1.8253861 & -0.2867631 & -1.2581601 & -1.2725643 & 0.0 & -1.1973642 & 0.9927066 & Recipient & Domestic & Domestic & Domestic & Domestic|POL & Positive\\
Poland & http://www.rp.pl/Rzad-PiS/180229632-Morawiecki-Polska-gotowa-na-kompromis-ws-budzetu-UE.html & 498 & rp.pl & Private/Non-Public & Online only & National & low = CP mentioned more times but NOT important part of story (mainly about others issues) & Institutional bargaining over funding & Factual & National & No myth & NA & NA & NA & NA & NA & NA & NA & NA & Poland & morawiecki: polska gotowa na kompromis ws. budżetu ue & 2018-02-23 & polityka spójności & - rozpoczynamy wielkie negocjacje ws. przyszłego unijnego budżetu - powiedział szef rządu przed rozpoczynającym się nieformalnym szczytem w brukseli. - polska ma bardzo jasno określone cele. chcemy, aby te cele, które do tej pory dobrze działały dla polski, czyli wspólna polityka rolna czy polityka spójności, były kontynuowane - zaznaczył mateusz morawiecki. premier wyraził zrozumienie w stosunku do nowych propozycji, przedstawianych przez inne państwa członkowskie, takich jak polityka obronna czy wspólne działania w zakresie migracji i uchodźców i pomoc potrzebującym na miejscu. - wiele z nich bardzo nam się podoba i jesteśmy do nich pozytywnie nastawieni - stwierdził szef rządu. jednocześnie zaznaczył, że "na nowe cele muszą się znaleźć nowe środki" w unijnym budżecie. zdaniem mateusza morawieckiego, źródłem pozyskania pieniędzy może być m.in eliminacja rabatów, z których korzystają niektóre państwa członkowskie, a także wyeliminowanie luki w ściąganiu podatku vat. - w całej unii europejskiej komisja europejska określa lukę w podatku vat na kwotę 155 mld euro. to jest kwota wyższa niż całoroczny budżet ue, a więc częściowe wyeliminowanie tej luki doprowadziłoby do znaczącego wzrostu dochodów wszystkich państw członkowskich i wzrostu budżetu unii - wskazywał szef polskiego rządu. prezes rady ministrów podkreślił także, że polska "jednoznacznie opowiada się przeciwko rajom podatkowym". - w tym kontekście chcemy prowadzenia realnej polityki przez wszystkie kraje członkowskie, by wyeliminować unikanie opodatkowania przez bardzo bogate firmy - mówił. premier mateusz morawiecki podkreślił, że chciałby aby "nowy budżet unii europejskiej był dla polski jak najlepszy i odpowiadał na wyzwania przyszłości, aby można było go przeznaczyć na badania, rozwój i innowacje". jak dodał "wskazujemy, że budżet musi być oparty o zdrowy, dobry kompromis i polska jest gotowa do tego kompromisu". & 262 & low & Low & Power & NA & NA & 2018-02-23 & 2018 & 3 & POL
Frame & low-medium & National & <500 & 1.0448136 & 1.8253861 & -0.2867631 & -1.2581601 & -1.2725643 & 0.0 & -1.1973642 & 0.9927066 & Recipient & Domestic & Domestic & Domestic & Domestic|POL & Neutral\\
Poland & http://www.gazetaprawna.pl/artykuly/1106236,ue-rzad-szykuje-sie-na-bitwe-o-pieniadze.html & 256 & gazetaprawna.pl & Private/Non-Public & Online and Offline & National & medium = CP is important part of story & Institutional bargaining over funding & Factual & National & No myth & NA & NA & NA & NA & NA & NA & NA & NA & Poland & rząd szykuje się na bitwę o unijne pieniądze & 2018-02-22 & polityka spójności & premier morawiecki ma gotową strategię. mamy coś do zaoferowania komisji europejskiej dyplomacja na starcie negocjacji budżetowych głównym zagrożeniem dla polski są plany cięć w funduszach spójności i polityce rolnej. po wyjściu z ue wielkiej brytanii i pojawieniu się w związku z tym luki w unijnej kasie inne pozycje wydatkowe mają być wzmacniane kosztem tych funduszy. poznaliśmy polskie stanowisko na pierwszy etap rozmów o budżecie. - jeśli mają być nowe cele, potrzebne są nowe pieniądze. polska ma konstruktywne stanowisko: oferujemy wsparcie dla komisji europejskiej, jeśli chodzi o podniesienie budżetu europejskiego i tym samym podwyższenie wysokości składki. jesteśmy gotowi do rozmowy o tym, by płacić więcej, ale jest to uwarunkowane generalnym kompromisem w sprawie tego, na co mamy płacić - mówi konrad szymański, wiceszef msz odpowiedzialny za politykę unijną. w podobnym tonie wypowiada się wiceminister inwestycji i rozwoju paweł chorąży. - nasze otwarcie na dyskusję o podwyższeniu składki zależy od tego, co będzie w budżecie. jeśli nasze priorytety nie będą uszczuplone, jesteśmy w większym stopniu gotowi partycypować w budżecie - podkreśla wiceszef resoru rozwoju. stanowisko rządu jest jednoznaczne. to próba uniknięcia zagrożenia, jakim są zakusy na przesunięcia w budżecie ue po wyjściu wielkiej brytanii. dla polski najbardziej bolesne mogłyby być cięcia w polityce spójności, której jesteśmy największym beneficjentem. wewnętrzne dokumenty - opracowane przez dyrekcje generalne komisji europejskiej - wskazują, że brane są pod uwagę trzy scenariusze. dwa z nich oznaczają cięcia na poziomie 15 i 30 proc. plan zmniejszenia funduszy o 15 proc. według cen nominalnych (26 proc. według cen z 2011 r.) zakłada, że z dotacji mogłyby korzystać jedynie mniej rozwinięte regiony oraz kraje, których dnb (dochód narodowy brutto) na jednego mieszkańca wynosi poniżej 90 proc. unijnej średniej. takie rozwiązanie wykluczyłoby większość regionów we francji i niemczech. gdyby wprowadzono bardziej radykalne cięcie - o 30 proc. - z polityki spójności nie mogłyby korzystać także mniej rozwinięte regiony włoch i hiszpanii. w takim scenariuszu na fundusze spójności mogłyby liczyć wyłącznie kraje spełniające warunek dotyczący unijnej średniej, a więc kraje europy środkowej. trzeci scenariusz zakłada, że polityka spójności będzie dostępna - jak dotychczas - dla wszystkich regionów. fundusze zostałyby "zamrożone" według cen z 2011 r., co oznaczałoby de facto ich zwiększenie o 15 proc. w stosunku do obecnej perspektywy finansowej 2014-2020. dziś trudno przewidywać, w którą stronę potoczą się negocjacje. ważne są nie tylko sumy, lecz także kryteria przyznawania funduszy. więc szacunki, ile stracimy, są w nowym budżecie bardzo różne. czarny scenariusz nawet gdyby wielka brytania nadal była w ue i tak w kolejnym budżecie środki dla polski byłyby relatywnie mniejsze. to efekt rosnącego pkb i nadrabiania zaległości w rozwoju gospodarczym wobec innych krajów. jak zwraca uwagę wiceminister chorąży, w kolejnej perspektywie cztery lub nawet pięć polskich regionów przekroczy poziom 75 proc. unijnego pkb, co oznacza dla nich mniejszy udział w funduszach spójności. to warszawa, która została wyodrębniona jako oddzielny region statystyczny, ale także województwa: dolnośląskie, wielkopolskie, pomorskie i śląskie. jeśli do tego dojdą cięcia budżetu ue z powodu brexitu czy finansowania innych priorytetów, pieniądze dla polski mogą się okazać sporo mniejsze. - gdyby cięcia odnieść wprost do propozycji komisji, to przy obecnej wysokości naszych funduszy przy 10-proc. zmniejszeniu wydatków na spójność byłoby to ok. 8 mld euro dla nas mniej. gdyby cięto fundusze o 20 proc., wówczas ubytek wyniósłby nawet 16 mld euro mniej. żadnego scenariusza nie można wykluczyć - mówi europoseł po jan olbracht. te sumy zależą od kryteriów, jakie zostaną zastosowane do rozdziału funduszy. od jednego ze źródeł unijnych usłyszeliśmy, że polski budżet może być niższy w kolejnej perspektywie o 20-30 mld euro. to olbrzymia suma, biorąc pod uwagę, że polskie korzyści w polityce spójności i wspólnej polityce rolnej to ponad 100 mld euro na całe siedem lat. ubytek wyniósłby więc nawet 30 proc. ale takich szacunków nie potwierdzają inne źródła. w tym źródła rządowe. choć nie spotkaliśmy się jednocześnie z zaprzeczeniami. możliwe, że to element rozgrywki negocjacyjnej, pozycjonowania różnych graczy i krzyżujących się interesów. wydaje się jednak, że tak duże cięcia są wątpliwe i byłyby nie do zaakceptowania z powodów politycznych dla rządu. wariant optymistyczny o wiele bardziej prawdopodobne jest, że cięcia będą mniejsze. jak wynika z raportu konferencji peryferyjnych regionów nadmorskich europy, jeśli zaakceptowane zostałyby obecne propozycje ke dotyczące cięć w polityce spójności po 2020 r., polska straciłaby zaledwie 5 proc. dotacji. konferencja zrzeszająca 160 regionów europejskich przedstawiła w raporcie symulacje, w jaki sposób rozważane przez ke warianty oszczędności wpłynęłyby na poszczególne kraje. wynika z niego, że na cięciach najmniej straciłyby kraje europy środkowej. zarówno w przypadku redukcji o 15 proc., jak i o 30 proc. dotacje dla polski zmniejszyłyby się zaledwie o 5 proc. podobnie wygląda sytuacja innych państw w regionie - czechy i słowacja straciłyby po 3 proc., węgry i rumunia - po 4 proc. to niedużo w porównaniu z niemcami (-94 proc. w obu wariantach cięć), francją (-76 proc. przy redukcji o 15 proc., -91 proc. - o 30 proc.), hiszpanią (-92 proc. przy redukcji o 15 proc., -98 proc. - o 30 proc.) czy belgią (-85 proc. w obu wariantach). wprowadzenie mniej radykalnego wariantu zakładającego zmniejszenie funduszy o 15 proc. wykluczałoby w praktyce z polityki spójności już większość regionów niemiec i hiszpanii oraz drastycznie ograniczyło możliwość korzystania ze wsparcia przez regiony we francji i belgii. niewykluczone, że kraje członkowskie postanowią podtrzymać finansowanie polityki spójności na dotychczasowym poziomie - co uwzględnia trzeci wariant proponowany przez komisję. ten optymistyczny scenariusz zakładałby de facto wzrost funduszy spójności o 15 proc. w stosunku do cen bieżących (bądź "zamrożenie" progów budżetowych według cen z 2011 r.). można się jednak spodziewać, że taki scenariusz oznaczałby zmianę celów i warunków przyznawania funduszy, jak zapowiadane powiązanie wypłaty z unijnej kasy z przestrzeganiem zasad państwa prawa. - to, jak polityka będzie wyglądać po 2020 r., zależeć będzie od procesu negocjacji i pozycji poszczególnych państw członkowskich. nasza w ostatnim czasie nie jest najmocniejsza - mówi dgp członek komitetu regionów i marszałek województwa zachodniopomorskiego olgierd geblewicz. ⓒⓟ najgorsza opcja zakłada, że dostaniemy 30 proc. mniej środków choć jesteśmy przygotowani do negocjacji, to nasza pozycja w rozmowach z güntherem oettingerem i jeanem-claude'em junckerem nie będzie najmocniejsza źródło: dziennik gazeta prawna & 990 & medium & Medium & Power & NA & NA & 2018-02-22 & 2018 & 3 & POL
Frame & low-medium & National & 500-1000 & 1.0448136 & 1.8253861 & -0.2867631 & -1.2581601 & -1.2725643 & 0.0 & -1.1973642 & 0.9927066 & Recipient & Domestic & Domestic & Domestic & Domestic|POL & Neutral\\
Poland & http://gospodarka.dziennik.pl/news/artykuly/545612,wiceminister-rozwoju-adam-hamryszczak-polska-gospodarka-rozwoj-wskaznik.html & 43 & gospodarka.dziennik.pl & Private/Non-Public & Online only & National & very high = CP is most important issue + CP is mentioned in title/headline & Economic development & Positive & EU & No myth & Jobs & Positive & EU & No myth & NA & NA & NA & NA & Poland & wiceminister rozwoju: polska gospodarka w latach 2004-2015 urosła o prawie 59 proc. & 2017-03-22 & polityka spójności & jeśli chodzi o wpływ polityki spójności na rozwój społeczno-gospodarczy polski, to on tak naprawdę był jednoznacznie pozytywny. i to trzeba wprost podkreślić - powiedział hamryszczak podczas posiedzenia sejmowej komisji gospodarki i rozwoju. w latach 2004-2015 polska tak naprawdę, obok słowacji, osiągnęła największy skumulowany wzrost pkb wśród państw członkowskich unii europejskiej. od momentu przystąpienia do unii europejskiej polska gospodarka urosła o prawie 59 proc., podczas gdy w ue średni skumulowany wzrost osiągnął 14,8 proc. - mówił. według szacunków resortu, blisko 17-proc. średnioroczny wzrost był efektem realizacji przedsięwzięć współfinansowanych z funduszy unijnych. - inwestycje finansowane z funduszy europejskich są również czynnikiem zmian sektorowej struktury gospodarki, wyrażającej się zwłaszcza wzrostem działań sektora drugiego, czyli przemysłu i budownictwa. tutaj mamy ten wskaźnik ponad 34 proc. w 2015 roku wartości dodanej brutto, przy spadku sektora pierwszego, czyli rolnictwa - 2,8 w 2015 r. - mówił. zauważył, że zmniejszył się dystans polski do pozostałych krajów unii europejskiej. - w okresie 2004-2015 dystans pomiędzy polską a ue, tą "28", mierzony pkb na mieszkańca, wyraźnie zmalał - o 21 punktów procentowych. czyli mieliśmy taką sytuację, że 48 proc. mieliśmy w 2003 r., do 69 proc. w 2015 r., z czego prawie jedna czwarta była efektem realizacji polityki spójności - powiedział hamryszczak. - obrazuje to, w jaki sposób polityka spójności naprawdę wpływa na te działania związane z konwergencją, czyli zbliżaniem się tych regionów, krajów biedniejszych, do silniejszych - dodał. wybiegając w przyszłość, to dzięki funduszom unijnym polska ma szanse w 2020 roku osiągnąć poziom pkb per capita sięgający od 75 do 78 proc. średniej unijnej - prognozował. wiceminister wskazał, że fundusze unijne przyczyniają się do inwestycji w polsce, a największy wpływ na nie wywarły w latach 2012-2014. - wtedy te największe inwestycje zarówno przez przedsiębiorców, ale też przez sektor publiczny były realizowane. proces inwestycyjny oczywiście trwa, ale 2012-2014 to była kulminacja, kiedy to dzięki wydatkom unijnym nakłady brutto na środki trwałe były wyższe o około 30 proc., średniorocznie o dokładnie 28,7 proc. niż w sytuacji, gdyby polityka spójności nie była realizowana. czyli to jest kolejny aspekt mówiący o tym, jak ważna jest polityka spójności w rozwoju poszczególnych krajów - wyliczał. podkreślił, że stopa inwestycji w 2015 roku wyniosła 20,1 proc. pkb i była wyższa od średniej unijnej o 0,5 pkt. proc. dodał, że w roku przystąpienia polski do ue sytuacja była odwrotna. - najwyższa stopa inwestycji występowała w polsce w latach 2007-2009 - od 21,4 proc. do 23,1 proc., głównie za sprawą inwestycji sektora prywatnego, ale także z rosnącym udziałem inwestycji publicznych, w tym inwestycji sektora rządowego i samorządowego - zauważył. średnioroczna stopa inwestycji w okresie 2004-2015 jest jednak (...) niższa niż w unii europejskiej i najniższa w regionie europy środkowo-wschodniej - dodał. przypomniał, że w okresie wejścia do ue w polsce był najniższy wskaźnik zatrudnienia w całej wspólnocie. - ale do 2015 roku polska odnotowała największy wzrost tego wskaźnika, o 10,7 pkt. proc. tutaj niemal 2,4 pkt. proc. przyrostu wskaźnika zatrudnienia w okresie 2004-2015 właśnie było efektem realizacji inwestycji współfinansowanych z funduszy unijnych - poinformował. liczbę nowych miejsc pracy powstałych dzięki zaangażowaniu unijnego budżetu mr szacuje na ponad 600 tys. - to niesamowity przyrost - ocenił hamryszczak. po akcesji do ue nastąpił szybki spadek stopy bezrobocia, z ponad 19 proc. w 2004 r. wiceminister zauważył, że w 2013 r., w okresie pogorszenia koniunktury, stopa bezrobocia sięgnęła 10,3 proc., ale w latach 2014-2015 ponownie spadła, do 7,5 proc. i jest niższa niż średnia w ue o 1,9 pkt. proc. zaznaczył, że do ograniczenia bezrobocia w dużej mierze przyczyniły się fundusze unijne. przewodniczący komisji jerzy meysztowicz (n) zauważył, że słowa wiceministra hamryszczaka to "jednak troszeczkę pochwała tego okresu". z informacji (...) będzie jasny przekaz, że nie do końca była "polska w ruinie" w tym czasie - mówił meysztowicz. antoni mężydło z po dodał, że rozwój kraju z lat 2004-2015 "bardzo dobrze widać, jak się jeździ po polsce". "natomiast teraz jeszcze to bardzo dobrze brzmi" - skomentował. & 645 & very high & High & Socio-Economic & Socio-Economic & NA & 2017-03-22 & 2017 & 2 & ECO
Frame & high-very high & National & 500-1000 & 1.0448136 & 1.8253861 & -0.2867631 & -1.2581601 & -1.2725643 & 0.0 & -1.1973642 & 0.9927066 & Recipient & European & European & European & European|ECO & Positive\\
Poland & https://fakty.interia.pl/polska/news-premier-rola-polski-w-unii-europejskiej-sie-umacnia,nId,2653099 & 612 & fakty.interia.pl & Private/Non-Public & Online only & National & very low = CP mentioned once & Political leverage & Factual & National & No myth & NA & NA & NA & NA & NA & NA & NA & NA & Poland & premier: rola polski w unii europejskiej się umacnia & 2018-11-02 & polityka spójności & "rola polski w unii europejskiej umacnia się. jesteśmy realnie coraz silniejszym partnerem, zarówno ze względu na swoją siłę gospodarczą, jak i nasze ogromne sukcesy w uszczelnianiu systemu podatkowego" - powiedział premier mateusz morawiecki w wywiadzie opublikowanym w piątek na portalu tvp info. "rola polski w unii europejskiej umacnia się. jesteśmy realnie coraz silniejszym partnerem, zarówno ze względu na swoją siłę gospodarczą, jak i nasze ogromne sukcesy w uszczelnianiu systemu podatkowego. to zostało tam zauważone na bardzo wielu forach. odzyskaliśmy też w polityce zagranicznej naszą wewnętrzną sterowność - to od nas zależą kierunki polityki zagranicznej"- oświadczył morawiecki. reklama "nie jesteśmy łódeczką przywiązaną do wielkich możnych tego świata, tylko kierując się interesami polskimi i rozumiejąc realia geopolityki, prowadzimy politykę polską, polską politykę zagraniczną w najlepszym polskim interesie" - podkreślił. "zjednoczeni w różnorodności" szef rządu był też pytany, czy dzisiaj temat europy kilku prędkości wciąż funkcjonuje. "jakoś ostatnio przycichł i nie słychać o nim" - powiedział premier. morawiecki przypomniał jednocześnie hasło ue: "zjednoczeni w różnorodności". "to jest hasło całej unii europejskiej i my rozumiemy ue jako europę ojczyzn. natomiast podstawowa polityka budżetowa unii europejskiej to jest polityka spójności i my tego się bardzo mocno trzymamy, my będziemy bardzo mocno walczyć o to, żeby ten przydział środków dla krajów europy środkowej i wschodniej był jak największy, a ponieważ zatwierdzenie budżetu odbywa się na zasadzie jednomyślności, to na pewno wywalczymy bardzo dobry budżet dla polski i dla europy środkowej" - zapowiedział. zarzuty opozycji morawiecki był też pytany o zarzuty opozycji dotyczące wyprowadzania polski z ue przez rząd pis. "wręcz przeciwnie. my naszą pozycję w unii europejskiej teraz umacniamy, bo unia europejska jest unią twardej gry interesów krajów członkowskich. my nie uważamy tak jak nasi poprzednicy, że poklepywanie po plecach i ustępowanie wszystkim to jest realizacja interesów polski" - zaznaczył premier. "my podkreślamy, że unię europejską musi przede wszystkim łączyć polityka w zakresie rozwoju infrastruktury, nasza polityka trójmorza, czyli infrastruktura północ - południe, via carpatia, połączenia gazowe, uniezależnienie od rosji - to są cele twarde, cele, które my realizujemy, a to, co udało nam się osiągnąć w zakresie dobrowolności relokacji migrantów, uchodźców jest naszym wielkim sukcesem" - ocenił. & 343 & very low & Low & Power & NA & NA & 2018-11-02 & 2018 & 3 & POL
Frame & v.low & National & <500 & 1.0448136 & 1.8253861 & -0.2867631 & -1.2581601 & -1.2725643 & 0.0 & -1.1973642 & 0.9927066 & Recipient & Domestic & Domestic & Domestic & Domestic|POL & Neutral\\
\addlinespace
Poland & http://www.gazetaprawna.pl/artykuly/993882,jest-porozumienie-ws-budzetu-ue.html & 552 & gazetaprawna.pl & Private/Non-Public & Online and Offline & National & medium = CP is important part of story & Institutional bargaining over funding & Factual & EU & No myth & NA & NA & NA & NA & NA & NA & NA & NA & Poland & ue: jest porozumienie ws. budżetu. większe wydatki na bezpieczeństwo & 2016-11-17 & europejskie fundusze strukturalne i inwestycyjne & negocjatorzy z państw członkowskich, pe i ke wypracowali w czwartek porozumienie ws. przyszłorocznego budżetu ue. "28" zwiększy wydatki na bezpieczeństwo oraz radzenie sobie z kryzysem migracyjnym; priorytetem ma pozostać wspieranie konkurencyjności i wzrostu. przyszłoroczne zobowiązania ue mają wynieść 157,88 mld euro, czyli o prawie 3 mld euro mniej niż chciał tego parlament europejski. kompromis przewiduje, że środki na realne płatności mają być na poziomie 134,48 mld euro (o 2 mld euro mniej niż postulowali europosłowie). ustalone w wyniku długich negocjacji liczby są zbliżone do czerwcowej propozycji komisji europejskiej. "mierzyliśmy wysoko, dlatego proces negocjacji nie był łatwy" - mówiła na konferencji prasowej w brukseli po niemal całonocnych rozmowach wiceszefowa ke ds. budżetu kristalina georgiewa. zobacz także:kukiz'15 złożył projekt nowelizacji ustawy o broni i amunicji " aby zaradzić kryzysowi migracyjnemu i wzmocnić bezpieczeństwo, rada ue, czyli przedstawiciele państw członkowskich oraz wysłannicy europarlamentu uzgodnili wzrost zobowiązań na te cele o ponad 11 proc. w porównaniu do obecnego roku. ue będzie miała do wydania w tym dziale 5,91 mld euro. środki te pójdą m.in. na wsparcie państw członkowskich w przyjmowaniu rozdzielanych uchodźców, tworzenie centrów recepcyjnych, wsparcie działań integracyjnych, ale także sfinansowanie powrotów tych, którzy nie spełniają wymogów azylowych. z drugiej strony ue ma wzmacniać ochronę granic, podejmować działania przeciw terroryzmowi oraz przestępstwom oraz chronić krytyczną infrastrukturę. prawie połowa z przyszłorocznej unijnej kasy (74,9 mld euro w zobowiązaniach) ma służyć stymulowaniu wzrostu gospodarczego, tworzeniu miejsc pracy oraz konkurencyjności. dla przykładu: 21,3 mld euro pójdzie na program finansowania badań naukowych i innowacji horyzont 2020, erasmus +, program służący konkurencyjności przedsiębiorstw cosme czy inicjatywę infrastrukturalną łącząc europę. wzrost w tym dziale wynosi 12 proc. w porównaniu do 2016 r. 2,7 mld euro, czyli o 25 proc. więcej niż w tym roku, zabezpieczono dla europejskiego funduszu inwestycji strategicznych, z którego pochodzą gwarancje na kredyty mające pomóc wygenerować 315 mld euro inwestycji w ciągu trzech lat. dodatkowe pieniądze - 500 mln euro - zostaną, tak jak chciał parlament europejski, skierowane na wsparcie inicjatywy mającej pomagać młodym odnaleźć się na rynku pracy. budżet programu wymiany studenckiej oraz praktyk erasmus + został zwiększony dzięki naciskowi europosłów o 19 proc. - do 2,1 mld euro. wiceszefowa ke poinformowała, że zabezpieczono również środki dla młodych ludzi, by mogli otrzymać bezpłatny bilet kolejowy, który umożliwi im podróż przez całą europę. z wcześniejszych zapowiedzi wynika, że taki bilet otrzymają osoby wchodzące w pełnoletność. ogromna część unijnego budżetu, bo 53,59 mld euro, zostanie skierowana na wspieranie wyrównywania poziomu gospodarczego państw unijnych i regionów poprzez europejskie fundusze strukturalne i inwestycyjne. w części dotyczącej polityki spójności w ue przewiduje on w przyszłym roku tąpnięcie, jeśli chodzi o płatności. na ten cel zarezerwowano 37 mld euro, czyli o ponad 11 proc. mniej niż w obecnym roku. nie wynika to jednak ze zmniejszenia nakładów na ważną dla polski politykę kohezyjną, ale z wolnego rozkręcania się projektów i tym samym mniejszej liczby faktur do opłacenia. inny duży dział wydatków to środki dla rolników, dla których zarezerwowano w przyszłym roku 42,6 mld euro. uzgodniono też wynoszące 500 mln euro wsparcie dla borykających się z kryzysem producentów mleka oraz wieprzowiny. program ich wsparcia na tę kwotę został ogłoszony jeszcze w wakacje. zobacz także:martin schulz: ue ma przeciwciała na populizm i retorykę skrajnej prawicy " formalne zatwierdzenie budżetu ma nastąpić podczas posiedzenia rady ue 29 października. parlament europejski ma go przegłosować 1 grudnia. & 554 & medium & Medium & Power & NA & NA & 2016-11-17 & 2016 & 2 & POL
Frame & low-medium & National & 500-1000 & 1.0448136 & 1.8253861 & -0.2867631 & -1.2581601 & -1.2725643 & 0.0 & -1.1973642 & 0.9927066 & Recipient & European & European & European & European|POL & Neutral\\
Germany & http://www.t-online.de/nachrichten/id\_83040650/europaabgeordnete-warnen-vor-einbruch-bei-eu-foerderung.html & 231 & T-online.de & Private/Non-Public & Online only & National & medium = CP is important part of story & Institutional bargaining over funding & Negative & National + Subnational & No myth & NA & NA & NA & NA & NA & NA & NA & NA & Germany & europaabgeordnete warnen vor einbruch bei eu-förderung & 2018-01-12 & strukturfonds & die brandenburger eu-abgeordneten haben vor einem massiven einbruch der eu-förderung nach dem brexit gewarnt. mit dem austritt großbritanniens würden wahrscheinlich 10 milliarden euro jährlich fehlen, sagte die spd-europaabgeordnete susanne melior am freitag in potsdam. dies werde zu einsparungen bei der eu-förderung auch in brandenburg führen. "wir hoffen aber auch auf höhere einnahmen etwa aus der transaktionssteuer, damit die einsparungen nicht zu hoch ausfallen", erklärte melior. der cdu-europa-abgeordnete christian ehler erklärte, man müsse sich bei den strukturfonds auf eine halbierung der mittel einstellen. wenn die großen mitgliedsstaaten bereit wären, mehr einzuzahlen, könnten die folgen des brexits aber weniger problematisch werden. wie melior begrüßte er die bereitschaft der sondierer von cdu und spd in berlin, sich für die eu finanziell stärker zu engagieren. forschung und innovationen sollten stärker gefördert werden, forderte ehlers. die grünen-europaabgeordnete ska keller betonte, die eu müsse den notwendigen ausstieg aus der braunkohleverstromung mit fördermitteln abfedern, um soziale und wirtschaftliche brüche zu vermeiden. andernfalls hätten rechtspopulisten dort leichteres spiel, mahnte auch der europa-abgeordnete der linken, helmut scholz. "es muss auch in der lausitz ein sozialer rahmen bleiben, um der ausbreitung von rechtem gedankengut entgegen zu treten." & 195 & medium & Medium & Power & NA & NA & 2018-01-12 & 2018 & 3 & POL
Frame & low-medium & National & <500 & -1.0405052 & -1.0641830 & 1.2185583 & -1.5781059 & 1.4491718 & 12.0 & 1.6295860 & 1.9019065 & Payer & Domestic & Domestic & Domestic & Domestic|POL & Negative\\
Germany & http://www.faz.net/1.5431736 & 271 & Frankfurter Allgemeine & Private/Non-Public & Online and Offline & National & very low = CP mentioned once & Institutional bargaining over funding & Balanced & EU & No myth & NA & NA & NA & NA & NA & NA & NA & NA & Germany & haushaltskommissar oettinger:: eu fährt agrarsubventionen zurück & 2018-02-04 & kohäsionsfonds & auf landwirte und die regionen in europa kommen nach aussagen von eu-haushaltskommissar günther oettinger einschnitte im nächsten eu-haushalt zu. es werde keinen kahlschlag geben, sagte oettinger der zeitung "welt am sonntag". aber auch in deutschland müssten sich landwirte und regionen auf kürzungen einstellen. die eu-kommission plane, die mittel für die agrar- und kohäsionsfonds im nächsten mehrjährigen haushalt um jeweils fünf bis zehn prozent zu verringern. im frühjahr beginnen die verhandlungen über den siebenjährigen finanzrahmen der europäischen union nach 2020. es gebe bereits vorschläge, wie die kürzungen im agrarsektor gestaltet werden könnten, sagte oettinger. so werde erwogen, die direktzahlungen pro hektar fläche künftig degressiv zu gestalten. die landwirte erhielten dann ab einer bestimmten schwelle weniger finanzielle unterstützung pro hektar. auf deutschland komme insgesamt eine mehrbelastung im einstelligen milliardenbereich zu. änderungen seien auch bei den gewinnen der europäischen zentralbank (ezb) denkbar, sagte oettinger weiter. so erwäge die eu-kommission, dass künftig ein kleiner teil der gewinne, welche die ezb mit der ausgabe von banknoten mache, als eigenmittel in den haushalt fließe. & 172 & very low & Low & Power & NA & NA & 2018-02-04 & 2018 & 3 & POL
Frame & v.low & National & <500 & -1.0405052 & -1.0641830 & 1.2185583 & -1.5781059 & 1.4491718 & 12.0 & 1.6295860 & 1.9019065 & Payer & European & European & European & European|POL & Neutral\\
Germany & http://www.spiegel.de/politik/ausland/brexit-debatte-in-grossbritannien-die-maer-vom-schweizer-modell-a-1096667.html & 277 & SPIEGEL ONLINE & Private/Non-Public & Online and Offline & National & very low = CP mentioned once & Institutional bargaining over funding & Balanced & EU + Other country & No myth & NA & NA & NA & NA & NA & NA & NA & NA & Germany & brexit-debatte in großbritannien: die mär vom schweizer modell - spiegel online & 2016-06-11 & kohäsionsfonds & es ist ein standardspruch von farage. seine partei fordert wie teile der konservativen tories, großbritannien solle die europäische union verlassen. sie wollen den brexit. am 23. juni stimmen die briten darüber ab und aktuelle umfragen zeigen: es wird wohl sehr knapp. die regierung von david cameron warnt vor massiven wirtschaftlichen einbußen, sollten die bürger für den brexit stimmen. reine panikmache sei das, schimpfen seine widersacher. die schweiz und norwegen seien schließlich auch nicht in der eu - und beiden ländern gehe es hervorragend. das argument klingt zunächst interessant. könnten die briten ihren sonderstatus, über den sie ohnehin in vielen bereichen der eu bereits verfügen, noch perfektionieren und zu einer "schweiz mit atomwaffen" mutieren? also wirtschaftlich und militärisch dabei sein, aber der bürokratie und den vorschriften aus brüssel aus dem weg gehen? doch so charmant das für viele briten klingen mag: die gedankenspiele sind eine illusion. die brexit-befürworter machen ihren anhängern etwas vor. weder das norwegische noch das schweizer modell ist auch nur annähernd realistisch für großbritannien. relativ einfach lässt sich das bei norwegen zeigen. die skandinavier haben es zweimal abgelehnt, mitglied der eu zu werden. norwegen gehört seit 1994 dem europäischen wirtschaftsraum (ewr) an. dieser trat 1994 in kraft und öffnete seinen mitgliedern den binnenmarkt mit seinen 500 millionen verbrauchern. die brexit-befürworter unterschlagen beim verweis auf norwegen jedoch den preis, den oslo für die vorteile des binnenmarkts entrichtet: ewr-mitglieder müssen in den europäischen kohäsionsfonds einzahlen, mit dem soziale unterschiede in der eu ausgeglichen werden. norwegen nimmt zudem an einer reihe weiterer eu-programme teil und überweist dafür die gleichen summen wie die mitgliedstaaten. insgesamt kommen so mehr als 850 millionen euro pro jahr zusammen. berücksichtigt man noch, wie viel geld großbritannien von der eu erhält, zahlt norwegen pro bürger nahezu genau so viel wie das vereinigte königreich. wenn man bedenkt, dass aufgrund des briten-rabatts 66 prozent der zahlungen wieder auf die insel zurückfließen, zahlen die norweger im verhältnis deutlich mehr für ihre light-mitgliedschaft. "die ökonomische irrationalität ist erschreckend" dazu kommt: norwegen muss zahlreiche eu-vorschriften wie die arbeitnehmerfreizügigkeit akzeptieren, ohne mitsprache in den europäischen institutionen zu haben. sollte großbritannien also aus der eu austreten und dem ewr beitreten, müsste das land weiter zahlen, die ungeliebten vorschriften aus brüssel weiter akzeptieren - und hätte keine chance mehr, über die bedingungen für den binnenmarkt zu verhandeln. "die briten würden ihren einfluss in der eu verlieren, müssten aber alle regeln befolgen", sagt dennis snower, chef des instituts für weltwirtschaft in kiel. der us-amerikaner hat knapp 20 jahre in london gelebt und geforscht, die "ökonomische irrationalität der brexit-befürworter" nennt er "erschreckend". auch den verweis auf das schweizer modell hält snower für eine "phantasie, die völlig unrealistisch ist". gefühle statt fakten, so laute die strategie der eu-gegner. die schweiz ist kein mitglied im ewr, sondern handelte in den vergangenen rund 25 jahren mehr als 120 bilaterale abkommen mit der eu aus. so können die schweizer weitgehend vom eu-freihandel profitieren - allerdings zum beispiel nicht bei finanzdienstleistungen. das wäre für großbritannien ein großes problem, schließlich hat der finanzsektor eine tragende bedeutung für die britische volkswirtschaft. außerdem übernimmt auch die schweiz fortlaufend europäische regeln, um die kompatibilität zur eu zu gewährleisten, hat aber kein mitspracherecht. die briten können nicht alles haben der status der schweiz würde den briten wohl kaum reichen, sagt stephan breitenmoser, europarechtler an der universität basel. dazu komme, dass es sehr schwierig sei, mit der eu individuelle vereinbarungen zu treffen. "je spezifischer der einzelne bereich, umso komplizierter wird es", sagt breitenmoser. dass großbritannien also wirklich innerhalb von zwei jahren - bis ein austritt realität werden könnte - jenen status erreichen würde, den die schweiz hat, ist schwer vorstellbar. zumal die eu wenig interesse daran haben dürfte, den briten ähnliche zugeständnisse zu machen wie der kleinen schweiz. ein beispiel: die verträge der schweiz mit der eu im rahmen der arbeitnehmerfreizügigkeit sind statisch. das heißt: bis zum 1. juli 1995 haben die schweizer alle eu-regeln in diesem bereich übernommen. alles, was brüssel danach beschlossen hat, musste die schweiz nicht automatisch übernehmen, sondern konnte darüber verhandeln. derzeit ist die zukunft dieses sonderstatus allerdings ungewiss, weil die schweiz jährliche höchstzahlen für zuwanderer festlegen will. "die statischen verträge sind für die eu wegen der raschen entwicklung des rechts kaum noch sinnvoll", sagt europarechtler breitenmoser. er hält es für undenkbar, dass brüssel derartige vereinbarungen mit großbritannien schließen würde. & 723 & very low & Low & Power & NA & NA & 2016-06-11 & 2016 & 2 & POL
Frame & v.low & National & 500-1000 & -1.0405052 & -1.0641830 & 1.2185583 & -1.5781059 & 1.4491718 & 12.0 & 1.6295860 & 1.9019065 & Payer & European & European & European & European|POL & Neutral\\
Germany & https://www.welt.de/newsticker/news1/article181506718/Parlament-EU-Parlament-erzwingt-Strafverfahren-gegen-Ungarn.html & 202 & DIE WELT & Private/Non-Public & Online and Offline & National & very low = CP mentioned once & Political leverage & Negative & National + Other country & No myth & NA & NA & NA & NA & NA & NA & NA & NA & Germany & parlament: eu-parlament erzwingt strafverfahren gegen ungarn - welt & 2018-09-12 & kohäsionsfonds & mitgliedstaaten sollen sanktionen gegen orbans regierung prüfen anzeige schwere niederlage für ungarns rechtskonservativen regierungschef viktor orban: das europaparlament hat am mittwoch wegen des vorwurfs der verletzung von eu-grundwerten ein strafverfahren gegen ungarn eingeleitet, das bis zum entzug von stimmrechten auf europäischer ebene führen kann. ob es sanktionen gibt, müssen aber die eu-mitgliedstaaten entscheiden. die ungarische regierung warf dem parlament einen racheakt wegen ihres harten kurses in der flüchtlingspolitik vor. für einen ungarn-kritischen bericht der niederländischen grünen-abgeordneten judith sargentini stimmten 488 eu-abgeordnete. dagegen waren 197 vertreter, und 48 enthielten sich. damit kam die notwendige zwei-drittel-mehrheit zustande. in dem bericht werden orbans regierung zahlreiche verstöße vorgeworfen. dazu gehören angriffe auf die unabhängigkeit der justiz, die einschränkung der medienfreiheit und der rechte von minderheiten sowie ein vorgehen gegen nichtregierungsorganisationen. "die entscheidung von heute ist nichts als ein kleinlicher racheakt gegen ungarn von politikern, die für einwanderung sind", sagte der ungarische außenminister peter szijjarto in budapest. orban war am dienstag selbst ins europaparlament gekommen und hatte den eu-abgeordneten "erpressung" vorgeworfen. die abstimmung galt als test für die europäische volkspartei (evp), zu der orbans fidesz-partei gehört. der deutsche fraktionschef manfred weber (csu), der kommendes jahr das amt des eu-kommissionspräsidenten anstrebt, hatte am dienstagabend angekündigt, er werde das verfahren gegen ungarn unterstützen. wegen der spaltung der fraktion in der frage stellte er den evp-abgeordneten aber frei, wie sie am mittwoch abstimmen. 115 evp-abgeordnete stimmten nun für die einleitung des verfahrens nach artikel 7 eu-vertrag. 57 votierten dagegen und 28 enthielten sich. anders als von parlamentsvertretern zunächst angegeben ist das artikel-7-verfahren durch das votum eingeleitet. "das verfahren ist mit der entscheidung ausgelöst", hieß es aus dem eu-rat. auch ein diplomat sagte: "ja, wir sind im artikel-7-verfahren, das nicht nur durch die eu-kommission und die mitgliedstaaten eingeleitet werden kann, sondern auch durch das parlament." wann ungarn auf die tagesordnung der zuständigen eu-europaminister kommt, hängt nun vom derzeitigen eu-ratsvorsitz österreich ab. die hürden bis zu einem möglichen stimmrechtsentzug sind jedoch hoch: um über sanktionen zu entscheiden, wäre zuerst ein einstimmiger beschluss der mitgliedstaaten nötig. die eu-kommission hatte erstmals ende vergangenen jahres ein artikel-7-verfahren gegen polen ausgelöst. grund waren eine reihe umstrittener reformen des polnischen justizsystems. in dem polnischen verfahren hat ungarn bereits angekündigt, sanktionen gegen warschau mit seinem veto zu blockieren. budapest dürfte darauf setzen, dass polen sich genauso verhält. die bundesregierung nahm die entscheidung des eu-parlaments zur kenntnis, wie regierungssprecher steffen seibert sagte. konkret zu ungarn wollte er sich nicht äußern. grundsätzlich könne die eu als wertegemeinschaft aber nur funktionieren, wenn alle regierungen ihre werte achteten und verteidigten. bundesaußenminister heiko maas (spd) sagte kurz vor der abstimmung, es sei "an der zeit ein zeichen zu setzen, dass es auf diese grundwerte keine rabatte gibt". es sei gut zu verdeutlichen, "dass die eu mehr ist als eine mischung aus binnenmarkt und kohäsionsfonds". & 488 & very low & Low & Power & NA & NA & 2018-09-12 & 2018 & 3 & POL
Frame & v.low & National & <500 & -1.0405052 & -1.0641830 & 1.2185583 & -1.5781059 & 1.4491718 & 12.0 & 1.6295860 & 1.9019065 & Payer & Domestic & European & Mixed & Domestic|POL & Negative\\
\addlinespace
Germany & https://www.tagesspiegel.de/politik/betrug-mit-eu-geldern-der-fehler-liegt-im-system/23850450.html & 204 & Der Tagesspiegel & Private/Non-Public & Online and Offline & National & very low = CP mentioned once & Fraud/Corruption & Negative & EU & 7.Fraud & NA & NA & NA & NA & NA & NA & NA & NA & Germany & "der fehler liegt im system" & 2019-01-10 & kohäsionsfonds & der europäische rechnungshof hält das bestehende verfahren zur bekämpfung des betrugs bei eu-fördergeldern für unzureichend. herr parts, wie viel geld geht der eu jährlich durch betrug verloren? es gibt keine verlässliche summe. und darin liegt das problem. der europäische rechnungshof hat am donnerstag einen prüfbericht zur betrugsbekämpfung in der eu veröffentlicht. wo liegen die schwachpunkte, wenn es darum geht, effizient gegen den betrügerischen umgang mit eu-fördergeldern vorzugehen? die betrugsfälle, die entdeckt werden, stellen gewissermaßen nur die spitze des eisbergs dar. betrug ist ein verbrechen ohne opfer, deshalb bleiben viele fälle unentdeckt. das betrifft nicht nur die eu allein. aber die europäische union steht in den augen ihrer bürger in diesem punkt nicht besonders gut da. laut umfragen sind fast 70 prozent der eu-bürger der ansicht, dass die vergabe von eu-geldern in irgendeiner weise betrügerische aktivitäten zur folge hat. diese zahl ist sehr hoch. wir bewerten diese wahrnehmung der eu-bürger nicht. aber es sollte wie ein weckruf wirken, wenn die bevölkerung derart eingestellt ist. in ihrem bericht kritisieren sie, dass die eu-kommission nicht über das volle ausmaß von betrugsfällen und korruption in den mitgliedstaaten im bilde ist. wie könnte dies geändert werden? es ist nicht mehr ausreichend, wenn sich die eu auf bloße vermutungen über das gesamte ausmaß der betrügereien verlässt, wenn sie eine maßgeschneiderte bekämpfungsstrategie entwickeln will. je nach ausgabengebiet nimmt die betrügerei unterschiedliche formen und ausmaße an. ein schlüssel für ein besseres verständnis und ein effektiveres vorgehen ist die verwendung von big data. wissenschaftler sind bereits dabei, das ausmaß von korruption mit der hilfe von big data zu analysieren. es gibt mehrere finanz-datenbanken auf eu-ebene und in den mitgliedstaaten, die als informationsquelle genutzt werden können. aus dem bericht des europäischen rechnungshofs geht zudem hervor, dass es nur in 45 prozent der fälle, die von der eu-betrugsbekämpfungsbehörde olaf untersucht werden, zu einer strafverfolgung kommt. zunächst einmal: der fehler liegt nicht bei den olaf-mitarbeitern in brüssel, sondern im system. das gegenwärtige verfahren erfordert zwei aufeinanderfolgende ermittlungen. auf die ermittlungen der behörde olaf folgt eine justizermittlung in dem betroffenen mitgliedstaat. diese beiden verfahren nehmen einfach zu viel zeit in anspruch. hier geht es um den abschreckungseffekt. wenn das system zur strafverfolgung nicht vernünftig funktioniert, dann haben betrüger nichts zu befürchten. bekommt die eu immer ihr geld zurück, wenn ein betrugsfall aufgedeckt wurde? leider nein. in diesem punkt zeigen sich die eu-institutionen sogar noch schwächer als in der frage der strafverfolgung. olaf übermittelt seine empfehlungen zur eintreibung der betrügerisch verwendeten mittel an die zuständigen generaldirektionen in der eu-kommission. die generaldirektionen sind dafür verantwortlich, dass das geld wieder zurückkommt. aber im durchschnitt werden nur 15 prozent der mittel, bei denen ein betrug festgestellt wurde, wieder nach brüssel zurücküberwiesen. das ist ein sehr geringer anteil. dabei geht es um unterschiedliche eu-gelder - von den kohäsionsfonds bis zur unterstützung von drittstaaten außerhalb der eu. was ist die ursache für die geringe rücküberweisungsquote? auch hier liegt der fehler im system: die verantwortung liegt nicht in einer hand. olaf führt die ermittlungen durch und gibt nur empfehlungen ab. aber die finale entscheidung über die rückforderung der gelder wird in den generaldirektionen der eu-kommission getroffen. es gibt einige fälle, in denen die generaldirektionen einen olaf-bericht so schwach finden, dass sie vor einer entscheidung zusätzliche ermittlungsarbeit leisten müssen. aber die ergebnisse, welche die generaldirektionen abliefern, sind überraschend dürftig. mehr zum thema europäische union "kein geld in nutzloses betongold umwandeln" albert funk albrecht meier das gespräch führte albrecht meier. & 582 & very low & Low & Governance & NA & NA & 2019-01-10 & 2019 & 3 & POL
Frame & v.low & National & 500-1000 & -1.0405052 & -1.0641830 & 1.2185583 & -1.5781059 & 1.4491718 & 12.0 & 1.6295860 & 1.9019065 & Payer & European & European & European & European|POL & Negative\\
Germany & https://www.tagesschau.de/ausland/rechtsstaat-eu-diskussion-101.html & 291 & tagesschau.de & Public & Online and Offline & National & medium = CP is important part of story & Political leverage & Balanced & National + Other country & 6.Does not defend EU values (eg.gender/law/democracy) & NA & NA & NA & NA & NA & NA & NA & NA & Germany & eu-diskussion: weniger rechtsstaat - weniger geld? & 2017-11-15 & kohäsionsfonds & immer wieder beklagt die eu demokratieverstöße in polen und ungarn - immer wieder laufen ihre mahnungen ins leere. nun wird diskutiert, fördergelder an die einhaltung rechtsstaatlicher regeln zu knüpfen. der vorschlag birgt viel sprengstoff. von kai küstner, ard-studio brüssel die idee gehört zu den brisantesten, die derzeit auf eu-ebene diskutiert werden. wer die demokratischen spielregeln nicht einhält, könnte das künftig dort zu spüren bekommen, wo es besonders wehtut - im portemonnaie nämlich. heißt konkret: auf deutsches betreiben hin beginnt jetzt eine debatte darüber, ob die eu in zukunft fördergelder an die einhaltung rechtsstaatlicher regeln knüpfen könnte. sowohl ungarn als auch polen liefern sich mit der eu-kommission eine art dauerfehde über von brüssel beklagte demokratieverstöße. doch alle instrumente, mit denen die eu das problem anzugehen sucht, haben sich bislang als vergleichsweise stumpf erwiesen. nun also wird über eine härtere gangart nachgedacht. beim jüngsten eu-gipfel in brüssel fand kanzlerin angela merkel zwar, wie sie zugab, keine gelegenheit, das thema rechtsstaat anzuschneiden, was eigentlich geplant war. sie kündigte aber an: "wir werden darauf zurück kommen." noch befindet sich die diskussion in einem frühen stadium. doch die bundesregierung steht mit ihrem vorstoß, wie eu-diplomaten bestätigen, keineswegs alleine da. da im nächsten eu-haushalt ab 2021 - und um den geht es - die britischen milliarden wegfallen, zeigen sich gerade die größten einzahler sparideen gegenüber durchaus aufgeschlossen. der vorschlag birgt jedenfalls jede menge sprengstoff, nicht nur für den eu-zusammenhalt. auch müssten noch reihenweise juristische fragen geklärt werden. schließlich sind die fördergelder einst ersonnen worden, um soziale und wirtschaftliche ungleichheit innerhalb der eu abzumildern. parallel zur debatte, die auf der ebene der einzelstaaten geführt wird, macht sich auch das eu-parlament gedanken darüber, wie man den druck insbesondere auf polen aufrechterhalten kann. "die eu-kommission hat sich viel mühe gemacht, den rechtsstaatsbruch und die missachtung europäischer werte in polen rückgängig zu machen", sagt der liberale eu-abgeordnete guy verhofstadt. "aber offensichtlich ist die regierung in warschau nicht willens, eine lösung für diese verstöße zu finden." verhofstadt gehört aber auch zu jenen, die gleichzeitig warnen, dass man beim kürzen von geldern nicht die falschen treffen dürfe - sprich: die bevölkerung. im parlament gibt es derzeit auch eher bestrebungen, einen anderen sanktionsweg einzuschlagen. doch von einer entscheidung hinsichtlich des vorhabens, geld als hebel zu nutzen, ist man ohnehin noch weit entfernt. zunächst geht es den befürworten darum, die diskussion voranzutreiben - und damit auch das drohpotenzial gegenüber rechtsstaatsündern aufrecht zu erhalten. dass kürzungen bei fördergeldern, den sogenannten kohäsionsfonds, gerade polen und ungarn empfindlich treffen dürften, liegt auf der hand: warschau ist der größte empfänger von eu-hilfen, auch ungarns wirtschaftsleistung wird nicht unerheblich mit den umverteilten geldern angekurbelt. & 440 & medium & Medium & Power & NA & NA & 2017-11-15 & 2017 & 2 & POL
Frame & low-medium & National & <500 & -1.0405052 & -1.0641830 & 1.2185583 & -1.5781059 & 1.4491718 & 12.0 & 1.6295860 & 1.9019065 & Payer & Domestic & European & Mixed & Domestic|POL & Neutral\\
Germany & http://www.handelsblatt.com/politik/international/geld-gegen-fluechtlingsaufnahme-eu-staaten-wehren-sich-gegen-merkels-fluechtlingsplan/20998918.html & 285 & Handelsblatt & Private/Non-Public & Online and Offline & National & low = CP mentioned more times but NOT important part of story (mainly about others issues) & Political leverage & Balanced & EU + Other country & No myth & NA & NA & NA & NA & NA & NA & NA & NA & Germany & eu-staaten wehren sich gegen merkels flüchtlingsplan & 2018-02-23 & strukturfonds & die eu debattiert über ihren haushalt. merkel will, dass fördergelder mit der aufnahme von flüchtlingen verknüpft werden - und stößt auf massiven widerstand. brüsseldie reaktionen fielen kühl aus. am donnerstag hatte bundeskanzlerin angela merkel sich dafür ausgesprochen, die überweisung von geldern aus den eu-strukturfonds mit der aufnahme von flüchtlingen zu verknüpfen. im kreise ihrer kollegen beim treffen der eu-staats- und regierungschefs in brüssel am freitag fand die idee wenig anklang. und zwar auch nicht bei jenen staaten, die merkel dabei weniger im sinn gehabt haben dürfte als die hardliner polen, ungarn oder tschechien. "wer wird nachher bestraft? nicht die regierungen, aber die bürger", sagte der luxemburgs premierminister xavier bettel bei seinem eintreffen beim sondergipfel. am ende würden bauern oder studenten dafür bestraft, dass ihre regierung sich nicht an abmachungen gehalten habe. auch österreichs bundeskanzler sebastian kurz lehnte die idee merkels ab. eu-hilfen an bedingungen zu knüpfen, könne er zwar grundsätzlich nachvollziehen, sagte er. "ich würde nur bitten, sich da nicht nur ständig auf flüchtlinge zu fokussieren." die position des konservativen politikers ist nicht sonderlich überraschend. kurz ist kein freund der umverteilung von asylsuchenden innerhalb europas. man müsse die menschen vielmehr bereits an den außengrenzen der eu stoppen, forderte der 31-jährige. bei dem sondergipfel wollen die eu-staats- und regierungschefs erstmals den haushaltsrahmen für die jahre ab 2021 erörtern. merkel hatte am vortag in ihrer regierungserklärung gefordert, die verteilung von strukturfördergeldern solle "künftig auch das engagement vieler regionen und kommunen bei der aufnahme und integration von migranten widerspiegeln". bereits im positionspapier der bundesregierung für das treffen hatte es geheißen, dieses engagement müsse sich in den verteilungskriterien zeigen, die für den europäischen struktur- und investitionsfonds gälten. eu-diplomaten interpretierten diese aussagen, dass die flüchtlingsaufnahme mehr als anreiz denn als strikte bedingung für die fördergelder zu interpretieren sei. da in den fördertöpfen aber nur eine bestimmte summe zur verfügung steht, wäre das ergebnis aber wohl dasselbe: für jene staaten, die sich in der flüchtlingspolitik verweigern, bliebe weniger eu-geld übrig. dabei ist noch völlig offen, wie groß der eu-haushalt nach dem austritt großbritanniens im märz 2019 sein wird. experten rechnen mit einem loch von rund zehn milliarden euro. nun müssen die übrigen 27 mitgliedsstaaten nach wegen suchen, dieses zu schließen. eu-haushaltskommissar günther oettinger schlägt vor, die hälfte der summe durch einsparungen im budget aufzubringen, die andere hälfte durch höhere einnahmen. mit deutschland und frankreich haben sich bereits die beiden größten beitragszahler zu einer moderaten erhöhung ihrer überweisungen nach brüssel bereit erklärt. österreich, die niederlande, schweden und dänemark lehnen das hingegen ab. nötig sei "eine eu, die versucht, schlanker zu werden", sagte kurz. zwar sei man bereit, für neue aufgaben auch geld zur verfügung zu stellen. gleichzeitig müsse man aber an anderen stellen sparen. der niederländische ministerpräsident mark rutte forderte, durch einsparungen bei der struktur- und der agrarpolitik den nötigen spielraum für andere prioritäten zu schaffen: "wir müssen den haushalt modernisieren", sagte er. mehr mittel seien etwa für die sicherung der außengrenzen, den digitalen binnenmarkt und den kampf gegen cyber-kriminalität nötig. das treffen ist die erste gelegenheit zur aussprache auf höchster ebene zu dem thema. von echten verhandlungen könne man noch nicht sprechen, sagte ein eu-diplomat. "heute geht es darum, signale zu senden: wo die eigenen roten linien sind, wo man womöglich verhandlungsspielraum hat." merkel geht aber noch weiter. sie pocht unter anderem darauf, die strukturfondsmittel an die einhaltung rechtsstaatlicher prinzipien zu knüpfen. das zielt vor allem auf polen: gegen das land läuft derzeit wegen der justizreform der nationalkonservativen regierung ein verfahren wegen des verstoßes gegen europäische grundwerte. das europaparlament drängt darauf, die haushaltsverhandlungen vor den europawahl im nächsten frühjahr abzuschließen. es bleibt also gut ein jahr. viele regierungschefs aber wollen sich nicht unter zeitdruck setzen lassen. die aushandlung des derzeitigen finanzrahmens für den zeitraum 2014 bis 2020 hatte zweieinhalb jahre gedauert. & 635 & low & Low & Power & NA & NA & 2018-02-23 & 2018 & 3 & POL
Frame & low-medium & National & 500-1000 & -1.0405052 & -1.0641830 & 1.2185583 & -1.5781059 & 1.4491718 & 12.0 & 1.6295860 & 1.9019065 & Payer & European & European & European & European|POL & Neutral\\
Germany & http://www.welt.de/politik/ausland/article129553319/Cameron-fuehlt-sich-verlassen-und-verraten.html & 263 & DIE WELT & Private/Non-Public & Online and Offline & National & very low = CP mentioned once & Political leverage & Factual & EU + Other country & No myth & NA & NA & NA & NA & NA & NA & NA & NA & Germany & cameron fühlt sich verlassen und verraten & 2014-07-05 & strukturfonds & ratspräsident herman van rompuy wählte einen tweet, um europa noch aus dem sitzungssaal in brüssel zu informieren: "entscheidung gefallen. der europäische rat nominiert jean-claude juncker als präsidenten der nächsten eu-kommission", schrieb er. kaum länger war die mitteilung der bundeskanzlerin in ihrer abschließenden pressekonferenz: "der europäische rat hat mit qualifizierter mehrheit jean-claude juncker als vorschlag für das amt des kommissionspräsidenten dem europäischen parlament übermittelt. wir haben damit eine entscheidung gefällt. " juncker, der 59-jährige ehemalige premierminister luxemburgs, war zwar angela merkels ( cdu ) kandidat, aber dennoch zeigten andere ihre freude über die nominierung deutlicher als die bundeskanzlerin: der amtierende kommissionschef josé manuel barroso etwa, der fraktionschef der christdemokraten im europaparlament, manfred weber ( csu ), sogar die sozialisten im parlament. juncker kann nun entspannt auf seine wahl durch die abgeordneten warten, die für den 16. juli angesetzt ist: eine mehrheit ist ihm viel sicherer, als sie noch bis vor kurzem in der runde der regierungschefs war. einer kämpfte bis zum schluss. "das ist ein schlechter tag für europa", sagte großbritanniens premierminister david cameron nach dem gipfel. die entscheidung "riskiert, die position der nationalen regierungen zu untergraben". cameron war gegen juncker. "der prozess war falsch, die herangehensweise war falsch, die person war falsch", sagte er im anschluss an eine gehörige, nie da gewesene niederlage. er und der ungarische regierungschef viktor orbán waren die einzigen, die bei der abstimmung über juncker gegen ihn die hand hoben. cameron hatte das votum entgegen der sitten und gebräuche im europäischen rat verlangt. das gremium hat bislang fast immer den konsens gesucht, aber in dieser frage war der brite zu keinem zugeständnis bereit. europa brauche reformen - und wähle einen, "der wahrscheinlich jede ecke in diesem gebäude kennt", sagte er in anspielung auf junckers langjährige eu-erfahrung als premierminister luxemburgs und chef der euro-gruppe. er fühlt sich verlassen, zum teil auch verraten, waren doch kürzlich noch andere reserviert gegenüber juncker: die schweden, die niederländer, auch die kanzlerin konnte, musste cameron noch vor ein paar wochen so verstehen. die agenda könne von juncker ebenso wie von anderen umgesetzt werden, sagte sie noch beim eu-gipfel zwei tage nach der europawahl. "es ist natürlich einfacher, mit dem strom zu schwimmen, als aufzustehen und für das einzutreten, was man für richtig hält", sagte cameron, ein märtyrer britischer interessen, vom rest europas mit unverständnis betrachtet. den siegesjubel immerhin ersparten die kollegen cameron: "ich glaube, dass die entscheidung eine ist, die uns einen kommissionspräsidenten gibt, der über europäische erfahrung verfügt und auf die wünsche der einzelnen mitgliedsstaaten und des parlaments eingehen wird", sagte merkel noch über die personalie - und sonst nichts mehr. cameron warf seinen kollegen einen "schweren fehler " vor: sie hätten den althergebrachten prozess geändert, nachdem sie und niemand anderes den kommissionspräsidenten bestimmen. dass juncker ein vorschlag der christdemokratischen mehrheit im rat war, dort 26 von 28 stimmen erhielt und somit von den regierungschefs recht eindeutig nominiert wurde, erwähnte cameron nicht, aber er nahm die niederlage sportlich: "manchmal muss man eine schlacht verlieren, um den krieg zu gewinnen. " die nächste schlacht steht in wenigen wochen an. am 16. juli wird das parlament mittags in straßburg den kommissionspräsidenten wählen. am abend wollen sich die regierungschefs in brüssel erneut treffen, um über die restlichen personalfragen zu beraten: ein nachfolger für van rompuy muss gefunden werden. auch das amt der scheidenden hohen eu-vertreterin für außenpolitik, catherine ashton, muss nachbesetzt werden. und womöglich muss auch noch ein künftiger chef der euro-gruppe bestimmt werden. "ein weiterer tag im paradies", sagte cameron. diskussionen gab es auch in anderen fragen: so betonte merkel, der europäische stabilitätspakt werde nicht angetastet, während italiens premierminister matteo renzi den pakt lieber ein wenig lockerer ausgelegt sähe. noch in der nacht zwischen beiden gipfeltagen hatten die mitgereisten beamten aus den eu-hauptstädten an einem formelkompromiss für die abschlusserklärung des gipfeltreffens gefeilt. frankreich und italien fordern, öffentliche investitionen für wachstum aus der defizitberechnung herauszuhalten, berlin lehnt das ab. in der abschlusserklärung heißt es nun, dass die im stabilitätspakt eingebaute flexibilität "bestens" genutzt werden soll. was das genau heißt, wird sich zeigen. aber schon jetzt zeichnet sich ab, dass die eu-kommission italien im jahr 2015 wahrscheinlich bei der nutzung von mitteln aus den milliardenschweren strukturfonds entgegenkommen wird - renzis druck zahlt sich aus. & 700 & very low & Low & Power & NA & NA & 2014-07-05 & 2014 & 1 & POL
Frame & v.low & National & 500-1000 & -1.0405052 & -1.0641830 & 1.2185583 & -1.5781059 & 1.4491718 & 12.0 & 1.6295860 & 1.9019065 & Payer & European & European & European & European|POL & Neutral\\
Germany & http://www.sueddeutsche.de/politik/raetsel-der-woche-muss-orbn-mit-konsequenzen-von-der-eu-rechnen-1.3483962 & 253 & Süddeutsche Zeitung & Private/Non-Public & Online and Offline & National & low = CP mentioned more times but NOT important part of story (mainly about others issues) & Political leverage & Balanced & EU + Other country & No myth & NA & NA & NA & NA & NA & NA & NA & NA & Germany & muss orbán mit  konsequenzen von der eu rechnen? & 2017-04-28 & strukturfonds & die eu-komission hat wegen des hochschulgesetzes ein vertragsverletzungsverfahren gegen ungarn eingeleitet. nun stellt sich die frage, was der regierung von viktor orbán droht. die europäische kommission ist nicht ohnmächtig, wenn mitgliedstaaten eu-recht verletzen. als "hüterin der verträge" hat die eu-kommission das recht und die pflicht, die nationalstaaten zur einhaltung der regeln zu zwingen. allein 2015 hat die eu-kommission 742 vertragsverletzungsverfahren eingeleitet. in solchen verfahren geht es darum, ob staaten durch nationale vorschriften oder praktiken gegen die regeln des binnenmarktes verstoßen. kompliziert wird es, wenn es um werte wie demokratie und rechtsstaatlichkeit geht, die in artikel 2 des eu-vertrages festgeschrieben sind. ein staat, der sie verletzt, kann seine mitgliedsrechte verlieren - in der theorie. praktisch sind die hürden so hoch, dass das verfahren als fast aussichtslos gilt. im falle ungarns hat die eu-kommission einen anderen weg gewählt, das klassische vertragsverletzungsverfahren. das umstrittene ungarische hochschulgesetz verletzt ihrer ansicht nach unter anderem die dienstleistungsfreiheit im binnenmarkt. die ungarische regierung hat nun zwei monate zeit für eine stellungnahme. danach kann die eu-kommission das verfahren weitertreiben. stellt ungarn den missstand dennoch nicht ab, kann die eu-kommission vor dem europäischen gerichtshof klagen. einen urteilsspruch aus luxemburg müsste orbán umsetzen. andernfalls könnte gegen ungarn ein zwangsgeld verhängt werden. sorgen bereitet orbán eher ein druckmittel anderer art. ungarn gehört zu den größten netto-empfängern von eu-geld pro kopf. europäische mittel machten 2015 etwa 4,4 prozent des bruttoinlandsprodukts aus. zwar wäre es unmöglich, dem land einfach mittel aus strukturfonds zu kürzen. diplomaten in brüssel erinnern aber daran, dass demnächst die verhandlungen über den mehrjährigen finanzrahmen der eu für die zeit nach 2020 beginnen. & 273 & low & Low & Power & NA & NA & 2017-04-28 & 2017 & 2 & POL
Frame & low-medium & National & <500 & -1.0405052 & -1.0641830 & 1.2185583 & -1.5781059 & 1.4491718 & 12.0 & 1.6295860 & 1.9019065 & Payer & European & European & European & European|POL & Neutral\\
\addlinespace
Germany & http://www.focus.de/finanzen/videos/post-brexit-plan-des-finanzministeriums-bundesregierung-will-osteuropaeern-eu-mittel-kuerzen\_id\_7203910.html & 247 & FOCUS Online & Private/Non-Public & Online and Offline & National & high = CP is most important issue in story (can also cover other issues) & Political leverage & Negative & National & No myth & NA & NA & NA & NA & NA & NA & NA & NA & Germany & bundesregierung will osteuropäern eu-mittel kürzen - video & 2017-06-02 & kohäsionsfonds & dem eu-haushalt fehlen durch den austritt großbritanniens bis zu 13 milliarden euro jährlich. das bundesfinanzministerium hat für die bundesregierung bereits einen plan entwickelt, wie das loch gestopft werden kann. für große netto-empfänger der eu-hilfen sind das schlechte nachrichten: sie müssten sich auf weniger hilfsgelder einstellen. wie die nachrichtenagentur reuters berichtet, schlägt die bundesregierung in dem papier zur zukunft der eu-kohäsionsfonds ab 2020 vor, die auszahlung von strukturmitteln auch an rechtsstaatliche reformen zu knüpfen. dies könnte etwa polen oder ungarn treffen, die hohe summen aus dem eu-haushalt erhalten, aber nach meinung der eu-kommission und vieler mitgliedstaaten gegen grundlegende rechtsstaatliche prinzipien der eu verstoßen. der kohäsionsfonds soll ärmere und reichere länder in der eu ein wenig angleichen, um den zusammenhalt innerhalb der eu zu festigen. der fonds verfügt in der periode von 2014 bis 2020 über gut 63 milliarden euro. amazon greift deutschen tv-markt an: channels kurz vor dem start? & 155 & high & High & Power & NA & NA & 2017-06-02 & 2017 & 2 & POL
Frame & high-very high & National & <500 & -1.0405052 & -1.0641830 & 1.2185583 & -1.5781059 & 1.4491718 & 12.0 & 1.6295860 & 1.9019065 & Payer & Domestic & Domestic & Domestic & Domestic|POL & Negative\\
Germany & https://www.focus.de/perspektiven/europawahl-2019-binnenmarkt-sicherheit-subventionen-wer-profitiert-wie-stark-von-der-eu\_id\_10751292.html & 272 & FOCUS Online & Private/Non-Public & Online and Offline & National & medium = CP is important part of story & Solidarity to poor countries/regions & Factual & EU + Other country & No myth & NA & NA & NA & NA & NA & NA & NA & NA & Germany & mehr als nur nettozahler und - empfänger: wer wirklich wie stark von eu profitiert & 2019-05-23 & kohäsionsfonds & wenn es um die profiteure in der eu geht, dann ist immer schnell von nettozahlern und nettoempfängern die rede. doch tatsächlich ist die frage, wer wie stark von der union profitiert, wesentlich komplizierter. focus online gibt einen überblick. frieden, freiheit, sicherheit, garantie der menschenrechte, eindämmung sozialer ungerechtigkeit und diskriminierung, fortschritt und eine nachhaltige entwicklung der wirtschaft - all das sind ziele und werte, die die europäische union im vertrag von lissabon und in ihrer grundrechte-charta festgelegt hat. doch was bedeutet das in der realität? wie stark profitieren die einzelnen nationalstaaten und regionen wirklich von einer mitgliedschaft in der eu? alle nationalstaaten müssen geld an die eu zahlen. wie hoch die beiträge sind, ergibt sich einerseits aus einem anteil an der von den staaten erhobenen mehrwertsteuer und andererseits aus abgaben, die sich am jeweiligen bruttonationaleinkommen orientieren. die länder, die mehr an die eu zahlen als sie eu-gelder bekommen, werden als nettozahler bezeichnet. länder, die mehr gelder bekommen als sie an brüssel überweisen, werden umgekehrt als nettoempfänger bezeichnet. betrachtet man den anteil der eu-zahlungen am bruttoinlandsprodukt (bip), waren im jahr 2017 deutschland, schweden, österreich, dänemark und großbritannien die größten nettozahler. litauen, bulgarien, ungarn, griechenland und estland waren demnach die größten nettoempfänger. doch diese betrachtungsweise wird seit jahren als zu eng kritisiert. eu-haushaltskommissar günther oettinger bezeichnete die nettozahlerdebatte im februar 2018 als "zunehmend sinnentleert." die salden sind von zahlreichen faktoren beeinflusst, zudem ist die frage danach, wie ein staat von der eu-mitgliedschaft profitiert oder nicht profitiert, vielschichtiger als der reine blick auf eu-gelder. daher lohnt es, die wirtschaftliche entwicklung der länder vor dem hintergrund des europäischen binnenmarkts und der währungsunion zu betrachten. ein binnenmarkt hat grundsätzlich wesentliche und unmittelbare positive auswirkungen auf beschäftigung und wachstum. er ermöglicht es unternehmen, effizienter zu arbeiten, er schafft arbeitsplätze, und ermöglicht es menschen, auch in anderen ländern als ihrer heimat zu arbeiten. "rein ökonomisch profitiert jedes land davon, dass es teil der eu ist - als teil der zollunion und der freihandelszone, ganz unabhängig zunächst von der politischen union", sagt politikwissenschaftler andreas maurer von der universität innsbruck im gespräch mit focus online. auch habe die einführung des euro beispielsweise dazu geführt, dass für alle roaming-gebühren beim mobilfunk und die kosten für elektronik gesunken sind. der konkurrenzdruck nimmt zu, dadurch werden produkte für verbraucher in allen ländern billiger. löhne und gehälter sind vielerorts gestiegen. in allen 28 mitgliedsländern ist das reale einkommen seit 2014 gestiegen - jedoch unterschiedlich stark. deutschland liegt weit vorne - dahinter folgen frankreich, großbritannien, die niederlande und italien. die schlusslichter sind litauen, estland, lettland, malta und zypern. doch auch das gilt es, differenziert zu betrachten. pro kopf hat jeder bürger in deutschland mittlerweile ungefähr 70.000 euro mehr in der tasche als es ohne den euro der fall wäre, sagt experte maurer. "aber dabei müssen wir auch bedenken, dass vier millionen hartz-iv-empfänger davon nur träumen können - innerhalb deutschlands gibt es eine extreme kluft zwischen den gewinnern und verlierern der währungsunion. denn der kontinuierliche wirtschaftsaufschwung in deutschland ging nicht einher mit einer entsprechenden lohnentwicklung." wie sieht es im rest europas aus? die südeuropäischen länder wie spanien, italien und portugal profitieren stark von den infrastrukturmaßnahmen der eu. maurer sagt: "ohne die fördergelder der eu und ohne die tatsache, dass diese länder über große verkehrs- und telekommunikationsachsen an europa angebunden sind, hätten viele südeuropäische staaten den ausbau ihrer infrastruktur niemals finanziert." damit gehen modernisierungsschübe einher. als die länder eu-mitglied wurden, waren sie von landwirtschaft und fischerei geprägt, hatten keine wirkliche industrie. "durch die eu-mitgliedschaft haben sie sich massiv industrialisiert." heute gibt es in südeuropa im bereich der digital-ökonomie oder der künstlichen intelligenz genauso startups wie in estland oder irland - das gäbe es ohne die strukturhilfen der eu oder kostengünstige kredite der europäischen investitionsbank nicht, argumentiert politikwissenschaftler maurer. auch in osteuropa gibt es dem experten zufolge kein land, das wirtschaftlich nicht massiv von der eu-mitgliedschaft profitiert hätte. "für viele unternehmer in der westeuropäischen verarbeitenden industrie ist osteuropa die verlängerte werkbank", erklärt maurer. das habe gleichzeitig zu zerwürfnissen in deutschland und anderen ländern geführt, aber das heiße für länder wie tschechien, polen oder die baltischen republiken, dass sie sich in sehr kurzer zeit in bestimmten high-tech-bereichen wirklich zu meistern entwickelt haben. und in den ersten zehn, fünfzehn jahren nach dem eu-beitritt waren die arbeitskräfte aus osteuropa sehr viel billiger und konkurrenzfähiger als westeuropäer. keine der großen industrien in osteuropa hätte überlebt, ohne dass die eu dort investiert hätte, sagt der politikwissenschaftler. die frage, wer aus wirtschaftlicher sicht wie von der eu profitiert, ist vielschichtig. manchmal verlieren einige, weil andere gewinnen. "das problem ist ein allgemeiner fahrstuhleffekt, der mit der integration einhergeht. das heißt, es gewinnen zwar alle. aber es gewinnen eben alle von unterschiedlichen niveaus aus. die niveaus seien zwar über die integrationszeit hochgefahren. damit habe sich aber der ursprüngliche unterschied, zum beispiel bezüglich der einkommensschere zwischen gut- und schlechtverdienern, nicht verändert. "es kriegen alle ein bisschen mehr, aber das verhältnis ändert sich nicht", sagt der politikwissenschaftler. kulturwissenschaftler jürgen neyer von europa-universität viadrina in frankfurt (oder) betont im gespräch mit focus online, dass sich die die frage nach dem wirtschaftlichen erfolg nicht nur aus nationalstaatlicher sicht bewerten lässt. die eigentliche grenze verläuft für ihn nicht zwischen ländern, sondern zwischen den bevölkerungsschichten: "wir haben einen großen anteil hochqualifizierter, für die das alles wunderbar ist - für die gibt es viele neue chancen, sie können in verschiedenen ländern arbeiten, sie sprechen verschiedene sprachen, europa ist für sie ein ort großartiger neuer gelegenheiten." ganz anders stelle sich die situation für die menschen dar, die in den peripherien der deutschen großstädte, in den banlieues in frankreich oder in den italienischen vorstädten leben und häufig relativ schlecht sind. "sie leben in hochlohnländern und müssen durch die eu auf dem arbeitsmarkt mit leuten konkurrieren, die ähnlich qualifiziert sind, aber bereit sind, für viel weniger geld zu arbeiten. diese menschen geraten durch die eu unter einen ganz anderen druck", sagt neyer. doch auch innerhalb der ärmeren schichten gilt es, zu differenzieren. während geringqualifizierte in hochlohnländern abgehängt zu werden drohen, kann die selbe schicht in ärmeren ländern davon profitieren. denn wer auf der unteren seite der ärmeren gesellschaften etwa in rumänien, bulgarien, polen oder portugal lebe, für den sei europa eine chance, erklärt kulturwissenschaftler neyer. die freizügigkeit im binnenmarkt gibt ihnen nämlich zumindest die möglichkeit, in einen anderen staat zu ziehen und dort ein besseres leben zu führen oder zumindest die eigene arbeitskraft für mehr geld anzubieten als in der heimat. ein weiterer vorteil der eu-mitgliedschaft ist der zugang zu den fördertöpfen. im rahmen von fonds werden hunderttausende projekte in ganz europa mit geldern unterstützt, um eine"ausgewogenere, nachhaltigere territoriale entwicklung" zu ermöglichen. es gibt insgesamt fünf struktur- und investitionsfonds. den europäischen fonds für regionale entwicklung (efre), den europäischen sozialfonds (esf), den kohäsionsfonds, den europäischen landwirtschaftsfonds für die entwicklung des ländlichen raums (eler) und den europäischen meeres- und fischerei-fonds (emff). es ist politisch gewollt, regionen mit entwicklungsrückstand in richtung des eu-durchschnitts zu entwickeln und zu fördern. das betrifft vor allem den kohäsionsfonds. der findet nur in eu-mitgliedstaaten einsatz, deren bip unter 90 prozent des eu-durchschnitts liegt. das betrifft aktuell bulgarien, estland, griechenland, kroatien, lettland, litauen, malta, polen, portugal, rumänien, die slowakei, slowenien, die tschechische republik, ungarn und zypern. "diese staaten werden im bereich umwelt-, klima-, energie und infrastrukturpolitik massiv gefördert, um überhaupt binnenmarktfähig zu sein", sagt politikwissenschaftler maurer. die restlichen fonds stehen allen anderen ländern zur verfügung. um die berechtigung und verteilung zu regeln, ist die eu in regionale einheiten, sogenannte nuts unterteilt. so können landwirte in süd- und osteuropa genauso von agrarsubventionen profitieren wie landwirte in innsbruck, nrw, schleswig-holstein oder ostdeutschland. nur nutzen nicht alle diese potenziellen fördergelder auf dieselbe weise, wie maurer erklärt: "das hängt an der frage, wer die kapazität und den politischen willen hat, die fördergelder zu beantragen. und da gibt es einen großen unterschied. staaten wie irland und zypern haben eine verwaltung, die extrem gut ausgebildet und ausgerichtet ist auf das geschickte abfragen von fördermitteln und deren umsetzung." doch letztlich geht es in der eu natürlich nicht nur ums geld. auch fragen der politischen stabilität und sicherheit sind relevant. "alle mitgliedstaaten sind heute sicherer", sagt kulturwissenschaftler neyer. "beim thema sicherheit können wir uns auf immanuel kant berufen: demokratien zusammen stabilisieren sich, geben sich vertrauen und sicherheit, keiner muss mehr angst vor dem anderen haben." vor allem kleinstaaten wie dänemark, luxemburg und malta profitieren stark davon. sie bräuchten eigentlich keine außenpolitik mehr zu machen, sondern könnten einfach bei den großen staaten mitschwimmen und hätten eine viel größere verhandlungsmacht gegenüber dritten, sagt neyer. und klar ist auch: in europa gibt es seit mehr als 70 jahren frieden - so lange wie noch nie in der geschichte europas. & 1450 & medium & Medium & Values & NA & NA & 2019-05-23 & 2019 & 3 & ECO
Frame & low-medium & National & +1000 & -1.0405052 & -1.0641830 & 1.2185583 & -1.5781059 & 1.4491718 & 12.0 & 1.6295860 & 1.9019065 & Payer & European & European & European & European|ECO & Neutral\\
Germany & https://www.welt.de/politik/article181506232/Strafverfahren-gegen-Ungarn-Nichts-anderes-als-die-kleinliche-Rache-migrationsfreundlicher-Politiker.html & 222 & DIE WELT & Private/Non-Public & Online and Offline & National & very low = CP mentioned once & Political leverage & Negative & National + Other country & No myth & NA & NA & NA & NA & NA & NA & NA & NA & Germany & strafverfahren gegen ungarn: "nichts anderes als die kleinliche rache migrationsfreundlicher politiker" - welt & 2018-09-12 & kohäsionsfonds & ungarn könnte stimmrechte im ministerrat verlieren. nun muss sich der rat der mitgliedsländer mit dem fall befassen. anzeige nach polen muss sich auch ungarn einem sanktionsverfahren wegen gefährdung von eu-grundwerten stellen. eine zwei-drittel-mehrheit im europaparlament stimmte am mittwoch für ein rechtsstaatsverfahren, das im äußersten fall zum entzug der stimmrechte im ministerrat führen könnte. nun muss sich der rat der mitgliedsländer mit dem fall befassen. für die auslösung des verfahrens stimmten 448 abgeordnete, 197 waren dagegen, 48 enthielten sich. grundlage des votums ist ein kritischer bericht, den die grünen-abgeordnete judith sargentini im frühjahr im auftrag des parlaments erstellt hatte. unter berufung auf offizielle befunde von institutionen wie vereinte nationen, organisation für sicherheit und zusammenarbeit in europa (osze) oder europarat ging dieser mit der regierung unter dem rechtsnationalen ministerpräsidenten viktor orban hart ins gericht. es herrsche eine "systemische bedrohung der demokratie, der rechtsstaatlichkeit und der grundrechte in ungarn". der bericht verwies auf einschränkungen der meinungs-, forschungs- und versammlungsfreiheit sowie auf eine schwächung des verfassungs- und justizsystems und das vorgehen der regierung gegen nichtregierungsorganisationen. darüber hinaus werden in ihm verstöße gegen die rechte von minderheiten und flüchtlingen aufgezählt sowie korruption und interessenkonflikte kritisiert. in dem nun angenommenen papier wird ein rechtsstaatsverfahren nach artikel 7 der eu-verträge gegen ungarn gefordert. ein solches verfahren hatte die eu-kommission im dezember gegen polen gestartet. beratung und etwaige entscheidung liegen beim rat der mitgliedsstaaten, der sich nun mit beiden ländern befassen muss. ungarn übt scharfe kritik ungarns regierung kritisierte das abstimmungsergebnis scharf. "dies ist nichts anderes als die kleinliche rache migrationsfreundlicher politiker", sagte außenminister peter szijjarto am mittwoch in budapest. "ungarn und seine menschen hat man bestraft, weil sie bewiesen haben, dass die migration kein naturgegebener vorgang ist und dass man sie aufhalten kann." der bericht sei "voll mit ausgewiesenen lügen", führte szijjarto weiter aus. lesen sie auch viktor orbán "das verletzt die ehre des ungarischen volkes" in berlin wurde das votum dagegen begrüßt. deutschlands außenminister heiko maas (spd) betonte am mittwoch im bundestag, die eu sei "mehr als eine mischung aus binnenmarkt und kohäsionsfonds". auf die grundwerte gebe es keine rabatte. bundeskanzlerin angela merkel (cdu) vertrete die auffassung, die eu als wertegemeinschaft könne nur funktionieren, wenn alle die werte auch achteten, sagte regierungssprecher steffen seibert. auch frankreichs regierungssprecher benjamin griveaux zeigte sich zufrieden mit dem abstimmungsergebnis: "ich glaube, es war wichtig, heute morgen daran zu erinnern, dass man nicht auf der einen seite von den vorteilen der union profitieren kann (...), und sich auf der anderen seite über ihre regeln und grundwerte hinwegsetzt", sagte er in paris. "unsere identität steht auf dem spiel." die hürden für eventuelle strafen gegen ungarn sind allerdings hoch. der ministerrat müsste in einem nächsten schritt mit der zustimmung von vier fünfteln der mitgliedsstaaten feststellen, dass die "eindeutige gefahr einer schwerwiegenden verletzung" der eu-werte besteht. nur, wenn im anschluss der rat der eu-staaten einstimmig beschließt, dass im fall ungarn tatsächlich eine solche verletzung vorliegt, können mögliche strafen durchgesetzt werden. im extremfall verliert das land stimmrechte im ministerrat. vor jedem schritt muss aber das betroffene mitgliedsland gelegenheit bekommen, sich zu äußern. im fall polen gab es bisher nur eine anhörung. die abstimmung galt als test für die europäische volkspartei (evp), zu der orbáns fidesz-partei gehört. der deutsche fraktionschef manfred weber (csu), der kommendes jahr das amt des eu-kommissionspräsidenten anstrebt, hatte am dienstagabend angekündigt, er werde das verfahren gegen ungarn unterstützen. wegen der spaltung der fraktion in der frage stellte er den evp-abgeordneten aber frei, wie sie am mittwoch abstimmen. & 585 & very low & Low & Power & NA & NA & 2018-09-12 & 2018 & 3 & POL
Frame & v.low & National & 500-1000 & -1.0405052 & -1.0641830 & 1.2185583 & -1.5781059 & 1.4491718 & 12.0 & 1.6295860 & 1.9019065 & Payer & Domestic & European & Mixed & Domestic|POL & Negative\\
Germany & https://www.waz.de/politik/ost-laenderchefs-bei-regierungsbildung-an-den-osten-denken-id212401285.html & 221 & WAZ & Private/Non-Public & Online and Offline & Regional/Local & very low = CP mentioned once & Economic development & Positive & National + Subnational & No myth & NA & NA & NA & NA & NA & NA & NA & NA & Germany & ost-länderchefs: bei regierungsbildung an den osten denken & 2017-10-31 & kohäsionspolitik & dresden auch viele jahre nach der deutschen wiedervereinigung stehen die bundesländer im osten in vielen punkten schlechter da als die im westen. die ost-landeschefs wollen endlich aufholen - und schicken dazu einen ganzen forderungskatalog nach berlin. inhalt artikel auf einer seite lesen > ... ... vorherige seite nächste seite ost-länderchefs: bei regierungsbildung an den osten denken die ministerpräsidenten der ostdeutschen länder haben bundeskanzlerin angela merkel (cdu) aufgefordert, bei der regierungsbildung ost-interessen im blick zu behalten. sachsens scheidender ministerpräsident stanislaw tillich (cdu) schickte dazu im namen seiner amtskollegen einen brief an die kanzlerin, wie die sächsische staatskanzlei am dienstag mitteilte. die ostdeutschen länder wiesen weiterhin eine "nahezu flächendeckende strukturschwäche" auf, schrieb tillich, der den vorsitz der ministerpräsidentenkonferenz ost innehat. diese schwäche müsse überwunden werden. dafür sei ostdeutschland weiter auf finanzielle förderung angewiesen - sowohl aus deutschen töpfen als auch im rahmen der eu-kohäsionspolitik. "ein abruptes ende der strukturförderung in ostdeutschland würde die erfolge der vergangenheit gefährden", hieß es in dem schreiben. auch müsse verhindert werden, dass ostdeutschland in eine ungünstige "sandwichposition" gerate - zwischen den hoch entwickelten regionen in westdeutschland und den sehr stark von der eu geförderten gebieten in osteuropa. daneben sprachen sich die landeschefs in dem schreiben gegen einen schnellen braunkohle-ausstieg aus. an der kohleverstromung hingen in ostdeutschland zehntausende arbeitsplätze. ein abruptes ende verbiete sich "schon aus respekt vor der lebensleistung der beschäftigten", hieß es. erst wenn es für sie nachhaltige zukunftsperspektiven gebe, dürfe das ende der kohlenutzung beschlossen werden. damit der anschluss an den wirtschaftsstarken westen gelinge, müsse der osten zudem besser an das bahnnetz und den luftverkehr angeschlossen werden, forderten tillich und seine kollegen weiter. außerdem müsse die künftige regierung eine flächendeckende versorgung mit schnellem internet und mobilfunk sicherstellen. in berlin laufen derzeit sondierungsgespräche zwischen vertretern von cdu, csu, fdp und grünen über eine mögliche künftige jamaika-koalition. bei den reizthemen klima und flüchtlinge hatte es zuletzt streit gegeben. nun gelangen bei den themen arbeit, rente, pflege, sicherheit und bildung und digitales deutliche fortschritte. inhalt artikel auf einer seite lesen > ... ... vorherige seite nächste seite auch interessant & 338 & very low & Low & Socio-Economic & NA & NA & 2017-10-31 & 2017 & 2 & ECO
Frame & v.low & Regional & <500 & -1.0405052 & -1.0641830 & 1.2185583 & -1.5781059 & 1.4491718 & 12.0 & 1.6295860 & 1.9019065 & Payer & Domestic & Domestic & Domestic & Domestic|ECO & Positive\\
Germany & https://www.hna.de/lokales/frankenberg/gefaehrdet-brexit-neueeu-projekte-in-waldeck-frankenberg-9587301.html & 209 & Hessisch Niedersachsische Allgemeine & Private/Non-Public & Online and Offline & Regional/Local & very high = CP is most important issue + CP is mentioned in title/headline & Institutional bargaining over funding & Negative & Subnational & No myth & NA & NA & NA & NA & NA & NA & NA & NA & Germany & gefährdet der brexit neue eu-projekte in waldeck-frankenberg? & 2018-02-05 & kohäsionsfonds & waldeck-frankenberg. welche auswirkungen hat der brexit auf den landkreis? zunächst erschließt sich dieser zusammenhang nicht. warum sollte der austritt großbritanniens aus der eu einfluss auf nordhessen haben. es gibt viele projekte, die aus eu-fördertöpfen bezuschusst werden. diese töpfe könnten durch den brexit schrumpfen. um die reduzierung von eu-mitteln zu verhindern und eine flächendeckende förderung zu erhalten, haben sich die kommunalen spitzenverbände und der ausschuss der regionen nun für den erhalt der europäischen regionalpolitik ausgesprochen. der deutsche städtetag, der deutsche landkreistag und der deutsche städte- und gemeindebund fordern, die förderperiode auch nach dem jahr 2020 zu erhalten. "durch den austritt des vereinigten königreiches aus der europäischen union muss künftig mit kürzungen im haushalt der eu gerechnet werden, die gerade in wirtschaftlich stärker entwickelten mitgliedstaaten dazu führen könnten, dass die zur verfügung stehenden mittel deutlich reduziert werden", argumentieren die kommunalen spitzenverbände. sie unterzeichneten deshalb eine grundsatzerklärung zur kohäsionspolitik. das ist die politik, die hunderttausende projekte in europa unterstützt, die mittel aus dem europäischen fonds für regionale entwicklung (efre), dem europäischen sozialfonds (esf) und dem kohäsionsfonds erhalten. darunter fällt auch das europäische leader-programm, mit dem seit 1991 modellhaft innovative aktionen im ländlichen raum gefördert werden (siehe hintergrund). aktionsgruppen erarbeiten vor ort entwicklungskonzepte. ziel ist es, die ländlichen regionen zu einer eigenständigen entwicklung zu unterstützen. in waldeck-frankenberg gibt es drei leader-regionen, in denen geförderte projekte umgesetzt werden: burgwald-ederbergland, kellerwald-edersee und diemelsee-nordwaldeck. jede leader-region erhält für den förderzeitraum von 2014 bis 2020 ein planungskontingent von circa zwei millionen euro, berichtet regionalmanagerin lisa küpper aus der region kellerwald-edersee. das sagen die regionalmanager es sei noch vieles unklar, sagen stefanie koch und bernd wecker vom planungsbüro bioline aus lichtenfels, das sich um das regionalmanagement in der förderregion diemelsee-nordwaldeck kümmert. die oben genannte auswirkung des brexits auf die europäische förderung der dorf- und regionalentwicklung in der region sei eine mögliche konsequenz. "aber die betonung liegt auf möglich", sagen koch und wecker. "aktuell ist hier noch sehr vieles unklar. da der europäische finanzrahmen bis 2020 festgesetzt ist, sind kurzfristig keine änderungen zu erwarten." aktuell werde diskutiert, ob der brexit dazu führe, dass die finanzielle ausstattung des europäischen landwirtschaftsfonds für die entwicklung des ländlichen raums (eler) zurückgefahren wird. dies könne dazu beitragen, dass die leader-förderung in der region zurückgefahren wird. "möglich ist aber auch, dass es von landesseite eine neue strategische ausrichtung zur ländlichen entwicklung gibt", sagen die beiden regionalmanager. nach ihren informationen werde ein erster entwurf für den neuen eu-haushalt von 2021 bis 2026 im mai 2018 behandelt. "vermutlich wissen wir dann mehr und können auswirkungen auf die region realistisch benennen." so lange will sich auch stefan schulte, regionalmanager für burgwald-ederbergland, "nicht an möglichen spekulationen beteiligen", sagt er. das wichtigste für die förderregionen ist: "die finanzierung für die laufende förderperiode 2014 bis 2020 ist gesichert", sagt regionalmanagerin lisa küpper. "welche auswirkungen der brexit auf die anschlussförderperiode in hessen haben wird, kann aktuell nicht beurteilt werden." küpper weist zudem darauf hin, "dass das land hessen und der bund die ländlichen räume über die eu-förderung hinaus mit zusätzlichen förderprogrammen unterstützen". & 515 & very high & High & Power & NA & NA & 2018-02-05 & 2018 & 3 & POL
Frame & high-very high & Regional & 500-1000 & -1.0405052 & -1.0641830 & 1.2185583 & -1.5781059 & 1.4491718 & 12.0 & 1.6295860 & 1.9019065 & Payer & Domestic & Domestic & Domestic & Domestic|POL & Negative\\
\addlinespace
Germany & http://www.faz.net/aktuell/politik/europaeische-union/eu-gipfel-brexit-a-la-bratislava-14436798.html & 224 & Frankfurter Allgemeine & Private/Non-Public & Online and Offline & National & very low = CP mentioned once & Institutional bargaining over funding & Factual & EU + Other country & No myth & NA & NA & NA & NA & NA & NA & NA & NA & Germany & eu-gipfel: brexit à la bratislava & 2016-09-16 & strukturfonds & binnenmarkt ohne freizügigkeit im austausch für verteidigungskooperation: wenn an diesem freitag der rat der 27 tagt, dann könnte der grundstein für ein zukunftsweisendes abkommen mit den briten gelegt werden. ein gastbeitrag. © reuters in bratislava nicht dabei, trotzdem in einer hauptrolle: die britische premierministerin, theresa may © reuters in bratislava nicht dabei, trotzdem in einer hauptrolle: die britische premierministerin, theresa may "lassen sie mich heute eine prognose wagen: das vereinigte königreich wird nicht das letzte land sein, das die europäische union verlässt." im überschwang des gewonnenen referendums prophezeite nigel farage der eu im straßburger parlament die nahende implosion. doch vom befürchteten oder - je nach sichtweise - ersehnten domino-effekt kann bislang keine rede sein. das vermeintliche häufchen elend eu strafft sich. die berliner spatzen pfeifen es gerade von den dächern: unter den staaten des alten kerneuropa, mit deutschland, frankreich und italien an der spitze, schält sich graduell ein neuer, alter konsens heraus. nach sechzig jahren der sogenannten "monnet-methode", der europäischen verständigung durch marktintegration, soll der frieden auf dem kontinent künftig auch durch eine echte verteidigungsunion abgesichert werden. es wäre dies eine rückbesinnung auf die europäische verteidigungsgemeinschaft, die 1954 in der französischen nationalversammlung gescheitert war. für den nahenden brexit würde sich dadurch eine ganz neue perspektive eröffnen: im austausch für eine vertiefte verteidigungskooperation könnten die briten am binnenmarkt auch ohne die dazugehörige arbeitnehmerfreizügigkeit teilnehmen. © privat christian freudlsperger ist co-präsident des berliner grassroots-thinktanks polis180. © privat christian freudlsperger ist co-präsident des berliner grassroots-thinktanks polis180. anders als von farage im juni erwartet, herrscht aufseiten der briten derweil kaum euphorie, dafür umso mehr konfusion und defätismus. die premierministerin wiederholt mantrahaft, sie werde den eu-austritt ihres landes zum erfolg machen. nur wie? die debatte auf der insel kreist dabei vor allem um die wirtschaftliche frage, ob und wie eine teilnahme am binnenmarkt organisiert werden kann. sie konzentriert sich auf drei optionen. da wäre das wto-modell, also der zugang zum zoll des jeweils meistbegünstigten. das problem dabei ist, dass das vereinigte königreich nicht die vereinigten staaten sind. knapp die hälfte ihres handels verzeichnen britische unternehmen mit der eu. hinzu kommt, dass ein großer teil der ausländischen direktinvestitionen in der vergangenheit gerade deshalb im land getätigt wurden, weil dort für den gesamten europäischen markt produziert werden konnte. das hatte farage den nissan-beschäftigten in sunderland natürlich vergessen mitzuteilen. eine weitere teilnahme am binnenmarkt wäre zutiefst im britischen interesse. doch zu welchen bedingungen? die norweger, die am binnenmarkt und seinen grundfreiheiten teilnehmen und zur einzahlung in die europäischen strukturfonds verpflichtet sind, ohne in brüssel mitreden zu können, scheiden als vorbild aus. geeigneter wäre da aus britischer sicht das eidgenössische modell der bilateralen verträge, die in langwierigen verhandlungen auf das künftige europäisch-britische verhältnis maßgeschneidert würden. natürlich stets unter der maßgabe, dass eine teilnahme am binnenmarkt nicht mit der arbeitnehmerfreizügigkeit verbunden wäre. das scheint die rote linie zu sein, hinter die theresa may nicht zurücktreten kann. mehr zum thema europa ist die lösung: steinmeier fordert eine mutige vision rede zur lage der eu: juncker will milliarden-programm gegen arbeitslosigkeit in europa tusks brandbrief vor dem eu-gipfel unter verweis auf das britische handelsdefizit mit der eu meint schatzkanzler hammond, die eu werde sich schon schlussendlich für letztere option entscheiden. unklar bleibt dabei, wieso die eu, aus sorge um französische autoexporte oder deutsche maschinenausfuhren, ihren eigenen fortbestand gefährden sollte. sobald es außerhalb der gemeinschaft den jeweils vorteilhafteren deal gäbe, wäre weiteren austrittsinteressenten tür und tor geöffnet. vor allem aber fällt einmal mehr die wirtschaftliche engführung des britischen europadiskurses ins auge. aus britischer sicht dient die eu alleine dem wohlstands-, aus kontinentaleuropäischer perspektive auch dem friedenserhalt. so gesehen allerdings stellt der brexit nicht nur eine wirtschaftspolitische, sondern auch eine geostrategische herausforderung dar. und als solche sollte ihm auch begegnet werden. das vereinigte königreich ist eben nicht norwegen oder die schweiz. 1 | 2 nächste seite | artikel auf einer seite & 640 & very low & Low & Power & NA & NA & 2016-09-16 & 2016 & 2 & POL
Frame & v.low & National & 500-1000 & -1.0405052 & -1.0641830 & 1.2185583 & -1.5781059 & 1.4491718 & 12.0 & 1.6295860 & 1.9019065 & Payer & European & European & European & European|POL & Neutral\\
Germany & http://www.pnp.de/lokales/stadt\_und\_landkreis\_passau/passau\_land/2903354\_Einig-ueber-Antrag-fuer-Rudertinger-Volksschule.html & 214 & Passauer Neue Presse & Private/Non-Public & Online and Offline & Regional/Local & very low = CP mentioned once & Environment/green/low-carbon & Positive & Subnational & No myth & NA & NA & NA & NA & NA & NA & NA & NA & Germany & einig über antrag für rudertinger volksschule & 2018-04-09 & europäischer fonds für regionale entwicklung & hausaufgaben gemacht hat der bau-, umwelt- und verkehrsausschuss des gemeinderates ruderting (landkreis passau) in der volksschule. die mitglieder waren sich einig, dem ratsplenum nahezulegen, den förderantrag für das neue "kommunalinvestitionsprogramm schulinfrastruktur", kurz kip-s genannt, zu stellen. "wir brauchen eine priorisierung", mahnte bürgermeister rudolf müller zur eile, weil die antragsfrist bei der regierung von niederbayern am 27. april ausläuft. nicht unerwähnt ließ der bürgermeister, dass die grundschule ruderting zugleich für das leuchtturm-projekt efre (europäischer fonds für regionale entwicklung) mit eu-mitteln in höhe von insgesamt rund 495 millionen euro für investitionen in bayern zur verringerung der co2-emissionen angemeldet ist. ausklammern aus dem neuerlichen förderantrag wollte der bürgermeister allerdings den bereich digitalisierung, um sich nicht selbst die chancen dafür in einem anderen programm zu blockieren. architekt willi neumeier aus tittling, der pläne mitgebracht hatte, ging ins detail. für die grundschule erachtete er den brandschutz als "sicher großes thema", das an erster stelle stehen müsse: "da geht's ums leben." erste entwürfe dazu gehen mit kostenschätzungen von 175000 euro einher, wie es in der runde hieß. - bp welche weiteren sanierungsmaßnahmen an der schule geplant sind, lesen sie in der dienstagsausgabe der pnp (landkreis passau). & 194 & very low & Low & Socio-Economic & NA & NA & 2018-04-09 & 2018 & 3 & ECO
Frame & v.low & Regional & <500 & -1.0405052 & -1.0641830 & 1.2185583 & -1.5781059 & 1.4491718 & 12.0 & 1.6295860 & 1.9019065 & Payer & Domestic & Domestic & Domestic & Domestic|ECO & Positive\\
Germany & http://www.handelsblatt.com/politik/international/analyse-wenn-macron-scheitert-koennte-die-eu-scheitern/21183486.html & 233 & Handelsblatt & Private/Non-Public & Online and Offline & National & very low = CP mentioned once & Political leverage & Factual & National & No myth & NA & NA & NA & NA & NA & NA & NA & NA & Germany & wenn macron scheitert, könnte die eu scheitern & 2018-04-17 & kohäsionsfonds & macron hält rede im europäischen parlament frankreichs präsident emmanuel macron (foto: dpa) straßburgemmanuel macron wendet sich direkt an seinen wichtigsten politischen partner: "ich schlage deutschland eine neue partnerschaft vor", sagte der französische staatspräsident. "wir werden nicht oder nicht sofort über alles einig sein, aber wir werden über alles diskutieren". das war im september 2017, als macron seine erste große europapolitische rede in der pariser universität sorbonne hielt. nun hat macron seine zweite große ansprache zur zukunft europas gehalten - im europaparlament in straßburg. dieses mal nahm er das wort deutschland in 20 minuten redezeit kein einziges mal in den mund. trotzdem enthielt seine rede wichtige botschaften an die adresse der regierung in berlin. frankreichs präsident hat deutschland gewarnt - und zwar vor europapolitischer untätigkeit. "bloß nichts überstürzen, alles auf die lange bank schieben - das wäre gleichbedeutend mit lähmung", sagte der mann, der mit einem europapolitischen integrationsprogramm in den wahlkampf zog und gewann. damals verhinderte macron, dass der zweitgrößte staat der eu in die hand der rechtsextremen marine le pen fiel - und darüber war die erleichterung insbesondere in deutschland sehr groß. doch dann musste macron feststellen, dass seine europapolitischen vorstellungen im rest der eu auf wenig gegenliebe stoßen. europa lässt den französischen präsidenten auflaufen. polen, ungarn, tschechien, österreich, großbritannien, italien und die niederlande erleben eine renaissance der nation. deshalb ruhen die hoffnungen macrons vor allem auf deutschland. seine mahnung dürfte sich vor allem an jene mitglieder der cdu/csu-fraktion richten, die jegliche integrationsschritte in der europäischen währungsunion verhindern wollen. dass die bundeskanzlerin reformen der euro-zone gegen wachsende widerstände in den eigenen reihen durchsetzen muss, scheint macron durchaus klar zu sein. seine eigenen ambitionen auf diesem gebiet fuhr der präsident deshalb bereits deutlich zurück. in straßburg sprach er zwar noch von fortschritten in der bankenunion, die bis zur europawahl im mai 2019 erreicht werden müssten. doch konkreter wurde er dabei nicht. die in deutschland verhasste eu-einlagensicherung blieb unerwähnt. macron wiederholte zwar seine forderung nach einer fiskalkapazität für die euro-zone, die eine art gemeinsamer finanztopf sein soll. doch auch hier nannte er keinerlei details. das eröffnet spielraum für kompromisse mit deutschland. macron zeigte sich als pragmatiker, der sich zwar grundsätzlich treu bleibt, den europäischen partnern aber zugleich entgegenkommt. zum einen bot er explizit an, den französischen beitrag zum eu-haushalt zu erhöhen. das dürfte in den mittel- und osteuropäischen staaten gut ankommen, die kürzungen der milliardenschweren strukturfonds aus dem brüsseler budget fürchten. zum anderen schlug macron vor, die versorgung von flüchtlingen künftig mit eu-haushaltsmitteln zu fördern. das müsste italien und griechenland gut gefallen, und auch deutschland. die kanzlerin hatte vor dem letzten eu-gipfel gefordert, dass regionen, die viele migranten aufnehmen, mehr geld aus den eu-kohäsionsfonds bekommen sollen. davon würden auch deutsche bundesländer profitieren - und macron hat diese idee nun unterstützt. der französische präsident hält an der europapolitischen linie fest, die er bereits im wahlkampf eingeschlagen hatte. er vertritt sie aber inzwischen defensiver. manche idee, die er noch im herbst in der sorbonne vorgetragen hatte, ist inzwischen verschwunden. das gilt für die europäische armee ebenso wie für die eu-innovationsagentur. macron ist in der realität des jahres 2018 angekommen. er begreift, dass nicht alle europäischen blütenträume durchsetzbar sind in einer zeit, in der nationalpopulisten wahlerfolge feiern. völlig ausbremsen will sich frankreichs präsident aber nicht lassen - zu recht. er ist der letzte, der die eu noch tatkräftig vorantreibt. wenn macron scheitert, könnte die eu scheitern. die wichtigsten neuigkeiten jeden morgen in ihrem posteingang. & 573 & very low & Low & Power & NA & NA & 2018-04-17 & 2018 & 3 & POL
Frame & v.low & National & 500-1000 & -1.0405052 & -1.0641830 & 1.2185583 & -1.5781059 & 1.4491718 & 12.0 & 1.6295860 & 1.9019065 & Payer & Domestic & Domestic & Domestic & Domestic|POL & Neutral\\
Germany & https://www.tagesspiegel.de/politik/europaeische-union-kein-geld-in-nutzloses-betongold-umwandeln/23801962.html & 279 & Der Tagesspiegel & Private/Non-Public & Online and Offline & Regional/Local & low = CP mentioned more times but NOT important part of story (mainly about others issues) & Mismanagement & Balanced & EU + Other country & No myth & Fraud/Corruption & Balanced & EU + Other country & No myth & NA & NA & NA & NA & Germany & "kein geld in nutzloses betongold umwandeln" & 2018-12-27 & kohäsionsfonds & der präsident des europäischen rechnungshofs, klaus-heiner lehne, über die verschwendung von eu-geldern und den kontrollverlust gegenüber der ezb. herr lehne, 2019 wird ein wichtiges jahr für europa. die briten treten aus, im mai wird ein neues europaparlament gewählt. dann kommt ein neuer haushaltsplan, der bis 2027 gelten soll. der europäische rechnungshof hat die aufgabe, die ausgaben zu kontrollieren und damit auch verschwendung von eu-geldern zu vermeiden. wie funktioniert das in der praxis? wir prüfen in erster linie die arbeit der eu-kommission. bei der vergabe von eu-fördergeldern hat sich der europäische rechnungshof in der vergangenheit sehr stark auf die frage konzentriert, ob die regularien eingehalten wurden - etwa die vorschriften bei der ausschreibung. in diesem bereich ist die eu in den vergangenen zwei jahrzehnten gewaltig vorangekommen: in der zeit, als die kommission wegen unregelmäßigkeitsvorwürfen 1999 zurücktreten musste, gab es bei der mittelvergabe prozentuale fehlerquoten im zweistelligen bereich. in der aktuellen eu-haushaltsperiode sind wir über den gesamten haushalt bei geschätzten 2,4 prozent angelangt. daher stehen für uns inzwischen andere fragen im vordergrund: ergibt ein projekt planerisch einen sinn? steht ein tragfähiges geschäftsmodell dahinter? können sie beispiele nennen? es stellt sich etwa die frage, ob es sinn ergibt, erhebliche summen in den ausbau des brenner-basistunnels zu stecken, wenn die anschlüsse in bayern und italien möglicherweise nicht rechtzeitig gebaut werden. wir erstellen gegenwärtig einen bericht zu den von der eu geförderten transeuropäischen verkehrsnetzen, mit denen die verbindungen im binnenmarkt verbessert werden sollen. welche folgen hat es, wenn sich der fokus ihrer arbeit auf die frage verlagert, welche eu-projekte überhaupt sinnvoll sind? in diesem jahr haben wir rund 50 berichte zu einzelnen eu-politikbereichen erstellt. der neue schwerpunkt in unserer arbeit führt dazu, dass die fachpolitiker in den betroffenen ausschüssen im europaparlament plötzlich interesse für unsere arbeit zeigen. ich weiß ja selbst aus meiner erfahrung als parlamentarier, dass man als abgeordneter seinen ganzen ehrgeiz daransetzt, ein bestimmtes gesetz zu verabschieden. aus meiner heutigen warte füge ich aber hinzu: die arbeit eines parlamentariers darf nicht beendet sein, wenn das gesetz verabschiedet ist. die umsetzung und die kontrolle der beteiligten verwaltungen ist genauso wichtig. derzeit verhandelt die eu-kommission mit den mitgliedstaaten über die verteilung der gelder in der nächsten finanzperiode zwischen 2021 und 2027. wie lautet ihr ratschlag an die verhandler? um es zugespitzt zu sagen: es darf keine flächendeckende subventionierung geben, und es darf auch kein geld in nutzloses betongold umgewandelt werden. unser besonderes augenmerk gilt der agrarpolitik in der nächsten haushaltsperiode. eigentlich hat sich die eu das ziel gesetzt, dörfliche strukturen zu fördern sowie die umwelt und das klima zu erhalten. dazu passt aber nicht, dass 80 prozent der landwirtschaftlichen fördermittel in die agrarindustrie fließen. dagegen bräuchten kleinere und mittlere betriebe in den mittelgebirgsregionen sehr viel mehr unterstützung. und wenn die großbetriebe dann umweltschäden produzieren, dann gibt es wieder andere eu-programme zur behebung dieser schäden. das ergibt doch keinen sinn. und kommt ihre botschaft bei den verantwortlichen an? in der agrarpolitik hat man es häufig mit verantwortlichen zu tun, die sich seit jahrzehnten mit ihrem fachbereich beschäftigen. entsprechend groß ist auch das beharrungsvermögen. die bereitschaft zu wirklich fundamentalen reformen wird nicht aus dem agrarsektor selbst kommen. der druck von außen muss wohl noch größer werden, damit sich etwas bewegt. aber wir können unseren beitrag dazu leisten, dass die mittel weniger einseitig auf die großbetriebe konzentriert werden. wie denn? machen wir uns nichts vor: die eu-kommission wird sich bei den etatverhandlungen voraussichtlich nicht komplett mit ihrer forderung durchsetzen, dass der nächste eu-haushaltsrahmen 1,1 prozent der europäischen wirtschaftsleistung umfassen soll. wahrscheinlich wird man in den verhandlungen irgendwo unterhalb dieser marke landen. wenn die eu aber gleichzeitig neue aufgaben wie die sicherung der außengrenzen ausbauen will, dann muss es umschichtungen zu lasten der größten blöcke im etat geben. und das sind nun einmal die agrarförderung und die kohäsionsfonds, also die wichtigsten strukturhilfen. reichen denn umschichtungen? der gesamte eu-haushalt umfasst derzeit 140 milliarden euro pro jahr. das klingt nach viel, ist aber tatsächlich eher wenig. daher ist das geld nur wirksam angelegt, wenn wir uns auf bestimmte ziele konzentrieren und nicht nach dem gießkannenprinzip ausschütten. es braucht mehr fokussierung auf wirklich wichtige aufgaben. fallen eigentlich bestimmte mitgliedsländer besonders auf, wenn der rechnungshof die verwendung von mitteln prüft? unsere normalen prüfungen sind zwangsläufig stichproben. und aus einzelnen fällen sollte man keine allgemeinen rückschlüsse ziehen. bei 1000 prüfungen in 28 mitgliedstaaten ist die statistische basis für eine einzelbewertung der mitgliedstaaten zu klein. aber meine allgemeine einschätzung ist, dass kein mitgliedsland negativ besonders auffällt. also auch die osteuropäischen länder nicht, die erst seit 2004 dabei sind? kommen die schon klar mit der verwendung von eu-mitteln? gerade ungarn hat ja recht viele vertragsverletzungsverfahren und auch mehrere ermittlungsverfahren der eu-sonderbehörde gegen korruption am hals. ich kann nur aus sicht des haushaltskontrolleurs sprechen, denn für betrugsverdachtsfälle sind wir nicht zuständig. und aus unserer sicht muss ich sagen, dass die osteuropäer nicht besser und nicht schlechter abschneiden bei der mittelverwendung als die anderen. es gibt natürlich regionale unterschiede, wenn es um die effizienz geht oder die fähigkeit, eigene kofinanzierungsmittel bereitzustellen. aber das hängt auch damit zusammen, dass die eu-programme eben zu wenig konzentriert sind. in deutschland hat der bund ja derzeit das problem, dass mittel für kommunale unterstützungsprogramme nicht im geplanten maß abfließen. mehr zum thema macrons rede im bundestag macron eine bühne zu bieten, reicht nicht aus albrecht meier das haben wir in der eu auch. immer wieder werden mittel nicht abgerufen. eben weil es an der verlangten mitfinanzierung hapert, oder weil planungen nicht vorankommen und verwaltungen nicht hinterherkommen. die mittel sind gebunden und müssen zu einem späteren zeitpunkt abgewickelt werden. neuerdings sogar drei jahre über die haushaltsperiode hinaus. das ist aber ein echtes problem geworden. denn diese noch abzuwickelnden mittelbindungen liegen mittlerweile bei nahezu dem doppelten des eigentlichen haushaltsvolumens. da sind also gewaltige beträge aufgelaufen. die kommission will sie nun abbauen. aber ich bin nicht sicher, dass das parlament und der europäische rat das mitmachen. & 998 & low & Low & Governance & Governance & NA & 2018-12-27 & 2018 & 3 & POL
Frame & low-medium & Regional & 500-1000 & -1.0405052 & -1.0641830 & 1.2185583 & -1.5781059 & 1.4491718 & 12.0 & 1.6295860 & 1.9019065 & Payer & European & European & European & European|POL & Neutral\\
Germany & https://www.sueddeutsche.de/news/politik/parteien---duesseldorf-laschet-kritisiert-merkels-europapolitik-zu-zoegerlich-dpa.urn-newsml-dpa-com-20090101-181116-99-845752 & 276 & Süddeutsche Zeitung & Private/Non-Public & Online and Offline & National & very low = CP mentioned once & Institutional bargaining over funding & Positive & EU + National & 2.Rich countries pay & NA & NA & NA & NA & NA & NA & NA & NA & Germany & laschet kritisiert merkels europapolitik: & 2018-11-16 & kohäsionsfonds & düsseldorf (dpa/lnw) - nordrhein-westfalens ministerpräsident armin laschet hat der bundesregierung und kanzlerin angela merkel (cdu) zu wenig einsatz in der europapolitik vorgeworfen. "die deutsche politik ist zu zögerlich", sagte der cdu-politiker dem "spiegel" (samstag). bei der europawahl im kommenden jahr würden möglicherweise "populisten von links und rechts große erfolge feiern", betonte laschet. "dem kann man nur begegnen, wenn man zeigt, dass europa handlungsfähig und entscheidungsfreudig ist. dabei kommt es vor allem auf deutschland und frankreich an." auch wegen des brexits werde deutschland einen höheren beitrag zum eu-budget beitragen müssen, sagte laschet. "es nützt übrigens gerade deutschland, wenn die kohäsionsfonds bleiben, wenn noch mehr in innovation, forschung, in künstliche intelligenz investiert wird. wir müssten begeisterter und nicht buchhalterisch über europa reden", sagte der frühere europaabgeordnete laschet. die europawahl findet ende mai nächsten jahres statt. & 136 & very low & Low & Power & NA & NA & 2018-11-16 & 2018 & 3 & POL
Frame & v.low & National & <500 & -1.0405052 & -1.0641830 & 1.2185583 & -1.5781059 & 1.4491718 & 12.0 & 1.6295860 & 1.9019065 & Payer & Domestic & European & Mixed & Domestic|POL & Positive\\
\addlinespace
Germany & http://www.spiegel.de/politik/deutschland/armin-laschet-beklagt-mangelndes-engagement-in-der-europapolitik-a-1238811.html & 235 & SPIEGEL ONLINE & Private/Non-Public & Online and Offline & National & very low = CP mentioned once & Solidarity to poor countries/regions & Positive & National & No myth & Research \& innovation & Positive & National & No myth & NA & NA & NA & NA & Germany & kritik vom cdu-vize: laschet wirft regierung mangelndes engagement in europapolitik vor - spiegel online - politik & 2018-11-16 & kohäsionsfonds & der nordrhein-westfälische ministerpräsident armin laschet hat der bundesregierung und kanzlerin angela merkel mangelndes engagement in der europapolitik vorgeworfen. "die deutsche politik ist zu zögerlich", kritisierte laschet im spiegel. "wir stehen vor einer europawahl, bei der möglicherweise die populisten von links und rechts große erfolge feiern werden", warnte der stellvertretende cdu-vorsitzende. "dem kann man nur begegnen, wenn man zeigt, dass europa handlungsfähig und entscheidungsfreudig ist. dabei kommt es vor allem auf deutschland und frankreich an." deutschland werde, auch wegen des brexits, einen höheren beitrag zum eu-budget beitragen müssen. "wir werden mehr bezahlen müssen, wenn die briten rausgehen" so der cdu-politiker. "es nützt übrigens gerade deutschland, wenn die kohäsionsfonds bleiben, wenn noch mehr in innovation, forschung, in künstliche intelligenz investiert wird. wir müssten begeisterter und nicht buchhalterisch über europa reden." & 132 & very low & Low & Values & Socio-Economic & NA & 2018-11-16 & 2018 & 3 & ECO
Frame & v.low & National & <500 & -1.0405052 & -1.0641830 & 1.2185583 & -1.5781059 & 1.4491718 & 12.0 & 1.6295860 & 1.9019065 & Payer & Domestic & Domestic & Domestic & Domestic|ECO & Positive\\
Germany & https://www.presseportal.de/pm/133076/4136613 & 232 & presseportal.de & Private/Non-Public & Online only & National & very high = CP is most important issue + CP is mentioned in title/headline & Institutional bargaining over funding & Factual & EU + National + Subnational & No myth & Bureaucracy and/or delays & Factual & EU + National + Subnational & No myth & NA & NA & NA & NA & Germany & die bedürfnisse der bürger im mittelpunkt: vorschläge der regionen für die eu-kohäsionspolitik 2021-2027 & 2018-12-06 & europäischer fonds für regionale entwicklung & brüssel (ots) - der europäische ausschuss der regionen (adr) plädiert für eine einfachere und flexiblere kohäsionspolitik. sie soll für eine inklusive nachhaltige entwicklung eingesetzt werden, um unterschiede zu verringern und allen europäischen bürgerinnen und bürgern mehr chancen zu eröffnen. die lokalen und regionalen mandatsträger verabschiedeten auf der adr-plenartagung am 5. dezember vier stellungnahmen mit empfehlungen für verbesserungen und legislative änderungen. darin reagieren sie auf die vorschläge der europäischen kommission für die kohäsionspolitik 2021-2027. die mit einem budget von 365 milliarden euro bis 2020 ausgestattete kohäsionspolitik bleibt auch im kommenden jahrzehnt das wichtigste instrument der eu, um den wirtschaftlichen, sozialen und territoriale zusammenhalt zu fördern und die maßnahmen der eu in den regionen für alle sichtbar zu machen. die lokalen und regionalen entscheidungsträger legten am 5. dezember vier stellungnahmen vor. darin sprechen sie sich für verbesserungen und legislative änderungen aus, die sich aus den vorschlägen der europäischen kommission für die kohäsionspolitik 2021-2027 ergeben. gleichzeitig bekräftigten sie ihre ablehnung der von der kommission vorgeschlagenen mittelkürzung um zehn prozent. die lokalen und regionalen mandatsträger bewerteten die von der europäischen kommission im mai vorgelegten legislativvorschläge für den zeitraum 2021-2027. darin geht es um die wichtigsten eu fonds (verordnung mit gemeinsamen bestimmungen), den europäischen fonds für regionale entwicklung, den europäischen sozialfonds plus und den europäischen fonds für territoriale zusammenarbeit. für eine flexiblere kohäsionspolitik wirbt eine der stellungnahmen. eine strategie dafür bestehe darin, den verwaltungsaufwand für begünstigte und fondsverwalter zu verringern. regionen, städten und lokalen interessenträgern soll dadurch aber kein nachteil entstehen. michael schneider (de/evp), staatssekretär und bevollmächtigter des landes sachsen-anhalt beim bund, und catiuscia marini (it/spe), präsidentin der region umbrien, hatten diese gemeinsam erarbeitet. zudem plädiert schneider für höhere kofinanzierungssätze, um die attraktivität der regionen zu steigern. er betonte außerdem, dass "der europäische landwirtschaftsfonds für die entwicklung des ländlichen raums (eler) daher wieder in die dachverordnung aufgenommen werden muss, um synergien mit den anderen fonds zu maximieren". "die heute vorgelegten vorschläge werden die kohäsionspolitik modernisieren, vereinfachen und verbessern und zeigen, dass sich die regionen und städte europas dafür einsetzen diese politik fit für die zukunft zu machen. wir brauchen mehr flexibilität und eine gemeinsame verwaltung, damit die mittel auf allen regierungsebenen so eingesetzt werden können und die eu die wirkung erzielen kann, die sich die bürger erwarten. dies wird es der kohäsionspolitik ermöglichen, die herausforderungen zu meistern, vor denen europa nicht nur heute, sondern morgen steht", sagte der präsident des europäischen ausschusses der regionen karl-heinz lambertz. die eu-kommissarin für regionalpolitik, corina cretu, fügte hinzu: "die wertvolle arbeit des europäischen ausschusses der regionen zur zukunft der kohäsionspolitik hat dazu beigetragen, das tempo für die verhandlungen zu bestimmen. ich begrüße seinen konstruktiven ansatz - insbesondere seine unterstützung für ein starkes partnerschaftsprinzip. seit beginn der debatte über die zukunft der europäischen union war der ausschuss einer der lautstärksten befürworter der kohäsionspolitik. ich bin zuversichtlich, dass unsere gute zusammenarbeit weiterhin früchte tragen und den weg für eine starke kohäsionspolitik ebnen wird." europäischer fonds für regionale entwicklung (efre) und kohäsionsfonds: der schwerpunkt von efre (derzeit mit etwa 200 milliarden euro über sieben jahre der größte investitionsfonds der eu) und des kohäsionsfonds (reserviert für länder, deren bruttoinlandsprodukt pro kopf weniger als 90 prozent des eu-durchschnitts beträgt) muss weiterhin auf dem wirtschaftlichen, sozialen und territorialen zusammenhalt liegen. in der stellungnahme fordern die regionen und städte, die von der kommission vorgeschlagene kürzung des kohäsionsfonds um 46 prozent zu überarbeiten und zu begrenzen. zudem empfehlen sie, die gleichbleibende mittelausstattung für den efre (+1 prozent) zu überdenken. europäischer sozialfonds plus (esf+): die regionen und städte wollen, dass der soziale zusammenhalt das zentrale ziel des esf bleibt. weiterhin sprechen sich die mandatsträger dafür aus, seine verknüpfung mit den länderspezifischen empfehlungen der kommission im rahmen des europäischen semesters zu verstärken. die bedürfnisse der bürgerinnen und bürger müssen an erster stelle stehen. in der stellungnahme zum esf+ wird die direkte verknüpfung zwischen dem esf+, der europäischen säule sozialer rechte und der koordinierung der makroökonomischen maßnahmen der mitgliedstaaten im rahmen des europäischen semesters begrüßt. die verbindung beider säulen stärkt die europäische soziale dimension des europäischen semesters und der kohäsionspolitik. europäische territoriale zusammenarbeit (etz): in der stellungnahme zum etz begrüßt der adr, dass eine spezifische verordnung für diese grundlegende politik der eu vorgelegt wird. allerdings lehnen die regionen und städte den vorschlag der kommission zur kürzung der etz haushaltsmittel um 1,847 milliarden euro ab. der adr ist entschlossen, gemeinsam mit dem europäischen parlament und dem rat auf eine rücknahme des vorschlags der kommission hinzuarbeiten. wenn die maritime zusammenarbeit als teil der transnationalen zusammenarbeit betrachtet wird, wie der vorschlag nahelegt, würden in der konsequenz die regionen von einer grenzübergreifenden maritimen zusammenarbeit abgeschnitten werden. lesen sie mehr zu den adr-vorschlägen für die europäischen struktur- und investitionsfonds 2021-2027: https://cor.europa.eu/en/news/documents/press-memo-de.docx die volle debatte zum anschauen: https://ec.europa.eu/avservices/video/player.cfm?ref=i164657 & 812 & very high & High & Power & Governance & NA & 2018-12-06 & 2018 & 3 & POL
Frame & high-very high & National & 500-1000 & -1.0405052 & -1.0641830 & 1.2185583 & -1.5781059 & 1.4491718 & 12.0 & 1.6295860 & 1.9019065 & Payer & Domestic & European & Mixed & Domestic|POL & Neutral\\
Germany & https://www.tagesspiegel.de/politik/eu-haushalt-guenther-oettinger-bald-weniger-geld-fuer-osteuropa/22619754.html & 234 & Der Tagesspiegel & Private/Non-Public & Online and Offline & Regional/Local & medium = CP is important part of story & Solidarity to poor countries/regions & Positive & EU & No myth & Institutional bargaining over funding & Balanced & EU & No myth & NA & NA & NA & NA & Germany & günther oettinger: bald weniger geld für osteuropa & 2018-05-29 & kohäsionsfonds & an diesem dienstag beschließt die eu-kommission ihre pläne für einen fonds, der die angleichung der lebensverhältnisse in europa fördern soll. bei der verteilung der gelder steht eine wende bevor. italien soll nach dem willen der eu-kommission künftig mehr geld aus den fördertöpfen der europäischen union bekommen, einige länder im osten der union dagegen weniger. für länder wie die slowakei, die baltischen länder oder polen sei im geplanten künftigen eu-haushalt weniger geld für die kohäsionspolitik vorgesehen, "weil sie wettbewerbsstärker geworden sind, weil sie wirtschaftlich zugelegt haben", sagte eu-haushaltskommissar günther oettinger am dienstag im eu-parlament in straßburg. "andere, die in den letzten jahren länger in der stagnation gewesen sind - italien -, bekommen mehr geld." er gehe fest davon aus, dass einige der erst nach dem zerfall des ostblocks beigetretenen staaten mit ihrer wirtschaftsleistung pro kopf im nächsten jahrzehnt den europäischen durchschnitt übersteigen würden, sagte oettinger. einige könnten damit nettozahler in den eu-haushalt werden. dies zeige den erfolg der eu-kohäsionspolitik, die "noch immer den gedanken der solidarität und stärkung der schwachen in sich trägt". die eu-kommission beschließt an diesem dienstag ihre pläne für die zukunft der regional- und kohäsionsfonds. sie sollen eine angleichung der lebensverhältnisse in der eu fördern und sind nach den agrarausgaben der größte posten im eu-budget. die eu-kommission hat vorgeschlagen, dafür im nächsten eu-finanzrahmen von 2021 bis 2027 rund 373 milliarden euro zur verfügung zu stellen. die eu-kommission unterstützt dabei auch den deutschen vorschlag, gebiete mit einer hohen zahl von flüchtlingen künftig stärker zu berücksichtigen. dies könnte zu lasten osteuropäischer staaten gehen, welche die flüchtlingsaufnahme verweigern. hauptankunftsländer für flüchtlinge wie italien oder griechenland, aber auch deutsche regionen könnten davon profitieren. mehr zum thema brexit-lücke deutschland soll bis zu 12 milliarden euro mehr in den eu-haushalt einzahlen allerdings wird deutschland wegen seiner großen wirtschaftskraft nach dem brüsseler vorschlag insgesamt weniger bekommen. denn wegen des eu-austritts großbritanniens und neuer aufgaben bei verteidigung, grenzschutz und forschung sollen bei der kohäsionspolitik einsparungen erfolgen. den plänen müssen das europaparlament und die mitgliedstaaten noch zustimmen. (afp, dpa) & 345 & medium & Medium & Values & Power & NA & 2018-05-29 & 2018 & 3 & ECO
Frame & low-medium & Regional & <500 & -1.0405052 & -1.0641830 & 1.2185583 & -1.5781059 & 1.4491718 & 12.0 & 1.6295860 & 1.9019065 & Payer & European & European & European & European|ECO & Positive\\
Germany & https://www.tagesschau.de/kommentar/groko-kommentar-eu-101.html & 259 & tagesschau.de & Private/Non-Public & Online and Offline & National & very low = CP mentioned once & Institutional bargaining over funding & Negative & National & No myth & NA & NA & NA & NA & NA & NA & NA & NA & Germany & kommentar: groko-sondierer auf dem weg zur mehrschichten-eu & 2018-01-12 & kohäsionsfonds & wer mehr geld gibt, soll auch mehr einfluss haben: die pläne von union und spd könnten die eu verändern. einige mitgliedstaaten müssen sich auf harte verteilungskämpfe einstellen. von andreas meyer-feist, ard-studio brüssel das war eine schwere geburt: 24 stunden tauziehen - herausgekommen ist ein papier, auf dem das thema europa ganz oben steht. falls nach den zähen sondierungen die koalitionsverhandlungen sauber über die bühne gehen, falls die neue groko tatsächlich kommt, falls die pläne niemand vergessen hat - was in solchen fällen leider oft passiert - dann könnte die eu bald anders aussehen als heute. die neue groko will den brüsseler apparat neu aufstellen. kein weiter so wie bisher. das ganze unterfüttert mit mehr geld aus berlin. brüssel soll finanziell gestärkt werden. das ziel: mehr einfluss der eu-hauptgeldgeber. weniger einfluss für diejenigen, die aus brüssel nur geld sehen wollen, aber nur wenig solidarität zeigen, wenn es um teilbare belastungen geht. zum beispiel bei der versorgung von flüchtlingen. für länder wie polen oder ungarn könnten schwere zeiten anbrechen - falls sie sich nicht auf die neue lage einstellen. ihnen wird ein kräftiger wind aus berlin entgegen wehen. vor allem frankreichs präsident emmanuel macron dürfte jetzt aufatmen. er will deutschland und frankreich enger zusammenwachsen lassen - und aus diesem neuen kraftzentrum heraus die eu verändern. im euroraum wird wohl bald enger kooperiert. hier hat sich der ehemalige eu-parlamentspräsident schulz wohl klar gegen die eher zurückhaltende kanzlerin durchgesetzt. am ende könnte eine mehrschichten-eu stehen: auf der einen seite die premium-eu, auf der anderen seite die eu-holzklasse. wer upgraden will, muss sich an die spielregeln halten. die neue groko will brüssel verändern. aber es gibt fragezeichen. wenn mehr geld aus berlin nach brüssel fließen soll, dann muss es dafür auch mehr messbare gegenleistungen geben. und zwar die richtigen. mehr zahlen und dann sparen an der falschen stelle - das geht nicht mehr. so könnte es aber kommen. nach dem brexit fehlt mit großbritannien einer der wichtigsten eu-nettozahler. die aussage im groko-papier "wir wollen die eu finanziell stärken, damit sie aufgaben besser wahrnehmen kann", reicht nicht aus. das ganze eu-fördersystem gehört auf den kopf gestellt, damit es mehr luft gibt für neue aufgaben. die landwirtschaft und der kohäsionsfonds sind riesen-batzen. beides verschlingt einen großteil des eu-haushalts. das wird wohl so nicht mehr bleiben können. es wird harte verteilungskämpfe geben. dann kommt es auch auf berlin an. & 397 & very low & Low & Power & NA & NA & 2018-01-12 & 2018 & 3 & POL
Frame & v.low & National & <500 & -1.0405052 & -1.0641830 & 1.2185583 & -1.5781059 & 1.4491718 & 12.0 & 1.6295860 & 1.9019065 & Payer & Domestic & Domestic & Domestic & Domestic|POL & Negative\\
Germany & https://www.dw.com/de/kommentar-orbans-zaun-ver\%C3\%A4nderte-europa/a-19549846 & 196 & Deutsche Welle & Private/Non-Public & Online and Offline & National & very low = CP mentioned once & Political leverage & Negative & EU + Other country & 6.Does not defend EU values (eg.gender/law/democracy) & Solidarity to poor countries/regions & Negative & EU + Other country & No myth & NA & NA & NA & NA & NA & kommentar: orbans zaun veränderte europa & 2016-09-14 & NA & NA & 738 & very low & Low & Power & Values & NA & 2016-09-14 & 2016 & 2 & POL
Frame & v.low & National & 500-1000 & -1.0405052 & -1.0641830 & 1.2185583 & -1.5781059 & 1.4491718 & 12.0 & 1.6295860 & 1.9019065 & Payer & European & European & European & European|POL & Negative\\
\addlinespace
Germany & https://www.t-online.de/region/id\_84417246/leibniz-institut-in-frankfurt-oder-wird-erweitert.html & 194 & T-online.de & Private/Non-Public & Online and Offline & National & low = CP mentioned more times but NOT important part of story (mainly about others issues) & Research \& innovation & Positive & EU + National + Subnational & No myth & NA & NA & NA & NA & NA & NA & NA & NA & NA & leibniz-institut in frankfurt (oder) wird erweitert & 2018-09-08 & NA & NA & 92 & low & Low & Socio-Economic & NA & NA & 2018-09-08 & 2018 & 3 & ECO
Frame & low-medium & National & <500 & -1.0405052 & -1.0641830 & 1.2185583 & -1.5781059 & 1.4491718 & 12.0 & 1.6295860 & 1.9019065 & Payer & Domestic & European & Mixed & Domestic|ECO & Positive\\
Germany & http://www.spiegel.de/politik/ausland/eu-gipfel-angela-merkel-fordert-polen-und-ungarn-heraus-a-1195021.html & 260 & SPIEGEL ONLINE & Private/Non-Public & Online and Offline & National & very low = CP mentioned once & Political leverage & Factual & National & No myth & NA & NA & NA & NA & NA & NA & NA & NA & Germany & eu-gipfel: merkel fordert osteuropa heraus - spiegel online - politik & 2018-02-23 & kohäsionsfonds & es ist der start von mammutverhandlungen, selbst für brüsseler verhältnisse. anfang mai will eu-haushaltskommissar günther oettinger einen ersten eu-budget-entwurf für die jahre 2021 bis 2027 vorstellen - und an diesem freitag wollen die staats- und regierungschefs der eu im idealfall schon einmal abklopfen, welche streitpunkte es geben könnte. denn immerhin geht es um viel geld: rund eine billion euro, etwa ein prozent des bruttonationaleinkommens der gemeinschaft, muss über den zeitraum verteilt werden. für den letzten sogenannten mehrjährigen finanzrahmen brauchten die eu-mitglieder 29 monate, um zu einem ergebnis zu kommen. wenig spricht dafür, dass es dieses mal viel schneller geht. im gegenteil, es dürfte dieses mal noch komplizierter werden als in der vergangenheit. das hat vor allem zwei gründe: den brexit und den frust von angela merkel. zunächst zum brexit: die eu muss sparen, weil großbritannien künftig nicht mehr zur gemeinschaft gehört. dadurch fallen etwa 12 bis 14 milliarden euro jährlich weg. auf der anderen seite kommen neue aufgaben auf die gemeinschaft zu: ein besserer schutz der außengrenzen, mehr hilfe für afrika, die integration von migranten. entsprechend heftig wird in brüssel bereits jetzt gerungen - obwohl völlig offen ist, ob das rahmenbudget, überhaupt noch vor den europawahlen im mai 2019 beschlossen werden kann. oettinger strebt das zwar an, wahrscheinlicher ist aber, dass es erst im jahr 2020 zu einem abschluss kommt. dann erst drängt die zeit, und druck hat den kompromiss in brüssel noch immer befördert. für berlin ist das datum nicht ohne bedeutung, denn im zweiten halbjahr 2020 übernimmt deutschland die rotierende ratspräsidentschaft. entsprechend ist die mögliche nächste bundesregierung schon ein bisschen in vorleistung gegangen und hat im koalitionsvertrag angekündigt, man sei bereit, künftig mehr für die eu auszugeben. wieviel, das ist offen, oettinger forderte in der "bild"-zeitung "mindestens 3 bis 3,5 milliarden euro" im jahr. andere länder wie österreich oder die niederlande nehmen die deutsche großzügigkeit dankend zur kenntnis, denken selbst allerdings nicht daran, mehr für die eu zu zahlen. wenn die deutschen freiwillig wollen, bitte schön! mehr geld gegen mehr solidarität die bundesregierung hat einen hintergedanken, wenn sie jetzt nicht bei jedem euro knausert - hier kommt nun der frust der kanzlerin ins spiel. merkel kann sich im kleinen kreis köstlich darüber amüsieren, wenn mit europäischem geld mal wieder ein radweg im münsterland gepflastert wird, der mit europa nun so gar nichts zu tun hat. entsprechend fordert die bundesregierung in einem fünfseitigen "positionspapier der bundesregierung zum mehrjährigen finanzrahmen der eu", das dem spiegel vorliegt, projekte mit "europäischen mehrwert" zu fördern, "insbesondere beim schutz der außengrenzen" oder "der bekämpfung von terrorismus und organisierter kriminalität". vor allem aber drängt deutschland darauf, dass die vergabe von regionalmitteln künftig auch daran geknüpft wird, dass die empfängerländer rechtsstaatliche kriterien einhalten, ein vorschlag mit sprengkraft. die einhaltung rechtstaatlicher grundprinzipen sei "eine wichtige vorrausetzung für ein gesundes investitionsumfeld", heißt es in dem papier. der passus ist vor allem eine kampfansage an länder wie ungarn oder polen, die sich, etwa bei der umstrittenen polnischen justizreform nicht an eu-werte halten oder urteile des europäischen gerichtshofs nicht umsetzen. genauso verhält es sich mit einer anderen idee, die merkel am donnerstag im bundestag erläuterte: die kanzlerin ist bereit, die hilfsfonds für strukturschwache regionen als druckmittel dafür zu nehmen, eine gleichmäßigere verteilung von flüchtlingen in der eu zu erreichen. dabei schwebt berlin offenbar ein neues anreizsystem vor, wie aus dem positionspapier hervorgeht. der "grundwert der solidarität" solle sich "auch in den eu-finanzen widerspiegeln", heißt es darin. in vielen mitgliedstaaten hätten regionen und kommunen die aufnahme und integration von schutzberechtigen oder bleibeberechtigten zuwanderern übernommen. daher sei es notwendig, "dass die eu-struktur- und kohäsionsfonds gerade auch jene mitgliedstaaten und regionen unterstützen, die sich dieser aufgabe gestellt haben". die begeisterung dafür dürfte sich vor allem in osteuropa in grenzen halten, der polnische außenminister konrad szymanski sprach bereits von "einem fehler". umso besser, wenn es wenigstens beim zweiten gipfelthema, der frage der spitzenkandidaten ein bisschen frieden gibt - zumindest vorerst. zwar denken die meisten staats- und regierungschefs gar nicht daran, wie vom parlament gefordert, denjenigen spitzenkandidaten automatisch als kommissionschef durchzuwinken, der die europawahl 2019 gewinnt. doch diese frage stellt sich erst in gut einem jahr, daher kann man sie erst mal offenlassen. & 693 & very low & Low & Power & NA & NA & 2018-02-23 & 2018 & 3 & POL
Frame & v.low & National & 500-1000 & -1.0405052 & -1.0641830 & 1.2185583 & -1.5781059 & 1.4491718 & 12.0 & 1.6295860 & 1.9019065 & Payer & Domestic & Domestic & Domestic & Domestic|POL & Neutral\\
Germany & http://www.dw.com/de/serbiens-zweckgemeinschaft-mit-der-eu/a-37781378 & 284 & Deutsche Welle (English) & Public & Online and Offline & National & very low = CP mentioned once & Solidarity to poor countries/regions & Negative & EU + Other country & No myth & Institutional bargaining over funding & Negative & EU + Other country & No myth & NA & NA & NA & NA & Germany & serbiens zweckgemeinschaft mit der eu | europa | dw.com | 02.03.2017 & 2017-03-02 & kohäsionsfonds & von der eu hoffiert: serbiens ministerpräsident aleksandar vucic offiziell ist die haltung der serbischen regierung ganz auf der eu-linie: "wir sollen die eu als einen seriösen partner verstehen, nicht als ein schatzkrug, den wir finden, und dann wird alles gut - bessere gehälter und renten, arbeitsplätze, weniger arbeitslosigkeit. das soll uns nicht die eu bringen, das sollen wir leisten", sagt in einem dw-interview die ministerin für öffentliche verwaltung und lokale selbstverwaltung in der regierung serbiens, ana brnabić. in der realität zeigt sich der weg in die eu als ein steiniger. seit anfang 2014, als die verhandlungen anfingen, sind bis heute nur acht von insgesamt 35 verhandlungskapiteln eröffnet. nur die beiden kapitel, die wissenschaft sowie kultur regeln, sind abgeschlossen. bei dem tempo wird serbien erst zwischen 2025 und 2030 beitrittsbereit sein, obwohl belgrad 2020 als wunschdatum des beitritts angepeilt hat. dazu müssen noch einige der schwierigsten themen erst verhandelt werden - wie etwa kapitel 35 über die normalisierung der beziehungen zum kosovo. kosovo und russlandliebe in dieser sache hat belgrad eine prorussische gebärde immer parat. mal organisiert man eine serbisch-russische militärparade, mal schenkt moskau serben einige alte kampflugzeuge. dabei bleibt der ost-west-spagat essenziell für die aufrechterhaltung belgrader kosovo-politik. diese ehemalige serbische provinz, in der mehrheitlich albaner leben, erklärte 2008 seine unabhängigkeit. die usa und 23 von 28 eu-mitgliedstaaten sprachen zügig die anerkennung aus. moskau dagegen verhindert durch sein vetorecht im un-sicherheitsrat bis heute die aufnahme des kosovo in die un. durch die russische unterstützung wurde die belgrader blockadepolitik ein ärgernis für die mehrheit in der eu. zwar verhandeln prishtina und belgrad unter der eu-leitung in brüssel über eine normalisierung der beziehungen, beide seiten haben dabei aber grundsätzlich gegensätzliche annahmen. für belgrad ist die "selbsternannte staatlichkeit des kosovo" nach wie vor einfach nicht existent, während für prishtina die serbische anerkennung der unabhängigkeit zu hauptzielen der verhandlungen gehört. gleichzeitig weigert sich serbien, die wegen des ukraine-konflikts verhängten sanktionen gegen russland, mitzutragen. sowohl-brüssel-als-auch-moskau ist eine konstante in der außenpolitischen positionierung des landes unter dem national-konservativen ministerpräsidenten aleksandar vučić. doch die unkritische liebe zum "serbischen kosovo" und zu russland bleiben eindeutig ernst zu nehmende hindernisse auf dem weg serbiens zur eu. spätestens bei der eröffnung des kapitels 31 (die gemeinsame außen- und sicherheitspolitik) wird dieses thema belgrad auf die füße fallen. verheerende wirtschaftsentwicklung zu alldem kommt ein unerfreulicher zustand der serbischen wirtschaft. dušan reljić von der stiftung wissenschaft und politik (swp) warnt seit langem davor, sechs westbalkan-länder (albanien, bosnien-herzegowina, kosovo, mazedonien, montenegro und serbien) ihrem ökonomischen schicksal zu überlassen. sie würden nicht in der lage sein, ein nennenswertes wirtschaftswachstum selbständig zu generieren. "eine berechnung der weltbank besagt: selbst wenn diese länder ein sechsprozentiges wachstum jährlich hätten, würden sie den eu-durchschnitt erst im jahre 2035 erreichen - vorausgesetzt die eu-länder hätten gar kein wachstum", so reljić. die vorschläge des balkankenners, wie etwa den sofortigen zugang zu eu-kohäsionsfonds für diese länder zu ermöglichen, dürfen allerdings heutzutage auf wenig gegenliebe in brüssel und berlin stoßen. nach dem brexit geht es in vielen europäischen hauptstädten um blankes politisches überleben. eine intensive finanzielle unterstützung für die schwächelnden eu-anwärter ist derzeit nirgends in der eu mehrheitsfähig. orbanisierung serbiens im aktuellen jahresbericht von amnesty international (ai) wird man einige hinweise auf weitere vorprogrammierte verhandlungsschwierigkeiten belgrads finden. beispielhaft für staatlich organisierte willkür und politische bevormundung der justiz und der polizei ist der fall "savamala". eine schar paramilitärisch organisierter, vermummter personen griff in einer aprilnacht des vergangenen jahres einige bewohner und ladenbesitzer in dem belgrader stadtviertel savamala an. die angegriffenen bürger und ihre immobilien standen dem prestigeprojekt der regierung - dem hochglanzgeschäftsviertel "belgrad am wasser"- im weg. damit die schleierhaften arabischen investoren des projekts nicht den legalen aber möglicherweise lang andauernden und teueren weg der räumung oder enteignung abwarten müssen, wurden zerstörungen der häuser veranlasst. trotz panischer hilfeanrufe blieb in dieser nacht die polizei auf der wache. eine journalistin brachte die passivität der polizei mit der verantwortung des innenministers in verbindung. die sonst langsame serbische justiz reagierte diesmal prompt und verurteilte die journalistin und ihre wochenzeitung zu einer geldstrafe: sechs durchschnittsmonatsgehälter (rund 2400 euro) kostet in serbien das aussprechen des offensichtlichen. der aktuelle fortschrittsbericht der europäischen kommission über serbien rundet das bild ab. im jahre 2016 findet man im bereich meinungsäußerungsfreiheit und korruption den lapidaren befund: no progress! zweckgemeinschaft und trotzdem hört man aus brüssel hauptsächlich lob für die reformbemühungen serbiens, gepaart höchstens mit einer diplomatisch verhüllten kritik. das offizielle serbien und die eu bilden eine zweckgemeinschaft. brüssel braucht die stabilität auf dem balkan. solange aleksandar vučić innenpolitisch unangefochten ist, gesprächsbereitschaft im zusammenhang mit kosovo zeigt, die flüchtlingspolitik moderat gestaltet und "nie wieder krieg" als mantra wiederholt, soll man ihn, so glaubt man in brüssel, nicht als selbstsüchtigen autokraten entblößen. in belgrad wiederum braucht man die eu als politisches ziel, mit dem die bürger eine vage hoffnung auf besseres leben verbinden. die eu als zuckerbrot, die serbische regierung als peitsche: diese arbeitsteilung birgt in sich das risiko, dass die eu in den augen vieler bürger serbiens zum komplizen von vučić verkommt. diesen verdacht sollte jetzt die eu-außenministerin federica mogherini zerstreuen. im rahmen ihrer balkantournee besucht sie am freitag belgrad, wo gerade der wahlkampf offiziell eröffnet wurde. der starke mann serbiens aleksandar vučić hat sich entschieden, den ministerpräsidentenposten gegen das präsidentenmandat auszutauschen - ganz nach dem modell putins oder erdogans. die wahl ist für den 2. april angesetzt. die regierungsfreundlichen medien stellen mogherinis besuch als willkommene wahlkampfhilfe für vučić dar. wenn auch ihre kritik der serbischen politik im verborgenen der hinterzimmer bleibt, wird das dem land wenig nützen - dem allmächtig erscheinenden vučić aber umso mehr. & 937 & very low & Low & Values & Power & NA & 2017-03-02 & 2017 & 2 & ECO
Frame & v.low & National & 500-1000 & -1.0405052 & -1.0641830 & 1.2185583 & -1.5781059 & 1.4491718 & 12.0 & 1.6295860 & 1.9019065 & Payer & European & European & European & European|ECO & Negative\\
Germany & http://www.sueddeutsche.de/news/politik/eu-oettinger-eu-mittel-fuer-bauern-und-regionen-werden-gekuerzt-dpa.urn-newsml-dpa-com-20090101-180204-99-922110 & 245 & Süddeutsche Zeitung & Private/Non-Public & Online and Offline & National & very low = CP mentioned once & Institutional bargaining over funding & Factual & EU + National & No myth & NA & NA & NA & NA & NA & NA & NA & NA & Germany & oettinger: eu-mittel für bauern und regionen werden gekürzt & 2018-02-04 & kohäsionsfonds & brüssel (dpa) - deutsche bauern und regionen werden ab 2021 wohl weniger geld aus brüssel bekommen. eu-haushaltskommissar günther oettinger sagte der "welt am sonntag": "es wird keinen kahlschlag geben, wie einige befürchten. aber auch in deutschland werden sich landwirte und regionen auf finanzielle kürzungen einstellen müssen." die entsprechenden fördertöpfe - die agrar- und kohäsionsfonds - sollen nach seinen worten um jeweils fünf bis zehn prozent gekürzt werden. oettinger bereitet derzeit seinen entwurf für den nächsten mehrjährigen haushaltsrahmen für die jahre nach 2020 vor. & 81 & very low & Low & Power & NA & NA & 2018-02-04 & 2018 & 3 & POL
Frame & v.low & National & <500 & -1.0405052 & -1.0641830 & 1.2185583 & -1.5781059 & 1.4491718 & 12.0 & 1.6295860 & 1.9019065 & Payer & Domestic & European & Mixed & Domestic|POL & Neutral\\
Germany & https://www.waz.de/staedte/duisburg/nord/problemviertel-foerderkredit-fuer-seenplatte-id213511649.html & 215 & WAZ & Private/Non-Public & Online and Offline & Regional/Local & low = CP mentioned more times but NOT important part of story (mainly about others issues) & Infrastructure & Balanced & Subnational & No myth & NA & NA & NA & NA & NA & NA & NA & NA & Germany & problemviertel-förderkredit für seenplatte & 2018-02-21 & europäischer fonds für regionale entwicklung & marxloh/wedau. stadt plant, 20 millionen euro kredit für das projekt sechs-seen-platte aufzunehmen. geld der nrw-bank, das für viertel wie marxloh gedacht ist. inhalt artikel auf einer seite lesen > ... ... vorherige seite nächste seite problemviertel-förderkredit für seenplatte für arbeiten im zusammenhang mit dem bau des projekts "sechs-seen-wedau" will die städtische baugesellschaft gebag einen eu-geförderten kredit der nrw-bank von 20 millionen euro aufnehmen. der kredit entspringt einem topf, der problembelasteten nrw-stadtteilen wie marxloh oder hochfeld zu gute kommen soll. die stadt argumentiert, der kredit nutze marxloh dennoch, weil im zuge der bauarbeiten 300 bis 400 ungelernte arbeitskräfte aus den genannten stadtteilen eingesetzt werden sollen. es mehren sich stimmen in der marxloher stadtteilpolitik, die dieses vorgehen kritisch sehen. 100 millionen euro teures großprojekt die entsprechende städtische beschlussvorlage, drucksache nummer 18-0045, sieht vor, die wedauer baumaßnahme - hier speziell den beabsichtigten einsatz ungelernter arbeiter - in das integrierte stadtentwicklungskonzept für marxloh und hochfeld einfließen zu lassen. dies sei voraussetzung für die zuteilung des günstigen, eu-subventionierten kredites. so schreibt es die stadt selbst im baugebietsentwicklungsplan für das 100 millionen euro teure großprojekt. die beschlussvorlage wurde am 1. februar in der hamborner bezirksvertretung mehrheitlich beschlossen, am 15. februar in einer sondersitzung der bezirksvertretung duisburg-süd. der rat entscheidet abschließend am 5. märz. linke fragt nach qualität der angekündigten jobs dort werde die linken-fraktion mit hoher wahrscheinlichkeit gegen die vorlage stimmen, sagt linken-politiker herbert fürmann: "sollen sich düsseldorfer millionäre in wedau mit förderung von efre-geldern [europäischer fonds für regionale entwicklung, anm. d. redaktion] villen zulegen?" in der bezirksvertretung hätten die linken sich enthalten: "was sollen dort für arbeitsplätze entstehen? ein-euro-jobs, die dann als beitrag zur stadtentwicklung in marxloh anerkannt werden?", sagt fürmann. außerdem sei das geld im fördertopf begrenzt: "ich habe so den eindruck, dass einige denken, das füllhorn für marxloh und hochfeld wird schon wieder aufgefüllt. das ist aber nicht so." der neumühler ratsherr karlheinz hagenbuck (sgu), selbst im aufsichtsrat der gebag, sieht das vorhaben ebenfalls kritisch: "wie lange bleiben die arbeiter in lohn und brot?" außerdem stelle er sich die frage, ob das geplante vorgehen der stadt nicht mit geltendem ausschreibungsrecht kollidiere. enzweiler (cdu) sieht förderrichtlinien erfüllt das sieht rainer enzweiler, marxloher rechtsanwalt und cdu-fraktionsvorsitzender im rat, grundlegend anders: "marxloh wird hier nichts weggenommen, der antrag entspricht den förderrichtlinien. auch wenn die vorlage etwas unglücklich formuliert ist." davon überzeugt ist auch der spd-landtagsabgeordnete frank börner: "marxloh geht nichts verloren, das hat mir auch der oberbürgermeister versichert." der ankauf von schrottimmobilien behalte weiter priorität, selbst wenn er dann mit geld aus anderen töpfen finanziert werden müsse: "das geld aus diesem topf würde verfallen, wenn wir den antrag nicht stellen würden." wohn- und renditeobjekte werden nicht gefördert der europäische fonds für regionale entwicklung - kurz efre - ermöglicht der landeseigenen nrw-bank seit 2014 die vergabe europäisch geförderter stadtentwicklungskredite. kreditnehmer können kommunen sein, unternehmen der öffentlichen hand aber auch private unternehmen, vereine oder initiativen. die kredite haben flexible laufzeiten und ermöglichen den kreditnehmern, bedingt durch die förderung und absicherung mit eu-geldern, sehr günstige kreditzinsen. einen solchen, mit eu-geldern geförderten kredit will die stadt duisburg durch die einbeziehung der wedauer baumaßnahme in das integrierte stadtteilentwicklungskonzept beantragen. in dem strategiepapier, das sich eigentlich mit perspektiven für die stadtteile hochfeld und marxloh befasst, sind maßnahmen aufgelistet, die zur verbesserung der lebensqualität in den sogenannten problem-stadtteilen beitragen sollen. lärmschutz und grünpflege sind förderungsfähig förderfähig sind laut der nrw-bank nur stadtentwicklungsprojekte, die in einklang mit einem integrierten kommunalen handlungskonzept stehen. durch die aufnahme des wedauer großprojekts in das handlungskonzept für marxloh und hochfeld will die kommune dieses förderkriterium erfüllen. ausgeschlossen von der förderung durch einen europäisch subventionierten stadtentwicklungskredit sind auch wohnprojekte. weil es sich beim wedauer großprojekt natürlich um eine groß angelegte wohnungsbauunternehmung handelt, hat die gebag im bauentwicklungsplan zum projekt zwischen förderfähigen und nicht-förderfähigen projektkosten unterschieden. die förderfähigen projektkosten in höhe von rund 24 millionen euro betreffen etwa lärmschutz und die umsiedlung und neubegrünung von schrebergarten-anlagen oder spielplätzen. gebag rechnet nicht mit großer rendite eine weitere voraussetzung für die zuteilung des gewünschten groß-kredits ist, dass das projekt keine oder eine sehr geringe rendite erwirtschaften darf. dass dies bei der bebauung des duisburger sahne-grundstücks schlechthin der fall sein wird, attestiert sich die gebag ebenfalls im eigenen bauentwicklungsplan. die vom geldgeber geforderte integrations-komponente, schließlich, will die kommune durch die einbeziehung ungelernter marxloher und hochfelder zuwanderer sicher stellen. kommentar von christian... in der grauzone gebag und stadt duisburg bewegen sich mit dem geplanten antrag auf einen mit eu-geldern geförderten stadtentwicklungskredit hart am rande der... inhalt artikel auf einer seite lesen > ... ... vorherige seite nächste seite auch interessant & 776 & low & Low & Socio-Economic & NA & NA & 2018-02-21 & 2018 & 3 & ECO
Frame & low-medium & Regional & 500-1000 & -1.0405052 & -1.0641830 & 1.2185583 & -1.5781059 & 1.4491718 & 12.0 & 1.6295860 & 1.9019065 & Payer & Domestic & Domestic & Domestic & Domestic|ECO & Neutral\\
\addlinespace
Germany & http://www.deutschlandfunk.de/griechenland-hilfen-da-wird-nicht-reflektiert-was-im.694.de.html?dram:article\_id=420995 & 197 & Deutschlandfunk & Public & Online and Offline & National & very low = CP mentioned once & Solidarity to poor countries/regions & Positive & National & 1.Poor regions funded only & Financial burden & Balanced & National & 2.Rich countries pay & Institutional bargaining over funding & Negative & National & NA & Germany & griechenland-hilfen - "da wird nicht reflektiert, was im eigenen land schiefläuft" & 2018-06-22 & kohäsionsfonds & stefan heinlein: mitgehört hat am telefon der cdu-haushaltsexperte klaus-peter willsch, lange jahre einer der schärfsten kritiker der griechenland-hilfen. guten abend, herr willsch. klaus-peter willsch: guten abend, herr heinlein. heinlein: reden wir, herr willsch, vielleicht zunächst über die positive nachricht des heutigen tages. für olaf scholz nicht, aber für die meisten ist es durchaus eine neuigkeit: deutschland hat seit 2010 - wir haben es gehört - fast drei milliarden euro an zinsgewinnen durch die griechenland-kredite eingenommen. waren die griechenland-hilfen, herr willsch, für deutschland durchaus ein gutes geschäft? willsch: nein, das stimmt ja nicht. um mal die sache ein bisschen einzuordnen: die ezb hat damals, als griechenland in höchster not war und von niemand mehr geld bekommen hat, im rahmen des sogenannten smp-programmes griechische staatsanleihen aufgekauft. die sind dann an die einzelnen zentralbanken der eurostaaten verteilt worden und die sind natürlich verzinst worden. dann haben aber schon im zuge des griechischen schuldenschnitts, der ja stattfand 2012, die finanzminister damals gesagt, wir wollen daran kein geld verdienen, sondern geben das den griechen. das ist auch ausgezahlt worden. unsere deutsche bundesbank hat dafür 2,74 milliarden ausgezahlt an die ezb, die das dann wiederum weitergereicht hat nach griechenland. darüber hinaus ist für das, was in 2014 stehen geblieben ist, weil tsipras die vereinbarungen nicht mehr einhalten wollte, offen gebrochen hat, auf ein sperrkonto eingezahlt worden beim esm (1,8 milliarden), und die sollen nun auch ausgekehrt werden. die meldung ist schlicht falsch! die grünen haben falsch gerechnet. und ich kann das auch gar nicht verstehen, wie man zu so einer zahl kommen kann. der cdu-haushaltsexperte klaus-peter willsch (imago/sven simon) heinlein: aber, herr willsch, der finanzminister olaf scholz hat das ja - wir haben es in dem bericht gerade gehört - durchaus bestätigt. er sagte sogar, das ist keine neuigkeit. willsch: das ist auch keine neuigkeit. natürlich werden für anleihen zinsen gezahlt. nur was mit diesen zinsen geschieht? normalerweise wäre der weg so: die bundesbank vereinnahmt die, weist sie später als überschuss aus und der überschuss fließt in den bundeshaushalt. genau das hat nicht stattgefunden. die bundesbank hat sie zwar vereinnahmt, weil das so ist. wenn man eine anleihe zeichnet, kriegt man zur endfälligkeit die anleihe hoffentlich zurück und kriegt zwischendurch zinsen. die sind aber auf separatem konto geführt worden und sind ja griechenland zurückgegeben worden. heinlein: herr willsch, viele fakten, viele zahlen. fakt ist aber auch: der untergang der eurozone, den sie und andere ja 2010 und später auch vorhergesagt haben, hat nicht stattgefunden. und griechenland - da sind sich die euro-finanzminister einig - steht schon bald wieder auf eigenen beinen. haben sie sich damals getäuscht, als sie athen die unterstützung verweigern wollten? willsch: nein, das glaube ich nicht. schauen sie sich doch mal die zahlen an! griechenland hatte einen schuldenstand, gemessen am bruttoinlandsprodukt, von 145 prozent, als 2010 dieser rettungszirkus begann. heute stehen wir bei 180 prozent - und das ganze, nachdem pakete ausgereicht wurden in einer größenordnung von über 300 milliarden und noch ein schuldenschnitt in einem gesamtvolumen, ein schuldenschnitt, ein schuldenrückkauf in einer größenordnung von 64 milliarden vorgenommen wurde. heinlein: sie bleiben dabei, herr willsch, es wäre besser gewesen, die griechen pleite gehen zu lassen? willsch: das wäre bis heute besser. das problem ist ja nicht gelöst. wir haben, um das auch noch mal geradezurücken, wir haben im zuge dieses schuldenschnitts bei unseren bad banks, die wir damals eingerichtet haben, um die maroden geschäftsbanken hre und so weiter abzuwickeln und abzuschirmen, bei diesem schuldenschnitt haben wir bei diesen bad banks ein volumen von über 7,5 milliarden euro verloren als deutscher steuerzahler. das geld ist futsch! darüber hinaus musste die bundesbank 2011, 2012 die iwf-mittel erhöhen. das eigenkapital beim iwf musste damals erhöht werden. da hat der iwf immer mitgemacht. inzwischen machen sie nicht mehr mit. das war eine einzahlung von 41,5 milliarden. und die bundesbank hat risikorückstellungen richtigerweise gebildet. risikorückstellungen mindern das ergebnis und damit die ausschüttung an den bundeshaushalt. die lagen früher immer so bei zwei milliarden. inzwischen sind die risikorückstellungen in einer größenordnung von 16,5 milliarden euro. das heißt, das geld wird richtigerweise von der bundesbank hingelegt nach dem motto, man weiß ja nicht, ob die wirklich zurückzahlen. das risiko ist nach wie vor manifest und es wird da vorm vorhang viel vorgespielt. aber wenn sie in die harten fakten reingehen, dann sieht die geschichte schon anders aus. ein junges mädchen in athen: innerhalb der eu leben die meisten armen kinder in griechenland. (louisa gouliamaki / afp) heinlein: herr willsch, eine frage. zu den harten fakten gehört ja auch, dass athen eisern gespart hat. es stand lange unter dem kuratel von brüssel, hat viele verpflichtungen eingegangen. griechischen rentnern, griechischen sozialhilfeempfängern geht es deutlich schlechter als noch vor jahren. sie haben recht: die arbeitslosigkeit ist nach wie vor hoch, aber die griechischen wirtschaftsdaten gehen langsam voran. es geht wieder aufwärts mit griechenland. und da ist es doch eine frage der solidarität, dass man auch in zukunft den griechen den rücken stärkt. oder sehen sie das nach wie vor anders? willsch: nein! solidarität ist im rahmen der eurozone keine maßgröße. solidarität ist in der europäischen union. dafür haben wir strukturfonds, kohäsionsfonds, was weiß ich was alles an fonds, insgesamt 50 finanzmittel, wo versucht wird, wirtschaftliche unterschiede auszugleichen. das ist der raison d'être der europäischen union. das tun wir, wir sind nettozahler, machen das gerne, weil wir damit eine gedeihliche entwicklung in ganz europa fördern. aber in der eurozone war das definitiv ausgeschlossen von anfang an: keiner darf für die schulden des anderen einstehen oder aufkommen. ausgeschlossen! das ist ja meine kritik von anfang an gewesen und ich habe damals gesagt in meiner rede im bundestag, geld kauft keine freunde. und wenn sie sich den zustand europas heute anschauen, dann habe ich damit auch nicht ganz unrecht gehabt, denn nie ist so schlecht übereinander geredet worden wie zurzeit. heinlein: ein lob, ein zartes lob, ein kleines lob für die griechen und ihre sparanstrengungen geht ihnen nicht über die lippen? willsch: na ja, es ist sehr viel beschlossen worden, ein bisschen was umgesetzt worden, aber es war viel mehr versprochen als beschlossen und erst recht als umgesetzt wurde. ich sehe keinen grundsätzlichen mentalitätswandel dort. natürlich auch durch diese art mit der kontrolle durch früher die troika, dann die institutionen hat es die griechische politik leicht. die sagen immer, wenn irgendein problem auftaucht, na ja, das müssen wir wegen brüssel, das müssen wir wegen berlin machen. da wird nicht reflektiert, was im eigenen land schiefläuft, und das ist eine weitere negative folge. es wird eigenverantwortung unterhöhlt durch eine solche form des bailout, der schuldenübernahme, und damit werden probleme nur vertagt und verschoben. 180 schuldenstand am bip, das ist das dreifache dessen, was nach den vertraglichen grundlagen der währungsunion zulässig ist. heinlein: der cdu-haushaltsexperte klaus-peter willsch heute abend hier im deutschlandfunk. vielen dank, herr willsch, für das gespräch und vielen dank, dass sie uns so spät noch rede und antwort gestanden haben. willsch: aber gerne doch! & 1161 & very low & Low & Values & Values & Power & 2018-06-22 & 2018 & 3 & ECO
Frame & v.low & National & +1000 & -1.0405052 & -1.0641830 & 1.2185583 & -1.5781059 & 1.4491718 & 12.0 & 1.6295860 & 1.9019065 & Payer & Domestic & Domestic & Domestic & Domestic|ECO & Positive\\
Germany & https://www.welt.de/politik/ausland/article158036101/Brexit-ASAP-London-kaempft-andere-Schlachten.html & 288 & DIE WELT & Private/Non-Public & Online and Offline & National & very low = CP mentioned once & Political leverage & Balanced & EU + Other country & No myth & NA & NA & NA & NA & NA & NA & NA & NA & Germany & großbritannien: in london tobt jetzt die nächste brexit-schlacht - welt & 2016-09-12 & strukturfonds & elmar brok kündigt für die verhandlungen bereits an, dass es für die briten keine rosinenpickerei geben wird. anzeige eu-ratspräsident donald tusk jedenfalls will nicht mehr warten. "der ball ist im feld großbritanniens, mit den verhandlungen zu beginnen. es ist in unser aller interesse, damit so schnell wie möglich anzufangen. asap", schrieb er am donnerstag dieser woche nach seinem ersten besuch bei premierministerin theresa may auf twitter. aus seinen worten sprach einmal mehr die machtlosigkeit der eu-staaten angesichts des politischen vakuums, in das der brexit ganz europa katapultiert hat. "as soon as possible" bedeutet tusks vier-buchstaben-kürzel ausgeschrieben. so schnell wie möglich soll klar sein, welchen status die briten künftig gegenüber der eu einnehmen. doch von konkreten plänen ist london zweieinhalb monate nach dem referendum weit entfernt. may hat noch nicht einmal erkennen lassen, wann sie den austrittsprozess überhaupt formell in gang setzen will. das kann laut artikel 50 des eu-vertrags nur das land, das sich zum ausstieg entschlossen hat. um in tusks fußballsprache zu bleiben: nach dem brexit ist vor dem brexit. mit dem ende der sommerpause ist die schonfrist vorbei und das nächste kapitel der brexit-schlacht auf der insel in vollem gange: die zwischen den "harten brexiteers", die keine kompromisse mit europa wollen, und den "weichen brexiteers", die zu zugeständnissen in der frage der zuwanderung von eu-arbeitnehmern bereit sind, wenn es im gegenzug zugang zum binnenmarkt gibt. ein kernversprechen im handstreich kassiert premier may steht zwischen den fronten und lässt bisher nicht erkennen, wie sie die quadratur des kreises zu schaffen gedenkt. mitte der woche machte die konservative wenigstens eine ansage: sie erteilte den forderungen nach einem punktesystem bei der einwanderung von eu-bürgern eine absage. ein solches system nach australischem vorbild, bei dem aufenthaltsquoten an qualifikation geknüpft werden, sei "keine wunderwaffe". es gebe der regierung keine wirkliche kontrolle über die zahl der ankommenden, weil "die leute automatisch reindürfen, nur weil sie die kriterien erfüllen". damit kassierte may im handstreich ein kernversprechen des brexit-lagers und dessen anführers boris johnson. nigel farage, gesicht der anti-eu-partei ukip und ebenfalls befürworter des punktesystems, konterte mays ansage umgehend mit einer drohung. "ihr rückzieher zeigt, dass die regierung bereits einknickt", so farage. "wenn das establishment denkt, dass es den brexit unterwandern kann, dann sollte es sich auf konsequenzen an der wahlurne gefasst machen." wie ihr künftiges zuwanderungsmodell aussieht, ließ sie aber weiter offen. offensichtlich versucht may, im hintergrund erst einmal abzuklopfen, was mit den eu-staaten machbar ist. zu denen sei die nähe ohnehin nicht mehr groß, spotten die kritiker zu hause und zeigen mit dem finger auf das familienfoto des g-20-gipfels in china. für dieses musste sich die 59-jährige mit einem platz ganz am rand begnügen. ein korb ausgerechnet vom "bruderland" druck auf may kommt auch von außerhalb europas. es sei "eine offene frage", ob unternehmen seines landes ihre hauptquartiere demnächst aus großbritannien in die eu verlagern, verkündete japans botschafter in london. in einem brief seiner regierung drängt tokio die britische regierung in aller deutlichkeit, auch künftig den zugang zum eu-binnenmarkt zu garantieren. "angesichts der großen zahl japanischer firmen, die aktiv in großbritannien investiert haben und das land als brücke nach europa gesehen haben, erbitten wir nachdrücklich, dass großbritannien diese tatsache ernsthaft berücksichtigt und in verantwortungsvoller weise reagiert, um nachteilige folgen für unsere unternehmen zu minimieren." kaum einen tag später gab ausgerechnet das "bruder-land" australien der britischen regierung einen korb in sachen handelsabkommen. erst nach dem endgültigen austritt aus der eu werde sein land formell darüber verhandeln, erklärte der zuständige minister steve ciobo diese woche. und auch dann müssten die briten warten. "das ist die folge der tatsache, dass unsere gespräche mit der eu (über ein freihandelsabkommen, die redaktion) weiter fortgeschritten sind als die mit großbritannien." für den brexit-hardliner liam fox, den may für das verhandeln neuer handelsabkommen verantwortlich gemacht hat, bedeutete ciobos abfuhr einen politischen tritt in die magengrube. schließlich führt fox in london das bataillon jener an, die der eu so schnell wie möglich den rücken kehren wollen, weil großbritannien angeblich eine viel bessere wirtschaftliche zukunft im verein mit den ehemaligen kolonialländern habe. "wir erlauben keine rosinenpickerei" gleichzeitig formiert sich in brüssel ein neues bataillon. das europäische parlament ernannte am donnerstag den belgier guy verhofstadt zu seinem offiziellen vertreter in den verhandlungen mit london. der liberale ex-premier gilt als wohl profiliertester verfechter einer föderalen eu. nigel farage beschreibt verhofstadt gern als "fanatischen" föderalisten, "der alles hasst, wofür wir stehen". "das eu-parlament wird konstruktiv und vernünftig die verhandlungen begleiten", so der cdu-abgeordnete elmar brok zur "welt". "aber wir erlauben keine rosinenpickerei. für den zugang zum binnenmarkt müssen die briten weiter bei der freizügigkeit mitmachen und in die strukturfonds einzahlen." beide punkte sind für die brexit-hardliner um theresa may aber nicht verhandelbar - zumindest bis zum jetzigen zeitpunkt. anzeige wenigstens die neuen wirtschaftszahlen halten die hardliner weiter bei laune, denn sie sind bei weitem nicht so dramatisch, wie pessimisten vorausgesagt hatten. am freitag gab die baubranche bekannt, dass die produktion auf demselben niveau geblieben sei, obwohl ökonomen nach dem brexit einen rückgang von 0,5 prozent vorausgesagt hatten. gleichzeitig ging das handelsdefizit zur eu zurück, weil die britischen exporte den stärksten zuwachs der vergangenen sechs jahre verzeichneten. & 872 & very low & Low & Power & NA & NA & 2016-09-12 & 2016 & 2 & POL
Frame & v.low & National & 500-1000 & -1.0405052 & -1.0641830 & 1.2185583 & -1.5781059 & 1.4491718 & 12.0 & 1.6295860 & 1.9019065 & Payer & European & European & European & European|POL & Neutral\\
Germany & https://www.faz.net/1.5568957 & 203 & Frankfurter Allgemeine & Private/Non-Public & Online and Offline & National & very low = CP mentioned once & Institutional bargaining over funding & Balanced & EU + National & No myth & NA & NA & NA & NA & NA & NA & NA & NA & Germany & finanzplanung: das steht drin im billionen-haushalt der eu & 2018-05-02 & strukturfonds & die eu-kommission wird in ihrem offiziellen haushaltsvorschlag für die finanzperiode 2021 bis 2027 voraussichtlich ausgaben von 1,13 prozent der eu-wirtschaftsleistung vorschlagen. diese zahl war am dienstag in der eu-behörde zu hören und entspricht früheren informationen der f.a.z. ursprünglich hatte haushaltskommissar günther oettinger angekündigt, er werde einen betrag zwischen 1,13 und 1,18 prozent vorschlagen. in der laufenden periode beträgt der anteil ein prozent. die kommission begründet die erhöhung mit zusätzlichen aufgaben, die die eu übernehmen soll. hinzu kommt die "brexit-lücke", die aufgrund des eu-austritt des bisherigen nettozahlers großbritannien entsteht. der vorschlag, den die kommissare an diesem mittwoch offiziell beschließen wollen, werde für alle seiten "vertretbare zumutungen" enthalten, hieß es in der kommission. am mittwoch soll nur die grobe struktur des vorschlags vorliegen, die details will die eu-behörde bis zum monatsende vorstellen. schon im juni sollen nach dem willen von kommissionschef jean-claude juncker das europaparlament und die vertreter der mitgliedstaaten die beratungen über den vorschlag aufnehmen. dem vernehmen nach sollen die haushaltsausgaben auf sieben statt bisher fünf titel aufgeteilt werden. der erste, auf binnenmarkt und innovation bezogene titel soll um digitales ergänzt werden. dazu kommen kohäsion (darunter fallen vor allem die strukturfonds), "natürliche ressourcen" (also vor allem die agrarpolitik), migration, sicherheits- und verteidigungspolitik, nachbarschaftspolitik sowie verwaltung. statt bisher 58 soll es künftig nur noch 37 ausgabenprogramme geben. diese reduktion geht aber vor allem auf eine neustrukturierung der programme zurück; komplett wegfallen soll keine einzige ausgabenkategorie. die ausgaben für die kohäsions- und die agrarpolitik will die kommission um 5 bis 7 prozent kürzen. auf diese töpfe entfallen bisher rund 75 prozent der eu-ausgaben. die genauen zahlen seien erst ermittelbar, wenn bis ende mai die strukturen der jeweiligen programme feststehen, hieß es seitens der behörde. bestimmte ausgaben, etwa für forschung und energie sowie das budget erasmus plus, mit dem die eu auslandsaufenthalte von studenten und auszubildenden unterstützt, sollen bekanntermaßen erhöht werden; für erasmus plus ist weiterhin eine ausgabenverdopplung vorgesehen. wie berichtet, will die kommission den haushaltsentwurf in absoluten zahlen anders als bisher zu laufenden preisen vorlegen, also unter berücksichtigung der inflation. sie unterstellt dabei eine inflationsrate von etwa 2 prozent. damit wird der umfang des budgets bis zum ende der finanzperiode rechnerisch aufgebläht. in der kommission wird dieses vorgehen damit begründet, dass mögliche zahler und empfänger schon jetzt wissen müssten, um welche beträge es in einigen jahren gehe. die kalkulationsmethode diene also der "transparenz", hieß es am dienstag. freilich weiß niemand, wie sich die inflationsrate bis 2027 entwickelt, daher ist die bisherige methode zu festen preisen die sauberere. sie soll laut kommission am mittwoch in einem anhang ebenfalls veröffentlicht werden und als "hauptberatungsgrundlage" für die weiteren verhandlungen dienen. auf der einnahmenseite will die kommission nach dem wegfall des briten-rabatts auch alle anderen haushaltsrabatte streichen. diese rabatte, von denen deutschland, die niederlande und schweden profitieren, wurden in den achtziger jahren als eine art ausgleich vom briten-rabatt eingeführt, um die anderen nettozahler ebenfalls etwas zu entlasten. nach den kommissionsplänen sollen diese rabatte bis 2027 schritt für schritt ganz abgeschafft werden. neu im haushaltsentwurf vorgesehen sind auch ausgaben für zwecke der währungsunion, allerdings in vergleichsweise geringer höhe voraussichtlich unter einer milliarde euro. so sollen eurostaaten, die wirtschaftsreformen verwirklichen, dafür "belohnt" werden; staaten, die dem euroraum beitreten wollen, sollen auf spezielle strukturhilfen zugreifen können. hilfen zur abfederung makroökonomischer schocks will die kommission nicht in den eu-haushalt einstellen. sie sind vielmehr als kreditlinie außerhalb des etats vorgesehen, sollen aber von diesem abgesichert werden. die rede ist hier von einem einstelligen milliardenbetrag über die sieben jahre. endgültig klar scheint, dass die kommission die einhaltung rechtsstaatlicher standards in den mitgliedstaaten an die vergabe von fördermitteln knüpfen will. sie will sich dabei offenbar für künftige verfahren großen ermessensspielraum einräumen: wann immer sie als hüterin der verträge einen verstoß gegen das rechtsstaatlichkeitsprinzip feststellt, soll sie künftig entscheiden können, mittel zu kürzen. dieser gesetzesvorschlag ist nicht teil des haushaltsentwurfs, sondern wird getrennt davon eingebracht. & 656 & very low & Low & Power & NA & NA & 2018-05-02 & 2018 & 3 & POL
Frame & v.low & National & 500-1000 & -1.0405052 & -1.0641830 & 1.2185583 & -1.5781059 & 1.4491718 & 12.0 & 1.6295860 & 1.9019065 & Payer & Domestic & European & Mixed & Domestic|POL & Neutral\\
Germany & https://www.heise.de/tp/features/EU-Budgetreform-Geld-gegen-Wohlverhalten-3939550.html & 200 & heise online & Private/Non-Public & Online only & National & very low = CP mentioned once & Solidarity to poor countries/regions & Balanced & EU & No myth & Institutional bargaining over funding & Balanced & No actor & No myth & NA & NA & NA & NA & Germany & eu-budgetreform: geld gegen wohlverhalten & 2018-01-11 & kohäsionsfonds & haushaltskommissar oettinger folgt den wünschen von kanzlerin merkel. künftig sollen finanzhilfen aus brüssel an konditionen gebunden werden. das zielt auf polen und ungarn - könnte aber auch italien oder frankreich treffen man kann nicht sagen, dass es grund zu eile gäbe. erst in zwei jahren braucht die europäische union ein neues rahmenbudget, das die aktuelle mittelfristige finanzplanung 2014-2020 ablöst und das rund eine billion euro schwere eu-budget reformiert. man könnte problemlos die europawahl im frühsommer 2019 abwarten und dann das neue budget festlegen - im lichte der wahlentscheidung. doch eu-haushaltskommissar günther oettinger macht druck. der cdu-mann möchte schon vor der wahl fakten schaffen. und er will noch während der laufenden deutschen koalitionsverhandlungen pflöcke einschlagen. und so fand schon in dieser woche eine "orientierungsdebatte" der eu-kommission zum nächsten eu-budget statt. oettinger drückt aufs tempo, im mai soll der fertige entwurf stehen. doch schon die wenigen andeutungen, die merkels kommissar am mittwoch in brüssel machte, lassen aufhorchen. einem neuen eurozonen-budget, wie es frankreichs staatschef emmanuel macron fordert, erteilte oettinger ebenso eine absage wie neuen schulden oder anleihen. auch eigene eu-steuern will der sparsame schwabe nur im notfall erheben. allenfalls könne man über eine plastiksteuer nachdenken. mit den einnahmen sollen die eu-staaten entlastet werden, die dennoch tiefer in die tasche greifen müssen: um rund zehn prozent "plus x" will oettinger das eu-budget aufstocken, gleichzeitig aber quer durch alle ressorts kürzen. mehr geld für weniger leistung, wie geht das zusammen? es gehe darum, den einnahmeausfall durch den brexit zu kompensieren, beschwichtigt der sparkommissar. das könnte man allerdings auch anders - etwa, indem man die anachronistischen agrarsubventionen zusammenstreicht oder die kohäsionsfonds neu konzipiert, so dass davon nur noch wirklich bedürftige regionen profitieren. doch das wagt oettinger nicht. er hat die kürzungen wohl nicht zufällig so justiert, dass auch die neuen deutschen bundesländer weiter in den "genuss" von eu-hilfen kommen. bluten sollen andere: länder wie polen oder ungarn, die sich nicht an die "grundwerte" der eu halten - oder staaten wie italien oder frankreich, die nicht aufs wort die "wirtschaftspolitischen empfehlungen" aus brüssel befolgen. das ist das größte und konfliktträchtigste novum in oettingers budgetpolitischem masterplan: er will die zahlung von eu-hilfen künftig an konditionen binden. geld wird von politischem wohlverhalten abhängig gemacht. dabei ist nicht einmal klar, ob dies nach eu-recht zulässig ist. "wir sind dabei, dies vertragsrechtlich auf seine machbarkeit hin zu prüfen", räumte oettinger ein. doch für eine intensive prüfung bleibt nicht viel zeit. schon nach ostern will die kommission die umstrittene konditionalität ausbuchstabieren. warum nach ostern? wenn nicht alles täuscht, weil ende märz das ultimatum an polen abläuft, die umstrittene justizreform rückgängig zu machen. danach kommt es zum schwur - und bisher hat die eu-kommission kaum trümpfe in der hand. das vor weihnachten eingeleitete artikel-7-verfahren, mit dem ein ernster verstoß gegen eu-prinzipien festgestellt werden soll, hat nämlich keine chance auf erfolg. ungarn hat schon angekündigt, mögliche sanktionen gegen polen - wie den entzug des stimmrechts im ministerrat - per veto zu verhindern. um das drohende waterloo zu verhindern, denkt die eu-kommission nun über andere strafmaßnahmen nach. der entzug von eu-mitteln wäre da besonders attraktiv; schließlich hängen polen und ungarn am tropf der brüsseler bürokraten. denkbar wäre laut oettinger aber auch, den ländern "ergänzende mittel" anzubieten, wenn sie bedenken in sachen rechtsstaatlichkeit und demokratie ausräumen. das system von zuckerbrot und peitsche könnte aber nicht nur zur durchsetzung der eu-rechtsordnung eingesetzt werden. oettinger denkt auch - nicht zuletzt auf drängen von kanzlerin merkel - darüber nach, es für die finanz- und wirtschaftspolitik zu nutzen. nur wer die kommissionsempfehlungen im rahmen des "europäischen semesters" umsetzt, soll künftig noch finanzspritzen aus brüssel erhalten. oettinger illustrierte das am beispiel der breitbandnetze: es könne ja wohl nicht sein, dass eu-länder geld für die infrastruktur erhalten, wenn sie veraltete netze haben. ein zweischneidiges argument, denn es ließe sich auch gegen deutschland wenden. vor allem ließe sich das prinzip aber auf jene staaten anwenden, die sich den neoliberalen "strukturreformen" verweigern und nicht so radikal den arbeitsmarkt "liberalisieren", wie dies die eu-kommission immer wieder fordert. merkel hatte das grundprinzip schon vor jahren formuliert: geld gegen reformen! damals, auf dem höhepunkt der eurokrise, wollte sie es noch mit nationalen "reformverträgen" durchsetzen. doch das scheiterte am widerstand vieler eu-staaten. nun kommt dieselbe idee in gestalt budgetpolitischer "konditionalität" wieder zum vorschein - ob sich oettinger vorher mit merkel abgestimmt hat? wir wissen es nicht. klar ist nur, dass oettingers vorschläge deutschen "ordnungspolitischen" vorstellungen viel näher stehen als französischen oder polnischen wünschen. die entscheidende frage wird nun sein, ob der cdu-mann dafür auch mehrheiten im eu-ministerrat organisieren kann. das neue rahmenbudget kann nämlich nur einstimmig beschlossen werden. polen oder ungarn könnten den entwurf also zu fall bringen, frankreich auch. mit einem ersten stimmungsbild wird bei einem sondergipfel ende februar gerechnet. österreich hat sich bereits gegen eine aufstockung des eu-budgets ausgesprochen, ausgerechnet ungarn fordert hingegen eine massive ausweitung. die meisten länder wollen aber abwarten, bis zahlen auf dem tisch liegen. doch danach, da sind sich alle in brüssel einig, beginnt die große budget-schlacht. sie könnte blutig werden. (eric bonse) & 849 & very low & Low & Values & Power & NA & 2018-01-11 & 2018 & 3 & ECO
Frame & v.low & National & 500-1000 & -1.0405052 & -1.0641830 & 1.2185583 & -1.5781059 & 1.4491718 & 12.0 & 1.6295860 & 1.9019065 & Payer & European & European & European & European|ECO & Neutral\\
Germany & https://www.augsburger-allgemeine.de/politik/EU-will-Deutschland-Milliarden-fuer-die-Aufnahme-von-Fluechtlingen-ueberweisen-id51264111.html & 237 & Augsburger Allgemeine Community & Private/Non-Public & Online and Offline & Regional/Local & low = CP mentioned more times but NOT important part of story (mainly about others issues) & Social awareness/inclusion & Balanced & National & No myth & Institutional bargaining over funding & Balanced & National & No myth & NA & NA & NA & NA & Germany & eu will deutschland milliarden für die aufnahme von flüchtlingen überweisen & 2018-06-01 & kohäsionsfonds & eigentlich wollte angela merkel länder bestrafen, die keine asylsuchenden aufnehmen. nun kam es anders. warum sich die kanzlerin trotzdem durchgesetzt hat. die kanzlerin hat sich am ende doch durchgesetzt. im nächsten haushalt der europäischen union bekommt deutschland einen zuschlag für seine bemühungen um die aufnahme von flüchtlingen. anstatt widerständler durch den entzug von fördermitteln zu bestrafen, sollen künftig besonders engagierte mitgliedstaaten entlastet werden. dabei hatte zunächst alles nach einer niederlage für angela merkel ausgesehen. als sich die staats- und regierungschefs im februar zu einem sondergipfel über die finanzen für die sieben jahre ab 2021 trafen, wurde die deutsche forderung nach einer bestrafung jener staaten, die keine migranten aufgenommen hatten, brüsk zurückgewiesen. daran hat sich zwar nichts geändert, aber inzwischen steht fest: die union wird stattdessen alle die länder, die ihren grenzen öffneten, auf andere weise "belohnen". bis zu 2800 euro soll es demnächst pro aufgenommenem flüchtling aus der gemeinschaftskasse geben. deutschland kann mit 4,5 milliarden euro rechnen, um die aufwendungen von bund, ländern und kommunen für die unterbringung und integration von asylsuchenden finanziell abzufedern. haushaltskommissar günther oettinger baute in seinen entwurf einen trick ein: aus dem kohäsionsfonds wurden bisher nahezu ausschließlich projekte der mitgliedstaaten für den erhalt der infrastruktur bezuschusst. ab 2021 will die kommission aber mehrere töpfe zusammenlegen und die vergabekriterien erweitern. somit können sich die aufnahmeländer nun auch für die integration von zuwanderern unterstützen lassen. besonders wichtig für deutschland: nach informationen aus dem umfeld der kommission sollen die gelder für alle migranten ausgeschüttet werden, die seit 2013 in die gemeinschaft gekommen sind. das statistische amt der europäischen union gibt deren zahl mit rund 1,7 millionen menschen an. etwa die hälfte davon nahm allein die bundesrepublik im jahr 2015 auf dem höhepunkt der flüchtlingswelle auf. struktur-gelder werden eigentlich nach einem bestimmten schlüssel vergeben. dabei spielt vor allem die wirtschaftskraft der staaten eine rolle. anders bei den finanzmitteln zur unterstützung der flüchtlingsaufnahme: die kommission will dabei die bevölkerungsgröße zugrunde legen sowie die zahl der eingereisten migranten - abgezogen werden diejenigen, die bereits wieder ausgereist sind. damit ist klar: auch jene deutschen bundesländer, die eigentlich statistisch zu den reicheren europäischen regionen gehören, können mit unterstützung aus brüssel rechnen. für die städte und gemeinden, die besonders belastet sind, ist das eine gute nachricht. denn sie mussten bisher darauf hoffen, dass ihre aufwendungen für zusätzliche plätze in aufnahmezentren durch zuschüsse des bundes ausgeglichen werden. politisch dürfte diese umstellung im europäischen haushalt auch für andere staaten wichtig werden: bisher hatten sich griechenland und besonders italien beschwert, weil sie von den partnern alleingelassen wurden. rom hatte immer wieder darauf hingewiesen, dass das land alleine in diesem jahr rund 700.000 flüchtlinge zu verkraften hat. die bundesrepublik erhält nach dem jetzigen entwurf der kommission in der zeit zwischen 2021 und 2027 rund 15,7 milliarden euro aus dem kohäsionsfonds, rechnet man den zu erwartenden inflationsausgleich hinzu, wären es sogar 17,7 milliarden euro. diese fördersumme fällt deutlich höher aus als erwartet, da wegen des brexit eigentlich deutlich größere einschnitte befürchtet worden waren. nun ist klar, dass fast ein drittel der künftigen subventionen ein ausgleich für die hohen aufwendungen in der flüchtlingskrise sind. & 518 & low & Low & Socio-Economic & Power & NA & 2018-06-01 & 2018 & 3 & ECO
Frame & low-medium & Regional & 500-1000 & -1.0405052 & -1.0641830 & 1.2185583 & -1.5781059 & 1.4491718 & 12.0 & 1.6295860 & 1.9019065 & Payer & Domestic & Domestic & Domestic & Domestic|ECO & Neutral\\
\addlinespace
Germany & http://www.wiwo.de/politik/europa/freytags-frage-wie-kann-der-brexit-fuer-beide-seiten-abgefedert-werden/20837200.html & 283 & Wirtschafts Woche & Private/Non-Public & Online and Offline & National & very low = CP mentioned once & Institutional bargaining over funding & Negative & EU + Other country & No myth & NA & NA & NA & NA & NA & NA & NA & NA & Germany & wie kann der brexit für beide seiten abgefedert werden? & 2018-01-12 & kohäsionsfonds & in diesem jahr entscheidet sich, ob die drohenden brexit-verluste für eu und großbritannien abgewendet werden können. warum drohungen dabei ins leere laufen - und welche punkte beide seiten lösen müssen. das gerade begonnene jahr wird für die beziehungen zwischen großbritannien und der europäischen union (eu) entscheidend werden. schon am 29. märz 2019 wollen die briten - oder zumindest ein teil von ihnen - aus der eu ausgeschieden sein. bis zu diesem datum geht es darum, die beziehungen zur eu neu zu regeln. dabei hat die britische regierung klare ziele: sie will möglichst wenig geld zum abschied zahlen, die zuwanderung strikt begrenzen und den zugang ihrer unternehmen zu den märkten für kapital, güter und dienste in europa offenhalten. die europäer ihrerseits haben die sorge, dass der brexit für andere von europakritischen politikern regierte länder zum vorbild wird, sofern es den briten gelingt, gute bedingungen auszuhandeln. deshalb kann man annehmen, dass die europäischen unterhändler zumindest nach außen während der nächsten monate hart auftreten werden. das erkennt man schon am engen zeitplan: bereits im oktober 2018 müsse der austrittsvertrag feststehen, bekräftigen eu-offizielle. der grund: das europäische parlament muss dem vertrag vor ende märz 2019 zustimmen. außerdem, so heißt es, wolle man nicht über die partnerschaft zwischen eu und großbritannien nach dem austritt verhandeln, bevor die details des austritts überhaupt geregelt sind. offenbar auch, um die europäische position aufzuweichen, haben diese woche die beiden britischen minister philipp hammond und david davis, die zu den stützen der ansonsten recht wackeligen britischen regierung zählen, deutschland besucht. in einem gastbeitrag für die frankfurter allgemeine zeitung warben sie dafür, nach dem 29. märz 2019 für eine übergangsphase die alten regeln beizubehalten, bis die neue europäisch-britische partnerschaft im detail ausgehandelt ist. brexit: die britische planlosigkeit ist ein europäisches problem der eu-gipfel ist auftakt der zweiten phase der brexit-verhandlungen. weil die briten aber keine klare vorstellung von ihrer zukunft außerhalb der eu haben, ist eine einigung nicht sicher. das wesentliche argument - und damit die implizite drohung - der beiden minister sind die hohen exporte der deutschen unternehmen nach großbritannien. die deutschen sind mit über 110 milliarden euro die größten exporteure nach großbritannien - das ist tatsächlich keine kleinigkeit. gäbe es in zukunft handelsbarrieren zwischen großbritannien und europa, würde das volumen dieser exporte also schrumpfen. ob das die bundesregierung überzeugt, ist zweifelhaft. und die europäische kommission wird diesem vorschlag ohnehin ungeachtet der möglichen positionierung der bundesregierung sehr skeptisch gegenüberstehen. das britische drohpotential ist denn auch geringer, als mancher minister in london glaubt. anders als von den brexit-initiatoren behauptet, wird sich die wirtschaftliche situation in großbritannien durch den brexit nicht verbessern. das gilt vermutlich besonders für die regionen, die vehement dafür gestimmt haben. sie nämlich haben oft besonders viele mittel aus den europäischen struktur- und kohäsionsfonds erhalten. dass die britische regierung diese zahlungen nach dem brexit kompensiert, kann durchaus bezweifelt werden. & 470 & very low & Low & Power & NA & NA & 2018-01-12 & 2018 & 3 & POL
Frame & v.low & National & <500 & -1.0405052 & -1.0641830 & 1.2185583 & -1.5781059 & 1.4491718 & 12.0 & 1.6295860 & 1.9019065 & Payer & European & European & European & European|POL & Negative\\
Germany & https://www.dw.com/de/frankreichs-junge-arbeitslose-von-der-eu-vergessen/a-48860922 & 206 & Deutsche Welle (English) & Public & Online and Offline & National & very low = CP mentioned once & Ineffective goal achievement & Negative & EU + Other country & No myth & NA & NA & NA & NA & NA & NA & NA & NA & Germany & jugendarbeitslosigkeit in frankreich & 2019-05-24 & strukturfonds & eine reihe junger menschen schwärmt in zweiergruppen in verschiedene richtungen aus, von les halles im zentrum der französischen hauptstadt paris. sie haben grüne flyer und große schwarze pappschilder in der hand. "ich lasse nicht die anderen meinen eu-abgeordneten aussuchen", steht auf den flyern. auf die pappschilder sind kontroverse zitate europäischer abgeordneter gedruckt, um passanten zur diskussion zu animieren - wie zum beispiel zur flüchtlingspolitik. doch was für die aktivisten und aktivistinnen offensichtlich scheint - dass europa wichtig für die jugend ist - leuchtet nicht allen jungen französinnen und franzosen ein. denn fast jede(r) vierte von ihnen ist arbeitslos. so manch eine(r) fühlt sich von der europäischen union (eu) alleine gelassen und hat auch keinen sinn für die anstehenden wahlen. "die eu verschafft uns einen jobvorteil" "europa gehört für mich einfach zur normalität," sagt léo allaire, der sich bei den sogenannten jungen europäern um aktionen wie der an diesem freitag kümmert. der verein setzt sich seit über 30 jahren für ein demokratisches und föderales europa ein. der 21-jährige studiert angewandte sprachen an einer pariser universität und lernt schon seit schulzeiten deutsch. für ihn ist es natürlich, freunde in deutschland, italien und spanien zu haben. aber die eu verschaffe jungen leuten auch einen ganz klaren jobvorteil: "ein freund von mir macht zum beispiel gerade ein praktikum in deutschland, unterstützt vom austauschprogramm erasmus. und es gibt viele andere initiativen für uns junge leute, die sich durch eu-gelder finanzieren. deswegen ist es ja gerade so wichtig, bei der wahl mitzuentscheiden", meint er. ein paar meter weiter steht die 23-jährige marie pouliquen, die politikwissenschaften mit schwerpunkt auf erneuerbaren energien an der pariser universität sciences po studiert. das motto der flugblätter steht auch auf ihrem grünen t-shirt. "eine welt ohne die europäische union kann ich mir nicht mehr vorstellen - das würde ja heißen, jedes land wäre wieder auf sich allein gestellt", sagt sie mit nachdruck. doch pouliquen hat schwierigkeiten, ihre flyer unter die jungen leute zu bringen, die oft abwinken und weitergehen. für sie liegt das auch am aktuellen wahlkampf. "der ist eine absolute katastrophe", sagt sie. "es dreht sich praktisch nur um nationalismus, und die politiker vergessen die wichtigsten themen wie die jugendarbeitslosigkeit. dabei sind wir doch europas zukunft." jugendarbeitslosigkeit kommt im wahlkampf kaum zur sprache der französische europa-wahlkampf ist zum duell zwischen europabefürwortern wie dem aktuellen präsidenten emmanuel macron und europagegnern wie der rechtsradikalen partei rassemblement national (kurz rn, ehemals front national) von marine le pen geworden. das ist auch eine konsequenz der monatelangen demonstrationen der sogenannten gelbwesten, die das land noch weiter gespalten haben. das thema jugendarbeitslosigkeit kommt praktisch nicht zur sprache - weder bei den zwei favoriten, dem rn und macrons partei lrem, noch bei den 32 weiteren parteien in frankreich. für manch eine(n) junge(n) arbeitslose(n) trägt das noch mehr dazu bei, dass die eu abstrakt erscheint. wie zum beispiel für die gruppe junger leute, die an diesem freitag nachmittag an einem seminar im jugendjobzentrum mission locale im departement seine-et-marne bei paris teilnehmen. zwar liegt die arbeitslosigkeit in dieser gegend etwas unter dem nationalen durchschnitt, der für leute bis 25 jahre gerade 19,2 prozent beträgt. aber einfach ist die jobsuche dennoch nicht, sagt inès hadbi, die seit monaten nach einer stelle im verwaltungsbereich sucht. anders als die studierenden des vereins der jungen europäer hat sie noch nie von austauschprogrammen wie erasmus gehört - genauso wie die anderen teilnehmer des seminars, von denen einige die schule nicht fertig gemacht haben. von der eu "alleine gelassen" "ich fühle mich von der eu alleine gelassen", meint die 19-jährige. "wir können uns nur selbst helfen. wenn wir einen job finden, ist das aus eigener kraft und nicht etwa, weil die eu etwas für uns tut. europäische politiker sehen uns doch gar nicht. wir sind ihnen egal." neben ihr sitzt der 19-jährige roméo oaw, der polizist werden will. für ihn ist die eu alles andere als präsent. "ich komme aus einem dorf mit 300 einwohnern" , sagt er. "bei uns ist zwar das europäische logo an allen öffentlichen gebäuden, aber das bedeutet gar nichts. was zählt, sind - wenn überhaupt - lokale initiativen." europas begrenzter handlungsspielraum? weder hadbi noch oaw haben vor, bei den europäischen wahlen ihre stimme abzugeben. christopher dembik, ökonom bei der saxo bank in paris, kann verstehen, dass die eu den jungen leuten weit weg erscheint. aber er meint, das liege auch daran, dass deren handlungsspielraum beschränkt sei. "die eu kann lediglich über strukturfonds gelder zur verfügung stellen. aber ihre programme sind doch nicht auf einzelne länder, bevölkerungsgruppen oder gar klassen gemünzt", sagt er. "europa gibt eine richtung vor, aber es ist an den nationalen regierungen, konkret zu handeln." frankreich habe es dabei besonders schwer, weil seine geburtenrate so hoch sei. deswegen müsse das land umso mehr stellen schaffen, um seine jugendarbeitslosigkeit zu senken. hagar akhrouf, vizepräsidentin der jungen europäer, findet dennoch, dass eu-programme viel zu ihrer qualifikation beitragen können. aber sie gibt zu, dass diese noch ausgeweitet werden müssten, um mehr jugendliche zu erreichen und zu überzeugen. "hier gibt es so viel kreativität", sagt die 24-jährige, die europawissenschaften an der pariser sorbonne-universität studiert. "wir könnten locker mit dem start-up-paradies silicon valley mithalten. wir sind jung, dynamisch und haben potenzial. aber damit wir das voll nutzen können, muss die eu uns mehr unterstützen." doch ob solche programme auch vielen jugendlichen ohne universitätsabschluss in den job helfen werden, ist fraglich. unter nicht-diplomierten ist die arbeitslosigkeit in frankreich bis zu viermal so hoch wie unter diplomierten. & 914 & very low & Low & Socio-Economic & NA & NA & 2019-05-24 & 2019 & 3 & ECO
Frame & v.low & National & 500-1000 & -1.0405052 & -1.0641830 & 1.2185583 & -1.5781059 & 1.4491718 & 12.0 & 1.6295860 & 1.9019065 & Payer & European & European & European & European|ECO & Negative\\
Germany & http://www.epochtimes.de/politik/europa/deutschland-eu-mittel-fuer-polen-und-ungarn-sollen-an-rechtsstaatlichkeit-geknuepft-werden-a2268551.html & 282 & Epoch Times Deutschland & Private/Non-Public & Online only & National & very low = CP mentioned once & Political leverage & Positive & EU + Other country & No myth & NA & NA & NA & NA & NA & NA & NA & NA & Germany & deutschland: eu-mittel für polen und ungarn sollen an "rechtsstaatlichkeit" geknüpft werden & 2017-11-15 & kohäsionspolitik & polen und ungarn würden die "demokratischen grundprinzipien" einschränken, heißt es aus brüssel. deshalb soll den ländern künftig möglicherweise die eu-mittel gekürzt oder gar gestrichen werden. angesichts des dauerstreits mit polen und ungarn um will deutschland die vergabe von eu-mitteln künftig an die "einhaltung der rechtsstaatlichkeit" knüpfen. beim treffen der eu-europaminister am mittwoch in brüssel forderte der parlamentarische staatssekretär im wirtschaftsministerium, uwe beckmeyer (spd), die eu-kommission auf, diese möglichkeit bei der neuausrichtung der mittelvergabe im rahmen der eu-kohäsionspolitik zu prüfen. die eu liegt seit jahren mit polen und ungarn im clinch, weil deren regierungen aus sicht brüssels wichtige "demokratische grundprinzipien" einschränken. gegen beide länder laufen deswegen vertragsverletzungsverfahren, gegen polen auch ein verfahren zur rechtsstaatlichkeit. die europaminister diskutierten am mittwoch erstmals über die neuausrichtung der milliardenschweren eu-kohäsionspolitik im nächsten mehrjährigen finanzrahmen von 2021 bis 2027. durch den eu-austritt großbritanniens stehen auch in diesem bereich deutliche kürzungen an, weil mit london der zweitgrößte nettozahler der union wegfällt. entscheidungen fielen am mittwoch noch nicht, sie werden erst im kommenden jahr erwartet. auch andere nettozahler, die mehr in den eu-haushalt einzahlen als sie zurückbekommen, unterstützen grundsätzlich das vorhaben, die rechtsstaatlichkeit zur bedingung für zahlungen aus der eu-kohäsionspolitik zu machen. ein diplomat nannte im vorfeld des treffens konkret frankreich, die niederlande, finnland, schweden, dänemark und belgien. ungarn und polen wiesen den deutschen vorstoß dem vernehmen nach am mittwoch zurück. die kohäsionspolitik soll eine angleichung der wirtschaftlichen und sozialen verhältnisse innerhalb der eu voranbringen und ist im laufenden finanzzeitraum von 2014 bis 2020 fast 352 milliarden euro schwer - dies ist gut ein drittel des gesamten eu-haushalts. unterstützt werden unter anderem projekte zur verbesserung der verkehrsinfrastruktur, umwelt, energie, beschäftigung und bildung vor allem weniger entwickelte regionen und mitgliedstaaten. (afp) & 291 & very low & Low & Power & NA & NA & 2017-11-15 & 2017 & 2 & POL
Frame & v.low & National & <500 & -1.0405052 & -1.0641830 & 1.2185583 & -1.5781059 & 1.4491718 & 12.0 & 1.6295860 & 1.9019065 & Payer & European & European & European & European|POL & Positive\\
Germany & http://www.hna.de/kassel/mio-flossen-bruessel-nach-kassel-bringt-6575201.html & 198 & Hessisch Niedersachsische Allgemeine & Private/Non-Public & Online and Offline & Regional/Local & high = CP is most important issue in story (can also cover other issues) & Bureaucracy and/or delays & Balanced & EU + National & No myth & Economic development & Positive & EU & No myth & Social justice & Positive & EU & No myth & Germany & 25 mio. flossen aus brüssel nach kassel: was die eu für uns bringt & 2016-07-16 & europäischer sozialfonds & kassel. nach der entscheidung der briten, die eu zu verlassen, fragen sich auch hier viele menschen: was bringt die europäische union für mich? die antwort im fall kassel lautet: eine ganze menge. das ergab eine umfrage der hna. so sind in den letzten zehn jahren rund 25 millionen euro aus brüssel direkt nach kassel geflossen (siehe unten). aber auch viele kleinere projekte werden mitfinanziert. geld fließt in straßen, arbeitsplätze, existenzgründungen und weiterbildung. dennoch: professor wolfgang schroeder, politikwissenschaftler an der universität kassel, meint, dass es auch hier eine wut gibt, die zunehmend gegen die eu gerichtet sei. damit verhalte sich ein teil der bevölkerung "gegen die eigene ökonomische interessenlage". zu der kritik an der ausufernden bürokratie in der eu sagt der wissenschaftler im interview mit der hna, etliche nationale politiker würden die eu als sündenbock nutzen, um von einer eigenen schwäche abzulenken. die europäische union diene somit als eine art "blitzableiter". den von den briten eingeschlagenen weg, also eine rückkehr in den nationalstaat, hält professor wolfgang schroeder für ein "unverantwortliches abenteuer". damit würde unser wohlstand reduziert. der hauptverlierer wären die ärmeren schichten in der bevölkerung. schroeder: "ohne eu gibt es weniger wachstum." entscheidend sei aber: "wir brauchen ein positives bild von europa, das bei den leuten ankommt." von grimmwelt bis dock 4 millionen-förderung aus brüssel für projekte in kassel kassel profitiert von zuschüssen aus verschiedenen eu-förderprogrammen. aus dem efre-programm (europäischer fonds für regionale entwicklung) seien zum beispiel von 2007 bis 2013 insgesamt 18,54 millionen euro als förderung für projekte in kassel bewilligt worden, berichtet stadt-sprecher ingo happel-emrich. nach seinen angaben waren dies folgende projekte: * lokale ökonomie: 2,1 millionen euro * gebiet soziale stadt wesertor: 1,4 millionen euro * umbau tulpenallee: 240 000 euro * standortentwicklung hauptbahnhof nord (fraunhofer-institut): 2,7 mio. euro * grimmwelt: sechs mio. euro * science park: 7,7 mio. euro. für kultur und schulen ein weiteres förderprogramm war laut stadt das urban-ii-programm. dieses lief in den jahren 2000 bis 2006 in den stadtteilen mitte, rothenditmold, nord-holland, wesertor, unterneustadt und bettenhausen. das fördervolumen betrug insgesamt rund zehn millionen euro. hieraus erhielt die stadt kassel unter anderem geld für folgende bau-projekte: * dock 4: 568 000 euro * mehrzweckhalle/kulturzentrum oberzwehren: 502 000 euro * max-eyth-schule: 385 000 euro für den bau der cafeteria zeughausruine * carl-anton-henschel-schule: 1,39 mio euro. ein drittes förderprogramm, von dem die stadt kassel profitiert, ist "jugend stärken im quartier". kassel erhält aus diesem esf-programm (europäischer sozialfonds) von 2015 bis 2018 insgesamt 902 993 euro für angebote, die jungen menschen helfen, soziale benachteiligungen und individuelle beeinträchtigungen am übergang von der schule in den beruf zu überwinden. zu den problemen, die von der stadt durch die eu gesehen werden, erklärt happel-emrich: "problematisch wirkt sich mitunter die fragmentierung der eu-regelungen aus, die nicht immer aufeinander abgestimmt sind." mangelnde abstimmung beispielhaft führt der stadt-sprecher luftreinhaltung und abgas-grenzwerte für kraftfahrzeuge an. "die eu müsste über die euro-abgasnorm die von der automobilindustrie einzuhaltenden grenzwerte so festlegen, dass die kommunen in der lage sind, die ebenfalls von der eu festgelegten grenzwerte für die luftqualität einzuhalten." & 520 & high & High & Governance & Socio-Economic & Socio-Economic & 2016-07-16 & 2016 & 2 & POL
Frame & high-very high & Regional & 500-1000 & -1.0405052 & -1.0641830 & 1.2185583 & -1.5781059 & 1.4491718 & 12.0 & 1.6295860 & 1.9019065 & Payer & Domestic & European & Mixed & Domestic|POL & Neutral\\
Germany & http://www.faz.net/1.5614013 & 293 & Frankfurter Allgemeine & Private/Non-Public & Online and Offline & National & medium = CP is important part of story & Institutional bargaining over funding & Balanced & EU + National & No myth & Political leverage & Balanced & EU + National & No myth & NA & NA & NA & NA & Germany & neue regionalpolitik: deutschland bekommt weniger geld von der eu & 2018-05-29 & kohäsionsfonds & die eu-kommission will die milliardenschweren regional- und kohäsionsfonds im nächsten jahrzehnt neu ausrichten. die mittel für deutsche regionen sollen dabei um 21 prozent gesenkt werden, wie die behörde am dienstag in straßburg mitteilte. dabei profitiert deutschland eigentlich noch davon, dass regionen mit vielen flüchtlingen künftig stärker gefördert werden sollen. die eu-kohäsionspolitik soll eine angleichung der lebensverhältnisse in der eu fördern. sie sind nach den agrarausgaben der größte posten im eu-budget. in konstanten 2018er preisen stehen 331 milliarden euro für den nächsten eu-finanzzeitraum von 2021 bis 2027 zur verfügung (373 milliarden euro in jeweiligen preisen einschließlich inflationseffekt). dem vorschlag müssen noch die mitgliedstaaten und das europaparlament zustimmen. staaten wie die slowakei, die baltischen staaten oder polen hätten wirtschaftlich aufgeholt und bräuchten künftig weniger finanzielle unterstützung, sagte eu-haushaltskommissar günther oettinger im europaparlament. "andere, die in den letzten jahren länger in der stagnation gewesen sind, (wie) italien, bekommen mehr geld." er gehe fest davon aus, dass einige der erst nach dem zerfall des ostblocks beigetretenen staaten mit ihrer wirtschaftsleistung pro kopf im nächsten jahrzehnt den europäischen durchschnitt übersteigen würden, sagte oettinger. einige könnten damit nettozahler in den eu-haushalt werden. dies zeige den erfolg der eu-kohäsionspolitik, die "noch immer den gedanken der solidarität und stärkung der schwachen in sich trägt". trotz kürzungen von gleichfalls mehr als einem fünftel bleibt polen in absoluten zahlen aber weiter spitzenreiter bei den kohäsionsmitteln. es soll von 2021 bis 2027 noch 64,4 milliarden euro erhalten. die zweithöchste summe bekommt dann aber schon italien (38,5 milliarden euro), gefolgt von spanien (34 milliarden euro). ihre zuwendungen steigen um sechs beziehungsweise fünf prozent. die kommission änderte dabei nun etwas den verteilungsschlüssel. er beruht weiter zu 80 prozent auf der wirtschaftsleistung pro kopf. hinzu kommen nun "anpassungen" auf basis von jugendarbeitslosigkeit, niedrigem bildungsstand und auswirkungen des klimawandels. aufgenommen hat die kommission auch den deutschen vorschlag, gebiete mit einer hohen zahl von flüchtlingen künftig stärker zu berücksichtigen. die berücksichtigung der aufnahme und integration von migranten werde aber nur "sehr klein" ausfallen, schränkte die für regionalpolitik zuständige eu-kommissarin corina cretu ein. dennoch wolle ihre behörde regionen einen "anreiz" geben, flüchtlinge aufzunehmen und regionen dabei unterstützen. cretu wies zurück, dass osteuropäische länder durch starke kürzungen für die verweigerung der flüchtlingsaufnahme bestraft werden sollten. die kohäsionspolitik sei weiter vor allem an wirtschaftlichen kriterien ausgerichtet, sagte sie. "je reicher man wird, desto weniger geld bekommt man." darüber hinaus will die kommission die auszahlung von mitteln auch stärker an die umsetzung von strukturreformen in den mitgliedstaaten knüpfen. hierzu wurden vorbedingungen für die auszahlung der mittel neu gefasst, erklärte die behörde. "die mitgliedstaaten werden bei der kommission keine zahlungsanträge für eu-finanzierte projekte mit nicht erfüllten vorbedingungen einreichen können." & 449 & medium & Medium & Power & Power & NA & 2018-05-29 & 2018 & 3 & POL
Frame & low-medium & National & <500 & -1.0405052 & -1.0641830 & 1.2185583 & -1.5781059 & 1.4491718 & 12.0 & 1.6295860 & 1.9019065 & Payer & Domestic & European & Mixed & Domestic|POL & Neutral\\
\addlinespace
Germany & https://www.zeit.de/wirtschaft/2016-07/sigmar-gabriel-eu-investitionen-brexit-wirtschaftskrise & 208 & ZEIT ONLINE & Private/Non-Public & Online and Offline & National & very low = CP mentioned once & Institutional bargaining over funding & Balanced & EU + National & No myth & NA & NA & NA & NA & NA & NA & NA & NA & Germany & eu-politik: gabriel will richtig geld ausgeben & 2016-07-01 & kohäsionsfonds & bundeswirtschaftsminister sigmar gabriel (spd) hat sich vor dem hintergrund des bevorstehenden britischen eu-austritts für eine europäische investitionsoffensive ausgesprochen. dafür wolle er ein "zentrales investitionsregelwerk" auf eu-ebene installieren, schrieb gabriel in einem brief an die mitarbeiter seines ministeriums. kanzleramtsminister peter altmaier (cdu) äußerte sich dazu allerdings skeptisch. "wachstumsschwäche, investitionsschwäche und beschäftigungskrise haben europa politisch gespalten wie nie zuvor seit den römischen verträgen", heißt es in dem schreiben. die überzeugung, dass europa für alle ein gewinn ist, könne nur dann wieder stark werden, wenn auswege aus der wirtschaftskrise gefunden würden. "deshalb gilt: vorrang muss jetzt ein neuer wirtschaftlicher aufschwung in europa haben", verlangte der minister. notwendig sei eine "wirtschaftliche trendwende". derzeit gebe es große wirtschaftliche und soziale ungleichheiten sowohl zwischen als auch innerhalb der eu-staaten, kritisierte gabriel. der stabilitäts- und wachstumspakt der eu müsse daher "wachstumsfreundlicher angewandt werden". von der investitionsoffensive sollen demnach unter anderem transeuropäische verkehrs- und energienetze profitieren. außerdem soll ein europäisches gigabyte-netz aufgebaut, die jugendarbeitslosigkeit bekämpft sowie die finanzierung von firmengründungen erleichtert werden. anfänglich sollten die notwendigen investitionen aus dem vorhandenen europäischen strukturfonds und kohäsionsfonds finanziert werden, schrieb gabriel weiter. gestützt werden sollten diese auch von dem europäischen fonds für strategische investitionen (efsi). dieser solle "bei einer zentralen instanz angesiedelt, mit höheren finanzmitteln ausgestattet und verstetigt werden". kanzleramtsminister altmaier sagte dazu dem swr, es sei richtig, dass in vielen europäischen ländern mehr wachstum gebraucht werde. die frage sei jedoch, "ob man das durch noch mehr öffentliche gelder erreicht, durch noch mehr brücken, die gebaut werden und öffentliche infrastruktur, oder ob man es dadurch erreicht, dass diese länder attraktiv werden für ausländische investitionen". der kanzleramtschef wies darauf hin, dass es in deutschland heute mehr jobs gebe als zu beginn des jahrhunderts, "und das haben wir nicht erreicht durch öffentliche ausgabenprogramme". jetzt müsse es auch anderen ländern gelingen, "aus eigener kraft ein attraktiver wirtschaftsstandort zu werden". auf ein positives echo stießen die pläne gabriels bei grünen und linken. "ein brief ins eigene haus reicht nicht", äußerte sich grünen-fraktionschef anton hofreiter allerdings zugleich vorsichtig. er forderte, gabriel müsse jetzt "endlich mal ernst machen und die investitionsblockade der regierung durchbrechen". die grünen würden einen "ernst gemeinten vorstoß für soziale investitionen, erneuerbare energien und öffentliche infrastruktur" unterstützen. linken-parteichef bernd riexinger wies darauf hin, dass die spd bislang "das deutsche kürzungs- und privatisierungsdiktat für griechenland mitgetragen" habe. es sei aber zu begrüßen, dass "offensichtlich ein umdenken stattgefunden hat". riexinger warnte jedoch vor einer einbeziehung des efsi. dieser diene vor allem dazu, geldgeber aus der privatwirtschaft anzulocken und führe zu einer "privatisierung durch die hintertür". & 425 & very low & Low & Power & NA & NA & 2016-07-01 & 2016 & 2 & POL
Frame & v.low & National & <500 & -1.0405052 & -1.0641830 & 1.2185583 & -1.5781059 & 1.4491718 & 12.0 & 1.6295860 & 1.9019065 & Payer & Domestic & European & Mixed & Domestic|POL & Neutral\\
Germany & http://www.epochtimes.de/politik/welt/staedte-praesident-eu-muss-foerdergelder-flexibler-fuer-fluechtlinge-einsetzen-a1307004.html & 251 & Epoch Times Deutschland & Private/Non-Public & Online only & National & very low = CP mentioned once & Social awareness/inclusion & Balanced & EU & No myth & NA & NA & NA & NA & NA & NA & NA & NA & Germany & städte-präsident: eu muss fördergelder flexibler für flüchtlinge einsetzen & 2016-02-15 & kohäsionsfonds & die vertreter von städten und gemeinden in europa fordern von der eu, in der flüchtlingskrise regionale fördergelder flexibler einzusetzen und schneller zu bewilligen: "europa sollte die regionalen eu-fördergelder, also die kohäsionsfonds, flexibler einsetzen, etwa vor ort für ausbildung, job-training und sprachkurse", sagte der präsident des ausschusses der regionen (adr) der europäischen union, markku markkula, in einem interview mit der "neuen osnabrücker zeitung" (montag). auf diese weise könnten flüchtlinge gleich eigenes geld verdienen und sich schneller in der gesellschaft zurechtfinden. "bisher dauert die bewilligung solcher gelder zwischen einem halben und einem jahr. das geht so nicht, das ist viel zu lange", kritisierte der finne. für die regionale förderung stünden in der förderperiode von sieben jahren bis 2020 rund 350 milliarden euro bereit. "das ist eine riesensumme, und nur ein kleiner teil des geldes würde ja schon enorm helfen", sagte markkula. bisher sei aber unklar, ob die eu dieses geld so einsetzen wolle. viele behörden stellten keine anträge, weil sie nicht wüssten, was sie mit diesem geld tun dürften. markkula sprach sich dafür aus, dass eu-länder, die sich nicht an der verteilung von 160.000 flüchtlingen beteiligen wollen, in anderer form hilfe stellen sollten. sie könnten zum beispiel die flüchtlingslager im libanon oder in der türkei finanziell unterstützen. "jeder - und ich betone: jeder - muss in dieser krise verantwortung übernehmen", betonte markkula. geldbußen für die staaten lehnte er ab. der finne sprach sich auch gegen die schließung von grenzen aus, weil dies der wirtschaft schaden werde. der ausschuss der regionen vertritt die interessen von städten und gemeinden aus den 28 eu-staaten in brüssel. mitglieder sind etwa regionale abgeordnete oder bürgermeister. der adr hat beratende funktion und kann auf die eu-kommission, den ministerrat oder das eu-parlament bei der gesetzgebung einwirken. & 292 & very low & Low & Socio-Economic & NA & NA & 2016-02-15 & 2016 & 2 & ECO
Frame & v.low & National & <500 & -1.0405052 & -1.0641830 & 1.2185583 & -1.5781059 & 1.4491718 & 12.0 & 1.6295860 & 1.9019065 & Payer & European & European & European & European|ECO & Neutral\\
Germany & http://www.rp-online.de/politik/eu/viktor-orban-nennt-fluechtlinge-muslimische-invasoren-aid-1.7307282 & 269 & RP Online & Private/Non-Public & Online and Offline & National & very low = CP mentioned once & Political leverage & Balanced & Other country & No myth & NA & NA & NA & NA & NA & NA & NA & NA & Germany & viktor orban nennt flüchtlinge 'muslimische invasoren' & 2018-01-08 & kohäsionsfonds & berlin. ungarns ministerpräsident orban verteidigt erneut die weigerung seines landes, flüchtlinge aufzunehmen. "wir betrachten diese menschen nicht als muslimische flüchtlinge. wir betrachten sie als muslimische invasoren." um aus syrien in ungarn einzutreffen, müsse man vier länder durchqueren, sagte der ungarische regierungschef viktor orban in einem interview mit der "bild"-zeitung. die menschen würden nicht "um ihr leben" rennen, sondern "ein besseres leben suchen". die flüchtlinge hätten vorher um ihre aufnahme bitten sollen, stattdessen aber hätten sie die grenze illegal durchbrochen. "das war keine flüchtlingswelle, das war eine invasion", sagte orban, der vergangene woche gast bei der klausurtagung der csu-landesgruppe im oberbayerischen seeon war. er habe nie verstanden, "wie in einem land wie deutschland [...] das chaos, die anarchie und das illegale überschreiten von grenzen als etwas gutes gefeiert werden konnte". orbans regierung verweigert die aufnahme von flüchtlingen nach einem von der eu vorgeschlagenen schlüssel und steht unter anderem deshalb in der kritik. auf die kritischen stimmen zu seinem auftritt in seeon reagierte orban irritiert. "ich finde, wir verdienen mehr respekt", sagte er. trotz der streitereien mit brüssel bekannte sich der ungarische ministerpräsident aber zur europäischen union. "die eu ist ein wunderbares projekt, in dem wir gerne teil sind und bleiben werden." ungarns ministerpräsident verwahrte sich in dem interview gegen den vorwurf, sein land nehme geld von der eu, weigere sich aber, flüchtlinge aufzunehmen. der sogenannte kohäsionsfonds, der der ungarischen wirtschaft zugutekomme, sei kein geschenk. "er ist ein fairer ausgleich, da wir unseren markt dem freien wettbewerb geöffnet haben. das hat absolut nichts mit der flüchtlingsfrage zu tun." orban bekräftigte, dass ungarn auch künftig keine flüchtlinge aufnehmen werde. "wir glauben, dass eine hohe zahl an muslimen notwendigerweise zu parallelgesellschaften führt", sagte er. "so etwas möchten wir nicht. und wir möchten uns nichts aufzwängen lassen." & 293 & very low & Low & Power & NA & NA & 2018-01-08 & 2018 & 3 & POL
Frame & v.low & National & <500 & -1.0405052 & -1.0641830 & 1.2185583 & -1.5781059 & 1.4491718 & 12.0 & 1.6295860 & 1.9019065 & Payer & European & European & European & European|POL & Neutral\\
Germany & http://www.welt.de/wirtschaft/article130410180/Italien-verfoerdert-voellig-ziellos-die-EU-Milliarden.html & 211 & DIE WELT & Private/Non-Public & Online and Offline & National & high = CP is most important issue in story (can also cover other issues) & Fraud/Corruption & Negative & EU + Other country & 8.Mismanaged & Mismanagement & Negative & Other country & 8.Mismanaged & NA & NA & NA & NA & Germany & eu-gelder : italien verfördert völlig ziellos die eu-milliarden & 2014-07-22 & europäischer fonds für regionale entwicklung & der trasimener see in umbrien ist ein kleinod mit klarem wasser, umgeben von bergen. er ist ein beliebtes ausflugsziel. dementsprechend bedurfte es seitens italien nicht allzu viel überredungskunst, um die eu für den bau eines radwegs zu gewinnen. 1997 wurde die kofinanzierung für das projekt beantragt und von brüssel genehmigt. danach lief so ziemlich alles schief, was schief gehen konnte. statt den radweg wie vereinbart in einem abstand vom see zu errichten, setzten ihn die italiener direkt ans ufer. wohlgemerkt ohne die eu-kommission darüber zu informieren. schlimmer noch. die zweiradschneise verläuft nun zum teil durch naturgeschützten schilf. zu weiten teilen ist sie überflutet und unbenutzbar. weil der radweg noch nicht die ganze runde um den see macht, erbat italien inzwischen weitere 1,8 millionen euro. auch die wurden bewilligt. inge gräßle sammelt beispiele wie diese. sie tut es berufsmäßig. die cdu-europaabgeordnete ist vorsitzende des eu-haushaltskontrollausschusses und überprüft, wie die mitgliedsländer gemeinschaftsgelder verwenden. schummeleien wie im fall des trasimener sees leiste sich italien viel zu häufig, findet gräßle. in der haushaltsperiode 2000 bis 2006 entfielen auf spanien, italien und griechenland zusammen 95 prozent aller rechtsverstöße. je ein drittel der regelverletzungen, die den europäischer fonds für regionale entwicklung (efre) und dem europäischen sozialfonds (esf) betreffen, lasse sich italien zuschulde kommen. zwischen 2000 und 2010 forderte die eu deswegen rund 900 millionen euro von italien zurück. "italien ist als gründungsmitglied der eu seit 20 jahren hinter spanien mit an der spitze bei fehlern und unregelmäßigkeiten in der umsetzung des eu-haushalts", sagt gräßle der "welt". "die ratspräsidentschaft wäre doch eine gute gelegenheit für die italienische regierung, sich dieser thematik zu stellen und eine reform der eigenen verwaltung anzupacken." seit dem 1. juli hat italien den eu-vorsitz. premier matteo renzi, 39, dringt darauf, dass italien von brüssel und berlin mehr finanziellen spielraum erhält. die auflagen des maastricht-vertrags und des europäischen fiskalpakts will er laxer interpretiert sehen. "flexibilität" ist das zauberwort der stunde. renzi will mehr ausgeben dürfen, um der italienischen wirtschaft einen schub zu verleihen. der grund: der aufschwung droht auszubleiben. volkswirte revidieren ihre prognosen nach unten. das forschungsinstitut prometeia sagt in seiner schätzung vom juli statt einem plus von 0,8 prozent für 2014 nur noch einen zuwachs von 0,3 prozent voraus. das würde bei weitem nicht ausreichen, um der arbeitslosigkeit und verschuldung beizukommen. doch die entscheidende frage lautet: wofür will italien das zusätzliche geld, das die flexibilität bringen soll, eigentlich ausgeben? der umgang mit den eu-fördergeldern unterstreicht, dass es nicht allein ausreicht, mehr mittel zur verfügung zu haben. damit das geld einen positiven effekt hat, muss es auch klug und gezielt verwendet werden. genau da hakt es in italien. erstens verteilt das land das geld auf tausende mikroprojekte, deren erfolg selten überprüft wird. zweitens wird ein großer teil der finanziellen mittel nicht abgerufen. fast die hälfte bleibt so ungenutzt. und drittens begeht italien bei den abgerufenen geldern oft rechtsverstöße, weil es sie nicht im sinne der förderregeln einsetzt. von geldmangel kann keine rede sein. laut der internetseite "opencoesione.gov.it", für die das amt für regionalentwicklung in rom zuständig ist, hatte italien im zeitraum von 2007 bis 2013 insgesamt 99,3 milliarden euro für die wirtschaftsförderung zur verfügung. fast 28 milliarden euro stammten aus europäischen töpfen, vor allem aus dem efre und esf. nach angaben von eu-haushaltsexpertin gräßle waren es sogar knapp 29 milliarden euro. mit dem geldausgeben tun sich die italiener hingegen schwer. die regionen, provinzen und städte balgen sich um fördergelder. um ja nicht jemand zu benachteiligen, werden die mittel breit gestreut. statt sich auf einige wenige großprojekte zu konzentrieren, verzettelt sich italien in tausenden von miniprogrammen. roberto perotti und filippo teoldi von der mailänder wirtschaftsuniversität bocconi haben sich in einer studie die verwendung der fördergelder für aus- und weiterbildung angeschaut. für den zeitraum von 2007 bis 2012 kommen sie auf 7,5 milliarden euro, die auf rund 500.000 projekte verteilt worden seien. ob die mittel etwas bewirkt hätten, sei unklar. die bisherigen evaluationen seien nicht aussagekräftig, monieren die wissenschaftler. eine expertengruppe kam beispielsweise zum schluss, dass mit dem geld aus den eu-töpfen 220.000 stellen in italien geschaffen wurden. für die bocconi-experten ist das noch kein beweis für den erfolg der programme. "jede stelle hat also 33.300 euro an fördergeld gekostet. ist das viel oder wenig? das ist schwierig zu beantworten", schreiben perotti und teoldi. schließlich wisse man nicht, um welche art von jobs es sich gehandelt habe. perottis fazit fällt dementsprechend kritisch aus: "das fördergeld wird in tausende projekte gesteckt, die derzeit weder gesteuert noch kontrolliert werden." nicht allein die frage, wie das geld ausgegeben wird, ist in italien akut. sondern auch, ob es überhaupt genutzt wird. italien hat eine der niedrigsten absorptionsraten von eu-geldern in der ganzen union. von den zur verfügung stehenden mitteln für zahlungen in der eu- haushaltsperiode 2007 bis 2013 hatte italien bis ende 2013 erst 47 prozent abgerufen: damit liegt das land auf dem viertletzten platz aller mitgliedstaaten. stand ende mai waren laut den angaben auf "opencoesione.gov.it" gerade einmal 56 prozent der verfügbaren mittel von der eu zertifiziert. das mag einerseits an der ideenlosigkeit der italienischen verwaltung liegen, die nichts mit den mitteln anzufangen weiß. andererseits spielen auch gesetzliche auflagen eine rolle. die städte und gemeinde unterliegen einem nationalen stabilitätspakt. er soll sicherstellen, dass sich die lokalen regierungen nicht in schulden zu stürzen. um die regeln nicht zu missachten, machen die kommunen häufig um die eu-fördertöpfe einen bogen. denn jeder euro aus brüssel muss mit einem euro aus der eigenen tasche kofinanziert werden. und das können sich viele städte einfach nicht leisten. wenn das geld schließlich doch abgerufen wird, wird es nicht selten gesetzeswidrig abgezweigt. korruption und vetternwirtschaft grassieren in italien. die skandale um die weltausstellung expo 2015 in mailand und ein flutwehr, das venedig vor hochwasser schützen soll, bestimmen momentan die schlagzeilen. bei den beiden mammutprojekten sollen firmen mit der politik auf kosten des steuerzahlers gemauschelt haben. symptomatisch ist der fall des sizilianischen politikers francantonio genovese. im mai stimmte das italienische parlament dafür, die immunität des abgeordneten der sozialdemokratischen partei aufzuheben. genovese wurde anschließend verhaftet. der vorwurf der staatsanwaltschaft lautet: er soll in seiner heimatstadt messina öffentliche gelder, die für den bildungssektor bestimmt waren, an firmen geleitet haben, bei denen freunde und verwandte das sagen haben. genovese bestreitet die vorwürfe. miniprogramme mit geringer wirkung, ungenutzte mittel, regelverstöße und korruption - braucht italien wirklich mehr geld für seine wirtschaft? eu-währungskommissar jyrki katainen jedenfalls hält die forderungen von roms regierungschef renzi für verkehrt: "italien hat einen zu hohen schuldenstand und es mangelt dem land an produktivität und wettbewerbsfähigkeit", sagte katainen der "welt". das seien keine zyklischen probleme, sondern strukturelle. und gegen die helfe kein wachstumspaket. & 1117 & high & High & Governance & Governance & NA & 2014-07-22 & 2014 & 1 & POL
Frame & high-very high & National & +1000 & -1.0405052 & -1.0641830 & 1.2185583 & -1.5781059 & 1.4491718 & 12.0 & 1.6295860 & 1.9019065 & Payer & European & European & European & European|POL & Negative\\
Germany & http://www.t-online.de/nachrichten/ausland/eu/id\_81321846/eu-hilfen-bundesregierung-will-osteuropaeern-gelder-streichen.html & 205 & T-online.de & Private/Non-Public & Online only & National & medium = CP is important part of story & Political leverage & Negative & National + Other country & No myth & NA & NA & NA & NA & NA & NA & NA & NA & Germany & eu-hilfen: bundesregierung will osteuropäern gelder streichen & 2017-05-31 & kohäsionsfonds & kanzlerin angela merkel und der ungarische premierminister viktor orban. (quelle: dpa) die bundesregierung will künftig finanziellen druck für die bessere umsetzung von reformen in den eu-staaten ausüben. in einem papier zur zukunft der eu-kohäsionsfonds ab 2020 will deutschland vorschlagen, die auszahlung solcher strukturmittel auch an rechtsstaatliche reformen zu knüpfen. mehr zum thema "stoppt brüssel": anti-flüchtlings-kampagne in ungarn proteste in budapest: massenproteste gegen orbán dies könnte etwa eu-ländern wie polen oder ungarn treffen, die hohe summen aus dem eu-haushalt erhalten, aber nach meinung der eu-kommission und vieler mitgliedstaaten gegen grundlegende rechtsstaatliche prinzipien der eu verstoßen. in der flüchtlingskrise hatte die bundesregierung sich noch dagegen ausgesprochen, strukturgelder etwa für osteuropa mit der umsetzung von eu-beschlüssen wie bei der aufnahme von flüchtlingen zu verbinden. das papier des wirtschaftsministeriums trägt den titel "stellungnahme der bundesregierung zur kohäsionspolitik der eu nach 2020" und ist innerhalb der bundesregierung bereits abgestimmt. die deutschen vorschläge fließen in die diskussion in brüssel ein, wie nach dem auslaufen der derzeitigen eu-finanzperiode die förderung 2020 geändert werden soll. dabei werden konflikte etwa zwischen den nettozahlerländern und den nehmerstaaten in der eu erwartet. "solidarität für alle" ausdrücklich bekennt sich die bundesregierung in der stellungnahme zu den hilfen reicher für arme eu-länder. "die herausforderungen der vergangenen jahre haben unterstrichen, dass solidarität für alle gelten muss", heißt es deshalb. mit diesen struktur- und kohäsionsfonds soll etwa die infrastruktur in ärmeren eu-staaten und -regionen gezielt gefördert werden. allerdings soll die eu-kommission künftig genauer prüfen, wieso die milliardenhilfen in einigen eu-staaten eine entwicklung fördern und in anderen nicht, fordert die bundesregierung. sie will zudem künftig die kofinanzierung - also einen gewissen nationalen eigenanteil - bei der zahlung von strukturhilfemittel beibehalten. bürokratische auflagen sollen reduziert und die frist wieder verkürzt werden, in der das geld aus brüssel ausgezahlt werden kann. der politisch heikelste punkt ist allerdings die nun von der bundesregierung vorgeschlagene verknüpfung von reformen und hilfen. hintergedanke ist es, die wirtschaftliche entwicklung in der eu stärker anzugleichen, wofür die bundesregierung und vor allem die nordischen eu-staaten strukturreformen in der gesamten eu für nötig halten. die eu-kommission legt dafür jährlich sogenannte länderspezifische empfehlungen vor. die umsetzung dieser empfehlungen soll nun nach dem willen berlins ab 2020 aber verbindlicher gemacht werden. brisanz für osteuropäische staaten die verknüpfung mit auszahlungen aus den kohäsionsfonds soll ein "zusätzlicher anreiz" zur umsetzung der von der eu-kommission vorgelegten länderspezifischen empfehlungen sein, heißt es in dem papier. sehr brisant könnte dies vor allem für osteuropäische eu-staaten wie polen und ungarn werden. diese bekommen hohe summen aus dem eu-haushalt - liegen mit der eu-kommission aber im heftigen streit wegen umstrittener reformen im rechtsstaatlichen bereich. so ermittelt die eu-kommission gegen die nationalkonservative pis-regierung in warschau wegen umstrittener justizgesetze. in ungarn hatte die eu-kommission zuletzt die drohende schließung der soros-universität in budapest bemängelt und die regierung aufgefordert, ein entsprechendes hochschulgesetz wieder zu ändern. in dem papier heißt es, dass generell eine stärkere verbindung zwischen empfehlungen der kommission und der auszahlung von geld geschaffen werden sollte. dann folgt der satz: "darüber hinaus sollte geprüft werden, ob der erhalt von eu-kohäsionsmitteln auch an die einhaltung von rechtstaatlichen grundprinzipien geknüpft werden kann." dies ist in der eu umstritten - vor allem in osteuropäischen staaten. & 544 & medium & Medium & Power & NA & NA & 2017-05-31 & 2017 & 2 & POL
Frame & low-medium & National & 500-1000 & -1.0405052 & -1.0641830 & 1.2185583 & -1.5781059 & 1.4491718 & 12.0 & 1.6295860 & 1.9019065 & Payer & Domestic & European & Mixed & Domestic|POL & Negative\\
\addlinespace
Germany & http://www.tagesspiegel.de/politik/westbalkan-konferenz-in-paris-eu-muss-balkanstaaten-perspektiven-aufzeigen/13826210.html & 220 & Der Tagesspiegel & Private/Non-Public & Online and Offline & Regional/Local & very low = CP mentioned once & Economic development & Positive & EU + Other country & No myth & NA & NA & NA & NA & NA & NA & NA & NA & Germany & eu muss balkanstaaten perspektiven aufzeigen & 2016-07-04 & strukturfonds & im nebel der eu-krise ist die beitrittsperspektive der westbalkanstaaten nicht mehr auszumachen. trotz brexits darf die bedeutung dieser länder für die stabilität der eu jedoch nicht vergessen werden. ein gastbeitrag. im jahr 1992 wurde die fußballmannschaft jugoslawiens wegen des krieges in diesem land aus dem em-turnier ausgeschlossen. dänemark rückte an jugoslawiens stelle und wurde sensationell europameister. nur tage zuvor hatten die dänen gegen den maastrichter vertrag gestimmt und dadurch die schnelle vertiefung der europäischen integration ausgebremst. neben der europäischen währungsunion widerstrebte der mehrheit der dänischen wählerschaft damals die erweiterung der gemeinschaftspolitik auf die gemeinsame außen- und sicherheitspolitik. infolge des "nej" zur vertieften integration wurde die fähigkeit der eu, als politischer akteur in südosteuropa aufzuspielen, deutlich geschwächt. die eu-staaten scheiterten damals als schlichter im jugoslawien-krieg. erst als die usa einschritten, konnten zerstörung und gewalt gestoppt werden. seitdem gilt die eingliederung des post-jugoslawischen raumes als prüfstein für die externe handlungsfähigkeit der union und ihre ambition, für dauerhaften frieden in europa zu sorgen. im jahr 1993 einigten sich die eu-staaten auf einen plan zur öffnung der europäischen gemeinschaft der zwölf länder für neue mitglieder. in der dänischen hauptstadt kopenhagen verabredete der krisengeschüttelte europäische rat die wichtigsten beitrittskriterien: zukünftige eu-mitglieder müssen menschenrechte, demokratie, rechtsstaat und eine wettbewerbsfähige marktwirtschaft garantieren. beim gipfel in thessaloniki 2013 schließlich wurde die "beitrittsperspektive" für den westbalkan verkündet und eine mitgliedschaft fest in aussicht gestellt. diese perspektive ist im nebel der krisen der europäischen integration nicht mehr auszumachen. dennoch kommen, just vor dem halbfinale der diesjährigen fußball-em, auf einladung françois hollandes am 4. juli die regierungschefs der westbalkanländer sowie der eu-mitglieder deutschland, italien, kroatien, österreich und slowenien in paris zusammen. auch der eu-erweiterungskommissar johannes hahn und die eu-außenbeauftragte federica mogherini sind mit von der partie. die menschen in den nachfolgestaaten des früheren jugoslawien und in albanien hoffen, dieses mal nicht aus dem spiel genommen zu werden. wie weiter mit der eu-erweiterungspolitik? bosnien und herzegowina, mazedonien, montenegro, albanien, kosovo und serbien wollen in die eu. die länder teilen viele strukturelle probleme in gesellschaft, politik und wirtschaft. ihre bemühungen, die politischen und ökonomischen kriterien für den eu-beitritt zu erfüllen, kommen, auch infolge der europäischen finanz- und schuldenkrise seit 2009, nur mit mühe voran. im august 2014 hat deshalb bundeskanzlerin angela merkel zu einem ersten treffen der westbalkanländer in berlin eingeladen. die konferenz fand unter dem eindruck der krise in der ukraine und der zuspitzung der spannungen mit russland statt. die eu musste geopolitisch flagge zeigen. obwohl es viele vorbehalte gegen weitere beitritte gibt, durfte kein zweifel an der eu-perspektive für die region entstehen. auch die feierlichkeiten zur erinnerung an den beginn des ersten weltkrieges verstärkten das interesse an einer dauerhaften friedensordnung des balkans in der eu. die als "berlin-prozess" bezeichnete initiative sollte die länder bis 2018, zum hundertsten jahrestag des weltkriegsendes, auf ihre zukunft in der eu vorbereiten. nach einem weiteren treffen in wien im sommer 2015 markiert jetzt paris die halbzeit in diesem diplomatischen spiel. der abschluss 2018 sollte ursprünglich in großbritanniens stattfinden. mit dem brexit dürfte nicht nur dieser tagungsort, sondern auch die erwartung in eu und beitrittsländern abhandengekommen sein, dass die erweiterungspolitik die gewünschten ergebnisse liefern kann. an der geostrategischen bedeutung des westbalkans für die eu und das transatlantische bündnis hat sich aber nichts geändert - im gegenteil: angesichts der ungewissheit, wie sich das westliche verhältnis mit russland weiterentwickeln wird, ist die anbindung des westbalkans an die eu wichtiger als je zuvor. die illustre gesellschaft in paris sollte sich deshalb eindeutig zur erweiterung der eu im westbalkan bekennen und klären, wie es weitergeht. vor allem müssen die projekte, die im rahmen des berliner prozesses angestoßen wurden, mit ergebnissen aufwarten. in der zweiten halbzeit sollten tore fallen. so verfolgt die eu erklärtermaßen den plan, bilaterale auseinandersetzungen wie etwa grenzstreitigkeiten zwischen den ländern zügig beizulegen. ungeklärte beziehungen, so wird befürchtet, könnten sonst später den beitrittsprozess blockieren. das wichtigste vorhaben in diesem zusammenhang, der "normalisierungsprozess" zwischen belgrad und priština, ist seit dem letzten herbst allerdings weitgehend zum stillstand gekommen, ohne dass die eu energische wiederbelebungsversuche unternommen hat. eine nachhaltige einhegung dieses konflikts wird auch ein entscheidender test der gerade vorgestellten "globalen strategie" der eu sein. spoods.de gründung eines balkan-jugendwerks priorität soll im berliner prozess auch die gemeinsame wirtschaftliche entwicklung bekommen, insbesondere durch eine bessere verzahnung der infrastrukturen der westbalkanländer untereinander und mit der eu. denn ohne zusätzliches, durch produktivitätssteigerungen in der region erzeugtes, wachstum wird der westbalkan nicht zur eu aufschließen können. die eu setzt daher auf die bereiche verkehr und energie, mit denen die region für investitionen weiter geöffnet werden soll. dies allein dürfte jedoch nicht reichen, wie die schleppende wirtschaftliche erholung in der region zeigt. deswegen benötigen die länder schon jetzt - noch vor dem beitritt - zugang zu den eu-strukturfonds, die kleinen und mittleren unternehmen zu innovation und kapital und die menschen zum europäischen arbeitsmarkt. am eigenen schopf werden sie sich niemals aus der ökonomischen und sozialen stagnation ziehen können. mindestens eine wichtige weichenstellung ist am montag in paris zu erwarten. inspiriert vom deutsch-französischen jugendwerk wollen die regierungen der westbalkanländer ein balkan-jugendwerk mit sitz in tirana gründen. durch begegnungen und austauschprogramme, die auch auszubildenden offenstehen, sollen sich die jungen menschen in der region besser kennenlernen, vorurteile abbauen und gemeinsame projekte entwickeln. wie wichtig junge menschen für die zukunft der europäischen integration sind, hat das referendum in großbritannien gerade erst wieder gezeigt - eine überwältigende mehrheit der jungen wähler hat für den verbleib in der eu gestimmt. dušan reljić forscht an der stiftung wissenschaft und politik (swp) u.a. zu aktuellen entwicklungen im westbalkan. er leitet das brüsseler büro der swp. tobias flessenkemper leitet die südosteuropa-projekte des "centre international de formation européenne" (cife) in nizza. in den jahren 2012-13 war er gastwissenschaftler an der swp. die stiftung berät bundestag und bundesregierung in allen fragen der außen- und sicherheitspolitik. der artikel erscheint auf der swp-homepage in der rubrik "kurz gesagt". & 988 & very low & Low & Socio-Economic & NA & NA & 2016-07-04 & 2016 & 2 & ECO
Frame & v.low & Regional & 500-1000 & -1.0405052 & -1.0641830 & 1.2185583 & -1.5781059 & 1.4491718 & 12.0 & 1.6295860 & 1.9019065 & Payer & European & European & European & European|ECO & Positive\\
Germany & http://www.stern.de/politik/ausland/griechenland-europartner-geben-tsipras-eine-letzte-chance-2181554.html & 255 & Stern magazine & Private/Non-Public & Online and Offline & National & very low = CP mentioned once & Institutional bargaining over funding & Factual & EU + Other country & No myth & NA & NA & NA & NA & NA & NA & NA & NA & Germany & griechenland-krise: europartner geben tsipras eine letzte chance & 2015-03-20 & strukturfonds & reparationsforderungen, pfändungsdrohungen und eine stinkefinger-debatte: einen monat lang drehte sich die griechenland-rettung im kreis, das klima war zunehmend vergiftet. angesichts der wachsenden gefahr, dass athen aus dem euro fliegt, wurde das thema bei einem "mini-gipfel" in brüssel zur chefsache. letztlich gaben bundeskanzlerin angela merkel und co. dem griechischen ministerpräsidenten alexis tsipras nochmal eine chance und schickten ihn zurück auf los. um auf finanzhilfe zu hoffen, muss er jetzt schnell liefern. fünf zeilen hat die erklärung des brüsseler spitzentreffens, an dem in der nacht zum freitag neben tsipras und merkel frankreichs präsident françois hollande und die spitzen von eu-kommission, eurogruppe, rat und europäischer zentralbank (ezb) teilnahmen. "wir halten vollständig an der vereinbarung der eurogruppe vom 20. februar 2015 fest", lautet der zentrale satz. "kein deut" habe sich an den beschlüssen der euro-finanzminister verändert, befand merkel, die mit alarmrotem blazer nach brüssel gekommen war. "europäische partner nicht erpressbar" die griechische regierung habe wohl "den ernst der lage jetzt erkannt" und begriffen, "dass die europäischen partner nicht erpressbar sind", sagte der deutsche eu-kommissar günther oettinger im deutschlandfunk. es sei aber "ärgerlich", dass sich merkel und die anderen dazu erst "eine nacht um die ohren schlagen" mussten. "vielleicht braucht das ego von herrn tsipras die augenhöhe am tisch einer nachtsitzung mit den regierungschefs." aber was hat tsipras außer dem treffen auf höchster ebene dafür bekommen, dass er zusagte, "in den kommenden tagen eine vollständige liste spezifischer reformen" vorzulegen? der griechische premier verbucht als erfolg, dass er die zusage erhalten habe, keine maßnahmen umsetzen zu müssen, die griechenlands wirtschaft in die rezession treiben könnten. zudem hätten ihm "alle teilnehmer" in aussicht gestellt, die finanzierung der griechischen wirtschaft "so schnell wie möglich" wiederherzustellen. "zeit für griechenland läuft ab" in der erklärung des spitzentreffen versprechen alle seiten, dass die arbeit "beschleunigt" werden soll. "die eurogruppe steht bereit, so schnell wie möglich wieder zusammenzukommen." ob das heißt, dass sich damit die hoffnung von tsipras erfüllt, bald teile des verbleibenden geldes aus dem verlängerten hilfsprogramm oder zinsgewinne der ezb mit griechischen staatsanleihen zu bekommen, blieb unklar. der griechische regierungschef setzt auch darauf, bald "hohe summen" aus den europäischen strukturfonds für sozialprogramme zu bekommen - eine bestätigung dafür gab es nicht. das treffen habe "keine neuen ergebnisse geliefert", befindet carsten brzeski, chefvolkswirt bei ing-diba. "die bedingungen der vereinbarungen vom 20. februar haben sich nicht verändert, und es ist noch immer unklar, worin die griechischen pläne genau bestehen." mehr als eine gelegenheit für einen "gesichtswahrenden kompromiss" habe der linkspolitiker tsipras nicht bekommen. und angesichts anstehender kreditrückzahlungen und der sich verschlechternden finanzlage "läuft die zeit für griechenland weiter ab". geduld der eurozone neigt sich dem ende auch merkel hörte sich nach dem nächtlichen treffen eher vorsichtig an. "jetzt müssen wir das einfach einmal beim wort nehmen", sagte sie mit blick auf die griechischen zusagen. "wir vertrauen darauf, dass es nun auch so kommt." für optimisten könne der mini-gipfel "der beginn konstruktiverer und weniger feindseliger verhandlungen" sein, meint volkswirt brzeski. "für pessimisten könnte es nur ein freundlicher und diplomatischer weg gewesen sein zu sagen, dass sich die geduld der eurozone schnell dem ende zuneigt." oettinger forderte, die reformliste müsse nun endlich "nachhaltig" und "belastbar" sein. wenn tsipras wieder nicht liefere, werde es "ganz schwierig". dann lasse sich "nichts mehr ausschließen". & 544 & very low & Low & Power & NA & NA & 2015-03-20 & 2015 & 1 & POL
Frame & v.low & National & 500-1000 & -1.0405052 & -1.0641830 & 1.2185583 & -1.5781059 & 1.4491718 & 12.0 & 1.6295860 & 1.9019065 & Payer & European & European & European & European|POL & Neutral\\
Germany & http://www.pnp.de/nachrichten/politik/2912156\_Scharfer-CDU-Widerspruch-gegen-EU-Reformplaene.html & 218 & Passauer Neue Presse & Private/Non-Public & Online and Offline & Regional/Local & very low = CP mentioned once & Institutional bargaining over funding & Negative & National & No myth & Mismanagement & Negative & National & 4.No added value & NA & NA & NA & NA & Germany & scharfer cdu-widerspruch gegen eu-reformpläne & 2018-04-16 & kohäsionsfonds & aus der cdu kommt scharfer widerspruch gegen die brüsseler vorschläge zur eu-reform. "das vorhaben der eu-kommission, den eurorettungsfonds ohne vertragsänderung zum europäischen währungsfonds auszubauen und die nationalen parlamente außen vor zu lassen, ist ein klarer affront aus brüssel. das werden wir nicht akzeptieren", sagte eckhardt rehberg (cdu), chefhaushälter der unionsfraktion im deutschen bundestag, im interview mit der "passauer neuen presse" (dienstagausgabe). beim europäischen währungsfonds pocht rehberg auf die beteiligung bundestags und fordert zudem: "dieser fonds darf nur eingreifen, wenn die gesamte eurozone in gefahr ist und nicht nur ein einzelnes land. außerdem gilt natürlich: haftung und risiko dürfen nicht entkoppelt werden." einen euro-finanzminister lehnt rehberg ebenso ab wie einen neuen investitionshaushalt. "die eu gibt dieses und nächstes jahr 15 milliarden euro an kohäsionsfonds-mitteln nicht aus. erstmal muss also geklärt werden, wie wir das vorhandene geld sinnvoll und effizient ausgeben, bevor wir neue geldtöpfe ins fenster stellen", sagte rehberg. auch sei der juncker-investitionsfonds noch lange nicht aufgebraucht. forderungen der spd zur eu-reform wies rehberg zurück: "die spd will auch, dass wir die finanziellen einbußen durch den brexit ausgleichen, und noch vieles mehr. in unserer finanzplanung ist das alles nicht vorgesehen. es steht auch nicht auf der liste der vorrangigen maßnahmen." er geht nicht davon aus, dass schnell beschlüsse zur eu-reform gefasst werden können. "die forderung der spd, beim eu-gipfel im juni müssten erste schritte zur reform der eurozone eingeleitet werden, nehme ich sehr gelassen. in acht wochen werden wir kaum zu endgültigen ergebnissen kommen", sagte rehberg. der cdu-politiker fordert, auch die kritischen stimmen zur eu-reform unter den mitgliedsstaaten zu berücksichtigen. "bei den finanzen und der europolitik sind wir gut beraten, die stimmen der acht nordländer zu hören, die nichts von macrons vorpreschen halten. frankreich und deutschland sind wichtig. aber daneben gibt es 25 weitere eu-mitglieder." das komplette interview lesen sie am dienstag in ihrer pnp. & 315 & very low & Low & Power & Governance & NA & 2018-04-16 & 2018 & 3 & POL
Frame & v.low & Regional & <500 & -1.0405052 & -1.0641830 & 1.2185583 & -1.5781059 & 1.4491718 & 12.0 & 1.6295860 & 1.9019065 & Payer & Domestic & Domestic & Domestic & Domestic|POL & Negative\\
Germany & http://www.dw.com/de/brok-beide-zypern-m\%C3\%BCssen-sich-bewegen/a-37183102 & 267 & Deutsche Welle (English) & Public & Online and Offline & National & very low = CP mentioned once & Social awareness/inclusion & Positive & National & No myth & NA & NA & NA & NA & NA & NA & NA & NA & Germany & brok: beide zypern müssen sich bewegen | europa | dw.com | 18.01.2017 & 2017-01-18 & strukturfonds & deutsche welle: aufgeschoben ist hoffentlich nicht aufgehoben. kommen die zypern-gespräche doch noch voran, obwohl beide seiten vergangene woche in genf nicht im einvernehmen auseinandergegangen waren? elmar brok: ich habe den eindruck, dass beide seiten ein abkommen wollen. mit den beiden gegenwärtigen präsidenten, nikos anastasiadis und mustafa akinci, sind wohl auch die gegebenheiten optimal, um ein abkommen auszuhandeln. man soll aber nicht einen sofortigen abschluss erwarten im sinne von: wenn es jetzt nicht klappt, haben wir eine krise. ich weiß von anastasiadis, aber auch von den türkischen zyprioten, dass sie einen abschluss wollen. deswegen bin ich zuversichtlich, dass es passiert. noch nie waren wir so nahe dran wie jetzt. zumindest für die athener regierung ist klar, dass die türkei die hauptverantwortung für das bisherige scheitern der gespräche trägt, da die türkischen soldaten weiterhin auf zypern bleiben. stimmt das oder müssen beide seiten zugeständnisse machen? wenn man einen kompromiss haben will, müssen sich immer beide seiten bewegen. das kann ja auch in einem bestimmten zeitrahmen geschehen, dass die dinge zusammenwachsen. das schwarzer-peter-spiel, dem einen oder dem anderen die schuld zuzuschreiben, das ist jetzt in dieser historischen sekunde, wo man eine einigung erreichen kann, nicht das richtige signal. auffallend ist, dass eu-kommissionspräsident jean-claude juncker bei den gesprächen in genf mit dabei war. ist das ein zeichen, dass die eu sich noch stärker in der zypern-frage engagieren soll? ich glaube, das sollten wir tun. denn hier geht es um ein mitgliedsland, und wenn wir diese probleme lösen, werden sich möglicherweise auch andere probleme lösen. wir haben ja schon damals (bei den zypern-verhandlungen im jahr 2004) mit eu-kommissar günter verheugen den fehler gemacht, dass die volksabstimmung in zypern leider gottes erst nach dem beitritt erfolgte. man hätte es vorher machen sollen. ich glaube, dass aus diesem grunde juncker richtigerweise jetzt die verantwortung der eu mit übernimmt. das heißt, sie erwarten, dass juncker bei den volksabstimmungen, die eventuell auf beiden teilen der insel noch anstehen, auch eine eigene rolle übernimmt? das kann ich mir gut vorstellen. die eu wird ja gefordert sein, durch die strukturfonds und manches andere mehr, das zusammenwachsen zu ermöglichen, einen friedlichen rahmen und eine bestimmte absicherung zu geben. deswegen finde ich schon gut, dass man im rahmen der eu eine lösung findet. wenn man solche konflikte überwinden will, ist es immer gut, wenn man in einem größeren hafen aufgehoben ist. elmar brok (cdu) ist vorsitzender des ausschusses für auswärtige angelegenheiten im europäischen parlament. & 409 & very low & Low & Socio-Economic & NA & NA & 2017-01-18 & 2017 & 2 & ECO
Frame & v.low & National & <500 & -1.0405052 & -1.0641830 & 1.2185583 & -1.5781059 & 1.4491718 & 12.0 & 1.6295860 & 1.9019065 & Payer & Domestic & Domestic & Domestic & Domestic|ECO & Positive\\
Germany & http://www.merkur.de/lokales/garmisch-partenkirchen/garmisch-partenkirchen/landkreis-garmisch-partenkirchen-drueckt-co2-bremse-6079580.html?cmp=defrss & 223 & Merkur.de & Private/Non-Public & Online and Offline & Regional/Local & very low = CP mentioned once & Environment/green/low-carbon & Positive & Subnational & No myth & NA & NA & NA & NA & NA & NA & NA & NA & Germany & landkreis kämpft gegen die klimabomben & 2016-01-29 & europäischer fonds für regionale entwicklung & landkreis - treibhausgase steigen überall in die atmosphäre - auch mitten in der natur. kohlenstoffdioxid entweicht etwa in großen mengen aus heimischen moorböden, beispielsweise in eschenlohe-weghaus. dort soll die co2-freisetzung spürbar reduziert werden. die menge an klima-gefährdendem kohlenstoffdioxid, die täglich von bayerischen mooren in die luft entweicht, ist beträchtlich. laut peter strohwasser von der unteren naturschutzbehörde ist diese mit dem co2-ausstoß im straßenverkehr durchaus vergleichbar. "das sind keine kinkerlitzchen, das ist eine kapitale größenordnung", teilte der experte vom landratsamt jüngst den erstaunten kreistags-mitgliedern im umwelt- und landwirtschafts-ausschuss mit. diesen treibhausgasen möchte der landkreis nun den kampf ansagen. nicht zuletzt deshalb, da man als eine von zwei schwerpunkt-regionen für das eu-programm efre (europäischer fonds für regionale entwicklung) zur förderung von moorprojekten ausgewählt worden ist. und hierbei fließen laut tessy lödermann (bündnis 90/die grünen) reichlich gelder aus brüssel. die ehemalige landtagsabgeordnete (1990 bis 1998) spricht in diesem zusammenhang von fünf millionen euro. nun wollte die grünen-fraktion wissen, in welch konkrete maßnahmen teile dieser beachtlichen summe investiert werden. klaus streicher, ansprechpartner im landratsamt, hatte für lödermann eine gute nachricht parat. auf der staatseigenen fläche im eschenloher weiler weghaus könnte in kooperation mit der landesanstalt für landwirtschaft und dem gestüt schwaiganger - es bewirtschaftet diesen grund - ein antrag für knapp 100 hektar sogenanntes niedermoor-grünland gestellt werden. die kosten für diese bislang erstmalige renaturierung beziffert streicher auf "mehrere hunderttausend euro". auf diesem weitläufigen gelände wurde der einst morastige boden zur bäuerlichen nutzung trockengelegt. mittels künstlicher gräben entzog man dem moor feuchtigkeit. dadurch sank der grundwasserspiegel. der nebeneffekt: an die fauligen pflanzen (torf) gelangte nun sauerstoff. dadurch wurde bislang gebundenes kohlenstoffdioxid freigesetzt und strömt seitdem in die atmosphäre, was in puncto klimaschutz suboptimal ist. auf vielen flächen deutschlandweit ist dieses durch menschenhand herbeigeführte phänomen zu beobachten. in weghaus will man jetzt aber die co2-bremse drücken - und zwar massiv. das problem bei dieser initiative: das areal soll derart, wie es im fachjargon heißt, "vernässt" werden, dass es auch künftig landwirtschaftlich genutzt werden kann. gedacht ist dabei an ein stauwehr an einem der hauptkanäle. mit dessen hilfe kann der grundwasserstand nach bedarf reguliert werden. daneben werden punktuell vorhandene gräben zugeschüttet. die operation weghaus soll in den kommenden fünf jahr in drei schritten ablaufen: voruntersuchung, konkrete umsetzung und beobachtung. zur genauen daten-erhebung werden laut streicher messpunkte eingerichtet. christof schnürer & 390 & very low & Low & Socio-Economic & NA & NA & 2016-01-29 & 2016 & 2 & ECO
Frame & v.low & Regional & <500 & -1.0405052 & -1.0641830 & 1.2185583 & -1.5781059 & 1.4491718 & 12.0 & 1.6295860 & 1.9019065 & Payer & Domestic & Domestic & Domestic & Domestic|ECO & Positive\\
\addlinespace
Germany & http://www.rp-online.de/kultur/fuer-ein-realistisches-europa-aid-1.5451778 & 199 & RP Online & Private/Non-Public & Online and Offline & Regional/Local & medium = CP is important part of story & Ineffective goal achievement & Negative & National & 4.No added value & NA & NA & NA & NA & NA & NA & NA & NA & Germany & für ein realistisches europa & 2015-10-07 & kohäsionsfonds & der publizist martin winter hält nichts von einem idealisierten europa-bild. von peter seidel am geld hat es nicht gefehlt. "nach den sozialfonds wurden der europäische fonds für regionale entwicklung und der kohäsionsfonds geschaffen. über diese fonds flossen allein zwischen 2000 und 2013 rund 700 milliarden euro in jene gebiete und länder, die wirtschaftlich hinterherhinken. (...) das system der fonds (...) hat nichts bewirkt, (...) sondern politische und wirtschaftliche lethargie verstärkt." was sich hier wie ein aktueller kommentar zu den jüngsten französischen forderungen zur "vertiefung" der eu-währungsunion anhört, stammt aus dem neuen buch des ehemaligen brüsseler korrespondenten martin winter, das gerade erschienen ist. titel: "das ende einer illusion. europa zwischen anspruch, wunsch und wirklichkeit". winter, der zwei jahrzehnte für "frankfurter rundschau" und "süddeutsche zeitung" gearbeitet hat, geht breit auf das thema eu-wirtschaftsregierung ein, die er unter verweis auf helmut schmidt für unrealistisch hält. was winter 2014 aber noch nicht so absehen konnte, ist, dass durch die neuen französischen vorschläge zu einer wirtschaftsregierung ein umschlag von der quantität in die qualität droht und die euro-krise zu einer beträchtlich ausgeweiteten dauer-alimentierung des südens durch zentralisierung, institutionalisierung der transferunion und (beispielsweise) vergemeinschaftung der deutschen einlagensicherungsfonds genutzt werden soll. die eu würde damit weiter den von winter ausführlich beschriebenen weg gehen, fundamentale probleme lediglich mit noch mehr geld zu "bekämpfen". winter ist ausgerechnet von wissenschaftlicher seite vorgeworfen worden, den "ideologischen faktor" der deutschen europapolitik mit seinem realistischen ansatz geringzuschätzen. winter, ein überzeugter, aber eben kein hurra-europäer, teilt zwar die meisten europäischen ziele, hält sie aber nicht (mehr) für realistisch. dies zieht sich durch das ganze buch, ebenso wie seine sorge, "einen disparaten haufen zusammenzuhalten". schwerpunkte seines buches sind die bisherigen vorstöße zu einer europäischen wirtschaftsregierung und einer europäischen sicherheits- und verteidigungspolitik, während die asylpolitik auf nur drei seiten abgehandelt wird. leider fehlt dem buch ein anmerkungsapparat, und die literaturauswahl ist leider dürftig und veraltet. winter hat ein mutiges, weil realistisches buch geschrieben. welcher schaden für europa entsteht, sollte sich brüssel zum verlängerten arm der defizitländer machen, konnte winter 2014 nur erahnen. doch wegen seiner präzisen schilderung ist dem buch eine weite verbreitung zu wünschen. denn, so der göttinger rechtsprofessor frank schorkopf, "es ist zeit, die integration nicht idealisiert, sondern pragmatisch-realistisch zu denken". & 371 & medium & Medium & Socio-Economic & NA & NA & 2015-10-07 & 2015 & 1 & ECO
Frame & low-medium & Regional & <500 & -1.0405052 & -1.0641830 & 1.2185583 & -1.5781059 & 1.4491718 & 12.0 & 1.6295860 & 1.9019065 & Payer & Domestic & Domestic & Domestic & Domestic|ECO & Negative\\
Germany & http://www.welt.de/debatte/kommentare/article143702869/EU-darf-sich-Athens-Gebaren-nicht-gefallen-lassen.html & 230 & DIE WELT & Private/Non-Public & Online and Offline & National & very low = CP mentioned once & Mismanagement & Negative & National & 8.Mismanaged & Political leverage & Negative & National & 6.Does not defend EU values (eg.gender/law/democracy) & NA & NA & NA & NA & Germany & eu darf sich athens gebaren nicht gefallen lassen & 2015-07-11 & strukturfonds & dass man die gemeinsame währung vor den griechen retten muss, ist konsens im volk, bei vielen abgeordneten und weitgehend sogar in der deutschen publizistik. die taktischen fähigkeiten der bundeskanzlerin werden daran gemessen, noch mehr aber ihre führungskraft in hochkomplexen europäischen regeln und machtstrukturen. dass man umgekehrt die griechen vor europa retten muss, haben tsipras, varoufakis \& co. zur entscheidungsfrage gemacht - und gewonnen. ihr va-banque-spiel hat sich ausgezahlt. es droht schule zu machen. denn das geschäftsmodell ist so einfach wie korrumpierend für wähler und gewählte in europa: schulden machen bis zum anschlag und dann das volk entscheiden lassen, ob es dafür zahlen will - oder doch eher nicht. sie wollen, wie die briten sagen, den kuchen essen und ihn behalten, also die transfers der eu einschließlich regionalfonds, strukturfonds und jeder menge kredite konsumieren, aber das kleingedruckte, das die geldgeber in brüssel, berlin und washington geduldig anmahnen, nicht befolgen - wenn sie es denn, wie der griechische steuer-, beamten- und rentnerstaat nun einmal beschaffen ist, überhaupt könnten. damit darf griechenland nicht durchkommen. es muss mit allen mitteln verhindert werden, dass die anbeter der griechischen regierung zuerst griechenland und dann europa in richtung lateinamerikanischer revolutionen steuern. eine generationenaufgabe scheint indes, griechenland aus der vormoderne richtung europa zu führen. die währungsunion ist unvermindert ein experiment in ungleichzeitigkeit mit lebenden figuren. die maastricht-kriterien sollten einheit herbeizwingen. am wenigsten glückte das in griechenland. eine währung aber, die nicht auf gold gegründet ist, beruht auf verträgen und vertrauen - und sonst nichts. wenn beides in den wind geschlagen wird, geht zuerst die währung verloren, danach das vertrauen in das europäische projekt, die notenbank, die regierung und die demokratie. es sind diese fragen, die hinter den palliativen in milliardenhöhe erkennbar werden. vielen europäern, am meisten den in gelddingen empfindlichen deutschen, ist die anhaltende rettungspolitik der "institutionen" (ezb, iwf, kommission) die in athen nicht einmal ihren namen nennen dürfen, längst zuwider. die eu kann nicht so weitermachen. natürlich kann europa an der währungsfrage zerbrechen. doch darf das nicht heißen, fortgesetzt schlechtem geld gutes nachzuwerfen. führung heißt europa gegen weichmacher und währungsverderber retten. & 342 & very low & Low & Governance & Power & NA & 2015-07-11 & 2015 & 1 & POL
Frame & v.low & National & <500 & -1.0405052 & -1.0641830 & 1.2185583 & -1.5781059 & 1.4491718 & 12.0 & 1.6295860 & 1.9019065 & Payer & Domestic & Domestic & Domestic & Domestic|POL & Negative\\
Germany & http://www.heise.de/tp/news/Muss-die-EU-Kommission-Portugal-und-Spanien-Gelder-streichen-3341551.html & 273 & heise online & Private/Non-Public & Online only & National & low = CP mentioned more times but NOT important part of story (mainly about others issues) & Political leverage & Negative & EU & 6.Does not defend EU values (eg.gender/law/democracy) & NA & NA & NA & NA & NA & NA & NA & NA & Germany & muss die eu-kommission portugal und spanien gelder streichen? & 2016-10-05 & kohäsionspolitik & im europaparlament wurde die streichung von fondsgeldern auf breiter front abgelehnt nachdem die eu-kommission wegen den defizitverstößen gegen spanien und portugal im juli auf eine strafzahlung verzichtet hatte, steht nun die frage der aussetzung von fondsgeldern auf der tagesordnung. wirtschaftskommissar jyrki katainen hatte schon darauf hingewiesen, dass "ein teil" der gelder ausgesetzt werden müsse, wenn festgestellt werde, dass "ein mitgliedsland keine effektiven maßnahmen" zur einhaltung der defizitziele ergriffen habe. als nun am montagabend mit europaparlamentariern über die aussetzung von fondsgeldern debattiert wurde, wiederholte katainen, es handele sich dabei um eine "pflicht". die für regionalpolitik zuständige kommissarin corina cretu erklärte in der ausschusssitzung in straßburg, dass man es gar nicht mit einer strafe zu tun habe und die auswirkungen ohnehin nur minimal sein würden. und "kurzfristig" gäbe es keine, bestenfalls würde es einige investitionen verzögern, argumentierten beide, ohne details zu nennen. ohnehin würden die strafen wieder aufgehoben, wenn beide länder die versprochenen maßnahmen zur defizitreduzierung umsetzen, sagten sie mit blick auf die haushalte für 2017, die bis zum 15. oktober vorgelegt werden müssen. sogar bis in die fraktion der konservativen gab es kritik an dem vorhaben. "kafkaesk", "absurd", "ungerecht", "kontraproduktiv" waren nur einige der bezeichnungen für die geplante aussetzung von geldern, die vertreter im wirtschafts- und regionalausschuss dafür fanden. gefragt wurden die eu-kommissare von diversen abgeordneten auch, ob das die antwort brüssels auf den brexit sein solle und so das gemeinschaftsgefühl gestärkt werden solle. das war auch die meinung der regionalpolitischen sprecherin der linken im europaparlament. martina michels sieht den zusammenhalt in der eu weiter gefährdet, denn eine "spirale aus sparauflagen", die von der kommission gefordert würden, führe zum "rückgang öffentlicher investitionen auf allen ebenen und damit verschlechterung der wirtschaftlichen und sozialen lage". das werde durch die aussetzung von fondsgeldern noch verstärkt. "das ist das gegenteil von förderung gleichwertiger lebensverhältnisse, dem kernziel der eu-kohäsionspolitik". möglich wäre es, bis zu 50\% der zahlungen für 2017 zu streichen. und wie telepolis schon aufgezeigt hatte, könnten auch projekte zu fall gebracht werden, die im zusammenhang des jugendgarantie-versprechens stehen. dazu zog die eu gerade sogar eine vorsichtige positive bilanz. das ist angesichts der tatsache, dass drei jahre nach dem versprechen in spanien noch immer mehr als 42\% und in griechenland sogar fast 48\% aller jungen menschen keinen job, ausbildungs- oder praktikumsstelle haben, ein hohn. denn das sollte ihnen nach dem versprechen in nur vier monaten garantiert werden. so befürchtet auch michels, dass projekte von der strafe betroffen sein könnten, "die sich darum bemühen, menschen aus armut und sozialer ausgrenzung" herauszuhelfen. sie forderte deshalb von der kommission, wenigstens zu sichern, dass keine geplanten und benötigten projekte zunichtegemacht werden. ohnehin ist auch klar, dass es eine sonderbare auslegung der stabilitätsziele in der kommission gibt. umso größer das land, umso nachsichtiger ist man in brüssel. frankreich und deutschland verstoßen massiv gegen stabilitätskriterien. frankreichs haushaltsdefizit liegt zum beispiel seit 2006 über der marke von 3\% und das land wird das ziel auch im laufenden jahr nicht einhalten. arbeitet man in brüssel an sanktionen? nein! kommissionspräsident jean-claude juncker hatte sogar offen erklärt, er tue seit jahren nichts anderes, als der regierung in paris ausnahmen von den regeln des stabilitätspakts zu gewähren. auf die frage, warum er das mache, antwortete er nur: "weil es frankreich ist." man müsse rücksicht auf seine spezielle mentalität und politische reflexe nehmen. hat man bisher schon etwas von einer sanktionsforderung gegen deutschland vernommen? dabei werden notorisch auch die dauernden deutschen verstöße gegen die leistungsbilanzüberschüsse ignoriert. seit 2012 liegt der überschuss über der höchstmarke von 6\%, ein ohnehin sehr hoher schwellenwert im stabilitätspakt, der für ungleichgewichte in der eu sorgt. 2016 dürfte mit 8,5\% sogar eine neue rekordmarke erklommen werden. man kann die ungleichbehandlung auch im vergleich spanien und portugal sehr deutlich sehen, wozu auch politische faktoren beitragen. hatten die konservativen in spanien ohnehin schon ein jahr mehr zeit bis 2016 erhalten, um das defizit wieder auf 3\% zu senken, wird portugal besonders hart angegangen, weil das land das nicht schon 2015 geschafft hat. und spanien gewährt die kommission nun sogar weitere zwei jahre, um auf 3\% zu kommen, während von portugal gefordert wird, es nun auf 2,5\% zu senken. die spanischen konservativen werden also erneut dafür belohnt, dass ihr defizit mit 5,2\% deutlich höher lag als das in portugal. der nachbar hätte sogar das 3\%-ziel sogar praktisch wieder eingehalten, wäre nicht eine von brüssel genehmigte bankenhilfe dazwischengekommen. anders als spanien dürfte es portugal 2016 aber schaffen, das defizitziel einzuhalten. spanien wird dagegen, angesichts der seit zehn monaten blockierten regierung, nicht einmal erreichen, zum geforderten termin einen haushalt in brüssel vorzulegen. klar ist, auch wegen der selbstzerfleischung bei den spanischen sozialdemokraten (), dass vor monatsende keine regierung gebildet werden kann. viel spricht derzeit sogar für neuwahlen an weihnachten, weil sich rajoy nun ausrechnet, mit den rechten "ciudadanos" (bürger) dann eine stabile mehrheit zu erhalten. deshalb muss man erneut damit rechnen, dass brüssel den konservativen freunden nicht in die parade fahren wird. steht keine regierung, wird die aussetzung von eu-geldern bestenfalls nur sehr sanft und symbolisch ausfallen, um den konservativen nicht vor den wahlen zu schaden. deshalb hatte man die entscheidung über sanktionen bisher immer wieder vertagt. eigentlich sollte die entscheidung über eine defizitstrafe und die aussetzung von fördergeldern schon im frühjahr fallen. letztlich wurde keine strafzahlung verhängt, um dem konservativen mariano rajoy nicht die rechnung für seine verfehlte politik auszustellen und damit die bildung einer möglichen rechtsregierung zu verhindern. dafür machte sich vor allem schäuble stark, der zuvor am lautesten nach strafen gerufen hatte. & 916 & low & Low & Power & NA & NA & 2016-10-05 & 2016 & 2 & POL
Frame & low-medium & National & 500-1000 & -1.0405052 & -1.0641830 & 1.2185583 & -1.5781059 & 1.4491718 & 12.0 & 1.6295860 & 1.9019065 & Payer & European & European & European & European|POL & Negative\\
Germany & https://www.waz.de/staedte/siegerland/cyberbrille-koennte-sogar-beim-schrankaufbauen-helfen-id215406205.html & 252 & WAZ & Private/Non-Public & Online and Offline & Regional/Local & very low = CP mentioned once & Research \& innovation & Positive & EU & No myth & NA & NA & NA & NA & NA & NA & NA & NA & Germany & cyberbrille könnte sogar beim schrankaufbauen helfen & 2018-09-24 & europäischer fonds für regionale entwicklung & siegen. die uni siegen tüftelt an einer neuartigen cyberbrille, die bei der arbeit an maschinen helfen soll. hologramm soll zeigen, was zu tun ist. inhalt artikel auf einer seite lesen > ... ... vorherige seite nächste seite cyberbrille könnte sogar beim schrankaufbauen helfen man stelle sich vor, beim aufbau eines möbels würde eine intelligente brille mit einer art hologramm schritt für schritt zeigen, was zu tun ist. unverständliche bedienungsanleitungen gehörten der vergangenheit an. die ergebnisse eines forschungsprojekts der universität siegen könnten so etwas ermöglichen: zwar nicht hobby-heimwerkern, aber maschinenbedienern in unternehmen. cyberbrille soll bei der arbeit an der maschine helfen was passiert mit dem wissen erfahrener fachkräfte, wenn sie aus dem beruf ausscheiden? diese frage war ausgangspunkt für projektleiter dr. christopher kuhnhen und sein team. die wissenschaftler wollten diese expertise mit methoden der industrie 4.0 nahtlos an jüngere kollegen weitergeben. das system sei für unternehmen konkret nutzbar und erleichtere die arbeit, so prof. bernd engel, lehrstuhl für umformtechnik. wissenschaft mehr eu-geld für forschung in südwestfalen in kooperation mit dem lehrstuhl für wirtschaftsinformatik und neue medien (prof. volker wulf), dem siegener mittelstandsinstitut (smi) und betrieben aus der region ist ein system entstanden, das maschinenbediener mit hilfe einer cyberbrille bei ihrer arbeit unterstützen und arbeitsabläufe verkürzen soll. im mittelpunkt stand ein biegeprozess, das system lässt sich aber auch auf kunststoff oder zerspanung übertragen. es wurde bereits diversen unternehmen aus der region vorgestellt. die reaktionen seien überragend gewesen, berichtet kuhnhen. vorteil für neue mitarbeiter 1. bevor die maschine das rohr biegen kann, muss sie gerüstet, also mit werkzeugen bestückt werden. der bediener setzt die "hololens", eine sogenannte mixed-reality-brille auf. blickt er auf die maschine, erscheint dort ein virtuelles modell. über die brille wird nun jeder arbeitsschritt eingeblendet. zudem können erläuternde videos abgespielt werden. so könnten auch unerfahrenere mitarbeiter maschinen rüsten. "eine art navigationsgerät für maschinenbediener", so kuhnhen. die dauer des prozesses werde so um bis zu 50 prozent verkürzt. lokales 2,6 millionen euro das projekt, titel "cyberrüsten 4.0", hat einen gesamtumfang von 2,6 millionen euro. auf die uni siegen entfallen fast 1,5 millionen euro, davon kommen 90 prozent vom fördermittelgeber efre (europäischer fonds für regionale entwicklung). 2. beim rotationszugbiegen müssen sieben verschiedene achsen eingestellt werden. für ein perfektes ergebnis ist viel erfahrung nötig - oder langes tüfteln. das vom hard- und softwarespezialisten lachmann \& rink (freudenberg) umgesetzte programm schlägt dem bediener die bestmöglichen parameter vor. damit das funktioniert, haben die wissenschaftler erfahrenen mitarbeitern des hilchenbacher unternehmens westfalia metallschlauchtechnik auf die finger geschaut und interviews geführt. ihr wissen fließt direkt in das programm ein. "prozesse mit den menschen weiterentwickeln" 3. das programm ist lernfähig: wird ein rohr mit fehlern produziert, werden bei ähnlichen problemen vorschläge zur fehlerbehebung geliefert. und es überprüft, welche einstellungen bei welchen problemen zum erfolg geführt haben. prof. engel hofft, dass unternehmen so die angst vor dem begriff industrie 4.0 genommen wird: "wir wollen nicht den menschen durch maschinen ersetzen, sondern prozesse mit den menschen weiterentwickeln." mehr nachrichten, fotos und videos aus dem siegerland gibt es hier.die lokalredaktion siegen ist auch auf facebook. inhalt artikel auf einer seite lesen > ... ... vorherige seite nächste seite kommentare & 519 & very low & Low & Socio-Economic & NA & NA & 2018-09-24 & 2018 & 3 & ECO
Frame & v.low & Regional & 500-1000 & -1.0405052 & -1.0641830 & 1.2185583 & -1.5781059 & 1.4491718 & 12.0 & 1.6295860 & 1.9019065 & Payer & European & European & European & European|ECO & Positive\\
Germany & https://www.merkur.de/politik/ost-laenderchefs-bei-regierungsbildung-an-osten-denken-zr-8941378.html & 213 & Merkur.de & Private/Non-Public & Online and Offline & Regional/Local & very low = CP mentioned once & Economic development & Positive & National + Subnational & No myth & NA & NA & NA & NA & NA & NA & NA & NA & Germany & ost-landeschefs schicken forderungskatalog an merkel & 2017-10-31 & kohäsionspolitik & auch viele jahre nach der deutschen wiedervereinigung stehen die bundesländer im osten in vielen punkten schlechter da als die im westen. die ost-landeschefs wollen endlich aufholen - und schicken dazu einen ganzen forderungskatalog nach berlin. dresden - die ministerpräsidenten der ostdeutschen länder haben bundeskanzlerin angela merkel (cdu) aufgefordert, bei der regierungsbildung ost-interessen im blick zu behalten. sachsens scheidender ministerpräsident stanislaw tillich (cdu) schickte dazu im namen seiner amtskollegen einen brief an die kanzlerin, wie die sächsische staatskanzlei am dienstag mitteilte. die ostdeutschen länder wiesen weiterhin eine "nahezu flächendeckende strukturschwäche" auf, schrieb tillich, der den vorsitz der ministerpräsidentenkonferenz ost innehat. diese schwäche müsse überwunden werden. dafür sei ostdeutschland weiter auf finanzielle förderung angewiesen - sowohl aus deutschen töpfen als auch im rahmen der eu-kohäsionspolitik. "ein abruptes ende der strukturförderung in ostdeutschland würde die erfolge der vergangenheit gefährden", hieß es in dem schreiben. ungünstige "sandwichposition" verhindern auch müsse verhindert werden, dass ostdeutschland in eine ungünstige "sandwichposition" gerate - zwischen den hoch entwickelten regionen in westdeutschland und den sehr stark von der eu geförderten gebieten in osteuropa. daneben sprachen sich die landeschefs in dem schreiben gegen einen schnellen braunkohle-ausstieg aus. an der kohleverstromung hingen in ostdeutschland zehntausende arbeitsplätze. ein abruptes ende verbiete sich "schon aus respekt vor der lebensleistung der beschäftigten", hieß es. erst wenn es für sie nachhaltige zukunftsperspektiven gebe, dürfe das ende der kohlenutzung beschlossen werden. damit der anschluss an den wirtschaftsstarken westen gelinge, müsse der osten zudem besser an das bahnnetz und den luftverkehr angeschlossen werden, forderten tillich und seine kollegen weiter. außerdem müsse die künftige regierung eine flächendeckende versorgung mit schnellem internet und mobilfunk sicherstellen. in berlin laufen derzeit sondierungsgespräche zwischen vertretern von cdu, csu, fdp und grünen über eine mögliche künftige jamaika-koalition. bei den reizthemen klima und flüchtlinge hatte es zuletzt streit gegeben. nun gelangen bei den themen arbeit, rente, pflege, sicherheit und bildung und digitales deutliche fortschritte. dpa & 312 & very low & Low & Socio-Economic & NA & NA & 2017-10-31 & 2017 & 2 & ECO
Frame & v.low & Regional & <500 & -1.0405052 & -1.0641830 & 1.2185583 & -1.5781059 & 1.4491718 & 12.0 & 1.6295860 & 1.9019065 & Payer & Domestic & Domestic & Domestic & Domestic|ECO & Positive\\
\addlinespace
Germany & https://www.n-tv.de/wirtschaft/EU-kuerzt-Unterstuetzung-deutscher-Bauern-article20267503.html & 275 & n-tv.de & Private/Non-Public & Online only & National & very low = CP mentioned once & Institutional bargaining over funding & Negative & EU + National & No myth & NA & NA & NA & NA & NA & NA & NA & NA & Germany & zölle für us-exporteure: eu kürzt unterstützung deutscher bauern & 2018-02-04 & kohäsionsfonds & der befürchtete kahlschlag soll ausbleiben, doch durch kürzungen des eu-haushalts im agrarsektor müssen deutsche landwirte mit einschnitten rechnen. auf zolldrohungen aus den usa hat haushaltskommissar günther oettinger eine klare antwort. nach worten des eu-haushaltskommissars günther oettinger müssen sich deutsche bauern und die bundesländer auf weniger geld aus brüssel einstellen. beim neuen mehrjährigen eu-haushalt werde es zwar "keinen kahlschlag geben, wie einige befürchten", sagte oettinger der "welt am sonntag". "aber auch in deutschland werden sich landwirte und regionen auf finanzielle kürzungen einstellen müssen." die eu-kommission plane, "die agrar- und kohäsionsfonds im neuen mehrjährigen haushalt jeweils um fünf bis zehn prozent zu verkleinern", sagte oettinger. demnach gibt es bereits überlegungen, wie die kürzungen im landwirtschaftssektor aussehen könnten: "in der agrarpolitik erwägen wir, die direktzahlungen pro hektar fläche künftig degressiv zu gestalten. das bedeutet: ab einer gewissen fläche gibt es dann pro hektar weniger finanzielle unterstützung als für den ersten hektar." die eu-kommission will im mai einen konkreten vorschlag zum nächsten finanzrahmen vorlegen, der ab 2021 gilt. die landwirtschaft und die strukturförderung machen zusammen bisher fast drei viertel aller eu-ausgaben aus. in der "welt am sonntag" reagierte oettinger auch auf die drohung von us-präsident donald trump, möglicherweise strafzölle auf produkte aus europa zu erheben: "wenn europäische exporteure zölle zahlen müssen, wird eine zweibahnstrasse daraus. dann werden auch us-exporteure bei uns zölle zahlen müssen", sagte der eu-kommissar. "wer das instrument zückt, muss wissen, dass wir es auch haben. und der europäische markt ist mindestens so groß wie der amerikanische." & 254 & very low & Low & Power & NA & NA & 2018-02-04 & 2018 & 3 & POL
Frame & v.low & National & <500 & -1.0405052 & -1.0641830 & 1.2185583 & -1.5781059 & 1.4491718 & 12.0 & 1.6295860 & 1.9019065 & Payer & Domestic & European & Mixed & Domestic|POL & Negative\\
Germany & https://www.tagesschau.de/inland/bauern-subventionen-101.html & 228 & tagesschau.de & Public & Online and Offline & National & low = CP mentioned more times but NOT important part of story (mainly about others issues) & Institutional bargaining over funding & Balanced & EU + National & No myth & NA & NA & NA & NA & NA & NA & NA & NA & Germany & weniger geld aus brüssel für deutsche bauern & 2018-02-04 & kohäsionsfonds & durch den brexit fehlen im eu-haushalt bald milliarden. eu-haushaltskommissar oettinger will umschichten - zuungunsten deutscher bauern. mehr geld soll in verteidigung und migrationspolitik fließen. deutsche bauern und regionen werden ab 2021 wohl weniger geld aus brüssel bekommen. eu-haushaltskommissar günther oettinger sagte der "welt am sonntag": "es wird keinen kahlschlag geben, wie einige befürchten. aber auch in deutschland werden sich landwirte und regionen auf finanzielle kürzungen einstellen müssen." die eu-kommission plane, die agrar- und kohäsionsfonds im neuen mehrjährigen haushalt jeweils um fünf bis zehn prozent zu verkleinern. "in der agrarpolitik erwägen wir, die direktzahlungen pro hektar fläche künftig degressiv zu gestalten. das bedeutet: ab einer gewissen fläche gibt es dann pro hektar weniger finanzielle unterstützung als für den ersten hektar", sagte oettinger. oettinger bereitet derzeit seinen entwurf für den nächsten mehrjährigen haushaltsrahmen für die jahre nach 2020 vor. im budget fehlen dann wegen des brexits voraussichtlich bis zu 14 milliarden euro an britischen beiträgen. gleichzeitig will oettinger für einige aufgaben wie verteidigung oder migrationspolitik mehr geld einplanen. im haushalt soll deshalb umgeschichtet werden. die landwirtschaft und die strukturförderung machen zusammen bisher fast drei viertel aller eu-ausgaben aus. geplant ist auch, dass die eu-länder zehn bis 20 prozent mehr einzahlen. deutschland werde laut oettinger eine mehrbelastung im einstelligen milliardenbereich zu schultern haben. oettinger hofft auch auf neue eigenmittel für den eu-haushalt. "wir erwägen auch, dass künftig ein kleiner teil der gewinne, die die europäische zentralbank mit der ausgabe von banknoten macht, als eigenmittel in den eu-haushalt fließt." der kommissar setzt zudem weiter auf seinen vorschlag einer "plastiksteuer" auf verpackungen. in der "welt am sonntag" reagierte oettinger auch auf die drohung von us-präsident donald trump, möglicherweise strafzölle auf produkte aus europa zu erheben: "wenn europäische exporteure zölle zahlen müssen, wird eine zweibahnstraße daraus. dann werden auch us-exporteure bei uns zölle zahlen müssen", sagte der eu-kommissar. "wer das instrument zückt, muss wissen, dass wir es auch haben. und der europäische markt ist mindestens so groß wie der amerikanische." mit dem künftigen haushaltsrahmen befassen sich am 23. februar erstmals die eu-staats- und regierungschefs. oettinger will seinen entwurf im mai vorlegen. über einzelheiten dürfte danach monatelang mit den mitgliedsländern und dem europaparlament gestritten werden. & 369 & low & Low & Power & NA & NA & 2018-02-04 & 2018 & 3 & POL
Frame & low-medium & National & <500 & -1.0405052 & -1.0641830 & 1.2185583 & -1.5781059 & 1.4491718 & 12.0 & 1.6295860 & 1.9019065 & Payer & Domestic & European & Mixed & Domestic|POL & Neutral\\
Germany & http://www.dw.com/de/fl\%C3\%BCchtlingsurteil-orban-bleibt-stur/a-40412763 & 262 & Deutsche Welle (English) & Public & Online and Offline & National & very low = CP mentioned once & Political leverage & Negative & Other country & No myth & NA & NA & NA & NA & NA & NA & NA & NA & Germany & flüchtlingsurteil: orban bleibt stur | aktuell europa | dw | 08.09.2017 & 2017-09-08 & kohäsionsfonds & ungarn wird trotz des flüchtlings-urteils des obersten eu-gerichts weiter keine migranten aufnehmen. ungarn sei kein einwanderungsland und wolle nicht von der eu dazu gezwungen werden, so regierungschef viktor orban. zwar müsse sein land das eugh-urteil zur kenntnis nehmen, "denn wir können nicht das fundament der eu untergraben - und die anerkennung von recht und gesetz ist das fundament der eu", sagte orban (artikelbild) im staatsrundfunk. "gleichzeitig ist dieser richterspruch für uns aber kein grund, unsere politik zu ändern, die flüchtlinge ablehnt." orban wies forderungen mehrerer mitgliedstaaten und aus der eu-kommission zurück, die zahlungen aus dem kohäsionsfonds zur förderung der finanzschwächeren eu-staaten an die bereitschaft zur aufnahme von flüchtlingen entsprechend den eu-beschlüssen zu koppeln. dies verstoße gegen die regeln der eu und sei unmoralisch, sagte orban. "bisher haben wir einen juristischen kampf geführt, jetzt müssen wir einen politischen kampf führen", so orban. budapest müsse erreichen, dass der quotenbeschluss von 2015 revidiert werde und kein anderer verteilungsmechanismus für asylbewerber an seine stelle trete. eu-quotenregel gilt für ungarn und die slowakei der europäische gerichtshof (eugh) hatte zuvor klagen von ungarn und der slowakei gegen die eu-quotenregel für die aufnahme von flüchtlingen abgewiesen. da keine berufung gegen das urteil möglich ist, müssten beide länder nach geltender rechtslage gegen ihren willen migranten entsprechend den im ministerrat beschlossenen verteilungsschlüssel aufnehmen. der beschluss sieht für jedes land der eu die aufnahme einer festgelegten anzahl an geflüchteten vor. bereits in einer ersten reaktion hatte der ungarische außenminister peter szijjarto das urteil als "empörend" zurückgewiesen. es sei ausfluss einer politik, die "das europäische recht vergewaltigt". in seiner argumentation bemüht orban auch die historie. im gegensatz zu anderen ländern sei ungarn in der vergangenheit keine kolonialmacht gewesen und habe anders als diese länder deshalb auch keine verpflichtung, fremde aufzunehmen. "keine solidarität à la carte" eu-kommissionspräsident jean-claude juncker hatte nach dem eugh-urteil gemahnt, solidarität sei nicht à la carte zu haben. dies stößt auch in deutschland auf große zustimmung. im zdf-politbarometer befürworteten 82 prozent der befragten, dass die länder, die sich weigern flüchtlinge aufzunehmen, ausgleichszahlungen leisten sollten. forderung nach eu-geld für den grenzzaun orban wiederholte auch seine forderung nach einer eu-zahlung von 440 millionen euro für den zaun an der ungarischen grenze. dieser schütze nicht nur die ungarischen eu-bürger, sondern auch österreicher, deutsche und andere länder vor unkontrollierter einwanderung. "wenn die europäische kommission statt der verteidigung der grenzen ausschließlich dazu bereit ist, maßnahmen und institutionen zu finanzieren, die die aufnahme von migranten anstreben, werden wir hunderttausenden von migranten bloß einen erneuten anreiz bieten, die sich in richtung europa auf den weg machen - anstatt die migration aufzuhalten." die eu-kommission hatte orbans forderung nach den 440 millionen euro bereits ende august abgelehnt und das land bei der flüchtlingsverteilung zur solidarität aufgerufen. & 463 & very low & Low & Power & NA & NA & 2017-09-08 & 2017 & 2 & POL
Frame & v.low & National & <500 & -1.0405052 & -1.0641830 & 1.2185583 & -1.5781059 & 1.4491718 & 12.0 & 1.6295860 & 1.9019065 & Payer & European & European & European & European|POL & Negative\\
Germany & https://www.tagesspiegel.de/wirtschaft/zuwendungen-von-der-eu-milliarden-fuer-bauern-forscher-und-arbeitslose/21011106.html & 242 & Der Tagesspiegel & Private/Non-Public & Online and Offline & Regional/Local & low = CP mentioned more times but NOT important part of story (mainly about others issues) & Solidarity to poor countries/regions & Balanced & National + Subnational & 2.Rich countries pay & NA & NA & NA & NA & NA & NA & NA & NA & Germany & milliarden für bauern, forscher und arbeitslose & 2018-02-27 & kohäsionsfonds & berlin und brandenburg können im laufenden eu-finanzrahmen, der sich von 2014 bis 2020 erstreckt, mit rückflüssen von mehr als drei milliarden euro rechnen. deutschland ist der größte nettozahler in der eu. 2016 überwies die bundesrepublik, die mit abstand das land mit der größten wirtschaftsleistung in der eu von 28 mitgliedstaaten ist, 10,08 milliarden euro mehr nach brüssel, als an zuwendungen von der eu zurück flossen. pro kopf kostet die eu damit jeden bundesbürger im schnitt 176 euro im jahr. schweden (226 euro), niederländer (219) und briten (178) zahlen jedoch pro kopf noch mehr ein. in brüssel laufen gerade die verhandlungen darüber an, wie viel geld die 27 mitgliedstaaten der eu im nächsten mittelfristigen finanzrahmen, der die haushaltsjahre 2021 bis 2027 abdecken wird, jeweils zum eu-haushalt beisteuern. am verhandlungstisch im rat, dem gremium der mitgliedstaaten, sitzen die staats- und regierungschefs. nicht direkt dabei sind die ministerpräsidenten der bundesländer. dabei sind die schwerpunkte des eu-haushalts entscheidend dafür, wie viel geld am ende aus brüssel in die länder zurück fließt. so können berlin und brandenburg im laufenden eu-finanzrahmen, der sich von 2014 bis 2020 erstreckt, mit rückflüssen von mehr als drei milliarden euro rechnen. von den so genannten kohäsionsfonds, die mitgliedstaaten bei investitionen in umwelt und verkehrsinfrastruktur helfen sollen, ist deutschland allerdings ausgeschlossen. diese gelder gehen an die 15 mitgliedstaaten mit der niedrigsten wirtschaftsleistung pro kopf. gurken stehen unter dem schutz der eu der größte posten aus dem eu-haushalt für berlin/brandenburg geht in die landwirtschaft. rund 1,05 milliarden euro stehen in berlin und brandenburg allein aus dem eu-landwirtschaftsfonds für nachhaltige entwicklung bereit. ein großteil der mittel fließt nach brandenburg, da dort die überwiegende zahl der landwirtschaftlichen betriebe ist. hinzu kommen direktzahlungen an die bauern. pro hektar land bekommt der landwirt in deutschland im schnitt 281 euro pro jahr. im schnitt machen diese direktzahlungen, die nur an die bewirtschaftete fläche, aber nicht an die produzierte menge gebunden ist, 40 prozent der einkommen der bauern aus. gurken und meerrettich stehen zudem als regionale produkte unter dem schutz der eu und dürfen nicht nachgeahmt werden. insgesamt stehen deutschland in der förderperiode 6,35 milliarden euro für die landwirtschaft zur verfügung. den zweitgrößten posten macht für berlin die regionalförderung aus. aus den sogenannten brüsseler efre-töpfen kann berlin den jahren 2014 bis 2020 mit zuschüssen in höhe von 635 millionen euro rechnen. wenn die eigenmittel, die die öffentliche hand sowie privatleute zusteuern, eingerechnet werden, stehen damit investitionsmittel in höhe von 1,27 milliarden euro zur verfügung. allein 2016 hatte berlin an regionalfördermitteln in brüssel 88,9 millionen euro abgerufen. die gelder werden zum beispiel eingesetzt, um benachteiligte städtische gebiete aufzuwerten. mittel für langzeitarbeitslose in berlin der drittwichtigste posten für berlin sind die mittel aus dem eu-sozialfonds (esf). daraus hat das land in den jahren 2014 bis 2020 rund 215 millionen euro aus dem eu-haushalt zur verfügung. die projekte sind meist darauf angelegt, langzeitarbeitslose in den arbeitsmarkt zu integrieren und beschäftigungschancen von arbeitnehmern mit vermittlungshemmnissen zu verbessern. allein 2016 hat berlin aus diesem topf rund 30 millionen euro abgerufen. mehr zum thema sozialer arbeitsmarkt wie die groko langzeitarbeitslosen helfen will marie rövekamp berlin hat beachtliche erfolge, wenn es darum geht, eu-forschungsförderung einzuwerben. auf der liste der erfolgreichsten deutschen städte liegt berlin nach münchen auf dem zweiten platz, gefolgt von köln, stuttgart und heidelberg. eu-weit stehen für die forschungsförderung 80 milliarden euro im laufe des finanzrahmens zur verfügung. wie viel davon an die spree geht, ist nicht ausgewiesen. & 582 & low & Low & Values & NA & NA & 2018-02-27 & 2018 & 3 & ECO
Frame & low-medium & Regional & 500-1000 & -1.0405052 & -1.0641830 & 1.2185583 & -1.5781059 & 1.4491718 & 12.0 & 1.6295860 & 1.9019065 & Payer & Domestic & Domestic & Domestic & Domestic|ECO & Neutral\\
Germany & https://www.stern.de/panorama/weltgeschehen/cdu-generalsekretaer-peter-tauber-musste-notoperiert-werden-7850010.html & 210 & Stern magazine & Private/Non-Public & Online and Offline & National & medium = CP is important part of story & Institutional bargaining over funding & Negative & EU + National & No myth & NA & NA & NA & NA & NA & NA & NA & NA & Germany & cdu-generalsekretär peter tauber musste notoperiert werden & 2018-02-04 & kohäsionsfonds & cdu-generalsekretär peter tauber verbrachte zwölf tage auf der intensivstation die wichtigsten meldungen im überblick: a400m defekt: deutsche soldaten sitzen in mali fest (08.35 uhr) bundesregierung gab 72,2 millionen euro für g20-gipfel aus (06.45 uhr) spanische polizei zerschlägt pädophilenring - 40 festnahme (05.05 uhr) umfrage: afd und fdp steigen in der wählergunst (04.15 uhr) schulz nicht ins kabinett? spd-debatte um parteichef (02.15 uhr) die nachrichten des tages im stern-ticker: +++ 08.35 uhr: a400m defekt: deutsche soldaten sitzen in mali fest +++ insgesamt 89 in mali eingesetzte deutsche soldaten warten seit tagen auf eine rückflugmöglichkeit aus dem westafrikanischen krisenstaat in den heimaturlaub. eigentlich hätten die soldaten schon vor einigen tagen nach vier monaten einsatz nach deutschland fliegen sollen, sagte ein sprecher des verteidigungsministeriums am sonntag. aber ein für die abholung eingeplanter a400m ist defekt. nun sollten die soldaten in den kommenden tagen auf zivile linienflüge von bamako nach europa gebucht werden, sagte der sprecher. zunächst berichtete die "bild am sonntag" über die verzögerung. der a400m gilt zwar als modernstes militärisches transportflugzeug der welt. doch an den maschinen gibt es immer wieder technische probleme. die bundeswehr verfügt über 14 a400m. +++ ticker +++ news des tages rebellen in syrien schießen russischen kampfjet ab dpa +++ 06.45 uhr: bundesregierung gab 72,2 millionen euro für g20-gipfel aus +++ die bundesregierung hat für den g20-gipfel in hamburg 72,2 millionen euro ausgegeben. das geht aus einer aufstellung des finanzministeriums hervor, die der deutschen presse-agentur vorliegt. danach kostete alleine der einsatz von bundespolizei, bundeskriminalamt, technischem hilfswerk und bundesamt für sicherheit in der informationstechnik 27,7 millionen euro. das bundespresseamt gab 22,1 millionen für die betreuung der tausenden akkreditierten journalisten und für die eigene kommunikation aus. für die organisatorische und logistische vorbereitung des besuchs der staats- und regierungschefs in hamburg veranschlagte das auswärtige amt 21,7 millionen euro. das verteidigungsministerium gibt die kosten für die "technische amtshilfe" der bundeswehr für die sicherheitskräfte des bundes und der länder mit 300.000 euro an. hinzu kommen 400.000 euro für eine veranstaltung des bundesfinanzministeriums mit dem titel "g20-finance track". die aufstellung gibt nur den teil der kosten wieder, die der bund zu tragen hat. die hamburger landesregierung hat noch keine kostenrechnung veröffentlicht. +++ 05.38 uhr: vier tote nach explosion in chinesischer chemiefabrik +++ bei einer explosion in einem chemiewerk in ostchina sind vier menschen ums leben gekommen. sechs weitere wurden verletzt, wie die chinesische nachrichtenagentur xinhua am sonntag berichtete. das unglück passierte demnach am samstag bei wartungsarbeiten im werk jinshan in einer wirtschaftszone der stadt linshu (provinz shandong). die polizei nahm den besitzer der fabrik in gewahrsam. behörden leiteten eine untersuchung ein, um die ursache der explosion zu klären. +++ 05.05 uhr: spanische polizei zerschlägt pädophilenring - 40 festnahmen +++ die spanische polizei hat einen landesweit tätigen pädophilenring zerschlagen und im rahmen einer groß angelegten operation 40 verdächtige festgenommen. sie sollen über das internet entsprechendes material ausgetauscht haben und spezielle software verwendet haben, um den zugriff der sicherheitskräfte zu erschweren, hieß es am samstag in einer mitteilung. der inhalt des sichergestellten materials sei "extrem schwerwiegend". unter den festgenommenen seien lehrer, ingenieure und beamte ebenso wie arbeitslose und rentner, hieß es. die verdächtigen wurden in 17 provinzen gefasst, darunter madrid, barcelona, alicante und córdoba. bei dutzenden hausdurchsuchungen beschlagnahmten einsatzkräfte zahlreiche laptops, mehr als 100 festplatten, dvds, fotografien und videos. pädophilie missbrauchsfall von staufen: viele wussten bescheid - keiner tat etwas von frauke hunfeld +++ 04.15 uhr: union und spd im schlussspurt der koalitionsverhandlungen +++ cdu, csu und spd gehen am sonntag in den endspurt ihrer koalitionsverhandlungen über eine fortsetzung der schwarz-roten regierung. es wurde erwartet, dass beide seiten in der spd-zentrale in berlin zunächst zu getrennten vorberatungen zusammenkommen. um 11.30 uhr sollte erneut die 15er-spitzenrunde der unterhändler um kanzlerin und cdu-chefin angela merkel, den spd-vorsitzenden martin schulz sowie csu-chef horst seehofer zusammenkommen. offen war, ob die verhandlungen tatsächlich wie ursprünglich geplant abgeschlossen werden können. 03.15 uhr: brasilianische rentenkasse verlangt von staatschef temer lebensbeweis +++ eine brasilianische rentenkasse hat von staatschef michel temer einen lebensbeweis verlangt - und ihm zahlungen gestrichen, weil er nicht persönlich vorstellig wurde. der 77-jährige kommentierte den vorfall am samstag gut gelaunt: er sei glücklich, wie jeder andere brasilianer behandelt zu werden, sagte der präsident im sender rede tv. er könne aber versichern, dass es ihm "gut, wirklich gut" gehe. brasilianer müssen ein mal im jahr bei ihrer rentenkasse vorstellig werden und damit beweisen, dass sie noch am leben sind. die nationale rentenkasse verlangte das gleiche von temer, der als früherer staatsanwalt eine pension bezieht. weil der konservative staatschef den behördengang nicht auf sich nahm und damit formell gesehen den lebensbeweis schuldig blieb, wurde die pension für november und dezember nicht ausgezahlt, wie die zeitung "o globo" kürzlich berichtete. der zufall will, dass der höchst unbeliebte präsident derzeit versucht, eine umstrittene rentenreform durchzuboxen. unter anderem soll das renteneintrittsalter erhöht werden. +++ 02.15 uhr: schulz nicht ins kabinett? spd-debatte um parteichef +++ in der spd gibt es wachsende bedenken gegen den gang von parteichef martin schulz als minister und vizekanzler in das kabinett der geplanten großen koalition. intern wird die frage nach informationen der deutschen presse-agentur verstärkt diskutiert, aber wegen der laufenden verhandlungen und mit blick auf die autorität von schulz sind nur wenige bereit, sich hierzu öffentlich klar zu äußern. "ich glaube, dass es im moment sehr schwer zu vermitteln ist, dass der vorsitz der partei vereinbar ist mit der organisationstätigkeit eines vizekanzlers und der reisetätigkeit eines außenministers", sagte die niedersächsische landtagsabgeordnete doris schröder-köpf der dpa. entscheidend sei der erneuerungsprozess. "die partei braucht jetzt sehr intensive betreuung, eine art wiederaufbau quasi." zuvor hatte unter anderem der designierte thüringische spd-chef wolfgang tiefensee in der "welt" schulz dazu aufgefordert, nicht in das geplante kabinett von kanzlerin angela merkel (cdu) zu gehen. +++ 01.49 uhr: umfrage: afd und fdp steigen in der wählergunst +++ während union und spd eine neuauflage der großen koalition verhandeln, steigen afd und fdp in der wählergunst. in einer emnid-erhebung für die "bild am sonntag" kommt die afd nun auf 13 prozent, die fdp auf 9 prozent. beide parteien verbesserten sich gegenüber der vorwoche um je einen zähler. die union verliert dagegen an zustimmung. die cdu/csu kommt auf 33 prozent, ein punkt weniger als vor einer woche. die spd bleibt bei 20 prozent. auch grüne (11 prozent) und linke (10 prozent) bleiben unverändert. +++ 01.45 uhr: komplikationen nach darm-op - tauber musste notoperiert werden +++ cdu-generalsekretär peter tauber hat sich einem medienbericht zufolge wegen gefährlicher komplikationen nach einer geplanten darmoperation einer weiteren not-op unterziehen müssen. wie "bild am sonntag" berichtete, verbrachte der 43-jährige nach dem eingriff zwölf tage auf der intensivstation. inzwischen befinde sich der politiker auf dem weg der besserung, er habe das krankenhaus am donnerstag verlassen. "das war eine extrem harte zeit", sagte tauber der zeitung. es gehe ihm deutlich besser, "aber ich werde noch einige zeit brauchen, um wieder vollständig gesund zu werden". dem bericht zufolge beginnt tauber kommende woche eine mehrwöchige reha. sein nächstes ziel: "in der sonne sitzen und eine eiskalte cola trinken." +++ 01.04 uhr: oettinger: eu-mittel für bauern und regionen werden gekürzt +++ deutsche bauern und regionen werden ab 2021 wohl weniger geld aus brüssel bekommen. eu-haushaltskommissar günther oettinger sagte der "welt am sonntag": "es wird keinen kahlschlag geben, wie einige befürchten. aber auch in deutschland werden sich landwirte und regionen auf finanzielle kürzungen einstellen müssen." die entsprechenden fördertöpfe - die agrar- und kohäsionsfonds - sollen nach seinen worten um jeweils fünf bis zehn prozent gekürzt werden. oettinger bereitet derzeit seinen entwurf für den nächsten mehrjährigen haushaltsrahmen für die jahre nach 2020 vor. im budget fehlen dann wegen des brexits voraussichtlich bis zu 14 milliarden euro an britischen beiträgen. gleichzeitig will oettinger für einige aufgaben wie verteidigung oder migrationspolitik mehr geld einplanen. im haushalt soll deshalb umgeschichtet werden. zudem sollen die eu-länder zehn bis 20 prozent mehr einzahlen. deutschland wird nach oettingers worten "eine mehrbelastung im einstelligen milliardenbereich zu schultern haben". +++ 00.42 uhr: mindestens elf soldaten bei selbstmordanschlag in pakistan getötet +++ bei einem selbstmordanschlag im nordwesten pakistans sind am samstag mindestens elf soldaten getötet worden. 13 weitere soldaten seien bei der attacke auf einen stützpunkt im swat-tal verletzt worden, teilte die pakistanische armee mit. demnach nahm der angreifer den sportplatz des stützpunkts ins visier. die radikalislamischen pakistanischen taliban nahmen die tat für sich in anspruch. im nordwesten pakistans verüben extremisten immer wieder anschläge, auch wenn die gewalt in der grenzregion zu afghanistan in den vergangenen jahren zurückgegangen ist. die pakistanischen taliban hatten zwischen 2007 und 2009 faktisch die kontrolle über das swat-tal und setzten dort die scharia durch, bis sie von der armee zurückgedrängt wurden. & 1447 & medium & Medium & Power & NA & NA & 2018-02-04 & 2018 & 3 & POL
Frame & low-medium & National & +1000 & -1.0405052 & -1.0641830 & 1.2185583 & -1.5781059 & 1.4491718 & 12.0 & 1.6295860 & 1.9019065 & Payer & Domestic & European & Mixed & Domestic|POL & Negative\\
\addlinespace
Germany & http://www.abendblatt.de/politik/ausland/article209896973/Guenther-Oettinger-stellt-Hilfsgelder-fuer-die-Tuerkei-infrage.html & 236 & Hamburger Abendblatt & Private/Non-Public & Online and Offline & Regional/Local & very low = CP mentioned once & Political leverage & Factual & EU & No myth & Institutional bargaining over funding & Factual & EU & No myth & NA & NA & NA & NA & Germany & günther oettinger stellt hilfsgelder für die türkei infrage & 2017-03-11 & kohäsionsfonds & entfernt sich die türkei von westlichen werten, hat das womöglich folgen für die beihilfe. ein eu-beitritt würde so unwahrscheinlich. brüssel. vor wenigen monaten ist günther oettinger vom eu-kommissar für digitale gesellschaft und wirtschaft zum haushaltskommissar aufgestiegen - und ist jetzt mittendrin auf zentralen baustellen der eu: ob hilfsgelder für die türkei, das verhältnis zu polen oder die kosten des brexit, der 63-jährige cdu-politiker scheut auch konflikte nicht. sein blick auf die zukunft der eu ist dennoch überraschend optimistisch. herr oettinger, mehrere eu-staaten verbieten wahlkampfauftritte von türkischen ministern. muss nicht auch die bundesregierung klare haltung zeigen und auftritte stoppen - erst recht nach den nazi-vorwürfen? günther oettinger: was einige türkische regierungsmitglieder äußern, ist in stil und inhalt nicht akzeptabel. der nazi-vergleich entbehrt jeder grundlage. generell ist meinungsfreiheit ein hohes gut, die versammlungsfreiheit auch. ich rate daher, unhaltbare behauptungen zurückzuweisen und im einzelfall abzuwägen, ob ein verbot erforderlich ist - aber ansonsten ist gelassenheit gefragt. video merkel - nazi-vergleiche durch türkische politiker müssen aufhören mit wenigen ländern habe deutschland so komplizierte, aber zugleich so vielfältige verbindungen wie mit der türkei, sagte merkel am... haben sie eine erklärung für die ausfälle? oettinger: präsident erdogan ist sichtbar nervös: es ist unklar, ob er das verfassungsreferendum gewinnt. wir sollten auf diese verbalen entgleisungen nicht so reagieren, wie er hofft. wäre eine gemeinsame eu-linie im umgang mit solchen auftritten nicht sinnvoll? oettinger: eine europäisierung ist in dem fall schwierig: in deutschland sind meinungs- und versammlungsfreiheit in der verfassung verankert. einige eu-mitglieder geraten beim thema meinungsfreiheit eher auf die schiefe ebene. warum bekommt die türkei von der eu hunderte millionen euro vor-beitrittshilfen - obwohl auf beiden seiten kaum noch jemand an den eu-beitritt glaubt? oettinger: wir geben solche hilfen an zahlreiche länder, etwa auch an die beitrittskandidaten auf dem westbalkan. aber wenn wir dauerhaft feststellen, dass sich die entwicklung von den werten europas entfernt, kann dies folgen für die finanzierung haben. es gibt einen klaren zusammenhang zwischen beitrittsziel, der verpflichtung, unsere werte zu übernehmen, und den finanzhilfen - mit denen wollen wir den weg nach europa ebnen, nicht das gegenteil fördern. die beitrittshilfen sind bei den beratungen über den haushalt 2018 und auch bei der vorbereitung zum mehrjährigen finanzrahmen ein thema. video erdogan: wenn ich nach deutschland will, komme ich der türkische präsident erdogan will sich ungeachtet der scharfen kritik an seinem nazi-vergleich nicht von einem deutschland-besuch... halten sie den eu-beitritt der türkei noch für möglich? oettinger: ein eu-beitritt der türkei kommt in diesem jahrzehnt sicher nicht, im nächsten jahrzehnt ist er nicht absehbar, und unter einem präsidenten erdogan ist er wenig wahrscheinlich. aber die türkei war und ist beitrittskandidat. diesen status würde sie sicher gefährden, wenn nicht verlieren, wenn sie zum beispiel im strafrecht die todesstrafe einführen würde. wir konzentrieren uns bei den verhandlungen derzeit auf fragen der rechtsstaatlichkeit, die sehen wir als gefährdet an. der dialog mit der türkei ist hilfreich und sicher besser, als die gesprächsfäden abzuschneiden. hat uns erdogan wegen des flüchtlingsabkommens in der hand, kann er die bundesregierung erpressen? oettinger: eine entwicklung der flüchtlingszahlen wie im spätsommer 2015 ist derzeit nicht zu befürchten. an der einhaltung des abkommens hat die eu ebenso wie die türkei ein interesse, es geht um leistung und gegenleistung. die eu zahlt die zugesagten mittel von drei milliarden in vollem umfang. damit unterstützen wir die menschenwürdige unterbringung in den flüchtlingslagern in der türkei. im übrigen beteiligt sich die türkei sehr konstruktiv an der gemeinsamen bekämpfung des islamischen terrorismus und an der lösung der konflikte im irak und in syrien. am mittwoch wählen die niederlande: wird geert wilders der erste rechtspopulistische regierungschef eines eu-gründungsstaates? oettinger: nein, ganz sicher nicht. alle gemäßigten parteien dort sind bereit, für eine regierungsbildung zusammenzuarbeiten. wilders hat keine mehrheit in aussicht, und seine umfragewerte gehen zurück. der stil von us-präsident trump, seine twitterei eingeschlossen, schadet den rechtspopulisten in europa. wo demokraten keine fehler machen, haben die populisten in europa ihren höhepunkt hinter sich. das könnte auch bei der afd der fall sein, ihre umfragewerte sind schon gekippt. video darum ist die wahl in den niederlanden auch für uns spannend geert wilders lag in den umfragen lange vorn, jetzt wurde der rechtspopulist eingeholt. wie groß seine chancen stehen und was die wahl... in frankreich sieht es anders aus. könnte die eu eine präsidentin marine le pen verkraften? oettinger: von einer präsidentin le pen gehe ich nicht aus. die franzosen sind in ihrer klaren mehrheit sehr europafreundlich. die demokratischen parteien dürfen allerdings nicht noch weitere fehler machen. polen ist auf klarem konterkurs zu den rechtsstaats- und demokratie-grundsätzen der eu. ist es strafe genug, dass die partner gegen den willen warschaus donald tusk als ratspräsident im amt bestätigt haben? oettinger: es ist schon ärgerlich, wie die polnische regierung hier europäische gremien für innenpolitische parteien-spiele zu nutzen versuchte. aber die anderen eu-staaten haben mit der wahl von tusk ein klares zeichen gesetzt. was die eingriffe in die meinungs- und pressefreiheit in polen betrifft, widerspricht dies den prinzipien der rechtsstaatlichkeit und den eu-verträgen. video showdown um tusk auf eu-gipfel die wiederwahl von donald tusk als eu-ratspräsident droht den eu-gipfel zu sprengen. polens ministerpräsidentin beata szydlo machte... der österreichische kanzler kern hat mit blick auf polen oder ungarn vorgeschlagen, eu-staaten gelder zu streichen, wenn sie grundwerte nicht einhalten. gute idee? oettinger: es ist nach dem haushaltsrecht gar nicht so leicht, ihnen zustehende gelder zu verweigern. wenn der eu wegen des brexit demnächst neun milliarden euro netto fehlen, ist die bereitschaft der anderen nettozahler, den einnahmerückgang auszugleichen und außerdem die kohäsionsfonds etwa für polen und ungarn auf hohem niveau zu halten, nicht selbstverständlich. dann kommt es auch auf das binnenklima in der eu an und darauf, welche signale die betroffenen regierungen aussenden. das soll keine drohung sein. aber es kommt jetzt darauf an, dass alle eu-partner klug handeln. in der debatte um die zukunft der eu ist wieder von einem europa der unterschiedlichen geschwindigkeiten die rede. aber taugt das als zauberformel? dieses europa haben wir ja schon ... oettinger: die unterschiedliche geschwindigkeit ist sicher nicht die lösung für alles, aber ein ergänzendes instrument. so sollte nicht immer der langsamste im geleitzug das tempo bestimmen oder gar im bremserhäuschen blockieren. wir sollten auf dem weg zu mehr europa etwa bei der migrations- und flüchtlingspolitik, bei der verteidigung oder terrorbekämpfung weiter gehen, in anderen bereichen die zahl der europäischen initiativen dagegen gering halten. in einigen feldern werden mitgliedsstaaten vorangehen, andere werden nicht dabei sein - wie das beim euro, dem schengen-abkommen oder jetzt der europäischen staatsanwaltschaft schon der fall ist. wie sieht europa in 20 jahren aus? oettinger: in 20 jahren werden wir nach dem brexit kein weiteres mitglied verloren haben und eine generation von briten erleben, die in parlament, regierung und öffentlichkeit einen erneuten beitritt überlegt. die staaten des westbalkans haben die beitrittsbedingungen erfüllt und sind eu-mitglieder, der frieden in dieser region ist gesichert. und die weiterentwicklung der verträge wird aus der heutigen kommission eine echte regierung machen, mit einem starken parlament und einem starken rat. so schwierig die entwicklung in der türkei ist, so sehr uns internationale krisen in atem halten und manche tweets aus washington irritieren, sie befördern doch auch eine erkenntnis: europa muss erwachsen werden. & 1206 & very low & Low & Power & Power & NA & 2017-03-11 & 2017 & 2 & POL
Frame & v.low & Regional & +1000 & -1.0405052 & -1.0641830 & 1.2185583 & -1.5781059 & 1.4491718 & 12.0 & 1.6295860 & 1.9019065 & Payer & European & European & European & European|POL & Neutral\\
Germany & http://www.waz.de/politik/article209896973.ece & 217 & WAZ & Private/Non-Public & Online and Offline & Regional/Local & very low = CP mentioned once & Political leverage & Balanced & National + Other country & No myth & NA & NA & NA & NA & NA & NA & NA & NA & Germany & eu-kommissar oettinger stellt hilfsgelder für türkei infrage & 2017-03-11 & kohäsionsfonds & brüssel entfernt sich die türkei von westlichen werten, hat das womöglich folgen für die beihilfe. ein eu-beitritt würde so unwahrscheinlich. vor wenigen monaten ist günther oettinger vom eu-kommissar für digitale gesellschaft und wirtschaft zum haushaltskommissar aufgestiegen - und ist jetzt mittendrin auf zentralen baustellen der eu: ob hilfsgelder für die türkei, das verhältnis zu polen oder die kosten des brexit, der 63-jährige cdu-politiker scheut auch konflikte nicht. sein blick auf die zukunft der eu ist dennoch überraschend optimistisch. herr oettinger, mehrere eu-staaten verbieten wahlkampfauftritte von türkischen ministern. muss nicht auch die bundesregierung klare haltung zeigen und auftritte stoppen - erst recht nach den nazi-vorwürfen? günther oettinger: was einige türkische regierungsmitglieder äußern, ist in stil und inhalt nicht akzeptabel. der nazi-vergleich entbehrt jeder grundlage. generell ist meinungsfreiheit ein hohes gut, die versammlungsfreiheit auch. ich rate daher, unhaltbare behauptungen zurückzuweisen und im einzelfall abzuwägen, ob ein verbot erforderlich ist - aber ansonsten ist gelassenheit gefragt. haben sie eine erklärung für die ausfälle? oettinger: präsident erdogan ist sichtbar nervös: es ist unklar, ob er das verfassungsreferendum gewinnt. wir sollten auf diese verbalen entgleisungen nicht so reagieren, wie er hofft. wäre eine gemeinsame eu-linie im umgang mit solchen auftritten nicht sinnvoll? oettinger: eine europäisierung ist in dem fall schwierig: in deutschland sind meinungs- und versammlungsfreiheit in der verfassung verankert. einige eu-mitglieder geraten beim thema meinungsfreiheit eher auf die schiefe ebene. warum bekommt die türkei von der eu hunderte millionen euro vor-beitrittshilfen - obwohl auf beiden seiten kaum noch jemand an den eu-beitritt glaubt? oettinger: wir geben solche hilfen an zahlreiche länder, etwa auch an die beitrittskandidaten auf dem westbalkan. aber wenn wir dauerhaft feststellen, dass sich die entwicklung von den werten europas entfernt, kann dies folgen für die finanzierung haben. es gibt einen klaren zusammenhang zwischen beitrittsziel, der verpflichtung, unsere werte zu übernehmen, und den finanzhilfen - mit denen wollen wir den weg nach europa ebnen, nicht das gegenteil fördern. die beitrittshilfen sind bei den beratungen über den haushalt 2018 und auch bei der vorbereitung zum mehrjährigen finanzrahmen ein thema. halten sie den eu-beitritt der türkei noch für möglich? oettinger: ein eu-beitritt der türkei kommt in diesem jahrzehnt sicher nicht, im nächsten jahrzehnt ist er nicht absehbar, und unter einem präsidenten erdogan ist er wenig wahrscheinlich. aber die türkei war und ist beitrittskandidat. diesen status würde sie sicher gefährden, wenn nicht verlieren, wenn sie zum beispiel im strafrecht die todesstrafe einführen würde. wir konzentrieren uns bei den verhandlungen derzeit auf fragen der rechtsstaatlichkeit, die sehen wir als gefährdet an. der dialog mit der türkei ist hilfreich und sicher besser, als die gesprächsfäden abzuschneiden. hat uns erdogan wegen des flüchtlingsabkommens in der hand, kann er die bundesregierung erpressen? oettinger: eine entwicklung der flüchtlingszahlen wie im spätsommer 2015 ist derzeit nicht zu befürchten. an der einhaltung des abkommens hat die eu ebenso wie die türkei ein interesse, es geht um leistung und gegenleistung. die eu zahlt die zugesagten mittel von drei milliarden in vollem umfang. damit unterstützen wir die menschenwürdige unterbringung in den flüchtlingslagern in der türkei. im übrigen beteiligt sich die türkei sehr konstruktiv an der gemeinsamen bekämpfung des islamischen terrorismus und an der lösung der konflikte im irak und in syrien. am mittwoch wählen die niederlande: wird geert wilders der erste rechtspopulistische regierungschef eines eu-gründungsstaates? oettinger: nein, ganz sicher nicht. alle gemäßigten parteien dort sind bereit, für eine regierungsbildung zusammenzuarbeiten. wilders hat keine mehrheit in aussicht, und seine umfragewerte gehen zurück. der stil von us-präsident trump, seine twitterei eingeschlossen, schadet den rechtspopulisten in europa. wo demokraten keine fehler machen, haben die populisten in europa ihren höhepunkt hinter sich. das könnte auch bei der afd der fall sein, ihre umfragewerte sind schon gekippt. in frankreich sieht es anders aus. könnte die eu eine präsidentin marine le pen verkraften? oettinger: von einer präsidentin le pen gehe ich nicht aus. die franzosen sind in ihrer klaren mehrheit sehr europafreundlich. die demokratischen parteien dürfen allerdings nicht noch weitere fehler machen. polen ist auf klarem konterkurs zu den rechtsstaats- und demokratie-grundsätzen der eu. ist es strafe genug, dass die partner gegen den willen warschaus donald tusk als ratspräsident im amt bestätigt haben? oettinger: es ist schon ärgerlich, wie die polnische regierung hier europäische gremien für innenpolitische parteien-spiele zu nutzen versuchte. aber die anderen eu-staaten haben mit der wahl von tusk ein klares zeichen gesetzt. was die eingriffe in die meinungs- und pressefreiheit in polen betrifft, widerspricht dies den prinzipien der rechtsstaatlichkeit und den eu-verträgen. der österreichische kanzler kern hat mit blick auf polen oder ungarn vorgeschlagen, eu-staaten gelder zu streichen, wenn sie grundwerte nicht einhalten. gute idee? oettinger: es ist nach dem haushaltsrecht gar nicht so leicht, ihnen zustehende gelder zu verweigern. wenn der eu wegen des brexit demnächst neun milliarden euro netto fehlen, ist die bereitschaft der anderen nettozahler, den einnahmerückgang auszugleichen und außerdem die kohäsionsfonds etwa für polen und ungarn auf hohem niveau zu halten, nicht selbstverständlich. dann kommt es auch auf das binnenklima in der eu an und darauf, welche signale die betroffenen regierungen aussenden. das soll keine drohung sein. aber es kommt jetzt darauf an, dass alle eu-partner klug handeln. in der debatte um die zukunft der eu ist wieder von einem europa der unterschiedlichen geschwindigkeiten die rede. aber taugt das als zauberformel? dieses europa haben wir ja schon ... oettinger: die unterschiedliche geschwindigkeit ist sicher nicht die lösung für alles, aber ein ergänzendes instrument. so sollte nicht immer der langsamste im geleitzug das tempo bestimmen oder gar im bremserhäuschen blockieren. wir sollten auf dem weg zu mehr europa etwa bei der migrations- und flüchtlingspolitik, bei der verteidigung oder terrorbekämpfung weiter gehen, in anderen bereichen die zahl der europäischen initiativen dagegen gering halten. in einigen feldern werden mitgliedsstaaten vorangehen, andere werden nicht dabei sein - wie das beim euro, dem schengen-abkommen oder jetzt der europäischen staatsanwaltschaft schon der fall ist. wie sieht europa in 20 jahren aus? oettinger: in 20 jahren werden wir nach dem brexit kein weiteres mitglied verloren haben und eine generation von briten erleben, die in parlament, regierung und öffentlichkeit einen erneuten beitritt überlegt. die staaten des westbalkans haben die beitrittsbedingungen erfüllt und sind eu-mitglieder, der frieden in dieser region ist gesichert. und die weiterentwicklung der verträge wird aus der heutigen kommission eine echte regierung machen, mit einem starken parlament und einem starken rat. so schwierig die entwicklung in der türkei ist, so sehr uns internationale krisen in atem halten und manche tweets aus washington irritieren, sie befördern doch auch eine erkenntnis: europa muss erwachsen werden. seite ... ... nächste seite & 1093 & very low & Low & Power & NA & NA & 2017-03-11 & 2017 & 2 & POL
Frame & v.low & Regional & +1000 & -1.0405052 & -1.0641830 & 1.2185583 & -1.5781059 & 1.4491718 & 12.0 & 1.6295860 & 1.9019065 & Payer & Domestic & European & Mixed & Domestic|POL & Neutral\\
Germany & http://www.faz.net/aktuell/wirtschaft/wirtschaftspolitik/was-bei-brexit-scheidungsverhandlungen-herauskommen-kann-14305986.html & 268 & Frankfurter Allgemeine & Private/Non-Public & Online and Offline & National & very low = CP mentioned once & Institutional bargaining over funding & Factual & EU + Other country & No myth & NA & NA & NA & NA & NA & NA & NA & NA & Germany & britischer eu-austritt: was bei der scheidung herauskommen kann & 2016-06-24 & kohäsionsfonds & © reuters für die briten heißt es nun: raus aus der eu - aber wie genau? © reuters für die briten heißt es nun: raus aus der eu - aber wie genau? viele haben bis zum schluss nicht daran geglaubt, doch jetzt ist es da - das brexit-votum. die briten wollen aus der eu austreten. niemals in ihrer geschichte hat die eu vor einer solchen herausforderung gestanden. wie werden sich großbritannien und der rest der gemeinschaft nun auseinanderdividieren? wie werden sie wirtschafltich weiter zusammenarbeiten? nach dem \#brexit-votum kommen die scheidungsverhandlungen. was könnte rauskommen? zunächst bleibt festzuhalten: nach dem brexit-votum sind die briten noch nicht sofort raus aus der eu. im eu-vertrag ist festgelegt, dass die regierung den austritt erst einmal offiziell bei der eu ankündigen muss. dann gibt es eine zwei-jahres-frist. in dieser zeit geht es sozusagen um die "scheidungsverträge", die die restlichen mitgliedstaaten mit großbritannien detailgenau ausverhandeln müssen. offiziell muss jedes land, das aus der eu austreten möchte, den europäischen rat darüber informieren. erst danach beginnen die austrittsverhandlungen. einige politiker der kampagne "vote leave", also austrittsbefürworter, haben nun vorgeschlagen, diese mitteilung nicht sofort abzugeben. statt dessen wollen sie in einen informellen verhandlungsprozess treten, also erst einmal die stimmungslage ausloten. ob das wirklich so kommt, ist aber noch unklar. norwegen-, schweiz- oder wto-modell das große fragezeichen ist jedenfalls: welches verhältnis wird großbritannien künftig zu anderen eu-staaten haben? es gibt dafür prinzipiell drei szenarien: eines funktioniert ähnlich wie das modell der norweger, eines wie das modell der schweiz; das dritte setzt das verhältnis großbritanniens zur eu eher dem von bangladesch oder anderer "drittstaaten" gleich. der britische schatzkanzler george osborne, eindeutig dem lager der eu-befürworter zuzuordnen, hat schon vor dem votum eine umfangreiche analyse der möglichen verwerfungen veröffentlicht, knapp 200 seiten lang, mit diversen berechnungen, differenziert nach der form der künftigen zusammenarbeit mit der eu (die frankfurter allgemeine sonntagszeitung berichtete). wählte großbritannien die engste variante einer zusammenarbeit, so wie sie zurzeit etwa norwegen praktiziert, wäre der schock für die wirtschaft am geringsten: knapp vier prozent der wirtschaftsleistung müsste großbritannien osbornes berechnungen zufolge dann für seine neue freiheit opfern. mehr zum thema rezessionsängste durch brexit: "zuerst überreagieren, später nachdenken" liveticker zum brexit - briten stimmen über eu-austritt ab brexit: cameron kündigt rücktritt an börsen-fusion wird nach brexit-votum schwieriger wie die deutschen unternehmen unter dem brexit leiden werden wie funktioniert das norwegische modell? das land ist seit 1994 teil des europäischen wirtschaftsraums (ewr), lehnt es aber beharrlich ab, eu-mitglied zu werden. der ewr öffnet gleichwohl seinen mitgliedern den eu-binnenmarkt - daher wird auch davon ausgegangen, dass diese variante zu relativ geringen schocks führen würde. doch die skandinavier müssen im gegenzug auch einen recht hohen preis für eben diesen zugang zum binnenmarkt zahlen und das wäre für großbritannien nicht anders: ewr-mitglieder müssen in den europäischen kohäsionsfonds einzahlen. der ist dafür da, zwischen armen und reichen eu-ländern umzuverteilen. norwegen überweist an die eu etwa genauso viel geld wie die "echten" mitgliedstaaten, allerdings nimmt das land auch noch an einer reihe weiterer eu-programme freiwillig teil, die geld kosten. es bestehen zweifel, ob die brexit-befürworter eine solche variante wollen. dem britischen freiheitsdrang würde daher wohl eher ein zweites modell entgegenkommen, in dem die zusammenarbeit ähnlich wie mit der schweiz von fall zu fall vereinbart würde - ein anstrengender prozess, der das bruttoinlandsprodukt laut osborne jährlich um bis zu 6 prozent drücken würde. die schweiz ist mitglied der europäischen freihandelszone efta, hat es aber anders als norwegen abgelehnt, dem ewr beizutreten. die regierung in bern handelte deshalb über jahrzehnte mit brüssel mehr als 120 bilaterale abkommen aus, die der schweiz gleichfalls eine weitgehende teilnahme am eu-binnenmarkt ermöglichen - allerdings bisher nicht, wenn es um dienstleistungen geht. sollte großbritannien ein ähnliches modell anstreben, wären die zwei jahre zeit für die austrittsverhandlungen äußerst knapp bemessen. \$stlekurz to view this video please enable javascript, and consider upgrading to a web browser that supports html5 video © dpa, reuters cameron kündig rücktritt an am schwierigsten wäre die so genannte wto-variante zu verkraften. diese ließe großbritannien die größte autonomie und knüpft lediglich an die bestehenden freihandelsregeln der welthandelsorganisation an. das würde osborne zufolge 7,5 prozent der wirtschaftsleistung kosten, heruntergebrochen auf jeden haushalt 6600 euro im jahr. die wto-variante würde zwar auch zu handelserleichterungen mit der eu führen, aber keinen zugang zum binnenmarkt bringen. zwar gibt es über die wto vereinbarte zollsenkungen, doch generell wären für einfuhren in die eu wieder zölle zu zahlen. die wto-option, stellte großbritannien im eu-verhältnis auf eine stufe mit ländern wie bangladesch. wie in den beiden anderen szenarien auch, hätte großbritannien auch keinen zugang zu den freihandelsvereinbarungen der eu mit anderen staaten mehr. auch die müsste london neu aushandeln. wie teuer das würde, hat wto-chef roberto azevedo kürzlich vorgerechnet. & 794 & very low & Low & Power & NA & NA & 2016-06-24 & 2016 & 2 & POL
Frame & v.low & National & 500-1000 & -1.0405052 & -1.0641830 & 1.2185583 & -1.5781059 & 1.4491718 & 12.0 & 1.6295860 & 1.9019065 & Payer & European & European & European & European|POL & Neutral\\
Germany & http://www.dw.com/de/polen-will-ein-anderes-europa/a-44525873 & 280 & Deutsche Welle (English) & Public & Online and Offline & National & very low = CP mentioned once & Solidarity to poor countries/regions & Positive & EU + Other country & No myth & NA & NA & NA & NA & NA & NA & NA & NA & Germany & polen will ein anderes europa & 2018-07-04 & kohäsionsfonds & polens premier morawiecki wirbt im eu-parlament für die eu als wirtschafts- und verteidigungsbündnis unabhängiger nationen ohne gemeinsame regeln. kritik an seiner justizreform weist er zurück. aus brüssel barbara wesel. es ist zufall, dass tadeusz morawiecki ausgerechnet in dieser woche an der reihe ist, im europaparlament seine vision für europa vorzutragen, gerade zwei tage nachdem die eu-kommission ein vertragsverletzungsverfahren gegen seine regierung eröffnet hat. während in polen richter gegen ihre zwangspensionierung protestieren, verteidigt der regierungschef die seit langem umstrittene justizreform: "jedes land hat das recht, sein eigenes rechtssystem mit hinblick auf seine eigenen traditionen einzurichten." was er in straßburg entwirft, ist die vision eines illiberalen europa ohne demokratische zwänge, eines bündnisses total unabhängiger nationalstaaten ohne politische verpflichtungen für die gemeinschaft. eu als loses bündnis von nationalstaaten "europa muss die beziehung zwischen mitgliedsländern und den eu-institutionen neu ausbalancieren", fordert morawiecki. und er nutzt dafür den begriff der freiheit der bürger, der allerdings beim polnischen premier nur von unabhängigen nationalstaaten ausgeübt werden soll. nationale identitäten müssten mehr respektiert werden und die zukunft der eu sei nicht vorherbestimmt: "es ist nicht unausweichlich, dass weitere macht von den mitgliedsstaaten auf die eu-ebene übergeht", sagt polens premier. weitere integration sei nie ein selbstzweck gewesen. andererseits aber will der polnische premier die wohltaten der eu unbedingt erhalten: man müsse die ungleichheit bekämpfen, und darum gebe es in den haushaltsverhandlungen jetzt große kämpfe. aber die kohäsionsfonds müssten unbedingt erhalten werden, weil sie helfen könnten, trennungen in der eu zu überwinden. morawiecki kämpft hier um den künftigen anteil seines landes an eu-hilfen. bisher war polen größter netto-empfänger, ist aber im nächsten budget von kürzungen bedroht: polens wirtschaftsleistung ist zuletzt gestiegen, zudem berät die eu über mögliche einschnitte wegen der polnischen weigerung, an einer eu-weiten verteilung der flüchtlinge mitzuarbeiten. die umverteilung der mittel also will der polnische premier unbedingt erhalten und darüber hinaus die eu als verteidigungsbündnis ausbauen, um der aggression russlands entgegenzutreten. außerdem will morawiecki die nato stärken; er sieht polen als bollwerk gegen einen neuen anti-amerikanismus und will weiter gute beziehungen zu den usa pflegen. polen verbittet sich kritik "es ist nicht gut, wenn stolze nationen von anderen belehrt werden": mit diesem kernsatz verbittet sich der regierungschef kritik der eu-kommission an der justizreform und an anderen von der pis-partei eingeführten neuregelungen in polen. so etwas schaffe nur polarisierung, und die "stolze polnische nation" lehne diese art von einmischung ab. die eu hat inzwischen in schier endlosen gesprächsrunden auf den erhalt der gewaltenteilung und der unabhängigkeit der justiz in polen gedrungen, weil alle mitgliedsländer die demokratischen grundrechte achten müssten. das führte allerdings nicht zu einem umdenken in warschau, so dass zum ersten mal in der geschichte ein artikel 7 verfahren wegen anhaltender und schwerer vertragsverletzung eingeleitet wurde. am ende könnte dies sogar zu einem entzug des stimmrechts für polen führen. aber mateusz morawiecki weist alle vorwürfe aus brüssel zurück und bestreitet rundweg ihren wahrheitsgehalt: die medien in polen seien völlig frei und die justiz werde nach der reform unabhängiger sein denn je. im alten system hätte es zu viele postkommunisten gegeben: "osteuropa wird manchmal in einem falschen licht gesehen", beklagt der premier. "wir brauchen kein neues europa" manfred weber, fraktionsvorsitzender der europäischen volkspartei im europäischen parlament, attackiert morawiecki auf ganzer linie für das verhalten seiner regierung: "es gibt so viele fragen: warum ist das staatliche fernsehen ein propagandainstrument und kein freies medium mehr? warum werden friedliche demonstranten verhaftet, während die gewalt von rechtsextremen nicht geahndet wird?", so weber. "es gibt keine freiheit ohne rechtsstaatlichkeit", betont er. und er stellt infrage, ob es richtig sei, für polen nur die souveränität der staaten zu beschwören und ein anti-eu klima im land zu schaffen. weber beschwört schließlich: "wir brauchen kein anderes europa, es gibt bereits eine eu." ein europa des nationalismus und des egoismus aber könne nicht funktionieren und für das wohl der bürger arbeiten. der evp-vorsitzende beklagt auch die von der pis-regierung geschürte anti-deutsche stimmung, die die verständigung erschwere. das problem bei diesem leidenschaftlichen appell für ein demokratisches europa ist nur, dass es glaubwürdiger wäre, wenn die christdemokraten auch den ungarn viktor orbàn kritisieren würden. er aber gehört zu ihrer parteiengruppe und wird deshalb verschont, obwohl die abschaffung der medienfreiheit und die gleichschaltung der justiz in ungarn ein mindestens ebenso großes problem und das land auf dem weg zur autokratie weit fortgeschritten ist. auch von den sozialdemokraten und vor allem vom wortgewaltigen chef der liberalen kommt scharfe kritik an den aktionen der regierung in warschau: "die eu ist eine gemeinschaft der werte. richter unter (politische) kontrolle zu stellen, ist inakzeptabel und kann nicht toleriert werden", erklärt guy verhofstadt. das sei keine frage der tradition, sondern eine frage der grundsätze. und er appelliert an morawiecki persönlich: "drehen sie das rad zurück und bringen sie polen zurück in die familie demokratischer staaten, wenden sie sich ab von der illusion der sogenannten illiberalen staaten, zurück in den mittelpunkt europäischer politik." die regierung in warschau wird diesen ruf ignorieren wird und eher darauf setzen, die neue teilung europas in populistisch gefärbte, nur noch dem namen nach demokratische staaten mit nationalistischer grundhaltung zu unterstützen. bislang hatte die pis-regierung ihre verbündeten nur in der visegrad-gruppe, jetzt kann sie auf neue freunde bei rechtsextremen etwa in österreich oder italien hoffen. mateusz morawiecki und seine parteifreunde sehen sich in der eu derzeit eher auf der gewinnerstraße, so viel wurde aus seiner extrem kämpferischen rede klar. & 902 & very low & Low & Values & NA & NA & 2018-07-04 & 2018 & 3 & ECO
Frame & v.low & National & 500-1000 & -1.0405052 & -1.0641830 & 1.2185583 & -1.5781059 & 1.4491718 & 12.0 & 1.6295860 & 1.9019065 & Payer & European & European & European & European|ECO & Positive\\
Germany & https://www.handelsblatt.com/politik/international/auftritt-im-europaparlament-polens-premier-geht-auf-konfrontationskurs-zur-eu/22766418.html & 201 & Handelsblatt & Private/Non-Public & Online and Offline & National & very low = CP mentioned once & Political leverage & Negative & National & No myth & NA & NA & NA & NA & NA & NA & NA & NA & Germany & polens premier geht auf konfrontationskurs zur eu & 2018-07-04 & kohäsionsfonds & morawiecki fordert in straßburg eine "union 4.0" - und erntet wütende proteste. die beziehungen zwischen der eu und polen sind vergiftet wie nie zuvor. mateusz morawiecki "80 bis 90 prozent der medien berichten kritisch über die regierung", beklagt polens premierminister. (foto: afp) straßburgals frankreichs präsident im frühjahr kam, war das plenum des europarlaments bis fast auf den letzten platz besetzt. kaum einer der 751 abgeordneten wollte verpassen, was emmanuel macron zur zukunft europas zu sagen hatte. nun ist der polnische premierminister nach straßburg gekommen. viele reihen in dem riesigen plenarsaal blieben leer, als mateusz morawiecki sein konzept einer "europäischen union 4.0" vorstellte. auch eu-kommissionspräsident jean-claude juncker glänzte durch abwesenheit. sein erster stellvertreter frans timmermans war ebenfalls verhindert. als lückenbüßer musste eu-kommissionsvize valdis dombrovski herhalten. der lette ist eigentlich für die währungsunion und für finanzmarktregulierung zuständig. nun musste er in straßburg den polnischen premier empfangen, weil juncker und timmermans offenkundig keine lust dazu hatten. damit war schon vor redebeginn klar, dass die atmosphäre zwischen der eu und ihrem größten osteuropäischen mitgliedstaat einen neuen tiefpunkt erreicht hat. und morawiecki tat nichts, um die stimmung zu verbessern. polen wolle "keinen europäischen superstaat" und erwarte "respekt" vor seinen nationalen besonderheiten, verkündete der pole. den vorwurf der kommission, dass polen die justiz gleichschalte, damit den rechtsstaat beschädige und eu-grundwerte verletze, wies morawiecki in bausch und bogen zurück. die polnischen richter seien "unabhängiger als früher". es stimme auch nicht, dass polen die pressefreiheit oder die rechte der frauen einschränke. "80 bis 90 prozent der medien berichten kritisch über die regierung", so morawiecki. die rechte der frauen würden von der regierung in warschau "überaus ernst genommen". was der polnische premier nicht erwähnte: auch in seinem eigenen land gibt es heftigen protest gegen die nationalpopulistische politik der regierungspartei pis. in den polnischen großstädten versammeln sich auch heute wieder tausende von menschen zu demonstrationen gegen die umstrittene justizreform. die höchste richterin des landes malgorzata gersdorf warf der regierung "politische säuberung" der gerichte vor. prominentes opfer ist gersdorf selbst. die 65-jährige wurde von der regierung heute in zwangspension geschickt. die richterin will das nicht hinnehmen und ging trotzdem an ihren arbeitsplatz. im europaparlament bekamen gersdorf und andere polnische pis-gegner am mittwoch rückendeckung. es ging hoch her im hohen haus der eu. manfred weber, chef der größten fraktion evp attackierte den polnischen premier frontal. "wieso ist das polnische fernsehen zum propagandasender der regierung geworden. und wieso werden richter entlassen aufgrund ihrer politischen meinung", fragte er. die regierungspartei pis startete die justizreform, kaum dass sie 2015 an die macht gekommen war. sie unterstellte die gerichte direkt dem justizminister und schickt unliebsame richter zwangsweise in den ruhestand. gegen polen läuft deshalb ein strafverfahren nach artikel sieben des eu-vertrags - das erste der eu-geschichte. die regierung in warschau zeigt sich davon völlig unbeeindruckt. sanktionen hat sie ja auch nicht zu befürchten. theoretisch könnte das verfahren für polen zwar mit einem entzug der stimmrechte im ministerrat enden, was faktisch einem rauswurf aus der eu gleichkommt. der dafür nötige einstimmige beschluss wird aber nie zustande kommen, da viele mittel- und osteuropäischen staaten dagegen stimmen wollen. die pis fühlt sich sicher in warschau. und in westeuropa wächst der zorn. viele eu-gründerstaaten, allen voran frankreich, hätten schlichtweg lust, den polen den europäischen geldhahn zuzudrehen. laut entwurf des neuen mittelfristigen eu-haushalts soll das land zwischen 2021 und 2027 rund 72 milliarden euro aus den eu-kohäsionsfonds für strukturschwache regionen bekommen - mehr als jeder andere eu-staat. polen selbst ist das noch zu wenig. doch andere eu-staaten könnten auf die idee kommen, die mittel zu kürzen. denn die pis provoziert immer wieder - auch an diesem mittwoch im europaparlament. ein abgeordneter von der pis wirft seinen kollegen aus deutschland, frankreich und anderen westeuropäischen staaten "selbstgefällige rechthaberische arroganz" vor. die eu-kommission habe seinem land "gar nichts zu sagen". der polnische premier meldet zweifel an, ob "die kommission ein ehrlicher makler ist". im dialog mit der kommission sei polen inzwischen an einem "punkt angekommen", an dem verständigung kaum noch möglich sei, so morawiecki brücken will die regierung in warschau wohl auch gar nicht bauen. sie sucht die konfrontation und hofft dabei auf immer mehr verbündete: ungarn, tschechien und die slowakei stehen bereits an der seite polens. nun haben auch in italien und in slowenien nationalpopulisten die wahlen gewonnen und in österreich wurde die eu-skeptische fpö junior-partner in der regierungskoalition. all diese regierungen legen auf die einhaltung demokratischer grundwerte, wie sie im eu-vertrag von lissabon festgeschrieben sind, wenig wert. die nationalpopulisten schicken sich an, die eu zu übernehmen. ob ihnen das gelingt, ist die große frage der europäischen zukunft. die wichtigsten neuigkeiten jeden morgen in ihrem posteingang. & 778 & very low & Low & Power & NA & NA & 2018-07-04 & 2018 & 3 & POL
Frame & v.low & National & 500-1000 & -1.0405052 & -1.0641830 & 1.2185583 & -1.5781059 & 1.4491718 & 12.0 & 1.6295860 & 1.9019065 & Payer & Domestic & Domestic & Domestic & Domestic|POL & Negative\\
\addlinespace
Germany & http://www.spiegel.de/politik/ausland/eu-kommissarin-vera-jourova-droht-polen-mit-geldentzug-a-1137262.html & 292 & SPIEGEL ONLINE & Private/Non-Public & Online and Offline & National & low = CP mentioned more times but NOT important part of story (mainly about others issues) & Political leverage & Balanced & EU + Other country & No myth & NA & NA & NA & NA & NA & NA & NA & NA & Germany & verstöße gegen grundwerte: brüsseler kommissarin droht polen mit geldentzug - spiegel online - politik & 2017-03-07 & kohäsionsfonds & in den niederlanden und frankreich wollen geert wilders und marine le pen demnächst wahlen gewinnen, in ungarn und polen sind rechtsnationale populisten bereits an der macht und schwächen den rechtsstaat. in polen hat die regierende partei recht und gerechtigkeit (pis) staatliche medien und geheimdienste gleichgeschaltet, missliebige beamte entlassen und das verfassungsgericht lahmgelegt. die eu-kommission aktivierte daraufhin zum ersten mal überhaupt den mechanismus zum schutz des rechtsstaats. doch die polnische regierung zeigte sich auch in ihrem jüngsten brief an die kommission uneinsichtig. damit ist das rechtsstaats-verfahren an seinem ende angelangt. der nächste schritt wäre die aktivierung von artikel 7 des eu-vertrags, was im extremfall zur aussetzung der stimmrechte polens führen könnte. dass es dazu kommt, ist aber unwahrscheinlich: notwendig wäre ein einstimmiger beschluss der eu-mitgliedstaaten, und ungarn hat sein veto bereits angekündigt. im interview mit dem spiegel schlägt eu-justizkommissarin vera jourová nun ein anderes mittel vor: sie fordert, die für polen und ungarn lebenswichtigen eu-gelder an die einhaltung von grundwerten zu koppeln - oder die förderung gleich ganz abzuschaffen. spiegel online: frau kommissarin, populisten scheinen zur größten inneren bedrohung der eu zu werden. in polen und ungarn sind sie bereits an der macht und greifen demokratische institutionen an. sind die regeln der eu zu schwach, um die einhaltung ihrer eigenen grundwerte zu sichern? jourová: das ist nicht nur eine frage der regeln. hier geht es auch um gesellschaftliche entwicklung, die ihren eigenen regeln folgt. aber wir dürfen nicht naiv sein im umgang mit denen, die prinzipien zum schutz der grundrechte missbrauchen. jeder, der in europa lebt, muss diese grundwerte akzeptieren. das schließt die rechtsstaatlichkeit, die gleichberechtigung von mann und frau und die nichtdiskriminierung von religiösen und anderen minderheiten ein. wer weder das gesetz noch unsere demokratische lebensweise respektieren will, sollte nicht in europa leben. spiegel online : die polnische regierung zeigt sich bisher allerdings wenig beeindruckt, der rechtsstaats-mechanismus ist weitgehend wirkungslos geblieben. zeigt sich hier, dass die kommission zahnlos ist? jourová: wir müssen mit den kompetenzen arbeiten, die uns gegeben wurden. mit der bewertung des letzten briefs der polnischen regierung sind unsere möglichkeiten ausgeschöpft, die angelegenheit in einem freundlichen dialog zu regeln. wir werden wohl die mitgliedstaaten bitten, sich in den dialog mit polen einzuschalten. dieser fall kann nicht ewig bei der kommission bleiben. er dauert inzwischen seit mehr als einem jahr an. spiegel online : sie sprechen über artikel 7 der eu-verträge - dessen aktivierung zur aussetzung der polnischen stimmrechte führen kann. das ist wenig realistisch, aber in einer rede haben sie kürzlich noch eine andere möglichkeit angedeutet. jourová: ja, ich habe die möglichkeit zur sprache gebracht, eu-fördermittel von der rechtsstaatlichkeit und der achtung der grundrechte abhängig zu machen. spiegel online : polen ist bei weitem der größte nettoempfänger in der eu. das land erhält fast ein viertel aller eu-gelder, sie sind für 2,3 prozent seines bruttoinlandsprodukts verantwortlich. ein ausbleiben dieser mittel wäre eine katastrophe für die polnische wirtschaft. jourová: deshalb ist das thema wichtig für polen. natürlich geht es hier nicht um die aktuelle haushaltsperiode, die im jahr 2020 endet. aber die vorbereitungen für die nächste periode beginnen schon in diesem jahr, und wir müssen über die bedingungen für eine weitere massive förderung reden - förderung, die auf solidarität basiert. ich habe fast 20 jahre im bereich der strukturfonds gearbeitet, und es gab immer bedingungen, aber nur im zusammenhang mit ökonomischen faktoren oder technischen details. wir müssen uns wieder auf die grundwerte besinnen und sie berücksichtigen. spiegel online : allerdings kann die kommission das nicht allein entscheiden. auch das europäische parlament und die mitgliedstaaten haben beim etat ein wörtchen mitzureden. jourová: natürlich. aber mit großbritannien verlässt einer der größten nettozahler die eu. wir müssen die mechanismen und kriterien für die weitere zahlung von fördermitteln neu bewerten. zwei faktoren werden dabei entscheidend sein. einer ist die grenze, die bestimmt, wann ein land für strukturförderung in frage kommt. bisher war dies der fall, wenn das pro-kopf-bip eines landes unter 75 prozent des eu-durchschnitts liegt. aber das muss nicht so bleiben. wir werden wieder bei null anfangen. spiegel online : die erhöhung der bip-grenze könnte die förderung für einige mitgliedstaaten massiv beeinflussen. jourová: ja, deshalb müssen wir das mit dem parlament und den mitgliedstaaten erörtern. spiegel online : was ist der zweite faktor? jourová: die geldsumme, die in zukunft für die eu-kohäsionspolitik zur verfügung stehen wird - falls es überhaupt noch eine kohäsionspolitik* geben wird. spiegel online : wollen sie damit sagen, dass der geldtransfer von reicheren zu ärmeren eu-ländern vollständig enden könnte? jourová: es könnte mindestens veränderungen geben. schauen sie sich das weißbuch an, das die kommission vergangene woche veröffentlicht hat. in einem der szenarien wurde die regionalpolitik als ein bereich definiert, der künftig den mitgliedstaaten überlassen werden könnte. oder vielleicht werden wir neu definieren, ob künftige mittel als subventionen oder darlehen ausgezahlt werden. aber derzeit ist das natürlich noch eine theoretische diskussion. spiegel online : ... die allerdings zu erheblicher aufregung in volkswirtschaften führen könnte, die massiv von den kohäsionsfonds der eu abhängen, etwa in polen oder ungarn. jourová: diese länder werden mit am tisch sitzen und haben ein mitspracherecht. aber sie werden verstehen müssen, dass wir die grundwerte der eu wahren. die menschen sind derzeit nervös und unsicher wegen der sicherheitslage. eine solche atmosphäre fördert angst und hass. sicherheit ist wichtig, aber wir dürfen nicht jenen alle macht geben, die sagen, sie wollten sicherheit schaffen. grundrechte sind heutzutage nicht selbstverständlich. wir müssen sie schützen. & 899 & low & Low & Power & NA & NA & 2017-03-07 & 2017 & 2 & POL
Frame & low-medium & National & 500-1000 & -1.0405052 & -1.0641830 & 1.2185583 & -1.5781059 & 1.4491718 & 12.0 & 1.6295860 & 1.9019065 & Payer & European & European & European & European|POL & Neutral\\
Germany & http://www.wiwo.de/politik/ausland/mehr-geld-fuer-sicherheit-eu-einigt-sich-auf-haushaltplan-fuer-2017-/14856004.html & 219 & Wirtschafts Woche & Private/Non-Public & Online and Offline & National & low = CP mentioned more times but NOT important part of story (mainly about others issues) & Institutional bargaining over funding & Factual & EU + National & No myth & NA & NA & NA & NA & NA & NA & NA & NA & Germany & eu einigt sich auf haushaltplan für 2017 & 2016-11-17 & regionalpolitik & die flüchtlingskrise verändert nachhaltig die ausgabenpolitik der eu. im gemeinschaftshaushalt für 2017 wird für sicherheit noch einmal mehr geld eingeplant - obwohl die gesamtausgaben sinken sollen. brüsselaus dem gemeinschaftshaushalt der eu soll im kommenden jahr weniger geld ausgezahlt werden als in diesem. eine am donnerstag in brüssel getroffene einigung sieht vor, lediglich für projekte in bereichen wie grenzschutz, migration sowie wachstum und beschäftigung deutlich höhere ausgaben zuzulassen. an anderen stellen wird der finanzbedarf für 2017 hingegen geringer gesehen als in diesem jahr. so sollen etliche milliarden euro weniger in die eu-regionalpolitik fließen. anzeige insgesamt sind im haushaltsplan für 2017 ausgaben in höhe von 134,5 milliarden euro vorgesehen. dies entspricht einem minus von 6,5 prozent im vergleich zur planung für das vorjahr. die mittel für den bereich, der den grenzschutz und die asylpolitik umfasst, werden um 25 prozent auf 3,8 milliarden euro aufgestockt, die für die wachstums- und beschäftigungsförderung um 11 prozent auf 19,3 milliarden euro. der verhandlungsführer der bundesregierung zeigte sich mit dem ergebnis zufrieden. "wir werden die ausgaben zur bewältigung der migrationskrise auf hohem niveau fortführen", kommentierte finanzstaatssekretär jens spahn (cdu) mit blick auf zusätzliche gelder für den schutz der europäischen außengrenzen. weitere haushaltsschwerpunkte seien die eu-initiative zur bekämpfung der jugendarbeitslosigkeit und die forschungsförderung. die gesamtsumme der auszahlungen kann spahns angaben zufolge vor allem deswegen gekürzt werden, weil für die regionalpolitik eingeplante gelder nicht wie vorgesehen abgerufen werden. sie stehen zum beispiel zur förderung der wirtschaft in mittel- und osteuropa zur verfügung. an den haushaltsverhandlungen war neben regierungsvertretern aus den 28 eu-staaten auch das europaparlament beteiligt. letzteres hatte zuletzt auszahlungen in höhe von rund 138 milliarden euro gefordert, während die eu-regierungen eigentlich nur 133,8 milliarden euro geben wollten. die verhandlungen seien hart, aber in konstruktiver atmosphäre geführt worden, kommentierte der spd-europaabgeordnete und parlamentsberichterstatter jens geier am donnerstag. am ende habe man die richtigen prioritäten gesetzt. & 315 & low & Low & Power & NA & NA & 2016-11-17 & 2016 & 2 & POL
Frame & low-medium & National & <500 & -1.0405052 & -1.0641830 & 1.2185583 & -1.5781059 & 1.4491718 & 12.0 & 1.6295860 & 1.9019065 & Payer & Domestic & European & Mixed & Domestic|POL & Neutral\\
Germany & https://www.welt.de/newsticker/news1/article164198259/London-empoert-nach-Bericht-ueber-kraeftig-erhoehte-EU-Austrittsrechnung.html & 290 & DIE WELT & Private/Non-Public & Online and Offline & National & very low = CP mentioned once & Institutional bargaining over funding & Negative & EU + Other country & No myth & NA & NA & NA & NA & NA & NA & NA & NA & Germany & brexit: london empört nach bericht über kräftig erhöhte eu-austrittsrechnung - welt & 2017-05-03 & kohäsionsfonds & anzeige die britische regierung hat empört auf einen bericht zu saftigen geldforderungen der eu beim austritt aus der europäischen union reagiert. "wir werden nicht 100 milliarden zahlen", sagte brexit-minister david davis am mittwoch im britischen sender itv. großbritannien werde lediglich seine "internationalen verpflichtungen" erfüllen. london gehe in die austrittsverhandlungen nicht als "bittsteller". großbritannien will ende märz 2019 aus der eu austreten. die eu pocht darauf, dass london dabei seine finanziellen verpflichtungen aus der mitgliedschaft vollständig erfüllt. bisherige schätzungen aus brüssel bezifferten den betrag auf zwischen 40 und 60 milliarden euro. eingerechnet wurden dabei etwa zugesagte zahlungen für eu-kohäsionsfonds zur angleichung der lebensverhältnisse sowie pensionszahlungen für eu-beamte. die "financial times" berichtete am mittwoch, frankreich und polen pochten nun darauf, auch agrarausgaben bis zum jahr 2020 einzurechnen, wenn der aktuelle, siebenjährige finanzrahmen der eu endet. deutschland wolle london wiederum nicht erlauben, die austrittsrechnung zu drücken, indem der britische anteil an eu-gebäuden und anderen vermögenswerten abgezogen werde. all dies könnte die brexit-rechnung nach einschätzung der zeitung auf bis zu 100 milliarden euro erhöhen. anzeige davis wies auch einen bericht der "times" zurück, wonach die eu-kommission premierministerin theresa may daran hindern wolle, mit anderen eu-staats- und regierungschefs zu verhandeln. "es gibt bei verhandlungen zwei seiten und die andere seite wird nicht entscheiden, wer was macht", sagte er der bbc. may war nach presseberichten über ein unerfreulich verlaufenes abendessen mit eu-kommissionspräsident jean-claude juncker unter druck geraten. die premierministerin, die vor beginn der brexit-verhandlungen überraschend vorgezogene neuwahlen angesetzt hatte, sagte daraufhin am dienstag, sie werde in den verhandlungen "eine verdammt schwierige frau" sein. die "frankfurter allgemeine sonntagszeitung" hatte unter dem titel "das desaströse brexit-dinner" berichtet, juncker habe nach dem treffen gesagt, er sei mit blick auf den ausgang der brexit-verhandlungen nun "zehnmal skeptischer als zuvor". beim brexit-sondergipfel am samstag sagte der kommissionschef öffentlich: "ich habe manchmal den eindruck, dass unsere britischen freunde - nicht alle - die technischen schwierigkeiten, vor denen wir stehen, unterschätzen." & 329 & very low & Low & Power & NA & NA & 2017-05-03 & 2017 & 2 & POL
Frame & v.low & National & <500 & -1.0405052 & -1.0641830 & 1.2185583 & -1.5781059 & 1.4491718 & 12.0 & 1.6295860 & 1.9019065 & Payer & European & European & European & European|POL & Negative\\
Germany & http://www.wiwo.de/politik/ausland/eu-streitet-ueber-fluechtlingsquote-es-hilft-nichts-wenn-die-bockigen-noch-bockiger-werden/12457932.html & 265 & Wirtschafts Woche & Private/Non-Public & Online and Offline & National & very low = CP mentioned once & Political leverage & Negative & EU + Other country & No myth & NA & NA & NA & NA & NA & NA & NA & NA & Germany & eu streitet über flüchtlingsquote: "es hilft nichts, wenn die bockigen noch bockiger werden" & 2015-10-15 & kohäsionsfonds & düsseldorfes soll ein kurzes brüsseler gipfeltreffen werden. obwohl zwei tage eingeplant sind, möchte ratspräsident tusk am abend schon fertig werden. kein wunder: die wichtigsten themen wurden bereits von den innen- und außenministern zu aller zufriedenheit verhandelt: außengrenzen sichern, flüchtlinge ohne triftigen asylgrund abschieben, hotspots wie griechenland und italien helfen, drittstaaten wie der türkei bei der flüchtlingsaufnahme unterstützen. dabei soll es bleiben. die staatschefs wollen so erstmals langfristige entscheidungen in der flüchtlingskrise treffen. dennoch zeigen sich europaparlamentarier enttäuscht: "dabei bleibt die frage ungelöst, wie wir flüchtlinge in der eu gerecht verteilen, wie wir sie integrieren und wie wir in den herkunftsstaaten die situation tatsächlich verbessern", sagt die spd-europaabgeordnete birgit sippel. bislang haben sich die eu-mitgliedsstaaten auf die verteilung von 160.000 flüchtlingen geeinigt. die tatsächliche verteilung hat vergangen freitag erst begonnen und wird noch monate dauern. dabei sind in diesem jahr nach angaben des flüchtlingswerkes der vereinten nationen allein über das mittelmeer knapp 600.000 menschen nach europa gekommen. und der strom reißt nicht ab. die in deutschland entflammte diskussion über transitzonen an den grenzen entzweit nicht nur die cdu, sondern das ganze land. ähnlich heftig wird in vielen europäischen ländern diskutiert. es ist höchste zeit, dass sich die eu auf eine lösung einigt. dafür spricht sich in der union europa auch eine große mehrheit der bevölkerung aus, wie eine neue umfrage des eurobarometers zeigt. zwei drittel der europäer meint, dass die eu, und nicht etwa nationale regierungen, die entscheidungen in der flüchtlingskrise fällen sollten. mehr als 28.000 menschen wurden für das eurobarometer befragt. dabei bestätigte sich erneut eine starke meinungsverschiedenheit zwischen ost- und westeuropäischen staaten. dreiviertel aller europäer sprachen sich für eine verteilung der flüchtlinge auf die eu-mitgliedsstaaten aus. in deutschland waren es sogar 97 prozent. in tschechien und der slowakei hingegen lehnten dies mehr als zwei drittel aller befragten ab. damit schließen sie sich ihren staatschefs an. bereits ende september, als sich die eu-mitgliedsstaaten mehrheitlich auf die verteilung von 120.000 flüchtlingen einigten, wiesen die osteuropäischen staaten ungarn, polen, tschechien und die slowakei dies ab. dass die verteilung der flüchtlinge nicht auf der agenda des brüsseler gipfeltreffens steht, hat offenbar einen ganz einfachen grund: noch immer ist es nicht gelungen, polen, slowakei, tschechien und ungarn von einem europäischen verteilungsschlüssel zu überzeugen. "in einer solch festgefahrenen situation hilft manchmal eine kurze abkühlphase", sagt der cdu/csu-europavorsitzende herbert reul gegenüber dem handelsblatt. es helfe nichts, wenn die bockigen nur noch bockiger werden. dennoch will er die überzeugungsarbeit nicht allein der zeit überlassen. "natürlich müssen wir mehr druck auf die osteuropäer aufbauen", sagt reul. dieser druck könne moralisch wie auch finanziell aufgebaut werden kann. ersteres zeige jetzt schon wirkung, etwa im falle polens, das offenbar stärker den eu-kurs einlenkt. "es kommt der tag, an dem auch die osteuropäer etwas möchten" zu finanziellen konsequenzen hatte sich bereits ende september der deutsche wirtschaftsminister sigmar gabriel geäußert. es könne nicht sein, dass alle nur mitmachten, wenn sie geld bekommen, dann aber kneifen, wenn verantwortung zu tragen ist. "klar ist auch, wenn wir uns nicht einigen, dann ist die mittelfristige finanzplanung europas schall und rauch", erklärte der vizekanzler. der finanzrahmen legt die budgetrichtlinien des jährlichen eu-haushaltes fest. finanzieller druck ließe sich ausüben, indem die europäische strukturförderung gekürzt würde. vor allem die verweigerer der verteilung träfe dies hart. alle vier staaten - polen, slowakei, tschechien und ungarn - gehören zu den zehn größten empfängern des sogenannten kohäsionsfonds. die slowakei erhält in den jahren 2014 bis 2020 pro kopf 2681 euro und liegt damit auf rang zwei der empfänger hinter estland. die anderen drei bekommen ebenfalls mehr als 2000 euro pro kopf. zum vergleich: das in vielen teilen ebenfalls strukturschwache eu-mitglied rumänien erhält 1157 euro pro einwohner. doch die sache hat gleich zwei haken: sie wäre langwierig und kaum umzusetzen. "als erstes müsste die eu dafür den mittelfristigen finanzrahmen öffnen, danach neu verhandeln und schließlich einstimmig beschließen", erklärt jens geier, der für die spd im europäischen haushaltsausschuss sitzt. das heißt: die osteuropäer müssten den mittelkürzungen selbst zustimmen. obwohl der eu-finanzrahmen anfang 2016 geprüft und justiert wird, scheint diese möglichkeit deshalb unwahrscheinlich. gleichwohl dürfte die anstehende überprüfung des finanzrahmes dazu beitragen, druck auf osteuropa auszuüben. den ländern dürfte klar sein, dass umschichtungen im eu-haushalt zu ihren lasten gehen, weil das zusätzliche budget für die flüchtlingskrise ja von anderen haushaltsposten umgeschichtet werden müsse. "irgendwann kommt der tag, an dem auch die slowaken, tschechen, ungarn oder polen etwas möchten. dann werden sich sicher viele parlamentarier an deren verhalten in der flüchtlingskrise erinnern", sagt der spd-abgeordnete geier. seine anspielung ist klar: wer keine solidarität zeigt, darf gefälligst auch selbst keine erwarten. der frust sitzt tief in europa. denn gerade in einer zeit, in der schnelles handeln gefragt ist, wird genau das blockiert. ein mittel, um die osteuropäischen staaten doch noch zum umdenken zu bewegen, könnte die finanzielle unterstützung der europäischen union sein. bereits im august hatte sie 2,4 milliarden euro bereitgestellt, um mitgliedsstaaten bei der aufnahme der flüchtlingsströme zu helfen. weitere 1,7 milliarden sollen 2015 und 2016 aus flexibilitätsfonds und durch die umschichtung nicht verwendeter budgets fließen. in polen, tschechien, der slowakei und ungarn haben die finanziellen anreize offenbar noch kein umdenken bewirkt. so wird auf dem gipfel am abend das reizthema verteilungsschlüssel nur eine untergeordnete rolle spielen. europaparlamentarier reul meint jedoch, dass die verteilung sehr bald wieder diskutiert werden muss. "ich bin sicher, dass der verteilungsschlüssel kommt", sagt der cdu-politiker. es gebe eben nur eine gemeinsame lösung für ganz europa - da habe die mehrheit der befragten des eurobarometers schon recht. das sei, wie auch die gemeinsame kontrolle der außengrenzen, eine zentrale frage der eu. "und wenn wir uns bei zentralen fragen nicht mehr einigen können, wäre die union nicht mehr handlungsfähig", sagt reul. & 955 & very low & Low & Power & NA & NA & 2015-10-15 & 2015 & 1 & POL
Frame & v.low & National & 500-1000 & -1.0405052 & -1.0641830 & 1.2185583 & -1.5781059 & 1.4491718 & 12.0 & 1.6295860 & 1.9019065 & Payer & European & European & European & European|POL & Negative\\
Germany & http://www.n-tv.de/politik/Ost-Laenderchefs-schreiben-Brandbrief-article20110283.html & 239 & n-tv.de & Private/Non-Public & Online and Offline & National & very low = CP mentioned once & Economic development & Positive & Subnational & No myth & NA & NA & NA & NA & NA & NA & NA & NA & Germany & forderungen an merkel: ost-länderchefs schreiben brandbrief & 2017-10-31 & kohäsionspolitik & damit die interessen der ostdeutschen länder bei den jamaika-sondierungen nicht zu kurz kommen, schreiben die länderchefs einen brief ans kanzleramt. darin wenden sie sich gegen eine ende der strukturförderung und einen ausstieg aus der braunkohle. sachsens scheidender ministerpräsident stanislaw tillich hat bundeskanzlerin angela merkel aufgefordert, bei der regierungsbildung ostdeutsche interessen in den fokus zu nehmen. dazu schickte der cdu-politiker auch im namen seiner ostdeutschen amtskollegen einen brief an die kanzlerin und cdu-chefin, wie die staatskanzlei mitteilte. die ostdeutschen länder wiesen eine "nahezu flächendeckende strukturschwäche" auf, schrieb tillich, der derzeit den vorsitz der ministerpräsidentenkonferenz ost innehat. diese schwäche müsse überwunden werden. dafür sei ostdeutschland weiter auf finanzielle förderung angewiesen - sowohl aus deutschen töpfen als auch im rahmen der eu-kohäsionspolitik. "ein abruptes ende der strukturförderung in ostdeutschland würde die erfolge der vergangenheit gefährden", heißt es in dem schreiben. auch müsse verhindert werden, dass ostdeutschland in eine ungünstige "sandwichposition" gerate - zwischen den hoch entwickelten regionen in westdeutschland und den sehr stark von der eu geförderten gebieten in osteuropa. daneben sprachen sich die landeschefs in dem schreiben gegen einen schnellen braunkohle-ausstieg aus. an der kohleverstromung hingen in ostdeutschland zehntausende arbeitsplätze. ein abruptes ende verbiete sich "schon aus respekt vor der lebensleistung der beschäftigten", hieß es. erst wenn es für sie nachhaltige zukunftsperspektiven gebe, dürfe das ende der kohlenutzung beschlossen werden. damit der anschluss an den wirtschaftsstarken westen gelinge, müsse der osten zudem besser an das bahnnetz und den luftverkehr angeschlossen werden, forderten tillich und seine kollegen. mehr bundesbehörden müssten im osten angesiedelt werden. außerdem müsse die künftige regierung eine flächendeckende versorgung mit schnellem internet und mobilfunk sicherstellen. auch die hausarztversorgung müsse ausgebaut werden. in berlin laufen derzeit sondierungsgespräche zwischen vertretern von cdu, csu, fdp und grünen über eine mögliche künftige jamaika-koalition. bei den reizthemen klima und flüchtlinge hatte es zuletzt streit gegeben. fortschritte gelangen zuletzt bei den themen arbeit und rente, pflege, sicherheit sowie bildung und digitales. & 319 & very low & Low & Socio-Economic & NA & NA & 2017-10-31 & 2017 & 2 & ECO
Frame & v.low & National & <500 & -1.0405052 & -1.0641830 & 1.2185583 & -1.5781059 & 1.4491718 & 12.0 & 1.6295860 & 1.9019065 & Payer & Domestic & Domestic & Domestic & Domestic|ECO & Positive\\
\addlinespace
Germany & https://rp-online.de/nrw/staedte/moers/gaerten-der-vielfalt-fuer-moers-meerbeck\_aid-35912367 & 226 & RP Online & Private/Non-Public & Online and Offline & Regional/Local & very low = CP mentioned once & Cultural heritage & Positive & Subnational & No myth & NA & NA & NA & NA & NA & NA & NA & NA & Germany & "gärten der vielfalt" für moers-meerbeck & 2019-01-28 & europäischer fonds für regionale entwicklung & moers-meerbeck der ursprünglich vorgesehene standort an der glückaufstraße - in unmittelbarer nähe zum chemiewerk - gilt mittlerweile als nicht geeignet. meerbeck soll grüner werden - noch grüner. im rahmen des integrierten handlungskonzeptes "neu\_meerbeck", das der stadtrat 2015 beschlossen hat, ist unter dem titel "gärten der vielfalt" vorgesehen, ein gartenprojekt in kooperation zwischen bewohnern, ehrenamtlichen, beschäftigungsprojekten, schulen und anderen umzusetzen. geeignete flächen sollen kultiviert und zu orten sozialer gemeinschaft werden. der ausschuss für stadtentwicklung, planen und umwelt hat die verwaltung jetzt beauftragt, im rahmen der städtebauförderung "soziale stadt" des bundes und des landes nordrhein-westfalen sowie des efre (europäischer fonds für regionale entwicklung) zu stellen. sicher ist: der ursprünglich vorgesehene standort an der glückaufstraße - in unmittelbarer nähe zum chemiewerk - gilt mittlerweile als nicht geeignet. der fokus von politik und verwaltung liegt nunmehr auf einer fläche westlich der st.-marien-grundschule, die einen teilbereich der geplanten öffentlichen grünfläche an der kirschenallee/römerstraße bildet. vorteilhaft sei dort insbesondere die zentrale lage mit anbindung an mehrere schulen, das jugendzentrum und weitere einrichtungen, heißt es. fest steht auch: die frühzeitige einbindung der künftigen nutzer soll ein hohes maß an identifikation bringen. eine betreuung, möglichst durch "kümmerer" aus dem quartier, soll sichergestellt werden. für das "gärten der vielfalt"-projekt ist eine förderantragstellung vorgesehen, die fördertechnisch im zusammenhang mit dem volkspark "neu\_meerbeck" steht. die zu erwartenden kosten, sagt die verwaltung, liegen bei voraussichtlich rund 30.000 euro. unter berücksichtigung der angestrebten neunzigprozentigen förderung bleibt ein städtischer eigenanteil von zehn prozent in höhe von 3000 euro. & 246 & very low & Low & Socio-Economic & NA & NA & 2019-01-28 & 2019 & 3 & ECO
Frame & v.low & Regional & <500 & -1.0405052 & -1.0641830 & 1.2185583 & -1.5781059 & 1.4491718 & 12.0 & 1.6295860 & 1.9019065 & Payer & Domestic & Domestic & Domestic & Domestic|ECO & Positive\\
Germany & http://www.spiegel.de/politik/ausland/viktor-orban-reagiert-auf-martin-schulz-ungarn-verdient-mehr-respekt-a-1186666.html & 227 & SPIEGEL ONLINE & Private/Non-Public & Online and Offline & National & very low = CP mentioned once & Political leverage & Balanced & Other country & No myth & NA & NA & NA & NA & NA & NA & NA & NA & Germany & orbán reagiert auf schulz: "wir verdienen mehr respekt" - spiegel online - politik & 2018-01-08 & kohäsionsfonds & der ungarische ministerpräsident viktor orbán hat sich von spd-chef martin schulz "mehr respekt" für sein land erbeten. in anspielung auf schulz' früheres amt als präsident des europäischen parlaments sagte orbán der "bild"-zeitung: "was gut und nett in brüssel war - wo es keine offensichtlichen konsequenzen gab - ist eine andere geschichte, als in deutschland parteichef zu sein und mit anderen ländern zu kommunizieren. wir finden, wir verdienen mehr respekt." orbán war am freitag ehrengast der csu bei der winterklausur im oberbayerischen kloster seeon. csu-landesgruppenchef alexander dobrindt und csu-chef horst seehofer berichteten nach dem gespräch mit "unserem freund viktor orbán" von einem "ausgesprochen erfolgreichen besuch" und einem "sehr ehrlichen und offenen" austausch. martin schulz hatte seehofer vor dem treffen aufgefordert, dem rechtsnationalen ungarischen regierungschef die grenzen aufzuzeigen. vor allem in der flüchtlingspolitik verfolge orbán eine "gefährliche logik", hatte schulz gesagt. "ich erwarte, dass herr seehofer ihm bei diesem thema und auch bei den themen presse- und meinungsfreiheit ganz klare grenzen aufzeigt." "ihr wolltet die migranten, wir nicht!" in dem zeitungsinterview verwahrte sich orbán nun gegen den vorwurf, ungarn nehme geld von der eu, weigere sich aber, flüchtlinge aufzunehmen. der sogenannte kohäsionsfonds, der der ungarischen wirtschaft zugutekomme, sei kein geschenk. "er ist ein fairer ausgleich, da wir unseren markt dem freien wettbewerb geöffnet haben. das hat absolut nichts mit der flüchtlingsfrage zu tun." orbán bekräftigte, dass ungarn auch künftig keine flüchtlinge aufnehmen werde. "wir glauben, dass eine hohe zahl an muslimen notwendigerweise zu parallelgesellschaften führt", sagte er. "so etwas möchten wir nicht. und wir möchten uns nichts aufzwängen lassen." die zeitung zitiert orbán zudem mit den worten. "ihr wolltet die migranten, wir nicht!" die ungarische regierung verweigert die aufnahme von flüchtlingen nach einem von der eu vorgeschlagenen schlüssel und steht unter anderem deshalb in der kritik. bei der winterklausur der csu sagte orbán, die flüchtlingskrise sei zu einer "demokratieproblematik" für europa geworden. & 312 & very low & Low & Power & NA & NA & 2018-01-08 & 2018 & 3 & POL
Frame & v.low & National & <500 & -1.0405052 & -1.0641830 & 1.2185583 & -1.5781059 & 1.4491718 & 12.0 & 1.6295860 & 1.9019065 & Payer & European & European & European & European|POL & Neutral\\
Germany & https://www.zeit.de/2015/28/grexit-griechenland-staatspleite & 240 & ZEIT ONLINE & Private/Non-Public & Online and Offline & National & very low = CP mentioned once & Political leverage & Negative & Other country & No myth & NA & NA & NA & NA & NA & NA & NA & NA & Germany & europa: vergesst den grexit & 2015-07-23 & kohäsionsfonds & in der griechenlandkrise offenbart sich ein altes strategisches prinzip: schwäche ist macht. and the winner is ... griechenland. auf den ersten blick ist dieses fazit der bestimmt nicht letzten krise absurd. athen ist doch schon pleite, seitdem es die fälligen 1,55 milliarden dollar nicht an den weltwährungsfonds zurückzahlen konnte. die banken bleiben zu, nur stockend rinnt das bargeld aus den automaten. und europas mächtige - angela merkel mit françois hollande im schlepptau - geben sich eiskalt, nachdem das griechenvolk der eu am sonntag den hochgereckten mittelfinger gezeigt hat. erst einmal keine weiteren gespräche, heißt es. "grexit", "kollaps", "notlage" wallen durch die hiesige presse, allen voran die bild-zeitung mit der parole "kein drittes hilfsprogramm für athen!", der berliner sagt hier: "det wolln wa mal sehn." strategen reden nüchterner von der "stärke der schwachen". kinder beherrschen diese kunst perfekt. "hau mich doch", trotzen sie im bus der mutter. und die traut sich nicht, weil sie sich selber noch schärfer bestrafen würde, indem sie protest und abscheu auf sich zieht. auch im miteinander der staaten funktioniert das prinzip "schwäche ist macht". der bündnisführer, der mindere mitglieder züchtigen will, riskiert die selbstbestrafung, weil er die gesamte allianz schwächen und sich selber schädigen würde. beispiel: gerade mal ein jahr hat obama sein waffenembargo gegen das ägyptische putschregime durchgehalten, dann lieferte er wieder f-16-jets an kairo. aufs feinste beherrscht der griechische premier alexis tsipras diese kunst. seit seiner wahl trotzt und trumpft er, lässt er die euro-gruppe und die "eiserne kanzlerin" (bild) zappeln und greinen. "schmeißt mich doch raus, schneidet mir doch die luft ab", signalisiert er stumm, "ihr werdet schon sehen, was ihr davon habt." wem wäre denn mit grexit-plus-default geholfen? den griechen nicht, der eu noch weniger. wie argentinien, das im jahr 2002 aus der quasi-währungsunion mit dem dollar ausbrach und den peso auf ein viertel abwertete, würde das höchst verschuldete land den zugang zu den kapitalmärkten verlieren und noch tiefer in die verarmung stürzen. soll es doch? gemäß dem vertrauten spruch aus voltaires candide, wonach es gelegentlich nützlich sei, jemanden zu exekutieren, pour encourager les autres - um die anderen zu ermuntern, auf dem pfad der tugend zu bleiben. in diesem fall jene länder, die sich wie portugal und spanien am riemen gerissen und ihre finanzen in ordnung gebracht haben. grexit-fans vergessen, dass hellas auch dann noch immer mitglied der eu und somit wohlfahrtsempfänger bliebe. zwar gäbe es keine stütze aus den euro-rettungsfonds mehr, auch keine ezb-liquiditätshilfe für die todkranken banken. aber eu-milliarden würden weiter fließen. da wären der struktur- und kohäsionsfonds, dazu kämen milliarden aus dem zahlungsbilanz-topf, der für not leidende eu-staaten bereitsteht. ein grexit und eine staatspleite würden aber die lage verschlimmern, weil die märkte keinen cent mehr gäben. der nutzen einer abwertung wäre demgegenüber begrenzt. ein beispiel: 40 prozent der griechischen exporte sind erdölprodukte. bloß muss das land jedes einzelne fass importieren. der kräftig verteuerte einkauf würde den exportvorteil zunichtemachen. erhebend ist solche analyse nicht, legt sie doch nahe, dass europa um fast jeden preis ein land retten müsse, das es nicht verdient, weil es seit jahren auf kosten der anderen lebt, selbst auf die der baltischen kleinstaaten. die griechen für ihre hybris zu bestrafen wäre gewiss seelenbalsam für merkel und kollegen, die von tsipras und seinem ex-finanzminister monatelang belogen und ausmanövriert worden sind. doch sind gerechtfertigte gefühle nicht "zielführend", wie es auf neudeutsch heißt. warum, das hat der weltwährungsfonds (iwf) am 26. juni unter dem titel preliminary draft debt sustainability analysis erklärt. alltagssprachlich ausgedrückt, lautete die einschätzung: "athen kann nicht zahlen und wird es jahrzehntelang nicht können." dabei hatte die kanzlerin den iwf unbedingt in den verhandlungen mit athen dabeihaben wollen: neben eu-kommission und ezb sollte der washingtoner iwf ein gegengewicht zu den gnädigeren europäern bilden. zwar gibt der bericht des währungsfonds der deutschen kanzlerin en passant recht: ja, die griechen müssen sparen und reformieren; hätten sie ihre pflicht brav erfüllt, wären sie heute auf einem guten weg. aber dann breitet der iwf auf 23 seiten die rüde wahrheit aus: erstens braucht athen in den nächsten drei jahren 60 milliarden euro an finanzhilfe. zweitens müssten seine schulden "umstrukturiert", sprich: die laufzeiten von 20 auf 40 jahre verlängert werden. drittens: bessert sich der patient nicht, steht ein "haircut" (ein schuldenschnitt) von 53 milliarden euro an. für tsipras klingt das wie weihnachten im juli. es wird nicht genauso kommen, wie der iwf es prophezeit. aber die stärke der schwachen wird ihre wirkung entfalten. denn wenn erst das bankensystem und dann die wirtschaft kollabiert, werden merkel und kollegen die verantwortung dafür tragen wollen, als meuchler griechenlands dazustehen? europa wird viel, athen wenig geben müssen. eine so kaltäugige analyse muss allerdings auch die horrorseite der medaille beleuchten. gewinnen die tsipristen, wird euro-land verlieren. warum sollten nun die reformwilligen in iberien, die reformmüden franzosen und italiener weiter opfer bringen, wenn die griechen mit ihrer selbstsüchtigen strategie davonkommen? seit 45 jahren sinkt das eu-durchschnittswachstum sachte, sachte um einen halben prozentpunkt pro dekade. der eu-anteil am globalen wirtschaftsprodukt ist um neun punkte gefallen. das müssen merkel und die troika wissen, wenn sie das handgemenge mit athen wieder aufnehmen. die alten griechen haben europa vor den persern gerettet. rettet europa die neuen griechen nicht vor sich selbst, indem es mit der unumgänglichen hilfe disziplin und erneuerung durchsetzt, wird der leise niedergang voranschleichen. die winzige griechische wirtschaft zu alimentieren könnte europa sich leisten, nicht aber die folgekosten für alle anderen im verbund der 28 mitgliedsstaaten. & 912 & very low & Low & Power & NA & NA & 2015-07-23 & 2015 & 1 & POL
Frame & v.low & National & 500-1000 & -1.0405052 & -1.0641830 & 1.2185583 & -1.5781059 & 1.4491718 & 12.0 & 1.6295860 & 1.9019065 & Payer & European & European & European & European|POL & Negative\\
Germany & http://www.deutschlandfunk.de/diskussion-bei-eu-ministertreffen-weniger-geld-fuer.1773.de.html?dram:article\_id=400654 & 238 & Deutschlandfunk & Public & Online and Offline & National & high = CP is most important issue in story (can also cover other issues) & Political leverage & Balanced & EU + Other country & No myth & NA & NA & NA & NA & NA & NA & NA & NA & Germany & diskussion bei eu-ministertreffen - weniger geld für rechtsstaatsünder? & 2017-11-14 & kohäsionsfonds & urteil zur brennelemente-steuer schäuble will milliarden aus laufendem haushalt zurückzahlen die idee gehört zu den brisantesten, die derzeit auf eu-ebene diskutiert werden: wer die demokratischen spielregeln nicht einhält, könnte das künftig dort zu spüren bekommen, wo es besonders wehtut - im portemonnaie nämlich. heißt konkret: auf deutsches betreiben hin beginnt jetzt gerade eine debatte darüber, ob die eu in zukunft fördergelder an die einhaltung rechtsstaatlicher regeln knüpfen könnte. sowohl ungarn als auch polen liefern sich mit der eu-kommission eine art dauerfehde über von brüssel beklagte demokratie-verstöße. doch alle instrumente, mit denen die eu das problem anzugehen sucht, haben sich bislang als vergleichsweise stumpf erwiesen. nun also wird über eine härtere gangart nachgedacht. beim jüngsten eu-gipfel in brüssel fand kanzlerin merkel zwar, wie sie zugab, keine gelegenheit, das thema rechtsstaat anzuschneiden, was eigentlich geplant war. kündigte aber an: "wir werden darauf zurückkommen." noch befindet sich die diskussion in einem frühen stadium. doch die bundesregierung steht mit ihrem vorstoß, wie eu-diplomaten bestätigen, keineswegs alleine da. da im nächsten eu-haushalt ab 2021, und um den geht es, die britischen milliarden wegfallen, zeigen sich gerade die größten einzahler sparideen gegenüber durchaus aufgeschlossen. der vorschlag birgt jedenfalls jede menge sprengstoff. nicht nur für den eu-zusammenhalt. auch müssten noch reihenweise juristische fragen geklärt werden. sind die fördergelder doch einst ersonnen worden, um soziale und wirtschaftliche ungleichheit innerhalb der eu abzumildern. parallel zur debatte, die auf der ebene der einzelstaaten geführt wird, macht sich auch das eu-parlament gedanken darüber, wie man den druck insbesondere auf polen aufrechterhalten kann. "die eu-kommission hat sich viel mühe gemacht, den rechtsstaatsbruch und die missachtung europäischer werte in polen rückgängig zu machen. aber offensichtlich ist die regierung in warschau nicht willens, eine lösung für diese verstöße zu finden", beklagt der liberale eu-abgeordnete guy verhofstadt. der allerdings zu jenen gehört, die gleichzeitig warnen, dass man beim kürzen von geldern nicht die falschen treffen dürfe, sprich: die bevölkerung. im parlament gibt es denn derzeit auch eher bestrebungen, einen anderen sanktionsweg einzuschlagen. doch von einer entscheidung, was das ansinnen angeht, geld als hebel zu nutzen, ist man ohnehin noch weit entfernt: zunächst geht es den befürworten darum, die diskussion voranzutreiben. und damit auch das drohpotential gegenüber rechtsstaatssündern aufrecht zu erhalten. dass kürzungen bei fördergeldern, den sogenannten kohäsionsfonds, gerade polen und ungarn empfindlich treffen dürften, liegt auf der hand: warschau ist der größte empfänger von eu-hilfen, auch ungarns wirtschaftsleistung wird nicht unerheblich mit den umverteilten geldern angekurbelt. & 412 & high & High & Power & NA & NA & 2017-11-14 & 2017 & 2 & POL
Frame & high-very high & National & <500 & -1.0405052 & -1.0641830 & 1.2185583 & -1.5781059 & 1.4491718 & 12.0 & 1.6295860 & 1.9019065 & Payer & European & European & European & European|POL & Neutral\\
Germany & https://www.abendblatt.de/politik/deutschland/article215997411/Umweltministerin-Schulze-will-mehr-EU-Geld-fuer-Kohleregionen.html & 281 & Hamburger Abendblatt & Private/Non-Public & Online and Offline & Regional/Local & very low = CP mentioned once & Economic development & Positive & EU + National & No myth & NA & NA & NA & NA & NA & NA & NA & NA & Germany & umweltministerin schulze will mehr eu-geld für kohleregionen & 2018-12-12 & strukturfonds & kattowitz. bundesumweltministerin svenja schulze fordert eine größere finanzielle unterstützung der eu für den strukturwandel in mehr als 40 kohleregionen europas. "ich würde es sehr begrüßen, wenn wir im nächsten eu-haushalt auch mehr mittel für die vom strukturwandel betroffenen regionen bereitstellen", sagte die spd-politikerin am mittwoch auf der un-klimakonferenz im polnischen kattowitz (katowice). konkrete zahlen nannte sie auf nachfrage nicht. es gebe zwar die eu-strukturfonds für die wirtschaftliche entwicklung, sagte schulze. sie sei aber dafür, unter dem titel "just transition" (gerechter übergang) die regionen zu betrachten, "die jetzt vom strukturwandel nochmal härter betroffen sein werden". sie gehe davon aus, dass auch deutschland dann profitieren würde. die finanzierung des strukturwandels ist einer der knackpunkte im streit um den kohleausstieg in deutschland. die vor allem betroffenen bundesländer nordrhein-westfalen, sachsen, brandenburg und sachsen-anhalt wollen feste zusagen vom bund, die ostländer fordern 60 milliarden euro für neue jobs und infrastruktur. eingeplant im bundeshaushalt sind bisher 1,5 milliarden euro bis 2021. finanzminister olaf scholz (spd) hat gesagt, es sei klar, dass es dabei nicht bleibe und "viele weitere milliarden" ausgegeben würden. in kattowitz beraten derzeit vertreter aus fast 200 staaten über regeln für die umsetzung des pariser klimaabkommens, das zum ziel hat, die erderwärmung auf unter zwei grad im vergleich zum vorindustriellen niveau zu begrenzen. die konferenz soll am freitag enden. welche rolle "just transition" als begriff darin spielen soll, ist noch umstritten. & 234 & very low & Low & Socio-Economic & NA & NA & 2018-12-12 & 2018 & 3 & ECO
Frame & v.low & Regional & <500 & -1.0405052 & -1.0641830 & 1.2185583 & -1.5781059 & 1.4491718 & 12.0 & 1.6295860 & 1.9019065 & Payer & Domestic & European & Mixed & Domestic|ECO & Positive\\
\addlinespace
Germany & https://www.t-online.de/nachrichten/id\_85757852/neues-online-portal-zeigt-eu-foerderung-in-sachsen.html & 207 & T-online.de & Private/Non-Public & Online only & National & very high = CP is most important issue + CP is mentioned in title/headline & Improve governance & Factual & Subnational & No myth & NA & NA & NA & NA & NA & NA & NA & NA & Germany & neues online-portal zeigt eu-förderung in sachsen & 2019-05-15 & europäischer fonds für regionale entwicklung & die menschen in sachsen könne sich online über projekte informieren, die im freistaat mit geld aus der europäischen union entstanden sind. elf tage vor der europawahl am 26. mai hat die staatsregierung dafür am mittwoch eine internetseite freigeschaltet. in der datenbank sind details zu allen seit 2014 geförderten projekten abrufbar, teilte die staatskanzlei mit. auf einer karte werden die standorte der von den programmen europäischer fonds für regionale entwicklung (efre) und europäischer sozialfonds (esf) finanzierten maßnahmen dargestellt. "sachsen profitiert in hohem maße von der eu-förderung. dies war und ist ein entscheidender baustein für die erfolgreiche entwicklung des freistaates", sagte staatskanzleichef und europaminister oliver schenk (cdu). seit 1990 hat der freistaat nach regierungsangaben mehr als 20 milliarden euro an eu-fördermitteln erhalten. darunter sind 14 milliarden aus efre und esf. allein zwischen 2014 und 2020 stehen aus beiden töpfen insgesamt rund 2,8 milliarden euro zur verfügung. "durch das projektportal wird nicht nur die menge an geförderten vorhaben deutlich, sondern auch ihre vielfalt", sagte sachsens wirtschaftsminister martin dulig (spd). & 169 & very high & High & Governance & NA & NA & 2019-05-15 & 2019 & 3 & POL
Frame & high-very high & National & <500 & -1.0405052 & -1.0641830 & 1.2185583 & -1.5781059 & 1.4491718 & 12.0 & 1.6295860 & 1.9019065 & Payer & Domestic & Domestic & Domestic & Domestic|POL & Neutral\\
Germany & http://www.stern.de/politik/ausland/werner-faymann-vergleicht-fluechtlingspolitik-von-viktor-orban-mit-holocaust-6448058.html & 278 & Stern magazine & Private/Non-Public & Online and Offline & National & very low = CP mentioned once & Political leverage & Balanced & Other country & No myth & NA & NA & NA & NA & NA & NA & NA & NA & Germany & werner faymann vergleicht flüchtlingspolitik von viktor orban mit holocaust & 2015-09-12 & strukturfonds & österreichs bundeskanzler werner faymann: "menschenrechte nach religionen zu unterteilen ist unerträglich." der österreichische bundeskanzler werner faymann hat das vorgehen des ungarischen ministerpräsidenten viktor orbán in der flüchtlingskrise mit der ns-rassenpolitik verglichen. "menschenrechte nach religionen zu unterteilen ist unerträglich", sagte der sozialdemokrat dem hamburger nachrichtenmagazin "der spiegel". "flüchtlinge in züge zu stecken in dem glauben, sie würden ganz woandershin fahren, weckt erinnerungen an die dunkelste zeit unseres kontinents." faymann brachte finanzielle sanktionen für eu-staaten wie ungarn ins gespräch, die sich einer quotenregelung für die aufteilung der flüchtlinge in der eu verweigern. "zur bewältigung der flüchtlingsbewegung brauchen wir strafen gegen solidaritätssünder", sagte der spö-chef. als beispiel nannte er die kürzung der mittel aus den strukturfonds, von denen vor allem die östlichen eu-staaten profitierten. die quotenregelung könne in der eu auch mit qualifizierter mehrheit durchgesetzt werden. & 137 & very low & Low & Power & NA & NA & 2015-09-12 & 2015 & 1 & POL
Frame & v.low & National & <500 & -1.0405052 & -1.0641830 & 1.2185583 & -1.5781059 & 1.4491718 & 12.0 & 1.6295860 & 1.9019065 & Payer & European & European & European & European|POL & Neutral\\
Germany & https://www.welt.de/politik/ausland/article162460437/In-einer-Europaeischen-Union-ist-kein-Platz-fuer-Grossbritannien.html & 258 & DIE WELT & Private/Non-Public & Online and Offline & National & low = CP mentioned more times but NOT important part of story (mainly about others issues) & Solidarity to poor countries/regions & Balanced & National & No myth & Economic development & Balanced & National & No myth & NA & NA & NA & NA & Germany & nachgefragt: & 2017-03-01 & kohäsionsfonds & der europapolitiker elmar brok erwartet, dass einige eu-staaten künftig enger zusammenarbeiten werden, eine "koalition der willigen". für die briten und die türkei sieht er eine sonderrolle vor. anzeige die debatte über die zukunft der europäischen union ist entbrannt. an diesem mittwoch stellt eu-kommissionspräsident jean-claude juncker im europaparlament pläne für eine weiterentwicklung vor. dringend nötig, findet elmar brok (cdu), der dienstälteste unter den europaabgeordneten und brexit-koordinator der europäischen volkspartei. die welt: ist die europäische union in ihrer existenz bedroht? elmar brok: wenn europa nun an den zentralen aufgaben scheitert, wird es legitimationsprobleme bekommen. es gibt viele herausforderungen: wir müssen den terror bekämpfen, dauerhafte lösungen für die flüchtlingsfrage finden, die äußere sicherheit bewahren und die ökonomischen und sozialen konsequenzen der globalisierung bewältigen. wir müssen die europäische union fit machen, um diese aufgaben, die kein nationalstaat allein lösen kann, anzugehen. anzeige die welt: die eu-kommission will ein europa der mehreren geschwindigkeiten. gut so? brok: ein europa der verschiedenen geschwindigkeiten sollte sich im rahmen der bestehenden eu-verträge bewegen. alles andere würde den zusammenhalt der europäischen union gefährden. die welt: was heißt das konkret? brok: wenn eine koalition der willigen bei bestimmten themen voranschreitet, sollte sie die bestehenden institutionen nutzen, also keine neue verwaltung aufbauen, und offen für alle anderen sein, die mitmachen wollen, und damit keine neuen mauern bauen. die welt: welche staaten sind die vorreiter? brok: ich glaube nicht, dass es immer dieselben staaten sind. in fragen der sicherheits- und verteidigungspolitik werden es andere länder sein als etwa im bereich der steuerpolitik. zum künftigen kern europas sollten auf jeden fall deutschland, frankreich und möglichst auch italien und polen zählen. die welt: und großbritannien? brok: in einer europäischen union ist kein platz für großbritannien. die briten haben sich klar für den austritt aus der eu entschieden. die welt: werden allen verbindungen gekappt? brok: wir müssen einen ring von freunden um die europäische union ziehen, zu denen auch großbritannien zählt. die schweiz und norwegen, die etwa beide zugang zum binnenmarkt haben und dem schengen-system angehören, zählen auch zu dieser gruppe. großbritannien will einen freihandelsvertrag und enge beziehungen etwa in der forschungs- und sicherheitspolitik. das sind gute themen. die welt: die türkei auch? brok: ein ring an freunden rund um die europäische union könnte ein attraktives modell für die türkei sein. das land ist doch nicht bereit, die notwendigen rechtsstaatlichen reformen anzugehen, die für eine eu-mitgliedschaft nötig wären. für eine gewisse zeit könnte dieser ring auch eine perspektive für die ukraine nach dem vorbild norwegens sein. die welt: führt eine verteidigungsunion europa aus der krise? brok: die bevölkerung erwartet, dass wir mehr gegen terror und für die äußere sicherheit tun. vor allem staaten in osteuropa fühlen sich von russland bedroht. die welt: was schlagen sie vor? brok: wir haben bereits europäische gefechtsverbände auf dem papier. nur eingesetzt werden sie derzeit nicht, weil es an praktischen dingen hakt. wir brauchen ein europäisches hauptquartier für zivile und militärische einsätze, um die einsatzfähigkeit dieser truppen zu erhöhen und die anwendung des instruments für strukturierte zusammenarbeit einer koalition der willigen. die eu-staaten geben doppelt so viel geld für verteidigung aus wie russland, haben mehr soldaten als die usa: alles verschwendung wegen mangelnder synergienutzung. die welt: sie wollen, ebenso wie juncker, die soziale dimension europas stärken. wie denn? brok: wir brauchen einen neuen sozialpakt in der eu, als teil der sozialen marktwirtschaft, der die ungleichheit verringert und das recht auf tarifverhandlungen überall verankert. im rahmen eines solchen paktes sollen die mitgliedstaaten ihre sozialpolitik besser koordinieren. in den römischen verträgen, deren 60. geburtstag wir ende märz feiern, steht das ziel festgeschrieben, die lebensverhältnisse in europa anzugleichen. dafür braucht es auch einen gewissen sozialen ausgleich, der über die kohäsionsfonds schon angestrebt wird. die welt: wie soll man sich das konkret vorstellen? brok: wir müssen die europäischen strukturfonds und zum beispiel den juncker-plan effizienter als bislang einsetzen und stärker darauf konzentrieren, kleine und mittlere unternehmen zu fördern und die arbeitslosigkeit zu bekämpfen. wir brauchen nicht noch mehr straßen und brücken, sondern mehr wettbewerbsfähigkeit im mittelstand. & 671 & low & Low & Values & Socio-Economic & NA & 2017-03-01 & 2017 & 2 & ECO
Frame & low-medium & National & 500-1000 & -1.0405052 & -1.0641830 & 1.2185583 & -1.5781059 & 1.4491718 & 12.0 & 1.6295860 & 1.9019065 & Payer & Domestic & Domestic & Domestic & Domestic|ECO & Neutral\\
Germany & https://www.hna.de/politik/ost-laenderchefs-bei-regierungsbildung-an-osten-denken-zr-8941378.html & 216 & Hessisch Niedersachsische Allgemeine & Private/Non-Public & Online and Offline & Regional/Local & very low = CP mentioned once & Economic development & Positive & National + Subnational & No myth & NA & NA & NA & NA & NA & NA & NA & NA & Germany & ost-länderchefs: bei regierungsbildung an den osten denken & 2017-10-31 & kohäsionspolitik & dresden (dpa) - sachsens scheidender ministerpräsident stanislaw tillich (cdu) hat bundeskanzlerin angela merkel (cdu) aufgefordert, bei der regierungsbildung ostdeutsche interessen in den fokus zu nehmen. dazu schickte tillich auch im namen seiner ostdeutschen amtskollegen einen brief an die kanzlerin, wie die staatskanzlei am dienstag mitteilte. die ostdeutschen länder wiesen eine "nahezu flächendeckende strukturschwäche" auf, schrieb tillich, der derzeit den vorsitz der ministerpräsidentenkonferenz ost innehat. diese schwäche müsse überwunden werden. dafür sei ostdeutschland weiter auf finanzielle förderung angewiesen - sowohl aus deutschen töpfen als auch im rahmen der eu-kohäsionspolitik. "ein abruptes ende der strukturförderung in ostdeutschland würde die erfolge der vergangenheit gefährden", hieß es in dem schreiben. andere forderung bezogen sich auf den ausbau der infrastruktur, auf die bessere hausarztversorgung und den vorläufigen verzicht auf den braunkohleausstieg. & 125 & very low & Low & Socio-Economic & NA & NA & 2017-10-31 & 2017 & 2 & ECO
Frame & v.low & Regional & <500 & -1.0405052 & -1.0641830 & 1.2185583 & -1.5781059 & 1.4491718 & 12.0 & 1.6295860 & 1.9019065 & Payer & Domestic & Domestic & Domestic & Domestic|ECO & Positive\\
Germany & http://www.dw.com/de/deutschland-eu-gelder-nur-an-staaten-mit-recht-und-gesetz/a-39070825 & 248 & Deutsche Welle (English) & Public & Online and Offline & National & high = CP is most important issue in story (can also cover other issues) & Political leverage & Balanced & National + Other country & No myth & NA & NA & NA & NA & NA & NA & NA & NA & Germany & deutschland: eu-gelder nur an staaten mit recht und gesetz | aktuell europa | dw | 31.05.2017 & 2017-05-31 & kohäsionsfonds & ihre länder stehen im fokus: polens ministerpräsidentin beata szydlo und ihr ungarischer kollege viktor orban das bundeswirtschaftsministerium in berlin bestätigte, dass die regierung in einer stellungnahme zur zukunft des eu-kohäsionsfonds eine "bindung an die einhaltung der rechtsstaatlichen grundwerte der eu" befürwortet. die stellungnahme werde "in kürze" an die eu-kommission geschickt, sagte eine ministeriumssprecherin. eu-kohäsionsfonds als hebel die bundesregierung nehme damit eine debatte auf, die im europaparlament und in der eu-kommission schon länger geführt werde. auch eu-justizkommissarin vera jourova hatte sich für die verknüpfung von mittelvergabe und achtung rechtsstaatlicher prinzipien ausgesprochen. länder wie polen oder ungarn, die aus sicht der eu rechtsstaatsprinzipien und demokratische grundwerte nicht einhalten, müssten dann mit mittelkürzungen rechnen. der eu-kohäsionsfonds soll einen ausgleich zwischen reicheren und ärmeren staaten schaffen. polen ist der mit abstand größte empfänger von mitteln aus diesem fonds. für das land sind in der siebenjährigen eu-haushaltsperiode von 2014 bis 2020 rund 23,2 milliarden euro vorgesehen - mehr als ein drittel aller mittel. bei ungarn sind es gut sechs milliarden euro. der neue vorschlag deutschlands bezieht sich allerdings erst auf die haushaltsperiode von 2021 bis 2027. szydlo: deutscher vorschlag widerspricht eu-verträgen die verknüpfung mit rechtsstaatlichen prinzipien sei "sehr vernünftig", sagte die sprecherin des wirtschaftsministeriums. die eu zeichne sich vor allem dadurch aus, dass sie eine wertegemeinschaft sei. "die basis einer solchen wertegemeinschaft und die basis der glaubwürdigkeit der eu ist auch die einhaltung der grundwerte." polen lehnte dagegen umgehend die pläne der bundesregierung ab. ministerpräsident beata szydlo sagte, der vorschlag, bestimmten mitgliedstaaten strukturhilfen vorzuenthalten, widerspreche den eu-verträgen. konrad szymanski, der für eu-angelegenheiten zuständige stellvertretende außenminister, äußerte skepsis, ob sich solche pläne überhaupt umsetzen ließen. die eu-kommission könne darüber gar nicht entscheiden. & 287 & high & High & Power & NA & NA & 2017-05-31 & 2017 & 2 & POL
Frame & high-very high & National & <500 & -1.0405052 & -1.0641830 & 1.2185583 & -1.5781059 & 1.4491718 & 12.0 & 1.6295860 & 1.9019065 & Payer & Domestic & European & Mixed & Domestic|POL & Neutral\\
\addlinespace
Germany & https://www.merkur.de/politik/berlin-will-eu-mittelvergabe-an-rechtsstaatliche-prinzipien-knuepfen-zr-8366308.html & 212 & Merkur.de & Private/Non-Public & Online and Offline & Regional/Local & high = CP is most important issue in story (can also cover other issues) & Political leverage & Balanced & National + Other country & NA & NA & NA & NA & NA & NA & NA & NA & NA & Germany & keine rechtsstaatliche prinzipien? keine eu-fördermittel & 2017-05-31 & kohäsionsfonds & keine rechtsstaatliche prinzipien? keine eu-fördermittel die bundesregierung will die vergabe von fördermitteln in der eu künftig von der achtung rechtsstaatlicher prinzipien und der grundrechte in den mitgliedstaaten abhängig machen. berlin - das wirtschaftsministerium in berlin bestätigte am mittwoch, dass die regierung in einer stellungnahme zur zukunft der eu-kohäsionsfonds eine "bindung an die einhaltung der rechtsstaatlichen grundwerte der eu" befürwortet. die stellungnahme werde "in kürze" an die eu-kommission geschickt, sagte eine ministeriumssprecherin. die bundesregierung nehme damit eine debatte auf, die im europaparlament und in der eu-kommission schon länger geführt werde. auch eu-justizkommissarin vera jourova hatte sich für die verknüpfung von mittelvergabe und achtung rechtsstaatlicher prinzipien ausgesprochen. länder wie polen oder ungarn, die aus sicht der eu demokratische grundwerte nicht einhalten, müssten dann mit mittelkürzungen rechnen. verknüpfung mit rechtsstaatlichen prinzipien sei "sehr vernünftig" die stellungnahme der bundesregierung bezieht sich auf die nächste siebenjährige haushaltsperiode von 2021 bis 2027. das vom wirtschaftsministerium erarbeitete papier wurde in den vergangenen monaten zwischen den ressorts abgestimmt. die eu-kommission soll nach dem wunsch berlins die einführung einer neuen konditionalität mit blick auf die nächste haushaltsperiode prüfen. die verknüpfung mit rechtsstaatlichen prinzipien sei "sehr vernünftig", sagte die sprecherin des wirtschaftsministeriums. die eu zeichne sich vor allem dadurch aus, dass sie eine wertegemeinschaft sei. "die basis einer solchen wertegemeinschaft und die basis der glaubwürdigkeit der eu ist auch die einhaltung der grundwerte." polen ist der mit abstand größte empfänger von mitteln aus dem eu-kohäsionsfonds, der einen ausgleich zwischen reicheren und ärmeren staaten schaffen soll. für das land sind in der siebenjährigen eu-haushaltsperiode von 2014 bis 2020 rund 23,2 milliarden euro vorgesehen - mehr als ein drittel aller mittel. bei ungarn sind es gut sechs milliarden euro. afp rubriklistenbild: © dpa & 287 & high & High & Power & NA & NA & 2017-05-31 & 2017 & 2 & POL
Frame & high-very high & Regional & <500 & -1.0405052 & -1.0641830 & 1.2185583 & -1.5781059 & 1.4491718 & 12.0 & 1.6295860 & 1.9019065 & Payer & Domestic & European & Mixed & Domestic|POL & Neutral\\
Germany & http://www.pnp.de/nachrichten/politik/2713504\_Ost-LaenderchefsBei-Regierungsbildung-an-den-Osten-denken.html & 261 & Passauer Neue Presse & Private/Non-Public & Online and Offline & Regional/Local & very low = CP mentioned once & Economic development & Positive & Subnational & No myth & NA & NA & NA & NA & NA & NA & NA & NA & Germany & ost-länderchefs: bei regierungsbildung an den osten denken & 2017-10-31 & kohäsionspolitik & sachsens scheidender ministerpräsident stanislaw tillich (cdu) hat bundeskanzlerin angela merkel (cdu) aufgefordert, bei der regierungsbildung ostdeutsche interessen in den fokus zu nehmen. dazu schickte tillich auch im namen seiner ostdeutschen amtskollegen einen brief an die kanzlerin, wie die staatskanzlei am dienstag mitteilte. die ostdeutschen länder wiesen eine "nahezu flächendeckende strukturschwäche" auf, schrieb tillich, der derzeit den vorsitz der ministerpräsidentenkonferenz ost innehat. diese schwäche müsse überwunden werden. dafür sei ostdeutschland weiter auf finanzielle förderung angewiesen - sowohl aus deutschen töpfen als auch im rahmen der eu-kohäsionspolitik. "ein abruptes ende der strukturförderung in ostdeutschland würde die erfolge der vergangenheit gefährden", hieß es in dem schreiben. andere forderung bezogen sich auf den ausbau der infrastruktur, auf die bessere hausarztversorgung und den vorläufigen verzicht auf den braunkohleausstieg. & 123 & very low & Low & Socio-Economic & NA & NA & 2017-10-31 & 2017 & 2 & ECO
Frame & v.low & Regional & <500 & -1.0405052 & -1.0641830 & 1.2185583 & -1.5781059 & 1.4491718 & 12.0 & 1.6295860 & 1.9019065 & Payer & Domestic & Domestic & Domestic & Domestic|ECO & Positive\\
Germany & https://www.taz.de/!5592086/ & 266 & taz.de & Private/Non-Public & Online and Offline & National & very low = CP mentioned once & Social justice & Positive & National & No myth & NA & NA & NA & NA & NA & NA & NA & NA & Germany & europas sozialpolitik: medizin gegen rechts & 2019-05-12 & kohäsionsfonds & europas sozialpolitik medizin gegen rechts sinn und zukunft der eu bestehen auch darin, für soziale sicherheit zu sorgen. aber warum kommt dieses soziale europa nur so langsam voran? kreativität, leistungsbereitschaft und zusammenhalt - eine pille gegen rechts in europa? foto: imago/science photo library es ist ein großer fortschritt, der zu wenig gewürdigt wird: die europäische union hat es geschafft, sich auf ein epochales prinzip zu einigen. es heißt "gleicher lohn für gleiche arbeit am gleichen ort". polnische bauarbeiter müssen den bundesdeutschen tariflohn erhalten, wenn sie ein gebäude in hannover errichten. deutsche kellnerinnen in österreichischen skihütten bekommen den dort gültigen lohn, wenn er höher ist als der hiesige. das ist der grundsatz der sogenannten entsenderichtlinie. das eu-parlament hat die reform bereits im jahr 2018 beschlossen. spätestens ab sommer 2020 wird sie für rund 450 millionen europäer*innen gelten. augenblicklich dürfen rumänische oder bulgarische firmen ihren leuten noch die niedrigen einheimischen löhne zahlen, wenn sie sie nach deutschland zum arbeiten schicken. die billig-konkurrenz nervt hiesige firmen und beschäftigte. doch bald ist schluss mit dieser art des lohndumpings. die botschaft lautet: die als wirtschaftslobby und bürokratenkonvent verrufene eu kann auch sozialpolitik! sie vertritt auch die interessen von arbeitnehmer*innen. was bringt mir europa eigentlich?, fragen viele bürger*innen. inzwischen bekommt die eu druck aus mehreren richtungen - von konkurrierenden mächten wie china, den sich verabschiedenden briten, autoritären regierungen in warschau oder budapest. auch in westlichen kernländern der union behaupten rechtspopulisten, das gemeinsame europa biete zu wenige vorteile und zu viele nachteile. weit weg ist die europäische union heute aus der sicht vieler einwohner*innen. alte pro-eu-argumente - 75 jahre frieden - ziehen nicht mehr richtig, neue sind zwiespältig: grenzüberschreitende mobilität für arbeitnehmer kann eine schöne sache sein, allerdings nicht für diejenigen, die ihre heimat verlassen müssen, weil sie dort keinen job mehr finden. europäische sozialversicherung könnte funktionieren persönliche kosten-nutzen-rechnungen sind ein politischer faktor. die bewegung der gelbwesten in frankreich entzündete sich unter anderem an der taxe carbone, der steigenden ökosteuer, die benzin auch für diejenigen verteuerte, die auf dem land keine arbeit mehr finden und mit dem wagen in die stadt zur arbeit pendeln müssen, wo sie allerdings nur magere einkommen erzielen. solche erwägungen müssen in der europäischen politik künftig eine größere rolle spielen. manche eu-politiker*innen haben das schon gemerkt. sie denken sich konzepte aus, die praktischen, finanziellen nutzen versprechen. so propagierte manfred weber (csu), spitzenkandidat der europäischen volkspartei, dass die eu allen abiturient*innen ein interrail-ticket für die bahn schenken solle, um sie zur erforschung des kontinents zu animieren. das großzügige vorhaben wurde jedoch heruntergekocht: 2018 erhielten nur rund 30.000 junge menschen den kostenlosen fahrschein. die botschaft lautet: die als wirtschaftslobby verrufene eu kann auch sozialpolitik! bundesfinanzminister olaf scholz (spd) nahm ebenfalls einen anlauf. von ihm stammt der vorschlag einer europäischen arbeitslosen-rückversicherung. dieser entstand unter anderem aus diskussionen zwischen den französischen und deutschen regierungen. scholz versteht ihn auch als ein mittel, um den rechtspopulisten das wasser abzugraben. grundsätzlich könnte diese neue europäische sozialversicherung funktionieren: in guten zeiten zahlen die mitgliedsländer milliarden euro in einen gemeinsamen topf, aus dem sie im falle von wirtschaftskrisen zuschüsse zu ihren nationalen arbeitslosenversicherungen erhalten. dieses geld verhindert, dass sie leistungen an ihre arbeitslosen kürzen, wenn der abschwung länger dauert. die unterstützung aus brüssel wirkt stabilisierend, sozial und ökonomisch. freilich ist auch dieses projekt nur die mini-ausgabe einer größeren version. dabei würden erwerbslose individuell überweisungen der eu erhalten, die das nationale arbeitslosengeld aufstocken. die zuwendung aus brüssel würde beweisen: die eu kümmert sich, sie verbessert das leben ihrer bürger*innen, sie hat einen unmittelbaren sinn. eine derart spürbare sozialpolitik ist jedoch schwer durchzusetzen. das liegt auch an der entstehungsgeschichte der staatengemeinschaft, erklärt simone leiber, politikprofessorin der uni duisburg-essen. anfangs hieß das ganze schließlich noch europäische wirtschaftsgemeinschaft. die eu begann als wirtschaftliche integration und vereinheitlichung von märkten. eine gute medizin gegen rechts diese logik dominiert noch heute. die mitglieder können sich eher darauf verständigen, beschränkungen für firmen zu verringern, als neue soziale standards festzusetzen. in den neoliberalen jahrzehnten zwischen 1980 und 2010 war diese tendenz besonders ausgeprägt. überhaupt besitzt die eu nur begrenzte kompetenzen für sozialpolitik. die sozialversicherungen, beiträge und löhne sind zum großen teil sache der nationalstaaten. cdu-chefin annegret kramp-karrenbauer betonte das unlängst noch einmal, als sie auf die reformvorschläge des französischen präsidenten emmanuel macron antwortete: "eine europäisierung der sozialsysteme und des mindestlohns wären der falsche weg." allerdings ist die eu auch heute keine unsoziale veranstaltung. "dank der struktur- und kohäsionsfonds konnten regionale ungleichheiten bekämpft werden", sagt sophie pornschlegel von der denkfabrik das progressive zentrum in berlin. "auch die teilhabe am gemeinsamen binnenmarkt hat den wohlstand der mitgliedstaaten steigen lassen - jedoch ohne unbedingt gerecht verteilt zu sein." während der vergangenen zehn jahre mussten mitglieder wie griechenland und portugal herbe rückschläge verkraften. manchmal klappt sogar eine gemeinsame sozialpolitik, die über das minimum hinausgeht - wie bei der entsenderichtlinie. entscheidend war dabei, dass viele regierungen lohndumping aus dem ausland als problem betrachteten. frankreich, deutschland und die benelux-länder waren ebenso betroffen wie polen, ungarn und rumänien. so ist die debatte über eine stärkere, gemeinsame soziale sicherung in europa im gang. zum beispiel schlägt die spd im wahlkampf vor, dass der mindestlohn in jedem staat 60 prozent des durchschnittsverdienstes betragen solle. das ist richtig, weil der sinn europas auch darin besteht, die soziale sicherheit der einwohner*innen zu erhöhen. so fördert man kreativität, leistungsbereitschaft und zusammenhalt - eine gute medizin gegen rechts. um europa gegen seine feinde zu schützen, holt man es am besten näher an die bürger*innen heran. & 919 & very low & Low & Socio-Economic & NA & NA & 2019-05-12 & 2019 & 3 & ECO
Frame & v.low & National & 500-1000 & -1.0405052 & -1.0641830 & 1.2185583 & -1.5781059 & 1.4491718 & 12.0 & 1.6295860 & 1.9019065 & Payer & Domestic & Domestic & Domestic & Domestic|ECO & Positive\\
Germany & http://www.wiwo.de/politik/europa/ukraine-die-eu-schuldet-kiew-eine-klare-ansage-naemlich-nein/10900534.html & 270 & Wirtschafts Woche & Private/Non-Public & Online and Offline & National & very low = CP mentioned once & Political leverage & Positive & EU + Other country & No myth & NA & NA & NA & NA & NA & NA & NA & NA & Germany & ukraine: die eu schuldet kiew eine klare ansage, nämlich: nein & 2014-10-28 & kohäsionsfonds & bislang traut sich in brüssel und den anderen europäischen hauptstädten niemand, auf dieses ansinnen so zu reagieren, wie es vernünftig wäre: nämlich in aller freundschaft klar zu machen, dass ein eu-beitritt der ukraine nicht in frage kommen kann, weil er die finanziellen und integrativen fähigkeiten der union völlig überlasten würde. zu befürchten ist aber, dass stattdessen wieder alles auf die klassische strategie der karotte zuläuft: die beitrittsperspektive vor der nase soll unsichere kandidaten zu wirtschaftlichen, politischen und rechtsstaatlichen reformen verlocken. stabilität und demokratie werden damit vom selbstzweck zur politischen handelsware degradiert. den mäßigen erfolg dieses schachers - ihr werdet gute demokraten, wir bezahlen - kann man in bulgarien und rumänien betrachten. die geschichte dieser hybriden erweiterungspolitik beginnt in den frühen 1970er jahren, als mit der machtübernahme der sozialdemokraten in deutschland und der sozialisten in frankreich die damalige eg die politisch instabilen, gerade erst von diktaturen befreiten und wirtschaftlich zurück gebliebenen mittelmeerländer zu stabilisieren trachtete. es drohte - gar nicht so unähnlich zur heutigen situation der ukraine - andernfalls die machtübernahme radikaler kräfte, nämlich der kommunisten. da schien die aufnahme in die eg und die entwicklung dieser länder mit hilfe der brüsseler kohäsionsfonds das mittel der wahl. war es ja auch: portugal, spanien und griechenland wurden nicht kommunistisch, sondern blieben demokratisch - und dank europäischer hilfen relativ wohlhabend. doch der sozialdemokratische integrationstraum wurde eben nur oberflächlich realität. die wirtschafts- und währungskulturen innerhalb der wachsenden union blieben unterschiedlich - und das nord-süd-gefälle durch den euro als gemeinsame währung überdeutlich. die rechnung für den traum ist noch immer nicht beglichen, sie wächst stattdessen stetig weiter in der ezb-bilanz. & 263 & very low & Low & Power & NA & NA & 2014-10-28 & 2014 & 1 & POL
Frame & v.low & National & <500 & -1.0405052 & -1.0641830 & 1.2185583 & -1.5781059 & 1.4491718 & 12.0 & 1.6295860 & 1.9019065 & Payer & European & European & European & European|POL & Positive\\
Germany & https://www.pnp.de/nachrichten/bayern/2062628\_Bayerische-Hochschulen-bekommen-mehr-Geld-von-der-EU.html & 195 & Passauer Neue Presse & Private/Non-Public & Online and Offline & Regional/Local & very low = CP mentioned once & Public services & Positive & EU + Subnational & No myth & NA & NA & NA & NA & NA & NA & NA & NA & NA & bayerische hochschulen bekommen mehr geld von der eu & 2016-06-05 & NA & NA & 99 & very low & Low & Socio-Economic & NA & NA & 2016-06-05 & 2016 & 2 & ECO
Frame & v.low & Regional & <500 & -1.0405052 & -1.0641830 & 1.2185583 & -1.5781059 & 1.4491718 & 12.0 & 1.6295860 & 1.9019065 & Payer & Domestic & European & Mixed & Domestic|ECO & Positive\\
\addlinespace
Germany & https://www.morgenpost.de/wirtschaft/article215477553/Rechnungshof-kritisiert-Schummelei-mit-EU-Foerdergeld.html & 241 & Berliner Morgenpost - Berlin & Private/Non-Public & Online and Offline & Regional/Local & medium = CP is important part of story & Mismanagement & Negative & EU & No myth & Fraud/Corruption & Negative & EU & No myth & NA & NA & NA & NA & Germany & rechnungshof: schummelei mit eu-fördergeldern & 2018-10-03 & strukturfonds & brüssel. die aussicht auf europäische fördergelder machte die bauernfamilie erfinderisch. eigentlich sollten die eu-prämien die rinderzucht unterstützen, indem bauern geld erhalten für den kauf von kühen aus anderen beständen, sofern die tiere noch nicht gekalbt hatten. doch die landwirtsfamilie aus polen teilte einfach ihre milchviehherde in zwei bestände auf, eine für den vater und eine für den sohn. dann verkaufte der sohn dem vater die färsen und kassierte dafür eu-fördermittel. anschließend verkaufte der vater dem sohn diese kühe zurück. wieder gab es fördermittel, obwohl sich im gemeinsamen kuhstall überhaupt nichts geändert hatte. erst eine kontrolle durch den europäischen rechnungshof deckte den falschen kuhhandel auf. kein einzelfall, wie der neue jahresbericht des rechnungshofs zeigt, der an diesem donnerstag im eu-parlament vorgestellt wird. bei der verwendung der milliardengelder, die die eu jährlich vor allem für die regionale strukturförderung und die agrarpolitik ausgibt, wird mitunter getrickst, mitunter geschlampt. allerdings stießen die prüfer insgesamt auf weniger unregelmäßigkeiten als in früheren jahren: die fehlerquote, die bei stichproben ermittelt wurde, beträgt insgesamt 2,4 prozent - im jahr zuvor hatte sie noch bei 3,1 prozent gelegen. "die kontrollsysteme auf nationaler ebene sind besser geworden", sagte rechnungshof-präsident klaus-heiner lehne. doch die bilanz ist durchwachsen. zollbetrug bei einfuhren aus drittstaaten die prüfer entdeckten 13 derart gravierende fälle, dass sie das eu-amt für betrugsbekämpfung (olaf) einschalteten - die behörde wiederum meldet für das vergangene jahr die rückforderung von drei milliarden euro vor allem aus den eu-strukturfonds. der rechnungshofbericht listet einerseits merkwürdige einzelfälle auf: so kassierte frankreich etwa geld für die neuansiedlung von flüchtlingen, die es gar nicht aufgenommen hat. stutzig machte prüfer auch eine aktion der eu, die wahlen im westafrikanischen guinea-bissau zu unterstützen, indem brüssel wahlurnen und stimmzettel lieferte. doch die transportkosten waren mit fast 300.000 euro doppelt so hoch wie der wert der waren, rügt der rechnungshof. aber es gibt es auch strukturelle probleme, die weit größere schäden anrichten: der rechnungshof moniert, dass die eu-staaten zollbetrug bei einfuhren aus drittstaaten nur unzureichend verfolgen. textilien und andere waren, die nach europa kommen, würden zum teil systematisch zu gering bewertet, wodurch milliarden an zolleinnahmen verloren gehen. allein im fall großbritannien, dem eine klage der eu-kommission droht, beträgt der schaden über zwei milliarden euro. als problematisch erweist sich die milliardenschwere förderung ökologischer landwirtschaft, für die die prüfer "hohe mitnahmeeffekte" bei gleichzeitig "geringen anforderungen" feststellen. viele fördermittel werden nicht abgerufen und bei den fonds zur regionalen strukturförderung, die auch für deutschland eine bedeutende rolle spielen, moniert der rechnungshof eine vergleichsweise hohe zahl von unregelmäßigkeiten. nur bei einem drittel der untersuchten projekte wurden die ziele vollständig erreicht. erneut besorgt äußerte sich lehne zu den hohen rückständen beim abfluss von eu-fördermitteln. 270 milliarden euro sind zugesagt, werden aber von den mitgliedstaaten nicht abgerufen - mangels förderfähiger projekte, verwaltungsproblemen oder fehlender kofinanzierung. "wir schieben bei den fördermitteln eine gewaltige bugwelle vor uns her", sagte lehne. "dieses problem muss man endlich angehen." & 487 & medium & Medium & Governance & Governance & NA & 2018-10-03 & 2018 & 3 & POL
Frame & low-medium & Regional & <500 & -1.0405052 & -1.0641830 & 1.2185583 & -1.5781059 & 1.4491718 & 12.0 & 1.6295860 & 1.9019065 & Payer & European & European & European & European|POL & Negative\\
Germany & https://www.stuttgarter-zeitung.de/inhalt.verhandlungen-in-bruessel-suedwesten-bekommt-milliarden-aus-eu-haushalt.14f1391a-b6e9-4650-aa90-bfeff1abfff7.html & 256 & Stuttgarter-Zeitung.de & Private/Non-Public & Online and Offline & Regional/Local & low = CP mentioned more times but NOT important part of story (mainly about others issues) & Jobs & Factual & EU & No myth & Institutional bargaining over funding & Balanced & EU & No myth & NA & NA & NA & NA & Germany & verhandlungen in brüssel: südwesten bekommt milliarden aus eu-haushalt & 2018-02-27 & kohäsionsfonds & brüssel/stuttgart - deutschland ist der größte nettozahler in der eu. 2016 überwies die bundesrepublik, die mit abstand das land mit der größten wirtschaftsleistung in der eu von 28 mitgliedstaaten ist, 10,08 milliarden euro mehr nach brüssel, als an zuwendungen von der eu zurückflossen. pro kopf kostet die eu damit jeden bundesbürger im schnitt 176 euro im jahr. schweden (226 euro), niederländer (219) und briten (178) zahlen jedoch pro kopf mehr ein. in brüssel laufen gerade die verhandlungen an, wie viel geld die 27 mitgliedstaaten der eu im nächsten mittelfristigen finanzrahmen, der die haushaltsjahre 2021 bis 2027 abdecken wird, jeweils zum eu-haushalt beisteuern. am verhandlungstisch im rat, dem gremium der mitgliedstaaten, sitzen die staats- und regierungschefs. nicht direkt dabei sind die ministerpräsidenten der länder. dabei sind die schwerpunkte des eu-haushalts entscheidend dafür, wie viel geld am ende aus brüssel in die bundesländer zurückfließt. so kann baden-württemberg im laufenden eu-finanzrahmen, der sich von 2014 bis 2020 erstreckt, mit rückflüssen von insgesamt knapp fünf milliarden euro rechnen. von den sogenannten kohäsionsfonds, die den ärmeren mitgliedstaaten bei investitionen in umwelt und verkehrsinfrastruktur helfen sollen, ist deutschland ausgeschlossen. diese gelder gehen nur an die 15 mitgliedstaaten mit der niedrigsten wirtschaftsleistung pro kopf. am meisten eu-geld geht in die landwirtschaft im südwesten. das geht aus der antwort der landesregierung auf eine kleine anfrage des landtagsabgeordneten joachim kößler (cdu) hervor, die unserer zeitung vorliegt. zwischen 2014 und 2020 bekommt baden-württemberg rund 710 millionen euro aus dem europäischen landwirtschaftsfonds (eler). das sind etwa 7,5 prozent der an deutschland fließenden mittel aus diesem topf. flächenmäßig macht baden-württemberg etwa zehn prozent der bundesrepublik aus. zudem bekommt baden-württembergs landwirtschaft jedes jahr rund 420 millionen euro. aus diesem posten werden vor allem die direktzahlzungen an die landwirte bestritten. pro hektar land bekommt ein landwirt in deutschland im schnitt 281 euro pro jahr. den zweitgrößten posten macht die forschungsförderung aus. allein von 2014 bis mai vergangenen jahres haben einrichtungen in baden-württemberg eu-zuschüsse in höhe von 729 millionen euro eingeworben. die einrichtungen mussten sich dafür in ausschreibungen gegen konkurrenz in der gesamten eu durchsetzen. 280 millionen euro gingen an hochschulen im land, 238 millionen an forschungseinrichtungen und 193 millionen an unternehmen. den drittwichtigsten posten machen mittel aus dem europäischen sozialfonds (esf) aus. zwischen 2014 und 2020 bekommen projekte, die teils von den kommunen organisiert werden, eu-mittel in höhe von rund 260 millionen euro. das land steuert hierzu einen beitrag in etwa der gleichen höhe als eigenmittel bei. die mittel aus dem sozialfonds setzt das land vor allem ein, um langzeitarbeitslose und andere arbeitnehmer mit vermittlungshemmnissen möglichst in den arbeitsmarkt zu integrieren. den viertwichtigsten posten machen mittel aus dem eu-programm für die regionale entwicklung aus. hier dürften im laufenden finanzrahmen rund 246 millionen euro an eu-mitteln in den südwesten fließen. auch bei diesem programm sind eigenmittel des landes gefordert. der anteil an eigenmitteln kann schwanken zwischen 25 und 75 prozent. brüssel unterstützt zudem studenten bei auslandssemestern. 2013/2014 konnten allein im südwesten 7100 studenten ins ausland gehen. das erasmus+-programm leistete dabei organisatorische hilfe sowie einen beitrag von bis zu 500 euro im monat für den lebensunterhalt. 4300 studenten aus der eu konnten 2013/2014 zum studium in den südwesten kommen. & 541 & low & Low & Socio-Economic & Power & NA & 2018-02-27 & 2018 & 3 & ECO
Frame & low-medium & Regional & 500-1000 & -1.0405052 & -1.0641830 & 1.2185583 & -1.5781059 & 1.4491718 & 12.0 & 1.6295860 & 1.9019065 & Payer & European & European & European & European|ECO & Neutral\\
Germany & https://www.gmx.net/magazine/politik/viktor-orban-kontert-spd-chef-martin-schulz-ungarn-verdient-respekt-32735400 & 274 & GMX News & Private/Non-Public & Online only & National & very low = CP mentioned once & Political leverage & Negative & National + Other country & No myth & NA & NA & NA & NA & NA & NA & NA & NA & Germany & viktor orban kontert spd-chef martin schulz: ungarn verdient mehr respekt & 2018-01-07 & kohäsionsfonds & ungarns ministerpräsident viktor orban kontert vorwürfe von spd-chef schulz und will weiter daran festhalten, keine flüchtlinge in sein land zu lassen. der ungarische ministerpräsident viktor orban hat sich von spd-chef martin schulz "mehr respekt" für sein land erbeten. in anspielung auf schulz' früheres amt als präsident des europäischen parlaments sagte orban der "bild"-zeitung (montag): "was gut und nett in brüssel war - wo es keine offensichtlichen konsequenzen gab - ist eine andere geschichte, als in deutschland parteichef zu sein und mit anderen ländern zu kommunizieren. wir finden, wir verdienen mehr respekt." schulz hatte den csu-vorsitzenden horst seehofer aufgefordert, dem rechtsnationalen ungarischen regierungschef, der am freitag ehrengast der csu bei der winterklausur im oberbayerischen kloster seeon war, die grenzen aufzuzeigen. vor allem in der flüchtlingspolitik verfolge orban eine "gefährliche logik", hatte schulz kritisiert. "ich erwarte, dass herr seehofer ihm bei diesem thema und auch bei den themen presse- und meinungsfreiheit ganz klare grenzen aufzeigt." ungarn verweigert flüchtlingsaufnahme weiter orban verwahrte sich in dem interview gegen den vorwurf, ungarn nehme geld von der eu, weigere sich aber, flüchtlinge aufzunehmen. der sogenannte kohäsionsfonds, der der ungarischen wirtschaft zugutekomme, sei kein geschenk. "er ist ein fairer ausgleich, da wir unseren markt dem freien wettbewerb geöffnet haben. das hat absolut nichts mit der flüchtlingsfrage zu tun." orban bekräftigte, dass ungarn auch künftig keine flüchtlinge aufnehmen werde. "wir glauben, dass eine hohe zahl an muslimen notwendigerweise zu parallelgesellschaften führt", sagte er. "so etwas möchten wir nicht. und wir möchten uns nichts aufzwängen lassen." bildergalerie starten fotos aus aller welt - augenblicke, die ohne viele worte auskommen bewegend, schockierend, traurig, bezaubernd oder einfach nur schön: hier sehen sie - ständig aktualisiert - die besten fotos. © dpa & 278 & very low & Low & Power & NA & NA & 2018-01-07 & 2018 & 3 & POL
Frame & v.low & National & <500 & -1.0405052 & -1.0641830 & 1.2185583 & -1.5781059 & 1.4491718 & 12.0 & 1.6295860 & 1.9019065 & Payer & Domestic & European & Mixed & Domestic|POL & Negative\\
Germany & https://www.welt.de/regionales/sachsen-anhalt/article170195231/Bei-Regierungsbildung-an-den-Osten-denken.html & 246 & DIE WELT & Private/Non-Public & Online and Offline & National & very low = CP mentioned once & Economic development & Positive & Subnational & No myth & NA & NA & NA & NA & NA & NA & NA & NA & Germany & bei regierungsbildung an den osten denken - welt & 2017-10-31 & kohäsionspolitik & anzeige dresden - sachsens scheidender ministerpräsident stanislaw tillich (cdu) hat bundeskanzlerin angela merkel (cdu) aufgefordert, bei der regierungsbildung ostdeutsche interessen in den fokus zu nehmen. dazu schickte tillich auch im namen seiner ostdeutschen amtskollegen einen brief an die kanzlerin, wie die staatskanzlei am dienstag mitteilte. die ostdeutschen länder wiesen eine "nahezu flächendeckende strukturschwäche" auf, schrieb tillich, der derzeit den vorsitz der ministerpräsidentenkonferenz ost innehat. diese schwäche müsse überwunden werden. dafür sei ostdeutschland weiter auf finanzielle förderung angewiesen - sowohl aus deutschen töpfen als auch im rahmen der eu-kohäsionspolitik. "ein abruptes ende der strukturförderung in ostdeutschland würde die erfolge der vergangenheit gefährden", hieß es in dem schreiben. andere forderung bezogen sich auf den ausbau der infrastruktur, auf die bessere hausarztversorgung und den vorläufigen verzicht auf den braunkohleausstieg. & 125 & very low & Low & Socio-Economic & NA & NA & 2017-10-31 & 2017 & 2 & ECO
Frame & v.low & National & <500 & -1.0405052 & -1.0641830 & 1.2185583 & -1.5781059 & 1.4491718 & 12.0 & 1.6295860 & 1.9019065 & Payer & Domestic & Domestic & Domestic & Domestic|ECO & Positive\\
Germany & http://www.deutschlandfunk.de/praesidentschaftswahl-in-frankreich-mit-praesident-macron.694.de.html?dram:article\_id=385650 & 289 & Deutschlandfunk & Public & Online and Offline & National & very low = CP mentioned once & Solidarity to poor countries/regions & Balanced & National + Other country & No myth & NA & NA & NA & NA & NA & NA & NA & NA & Germany & präsidentschaftswahl in frankreich - "mit präsident macron schlagen wir ein neues kapitel auf" & 2017-05-08 & kohäsionsfonds & peter kapern: bei uns am telefon ist gunther krichbaum von der cdu, vorsitzender des europaausschusses im bundestag. guten tag, herr krichbaum. gunther krichbaum: schönen guten tag, herr kapern. kapern: erst schleift macron die französische parteienlandschaft und dann schäubles austeritäts-bollwerk? krichbaum: nein. ich denke, so kann man das nicht simplifizieren. denn zunächst ist es sehr erfreulich, dass herr macron doch die satte mehrheit der franzosen hinter sich weiß. zwei drittel haben für ihn gestimmt. und wir brauchen auch in der tat dieses positive signal für die weitere deutsch-französische zusammenarbeit. aber manfred weber hat natürlich völlig recht: die hausaufgaben sind hausaufgaben und sie müssen zunächst in paris gelöst werden. da kommt natürlich einiges auf ihn zu. und er wird auch liefern müssen. kapern: alles bleibt so, wie es ist? krichbaum: es kann nicht alles so bleiben wie es ist, denn präsident hollande hatte zuletzt nur noch umfragewerte von 12 bis 14 prozent. er war im ansehen der franzosen sehr unbeliebt geworden. die probleme sind frappierend, wenn man alleine weiß, dass die jugendarbeitslosigkeit sehr hoch ist, die arbeitslosigkeit generell sehr hoch ist. deswegen: da muss frankreich sehr viel tun, um wettbewerbsfähiger zu werden. kapern: aber was tut deutschland, herr krichbaum, um frankreich zu helfen? denn es gehen ja sehr viele experten davon aus, dass ohne eine geänderte politik in ländern wie frankreich, italien, spanien, griechenland, ohne eine geänderte deutsche politik dort die wirtschaft nicht wieder zum leben erweckt wird. krichbaum: ja gut, die politik muss alles dafür tun, damit in den jeweiligen ländern - und das sind hausaufgaben - wettbewerbsfähigere strukturen platzgreifen können. denn wir stehen ja nicht nur in einem wettbewerb innerhalb der europäischen union; wir sind ja keine insel, sondern wir stehen im zeitalter der globalisierung. und da muss man sich natürlich auch auf den weltmärkten entsprechend behaupten. da hat frankreich in der tat sehr viel nachholpotenzial. es gibt den sogenannten davos-index, der die wettbewerbsfähigkeit der länder weltweit misst durch das weltwirtschaftsinstitut. hier liegen wir, die bundesrepublik deutschland, auf platz fünf roundabout, frankreich aber circa 23/24, und das erklärt die malaise. man ist in den vergangenen jahren einfach dort nicht vorangekommen, auch bei den arbeitsmarktreformen. das ist im übrigen in italien dasselbe. und das erklärt im übrigen auch die hohe frustration der ... kapern: das heißt, die deutschen politiker lehnen sich jetzt erst mal zurück und gucken, was macron macht? krichbaum: das erklärt die hohe frustration auch der franzosen selbst. es ist eigentlich kein front national, es ist ein frust national, der in frankreich vorherrscht. und da muss natürlich dann auch die politik antworten finden. was das europäische und die europäische dimension angeht, so darf man nicht übersehen: es gibt ja sehr viele solidarinstrumente, die wir heute schon haben - denken sie an die kohäsionsfonds, denken sie an die infrastrukturfonds -, dass wir gerade auch regionen weiterhelfen in europa, die entwickelt werden müssen, und da leistet auch deutschland sehr, sehr viel. man darf nicht vergessen: der finanzierungsanteil der bundesrepublik deutschland am gesamthaushalt der europäischen union ist über ein fünftel. kapern: herr krichbaum, das waren jetzt - ich glaube, ich habe richtig mitgezählt - drei antworten, sogar eigentlich recht ausführlich. ich habe eigentlich nur einen satz gehört: die deutsche politik bleibt wie sie ist. krichbaum: ich habe gesagt, dass sich die deutsche politik gerade in der europäischen integration sehr stark einbringt, gerade auch finanziell, nicht nur im übrigen für frankreich. kapern: da wird nicht draufgesattelt. krichbaum: aber es bleibt dabei: die länder selbst, die nationalen regierungen müssen selbst auch dafür sorgen, weil die europäische union dafür auch gar keine kompetenzen besitzt. wir können ja auch nicht eingreifen in die kompetenzen eines anderen landes. hier ist jedes land selbst gefordert. die bundesrepublik deutschland war es im übrigen auch. durch die agenda 2010 hatte der damalige bundeskanzler schröder dafür gesorgt, dass tatsächlich dann auch der anfang eingeleitet wurde für mehr wettbewerbsfähigkeit in deutschland für europa. und es nutzt im übrigen auch keinem land, wenn wir uns selber schwächen würden. dadurch werden die anderen nicht schneller, sondern wir müssen in der tat dafür sorgen, dass mit unserem einfluss die länder selbst dann auch die entsprechenden reformen machen. macron im übrigen hat das auch getan, die sogenannten macron-gesetze, wenn man so möchte, und da wurden schon die richtigen schritte in die richtige richtung unternommen. kapern: herr krichbaum, lassen sie uns doch ganz schnell noch ein paar europäische dinge durchdeklinieren, die möglicherweise wahrscheinlich aus frankreich gefordert werden. wie sieht es damit in deutschland aus? ein eu-finanzminister - mit der union zu machen, ja oder nein? krichbaum: sie müssen mich schon auch ausreden lassen, herr kapern. es würde dann der zweite schritt vor den ersten gemacht. ein finanzminister ist dann denkbar, wir werden auch darüber reden müssen, dass dann in nationale kompetenzen eingegriffen werden kann und muss. aber erst muss dann auch klar sein, dass hier keine vergemeinschaftung der schulden stattfindet, was im übrigen die europäischen verträge ausschließen. kapern: ein eigener haushalt der eurozone, ja oder nein? krichbaum: die eurozone sollte keinen eigenen haushalt haben. wir haben einen haushalt der europäischen union, weil auch andere staaten der eurozone beitreten sollen und wollen, und wir sollten nicht zu viele unionen in einer union haben. das leistet nur der fragmentierung in der eu auch weiter vorschub. kapern: ein europäischer schuldentilgungsfonds, wie er von den deutschen wirtschaftsweisen seit langem gefordert wird? krichbaum: keine vergemeinschaftung von schulden. da gilt das no bail-out-verbot. das lassen die verträge gar nicht zu. kapern: dann schauen wir mal, ob emmanuel macron mit diesem angebot aus berlin dann tatsächlich zufrieden sein wird. krichbaum: ich bin zuversichtlich für weitere impulse in der deutsch-französischen zusammenarbeit. und mit präsident macron schlagen wir tatsächlich auch ein neues kapitel auf. kapern: das war gunther krichbaum von der cdu, der vorsitzende des europaausschusses im bundestag. danke und auf wiederhören. krichbaum: ich danke ihnen, herr kapern. auf wiederhören! & 965 & very low & Low & Values & NA & NA & 2017-05-08 & 2017 & 2 & ECO
Frame & v.low & National & 500-1000 & -1.0405052 & -1.0641830 & 1.2185583 & -1.5781059 & 1.4491718 & 12.0 & 1.6295860 & 1.9019065 & Payer & Domestic & European & Mixed & Domestic|ECO & Neutral\\
\addlinespace
Germany & https://www.ksta.de/politik/forderungen-an-merkel-ost-landeschefs--neue-regierung-darf-osten-nicht-vergessen-28746304 & 250 & Kolner Stadt-Anzeiger & Private/Non-Public & Online and Offline & Regional/Local & very low = CP mentioned once & Economic development & Positive & Subnational & No myth & NA & NA & NA & NA & NA & NA & NA & NA & Germany & forderungen an merkel: ost-landeschefs: neue regierung darf osten nicht vergessen | kölner stadt-anzeiger & 2017-10-31 & kohäsionspolitik & die ministerpräsidenten der ostdeutschen länder haben bundeskanzlerin angela merkel (cdu) aufgefordert, bei der regierungsbildung ost-interessen im blick zu behalten. sachsens scheidender ministerpräsident stanislaw tillich (cdu) schickte dazu im namen seiner amtskollegen einen brief an die kanzlerin, wie die sächsische staatskanzlei am dienstag mitteilte. die ostdeutschen länder wiesen weiterhin eine "nahezu flächendeckende strukturschwäche" auf, schrieb tillich, der den vorsitz der ministerpräsidentenkonferenz ost innehat. diese schwäche müsse überwunden werden. dafür sei ostdeutschland weiter auf finanzielle förderung angewiesen - sowohl aus deutschen töpfen als auch im rahmen der eu-kohäsionspolitik. "ein abruptes ende der strukturförderung in ostdeutschland würde die erfolge der vergangenheit gefährden", hieß es in dem schreiben. auch müsse verhindert werden, dass ostdeutschland in eine ungünstige "sandwichposition" gerate - zwischen den hoch entwickelten regionen in westdeutschland und den sehr stark von der eu geförderten gebieten in osteuropa. daneben sprachen sich die landeschefs in dem schreiben gegen einen schnellen braunkohle-ausstieg aus. an der kohleverstromung hingen in ostdeutschland zehntausende arbeitsplätze. ein abruptes ende verbiete sich "schon aus respekt vor der lebensleistung der beschäftigten", hieß es. erst wenn es für sie nachhaltige zukunftsperspektiven gebe, dürfe das ende der kohlenutzung beschlossen werden. damit der anschluss an den wirtschaftsstarken westen gelinge, müsse der osten zudem besser an das bahnnetz und den luftverkehr angeschlossen werden, forderten tillich und seine kollegen weiter. außerdem müsse die künftige regierung eine flächendeckende versorgung mit schnellem internet und mobilfunk sicherstellen. in berlin laufen derzeit sondierungsgespräche zwischen vertretern von cdu, csu, fdp und grünen über eine mögliche künftige jamaika-koalition. bei den reizthemen klima und flüchtlinge hatte es zuletzt streit gegeben. nun gelangen bei den themen arbeit, rente, pflege, sicherheit und bildung und digitales deutliche fortschritte. (dpa) & 273 & very low & Low & Socio-Economic & NA & NA & 2017-10-31 & 2017 & 2 & ECO
Frame & v.low & Regional & <500 & -1.0405052 & -1.0641830 & 1.2185583 & -1.5781059 & 1.4491718 & 12.0 & 1.6295860 & 1.9019065 & Payer & Domestic & Domestic & Domestic & Domestic|ECO & Positive\\
Germany & http://www.zeit.de/politik/ausland/2017-09/ungarn-viktor-orban-eugh-fluechtlinge-verteilung & 229 & ZEIT ONLINE & Private/Non-Public & Online and Offline & National & very low = CP mentioned once & Political leverage & Balanced & Other country & No myth & NA & NA & NA & NA & NA & NA & NA & NA & Germany & ungarn: orbán will eugh-urteil nicht umsetzen & 2017-09-08 & kohäsionsfonds & ungarn wird trotz des flüchtlings-urteils des europäische gerichtshofs (eugh) keine migranten aufnehmen. zwar müsse sein land das eugh-urteil zur kenntnis nehmen, "denn wir können nicht das fundament der eu untergraben - und die anerkennung von recht und gesetz ist das fundament der eu", sagte ministerpräsident viktor orbán im staatsrundfunk. "gleichzeitig ist dieser richterspruch für uns aber kein grund, unsere politik zu ändern, die flüchtlinge ablehnt." der rechts-konservative ministerpräsident sagte, dass er sich von anderen eu-staaten nichts vorschreiben lassen wolle. "die einwanderungsländer wollen uns ihre logik aufzwingen, aber wir haben niemanden zu uns eingeladen, wir wollen kein einwanderungsland werden." orbán wies forderungen mehrerer mitgliedstaaten und aus der eu-kommission zurück, die zahlungen aus dem kohäsionsfonds zur förderung der finanzschwächeren eu-staaten an die bereitschaft zur aufnahme von flüchtlingen zu koppeln. dies verstoße gegen die regeln der eu und sei unmoralisch, sagte orbán. das höchste eu-gericht hatte klagen von ungarn und der slowakei gegen die eu-quotenregel für die aufnahme von flüchtlingen abgewiesen. da keine berufung gegen das urteil möglich ist, müssten beide länder nach geltender rechtslage gegen ihren willen migranten entsprechend den im ministerrat beschlossenen verteilungsschlüssel aufnehmen. seitennavigation startseite & 192 & very low & Low & Power & NA & NA & 2017-09-08 & 2017 & 2 & POL
Frame & v.low & National & <500 & -1.0405052 & -1.0641830 & 1.2185583 & -1.5781059 & 1.4491718 & 12.0 & 1.6295860 & 1.9019065 & Payer & European & European & European & European|POL & Neutral\\
Germany & https://www.welt.de/wirtschaft/article164211044/Und-wenn-die-Briten-gehen-ohne-zu-zahlen.html & 243 & DIE WELT & Private/Non-Public & Online and Offline & National & very low = CP mentioned once & Institutional bargaining over funding & Balanced & EU + Other country & No myth & NA & NA & NA & NA & NA & NA & NA & NA & Germany & brexit: und wenn die briten gehen, ohne zu zahlen? - welt & 2017-05-03 & kohäsionsfonds & 100 milliarden euro? die brexit-rechnung könnte deutlich höher ausfallen als gedacht. die briten sind empört. der eu-verhandlungsführer betont, das sei keine bestrafung. so kommen die forderungen zustande. anzeige 40, 60 oder gar 100 milliarden euro: der streit ums geld hat wochen vor dem start der brexit-verhandlungen mit großbritannien bereits begonnen. das land will ende märz 2019 aus der eu austreten. diese pocht darauf, dass london dabei seine finanziellen verpflichtungen aus der mitgliedschaft vollständig erfüllt. bisherige schätzungen aus brüssel bezifferten den betrag auf zwischen 40 und 60 milliarden euro. die eu will aber, dass london "alle" finanziellen verpflichtungen aus der unionsmitgliedschaft erfüllt - selbst nach dem austrittsdatum. london hält dagegen. die brexit-rechnung dürfte eines der schwierigsten kapitel der austrittsgespräche werden: anzeige warum fordert die eu von london geld? die eu plant ihren haushalt über zeiträume von sieben jahren. der aktuelle läuft bis ende 2020 - das sind fast zwei jahre nach dem geplanten brexit im märz 2019. bis dahin sind praktisch alle eu-mittel schon verplant - und ein teil der eingegangenen verpflichtungen bezieht sich sogar noch auf die jahre nach 2020. "das ist keine bestrafung", sagte der brexit-verhandlungsführer der eu, michel barnier, am mittwoch zu der geldforderung. es gehe nur darum, dass großbritannien das bezahlt, was es in der eu mitbeschlossen habe. um was für posten geht es? anzeige betroffen ist jegliche form von eu-ausgaben einschließlich des eu-kohäsionsfonds zur angleichung der lebensverhältnisse zwischen armen und reichen mitgliedstaaten. ein zentraler punkt sind auch pensionszahlungen für eu-beamte. "es geht dabei nicht um die pensionen für britische beamte, sondern den anteil an den pensionen für alle eu-beamte", sagte ein diplomat. als daneben besonders problematisch gilt der britische anteil an krediten, die über die europäische investitionsbank (eib) vergeben werden. wie die "financial times" berichtet, pochten zum beispiel frankreich und polen darauf, auch agrarausgaben bis zum jahr 2020 einzurechnen. deutschland wolle london demnach nicht erlauben, die austrittsrechnung zu drücken, indem der britische anteil an eu-gebäuden und anderen vermögenswerten abgezogen werde. lesen sie auch forderungen an großbritannien eu erhöht brexit-schlussrechnung mal eben um 66 prozent wie hoch könnte die rechnung ausfallen? luxemburgs regierungschef xavier bettel bestätigte am samstag beim brexit-sondergipfel inoffizielle schätzungen aus brüssel von 40 bis 60 milliarden euro. für empörung in london sorgte am mittwoch ein bericht der "financial times", wonach die austrittsrechnung sich gar auf bis zu 100 milliarden euro belaufen könnte. welche position nimmt london ein? "wir werden nicht 100 milliarden zahlen", sagte der britische brexit-minister david davis am mittwoch. schon davor waren die finanzforderungen in london regelmäßig auf ablehnung gestoßen. premierministerin theresa may sagte, großbritannien werde sicher nicht "gewaltige summen" an die eu überweisen, bleibt aber bisher vage. offizielle sprachregelung ist vorerst, london werde nur seine "internationalen verpflichtungen" erfüllen. bis wann soll die geldfrage geregelt sein? die eu will die finanzen direkt in der ersten verhandlungsphase angehen. zunächst werde dabei über die "methodik" verhandelt - also welche posten einbezogen werden, heißt es im entwurf der eu-kommission für das verhandlungsmandat. diese phase soll möglichst bis herbst abgeschlossen werden. im fertig ausgehandelten austrittsabkommen soll dann ein "globaler betrag" stehen, der in den folgenden jahren aber auch noch "technischen anpassungen" unterworfen werden kann. barnier zufolge verlangt die eu aber "keinen blankoscheck" von london. und wenn london geht, ohne zu zahlen? eine studie des britischen oberhauses kommt zu dem schluss, dass großbritannien die eu auch ohne weitere zahlungen verlassen könnte. es gebe aber auch "mögliche gewinne aus anderen bereichen der verhandlungen", auf die london dann verzichten würde, warnte die vorsitzende des finanzausschusses, kishwer falkner, im märz. hier könnte es etwa um den künftigen zugang britischer unternehmen zum europäischen binnenmarkt gehen. "wenn es keine einigung auf die austrittsrechnung gibt, gibt es überhaupt kein austrittsabkommen", sagt ein eu-vertreter. & 622 & very low & Low & Power & NA & NA & 2017-05-03 & 2017 & 2 & POL
Frame & v.low & National & 500-1000 & -1.0405052 & -1.0641830 & 1.2185583 & -1.5781059 & 1.4491718 & 12.0 & 1.6295860 & 1.9019065 & Payer & European & European & European & European|POL & Neutral\\
Germany & https://www.presseportal.de/pm/29188/3833256 & 249 & presseportal.de & Private/Non-Public & Online only & National & very low = CP mentioned once & Institutional bargaining over funding & Negative & EU + National & No myth & Environment/green/low-carbon & Negative & No actor & No myth & NA & NA & NA & NA & Germany & umwelt- und klimaschutz müssen grundlage für neuen eu-haushalt sein & 2018-01-08 & strukturfonds & brüssel/berlin (ots) - über 20 verbände aus dem natur-, tier- und umweltschutz haben heute ihre forderungen zum eu-budget nach 2020 veröffentlicht. hintergrund ist die zurzeit stattfindende eu-konferenz, bei der hochrangige vertreter aus politik und gesellschaft mit haushaltskommissar günter oettinger bereits jetzt die möglichkeiten und entscheidungen des neuen mehrjährigen eu-finanzrahmens (mfr) diskutieren. dieser ist das grundlegende instrument der finanzplanung und spiegelt die politischen prioritäten der eu ab 2021 wider. die verbände fordern die eu-kommission auf, den mfr stärker als bislang an der gesundheit und lebenswerten zukunft der 500 millionen menschen in europa auszurichten. dazu gehört, den eingegangenen verpflichtungen des klimaschutzabkommens von paris und der un-nachhaltigkeitsziele (sdgs) nachzukommen. eine nachhaltige entwicklung, die erhaltung der artenvielfalt und die bekämpfung des klimawandels müssen bei der mittelverteilung zur alternativlosen grundlage erklärt werden. "für ein lebenswertes europa brauchen wir eine haushaltspolitik, die eine gesunde zukunft gestaltet statt die fossile vergangenheit zu zementieren. wenn wir wollen, dass sich die menschen nachhaltig-moderne lebensstile leisten können, muss bei der eu-mittelverteilung grundlegend umgesteuert werden", sagt kai niebert, präsident des umweltdachverbands deutscher naturschutzring (dnr). die deutschen ngos befürchten, dass die sicherstellung von sauberer luft, boden und wasser zum schutz des menschen auch weiterhin nicht ausreichend beachtet wird. bereits für den laufenden haushalt hatte sich die eu-kommission vorgenommen, mindestens 20 prozent des gesamtbudgets für den klimaschutz aufzuwenden. die entsprechenden rund 200 milliarden euro sollten für maßnahmen in den bereichen strukturfonds, forschung, landwirtschaft, meerespolitik sowie fischerei und entwicklung eingesetzt werden. aktuellen schätzungen der eu zufolge wird dieses ausgabenziel nicht erreicht. auch beim derzeit größten ausgabenpunkt, der gemeinsamen agrarpolitik, fließt immer noch mehrheitlich geld in eine umweltschädigende landwirtschaft, während der naturschutz weiterhin unterfinanziert ist. ebenso stehen projekte wie der bau von erdgaspipelines oder straßen- und flugverkehrsprojekte den nachhaltigkeitszielen entgegen. stattdessen sollte die eu ihre zahlungen auf natur- und umweltschutz, tierwohl und unbelastete nahrungsmittel verlagern. "wir leben in einer zeit, in der es darum geht, die internationalen umweltschutz-verträge umzusetzen und zu einem erfolg werden zu lassen. die eu hat bei der ausgestaltung des neuen mehrjährigen haushaltsplans die einmalige chance, mit weitblick zu agieren und zum globalen vorreiter für ein gesundes leben und arbeiten im europäischen wirtschaftsraum zu werden", unterstützt ernst-ulrich von weizsäcker, ko-präsident des club of rome die forderungen der umweltverbände. neben den zusätzlichen herausforderungen - brexit, migration, verteidigung und sicherheit - und damit einhergehend einem voraussichtlich sinkenden budget bei zunehmenden ausgaben, fordern die verbände, die internationalen klima- und umweltschutz-verpflichtungen als blaupause zu nutzen: bei sämtlichen ausgaben muss genauestens geprüft werden, ob dadurch nachhaltigkeitsziele gefährdet werden. maßnahmen, die umwelt, gesundheit und klima und infolgedessen die europäischen volkswirtschaften unumkehrbar schädigen, dürfen nicht mehr subventioniert werden. die finanzmittel für natur- und umweltschutz müssen deutlich erhöht werden. der mfr bietet nach überzeugung der verbände eine entscheidende chance der neuausrichtung. durch ihn kann eine energie- und verkehrswende eingeleitet sowie eine nachhaltige industrie- und agrarwende ermöglicht werden. ferner kann dadurch die richtung vorgegeben werden, um eine dringend notwendige offensive im natur- und artenschutz durch die einrichtung eines ausreichend ausgestatteten eu-naturschutzfonds zu unterstützen. die natur- und umweltschutzorganisationen sehen die eu in einer zentralen rolle innerhalb der internationalen staatengemeinschaft auf den gebieten klimaschutz, nachhaltige energiewende, nachhaltige lieferketten sowie naturschutz. die damit verbundene verantwortung ist herausforderung und chance zugleich für eine lebenswerte, freiheitliche und solidarische gesellschaft und eine intakte umwelt für ein zukunftsfähiges und demokratisches europa. & 554 & very low & Low & Power & Socio-Economic & NA & 2018-01-08 & 2018 & 3 & POL
Frame & v.low & National & 500-1000 & -1.0405052 & -1.0641830 & 1.2185583 & -1.5781059 & 1.4491718 & 12.0 & 1.6295860 & 1.9019065 & Payer & Domestic & European & Mixed & Domestic|POL & Negative\\
Germany & https://web.de/magazine/politik/viktor-orban-kontert-spd-chef-martin-schulz-ungarn-verdient-respekt-32735400 & 257 & WEB.DE & Private/Non-Public & Online only & National & low = CP mentioned more times but NOT important part of story (mainly about others issues) & Political leverage & Balanced & EU + National & No myth & NA & NA & NA & NA & NA & NA & NA & NA & Germany & viktor orban kontert spd-chef martin schulz: ungarn verdient mehr respekt & 2018-01-07 & kohäsionsfonds & ungarns ministerpräsident viktor orban kontert vorwürfe von spd-chef schulz und will weiter daran festhalten, keine flüchtlinge in sein land zu lassen. der ungarische ministerpräsident viktor orban hat sich von spd-chef martin schulz "mehr respekt" für sein land erbeten. in anspielung auf schulz' früheres amt als präsident des europäischen parlaments sagte orban der "bild"-zeitung (montag): "was gut und nett in brüssel war - wo es keine offensichtlichen konsequenzen gab - ist eine andere geschichte, als in deutschland parteichef zu sein und mit anderen ländern zu kommunizieren. wir finden, wir verdienen mehr respekt." schulz hatte den csu-vorsitzenden horst seehofer aufgefordert, dem rechtsnationalen ungarischen regierungschef, der am freitag ehrengast der csu bei der winterklausur im oberbayerischen kloster seeon war, die grenzen aufzuzeigen. vor allem in der flüchtlingspolitik verfolge orban eine "gefährliche logik", hatte schulz kritisiert. "ich erwarte, dass herr seehofer ihm bei diesem thema und auch bei den themen presse- und meinungsfreiheit ganz klare grenzen aufzeigt." ungarn verweigert flüchtlingsaufnahme weiter orban verwahrte sich in dem interview gegen den vorwurf, ungarn nehme geld von der eu, weigere sich aber, flüchtlinge aufzunehmen. der sogenannte kohäsionsfonds, der der ungarischen wirtschaft zugutekomme, sei kein geschenk. "er ist ein fairer ausgleich, da wir unseren markt dem freien wettbewerb geöffnet haben. das hat absolut nichts mit der flüchtlingsfrage zu tun." orban bekräftigte, dass ungarn auch künftig keine flüchtlinge aufnehmen werde. "wir glauben, dass eine hohe zahl an muslimen notwendigerweise zu parallelgesellschaften führt", sagte er. "so etwas möchten wir nicht. und wir möchten uns nichts aufzwängen lassen." bildergalerie starten fotos aus aller welt - augenblicke, die ohne viele worte auskommen bewegend, schockierend, traurig, bezaubernd oder einfach nur schön: hier sehen sie - ständig aktualisiert - die besten fotos. © dpa & 278 & low & Low & Power & NA & NA & 2018-01-07 & 2018 & 3 & POL
Frame & low-medium & National & <500 & -1.0405052 & -1.0641830 & 1.2185583 & -1.5781059 & 1.4491718 & 12.0 & 1.6295860 & 1.9019065 & Payer & Domestic & European & Mixed & Domestic|POL & Neutral\\
\addlinespace
Germany & https://www.tagesschau.de/ausland/eu-finanzen-103.html & 244 & tagesschau.de & Public & Online and Offline & National & low = CP mentioned more times but NOT important part of story (mainly about others issues) & Political leverage & Negative & EU + Other country & 6.Does not defend EU values (eg.gender/law/democracy) & NA & NA & NA & NA & NA & NA & NA & NA & Germany & ungarn nennt eu-haushaltsplan "erpressung" & 2018-05-03 & kohäsionsfonds & nettoempfänger in der eu kritisieren die neuen eu-haushaltspläne - wenngleich aus unterschiedlichen gründen. die eu nimmt die kritik gelassen. dabei stehen harte verhandlungen ins haus. ungarn hat die von der eu-kommission geplante koppelung von geldzahlung an die einhaltung rechtsstaatlicher prinzipen als "erpressung" zurückgewiesen. außenminister peter szijjarto sagte, die verträge der eu beschrieben die rechte und pflichten der mitgliedstaaten genau. "wir stimmen keinem vorschlag zu, der im hinblick auf die auszahlung von eu-fonds, die den ländern aufgrund der verträge zustehen, die möglichkeit der erpressung einräumen würde." die eu-kommission wirft der rechtsnationalen regierung von ministerpräsident viktor orban vor, die unabhängigkeit der justiz zu untergraben und die pressefreiheit einzuschränken. ähnliche vorwürfe hinsichtlich der rechtsstaatlichkeit gibt es auch gegen polen. ungarn und polen gehören zu den größten netto-empfängerländern in der eu. geld gegen menschenrechte - mit dieser koppelung will sich die kommission für den fall rüsten, dass etwa mittel aus den strukturfonds in mitgliedsstaaten missbraucht werden. auch bulgarien lehnt die vorgeschlagene kürzung des kohäsionsfonds der eu scharf ab. der kohäsionsfonds wurde für eu-staaten eingerichtet, die mit einem bruttonationaleinkommen pro einwohner unter 90 \% des eu-durchschnitts eingerichtet. sein ziel ist der ausgleich der wirtschaftlichen und sozialen ungleichheit. "die europäische sicherheit und verteidigung sind absolute priorität, doch ohne kohäsion riskieren wir, die teilung innerhalb europas zu vertiefen", erklärte präsident rumen radew zum vorschlag der eu-kommission eines haushaltsrahmens für die jahre 2021 bis 2027. bulgarien gehört ebenfalls zu den nettoempfängern der eu und ist für den ausbau seiner infrastruktur auf den von der eu eingerichteten kohäsionsfonds angewiesen. neben empfängerländern kritisieren auch zahlerländer wie österreich die haushaltspläne scharf. die eu ist jedoch zuversichtlich, dass österreich den plänen am ende zustimmt. "wir sind überzeugt, dass für österreich die vorteile überwiegen", sagte der vertreter der eu in österreich, jörg wojahn. so werde nicht nur der dringende wunsch der alpenrepublik nach einem besseren eu-außengrenzschutz im budgetentwurf berücksichtigt. das land könne auch von einer neuerung bei den agrarsubventionen profitieren, meinte wojahn. die zahlungen an bauern sollen nach den plänen künftig bei 60.000 euro gedeckelt werden, die so eingesparten gelder könnten in nationaler regie auf kleinere bäuerliche betriebe verteilt werden. kanzler sebastian kurz hatte sich nach bekanntgabe der eu-pläne für sparmaßnahmen ausgesprochen. ziel müsse es sein, "dass die eu nach dem brexit schlanker, sparsamer und effizienter wird", schrieb kurz auf twitter. eu-kommissionspräsident jean-claude juncker und haushaltskommissar günther oettinger nehmen die kritik an ihren vorschlägen für die künftigen eu-finanzen demonstrativ gelassen auf. "das war immer so", sagte juncker zu dem widerstand aus ländern wie österreich und frankreich. juncker und oettinger hatten am mittwoch vorgeschlagen, trotz der verkleinerung der eu nach dem brexit den gemeinschaftshaushalt im nächsten jahrzehnt deutlich aufzustocken, aber dennoch die finanzhilfen für landwirte und strukturschwache regionen zu kürzen. & 457 & low & Low & Power & NA & NA & 2018-05-03 & 2018 & 3 & POL
Frame & low-medium & National & <500 & -1.0405052 & -1.0641830 & 1.2185583 & -1.5781059 & 1.4491718 & 12.0 & 1.6295860 & 1.9019065 & Payer & European & European & European & European|POL & Negative\\
Germany & https://www.presseportal.de/pm/133076/4202473 & 264 & presseportal.de & Private/Non-Public & Online only & National & very low = CP mentioned once & Economic development & Factual & Other country & No myth & NA & NA & NA & NA & NA & NA & NA & NA & Germany & führende kommunal- und regionalpolitiker: "harter" brexit schadet der lokalen wirtschaft & 2019-02-25 & kohäsionspolitik & kommunal- und regionalpolitiker gaben ihrer wachsenden besorgnis angesichts des möglichen austritts des vereinigten königreichs aus der europäischen union ohne abkommen. unter verweis auf die vom 22. februar veröffentlichte studie eines forschungskonsortiums unter federführung der universität von birmingham warnte der europäische ausschuss der regionen (adr), die versammlung der regional- und kommunalvertreter der eu, vor den verheerenden wirtschaftlichen und politischen auswirkungen eines ungeregelten brexits auf die lokale wirtschaft. adr-präsident karl-heinz lambertz: "es zeichnet sich immer deutlicher ab, dass ein austritt ohne abkommen riesige lokal- und regionalwirtschaftliche schäden in sowohl der eu als auch dem vereinigten königreich verursachen wird. die rechnung müssen dann die bürger bezahlen. die eu wird mittel zum schutz ihrer am meisten betroffenen regionen bereitstellen, aber die warnung ist unmissverständlich: die wirtschaft des vereinigten königreichs ist fast um ein fünffaches stärker vom brexit betroffen als die übrige europäische union. wir wollen alle einen geregelten brexit, und das ausgehandelte austrittsabkommen ist unvergleichlich viel besser als ein katastrophaler no-deal". françois decoster (alde/fr), vorsitzender der interregionalen gruppe brexit des adr, forderte jüngst die bereitstellung neuer eu-mittel im rahmen der regionalfonds, d. h. der eu-kohäsionspolitik, für die regionen an der eventuellen neuen außengrenze. die eu hat dem mittlerweile zugestimmt. hinsichtlich der besorgniserregenden aussichten eines ungeregelten brexit für die irische insel fügte präsident lambertz (be/spe) hinzu: "es ist an der zeit, die politischen machtspiele zu beenden und endlich den anliegen der bürger priorität einzuräumen. dazu ist es dringend erforderlich, die rechte aller bürger sicherzustellen und die gefährlichen folgen eines austritts ohne abkommen, der zu einer harten grenze auf der irischen insel führen würde, abzuwenden. das im november 2018 vereinbarte austrittsabkommen ist nach wie vor die bislang beste und einzige verhandlungslösung." michael murphy (ie/evp), mitglied des grafschaftsrats von tipperary, leiter der irischen delegation im adr und stellvertretender vorsitzender der interregionalen gruppe brexit, bekräftigte dies und fügte hinzu: "der brexit ist für alle beteiligten seiten ein großes verlustgeschäft, was durch die veröffentlichte studie bestätigt wird. wir können mit sicherheit sagen, dass die lokale und regionale ebene zuerst die auswirkungen des brexits spüren wird. der geplante brexit wirkt sich bereits auf meine region aus - eine ganze reihe von unternehmen aus der agrar- und die lebensmittelbranche müssen aufgrund der schwankenden wechselkurse des britischen pfunds schließen. die regionen mit einem besonders großen handelsvolumen mit dem vereinigten königreich werden besonders unter den konsequenzen zu leiden haben. ihre lokale wirtschaft steht untersuchungen zufolge in vielen fällen bereits auf schwachen füßen." in zwei politischen entschließungen vom märz 2017 und mai 2018 hat sich der adr für die sicherung des friedens und die vermeidung einer außengrenze zwischen irland und nordirland eingesetzt und die eu aufgefordert, sicherzustellen, dass die lokalen und regionalen gebietskörperschaften mit den auswirkungen des brexit nicht alleingelassen werden. der adr befürwortet eine künftige beziehung zwischen dem vereinigten königreich und der eu, die eine enge zusammenarbeit mit den regionen, städten, unternehmen und hochschulen in england, schottland, wales und nordirland ermöglicht. original-content von: europäischer ausschuss der regionen, übermittelt durch news aktuell & 496 & very low & Low & Socio-Economic & NA & NA & 2019-02-25 & 2019 & 3 & ECO
Frame & v.low & National & <500 & -1.0405052 & -1.0641830 & 1.2185583 & -1.5781059 & 1.4491718 & 12.0 & 1.6295860 & 1.9019065 & Payer & European & European & European & European|ECO & Neutral\\
Germany & http://www.sueddeutsche.de/politik/europaeische-union-zeiten-des-zweifels-1.3096887 & 286 & Süddeutsche Zeitung & Private/Non-Public & Online and Offline & National & very low = CP mentioned once & Political leverage & Negative & EU + Other country & No myth & NA & NA & NA & NA & NA & NA & NA & NA & Germany & europäische union - zeiten des zweifels & 2016-07-27 & kohäsionsfonds & brüssel will doch keine geldstrafen gegen die defizitsünder spanien und portugal verhängen. man wolle das vertrauen der menschen in die eu nicht erschüttern. wie viel ist der stabilitätspakt noch wert? trotz anhaltender verstöße gegen die europäischen defizitregeln sollen portugal und spanien nun doch straffrei davonkommen. die eu-kommission beschloss am mittwoch, keine bußen zu fordern. die maximal mögliche strafe wären - wenn die euro-finanzminister zugestimmt hätten - 0,2 prozent der wirtschaftsleistung gewesen: im falle spaniens 2,2 milliarden euro, bei portugal etwa 360 millionen. als grund für die milde nannte eu-währungskommissar pierre moscovici zum einen die schwierige wirtschaftliche lage der beiden länder, die schwere krisen hinter sich hätten und bedeutende anstrengungen unternommen hätten, um wieder auf die beine zu kommen. ebenso wichtig aber sei die derzeitige politische lage in europa. "strafen auszusprechen erschien uns nicht als angemessen in einer zeit, in der die völker an europa zweifeln", sagte moscovici. zum ersten mal in der geschichte der währungsunion hatten die euro-finanzminister mitte juli bußgeldverfahren gegen zwei euro-staaten gestartet. zuvor hatte die kommission festgestellt, dass spanien und portugal anhaltend gegen die regeln des stabilitätspakts verstießen. spaniens defizit liegt schon seit der finanzkrise vor acht jahren über der zugelassenen grenze von drei prozent des bruttoinlandsprodukts, doch erhielt das land immer neue schonfristen von der eu. 2015 sollte es das defizit auf 4,2 prozent senken, stattdessen landete es bei 5,1 prozent, nicht zuletzt wegen steuersenkungen vor der wahl im dezember. im laufenden jahr sowie vermutlich auch im kommenden jahr werden die vorgaben nicht eingehalten. das strukturelle, um konjunktureinflüsse bereinigte defizit konnte zwischen 2013 und 2015 nur um 0,6 prozent statt wie von der eu gefordert um 2,7 prozent verringert werden. portugal wiederum verzeichnete trotz gegenteiliger versprechen 2015 ein defizit von 4,4 prozent. im laufenden jahr soll es auf 2,2 prozent sinken, allerdings hinkt portugal auch bei der reduktion des strukturellen defizits weit hinter den zielen her. beide länder hatten sich gegen sanktionen verwahrt und besserung versprochen. während madrid zusätzliche "fiskalische maßnahmen" ankündigte, verwies lissabon auf eine reserve von 350 millionen euro, mit denen das ziel erreicht werde. kritiker monieren seit jahren, die eu-defizitregeln seien nur glaubwürdig, wenn sie auch konsequent angewandt und nicht jedesmal von politischen überlegungen und machtverhältnissen überlagert würden. deutschland verstieß in den jahren nach der einführung des euro insgesamt sieben mal gegen die defizitgrenze, von 2001 bis 2005 sogar fünf jahre in folge. eine strafe wurde nach politischem druck aus berlin niemals ausgesprochen. auch frankreich bricht seit jahren die regeln und müsste bei strenger auslegung bestraft werden. eu-kommissionspräsident jean-claude juncker begründete die zurückhaltung jüngst mit den worten: "weil es frankreich ist". in brüssel wurde am mittwoch spekuliert, bundesfinanzminister wolfgang schäuble (cdu), der harte sanktionen zum gegenwärtigen zeitpunkt für kontraproduktiv hält, könnte druck auf die eu-kommission ausgeübt haben. "ideal" sei die jetzige lösung nicht, räumte moscovici ein. aber es sei die wirtschaftlich wie politisch weiseste. spanier und portugiesen hätten viele opfer gebracht in den vergangenen jahren. man wolle das vertrauen der menschen nicht zerstören. selbst symbolische sanktionen "wären von der öffentlichkeit nicht verstanden worden". man müsse auch sehen, dass die regeln wirkten, dass sie das verhalten der regierungen in die richtige richtung gelenkt hätten. "der stabilitätspakt ist effizient, er ist nicht dumm", so der franzose, der wiederholt für eine "intelligente" auslegung des regelwerks plädiert hat. bis zum 15. oktober sollen spanien und portugal nun neue pläne präsentieren, wie sie ihre haushalte in ordnung bringen wollen. sie bekamen dafür neue fristen gesetzt. im fall von portugal erwartet die eu-kommission, dass das land sein haushaltsdefizit bis ende des jahres auf 2,5 prozent der wirtschaftsleistung drückt. spanien soll bis ende 2018 schrittweise auf 2,2 prozent kommen. gelingt das nicht, könnten eu-fördermittel aus dem kohäsionsfonds gekürzt werden. die kommission kündigte dazu einen "rigorosen" vorschlag an. darüber will sie im september aber erst mit vertretern des europaparlaments beraten. die entscheidung der kommission sei eine gute nachricht für sein land, aber auch für den "europäischen geist", sagte der portugiesische außenminister augusto santos silva. der eu-abgeordnete fabio di masi (linke) lobte den "realismus" der brüsseler behörde und fügte hinzu: "strafen für vermeintliche defizitsünder wären ohnehin so absurd, wie koma-patienten blut abzuzapfen." & 701 & very low & Low & Power & NA & NA & 2016-07-27 & 2016 & 2 & POL
Frame & v.low & National & 500-1000 & -1.0405052 & -1.0641830 & 1.2185583 & -1.5781059 & 1.4491718 & 12.0 & 1.6295860 & 1.9019065 & Payer & European & European & European & European|POL & Negative\\
Germany & https://www.hna.de/politik/berlin-will-eu-mittelvergabe-an-rechtsstaatliche-prinzipien-knuepfen-zr-8366308.html & 254 & Hessisch Niedersachsische Allgemeine & Private/Non-Public & Online and Offline & Regional/Local & low = CP mentioned more times but NOT important part of story (mainly about others issues) & Political leverage & Negative & EU & No myth & NA & NA & NA & NA & NA & NA & NA & NA & Germany & keine rechtsstaatliche prinzipien? keine eu-fördermittel & 2017-05-31 & kohäsionsfonds & keine rechtsstaatliche prinzipien? keine eu-fördermittel die bundesregierung will die vergabe von fördermitteln in der eu künftig von der achtung rechtsstaatlicher prinzipien und der grundrechte in den mitgliedstaaten abhängig machen. berlin - das wirtschaftsministerium in berlin bestätigte am mittwoch, dass die regierung in einer stellungnahme zur zukunft der eu-kohäsionsfonds eine "bindung an die einhaltung der rechtsstaatlichen grundwerte der eu" befürwortet. die stellungnahme werde "in kürze" an die eu-kommission geschickt, sagte eine ministeriumssprecherin. die bundesregierung nehme damit eine debatte auf, die im europaparlament und in der eu-kommission schon länger geführt werde. auch eu-justizkommissarin vera jourova hatte sich für die verknüpfung von mittelvergabe und achtung rechtsstaatlicher prinzipien ausgesprochen. länder wie polen oder ungarn, die aus sicht der eu demokratische grundwerte nicht einhalten, müssten dann mit mittelkürzungen rechnen. verknüpfung mit rechtsstaatlichen prinzipien sei "sehr vernünftig" die stellungnahme der bundesregierung bezieht sich auf die nächste siebenjährige haushaltsperiode von 2021 bis 2027. das vom wirtschaftsministerium erarbeitete papier wurde in den vergangenen monaten zwischen den ressorts abgestimmt. die eu-kommission soll nach dem wunsch berlins die einführung einer neuen konditionalität mit blick auf die nächste haushaltsperiode prüfen. die verknüpfung mit rechtsstaatlichen prinzipien sei "sehr vernünftig", sagte die sprecherin des wirtschaftsministeriums. die eu zeichne sich vor allem dadurch aus, dass sie eine wertegemeinschaft sei. "die basis einer solchen wertegemeinschaft und die basis der glaubwürdigkeit der eu ist auch die einhaltung der grundwerte." polen ist der mit abstand größte empfänger von mitteln aus dem eu-kohäsionsfonds, der einen ausgleich zwischen reicheren und ärmeren staaten schaffen soll. für das land sind in der siebenjährigen eu-haushaltsperiode von 2014 bis 2020 rund 23,2 milliarden euro vorgesehen - mehr als ein drittel aller mittel. bei ungarn sind es gut sechs milliarden euro. afp rubriklistenbild: © dpa & 287 & low & Low & Power & NA & NA & 2017-05-31 & 2017 & 2 & POL
Frame & low-medium & Regional & <500 & -1.0405052 & -1.0641830 & 1.2185583 & -1.5781059 & 1.4491718 & 12.0 & 1.6295860 & 1.9019065 & Payer & European & European & European & European|POL & Negative\\
Germany & http://www.spiegel.de/politik/ausland/olaf-scholz-als-finanzminister-was-kommt-da-auf-die-eu-zu-a-1192263.html & 225 & SPIEGEL ONLINE & Private/Non-Public & Online and Offline & National & low = CP mentioned more times but NOT important part of story (mainly about others issues) & Institutional bargaining over funding & Positive & National & No myth & NA & NA & NA & NA & NA & NA & NA & NA & Germany & eu zur groko-einigung: olaf - wer? - spiegel online - politik & 2018-02-07 & kohäsionspolitik & die erleichterung in brüssel ist groß: mehr als vier monate sind seit der bundestagswahl vergangen, und endlich ist klar, wie es in berlin weitergehen wird. "eine gute nachricht für europa!", twitterte eu-finanzkommissar pierre moscovici. er habe "großen respekt" vor der konstruktiven haltung der spd. manfred weber (csu), chef der evp-fraktion im europaparlament, bezeichnete die einigung als "gutes signal an das volk und ganz europa". die künftige bundesregierung sei bereit, "zu einem stärkeren und besseren europa beizutragen". dies sei ein "eindeutig pro-europäischer ansatz und eine antwort an die populisten". tatsächlich liest sich das europakapitel im koalitionsvertrag an manchen stellen noch europafreundlicher als die entsprechenden passagen im sondierungspapier von spd und union. eu-freundlich liest sich zunächst auch eine neue passage zur kohäsionspolitik der eu, die unterschiede zwischen armen und reichen mitgliedsländern ausgleichen soll. "wir wollen die wichtigen strukturfonds der eu erhalten", heißt es - auch wenn der austritt großbritanniens ein loch von bis zu 14 milliarden euro pro jahr in den eu-haushalt reißt. die koalitionäre fordern "eine starke eu-kohäsionspolitik in allen regionen". die eurozone "nachhaltig und stärken und reformieren" das heißt allerdings auch: nicht nur ost- oder südeuropa sollen geld bekommen, sondern auch nettozahler wie etwa deutschland. im aktuellen mehrjahreshaushalt der eu, der von 2014 bis 2020 läuft, stehen deutschland rund 19 milliarden euro aus dem struktur- und investitionsfonds zu, mehr als die hälfte davon aus dem topf für regionale entwicklung. die eu-kommission aber spielt derzeit mit dem gedanken, reicheren ländern überhaupt keine kohäsionsmittel mehr zu zahlen. in brüssel rechnet man mit heftiger gegenwehr der deutschen bundesländer - und der koalitionsvertrag lässt ahnen, dass sie auf die hilfe berlins hoffen dürfen. überhaupt, das geld: die deutsche bereitschaft, künftig mehr in den eu-haushalt einzuzahlen, steht weiterhin im vertrag, ebenso wie der investivhaushalt für die eurozone, die man "nachhaltig stärken und reformieren" wolle. allerdings betonen die koalitionäre nun: "die rechte der nationalen parlamente bleiben davon unberührt." das heißt: wie viel geld deutschland letztlich zahlt, entscheidet der bundestag. die passage soll auf druck der union hinzugekommen sein. wie wird sich olaf scholz auf eu-ebene verhalten? derzeit überweisen die mitgliedsländer ein prozent ihres bruttoinlandprodukts nach brüssel. dass berlin künftig mehr zahlt, hält man zumindest in der spd für ausgemacht. "klar ist, dass es nicht bei einem prozent bleiben kann", sagt jens geier, vorsitzender der europa-spd. "das deutsche 'wir geben nichts'-dogma ist ein stück weit aufgelöst." das allerdings bedeute keineswegs, dass nur deutschland mehr einzahlen werde, betont der cdu-europapolitiker elmar brok. "wenn der eu-haushalt aufgestockt wird, müssen alle mitgliedsländer mehr einzahlen." zugleich wollen spd und union den kampf gegen steuervermeidung verschärfen. "wenn der deutsche finanzminister im rat entsprechend vorgeht, dann ändern sich die zeiten", sagt geier. dieser job fällt nun olaf scholz zu, der künftig das finanzressort leiten soll. in sachen eu-politik aber ist der spd-mann noch unerfahren. als hamburgs erster bürgermeister musste scholz sich um dieses thema bisher kaum kümmern - wie er in brüssel vorgehen wird, weiß so genau niemand . "wofür er europapolitisch steht, weiß ich nicht", sagt etwa der grünen-europaabgeordnete sven giegold. wie groß die bandbreite der erwartungen in brüssel ist, zeigen die reaktionen von fdp und linkspartei. mit ihrem beschluss, den eu-haushalt aufstocken zu wollen, "macht sich die groko zur komplizin der linksregierung in griechenland", sagt wolf klinz, wirtschaftspolitischer sprecher der fdp im europaparlament. "deutschland wird vom zuchtmeister zum zahlmeister europas." der linken-bundestagsabgeordnete fabio de masi ist vom gegenteil überzeugt: "die von der spd behauptete abkehr vom kürzungszwang der eu ist fake news." der stabilitäts- und wachstumspakt werde nicht in frage gestellt, die mitgliedsstaaten seien weiterhin zur kürzung von investitionen und sozialstaat gezwungen. unklar bleibt im koalitionsvertrag unterdessen die deutsche position zu den wirklich großen fragen rund um europas zukunft. "zu den reformvorschlägen macrons, einen eu-finanzminister oder transnationale listen bei der nächsten europawahl einzuführen, steht im vertrag kein wort", kritisiert giegold. das gleiche gelte für die die bankenunion oder mehr transparenz im finanzgebahren von großunternehmen. "dieser vertrag", so giegold, "sieht aus, als stammte er aus der zeit vor der krise." & 673 & low & Low & Power & NA & NA & 2018-02-07 & 2018 & 3 & POL
Frame & low-medium & National & 500-1000 & -1.0405052 & -1.0641830 & 1.2185583 & -1.5781059 & 1.4491718 & 12.0 & 1.6295860 & 1.9019065 & Payer & Domestic & Domestic & Domestic & Domestic|POL & Positive\\
\addlinespace
Germany & http://www.dw.com/de/visegrad-staaten-f\%C3\%BCr-ein-besseres-europa/a-37788500 & 287 & Deutsche Welle (English) & Private/Non-Public & Online and Offline & National & very low = CP mentioned once & Solidarity to poor countries/regions & Positive & EU + Other country & No myth & NA & NA & NA & NA & NA & NA & NA & NA & Germany & visegrad-staaten: "für ein besseres europa" | europa | dw.com | 02.03.2017 & 2017-03-02 & kohäsionspolitik & so geschlossen hat man die visegrad-staaten selten erlebt. auf ihrem gipfel in warschau, das derzeit in der gruppe den vorsitz führt, richteten die vier staaten (polen, ungarn, tschechien, slowakei) klare forderungen an die übrigen eu-mitglieder. geschlossen will die gruppe auch in den eu-gipfel ende märz in rom ziehen. der premier der slowakei, robert fico, äußerte eine scharfe warnung: "der stand der vorbereitungen des gipfels in rom, der so wichtig ist für die zukunft der eu, ist jämmerlich." es drohe eine begegnung, bei der am ende "keine vision eines europas der zukunft, sondern eine ansammlung individueller, nationaler interessen herauskommt". die gastgeberin, polens regierungschefin beata szydlo, hielt eine detaillierte erklärung vor die kameras, die in mehreren punkten die positionen der visegrad-gruppe zusammenfasst. eu-ratschef donald tusk solle auf dieser grundlage vorschläge für rom erarbeiten. "es geht nicht um mehr oder weniger europa, sondern um ein besseres europa", sagte szydlo. "für rechtsstaat, gegen protektionismus" in dem text nennen die vier staaten die eu eindeutig das "beste instrument, um die vor uns liegenden herausforderungen zu meistern". sie bekennen sich auch zu ihren grundwerten wie demokratie und rechtsstaat und wenden sich gegen "protektionismus, sowohl innerhalb wie außerhalb der eu" - offenbar eine anspielung auf die usa unter donald trump. außerdem solle die eu offen bleiben für "jene länder, die diese werte teilen, vor allem die länder des westlichen balkans und unsere östlichen nachbarn". auch das thema "europa der zwei geschwindigkeiten" kam zur sprache. innerhalb der eu solle jede form einer engeren zusammenarbeit "für jeden mitgliedsstaat offen stehen, um jegliche desintegration des binnenmarktes, des schengen-raums und der eu selbst absolut zu vermeiden", heißt es im text. es gebe noch "großes potenzial" für eine vertiefung des binnenmarktes, etwa in energiefragen, im digitalen bereich und bei der freizügigkeit der dienstleistungen. außerdem müssten die nationalen parlamente und der europäische rat, also die vertretung der regierungen, gestärkt werden - was offenbar zugleich heißt, dass die eu-kommission geschwächt werden soll. "lebensmittel zweiter klasse" für osteuropa? ungarns premier viktor orban stützt nach eigenen worten "zu hundert prozent" die gemeinsamen visegrad-positionen. ein weiteres thema des warschauer treffens war die unterschiedliche qualität von lebensmitteln in europa. laut untersuchungen und medienberichten bringen konzerne in den neuen eu-ländern unter gleichem namen schlechtere produkte auf den markt als in den alten. diese produktion "mit zweierlei maß", so orban, müsse ein ende haben. szydlo sagte, man habe eine gemeinsame arbeitsgruppe zu dem thema gebildet und werde es auf eu-ebene ansprechen. zugleich trafen sich in warschau die finanzminister der visegrad-gruppe und ihre kollegen aus südosteuropa sowie eu-kommissarin corina cretu. die minister lobten die eu-kohäsionspolitik, die auch im neuen finanzrahmen ab 2020 fortgesetzt werden müsse. experte: immer engere zusammenarbeit in visegrad-gruppe offenbar findet die oft unterschätzte visegrad-gruppe nach und nach zu einer engeren zusammenarbeit. jakub groszkowski, experte am zentrum für oststudien (osw) in warschau, sagte im dw-gespräch, die visegrad-staaten würden oft irrtümlich als "monothematischer verein" gesehen, etwa in der flüchtlingsfrage. "aber visegrad ist inzwischen viel mehr. die gruppe ist ein nützliches instrument für austausch und kooperation innerhalb von eu und nato. und es gibt viele formen ihrer zusammenarbeit, alle ministerien in diesen ländern und weitere institutionen treffen sich regelmäßig und verwirklichen gemeinsame projekte. sie unterstützen gemeinsam die sanktionen gegen russland, sie helfen auch gemeinsam der ukraine." die gruppe akzeptiere ein "europa verschiedener geschwindigkeiten, das es ja längst gibt". mit ihrer erklärung richte sie sich aber gegen "abgeschottete, exklusive clubs" innerhalb der gemeinschaft und wolle eine gemeinsame richtung in der entwicklung der eu-staaten erhalten. nur bei einem thema wurden differenzen deutlich: eine neue amtszeit für den eu-ratspräsidenten donald tusk. in polen gilt tusk als der gefährlichste politische und persönliche gegner des chefs der polnischen regierungspartei, jaroslaw kaczynski, der gegen eine neue amtszeit ist. "in dieser frage haben wir in der visegrad-gruppe keine gemeinsame position", sagte der tschechische premier bohuslav sobotka in warschau. verwirrung um gegenkandidaten zu tusk in dieser woche hatte polens regierung überraschend einen gegenkandidaten für das hohe amt aus dem hut gezaubert: jacek saryusz-wolski. der erfahrene europa-abgeordnete gehört paradoxerweise der bürgerplattform an, also derselben partei wie tusk. dass er offenbar einverstanden war, gegen seinen parteifreund als kandidat aufgestellt zu werden, führen kenner der materie darauf zurück, dass sich saryusz-wolski "innerlich schon lange von seiner partei entfernt" habe. noch verwirrender war, dass sich saryusz-wolski seit drei tagen weigert, seine kandidatur zu bestätigen oder zu dementieren. am donnerstag gelang es einem journalisten des polnischen senders tvn24, gemeinsam mit saryusz-wolski in straßburg einen fahrstuhl zu besteigen. auch jetzt verweigerte der politiker jede antwort und sagte nur: "journalisten kennen doch so einen begriff: no comment." & 774 & very low & Low & Values & NA & NA & 2017-03-02 & 2017 & 2 & ECO
Frame & v.low & National & 500-1000 & -1.0405052 & -1.0641830 & 1.2185583 & -1.5781059 & 1.4491718 & 12.0 & 1.6295860 & 1.9019065 & Payer & European & European & European & European|ECO & Positive\\
Spain & http://www.lne.es/cuencas/2017/05/25/langreo-dispondra-fondos-europeos-recuperar/2110264.html & 655 & lne.es & Private/Non-Public & Online and Offline & Regional/Local & medium = CP is important part of story & Institutional bargaining over funding & Negative & EU + National + Subnational & No myth & NA & NA & NA & NA & NA & NA & NA & NA & Spain & langreo no dispondrá de fondos europeos para recuperar los terrenos de nitrastur & 2017-05-25 & fondo europeo de desarrollo regional & el proyecto de desarrollo urbano presentado por el ayuntamiento no logra la puntuación suficiente para recibir los 2,7 millones solicitados el ayuntamiento de langreo no recibirá fondos europeos para desarrollar el proyecto de desarrollo urbano que incluía actuaciones como la recuperación de los terrenos de nitrastur para crear un espacio verde, de ocio y deportivo. el programa presentado por el consistorio a la convocatoria de las estrategias de desarrollo urbano sostenible integrado (dusi), que tiene financiación del fondo europeo de desarrollo regional (feder) de la unión europea (ue), no ha recibido el visto bueno de la secretaría de estado de presupuestos y gastos del ministerio de hacienda. el proyecto langreano no obtuvo la puntuación suficiente para conseguir la financiación necesaria. el importe total de las actuaciones es de diez millones de euros aunque la cuantía a subvencionar con los fondos europeos no superaría los 2,7 millones, a lo que se sumaría alrededor de un millón del ayuntamiento. un municipio asturiano, oviedo, recibió el visto bueno, con 10,3 millones de euros. siero y gijón tampoco dispondrán de ayudas europeas para desarrollar sus proyectos. el plan langreano fue apoyado en el pleno municipal por todos los grupos (iu, somos, psoe, pp y ciudadanos). en el plan se incluían propuestas trasladadas en los contactos que el gobierno local mantuvo con los grupos de la oposición, asociaciones, alcaldes de barrio y empresarios. entre las actuaciones previstas figuraba la recuperación de los terrenos de nitrastur para convertirlos en un espacio verde, de ocio y deportivo. estos terrenos, ubicados entre la felguera y barros, tienen una superficie de 200.000 metros cuadrados y son propiedad, en gran parte -unos 122.000 metros cuadrados- de iberdrola, que los adquirió en subasta pública en 2001. el ayuntamiento de langreo apostaba además por la conexión de los polígonos de la moral y valnalón, que permitiría generar un nuevo espacio industrial en la zona, junto con la actuación sobre áreas abandonadas en el entorno urbano para disponer de nuevos espacios de ocio y aparcamiento. también se planteó la eliminación de barreras arquitectónicas en espacios públicos y en edificios de propiedad municipal así como la instalación de ascensores y la construcción de sendas peatonales que unan el entorno urbano con el medio rural. las actuaciones alcanzarían la modernización de la gestión de los museos mediante la instalación de pantallas interactivas de información sobre el equipamiento cultural y el municipio para fomentar la visita de los turistas a otros activos del municipio. también se construirían sendas peatonales que unan el entorno urbano con el medio rural. todos estas actuaciones están incluidas en el proyecto elaborado por el ayuntamiento. & 439 & medium & Medium & Power & NA & NA & 2017-05-25 & 2017 & 2 & POL
Frame & low-medium & Regional & <500 & 0.0141942 & -0.1980960 & -0.1536394 & -0.9382144 & 0.7927531 & 12.6 & 1.2981504 & 0.5794339 & Recipient & Domestic & European & Mixed & Domestic|POL & Negative\\
Spain & http://sevilla.abc.es/internacional/abci-bruselas-y-berlin-piden-pagar-mas-para-cubrir-agujero-brexit-201802230246\_noticia.html & 600 & ABC de Sevilla & Private/Non-Public & Online and Offline & Regional/Local & low = CP mentioned more times but NOT important part of story (mainly about others issues) & Institutional bargaining over funding & Negative & EU + Other country & No myth & NA & NA & NA & NA & NA & NA & NA & NA & Spain & bruselas y berlín piden pagar más a la ue para cubrir el agujero del brexit & 2018-02-23 & fondos estructurales & los jefes de estado y de gobierno de los 27 países que permanecerán en la ue después de la salida de los británicos se reúnen este viernes en bruselas para analizar el futuro de la unión. y en el primer punto de la discusión figura el hecho de que la salida de un país contribuyente neto va a dejar un hueco en el presupuesto comunitario. las negociaciones para el próximo periodo presupuestario, que van a empezar en los próximos meses, amenazan con ser más complejas que las del propio brexit. el agujero que deja la salida del reino unido se calcula en unos 10.000 millones de euros anuales. y eso sucede en un momento en el que la propia unión ha emprendido una ambiciosa carrera para actuar en otros campos en los que hasta ahora no se había interesado, como la defensa, la protección de las fronteras exteriores y una verdadera política de seguridad. la conclusión es clara: si hay menos ingresos y se quiere actuar en políticas nuevas, será imprescindible que haya recortes en las principales ayudas comunitarias: en la política agrícola común (pac) y en los fondos de cohesión, que suman tradicionalmente dos tercios del presupuesto comunitario. alemania encabeza la lista de los países que están de acuerdo en que todos aporten más (y acepten recibir menos) para mantener la misma ambición financiera para la ue y no transmitir a las políticas europeas la sensación de desánimo que puede generar el brexit. pero hay otros gobiernos que insisten en que hay que actuar de forma similar a como se hizo con el tamaño del parlamento europeo, en el que el brexit servirá para reducir el número de diputados. es decir, prefieren que la reducción del tamaño signifique también una reducción paralela en las ambiciones del proyecto europeo. incluso hay propuestas, animadas por países como holanda, para introducir elementos de condicionalidad a la hora de beneficiarse de las ayudas europeas, como ha sucedido con los países rescatados. el presidente del consejo europeo, donald tusk, que viene de polonia, el país que en estos momentos es el más beneficiado por los fondos estructurales de cohesión, ha pedido a los gobiernos nacionales que empiecen cuanto antes la discusión sobre plazos y objetivos, para evitar las tradicionales negociaciones de madrugada a cara de perro a la hora de aprobar el próximo periodo plurianual de presupuestos que comienza en 2020. el parlamento europeo, por su parte, aprobó este jueves mismo un informe en el que reclaman aumentar el presupuesto. según los cálculos de los eurodiputados, los límites máximos de gastos del marco presupuestario deberían elevarse al 1,3 por ciento del pib de la ue, frente al 1 por ciento actual, que ya es una reducción respecto al porcentaje precedente. la comisión europea también pide elevar la contribución de los países miembros. ha adelantado su propuesta legislativa, que será hecha pública el próximo 2 de mayo, y que prevé recortes en los fondos agrícolas y de cohesión, pero al mismo tiempo pide a los estados una subida "moderada" en sus aportaciones, que estaría entre el 1,1\% y el 1,2\% para poder actuar en nuevas políticas. los gobiernos están en una situación perversa, porque unos creen que aumentar el presupuesto comunitario puede aumentar el sentimiento antieuropeo, mientras que otros consideran que no hacerlo sería aún más perjudicial para animar a los euroescépticos. tusk y el responsable de los presupuestos en la comisión, el alemán gunter oettinguer, creen que las capitales deberían hacer un esfuerzo excepcional para cerrar las negociaciones antes de las elecciones de mayo del año que viene. para españa la situación es relativamente compleja. antes de la crisis estaba claro que los números de nuestra economía nos situaban por encima de la renta media de la ue, por lo que dejábamos de estar entre los países beneficiarios. la crisis nos ha mantenido con un saldo neto positivo, pero todo indica que después de la salida del reino unido es muy posible que volvamos a una situación de contribuyente neto. el gobierno cree que es posible buscar un equilibrio, es decir, aumentar las contribuciones nacionales, de manera que no se pierdan los fondos de cohesión y que, como dijo una fuente diplomática española "todos los ciudadanos mantengan la posibilidad de ver los efectos benéficos de europa en su vida". & 719 & low & Low & Power & NA & NA & 2018-02-23 & 2018 & 3 & POL
Frame & low-medium & Regional & 500-1000 & 0.0141942 & -0.1980960 & -0.1536394 & -0.9382144 & 0.7927531 & 12.6 & 1.2981504 & 0.5794339 & Recipient & European & European & European & European|POL & Negative\\
Spain & http://www.rtve.es/alacarta/videos/noticias-de-extremadura/ext-20160309ext2/3516655/ & 632 & RTVE.es & Public & Online only & National & very low = CP mentioned once & Jobs & Positive & National & No myth & NA & NA & NA & NA & NA & NA & NA & NA & Spain & noticias de extremadura 2 - 09/03/16 & 2016-03-09 & fondo social europeo & extremadura dispondrá de más 330 millones de euros del fondo social europeo. es el 7 por ciento del montante que llegará a españa de ese programa operativo hasta 2020, prorrogable hasta 2023. serán 131 millones para educación y formación. 128 para fomentar el empleo de calidad. y 70 millones para inclusión social. la junta incide en los recursos para promover la contratación indefinida... el constitucional anula la prohibición de que los ayuntamientos ofrezcan servicios sociales y sanitarios, pero avala su fusión. da la razón así, parcialmente, al recurso presentado por extremadura contra la ley de la administración local. el psoe espera que sirva para enterrar la norma. el pp insiste en la duplicidad de competencias... informar en positivo para prevenir. es el objetivo del blog 'cómete el mundo tca'. un punto de encuentro para profesionales, familiares y afectados por trastornos alimentarios. y una reacción contraria a toda la información que existe en la red, que puede acabar en anorexia o bulimia... televisión española emitirá en directo tres procesiones de la semana santa de cáceres. la de la vera cruz y la del amor, el jueves santo. y el santo entierro el viernes santo. hoy se ha presentado la guía con la información de todos los itinerarios... & 206 & very low & Low & Socio-Economic & NA & NA & 2016-03-09 & 2016 & 2 & ECO
Frame & v.low & National & <500 & 0.0141942 & -0.1980960 & -0.1536394 & -0.9382144 & 0.7927531 & 12.6 & 1.2981504 & 0.5794339 & Recipient & Domestic & Domestic & Domestic & Domestic|ECO & Positive\\
Spain & http://www.periodistadigital.com/america/cultura/2015/04/13/ninos-gitanos-reaccionan-diccionario-rae.shtml & 629 & Periodista digital & Private/Non-Public & Online only & National & very low = CP mentioned once & Social awareness/inclusion & Positive & National + Subnational & No myth & NA & NA & NA & NA & NA & NA & NA & NA & Spain & [vídeo] así reaccionan estos niños gitanos al leer su 'definición' en el diccionario de la rae & 2015-04-13 & fondo social europeo & este proyecto ha sido financiado por el ministerio de sanidad, servicios sociales e igualdad español y el fondo social europeo no deja uno de sentir un escalofrío, entre de vergüenza y dolor, al observar a estos niños de raza gitana. "no me gusta que digan esto en los diccionarios [...] me están insultando". el consejo estatal del pueblo gitano de españa ha lanzado una conmovedora campaña para que la real academia española (rae) cambie en sus diccionarios una definición "discriminatoria" del término 'gitano'. este proyecto ha sido financiado por el ministerio de sanidad, servicios sociales e igualdad español y el fondo social europeo. & 102 & very low & Low & Socio-Economic & NA & NA & 2015-04-13 & 2015 & 1 & ECO
Frame & v.low & National & <500 & 0.0141942 & -0.1980960 & -0.1536394 & -0.9382144 & 0.7927531 & 12.6 & 1.2981504 & 0.5794339 & Recipient & Domestic & Domestic & Domestic & Domestic|ECO & Positive\\
\addlinespace
Spain & http://sevilla.abc.es/andalucia/20150427/sevp-ayudas-formacion-moldes-prensar-20150426.html & 642 & ABC de Sevilla & Private/Non-Public & Online and Offline & Regional/Local & medium = CP is important part of story & Fraud/Corruption & Negative & Subnational & No myth & NA & NA & NA & NA & NA & NA & NA & NA & Spain & subvenciones de formación para hacer moldes de «prensar cocaína» & 2015-04-27 & fondo social europeo & centro de formación en málaga que impartió el curso de soldador una "chapuza". así resumió un alumno su experiencia en un curso de soldador de estructuras metálicas ligeras impartido en 2011 por el centro de formación profesional ocupacional feijo sl, una empresa radicada en la capital malagueña que recibió una subvención de 131.625 euros de la junta de andalucía. éste es uno de los cursos de formación profesional con compromiso de contratación que están bajo sospecha. en un demoledor informe remitido a la fiscalía y al juzgado de instrucción 8 de málaga, la unidad central de delincuencia económica y fiscal (udef) de la policía denuncia el "incumplimiento sistemático" de los requisitos legales en el manejo de los fondos por parte de empresas beneficiarias de fondos públicos y describe un extenso catálogo de irregularidades que abarcan desde la confección de facturas y contratos ficticios, la falsificación de los partes de asistencia de alumnos, la suplantación de monitores y "algún tipo de tráfico de influencias" entre altos cargos autonómicos y empresarios beneficiados. los testimonios de alumnos (la policía ha entrevistado a más de 3.000 de diferentes cursos) han sido claves para arrojar luz sobre la investigación de los agentes, que mostraron su sorpresa por las "inexistentes" labores de control e inspección ejercidas por la administración. del descontrol instalado en la gestión de las ayudas dan prueban los encargos que hacían a los participantes durante su formación. algunos tan inimaginables como hacer "una especie de moldes para prensar cocaína, de tres tamaños" que desaparecieron del taller una vez terminados, según relató j.l.f.l., uno de los jóvenes que se instruyó como soldador en la sede de la academia en el polígono santa cruz de málaga. así lo recoge el acta de la declaración incorporada del sumario del caso de formación. durante las horas lectivas (975) del curso cofinanciado con el fondo social europeo, que se desarrolló entre marzo y diciembre de 2011, aseguró que fueron obligados a realizar obras de albañilería dentro de la nave del taller que no guardaban relación con la materia e incluso los mandaron a pintar la nave y maquinaria "con vista a la realización de futuros cursos". "se lo llevó calentito" en su declaración policial el 8 de abril de 2014, contó que el dueño del centro "organizaba barbacoas en su domicilio y en la misma empresa fanfarroneando de que pagaba la junta de andalucía". el empresario blasonaba de "haber recibido unos 240.000 euros" de la junta, afirmó. en realidad, fue beneficiado con 133.515 euros en 2009 y 131.625 al año siguiente, más de 265.000 euros en total. "se lo han llevado calentito", concluye este alumno, quien un día vio por la academia a una persona que podría ser un inspector de la junta, pero que "se marchó sin ver las dependencias", precisó. actualmente la empresa no tiene actividad. de hecho, la nave está en venta. otro alumno, r.a.p., corroboró a la policía que durante la formación les encomendaban los trabajos que entraban en la carpintería "como si fuesen trabajadores asalariados de la empresa". "el curso era una chapuza y no se aprendía apenas nada", afirma. & 528 & medium & Medium & Governance & NA & NA & 2015-04-27 & 2015 & 1 & POL
Frame & low-medium & Regional & 500-1000 & 0.0141942 & -0.1980960 & -0.1536394 & -0.9382144 & 0.7927531 & 12.6 & 1.2981504 & 0.5794339 & Recipient & Domestic & Domestic & Domestic & Domestic|POL & Negative\\
Spain & http://ecodiario.eleconomista.es/espana/noticias/8393936/05/17/Aspaym-destaca-en-la-campana-Yo-soy-los-beneficios-que-la-contratacion-de-discapacitados-puede-tener-para-las-empresas.html & 649 & El Economista (EcoDiario) & Private/Non-Public & Online and Offline & National & very low = CP mentioned once & Social awareness/inclusion & Positive & National + Subnational & No myth & NA & NA & NA & NA & NA & NA & NA & NA & Spain & aspaym destaca en la campaña 'yo soy' los beneficios que la contratación de discapacitados puede tener para las empresas & 2017-05-30 & fondo social europeo & la exposición de artistas figurativos se puede contemplar en hotel boston hasta este martes (09:29) aspaym castilla y león ha puesto en marcha la campaña 'yo soy', en la que empresarios y directivos, por un lado, y trabajadores con discapacidad, por el otro, destacan los beneficios que puede tener para las empresas la contratación de personas de este colectivo y el papel que éstas pueden jugar en la organización de la empresa. valladolid, 29 (europa press) la campaña se ha presentado en valladolid este martes en un acto que ha contado con la presencia, además del director general de aspaym, julio herrero, con la gerente de el campo, silvia muñoz; el director general de quesos entrepinares, carlos tejedor; y el responsable de rsc de matarromera, francisco sardón; así como dos participantes en los itinerarios de empleo de aspaym, federico vallmitjana y gorane del caso. con la iniciativa se trata de poner de manifiesto una realidad social, como es el alto porcentaje de paro en el sector de la discapacidad y concienciar a la sociedad y a las empresas del papel tan importante que juegan en este campo, pues como ha destacado sardón, los trabajadores demuestran, cuando se les da la oportunidad, que "una persona con discapacidad puede ser y es igual de productiva, capaz, eficiente y eficaz" que cualquier otra. la campaña consiste en un vídeo promocional y un portal (www.yosoytrabajo.es) en el cual se pueden conocer los programas de empleabilidad que gestiona la entidad, cofinanciados por el fondo social europeo e iniciativa de empleo juvenil que tiene en marcha la consejería de familia e igualdad de oportunidades, así como solicitar la documentación necesaria para formar parte de dichos programas. en el vídeo promocional silvia muñoz, gerente de el campo, el presidente de matarromera, carlos moro; y el director general de quesos entrepinares, carlos tejedor, explican la relación de sus empresas con personas discapacitadas y las aportaciones que realizan ellas a sus empresas; mientras que usuarios de la bolsa de empleo de la entidad como federico vallmitjana, gorane del caso y jorge isabel explican sus casos y sus expectativas de futuro. muñoz, tejedor y el responsable de rsc de matarromera y presidente de predif castilla y león, francisco sardón, han participado este martes en la presentación, ya que sus empresas tienen ya un recorrido en lo relativo a la contratación de personas discapacitadas. el campo, una empresa familiar agroalimentaria segoviana, comenzó hace unos años con la contratación de un joven sordomudo, mientras que hace poco tiempo incorporó a una chica con discapacidad, quien ha enseñado a los responsables de la empresa que al margen de su situación están "más que capacitados para transmitir el esfuerzo y para aportar al crecimiento económico". por su parte, carlos tejedor ha indicado que entrepinares formaliza contratos desde hace tiempo con personas con discapacidad auditiva, hasta el punto de que se decidió formar a los jefes de turno para que tuvieran unas nociones de lengua de signos con la que mejorar la comunicación con estos trabajadores. "quizás podríamos incluir otro tipo de discapacitados", ha apuntado el directivo de la empresa láctea. mientras tanto, fran sardón ha recordado que matarromera incorporó a sus filas a la primera persona con discapacidad hace 16 años y que la experiencia fue positiva, pues demostró que este colectivo "puede ser y es igual de productivo, capaz, eficiente y eficaz" que cualquier otra persona. "con una empresa más diversa mejorará la cuenta de resultados", ha aseverado. principio de igualdad de oportunidades sardón ha recalcado que matarromera ha iniciado, desde entonces, un "principio rector" de igualdad de oportunidades por el cual "no se mira la capacidad de las personas, sino el perfil laboral". de hecho, se da la circunstancia de que el mismo tiene una discapacidad física. por otro lado, el participante en los itinerarios de aspaym federico vallmitjana ha explicado su perfil y la incidencia que ha tenido en él a participación en esta iniciativa. este licenciado en económicas, con un máster mba y otro en gestión de sistemas informáticos que además ha trabajado en una universidad norteamericana, así como en holanda y alemania, llegó a través de los itinerarios a trabajar como responsable de posicionamiento seo en el grupo de aspaym y gestiona una veintena de páginas web. sin embargo, gorane del caso ha explicado que nunca ha podido encontrar trabajo como restauradora pese a su formación como titulada en historia del arte y en documentación gráfica, pues o bien le reclamaban experiencia laboral o dudaban de su capacidad por estar en silla de ruedas. pese a que ha reconocido que se ha llegado a sentir "un poco acorralada" por el tema laboral, ha destacado que tras pasar por los itinerarios de aspaym ha decidido prepararse oposiciones, pues en este caso si que hay plazas reservadas para personas con discapacidad. & 803 & very low & Low & Socio-Economic & NA & NA & 2017-05-30 & 2017 & 2 & ECO
Frame & v.low & National & 500-1000 & 0.0141942 & -0.1980960 & -0.1536394 & -0.9382144 & 0.7927531 & 12.6 & 1.2981504 & 0.5794339 & Recipient & Domestic & Domestic & Domestic & Domestic|ECO & Positive\\
Spain & http://www.lne.es/asturias/2018/06/12/programa-empleo-caixa-crea-76/2301891.html & 635 & lne.es & Private/Non-Public & Online and Offline & Regional/Local & very low = CP mentioned once & Jobs & Positive & EU + National + Subnational & No myth & NA & NA & NA & NA & NA & NA & NA & NA & Spain & el programa "más empleo" de la caixa crea 76 empleos en asturias & 2018-06-12 & fondo social europeo & en la convocatoria también se ha impulsado la cualificación de 173 asturianos el primer año de más empleo de "la caixa" ha dado como resultado 76 contrataciones laborales en asturias, donde se han desarrollado 4 proyectos de las siguientes entidades sociales: faedis, fundación acción contra el hambre asturias, fundación juansoñador asturias y fundación secretariado gitano asturias. asimismo, en el marco de la convocatoria de la fundación bancaria "la caixa" se ha impulsado la formación cualificada de 173 personas en asturias. en toda españa, se han conseguido 2.010 contrataciones y se ha formado un total de 3.137 personas. se trata, en el 70\% de los casos, de formación cualificada por medio de la obtención de carnets profesionales oficiales que facilitan el acceso a un empleo de calidad. esta nueva convocatoria tiene como objetivo seleccionar proyectos que favorezcan la inserción sociolaboral de personas en situación de vulnerabilidad en el mercado de trabajo. todo ello a través de entidades sociales, centros especiales de empleo y empresas de inserción que facilitan el acceso al empleo a colectivos en situación o riesgo de exclusión social a través de itinerarios integrados personalizados. la convocatoria más empleo de "la caixa" está cofinanciada por el fondo social europeo con 30.576.923 € y por fundación bancaria "la caixa" con 10.769.309,75 €. los proyectos seleccionados han obtenido una financiación anual de entre 80.000 € y 100.000 € hasta el año 2023 e incluyen itinerarios individualizados, formación para mejorar la empleabilidad, prospección e intermediación laboral, orientación laboral en la búsqueda activa de empleo, y apoyo y seguimiento, tanto a la persona como a la empresa, en caso de lograr la inserción. la fundación bancaria "la caixa" fue designada organismo intermedio del fondo social europeo en el marco del cual ha desarrollado esta iniciativa que forma parte del programa poises del fondo social europeo para españa 2014-2020. poises es una respuesta comunitaria diseñada para promover la inclusión social y luchar contra la pobreza y cualquier forma de discriminación. & 332 & very low & Low & Socio-Economic & NA & NA & 2018-06-12 & 2018 & 3 & ECO
Frame & v.low & Regional & <500 & 0.0141942 & -0.1980960 & -0.1536394 & -0.9382144 & 0.7927531 & 12.6 & 1.2981504 & 0.5794339 & Recipient & Domestic & European & Mixed & Domestic|ECO & Positive\\
Spain & http://www.farodevigo.es/portada-pontevedra/2016/09/16/programa-depoemprende-beneficiara-120-jovenes/1534251.html & 665 & Faro de Vigo & Private/Non-Public & Online and Offline & Regional/Local & very low = CP mentioned once & Jobs & Positive & EU + National + Subnational & No myth & NA & NA & NA & NA & NA & NA & NA & NA & Spain & el programa depoemprende beneficiará a 120 jóvenes menores de 30 y en paro & 2016-09-16 & fondo social europeo & la diputación organizó una segunda jornada informativa debido al interés suscitado el edificio administrativo acogió ayer una nueva jornada informativa sobre el programa de la diputación de pontevedra "depoemprende", que está dotado con 260.000 euros. tiene como finalidad el impulso del emprendimiento y la cultura emprendedora en los jóvenes en la provincia. esta actuación está enmarcada en el programa de itinerarios de emprendimiento juvenil de la fundación incyde y financiada por el fondo social europeo y podrán participar 120 jóvenes de la provincia de pontevedra, menores de 30 años, registrados en el sistema nacional de garantía juvenil, sin ocupación y sin estar integrados en los sistemas de educación o formación. en esta jornada, la diputación tuvo que añadir propuestas a las programadas inicialmente debido al interés suscitado. "depoemprende" ya recibió más de 150 solicitudes de participación, superando las plazas ofertadas. el personal de la institución dio a conocer entre los jóvenes los objetivos y contenidos del programa para formalizar la inscripción. & 162 & very low & Low & Socio-Economic & NA & NA & 2016-09-16 & 2016 & 2 & ECO
Frame & v.low & Regional & <500 & 0.0141942 & -0.1980960 & -0.1536394 & -0.9382144 & 0.7927531 & 12.6 & 1.2981504 & 0.5794339 & Recipient & Domestic & European & Mixed & Domestic|ECO & Positive\\
Spain & https://www.farodevigo.es/portada-deza-tabeiros-montes/2018/09/12/concello-primara-estetica-proyecto-renovacion/1960126.html & 660 & Faro de Vigo & Private/Non-Public & Online and Offline & Regional/Local & very low = CP mentioned once & Infrastructure & Positive & Subnational & No myth & NA & NA & NA & NA & NA & NA & NA & NA & Spain & el concello primará la estética en el proyecto de renovación de las calles principal y loriga & 2018-09-12 & fondo europeo de desarrollo regional & el concello de lalín licita la redacción del proyecto de reurbanización de las calles principal y joaquín loriga con un presupuesto de 34.727 euros. de esta forma, pretende obtener un proyecto constructivo "óptimo" que permita una humanización de ambas calles, tanto en su movilidad como en su estética, tal como explica el teniente de alcalde, nicolás gonzález casares. como arterias centrales de lalín, las calles principal y joaquín loriga soportan un tránsito importante, de modo que se pretende favorecer una movilidad peatonal, ciclista y no motorizada en condiciones de alta seguridad y baja siniestralidad, disminuyendo los desplazamientos en automóvil. la arteria única resultante pasará a tener "prioridad peatonal" mediante plataforma única, pero con las características técnicas que permitan mantener el tráfico rodado no pesado por ella, como viene sucediendo hasta ahora. el pliego de prescripciones técnicas para la contratación de la redacción del proyecto de ejecución de plataformas peatonales y de accesibilidad para principal y loriga quedó colgado ayer en la plataforma de contratación del estado. así que, desde hoy, se abre un plazo de 21 días naturales para que los profesionales o empresas interesadas puedan presentar sus ofertas. la iniciativa se enmarca en el objetivo temático número 4 de la estrategia dusi lalín ssuma 21, cofinanciado en un 80\% por el fondo europeo de desarrollo regional (feder). el ámbito de actuación comprende una superficie aproximada de 4.500 metros cuadrados entre las rúas principal y joaquín loriga, si bien se realizará una renovación parcial de las instalaciones urbanas según su necesidad. la principal tiene aceras de piedra y la calzada asfaltada, por lo que se actuará sobre la calzada para crear una plataforma única y se renovará el mobiliario urbano y las instalaciones que así se estimen. la intersección entre ambas calles -kilómetro 0- está acabada en piedra, si bien es necesario elevar las cuotas para adaptarse a la nueva plataforma. en cuanto a joaquín loriga, se propone la construcción de nuevas aceras en piedra y una calzada en plataforma única, teniendo en cuenta la renovación del mobiliario y parcial de otras instalaciones. el concello dispone de un levantamiento topográfico al que tendrán acceso los licitadores del contrato. los criterios estéticos serán fundamentales a la hora de valorar las propuestas que se presenten. de este modo, tendrán especial importancia la calidad arquitectónica de la propuesta, el material empleado para resolver la plataforma única, las aceras y el mobiliario urbano; la solución funcional de la situación de los elementos propuestos o la colocación de la estatua de laxeiro, situada a día de hoy en el museo municipal. también serán evaluables criterios estéticos tales como la relación entre las soluciones de despiece de los materiales de las aceras y el material elegido para la plataforma única, la propuesta de mobiliario, la de iluminación y la de vegetación para la zona. además, la redacción del proyecto deberá tener en cuenta la participación ciudadana, por lo que deberá acompañarse la propuesta definida de la misma en la ejecución del servicio. para la valoración del contrato se estima que el valor total de la obra a ejecutar posteriormente es de 415.172 euros. aunque casares admite que es posible que el coste real sea más alto, porque no se contemplan servicios como smart city, que regularía los controles de acceso y permitiría retirar la actual valla o la iluminación. una vez rematado el proceso de presentación de ofertas y en cuanto estén analizadas las mismas, el plazo previsto para la redacción del proyecto de renovación de principal y loriga es de dos meses. por lo tanto, casares estima que se podría conocer a finales de noviembre o a principios de diciembre. entonces se sacaría a licitación para la ejecución de los trabajos, que se evitará que coincidan con eventos como la feira do cocido o las fiestas patronales. & 635 & very low & Low & Socio-Economic & NA & NA & 2018-09-12 & 2018 & 3 & ECO
Frame & v.low & Regional & 500-1000 & 0.0141942 & -0.1980960 & -0.1536394 & -0.9382144 & 0.7927531 & 12.6 & 1.2981504 & 0.5794339 & Recipient & Domestic & Domestic & Domestic & Domestic|ECO & Positive\\
\addlinespace
Spain & http://www.farodevigo.es/portada-ourense/2018/02/28/adif-licita-contrato-adaptara-vias/1846082.html & 688 & Faro de Vigo & Private/Non-Public & Online and Offline & Regional/Local & very low = CP mentioned once & Infrastructure & Positive & National & No myth & NA & NA & NA & NA & NA & NA & NA & NA & Spain & adif licita el contrato que adaptará las vías y andenes de la estación para la llegada del ave & 2018-02-28 & fondo europeo de desarrollo regional & las actuales vías del ferrocarril en la estación empalme que serán modificadas para llevar a cabo el proyecto. // iñaki osorio los anuncios sobre las actuaciones relacionadas con la llegada del ave se suceden. después de que la conselleira de infraestructuras, ethel vázquez, presentase el proyecto de la futura estación de autobuses y la construcción de un aparcamiento, por parte de adif se daba a conocer que el consejo de administración aprobó, en su última reunión, la licitación del contrato de obras de ejecución del proyecto de construcción de remodelación de vías y andenes en la estación empalme de ourense, en el barrio de a ponte, que acogerá los nuevos servicios de alta velocidad madrid-galicia. el presupuesto de licitación de este contrato asciende a 16.699.296 euros (iva incluido) y cuenta con un plazo de ejecución de 12 meses. las obras licitadas se ciñen exclusivamente a la zona de vías y andenes y son independientes del proyecto de estación intermodal diseñado para la terminal ourensana. una actuación que va a ser cofinanciada por fondo europeo de desarrollo regional (feder) a través del p.o. plurirregional de españa 2014-2020, objetivo temático 7: transporte sostenible. cambios el proyecto tiene como objetivo la remodelación de vías y andenes de la estación de ourense para que la explotación actual de la terminal sea compatible con la llegada a la misma de trenes de alta velocidad en ancho de vía estándar, que es de 1.435 mm. la ejecución de las actuaciones previstas llevará consigo un corte parcial de las actuales vías 1 y 3, y de esa forma proceder a la ampliación del andén principal. en el estudio técnico se plasma que eso supondrá que se formen lo que denominan "dos vías mango (vías muertas de corta longitud que sirven para apartar material rodante durante las maniobras)" a cada uno de los lados del edificio de viajeros. ora de las actuaciones es la instalación de un "bretelle" (aparato de vía en forma de x) entre las vías 2 y 4, que permitirá ampliar su funcionalidad operando en ellas hasta tres composiciones de 200 metros simultáneamente. también se modificarán las actuales vías, y de esa forma conservar la funcionalidad actual, es decir, que se pueda simultanear las entradas y salidas a la estación de los trenes que circulan por las líneas monforte-vigo y ourense-santiago-a coruña. eso supone que se eliminen algunas vías por la ampliación de dos andenes. en cuanto a las actuaciones en andenes, paso inferior y marquesinas, se llevará a cabo la unión de los andenes 1 y 2 en la zona central del edificio de viajeros. el nuevo espacio peatonal resultante se cubrirá con una nueva marquesina, permitiendo redistribuir el espacio actual para formación de filas de espera y control de accesos. otras de las actuaciones es la elevación, en unos casos, y la prolongación, en otros, de andenes que tienen como objetivo mantener la actividad mientras se desarrollan las obras y conformar el nuevo proyecto, que contempla la ejecución de nuevas salidas y pasos en los extremos de cada andén, mediante rampas, caminos pavimentados y pasos accesibles sobre vías, incluyendo todos los elementos adicionales necesarios para garantizar la evacuación en condiciones de seguridad. eso incluye la ubicación de un nuevo ascensor para hacer accesible el andén 3. implica la ocupación y cancelación de una de las 2 escaleras existentes de acceso a dicho andén desde el paso inferior. un proyecto que también contempla actuaciones en electrificación e instalaciones de seguridad, como es la sustitución del actual enclave eléctrico de la estación por uno electrónico, así como la incorporación de nuevos elementos relacionados con el control de tráfico de trenes. en cuanto a las actuaciones en edificios, el proyecto de adif incluye un nuevo edificio técnico para alojar el nuevo sistema electrónico, situado en el que actualmente aloja las oficinas de señalización, comunicaciones y taller, siendo necesario demoler previamente la cubierta adosada al mismo y ocupando el final de tres vías. por lo que respeta al edificio de viajeros, se procederá a la retirada de todos los equipos existentes. las diferencias entre el gobierno popular y el grupo socialista sobre todo del proyecto relacionado con la llegada del ave, que son conocidas, ayer se volvieron a escenificar, en este caso con cruce de declaraciones diametralmente opuestas. así, la proclama del alcalde, jesús vázquez, relativa a que el proyecto de la xunta de la nueva estación "es un paso en la gran transformación" de la ciudad, contrasta con las denuncias del portavoz socialista, josé ángel vázquez barquero, que no duda en airear "la gran mentira" y "los engaños" que se visualizó "en el acto del partido popular en la explanada de la estación". pero va más allá, y mientras el regidor asegura que "después de más de dos años de gobierno ahora se en los frutos", y de un "trabajo duro", vázquez barquero alerta de que "el pleno está secuestrado", en alusión a que no ratificó el convenio firmado a tres bandas: adif, xunta y concello. las diferencias en el proyecto de la futura estación intermodal fueron una constante entre el gobierno no popular y los socialistas, sobre todo después de que por parte del ministerio de fomento, a través de adif, se procediese a ese "ajuste mágico" del proyecto de norman foster que había resultado ganador de un concurso. aunque ahora el foco también se pone sobre los terrenos de la actual estación de autobuses, en la zona de o pino. barquero es rotundo al afirmar que "el alcalde volvió engañar a los vecinos", en alusión al anuncio por el que garantizaba su uso público. y advierte que en el convenio firmado con adif y xunta, en 2016, se contempla su recalificación para aprovechamiento lucrativos, y contribuir a pagar las obras de la intermodal. & 967 & very low & Low & Socio-Economic & NA & NA & 2018-02-28 & 2018 & 3 & ECO
Frame & v.low & Regional & 500-1000 & 0.0141942 & -0.1980960 & -0.1536394 & -0.9382144 & 0.7927531 & 12.6 & 1.2981504 & 0.5794339 & Recipient & Domestic & Domestic & Domestic & Domestic|ECO & Positive\\
Spain & http://www.heraldo.es/noticias/aragon/teruel-provincia/2017/08/09/santa-catalina-rodenas-declarada-bic-comienza-restauracion-1191028-1101027.html & 651 & HERALDO & Private/Non-Public & Online and Offline & Regional/Local & very low = CP mentioned once & Cultural heritage & Positive & Subnational & No myth & NA & NA & NA & NA & NA & NA & NA & NA & Spain & santa catalina de ródenas, declarada bic, comienza su restauración & 2017-08-09 & fondo europeo de desarrollo regional & la iglesia de santa catalina de ródenas, cuya construcción data del siglo xvi y está declarada bien de interés cultural, comienza este mes la restauración de su cubierta, que cuenta con un presupuesto público de adjudicación de 125.000 euros, según ha publicado este miércoles el boletín oficial de aragón. las obras, que terminarán en noviembre, están cofinanciadas por el fondo europeo de desarrollo regional dentro del programa operativo 2014-2020, con el principal objetivo de resolver las afecciones de la techumbre e impedir las filtraciones de agua al interior de la iglesia. según informa el gobierno de aragón, el edificio religioso presenta diversas deficiencias, como grietas en las bóvedas y el deterioro de la estructura de madera, así como dichas filtraciones. de la misma manera, los trabajos de acondicionamiento pretenden evitar el degradado de los bienes que alberga el templo, cuya antigüedad se remonta en algunos casos al siglo xv, como el retablo de san juan bautista o el del calvario, y al período renacentista, como en el púlpito o la predela (ambas del siglo xvi). igualmente, sobresalen otras joyas artísticas como el retablo mayor, dedicado a santa catalina (del siglo xvii), el de santa marina (siglo xvii) y el de la virgen del rosario (siglo xviii). & 208 & very low & Low & Socio-Economic & NA & NA & 2017-08-09 & 2017 & 2 & ECO
Frame & v.low & Regional & <500 & 0.0141942 & -0.1980960 & -0.1536394 & -0.9382144 & 0.7927531 & 12.6 & 1.2981504 & 0.5794339 & Recipient & Domestic & Domestic & Domestic & Domestic|ECO & Positive\\
Spain & http://ecodiario.eleconomista.es/espana/noticias/7437369/03/16/UGT-reanuda-el-Centro-de-Informacion-y-Asesoramiento-para-Inmigrantes.html & 614 & El Economista (EcoDiario) & Private/Non-Public & Online and Offline & National & very low = CP mentioned once & Social awareness/inclusion & Positive & National + Subnational & No myth & NA & NA & NA & NA & NA & NA & NA & NA & Spain & ugt reanuda el centro de información y asesoramiento para inmigrantes & 2016-03-21 & fondo social europeo & hallan tres inmigrantes, uno de ellos menor, en los bajos de un camión (10:02) la unión general de trabajadores de la rioja ha retomado este mes de marzo el centro de información y asesoramiento sociolaboral para inmigrantes una vez recibida la financiación necesaria, tal y como ha informado la responsable del departamento de migraciones, beatriz fernández. logroño, 21 (europa press) en rueda de prensa, fernández ha recordado que el ciasi, presente en once ciudades de españa, está subvencionado por el ministerio de empleo y seguridad social y cofinanciado por el fondo social europeo. en la rioja está en marcha desde el año 2005. fernández ha explicado que, aunque tenga esa trayectora, este año 2016 no había comenzado a desarrollarse hasta este mes de marzo al depender de subvención. se prolongarán hasta el 31 de diciembre. así, desde este mes, los inmigrantes que tengan dudas con el ámbito sociolaboral (búsqueda de empleo, papeles o ayudas) pueden acudir a ugt y contar con una asesora que, de forma personal, tratará su caso. el modo de funcionamiento se basa en itinerarios personalizados. el año pasado se atendió a 149 personas (además de a más de cuatrocientas consultas), por lo que lo primero que hará el centro, una vez reanudado, es ponerse en marcha con las que tenían itinerarios pendientes. & 217 & very low & Low & Socio-Economic & NA & NA & 2016-03-21 & 2016 & 2 & ECO
Frame & v.low & National & <500 & 0.0141942 & -0.1980960 & -0.1536394 & -0.9382144 & 0.7927531 & 12.6 & 1.2981504 & 0.5794339 & Recipient & Domestic & Domestic & Domestic & Domestic|ECO & Positive\\
Spain & http://www.farodevigo.es/portada-pontevedra/2018/05/01/plan-insercion-73/1883026.html & 615 & Faro de Vigo & Private/Non-Public & Online and Offline & Regional/Local & very low = CP mentioned once & Social awareness/inclusion & Positive & National + Subnational & No myth & NA & NA & NA & NA & NA & NA & NA & NA & Spain & un plan con una inserción del 73\% & 2018-05-01 & fondo social europeo & el programa de trabajo de la cruz roja en pontevedra atendió el año pasado a 306 personas en la comarca y empleó a 225 el plan de empleo de cruz roja encontró trabajo el año pasado a 225 personas, el 73 por ciento de las que tomaron parte en alguna de sus actividades. la mayoría de ellas son mujeres, un 61 por ciento, las más castigadas por la crisis en el mercado laboral. daisy funes, procedente de honduras y auxiliar de enfermería, está trabajando actualmente gracias a la iniciativa de la ong. "encontrar este puesto ha sido toda una alegría. estoy muy agradecida", reconoce. el plan de empleo de cruz roja en pontevedra atendió el año pasado en la comarca a 306 personas, que participaron de una u otra forma en las diferentes iniciativas de inserción laboral. de ellos más de un 73 por ciento lograron un empleo al sumar 225. en total, fueron más de 220 las empresas que colaboraron, permitiendo gestionar más de 206 ofertas de empleo y prácticas formativas en diferentes sectores que necesitan profesionales, además de suponer una oportunidad a aquellos trabajadores que más lo necesitan. el plan nació con la finalidad de apoyar a todas aquellas personas que tienen dificultades para acceder al mercado laboral, bien porque sean parados de larga duración, bien porque superan cierta edad o procedan de otros países. de las más de 300 personas que tomaron parte en el programa, la mayoría, 188, fueron mujeres, es decir, el 61 por ciento. las trabajadoras siguen siendo el grupo con más dificultades para acceder al mercado laboral, seguido de las personas migrantes y refugiadas, los jóvenes sin cualificación y las personas desempleadas con más de 45 años. en base a ello, el perfil mayoritario es mujer, migrante, con una edad comprendida entre los 25 y 54 años, con baja o nula cualificación y en desempleo desde hace más de un año. orientación y formación el plan de empleo de cruz roja se puso en marcha en el año 2000 para apoyar en este sentido a quienes lo tienen más difícil a través de la orientación a la búsqueda de trabajo hasta la propia inserción laboral, pasando por acciones formativas y prácticas profesionales que permitan trabajar en un sector determinado. además, la iniciativa contó el año pasado con la colaboración de más de 20 personas voluntarias de cruz roja en pontevedra que aportan su tiempo, sus conocimientos y su experiencia. "el plan de empleo es un puente entre las personas vulnerables y el mercado laboral. la fórmula funciona cuando sumamos la colaboración de las empresas, que valoran mucho la motivación y la preparación de los participantes, así como la profesionalidad y dedicación de los equipos de empleo de cruz roja", señala antoni bruel, coordinador general de cruz roja. además, en este esfuerzo, el plan de empleo de cruz roja española cuenta con el apoyo de las administraciones públicas a través de los programas operativos de empleo, poej y poises, financiados por el ministerio de empleo y seguridad social y el fondo social europeo. & 509 & very low & Low & Socio-Economic & NA & NA & 2018-05-01 & 2018 & 3 & ECO
Frame & v.low & Regional & 500-1000 & 0.0141942 & -0.1980960 & -0.1536394 & -0.9382144 & 0.7927531 & 12.6 & 1.2981504 & 0.5794339 & Recipient & Domestic & Domestic & Domestic & Domestic|ECO & Positive\\
Spain & http://www.expansion.com/empresas/tecnologia/2015/12/09/5667ef9d22601d1d2f8b462f.html & 641 & Expansión & Private/Non-Public & Online and Offline & National & very low = CP mentioned once & Economic development & Positive & EU + National & No myth & NA & NA & NA & NA & NA & NA & NA & NA & Spain & la digitalización llega al mundo de las pymes & 2015-12-09 & fondo europeo de desarrollo regional & de izquierda a derecha, inmaculada riera, directora general de la cámara de comercio de españa, y daniel noguera, director general de red.es.p. davilaexpansion las nuevas tecnologías de la información suponen una oportunidad barata y accesible para potenciar la competitividad e internacionalización de las pequeñas y medianas empresas. la nueva era tecnológica ha empujado a las empresas a una carrera por digitalizar sus procesos para ser más eficientes y ganar en competitividad. una carrera en la que las pequeñas y medianas empresas parten con desventaja debido a su reducido tamaño, las secuelas de la crisis económica, que han lastrado las inversiones en tecnología, y la ausencia de una cultura empresarial que potencie la innovación. ésta fue una de las principales conclusiones a las que se llegó en la jornada sobre tecnologías de la información: una oportunidad para pymes y autónomos, organizada por el diario expansión, en colaboración con la unión europea, a través del fondo europeo de desarrollo regional, y la cámara de comercio de españa. "las pymes son una parte muy importante de nuestro tejido productivo. han sido, además, las más afectadas por la crisis económica, que ha hecho que las inversiones tecnológicas se hayan reducido drásticamente", indicó inmaculada riera, directora general de la cámara de comercio de españa, y encargada de inaugurar el foro. "la apuesta de las pymes por innovar ha de ser rotunda. la innovación es el eje básico que da solidez a los procesos de internacionalización", añadió. una internacionalización que se antoja imposible para muchas pymes a las que su reducido tamaño les hace muy difícil exportar sus productos. cabe recordar que en españa el 97\% de las empresas exportadoras son pymes y, sin embargo, éstas sólo venden fuera el 47\% de los bienes y servicios españoles. o, dicho de otro modo, un 3\% de las compañías de nuestro país son responsables de más de la mitad de las exportaciones. una opinión compartida por el presidente de la fundación españa digital, martín pérez, quien constató como "las pymes que usan las tecnologías digitales exportan dos veces más, crecen el doble y son diez veces más eficientes". en este sentido, daniel noguera, director general de red.es, apostó por que las pequeñas y medianas compañías afronten con ambición sus procesos de digitalización o, de lo contrario, tendrán problemas en el futuro. "o las empresas se digitalizan o el futuro no será todo lo halagüeño que pueda ser para ellas. seguimos estando por detrás de la media europea en cuanto a oferta de productos digitales. es difícil de entender cómo el comercio electrónico es un factor de competitividad en europa y en españa no lo estamos aprovechando", subrayó noguera. para carlos prieto, director general de la cámara de comercio de madrid, no es casual que el ecommerce no esté tan extendido en españa como en los países del norte de europa y achacó este retraso al clima y la cultura. "estamos ante un problema cultural y, además, el entorno no es el más propicio. no es casual que españa esté a la cola de europa ni que nuestras pymes vean el mundo digital como algo lejano", apuntó prieto. otro de los aspectos en el que incidieron los participantes en el foro fue en la gama de servicios tecnológicos baratos, incluso gratuitos, en los que se pueden apoyar las pymes para ganar en competitividad y promocionar su actividad. "las tic no son caras, hay multitud de servicios pensados para pymes. no hay que enfocarlo como un gasto ni como una pérdida de tiempo", señaló carmelo muñoz, director general del observatorio nacional de las telecomunicaciones para la sociedad de la información (ontsi). sobre todos ellos, los ponentes destacaron las redes sociales. "son un servicio barato que llega a mucha gente y que nos permite contactar directamente con nuestros clientes. el único indicador dónde las pymes están al nivel de las grandes empresas es en el uso de las redes sociales", destacó alfonso arbaiza, director general de fundetec. de la clausura del evento se encargó el subdirector general de apoyo a la pyme del ministerio de industria, energía y turismo, antonio fernández ecker, quien señaló "la digitalización como un factor crítico de competitividad". "nuestro principal objetivo es trabajar para mejorar y facilitar la competitividad de la pequeña y mediana empresa que pasa, obligatoriamente, por la adopción de las tic en sus procesos", concluyó ecker. exportar a través del 'ecommerce' en los últimos cuatro años, las pymes españolas que exportan a través de ebay han aumentado un 213\%, convirtiendo esta plataforma de comercio electrónico en uno de los principales canales para que las pequeñas y medianas empresas vendan sus productos en el exterior. "ebay es un claro protagonista en la evolución del 'ecommerce' en españa y una buena opción para las pymes que quieran un canal adicional en su estrategia ya sea para escalar en sus ventas o para internacionalizar su actividad", indicó manuel pérez, responsable de vendedores de ebay españa. asimismo, pérez hizo hincapié en el escaso porcentaje de digitalización del comercio minorista en el país y animó a las pymes a modernizarse. "hay muchos compradores que están adquirieron productos de otros países porque en españa no tienen esa oferta", apuntó. & 862 & very low & Low & Socio-Economic & NA & NA & 2015-12-09 & 2015 & 1 & ECO
Frame & v.low & National & 500-1000 & 0.0141942 & -0.1980960 & -0.1536394 & -0.9382144 & 0.7927531 & 12.6 & 1.2981504 & 0.5794339 & Recipient & Domestic & European & Mixed & Domestic|ECO & Positive\\
\addlinespace
Spain & http://www.europapress.es/andalucia/noticia-caritas-expone-obispos-sur-labor-andalucia-donde-invertido-394-millones-20171025133140.html & 677 & europa press & Private/Non-Public & Online and Offline & National & very low = CP mentioned once & Social awareness/inclusion & Positive & National + Subnational & No myth & NA & NA & NA & NA & NA & NA & NA & NA & Spain & cáritas expone a los obispos del sur su labor en andalucía, donde ha... & 2017-10-25 & fondo social europeo & publicado 25/10/2017 13:31:40cet el vicario general de la archidiócesis de sevilla, teodoro león, ha sido nombrado nuevo secretario general de los obispos del sur de españa córdoba, 25 oct. (europa press) - el presidente de cáritas regional de andalucía, mariano pérez de ayala, ha presentado a la cxxxviii asamblea ordinaria de los obispos del sur, reunida en córdoba desde este martes y hasta este miércoles, la memoria anual de cáritas correspondiente al año 2016 y los nuevos retos que tienen las distintas cáritas de andalucía, comunidad autónoma ésta donde cáritas invirtió 39,4 millones de euros el pasado año. así, según ha informado odisur, ante los obispos de las diócesis de sevilla, granada, almería, cádiz y ceuta, córdoba, guadix, huelva, jaén, asidonia-jerez y málaga, pérez de ayala ha destacado que, en andalucía, cáritas constituye, probablemente, "la red de ayuda más extensa y cercana de apoyo para las personas en situaciones de exclusión y vulnerabilidad". respecto a cifras concretas, ha detallado que en 2016 cáritas invirtió en andalucía 39.414.451 euros, de los que 26 millones provenían de fondos privados (socios, donantes y legados) y 13 millones de fondos públicos (fondo social europeo y administraciones públicas), habiéndose beneficiado la acción de las cáritas andaluzas 342.893 personas, en los distintos programas emprendidos, para los que se ha contado con 781 personas contratadas y 13.252 voluntarios. la acogida y la asistencia a las personas necesitadas suponen buena parte de la labor diaria de cáritas. en las diez diócesis andaluzas hay 1.201 cáritas parroquiales, cuya misión es hacer de la parroquia un lugar de servicio y de referencia para la integración de las personas en situación de pobreza. no se limitan a dar respuesta a las demandas de los usuarios (alimentación, ropa, suministros y vivienda, principalmente), sino que su tarea consiste también en hacerse cercanos a las personas, reivindicar su dignidad, buscar con ellos respuestas a sus necesidades y descubrir sus potencialidades. por los programas de empleo de las cáritas de andalucía han pasado, durante 2016, un total de 11.092 personas, para seguir un itinerario personalizado de inserción social y laboral. además, en cáritas se ha apostado por la economía solidaria, desarrollando proyectos que ponen a las personas en el centro de la actividad económica y que facilitan su acceso a un trabajo digno: empresas de reciclaje textil, de recogida de papel y aceites, de eco-agricultura, de restaurante-escuela y talleres de inserción (imprenta, serigrafía y costura). otros programas desarrollados por cáritas andalucía son los referidos a 'personas sin hogar', en los que se han atendido en 2016 a 7.540 personas; 'menores', con 6.361 beneficiarios; 'inmigrantes', con 5.981 personas atendidas; habiéndose hecho también "presente la respuesta de la iglesia a las demandas de las personas más vulnerables y necesitadas" en los programas de 'mayores', 'formación de voluntariado', 'atención a personas con discapacidad', 'cooperación internacional' y 'comercio justo', según ha resaltado el presidente de cáritas en andalucía, mariano pérez de ayala nombramientos por otro lado, los obispos del sur han nombrado a maría dolores megina navarro, de la diócesis de jaén, como presidenta de la hoac de andalucía, a propuesta de las delegaciones diocesanas de hoac en andalucía y por el pleno andaluz de representantes de este movimiento eclesial. junto a ello, los obispos andaluces han aceptado la renuncia del secretario general de la asamblea, antonio hiraldo velasco, que deja el cargo por motivos de edad y que ha estado al frente de la asamblea desde 1983. los obispos del sur han tenido palabras de reconocimiento a su gran labor desarrollada en estos años, programando y coordinando los encuentros. al mismo tiempo, han nombrado a teodoro león como nuevo secretario general de la asamblea de obispos del sur de españa. león es doctor en derecho canónico y diplomado en jurisprudencia de la rota romana, actual vicario general de la archidiócesis de sevilla, deán presidente de la catedral hispalense y delegado episcopal para las causas de los santos. teodoro león nació en puertollano (ciudad real) el 27 de noviembre de 1964, y fue ordenado sacerdote en sevilla el año 1991. ha sido párroco de nuestra señora de las veredas, en guadalema de los quinteros, y vicario parroquial en san isidoro de sevilla. finalmente, la cxxxviii asamblea ordinaria de los obispos del sur ha aprobado iniciar el proceso diocesano de la causa de canonización de la sierva de dios maría antonia de jesús tirado ramírez, de la tercera orden de santo domingo y fundadora de la casa madre de las dominicas del santísimo sacramento, en jerez de la frontera (cádiz). nació en esta ciudad gaditana en 1740 y murió santamente en 1810. cáritas córdoba & 780 & very low & Low & Socio-Economic & NA & NA & 2017-10-25 & 2017 & 2 & ECO
Frame & v.low & National & 500-1000 & 0.0141942 & -0.1980960 & -0.1536394 & -0.9382144 & 0.7927531 & 12.6 & 1.2981504 & 0.5794339 & Recipient & Domestic & Domestic & Domestic & Domestic|ECO & Positive\\
Spain & https://www.diariosur.es/accion-solidaria/mujeres-gitanas-decir-20180923012246-nt.html & 599 & Sur & Private/Non-Public & Online and Offline & Regional/Local & very low = CP mentioned once & Social awareness/inclusion & Positive & National + Subnational & No myth & NA & NA & NA & NA & NA & NA & NA & NA & Spain & "las mujeres gitanas tenemos mucho que decir en el movimiento por la igualdad" & 2018-09-23 & fondo social europeo & jennifer amador y sonia cortés, profesionales con estudios universitarios, rompen estereotipos y se convierten en referentes para los jóvenes de su comunidad sonia cortés y jennifer amador son responsables de dos de los programas más importantes de la fundación secretariado gitano de málaga, que trabaja para conseguir la promoción integral de la comunidad gitana desde la atención a la diversidad cultural. ambas tienen formación universitaria, son profesionales y gitanas, que rompen estereotipos y se convierten en un referente para animar a los jóvenes de su etnia a formarse y a que trabajen para llegar a ser lo que se propongan, además de promover la igualdad en su entorno. cortés (granada, 40 años) es licenciada en sociología, orientadora educativa y coordinadora del plan promociona, financiado por el fondo social europeo que acaba de cumplir diez años acompañando a los niños gitanos en sus estudios. un programa que ha logrado en este tiempo rebajar las cifras de abandono prematuro de los estudios y mejorar las condiciones de acceso al mercado laboral de los jóvenes gitanos. "las familias gitanas son cada vez más conscientes de la importancia de la educación y esto está cambiando muchas cosas dentro de la comunidad", señala cortés. antes, asegura, los chicos solo se planteaban terminar la educación obligatoria y ahora empiezan a mirar más allá. "lo importante es que los jóvenes gitanos conozcan otras opciones y puedan elegir; no todos tienen que ir a la universidad, pero hay que tener una preparación", añade. por su parte, jennifer amador (nacida también en granada, 27 años), se licenció en magisterio de primaria. luego terminó humanidades y tiene dos másteres, uno en historia y otro de proferorado en secundaria. además, es técnica de igualdad del programa calí, que por primera vez se centra en la mujer gitana para promover su acceso al mercado laboral. "yo nunca me he sentido limitada en mi familia por ser mujer para querer estudiar o salir fuera", dice amador. el ejemplo de jóvenes gitanas que, como ella, quieren formarse, trabajar y ser independientes económicamente rompe barreras psicológicas entre los más tradicionales. "cuando las niñas empiezan a estudiar y descubren que tienen otras alternativas ven más allá de querer casarse y tener hijos jóvenes, algo que es tan respetable como estudiar; lo bueno es que puedan decidir", dice amador. "muchos padres gitanos quieren que sus hijas se formen pero que a la vez se casen pronto; yo les digo que el hecho de casarse o tener pareja no puede frenar sus sueños", opina cortés, por su parte. como mujeres, ambas asisten expectantes al impulso que ha tomado el movimiento por la igualdad a raíz del pasado 8-m. "somos mujeres antes de ser de cualquier etnia; este movimiento tiene que contar con la mujer gitana porque tenemos mucho que decir y aportar", señala jennifer. la mujer gitana, indica cortés, "no puede permanecer al margen de esta lucha porque sufrimos una doble discriminación, por ser mujeres y por ser gitanas". aseguran que a pesar de que la sociedad piensa habitualmente que la comunidad gitana es patriarcal, lo cierto es que en los hogares las mujeres son el pilar fundamental de la familia. "la imagen de mujer cuidadora y sumisa es totalmente opuesta a la realidad; somos las que llevamos el peso de la casa y de las decisiones en una particular revolución silenciosa", asegura sonia. hay aún mucho por hacer, dicen, en el reparto de las tareas domésticas; pero en esto, como en el resto de la sociedad española, se están dando pasos. "mi padre no hacía nada en casa y ahora sí se dividen las tareas; esto ocurre más ahora porque cada vez son más las mujeres que trabajan fuera de casa", explica cortés. & 617 & very low & Low & Socio-Economic & NA & NA & 2018-09-23 & 2018 & 3 & ECO
Frame & v.low & Regional & 500-1000 & 0.0141942 & -0.1980960 & -0.1536394 & -0.9382144 & 0.7927531 & 12.6 & 1.2981504 & 0.5794339 & Recipient & Domestic & Domestic & Domestic & Domestic|ECO & Positive\\
Spain & http://vozpopuli.com/economia-y-finanzas/87324-el-coste-del-bloqueo-bruselas-sancionara-a-espana-con-mas-de-6-000-millones-si-no-hay-pactos & 613 & vozpopuli.com & Private/Non-Public & Online only & National & very low = CP mentioned once & Political leverage & Negative & EU + National & No myth & NA & NA & NA & NA & NA & NA & NA & NA & Spain & el coste del bloqueo: bruselas sancionará a españa con más de 6.000 millones si no hay pactos & 2016-08-02 & fondos estructurales & el bloqueo político le puede salir caro a españa. en concreto, la factura podría rondar los 6.100 millones de euros. el gobierno en funciones que preside mariano rajoy considera inevitable una sanción de la unión europea si no se produce un pacto entre las distintas fuerzas políticas que permita presentar un presupuesto a bruselas el próximo 15 de octubre, fecha en la que el ejecutivo de turno debe presentar un plan presupuestario de ingresos y gastos a las instituciones comunitarias en el que demuestre que adoptará las medidas con las que españa logró salvarse de la multa por incumplir el déficit la semana pasada. "esta vez se ha podido parar la multa, pero la próxima en octubre sería ya imparable", aseguran fuentes gubernamentales al periódico el mundo, añadiendo que tal plan, para tener credibilidad, sólo puede basarse en el proyecto de ley de presupuestos de 2017 que, a su vez, sólo es posible si hay un gobierno activo a partir de septiembre. si rajoy, sánchez, iglesias y rivera siguen como hasta ahora y son incapaces de superar el bloqueo político que experimenta el país desde el pasado 20 de diciembre, eso supondría reincidir en el incumplimiento del déficit del año pasado que ya dio lugar al primer expediente sancionador, junto a portugal, de la historia de la eurozona. además, conforme un estado infractor reincide, el procedimiento sancionador se endurece. así, el amplio abanico de sanciones al que se enfrentaría españa en aplicación del artíuclo 126, apartado 11 del tratado de la unión europea se desgrana de la siguiente manera: - la multa ya no sería de hasta el 0,2\% del pib (2.000 millones de euros), como la salvada la semana pasada, sino de hasta el 0,5\% (5.000 millones). - a estos 5.000 millones habría que sumar otros 1.100 millones de la congelación de los fondos estructurales. según el rotativo de unidad editorial, que cita a una portavoz de la comisión europea, esta medida se adoptará a la vuelta del verano tras un diálogo con el parlamento europeo. sólo si españa presenta el 15 de octubre las medidas que reclama bruselas, se desbloquearán estos fondos. - a estas sanciones económicas es probable que se sume otra más: el regreso de los temidos "hombres de negro". es decir, "la posibilidad de una supervisión reforzada mediante misiones de la comisión a las que se puede invitar al banco central europeo". - el informe gubernamental contempla también la opción de otras sanciones que podrían proceder del banco europeo de inversiones, que en 2015 prestó alrededor de 12.000 millones a entidades españolas. - por si fuera poco, la comisión europea y el consejo de ministros de finanzas de la ue también pueden poner nuevas trabas a la emisión de bonos y obligaciones del tesoro e incluso una "espectacular" sanción adicional para asegurar que españa no incurre en más gasto público y respeta el objetivo de déficit. fuentes gubernamentales no creen que a españa se le aplique todo este abanico de medidas pero tampoco ven argumentos para volver a librarse de la multa. a las posibles sanciones derivadas de la ausencia de pactos podría añadirse, además, el estigma ante la comunidad internacional de que la cuarta economía del euro es incapaz de presentar unos presupuestos del estado tras dos elecciones generales en seis meses. es el precio del bloqueo. & 556 & very low & Low & Power & NA & NA & 2016-08-02 & 2016 & 2 & POL
Frame & v.low & National & 500-1000 & 0.0141942 & -0.1980960 & -0.1536394 & -0.9382144 & 0.7927531 & 12.6 & 1.2981504 & 0.5794339 & Recipient & Domestic & European & Mixed & Domestic|POL & Negative\\
Spain & http://ecodiario.eleconomista.es/espana/noticias/9165514/05/18/La-Junta-resalta-que-en-Andalucia-el-riesgo-de-despoblacion-rural-es-bajo-gracias-a-las-politicas-de-desarrollo.html & 670 & El Economista (EcoDiario) & Private/Non-Public & Online and Offline & National & low = CP mentioned more times but NOT important part of story (mainly about others issues) & Economic development & Factual & EU + National & No myth & NA & NA & NA & NA & NA & NA & NA & NA & Spain & la junta resalta que en andalucía el riesgo de despoblación rural es bajo "gracias a las políticas de desarrollo" & 2018-05-27 & fondo europeo de desarrollo regional & el director general de desarrollo sostenible del medio rural, manuel garcía benítez, ha destacado que andalucía tiene en estos momentos menos riesgo de sufrir la despoblación de su medio rural, un fenómeno que sí está afectando con fuerza a otras comunidades autónomas, "gracias a las políticas globales de desarrollo que se están ejecutando por parte de la administración autonómica". sevilla, 27 (europa press) en este sentido, garcía benítez ha considerado que es clave "garantizar servicios básicos en todos los territorios", así como llevar a cabo acciones concretas que "incentiven no sólo la permanencia sino también la llegada de jóvenes y mujeres", según ha informado la junta en un comunicado. garcía benítez ha participado en soria en las jornadas 'combatiendo la despoblación en áreas rurales', un taller transnacional enmarcado en el programa europeo espon, para incentivar el análisis y debate sobre un problema que está afectando de manera creciente a zonas de españa y toda europa. en este marco, ha presentado las líneas estratégicas de la consejería de agricultura, pesca y desarrollo rural para evitar este despoblamiento rural, basadas en cinco claves, concretamente, "la participación, la creación de empleo, la igualdad de oportunidades, la innovación y la cooperación, especialmente entre territorios para el intercambio de experiencias". el responsable andaluz de desarrollo sostenible del medio rural ha destacado que, "en comparación con otras regiones españolas, la pérdida de población en las zonas rurales de andalucía es baja, amortiguada por las acciones emprendidas por la administración andaluza". en esta línea ha apuntado la mejora de las infraestructuras y servicios en todos los espacios rurales, con la ejecución de programas sociales, educativos y sanitarios, así como el impulso a las vías de comunicación. "llevamos más de 30 años haciendo los deberes, y por eso mantenemos vivos a los pueblos pequeños", ha apostillado. a esto hay que unir las políticas propias de la consejería de agricultura, resaltando los trabajos realizados desde 1991 dentro de la iniciativa comunitaria leader en sus distintas versiones y en el marco actual, con una catalogación de 52 territorios rurales. en el periodo comunitario ahora vigente (2014-2020) se han presentado hasta la fecha más de 2.700 proyectos de desarrollo por valor de 72 millones de euros, iniciativas destinadas a la adecuación y dotación de infraestructuras y servicios, al desarrollo de actividades empresariales generadoras de empleo, a potenciar la empleabilidad de los jóvenes, a mejorar la competitividad de las pequeñas y medianas empresas (pymes) y a la conservación y promoción del patrimonio rural. coordinación y cooperación garcía benítez ha incidido en que en las zonas rurales se actúa "con un enfoque territorial e integrador, conforme a sus potencialidades y recursos y con una coordinación y cooperación administrativa eficiente", dando preferencia a inversiones y actuaciones de carácter comarcal que incidan en más de un municipio para lograr un crecimiento más homogéneo. a esto se une el impulso a acciones complementarias y suplementarias entre zonas rurales y urbanas, aprovechando los recursos endógenos y promoviendo la participación de la población de los pueblos. a estas medidas, en los casos de riesgo de despoblamiento o necesidad de mantenimiento de la población se unen otras eminentemente formativas, especialmente para jóvenes y mujeres. eso se complementa con medidas e incentivos específicos, así como con la puesta en valor de los servicios e infraestructuras de los municipios dirigidos a la creación de empleo, ubicación de empresas y nuevos pobladores. el director general ha subrayado que toda esta estructura que ha dado buenos resultados puede verse "seriamente afectada" si se concreta la amenaza de recorte al presupuesto de la política agrícola común (pac), de un 16 por ciento de media según el marco financiero plurianual presentado por la comisión europea. este recorte es "especialmente grave" en el caso de los fondos feader, los destinados a desarrollo rural, ya que la reducción se eleva hasta el 26 por ciento, de ahí su reclamación al gobierno central para que reclame en europa que no pueden perderse ayudas, "si al final esto se concreta estarán condenadas a desaparecer las asociaciones de desarrollo rural que ahora trabajan en el territorio en toda andalucía". tendencia al despoblamiento para el año 2050 se prevé que la población de las regiones rurales disminuirá en casi ocho millones de personas, afectando especialmente a países del centro, este y norte de europa, así como a los del mediterráneo, entre ellos españa y portugal. el despoblamiento es uno de los grandes retos que ha llevado a las distintas administraciones a poner en marcha políticas y medidas innovadoras en las zonas rurales. en andalucía, que es la segunda región en superficie y la primera en población de españa, el 90 por ciento del territorio es rural y en él vive el 43 por ciento de sus habitantes. la comunidad andaluza cuenta con 713 municipios rurales, la mayoría con menos de 10.000 habitantes, considerándose que están en mayor riesgo de despoblación pequeñas localidades de áreas como la alpujarra, el altiplano granadino o el desierto de almería. el programa europeo espon es una iniciativa que da soporte a políticas de desarrollo y a la construcción de una comunidad científica europea en el campo del desarrollo y la cohesión territorial. está financiado por el fondo europeo de desarrollo regional (feder), así como por contribuciones de diversos países, entre ellos españa. & 879 & low & Low & Socio-Economic & NA & NA & 2018-05-27 & 2018 & 3 & ECO
Frame & low-medium & National & 500-1000 & 0.0141942 & -0.1980960 & -0.1536394 & -0.9382144 & 0.7927531 & 12.6 & 1.2981504 & 0.5794339 & Recipient & Domestic & European & Mixed & Domestic|ECO & Neutral\\
Spain & http://www.farodevigo.es/economia/2016/07/16/economia-inyecta-19-millones-investigar/1499247.html & 672 & Faro de Vigo & Private/Non-Public & Online and Offline & Regional/Local & very low = CP mentioned once & Research \& innovation & Positive & National + Subnational & No myth & NA & NA & NA & NA & NA & NA & NA & NA & Spain & economía inyecta 19,2 millones para investigar con la xunta el uso de drones en galicia & 2016-07-16 & fondo europeo de desarrollo regional & el ejecutivo gallego completa la inversión, de 24 millones en total el ministerio de economía y competitividad destinará 19,2 millones de euros, cofinanciados por el fondo europeo de desarollo regional (feder), a un proyecto que fomentará la investigación y la tecnología aeronáutica para la utilización de sistemas aéreos no tripulados y drones en ámbito civil en el aeródromo de rozas, en castro de rei (lugo). así lo aprobó ayer el consejo de ministros, que dio luz verde a un convenio de colaboración entre el departamento que dirige luis de guindos y la xunta. en concreto, las entidades gallegas participantes son la axencia galega de innovación (gain) y la axencia para a modernización tecnolóxica de galicia (amtega) y, la denominación del proyecto, civil uavs initiative. de este modo, según explicó el ministerio, se busca mejorar la prestación de determinados servicios públicos de la xunta en diferentes sectores, como el agrario, forestal, marítimo y medioambiental, a través de la explotación de las potencialidades de los sistemas aéreos no tripulados mediante una propuesta de compra pública innovadora (cpi). el presupuesto del proyecto civil uavs initiative asciende a 24 millones de euros, de los que el ministerio de economía y competitividad aportará 19,2 millones en anticipos reembolsables con la cofinanciación del fondo europeo de desarrollo regional. mientras, la axencia galega de innovación contribuirá al proyecto con 4 millones y, la amtega, con 800.000 euros. & 233 & very low & Low & Socio-Economic & NA & NA & 2016-07-16 & 2016 & 2 & ECO
Frame & v.low & Regional & <500 & 0.0141942 & -0.1980960 & -0.1536394 & -0.9382144 & 0.7927531 & 12.6 & 1.2981504 & 0.5794339 & Recipient & Domestic & Domestic & Domestic & Domestic|ECO & Positive\\
\addlinespace
Spain & http://www.farodevigo.es/sociedad/2017/09/26/firma-vigo-beneficiadas-programa-empleo/1756124.html & 605 & Faro de Vigo & Private/Non-Public & Online and Offline & Regional/Local & very low = CP mentioned once & Jobs & Positive & Subnational & No myth & NA & NA & NA & NA & NA & NA & NA & NA & Spain & una firma de vigo, entre las beneficiadas por el programa de empleo joven "la caixa" & 2017-09-26 & fondo social europeo & el joven empleado despacha en la empresa que se benefició de la ayuda a la contratación. // r. grobas el joven eric balbuena, un vigués de 22 años, ha sido uno de los beneficiados de las ayudas para incetivar la contratación de jóvenes de entre 16 y 29 años, que puso en marcha este año "la caixa". es su segundo trabajo y ya cuenta con un contrato indefinido en "sureya", una empresa distribuidora de suministros para automóviles localizada en lavadores, vigo. "trabajé antes en otra empresa del sector, pero me llegó esta oferta y estoy encantado", asegura el joven que cursó un ciclo de electromecánica y ahora atiende y clasifica pedidos telefónicos de empresas de automoción. la responsable de administración de "sureya", montse caride, explica que recibieron información de la iniciativa de la entidad bancaria, de quienes son clientes, vía mail. "nos pareció muy interesante, porque habíamos pensado en contratar a alguien para cubrir vacaciones. en vista de la ayuda, decidimos hacer un contrato indefinido y ahorramos costes, porque subvencionan hasta el 80 \% de los gastos de la contratación. así reforzamos la plantilla", reconoce la administradora de "sureya"; una firma asentada desde hace tres décadas en vigo y que actualmente tiene 19 trabajadores. el programa contemplaba ayudas de hasta 9.600 eros "la caixa" abrió este año la convocatoria del programa empleo joven para facilitar ayudas a empresas, incluidos empresarios autónomos, asociaciones, fundaciones y otras entidades sin ánimo de lucro. ahora ha decidido prorrogarlo, por lo que permanecerá activo hasta febrero de 2018. su objetivo es fomentar el empleo joven, estable y de calidad, dinamizando a su vez la actividad empresarial y el crecimiento económico. el proyecto contó con un presupuesto total de 21,7 millones de euros. el fondo social europeo aporta 20 millones de euros, cuya gestión asume la obra social "la caixa". esta, a su vez, destinará 1,7 millones adicionales a la iniciativa. las solicitudes pueden presentarse, junto con la documentación requerida en la página web www.lacaixaempleojoven.org. se podrán conceder ayudas a un máximo de diez personas jóvenes por entidad o empresa (se entiende por nif). los destinatarios finales de este programa son todas las personas jóvenes que cumplan los siguientes requisitos: tener entre 16 y 29 años (ambos incluidos) en el momento de la formalización del contrato, estar inscritas en el sistema nacional de garantía juvenil de manera previa a la formalización del contrato. en su defecto, haberse apuntado como demandante de trabajo en un servicio público de empleo o haber realizado la renovación de la demanda de empleo, en ambos casos a partir del 25 de diciembre de 2016 y de manera previaa la formalización del contrato y no haber trabajado en el día natural anterior al de inicio del contrato subvencionado. & 459 & very low & Low & Socio-Economic & NA & NA & 2017-09-26 & 2017 & 2 & ECO
Frame & v.low & Regional & <500 & 0.0141942 & -0.1980960 & -0.1536394 & -0.9382144 & 0.7927531 & 12.6 & 1.2981504 & 0.5794339 & Recipient & Domestic & Domestic & Domestic & Domestic|ECO & Positive\\
Spain & http://www.lavanguardia.com/local/sevilla/20140816/54413097581/ayuntamiento-de-cadiz-preve-que-el-yacimiento-gadir-alcance-los-20-000-visitantes-este-mes-de.html & 687 & LaVanguardia & Private/Non-Public & Online and Offline & National & very low = CP mentioned once & Cultural heritage & Positive & EU + Subnational & No myth & NA & NA & NA & NA & NA & NA & NA & NA & Spain & ayuntamiento de cádiz prevé que el yacimiento 'gadir' alcance los 20.000 visitantes este mes de agosto & 2014-08-16 & fondo europeo de desarrollo regional & cádiz, 16 (europa press) el yacimiento arqueológico 'gadir' ha recibido un total de 18.885 visitas hasta la primera semana del presente mes de agosto, mes en el que el ayuntamiento de cádiz prevé que se alcance la cifra de las 20.000 visitas. el 48 por ciento de las 18.885 personas que han visitado el yacimiento desde que se abriera a finales de marzo de 2014 hasta la primera semana de agosto del presente año son vecinos de la ciudad de cádiz. según los datos de visitas al yacimiento arqueológico 'gadir', a los que ha tenido acceso europa press, el mes en el que el yacimiento ha recibido más visitas desde que abriera es el de julio, cuando un total de 5.053 visitantes se acercaron a ver el trazado urbano fenicio que se encuentra justo debajo del teatro de títeres de la tía norica. precisamente en el mes de julio es cuando cambio la tendencia del número de visitas, en descenso desde que abril se registrara el acceso de un total de 4.684 personas para contemplar el yacimiento. en mayo fueron 4.513 las personas que observaron los restos de la antigua ciudad fenicia construida en el siglo ix a.c., mientras que en junio esta cifra descendió hasta los 3.472 visitantes. el teniente de alcaldía y delegado de fomento y turismo del ayuntamiento de cádiz, bruno garcía, afirma en declaraciones a europa press que estos datos "son prueba de la buena acogida que está teniendo el yacimiento, que se ha convertido en una de las joyas de la oferta turística y cultural de la ciudad de cádiz". por procedencia la mayoría de visitantes del yacimiento arqueológico 'gadir' (9.006 de 18.885) son vecinos de la ciudad de cádiz, mientras que 3.284 son personas que residen en algún punto de la provincia de cádiz. de las restantes siete provincias andaluzas, han acudido a ver los restos de viviendas fenicias más antiguos del mediterráneo occidental un total de 1.613 personas. del resto del territorio nacional han llegado a visitar el yacimiento 3.363 personas --el segundo grupo, por procedencia, más numeroso--; del resto de europa 1.031 y del resto del mundo 588. el yacimiento la recuperación del yacimiento fue gestionada por el ayuntamiento de cádiz a través del instituto de fomento, empleo y formación (ifef) y se enmarca en el proyecto 'namae', cofinanciado en un 75 por ciento por la unión europea a través del fondo europeo de desarrollo regional feder 2007-2013 y un 25 por ciento por el instituto de empleo y desarrollo socioeconómico y tecnológico de diputación (iedt) de la diputación de cádiz en el marco del 'programa operativo de cooperación transfronteriza españa fronteras exteriores' (poctefex). la recuperación y adaptación de este espacio para su apertura al público supuso una inversión de 949.826 euros, y se enmarca en la apuesta del ayuntamiento de cádiz por la recuperación de su legado histórico. el yacimiento fenicio 'gadir', situado bajo el antiguo teatro de títeres de la tía norica, recrea la cádiz fenicia y muestra al público los tramos de dos calles y de ocho viviendas datadas en el siglo ix a.c., así como algunos restos humanos de fenicios del siglo vi a.c. además de las antiguas estructuras arquitectónicas, el visitante puede contemplar el rostro de mattan, un fenicio que murió en un gran incendio que tuvo lugar en la ciudad allá por el siglo vii a.c., y del cual se ha hecho una reconstrucción facial empleando la tecnología digital y forense "más avanzada". este yacimiento está considerado "el más importante de la cuenca mediterránea occidental" porque "demuestra científicamente la antigüedad de cádiz, con una ocupación continuada de más de 3.000 años". la visita al yacimiento arqueológico 'gadir' es gratuita y ha de realizarse por grupos de entre 20 y 25 personas por un tiempo aproximado de entre 20 y 30 minutos. & 655 & very low & Low & Socio-Economic & NA & NA & 2014-08-16 & 2014 & 1 & ECO
Frame & v.low & National & 500-1000 & 0.0141942 & -0.1980960 & -0.1536394 & -0.9382144 & 0.7927531 & 12.6 & 1.2981504 & 0.5794339 & Recipient & Domestic & European & Mixed & Domestic|ECO & Positive\\
Spain & https://www.lavozdegalicia.es/noticia/galicia/2019/02/15/galicia-ante-pelea-fondos-ue/0003\_201902G15P11992.htm & 620 & La Voz de Galicia & Private/Non-Public & Online and Offline & Regional/Local & very high = CP is most important issue + CP is mentioned in title/headline & Institutional bargaining over funding & Negative & EU + National + Subnational & No myth & NA & NA & NA & NA & NA & NA & NA & NA & Spain & galicia se prepara para una batalla feroz por los fondos de la ue & 2019-02-15 & fondo social europeo & españa deberá negociar duro para lograr el desembolso de 2.771 millones para la comunidad galicia se la juega en el próximo marco presupuestario europeo (2021-2027). nunca antes la región se había visto obligada a lidiar con tantos frentes abiertos amenazando el futuro de sus fondos. se acabaron los años de bonanza y mimos de la ue. el brexit, el lastre de la crisis, la anémica inversión pública, la burocracia, la crisis demográfica y el hambre de recortes de bruselas añaden nubarrones al futuro de la comunidad. ¿qué amenazas se ciernen sobre los fondos? la ue ansía recortar las ayudas a los programas de cohesión y desarrollo regional, de los que se alimenta galicia. la presión de las capitales más ricas y el auge de los partidos eurófobos obligará a los gobiernos a pasar la tijera a esta partida presupuestaria (34 \% del total del presupuesto de la ue). otro de los peligros es el brexit. el divorcio británico dejará un agujero de entre 10.000 y 15.000 millones de euros anuales en la hucha comunitaria, que nadie está dispuesto a colmar. bruselas no tiene previsto aumentar las ayudas existentes del fondo europeo marítimo y pesquero (femp) para compensar la eventual expulsión de la flota gallega de aguas británicas. se inclina por suplir los efectos del brexit endureciendo las condiciones de financiación a las regiones como la gallega que, tras casi tres décadas de apoyo financiero europeo, no han sido capaces de alcanzar el vagón de cabeza en europa. galicia, que atraviesa años de sequía en sus inversiones públicas (cayeron un 65\% durante la crisis), tendría graves problemas para poder poner en marcha los planes operativos. en el peor de los casos, la xunta estima que la región podría perder el 46\% de los 2.771 millones de euros en ayudas estructurales que venía recibiendo en los últimos siete años. ¿a qué máximo puede aspirar galicia? a quedarse como está. ese es el escenario preferido de la eurocámara que esta semana echó un pulso a la comisión y el consejo para defender el mantenimiento de los actuales niveles de ayuda a las regiones que caen de categoría, como el caso galaico. la lista de deseos incluye un aumento de la cofinanciación, eliminar cualquier condicionalidad de cumplimiento del déficit para el desembolso de los fondos, reducir la burocracia, ampliar el calendario de ejecución y poner en la balanza los efectos de la sangría demográfica, que tanto castiga a galicia. las negociaciones sobre los presupuestos siguen estancadas en bruselas por el caos que se asoma con el brexit y la proximidad de las elecciones europeas. a pesar de que españa tendrá que batallar duro en las negociaciones, cuando estas cojan velocidad, el gobierno contará con el parlamento europeo como aliado. la institución aceptó ayer la propuesta promovida por el grupo de expertos de cohesión del consejo de municipios y regiones (cmre) de dedicar al reto demográfico un mínimo del 5 \% de los fondos europeos de desarrollo regional (feder) que reciba españa. las comarcas con población inferior a los 12.5 habitantes por kilómetro cuadrado o con un descenso anual del 1 \% de sus habitantes desde el inicio de la crisis deberían estar sujetas a planes nacionales y regionales con financiación específica. es poco probable, pero si las cosas no se tuercen, los gallegos podría seguir recibiendo 913.6 millones de euros del feader, 322.2 millones de euros del fondo social europeo y 371 millones del femp. ¿por qué necesita galicia los fondos europeos? existe un déficit de inversión pública en ciertas áreas fundamentales para el desarrollo de la región. el más importante tiene que ver con el atraso tecnológico que sufren las pymes. en este sentido, será crucial mantener los fondos feader intactos. si esa partida se cercena, galicia perderá el tren de la innovación. la propia industria clama por invertir más esfuerzos en el impulso de los sectores de alto valor añadido, como el tecnológico. en ese fondo está el germen de la economía gallega del futuro. otro flanco que es importante cubrir es el del desempleo (en el 12\%) y ahí será crucial jugar bien las cartas para que el fondo social europeo (fse) no acabe siendo un recuerdo de la época de bonanza. las estadísticas indican que galicia ha sido una de las regiones españolas donde la crisis se ha cebado más con los salarios. en un entorno de pérdida neta de población y estancamiento económico con respecto a otras regiones europeas, la red de seguridad del fse es vital. bruselas ya lo tuvo en cuenta cuando en sus cuentas aumentó los fondos a españa en un 5 \%. ¿por qué razón? por la precaria situación del mercado laboral y su alta cifra de desempleo (14.3 \%). & 789 & very high & High & Power & NA & NA & 2019-02-15 & 2019 & 3 & POL
Frame & high-very high & Regional & 500-1000 & 0.0141942 & -0.1980960 & -0.1536394 & -0.9382144 & 0.7927531 & 12.6 & 1.2981504 & 0.5794339 & Recipient & Domestic & European & Mixed & Domestic|POL & Negative\\
Spain & http://www.elmundo.es/internacional/2015/02/25/54ecaa2d268e3ecf2a8b4579.html & 661 & EL MUNDO & Private/Non-Public & Online and Offline & National & low = CP mentioned more times but NOT important part of story (mainly about others issues) & Fraud/Corruption & Negative & Other country & 7.Fraud & NA & NA & NA & NA & NA & NA & NA & NA & Spain & los símbolos de la corrupción política y la evasión fiscal & 2015-02-25 & fondos estructurales & la plaza ante el parlamento en atenas, escenario de concentraciones contra la troika y a favor de tsipras. reuters en la hora once, pero llegó a tiempo. la lista de reformas exigida por el eurogrupo el pasado viernes al nuevo gobierno griego a cambio de una prórroga de cuatro meses del rescate en vigor llegó a bruselas el lunes por la noche. el documento de seis folios filtrado horas antes, con un proyecto de reforma de la administración pública y el plan más serio aprobado nunca por atenas contra la corrupción y la evasión fiscal, no convenció a algunos miembros del eurogrupo por falta de concreción. devuelto y retocado por el equipo del ministro griego yanis varufakis, la versión final estaba lista en la madrugada del martes en la mesa del presidente del eurogrupo. faltaba el visto bueno, por teleconferencia, de los 18 ministros de finanzas y, si como parecía desde primera hora, recibía luz verde, se iniciará el proceso de ratificación de la prórroga del rescate en los parlamentos nacionales donde es preceptivo (alemania, finlandia, estonia y países bajos) y en los gobiernos de los otros 14 miembros del club. para varufakis, "es una lista muy completa de reformas" y "se ha acabado la austeridad con piloto automático". para el primer ministro griego, alexis tsipras, "grecia ha recuperado la soberanía nacional perdida". para la troika, rebautizada como "las instituciones", tsipras simplemente ha vuelto al redil, aunque tenga que defender lo indefendible dentro de casa. los cuatro meses de ampliación conseguidos serán un vía crucis, con vigilancia y exámenes igual o más rigurosos que los de la troika, y pueden acabar en nuevas elecciones anticipadas o con grecia fuera de la unión. de momento, la ue ha preferido el mal menor a una crisis que hubiera dejado fuera de la unión al país que, hace un año por estas fechas, la estaba presidiendo a pesar de todos sus males. de acuerdo con eeuu, las grandes potencias europeas han preferido no perder el control de un país crucial para la seguridad occidental en el mediterráneo oriental desde hace 70 años. en su defensa y en la de turquía, frente al comunismo soviético, nació, en los orígenes de la guerra fría, la doctrina truman y, en el pulso con la nueva rusia por la revisión del mapa de europa, lo último que necesita occidente es una grecia en el limbo a merced de créditos rusos o chinos como la venezuela poschavista. habiendo recibido ya unos 240.000 millones de euros en tres años, hubiera sido absurdo dejar caer a la banca y al gobierno griegos por los 18.000 millones pendientes de cobro. grecia y la ue tienen ahora cuatro meses para diseñar un nuevo plan que, además de reducir el déficit y la deuda, devuelvan a los griegos un mínimo de estabilidad y de dignidad. la lucha contra la corrupción y contra la evasión fiscal es sólo un símbolo de propósito de enmienda de lo peor que ha ofrecido grecia al mundo en los últimos decenios, pero nada de lo que ha pasado en esta crisis ni su solución son posibles sin hacer frente a ese desafío. con los nuevos compromisos, el gobierno de tsipras sencillamente está institucionalizando y reforzando lo que ya empezó la coalición saliente. el ex ministro de transporte, michalis liapis, fue detenido el año pasado por conducir sin carnet y con matrículas falsas, pero en la investigación abierta se demostró que llevaba mucho tiempo desviando fondos estructurales de la ue para sus fincas, palacete y cuentas particulares. uno más de muchos. a vasilis papageorgopoulos, ex alcalde de tesalónica, le cayó una cadena perpetua en febrero de 2013 por malversación de fondos públicos. al otrora todopoderoso ministro de defensa akis tsochatzopoulos, ex número dos del pasok de papandreu, lo condenaron a veinte años en octubre de 2013 por lavado de dinero. cuando empezaron a rodar cabezas, hace dos años, grecia recordó a muchos la italia de "manos limpias" en vísperas de la toma del poder por silvio berlusconi, a comienzos de los noventa. sus jueces se presentaban como una nueva generación de jóvenes salvadores de la patria. gran error. la judicatura griega es una institución tan corrupta o más que la clase política barrida en las últimas elecciones por la macrocoalición syriza, refugio de docenas de grupos, movimientos y asociaciones de izquierda, un arcoíris sin parangón en ningún otro país europeo, al que, por oportunismo, trata de imitar, mejor o peor, podemos en españa. abusando de su poder, los jueces han tumbado todos los recortes salariales que les afectaban personalmente, mientras en el resto de la administración se recortaban drásticamente. son los mismos jueces que dejaron escapar, durante un permiso, a christodoulos xiros, condenado a seis cadenas perpetuas por 33 ataques terroristas, en los que murieron seis personas, del grupo noviembre 17. @sahagunfelipe & 805 & low & Low & Governance & NA & NA & 2015-02-25 & 2015 & 1 & POL
Frame & low-medium & National & 500-1000 & 0.0141942 & -0.1980960 & -0.1536394 & -0.9382144 & 0.7927531 & 12.6 & 1.2981504 & 0.5794339 & Recipient & European & European & European & European|POL & Negative\\
Spain & https://elpais.com/sociedad/2018/11/20/actualidad/1542740767\_848060.html & 668 & EL PAÍS & Private/Non-Public & Online and Offline & National & medium = CP is important part of story & Territorial cooperation & Positive & EU + National & No myth & NA & NA & NA & NA & NA & NA & NA & NA & Spain & la raya: tierra de los 50 entierros y los 13 bautizos & 2018-11-20 & política de cohesión & los municipios de la frontera entre españa y portugal pierden población y denuncian abandono en rabanales hay unas 300 farolas para algo más de 500 habitantes. "en unos años tendremos más farolas que personas", cuenta el alcalde, domingo ferrero. "prefiero ni mirar el número de vecinos mayores de 80", dice. el municipio forma parte de la comarca de aliste (zamora), que aglutina unos 70 pueblos repartidos en más de una decena de ayuntamientos y linda con portugal, una frontera que se conoce como la raya. son el límite de un país, de una comunidad autónoma, de una provincia. la periferia de la periferia. y tienen los mismos problemas que sus vecinos portugueses: viven en un lugar donde muchos mueren, pocos nacen y los jóvenes se marchan. más información el 30\% del territorio español concentra el 90\% de la población "la demografía no admite más demoras, hay que actuar ya" las casas cuartel echan el cierre extremadura se ahoga vivir como un cura ya no es lo que era en estas tierras de castaños y de robles el diagnóstico es compartido, no tanto las soluciones. sus habitantes se quejan de las trabas burocráticas para abrir negocios y explotaciones ganaderas, de la baja calidad del tendido eléctrico, de la mala cobertura y conexión a internet. de vivir "olvidados" por las grandes compañías y capitales. con sus pensiones o sus trabajos, con sus cafés en el bar o las compras en la tienda, mantienen las constantes vitales de este enfermo rural que sufre una sangría demográfica. zamora es uno de esos puntos oscuros de la despoblación en españa. fue la provincia que, porcentualmente, más población perdió el año pasado, según el ine: el 1,5\%. unos 2.600 habitantes. en las provincias de la raya, el 94\% de los municipios tiene más mayores de 65 años que menores de 15. el 84\% ha perdido población entre 2000 y 2017. el 90\% tiene menos de 5.000 habitantes y la mitad, menos de 500. en siete de cada 10 hay más hombres que mujeres. "salvo en algunos puntos como la zona de influencia de vigo, badajoz o el valle del guadiana, casi 1.000 de los más de 1.200 kilómetros de frontera son carne de cañón. se pierde población desde los sesenta", apunta lorenzo lópez trigal, catedrático de geografía humana de la universidad de león. "en el ángulo entre zamora, ourense y el nordeste de portugal se ve. en el pasado se dejó de lado la actividad económica, no se creó industria", prosigue. premiar al emprendedor domingo ferrero, de 54 años, está al frente del ayuntamiento de rabanales desde 2007, primero por el psoe y después por el pp. seis pueblos conforman el municipio. el mayor, con el mismo nombre, tiene unos 200 habitantes. la escuela cerró. "hay dos velatorios y una residencia", indica. "los jóvenes se marchan a estudiar y aquí no hay trabajo. podrá venir algún abogado, pero no 10", continúa. "hay que mejorar la conexión a internet; terminar la concentración parcelaria; ayudar al emprendedor. si alguien quiere montar un negocio aquí, habría que darle un premio", pide. una estrategia conjunta españa y portugal celebran este miércoles en valladolid la xxx cumbre hispano portuguesa, un encuentro en el que se abordará, entre otros asuntos, el reto demográfico. "es considerado prioritario establecer una estrategia entre ambos países, que pueda combatir la despoblación de los territorios de frontera, que pueda tener influencia en la negociación de los instrumentos de política de cohesión territorial, económica, social y ambiental con la unión europea", sostienen fuentes de la secretaría de estado de la valorización del interior portuguesa, creada en octubre. en la raya, varias asociaciones potencian la cooperación transfronteriza entre ambos países, lanzando proyectos conjuntos. por ejemplo, las agrupaciones europeas de cooperación territorial zasnet o duero-douro. "los problemas que tenemos son prácticamente los mismos. nos hemos dado la espalda mucho tiempo y ahora se está avanzando, pero aún hay terreno por recorrer", apunta josé luis pascual, director general de duero-douro. "todo cuesta el doble de esfuerzo. yo pago los mismos impuestos que en una ciudad, pero en madrid pasarían miles de personas delante de mi restaurante y aquí, 20", se queja josé martín, de 47 años, que vive en san vitero, de unos 250 habitantes. míriam moral, de 32, trabaja en otro restaurante en rabanales, un "negocio familiar". allí lidian con la cobertura: "a veces nos fallan los datáfonos. y en febrero tuvimos una avería en el pueblo y tardaron una semana en repararla: una semana sin conexión". su negocio "funciona bien", sobre todo gracias a la gente de fuera. "aquí hay mucha caza, y la comarca forma parte de la reserva de la biosfera transfronteriza meseta ibérica. el turismo rural empieza a moverse", apunta moral. para que despegue, el alcalde de rabanales reclama infraestructuras, "como la señalización de rutas". javier faúndez (pp), presidente de la mancomunidad tierras de aliste, que agrupa a 12 ayuntamientos y unos 60 pueblos, va en la misma línea: "oportunidades hay, pero también demasiada burocracia. no hay fondos a proyectos que creen empleo. desde el principio debería haber ayudas". "necesitamos una fiscalidad favorable, bonificaciones y mejor financiación para pequeños municipios", zanja. ángel mezquita, de 42 años, es la prueba andante de lo que cuesta emprender. "tengo una explotación de cerdos. tardaron un año en darme el permiso para empezar a construir. tardé otro en arrancar. yo pude hacerlo porque me ayudaron mis padres, pero es demasiado tiempo", cuenta. vive en san juan del rebollar, un pueblo que depende de san vitero y en el que viven unos 160 habitantes. allí ya no hay tienda. a estas localidades les cuesta fijar población. en 2016, san vitero lanzó una campaña para atraer familias y lograr que el colegio no cerrara. ahora hay siete alumnos de tres a 11 años. están a la última: pizarra digital, casi un ordenador por niño. "estamos como en familia", dice la directora, yoana prieto, de 38 años. va y viene cada día desde zamora: "allí tengo mi vida, allí nació mi hija. y 45 minutos no es tanto". fueron los padres y el alcalde los que más se movieron para salvar el centro. ofrecieron dos casas gratis. hubo 200 solicitudes. llegaron dos familias. una no se adaptó. otra sigue allí, en la casa de la ermita. "cuando vimos la oportunidad, no lo dudamos, pero hay que combatir la imagen idílica de los pueblos. tiene que gustarte esta vida, pasas mucho tiempo sola", afirma esmeralda folgado, de 36 años, que está "encantada". "y si quieres ir al teatro, lo tienes solo a 45 minutos", añade. reconoce, sin embargo, limitaciones, como que el pediatra solo vaya dos veces a la semana. si los niños se ponen malos, tiene que ir a alcañices. esta localidad, con unos mil habitantes, es la cabecera de comarca. en palabras de su alcalde, jesús maría lorenzo mas (72 años, del pp), tampoco se libran de la despoblación, aunque aquí "el éxodo ha sido menor". "si lográramos que unos 100 funcionarios o empleados que trabajan aquí se mudaran, ganaríamos mucho. pero la gente se marcha. se matan por conseguir una plaza de guardia civil, por ejemplo, y después se van a vivir a zamora", se queja. alcañices linda con portugal. luis augusto lucas, de 65 años, preside la junta de freguesia (concejo vecinal) de são martinho de angueira, que con unos 200 habitantes depende de miranda do douro, uno de los municipios que más población ha perdido en el país luso en los últimos años. se diferencia poco de pueblos españoles con los que hace frontera. también ellos reclaman trabajo, oportunidades. "en el mundo rural no hay que defender la sostenibilidad, sino la justicia social", defiende teófilo nieto, cura en 15 pueblos de aliste. llegó en 1995 y lo ha visto evolucionar. mientras avanza por la calle, es capaz de contar las casas que ha ido cerrando. el año pasado, 50 entierros frente a 13 bautizos. "de ellos, solo un niño vivirá aquí", dice. confía en que la solución llegará. cree en las posibilidades de david contra goliat. y cita a josé antonio labordeta: "ni tú, ni yo, ni el otro lo lleguemos a ver, pero habrá que empujarlo para que pueda ser". & 1369 & medium & Medium & Socio-Economic & NA & NA & 2018-11-20 & 2018 & 3 & ECO
Frame & low-medium & National & +1000 & 0.0141942 & -0.1980960 & -0.1536394 & -0.9382144 & 0.7927531 & 12.6 & 1.2981504 & 0.5794339 & Recipient & Domestic & European & Mixed & Domestic|ECO & Positive\\
\addlinespace
Spain & http://www.elmundo.es/comunidad-valenciana/castellon/2017/09/07/59b18ef4268e3eab208b4775.html & 662 & EL MUNDO & Private/Non-Public & Online and Offline & National & very low = CP mentioned once & Research \& innovation & Positive & National + Subnational & No myth & NA & NA & NA & NA & NA & NA & NA & NA & Spain & ube investiga nuevos aditivos poliméricos para tintas ink-jet de efecto cerámico & 2017-09-07 & fondo europeo de desarrollo regional & esta actividad de innovación se está desarrollando en el centro de i+d de ube la empresa ube corporation europe está llevando a cabo desde 2016 y hasta 2018 un proyecto de innovación denominado 'desarrollo de nuevos aditivos poliméricos para tintas ink-jet de efecto cerámico', cuyo principal objetivo es la síntesis de poliméricos, de uso en procesos de fabricación de tintas cerámicas no pigmentadas. dichos aditivos permiten compatibilizar la parte orgánica de las tintas con los componentes de uso cerámico que se utilizan en las denominadas tintas 'de efecto', compuestas fundamentalmente por materias primas cerámicas y/o por fritas, ofreciendo unas mejores prestaciones técnicas, según ha informado la empresa en un comunicado este jueves. con este proyecto ube busca una mayor diversificación de su cartera de productos, partir de materias primas y subproductos producidos en la actualidad por la propia empresa, entrando en el campo de los aditivos para tintas. esta actividad de innovación se está desarrollando en el centro de i+d de ube en castellón, y está cofinanciada por el fondo europeo de desarrollo regional (feder), dentro del programa operativo de crecimiento inteligente 2014-2020, con el objetivo depotenciar la investigación, el desarrollo tecnológico y la innovación. & 200 & very low & Low & Socio-Economic & NA & NA & 2017-09-07 & 2017 & 2 & ECO
Frame & v.low & National & <500 & 0.0141942 & -0.1980960 & -0.1536394 & -0.9382144 & 0.7927531 & 12.6 & 1.2981504 & 0.5794339 & Recipient & Domestic & Domestic & Domestic & Domestic|ECO & Positive\\
Spain & http://www.telecinco.es/informativos/economia/Hacienda-contrata-auditoras-controlar-UE\_0\_2185575120.html & 602 & Telecinco & Private/Non-Public & Online only & National & high = CP is most important issue in story (can also cover other issues) & Improve governance & Factual & EU + National & No myth & NA & NA & NA & NA & NA & NA & NA & NA & Spain & hacienda contrata por 210.000 euros a varias auditoras para controlar el uso de los fondos de la ue & 2016-05-26 & fondo social europeo & en concreto, las auditoras adjudicatarias, entre las que se encuentra deloitte, se encargarán de realizar controles financieros sobre proyectos financiados a través del fondo europeo de desarrollo regional (feder), el fondo social europeo (fse), el fondo europeo de pesca (fep) y el feder po españa-fronteras exteriores (poctefex). en el caso de los fondos feder y fep se realizarán 80 controles en cada uno de ellos y la auditora encargada será deloitte por importe de 111.804 euros (55.902 euros en cada uno de los fondos). además, la auditora abante amb auditores llevará a cabo también 50 controles más en los fse por 33.500 euros. esta misma auditora controlará, asimismo, operaciones cofinanciadas por los fondos españa-fronteras exteriores (25 controles) por 14.973 euros. por último, una ute formada por audipublic auditores, compañía de auditoría consejeros auditores, auditeco y seiquer auditores y consultores llevará a cabo 70 controles en los fondos fse en materia de lucha contra la discriminación y asistencia técnica por importe de 49.973 euros. & 170 & high & High & Governance & NA & NA & 2016-05-26 & 2016 & 2 & POL
Frame & high-very high & National & <500 & 0.0141942 & -0.1980960 & -0.1536394 & -0.9382144 & 0.7927531 & 12.6 & 1.2981504 & 0.5794339 & Recipient & Domestic & European & Mixed & Domestic|POL & Neutral\\
Spain & http://www.europapress.es/madrid/noticia-madrid-ultima-estrategia-localizacion-objetivos-desarrollo-sostenible-ods-20180116181828.html & 637 & europa press & Private/Non-Public & Online only & National & very low = CP mentioned once & Environment/green/low-carbon & Positive & EU + National + Subnational & No myth & NA & NA & NA & NA & NA & NA & NA & NA & Spain & madrid ultima su estrategia de localización de los objetivos de... & 2018-01-16 & política de cohesión & madrid, 16 ene. (europa press) - el ayuntamiento de madrid está inmerso en la redacción de una estrategia de 'localización de los objetivos de desarrollo sostenible (ods) en la ciudad de madrid con la intención de presentarla a los actores sociales y políticos a finales de este mes para su discusión y posterior aprobación. "es un compromiso muy importante de nuestra generación para acabar con la pobreza y es imprescindible que haya un proceso de sensibilización social", ha indicado este martes la alcaldesa, manuela carmena, en soria, durante la reunión anual del consejo político del consejo de municipios y regiones de europa (cmre), un encuentro que reúne a más de 250 líderes locales y regionales que representan a más de 130.000 gobiernos locales y regionales de nuestro continente. bajo el lema think europa. 'compromiso 2030', la agenda del consejo se centra en la estrategia europea de implementación de la agenda 2030 en el marco de la futura política de cohesión de la ue. la inauguración ha corrido a cargo de la vicepresidenta del gobierno, soraya sáenz de santamaría. "los ods representan un compromiso muy importante de nuestra generación para acabar con la pobreza y es imprescindible que haya un proceso de sensibilización social", ha indicado la alcaldesa momentos antes del inicio de la mesa 'el impacto de la política internacional y de cooperación', en el que ha compartido experiencias con carlos martínez, alcalde de soria y vicepresidente europeo de cglu; leire pajín, del instituto de salud global; antonio fernández galiano, presidente de unidad editorial; joan clos, ex director ejecutivo de onu-habitat, y luis tejada, director de la agencia española de cooperación internacional para el desarrollo. carmena ha reiterado el compromiso de madrid con los principios y valores de la agenda 2030 que enlazan con la apuesta del gobierno municipal por el futuro de la ciudad. "en el ayuntamiento de madrid consideramos que los gobiernos municipales, al ser las administraciones más cercanas a la ciudadanía y tener competencias clave para la promoción del desarrollo sostenible, tienen que asumir el liderazgo en el compromiso público de las transformaciones que establece la agenda", ha apuntado. la futura estrategia asume el enfoque estratégico de la agenda 2030: multidimensional, integral, universal, multinivel y multiactor. "la agenda lo deja claro, las metas son universales y debemos articularlas desde todos los niveles de la administración y con la participación de todos los actores", ha indicado manuela carmena. ayuntamiento de madrid & 404 & very low & Low & Socio-Economic & NA & NA & 2018-01-16 & 2018 & 3 & ECO
Frame & v.low & National & <500 & 0.0141942 & -0.1980960 & -0.1536394 & -0.9382144 & 0.7927531 & 12.6 & 1.2981504 & 0.5794339 & Recipient & Domestic & European & Mixed & Domestic|ECO & Positive\\
Spain & http://www.abc.es/internacional/abci-dastis-asegura-espana-esta-dispuesta-aportar-mas-presupuesto-201803201924\_noticia.html & 685 & ABC TU DIARIO EN ESPAÑOL & Private/Non-Public & Online and Offline & National & medium = CP is important part of story & Institutional bargaining over funding & Negative & EU + National & No myth & NA & NA & NA & NA & NA & NA & NA & NA & Spain & dastis asegura que españa está dispuesta a aportar más al presupuesto de la ue & 2018-03-20 & política de cohesión & el ministro confía en que la postura "constructiva" de londres facilite un acuerdo sobre el futuro de gibraltar el ministro de asuntos exteriores, alfonso dastis, aseguró este martes en el congreso de los diputados que españa está dispuesta a aumentar su contribución al presupuesto de la unión europea, siempre que se garantice la atención a lo que calificó de "nuevas prioridades" y se mantenga una adecuada financiación de la pac (política agrícola común) y las políticas de cohesión. dastis compareció ante la comisión mixta congreso-senado para la ue con el fin de explicar la situación de las negociaciones del brexit y otras cuestiones relacionadas con la unión europea, entre ellas el debate sobre el próximo marco financiero plurianual, para el periodo 2021-2027. en este último punto, el ministro reconoció que la salida del reino unido del club comunitario tendrá efectos negativos, como por ejemplo que dejarán de entrar ingresos por entre 10.000 y 14.000 millones de euros. "si se quiere mantener activa la ue hay que cubrir ese déficit y españa está dispuesta a aumentar su contribución. parece lógico que así sea, porque hay que modernizar las políticas", dijo dastis. sin embargo, señaló que "es pronto para saber si españa se convertirá en contribuyente neto", como consecuencia de esa mayor aportación, porque eso dependerá de los equilibrios que se establezcan. en cualquier caso, el titular de exteriores insistió en que españa pondrá el acento en que, durante la negociación del marco financiero que comenzará el 2 de mayo, se garantice la financiación de las "nuevas prioridades", entre las que citó la seguridad y la defensa y la protección de las frontera. al propio tiempo, las autoridades españolas son conscientes de que va a haber una dura batalla en torno a la financiación de la pac y de las políticas de cohesión de las que españa se ha venido beneficiando desde su ingreso en la ue. en concreto dastis, que cree que todo esto ha de ser posible realizarlo con un presupuesto anual que ascienda al 1,1 por ciento del pib europeo, subrayó ante los parlamentarios que la pac es una importante política de cohesión, que previene la despoblación rural y garantiza la seguridad alimentaria. por ese motivo, el gobierno, aunque apuesta por la modernización de la pac y por adaptarla a nuevos retos como la lucha contra la desertización o el cambio climático, es contrario a su renacionalización y a la cofinanciación nacional de los pagos directos. por otra parte, respondiendo a preguntas de los parlamentarios, el ministro explicó alguno de los aspectos del brexit que afectan a gibraltar, insistiendo en que lo que españa desea es que no hay efectos negativos en la situación de los trabajadores transfronterizos y, si es posible, que mejore esa situación, manteniendo una posición pragmática compatible con no renunciar a la reclamación sobre la soberanía del peñón. dastis reiteró que españa no es contraria a que se aplique a gibraltar el régimen del periodo transitorio acordado entre la comisión europea y el reino unido, pero recordó que eso será fruto de la negociación que se mantiene con londres. según dijo, ha observado en el gobierno británico una actitud "receptiva y constructiva" para llegar a un acuerdo, lejos de las amenazas llegadas desde la colonia en el sentido de posibles rescisiones de contratos a los trabajadores que cruzan la verja o de emprender acciones legales. "ese no es mi discurso -dijo-. si ellos lo hacen así son ellos quienes ponen en peligro el futuro de la relación". dastis, que confía en acabar esas negociaciones con el reino unido en octubre, reconoció que uno de los asuntos que están sobre la mesa es la utilización conjunta del aeropuerto de gibraltar, construido sobre el istmo, una lengua de tierra no cedida por españa en el tratado de utrecht, pero que fue ocupada ilegalmente por los británicos a lo largo del siglo xix. además, apunto otros asuntos como el contrabando de tabaco, una mayo transparencia fiscal y resolver cuestiones medioambientales. & 667 & medium & Medium & Power & NA & NA & 2018-03-20 & 2018 & 3 & POL
Frame & low-medium & National & 500-1000 & 0.0141942 & -0.1980960 & -0.1536394 & -0.9382144 & 0.7927531 & 12.6 & 1.2981504 & 0.5794339 & Recipient & Domestic & European & Mixed & Domestic|POL & Negative\\
Spain & http://www.diariosur.es/economia/201607/30/bruselas-amenaza-espana-recortar-20160730132557-rc.html & 627 & Sur & Private/Non-Public & Online and Offline & Regional/Local & very high = CP is most important issue + CP is mentioned in title/headline & Political leverage & Negative & EU + National & No myth & NA & NA & NA & NA & NA & NA & NA & NA & Spain & bruselas amenaza a españa con recortar fondos estructurales si no pone orden en sus cuentas & 2016-07-30 & fondos estructurales & el comisario europeo para la economía digital aumenta la presión pese a que la comisión europea decidió esta semana anular la multa por incumplir los compromisos de reducción del déficit el comisario europeo para la economía digital, el alemán günhter oettinger, ha avisado hoy a españa y portugal de que deben "poner orden con urgencia" en sus cuentas públicas si no quieren "poner en riesgo" las transferencias millonarias de los fondos estructurales. "tenemos una segunda opción. nuestra amenaza de recortar los fondos estructurales del presupuesto europeo va en serio", ha subrayado oettinger en unas declaraciones al semanario alemán 'der spiegel', después de que la comisión europea (ce) decidiera esta semana anular la multa a españa y portugal por incumplir sus compromisos de reducción de déficit público. oettinger ha apuntado que en la reunión del colegio de comisarios en la que se adoptó esa decisión él defendió "claramente" la imposición de multas "moderadas" y asegura que otros colegas también abogaron por sanciones en aplicación del pacto de estabilidad y crecimiento. este acuerdo, afirma, no está muerto, ya que todavía queda la opción de recortar los fondos estructurales que reciben los dos países si no cumplen las recomendaciones y los objetivos presupuestarios marcados. dos años adicionales la reunión de los comisarios fue precedida, según diversas fuentes, de llamadas del ministro alemán de finanzas, wolfgang schäuble, a algunos comisarios "populares" para evitar las multas. oettinger ha reconocido que en el debate de las sanciones unos y otros han intentando pasarse la patata caliente y ha subrayado la necesidad de evitar que la política partidista desempeñe cada vez un papel mayor en europa. en su reunión del pasado miércoles, que duró cerca de tres horas, la comisión europea decidió finalmente por consenso cancelar las multas a españa y portugal a cambio de aplicar duros ajustes. españa recibió dos años adicionales para situar su déficit público por debajo del 3\% del pib, algo que madrid tenía que haber hecho inicialmente este año. según el nuevo calendario, el país deberá rebajar su déficit desde el 5,1\% actual al 4,6\% del pib en 2016, al 3,1\% en 2017 y al 2,2\% en 2018. & 360 & very high & High & Power & NA & NA & 2016-07-30 & 2016 & 2 & POL
Frame & high-very high & Regional & <500 & 0.0141942 & -0.1980960 & -0.1536394 & -0.9382144 & 0.7927531 & 12.6 & 1.2981504 & 0.5794339 & Recipient & Domestic & European & Mixed & Domestic|POL & Negative\\
\addlinespace
Spain & http://www.larazon.es/economia/la-comision-europea-dice-que-suspension-de-fondos-a-espana-no-tendria-impacto-hasta-2020-CB13654745 & 638 & La Razon & Private/Non-Public & Online and Offline & National & very high = CP is most important issue + CP is mentioned in title/headline & Political leverage & Balanced & EU + National & No myth & NA & NA & NA & NA & NA & NA & NA & NA & Spain & la comisión europea dice que suspensión de fondos a españa no tendría impacto hasta 2020 & 2016-10-03 & fondos estructurales & sólo se tomaría esta medida si el país no adopta ninguna medida para cumplir, dice la comisaria europea de política regional la comisaria europea de política regional, corina cretu, afirmó hoy que la suspensión de fondos estructurales a españa "no tendría un impacto a corto plazo en absoluto" sobre la aplicación de los programas europeos hasta 2020 y "solo si el país no adopta ninguna medida". "si suspendemos una parte de los compromisos de 2017 esto afectará únicamente a finales de 2020, pero solo si el país no adopta ninguna medida para cumplir", afirmó cretu en una audiencia pública en el parlamento europeo (pe), el primer paso del diálogo entre ambas instituciones antes de que el ejecutivo tome una decisión. el pe debe posicionarse sobre la posible suspensión de fondos estructurales a españa por incumplir el déficit, un procedimiento inédito en la ue que presenta muchas incógnitas y que podría suponer la congelación de hasta un 20 \% de los compromisos de 2017. "la posible suspensión de los fondos no es una sanción, (...) forma parte de unas medidas más amplias y vinculadas a las recomendaciones especificas al país", afirmó cretu. recordó que se trata de una medida contemplada en el procedimiento de déficit excesivo abierto a españa y portugal, una vez que ambos países se libraron de una multa el pasado julio. en el caso de españa, esto podría suponer la congelación de hasta un 20 \% de los compromisos de 2017. según cretu, se trataría de "una medida temporal" que podría no llegar a materializarse si ambos países cumplen las condiciones acordadas con la ce. "la ce retirará esa suspensión en cuanto los países se ajusten a los criterios que están establecidos en la legislación", añadió, en referencia al umbral del 3 \% al que españa debe reducir su déficit en los próximos dos años. cretu también aseguró que "las condiciones económicas y sociales se van a tener en cuenta", en particular el empleo, lo que también puede llevar a la ce a "priorizar" si suspende unos u otros programas. la suspensión de compromisos presupuestarios -el importe máximo de pagos a los que se puede comprometer la unión europea en un periodo- podría suponer una reducción de la financiación de programas de empleo, agrícolas y regionales, entre otros. "habría que hacer elecciones difíciles, con una justificación sólida de por qué algunos programas se consideran más importantes que otros", precisó. por su parte, el vicepresidente de la ce para empleo, jyrki katainen, garantizó que "no se perderá ni un céntimo si los países cumplen con las normas". "todo está en manos de los estados miembros y una vez que hagan lo que han prometido, no se perderá ni un céntimo", dijo. españa tiene hasta el 15 de octubre para presentar un nuevo proyecto presupuestario para 2017, pero el bloqueo político y la incertidumbre sobre la formación de un nuevo gobierno genera dudas sobre la posibilidad de que el país cumpla con ese plazo. katainen afirmó que "la ce tiene la obligación legal de actuar" y que "no es una cuestión de deseo político". la ce no ha dado por el momento más pistas sobre la decisión que va a tomar. cretu aseguró que, por el momento, los países tienen "dinero que gastar". "si hay una pequeña parte que se suspende, aun así habría fondos remanentes", concluyó. & 553 & very high & High & Power & NA & NA & 2016-10-03 & 2016 & 2 & POL
Frame & high-very high & National & 500-1000 & 0.0141942 & -0.1980960 & -0.1536394 & -0.9382144 & 0.7927531 & 12.6 & 1.2981504 & 0.5794339 & Recipient & Domestic & European & Mixed & Domestic|POL & Neutral\\
Spain & http://www.europapress.es/cantabria/cantabria-social-00674/noticia-cantabria-gobierno-cantabria-quiere-potenciar-relacion-trabajo-coordinado-sector-social-20141217130804.html & 673 & europa press & Private/Non-Public & Online only & National & very low = CP mentioned once & Social awareness/inclusion & Positive & EU + National + Subnational & No myth & NA & NA & NA & NA & NA & NA & NA & NA & Spain & el gobierno quiere potenciar la relación con el sector social & 2014-12-17 & fondo social europeo & publicado 17/12/2014 13:08:04cet a través del programa operativo 2014-2020, cantabria buscará obtener de europa recursos para realizar programas con el sector santander, 17 dic. (europa press) - el gobierno de cantabria quiere "potenciar" el "trabajo coordinado" y la "relación" con el sector social de la comunidad autónoma y "garantizar" que ésta sea "estable" durante la próxima década gracias a la captación de recursos del fondo social europeo, a través del programa operativo 2014-2020 que la región ha presentado a la ue, para poder llevar a cabo programas con las organizaciones que integran este sector. así lo ha afirmado este miércoles el presidente de cantabria, ignacio diego, quien ha opinado que las organizaciones que componen este sector social y, más concretamente, las que atienden a personas con discapacidad, son un "auténtico privilegio" por las "capacidades demostradas año a año". profesionalización de personas con discapacidad intelectual y como ejemplo se ha referido a la asociación cántabra en favor de las personas con discapacidad intelectual (ampros), cuyo centro ocupacional ha visitado hoy para participar en la clausura de un taller de empleo sobre agricultura ecológica desarrollado por 15 personas con discapacidad intelectual y repartirles a todos ellos los certificados de profesionalidad que han obtenido. el director-gerente de ampros, roberto álvarez, ha explicado que estos certificados de nivel i son los mismos que los que se entregan a los alumnos que superan cualquier otro taller de empleo, independientemente de que sufran o no discapacidad intelectual. ampros ha sido la primera organización de este tipo que ha desarrollado un taller de empleo en cantabria, y ha supuesto la creación de un huerto ecológico en el que los participantes han podido formarse. el que acaba de concluir es el segundo taller desarrollado por la asociación, con un coste de 113.000 euros financiado por el fondo social europeo y el servicio cántabro de empleo, ya que primero hubo otro, también de agricultura ecológica, en el que participaron 8 alumnos. estos talleres tienen una metodología y didáctica "adaptada" a personas con discapacidad. los productos que obtienen de este huerto son usados en el servicio de catering que ofrece ampros, que reparte unos mil menús diarios, de los que aproximadamente el "40 o 50\%" va a colegios. actualmente, tienen como clientes de su servicio de catering a cinco colegios de la comunidad autónoma y, según el gerente de ampros, el objetivo es "crecer" y también "mejorar" la calidad de la alimentación que se ofrece en ellos. en declaraciones realizadas a los medios de comunicación previas al acto, diego ha resaltado que con ampros, que en 2015 cumplirá 50 años en cantabria, el gobierno regional tiene una relación de "cordial" y de "colaboración" de "muchos años" y que ha sido "proactiva". además, diego ha aprovechado para recordar que el gobierno de cantabria creará una nueva bonificación del 100\% en las donaciones al patrimonio protegido de las personas con discapacidad para aportaciones de hasta 100.000 euros; adelantará el pago de las subvenciones a los centros especiales de empleo para cubrir la mitad de los costes laborales, de modo que recibirán a primeros de cada año la cantidad correspondiente al primer semestre más la paga extra, y ha incrementado al máximo los mínimos. también ha insistido en que la reforma fiscal de 2015 supone un incremento de los mínimos familiares por discapacidad. por ello, y al igual que ya reivindicó en el acto celebrado en el parlamento regional con motivo del día internacional de las personas con discapacidad, el pasado 3 de diciembre, diego ha señalado que cantabria cuenta con la regulación de españa que "más favorece" a este colectivo. antes de la clausura, diego y la consejera de economía, hacienda y empleo, cristina mazas, se han reunido con el gerente de ampros y su presidenta, maría del carmen sánchez, y otros miembros de la directiva. sánchez ha resaltado la "buena formación" que supone para los participantes talleres de empleo como el desarrollado. & 659 & very low & Low & Socio-Economic & NA & NA & 2014-12-17 & 2014 & 1 & ECO
Frame & v.low & National & 500-1000 & 0.0141942 & -0.1980960 & -0.1536394 & -0.9382144 & 0.7927531 & 12.6 & 1.2981504 & 0.5794339 & Recipient & Domestic & European & Mixed & Domestic|ECO & Positive\\
Spain & http://www.europapress.es/extremadura/noticia-vara-defiende-futuro-pasa-facilitar-inversiones-eolica-fotovoltaica-retirar-impuesto-sol-20180626202625.html & 624 & europa press & Private/Non-Public & Online only & National & very low = CP mentioned once & Environment/green/low-carbon & Positive & Subnational & No myth & NA & NA & NA & NA & NA & NA & NA & NA & Spain & vara defiende que el futuro pasa por "facilitar inversiones en... & 2018-06-26 & política de cohesión & mérida, 26 jun. (europa press) - el presidente de la junta de extremadura, guillermo fernández vara, ha defendido este martes que el futuro en materia energética pasa por "facilitar las inversiones en eólica y fotovoltaica", así como por "retirar el impuesto al sol". según los datos que ha destacado, en españa "más del 40 por ciento de la energía eléctrica se produce con energías limpias", un porcentaje que ha tildado de "insuficiente", ya que europa "obliga a que ese 40 por ciento aumente hasta el 60 por ciento en 2030". fernández vara se ha pronunciado de esta forma en su discurso de apertura del debate sobre el estado de la región que se celebra este martes y miércoles en la asamblea de extremadura, y en el que ha reafirmado que la política energética es "uno de los capítulos que el nuevo gobierno pretende abordar cuanto antes, con especial incidencia en el futuro de las centrales nucleares". se trata de un proceso, ha señalado fernández vara, que "debe conjugar el suministro eléctrico mientras se incrementa la producción con energías renovables", tras lo que ha señalado que en la actualidad, extremadura "produce solo de fuentes renovables una cantidad de energía un 16 por ciento superior a su demanda", por lo que "puede abastecerse exclusivamente de fuentes renovables y aun así exportar energía limpia". así, ha avanzado que en el futuro más inminente, "los expertos aseguran que la sostenibilidad no es una opción, es una cuestión de supervivencia", ha reafirmado fernández vara, quien ha reivindicado la necesidad de "impulsar avances en la legislación nacional que beneficien a extremadura". hay margen de crecimiento y es que, según ha aseverado el presidente extremeño, "existe margen para el crecimiento", ya que las energías renovables suponen en extremadura sólo el 27,5 por ciento de la producción de energías, inferior a la media española". en este sentido, vara ha defendido que la producción de energías limpias es "un elemento más para la revalorización de la vida rural en nuestra región", ya que éste es un medio "esencial" para revitalizar los ecosistemas, los canales cortos de comercialización de alimentos, la práctica de una ganadería y agricultura de proximidad, las mejoras en bienestar social, salud o educación, o para que avancen también en nuevas relaciones con nuestras ciudades. a ello deben sumarse otras políticas externas como la política de cohesión, la pac y la reforma de la financiación autonómica que "deben incluir entre sus variables retos tan acuciantes y transversales como el aquí planteado, para no tratar injustamente a regiones con situaciones muy distintas". vara ha defendido la necesidad de que la relación entre lo urbano y lo rural sea "un elemento sobre el que hay que trabajar para hacer frente al reto demográfico", ya que según ha destacado, en europa gran parte de la población urbana vive en ciudades pequeñas y medianas, que "en muchos casos" tienen "un papel fundamental en las economías regionales como centros de servicios públicos y privados, de producción de conocimientos, innovación e infraestructuras local y regional". este modelo determinará, a juicio de fernández vara, "el futuro desarrollo económico, social y territorial de todas las regiones que conformamos la unión europea", ante lo que ha defendido que territorios como extremadura, que cuentan con una "reducida densidad de población y una elevada dispersión, son esenciales para el equilibrio medioambiental de europa". desafios globales durante su discurso, el presidente extremeño ha considerado que la comunidad autónoma "no puede ser ajena a las profundas transformaciones a realizar para responder a los desafíos globales y locales que exige nuestro planeta", tras lo que ha reivindicado la necesidad de buscar la "sostenibilidad" de la región, "pero también la del planeta y la ciudadanía global". para ello, ha considerado necesario "desarrollar políticas más allá de la idea cooperación", ya que sectores como la educación, la salud, el comercio, agricultura, migraciones, igualdad, cambio climático, energía, "forman parte de la nueva agenda 2030 en la que extremadura trabaja dentro de los objetivos de desarrollo sostenible". en este sentido, el presidente extremeño ha planteado la puesta en marcha de una secretaría de coherencia de políticas "para la aplicación de la agenda 2030 de los objetivos de desarrollo sostenible", y que permita "reforzar la ejecución coherente de dicha agenda". así, vara ha instado a "asumir que los conflictos no deberían relacionarse con el establecimiento de nuevas fronteras", sino que ha abogad por "avanzar hacia las conexiones y el equilibrio entre las necesidades locales y las conexiones globales", en el que los territorios "puedan aliarse en torno a recursos compartidos para progresar". & 756 & very low & Low & Socio-Economic & NA & NA & 2018-06-26 & 2018 & 3 & ECO
Frame & v.low & National & 500-1000 & 0.0141942 & -0.1980960 & -0.1536394 & -0.9382144 & 0.7927531 & 12.6 & 1.2981504 & 0.5794339 & Recipient & Domestic & Domestic & Domestic & Domestic|ECO & Positive\\
Spain & http://www.europapress.es/epsocial/igualdad/noticia-mas-200-jovenes-discapacidad-intelectual-pasaran-curso-universidad-reto-lograr-empleo-20171024180046.html & 619 & europa press & Private/Non-Public & Online only & National & very low = CP mentioned once & Social awareness/inclusion & Positive & National + Subnational & No myth & NA & NA & NA & NA & NA & NA & NA & NA & Spain & más de 200 jóvenes con discapacidad intelectual pasarán este curso... & 2017-10-24 & fondo social europeo & publicado 24/10/2017 18:00:46cet gracias a los programas de formación impulsados por fundación once, en los que participan 15 universidades madrid, 24 oct. (europa press) - más de 200 jóvenes con discapacidad intelectual pasarán este curso por una de las 15 universidades españolas que han sido seleccionadas para el desarrollo de programas universitarios de formación para el empleo de jóvenes con discapacidad intelectual, en el marco de la convocatoria de ayudas de fundación once. el objetivo de este programa es mejorar la formación y el empleo de las personas con discapacidad intelectual, un grupo especialmente complicado entre las personas con discapacidad a la hora de encontrar trabajo, según indican desde fundación once. la institución recuerda que las tasas de actividad de los jóvenes con discapacidad son 14 puntos inferiores a las del mismo grupo de edad sin discapacidad, algo que ocurre igualmente con las tasas de empleo y paro, que también se duplican. esta situación se agrava más todavía en el caso de los jóvenes con discapacidad intelectual, que representan el 40\% del total de la juventud con discapacidad, ya que en su mayoría no obtienen el título de educación secundaria y no acceden, por tanto, con facilidad al mercado laboral. en concreto, las universidades seleccionadas para desarrollar este programa son la universidad de a coruña, universidad de alicante, universidad de alcalá de henares, universidad pablo de olavide, universidad pública de navarra, universidad miguel hernández, universidad complutense de madrid, universidad de murcia, universidad de burgos, universidad de castilla-la mancha, universidad de jaén, universidad de málaga, universidad de almería, universidad de extremadura y universidad de granada. la puesta en marcha de estos programas cuenta con la cofinanciación del fondo social europeo y la iniciativa de empleo juvenil, y por ellos pasarán más de 200 jóvenes con discapacidad intelectual. dichos programas de formación se han presentado este martes en madrid, en una jornada celebrada en la sede de fundación once, que ha contado con la participación del secretario general de universidades, jorge sáinz; el vicepresidente ejecutivo de fundación once, alberto durán; el presidente de down españa, josé fabián cámara; el vicepresidente de plena inclusión, juan pérez, y el presidente de la cooperativa altavoz, óscar pueyo. en su intervención, sáinz ha afirmado que los programas universitarios de formación para el empleo de jóvenes con discapacidad intelectual que se presentan este martes se ajustan al modelo de universidad por el que apuesta el gobierno de españa, que no es otro que el de "una universidad en la que quepamos todos". a juicio del representante del ministerio de educación, "ser excelente significa ser mejor cada día" y la universidad ha de serlo en todos los ámbitos en los que trabaja, por lo tanto, también en el de la inclusión. a este respecto, durán ha destacado la importancia que tienen programas como los presentados en esta jornada para conseguir "una sociedad mejor, jóvenes más capacitados y empresas más comprometidas". por este motivo, el vicepresidente ejecutivo de fundación once ha animado a todas las universidades a implicarse en proyectos que incorporen a personas con discapacidad intelectual y a trabajar en colaboración con las empresas para que se involucren en la contratación de este colectivo. finalmente, ha pedido al gobierno que visibilice estos programas "vanguardistas" en el ámbito europeo porque, según argumentó, ayudarán a conseguir un cambio de mentalidad en todas las personas. durante la jornada, la directora de la cátedra uam-fundación prodis y del programa promentor de la universidad autónoma de madrid, lola izuzquiza, y la directora del programa demos de la universidad de comillas, ana berástegui, expusieron las distintas experiencias y el impacto que tienen los programas pioneros en la formación universitaria de jóvenes con discapacidad intelectual. asimismo, se presentó el programa eca-tic, una plataforma virtual de empleo con apoyo, y se debatió sobre las propuestas y retos de futuro de los nuevos proyectos de formación universitaria dirigidos a personas con discapacidad intelectual. el acto ha concluido con la propuesta de creación de una red de universidades inclusivas por la discapacidad intelectual. discapacidad universidades once & 674 & very low & Low & Socio-Economic & NA & NA & 2017-10-24 & 2017 & 2 & ECO
Frame & v.low & National & 500-1000 & 0.0141942 & -0.1980960 & -0.1536394 & -0.9382144 & 0.7927531 & 12.6 & 1.2981504 & 0.5794339 & Recipient & Domestic & Domestic & Domestic & Domestic|ECO & Positive\\
Spain & http://www.expansion.com/economia/2018/02/22/5a8e90e2e5fdeafa2c8b4670.html & 675 & Expansión & Private/Non-Public & Online and Offline & National & very low = CP mentioned once & Solidarity to poor countries/regions & Factual & EU + Other country & No myth & NA & NA & NA & NA & NA & NA & NA & NA & Spain & merkel: la estabilidad presupuestaria sigue siendo "brújula" de alemania para la unión europea & 2018-02-22 & política de cohesión & la canciller alemana, angela merkel, ha subrayado hoy que la estabilidad presupuestaria seguirá siendo la "brújula" de alemania para la política económica de la unión europea y la eurozona. merkel ha recalcado en una declaración de gobierno ante el bundestag (cámara baja) antes del consejo europeo informal de mañana en bruselas que "el pacto de estabilidad y crecimiento se mantiene en el futuro como la brújula de nuestras negociaciones" en el ámbito comunitario. la canciller aboga por que "todos los estados miembros" ejerzan su "responsabilidad" y apliquen "reformas ambiciosas" para "reforzar el crecimiento y la estabilidad", aprovechando el resurgir de la economía europea y las "perspectivas positivas" generales. en este contexto, el "mercado común interior" tiene un "significado decisivo". merkel insiste en que la "responsabilidad y controles" en la ue deben ir de la mano, rebatiendo implícitamente cualquier debate sobre la mancomunización de las deudas en el ámbito comunitario. considera que la eurozona debe seguir avanzando en la integración, un proceso "en absoluto" completado, y abogó por mejorar la competitividad de forma "más efectiva" y avanzar hacia la unión bancaria. con respecto al próximo presupuesto de la ue para el período 2021-2027, que se empezará a analizar mañana, merkel apuesta por reforzar con "medios europeos" la investigación y la innovación para afrontar la digitalización, así como las infraestructuras. las regiones menos avanzadas necesitan "apoyo" dentro de la política de cohesión para afrontar los retos futuros, una cuestión que denominó de "solidaridad", aunque, ha precisado, también ésta "no es una calle de sentido único. & 254 & very low & Low & Values & NA & NA & 2018-02-22 & 2018 & 3 & ECO
Frame & v.low & National & <500 & 0.0141942 & -0.1980960 & -0.1536394 & -0.9382144 & 0.7927531 & 12.6 & 1.2981504 & 0.5794339 & Recipient & European & European & European & European|ECO & Neutral\\
\addlinespace
Spain & http://www.expansion.com/agencia/europa\_press/2016/10/10/20161010075947.html & 669 & Expansión & Private/Non-Public & Online and Offline & National & medium = CP is important part of story & Political leverage & Negative & EU + National & No myth & NA & NA & NA & NA & NA & NA & NA & NA & Spain & bruselas informa este lunes al eurogrupo sobre la congelación de fondos a españa por el déficit & 2016-10-10 & fondos estructurales & bruselas, 10 (europa press) la comisión europea informará este lunes a los ministros de economía y finanzas de la eurozona (eurogrupo) sobre el "diálogo estructurado" con el parlamento europeo en relación a la posible congelación de fondos europeos por haber incumplido el objetivo de déficit el pasado año. el ejecutivo comunitario debe presentar una propuesta para suspender parte de los fondos estructurales y de inversión de ambos países para 2017 después de que incumplieran sus objetivos en 2015. no obstante, debe consultar antes con el parlamento europeo, aunque su opinión no sea vinculante. así, un alto funcionario del eurogrupo ha asegurado que espera que bruselas exponga brevemente sus conclusiones tras esta primera sesión en el parlamento europeo "para que los ministros tengan una mejor apreciación" sobre la cuestión. en cualquier caso, las mismas fuentes han recordado que el diálogo con la eurocámara es únicamente un "requisito legal" antes de que la comisión pueda dar los siguientes pasos. el vicepresidente del ejecutivo comunitario responsable de empleo y crecimiento, jyrki katainen, y la comisaria de política regional, corina cretu, defendieron el pasado 3 de octubre ante las comisiones parlamentarias de asuntos monetarios y de desarrollo regional que el ejecutivo comunitario está "legalmente obligado" a congelar parte de estos fondos. no obstante, la mayoría de los grupos políticos mostraron su rechazo a esta posibilidad, que algunos eurodiputados tildaron como "absurda", y pidieron a bruselas que reconsiderara su posición. tras este primer encuentro, la eurocámara decidió continuar con el diálogo estructurado al estimar que es necesario recabar más información sobre la suspensión de los fondos. por este motivo, los líderes de los grupos parlamentarios y el presidente del parlamento europeo, martin schulz, han afirmado que invitarán a una nueva audiencia a los ministros de economía de españa y portugal, y que posteriormente decidirán sobre los siguientes pasos. la comisión europea, por su parte, no se plantea presentar su propuesta sobre la congelación de fondos hasta que el parlamento europeo de por cerrado el "diálogo estructurado", tal y como han informado fuentes comunitarias. en cualquier caso, el ejecutivo comunitario ha dejado claro que levantará "inmediatamente" la suspensión de los fondos una vez que certifique que españa ha tomado medidas efectivas para corregir el desvío de sus cuentas públicas. a este efecto, el gobierno en funciones debe presentar, antes del 15 de octubre, una batería de medidas y el borrador presupuestario para el próximo año. proximo desembolso de 2.800 millones a grecia por otro lado, los ministros de economía y finanzas de la zona euro abordarán los avances logrados en el marco del tercer rescate de grecia y evaluarán si las autoridades helenas han cumplido con las condiciones necesarias para dar la aprobación política al desembolso del siguiente tramo, de 2.800 millones de euros. esta cantidad forma parte de los 10.300 millones de euros que la troika desbloqueó en mayo, de los cuales ya se han desembolsado 7.500 millones. las acciones pendientes para el montante pendiente están relacionadas con privatizaciones, reformas del sector energético, la puesta en marcha de una agencia tributaria independiente y medidas relacionadas con la gobernanza bancaria. precisamente, la comisión europea explicó el pasado viernes que las instituciones comunitarias están evaluando el cumplimiento de las condiciones necesarias tras el "considerable progreso" de los últimos días, pero ha recordado que corresponde al eurogrupo y al mecanismo europeo de estabilidad (mede) decidir sobre los próximos desembolsos. & 565 & medium & Medium & Power & NA & NA & 2016-10-10 & 2016 & 2 & POL
Frame & low-medium & National & 500-1000 & 0.0141942 & -0.1980960 & -0.1536394 & -0.9382144 & 0.7927531 & 12.6 & 1.2981504 & 0.5794339 & Recipient & Domestic & European & Mixed & Domestic|POL & Negative\\
Spain & http://www.abc.es/local-canarias/20141118/abci-ausserd-inversiones-201411182149.html?utm\_source=abc\&utm\_medium=rss\&utm\_content=uh-rss\&utm\_campaign=traffic-rss & 596 & ABC TU DIARIO EN ESPAÑOL & Private/Non-Public & Online and Offline & National & low = CP mentioned more times but NOT important part of story (mainly about others issues) & Economic development & Positive & EU + National + Subnational & No myth & Territorial cooperation & Positive & EU + National + Subnational & No myth & NA & NA & NA & NA & Spain & ausserd competes to attract investments in the buchmesse salt of fuerteventura & 2014-11-19 & fondo europeo de desarrollo regional & la cita de logística y transporte se celebra por primera vez en una isla no capitalina ausserd regresa al salón atlántico de logística y transporte (salt) con la finalidad de "fomentar el conocimiento de esta zona y potenciar los factores geográficos y el nivel de desarrollo de las infraestructuras a fin de captar inversión y localización de empresas en esta zona de dakhla". ausserd, que preside ahmed abdellaoui, formado en la universidad de la laguna en económicas y empresariales, se encuentra cerca de mauritania, un área grande y con acceso directo al mar "que no provoca altos costes de implantación", informaron en un comunicado. esta zona se beneficia de dos crecimientos económicos: el propio de marruecos y el de mauritania con la actividad minera, pesca y construcción de infraestructuras. las autoridades de ausserd, a través de la asociación nass, regresan al salt porque están "trabajando para mejorar el entorno empresarial: nuevos incentivos para que los inversores y procedimientos simplificados. todo esto se hace sólo para mejorar el entorno empresarial y aumentar la capacidad de atracción de inversiones de ausserd". uno de los factores con los que ausserd se presenta en el salt es su infraestructura tecnológica para acoger a empresas que quieran desarrollar programas de atención a clientes o seguimientos de flotas. asimismo, ausserd lanza en el salt indicadores como "la eficacia de mercado de trabajo", que incluye la relación del nivel de productividad y los salarios, para captar inversión en esta zona. ausserd viene a una nueva edición del salt junto al centro regional de inversión (cri) de la región de dakhla, la agencia de promoción de inversiones del sur, el operador aéreo canary fly, el operador de viajes de negocios y turismo río de oro, entre otros. uno de los aspectos que ausserd promociona en salt es la invariabilidad de los tipos impositivos, incluyendo iva e impuestos especiales. además, la estabilidad está asegurada desde el cambio de las regulaciones con respecto a la atracción de mano de obra extranjera durante el tiempo del contrato de inversión. el encuentro, organizado por el clúster canario del transporte y la logística con la colaboración del gobierno de canarias y el cabildo de fuerteventura, se enmarca dentro de la segunda fase del programa de actuaciones para potenciar la conectividad entre canarias y el sur de marruecos (transmaca ii) y, en concreto, dentro de la actividad dirigida a la consolidación del espacio de intercambio institucional y empresarial de la logística y el transporte. salt forma parte del programa operativo de cooperación transfronteriza españa-fronteras exteriores (poctefex), que tiene como objetivo global mejorar la conectividad en el ámbito del transporte marítimo y aéreo entre canarias y marruecos para fortalecer y fomentar el intercambio comercial y de personas entre ambas regiones, y está cofinanciado al 75\% por la unión europea a través del fondo europeo de desarrollo regional (feder). & 474 & low & Low & Socio-Economic & Socio-Economic & NA & 2014-11-19 & 2014 & 1 & ECO
Frame & low-medium & National & <500 & 0.0141942 & -0.1980960 & -0.1536394 & -0.9382144 & 0.7927531 & 12.6 & 1.2981504 & 0.5794339 & Recipient & Domestic & European & Mixed & Domestic|ECO & Positive\\
Spain & https://www.diariosur.es/axarquia/nerja-incorpora-decena-20181028003848-ntvo.html & 657 & Sur & Private/Non-Public & Online and Offline & Regional/Local & very low = CP mentioned once & Infrastructure & Positive & Subnational & No myth & NA & NA & NA & NA & NA & NA & NA & NA & Spain & nerja incorpora una decena de trámites a la administración electrónica & 2018-10-27 & fondo europeo de desarrollo regional & el ayuntamiento adjudica por 148.760 euros el contrato para implantar la digitalización en los procesos de padrón, registro, gestión y archivo cabezas. con varios años de retraso con respecto a otros muchos municipios de un tamaño similar e incluso mucho menores, el ayuntamiento de nerja ha terminado de sumarse al carro de las nuevas tecnologías, al completar la implementación digital de una decena de trámites con los sistemas de la llamada administración electrónica. en concreto, a partir de ahora, los empleados municipales, vecinos y cargos públicos, podrán realizar a través de internet tareas tan habituales como el registro de entrada y salida de documentos, la gestión de expedientes electrónicos, de órganos colegiados, el portafirmas electrónico (firma electrónica), la sede electrónica, el padrón de habitantes y el archivo municipal. el contrato, con un presupuesto base de 148.760,33 euros y una duración de cuatro años, es cofinanciado por el fondo europeo de desarrollo regional (feder) en un 80\%, en el marco de las ayudas para la realización de estrategias de desarrollo urbano sostenible integrado, dado que la operación es llevada a cabo conforme al plan de implementación y selección de operaciones de la edusi 'nerja adelante', otorgado en 2016, con un montante de cinco millones. la alcaldesa, rosa arrabal (psoe), y el concejal de nuevas tecnologías, jorge bravo (iu), se reunieron esta pasada semana con representantes de la empresa adjudicataria para la firma del contrato para ofrecer estos servicios de software de administración electrónica, padrón de habitantes y otros servicios complementarios en el ayuntamiento de nerja. "desde hace más de diez años, era una herramienta necesaria para facilitar la gestión administrativa a los usuarios y funcionarios", afirmó la alcaldesa. "cualquier documento se sabrá en qué trámite está y quién falta por firmar, se escanearán los escritos con sus documentos y siempre habrá rastro de ellos. ahora entiendo porqué el pp no lo instaló en sus 20 años al frente del ayuntamiento", añadió. por su parte, el concejal nerjeño consideró que estos cambios van a suponer "un giro radical de 180 grados" en el funcionamiento de la administración, posibilitando la realización de numerosos trámites desde casa, el móvil, el ordenador o la 'tablet'. & 363 & very low & Low & Socio-Economic & NA & NA & 2018-10-27 & 2018 & 3 & ECO
Frame & v.low & Regional & <500 & 0.0141942 & -0.1980960 & -0.1536394 & -0.9382144 & 0.7927531 & 12.6 & 1.2981504 & 0.5794339 & Recipient & Domestic & Domestic & Domestic & Domestic|ECO & Positive\\
Spain & http://ecodiario.eleconomista.es/politica/noticias/8781812/11/17/Madrid-firma-la-Alianza-por-la-Cohesion-de-la-UE-en-defensa-de-esa-partida-presupuestaria.html & 679 & El Economista (EcoDiario) & Private/Non-Public & Online and Offline & National & medium = CP is important part of story & Institutional bargaining over funding & Negative & EU + National + Subnational & No myth & NA & NA & NA & NA & NA & NA & NA & NA & Spain & madrid firma la alianza por la cohesión de la ue en defensa de esa partida presupuestaria & 2017-11-30 & política de cohesión & bruselas, 30 nov (efe).- la presidenta de la comunidad de madrid, cristina cifuentes, ha firmado hoy la adhesión de madrid a la alianza por la cohesión, una declaración en defensa de esta política del presupuesto de la ue a la que ya se han unido más de 750 alcaldes, presidentes regionales y representantes electos de toda europa. en un acto celebrado en bruselas previo a la sesión plenaria del comité europeo de las regiones (cdr), cifuentes confirmó junto al presidente de esta institución, karl-heinz lambertz, la adhesión de la comunidad a esta declaración, en marcha desde el pasado 9 de octubre. cifuentes transmitió a lambertz su agradecimiento por la invitación a intervenir en la sesión plenaria de hoy, donde participará en un debate sobre el 'brexit', y por facilitar la posibilidad de que madrid se una a la alianza para la cohesión. "la comparto absolutamente, creo que es una gran iniciativa. es recuperar desde la ue, desde la europa de las regiones, esa idea inicial (...) de cohesión y de unidad entre todas las regiones", afirmó la presidenta de la región autónoma madrileña. "esperamos que sean muchas más las regiones que participen de esta alianza, que viene a recuperar la esencia verdadera de lo que tiene que significar la unión europea", agregó. por su parte, lambertz calificó la adhesión de madrid a la alianza de "particularmente simbólica y fuerte", ya que una política de cohesión potente requiere la participación de todas las regiones, no solo aquellas "en dificultades", opinó. "madrid es, afortunadamente, una región fuerte en españa y en europa. pero también necesitamos a esas regiones si queremos una cohesión completa en una dimensión económica, social y territorial", afirmó. el presidente de las regiones europeas señaló que el desarrollo de políticas para "compartir entre estados, regiones y europa" está "en el adn de la ue", con vistas a cumplir juntos "los objetivos y proyectos que permiten la innovación, el desarrollo y la cohesión social". la política de cohesión, que cuenta en el presupuesto europeo con más de 630.000 millones de euros para el periodo entre 2014 y 2020, es un importante instrumento de inversión de la ue destinado a paliar los desequilibrios entre las regiones comunitarias, y la alianza que la respalda defiende su continuidad en el presupuesto europeo. esta política podría ser una de las partidas del presupuesto que sufran recortes por el agujero que dejará el 'brexit' (la salida del reino unido de la unión para marzo de 2019) en las cuentas comunitarias y por la llegada de nuevas prioridades de financiación, como una estrategia de defensa europea o iniciativas para gestionar los flujos migratorios. a día de hoy, más de 750 representantes locales de toda la ue y 80 asociaciones de diversos campos han firmado la alianza por la cohesión. & 464 & medium & Medium & Power & NA & NA & 2017-11-30 & 2017 & 2 & POL
Frame & low-medium & National & <500 & 0.0141942 & -0.1980960 & -0.1536394 & -0.9382144 & 0.7927531 & 12.6 & 1.2981504 & 0.5794339 & Recipient & Domestic & European & Mixed & Domestic|POL & Negative\\
Spain & http://www.20minutos.es/noticia/3032670/0/diputacion-promueve-dia-europa-respaldo-instituciones-onubenses-union-europea/ & 658 & 20 minutos & Private/Non-Public & Online and Offline & National & very low = CP mentioned once & Jobs & Positive & Subnational & No myth & NA & NA & NA & NA & NA & NA & NA & NA & Spain & la diputación promueve en el día de europa el respaldo de las instituciones onubenses a la unión europea & 2017-05-09 & fondos estructurales & el centro de información europea 'europe direct huelva' de la diputación de huelva ha celebrado este martes la conmemoración del día de europa, enmarcado en el 60 aniversario de los tratados de roma, con variadas actividades en la plaza de las monjas de la capital, que han dado comienzo con un acto institucional seguido del izado de la bandera de la ue en el que han participado autoridades y representantes de las principales instituciones en la provincia. la diputada de desarrollo local, lourdes martín, ha abierto el acto apelando a las instituciones a brindar su apoyo para seguir trabajando por la construcción de una europa sin fronteras, "en unos momentos especialmente críticos, en los que asuntos de calado como el brexit o el auge de los populismos que proclaman la salida de europa como solución amenazan la estabilidad y el bienestar de los estados miembros". por su parte, el delegado territorial de economía, innovación, ciencia y empleo en huelva, manuel josé ceada, ha querido poner en valor lo que ha significado la unión europea (ue) para andalucía, "gracias a la llegada de fondos estructurales que han contribuido al avance de la comunidad y la lucha contra el desempleo". el delegado ha insistido en que "es el momento de cambiar el modelo social de la austeridad y luchar por una europa donde prime la igualdad, el derecho social y el bienestar de las personas por encima de otros intereses". en representación del ayuntamiento de huelva, el concejal de urbanismo y patrimonio municipal, manuel francisco gómez, ha abogado por trabajar en la construcción de "más y mejor europa" como solución a los problemas que en estos momentos atentan contra la unión europea "tanto por las amenazas internas, como la salida de estados miembros, como por los problemas externos que representan el terrorismo o la crisis de los refugiados". por último, desde erasmus students network, (esn), javier rodríguez, como representante de esta asociación que acoge a los universitarios erasmus que llegan a huelva desde otros países, ha hecho un llamamiento por una europa sin fronteras y ha instado a las instituciones a seguir trabajando en la construcción de "este gran país llamado europa". durante todo el día estará abierto un stand informativo atendido por voluntarios de la universidad de huelva y estudiantes erasmus de otros países que estudian en la ciudad. además, a las 18,00 horas se realizarán actividades de animación y talleres informativos. paralelamente, habrá un flashmob con temática europea llevado a cabo por la asociación juvenil de huelva, k2dance. también se ha organizado una nueva edición de la actividad 'cuentacuentos sobre la unión europea', con la que se quiere acercar a los más jóvenes de una forma divertida y dinámica, el funcionamiento de las instituciones de la unión europea, así como de sus países miembros. el objetivo fundamental del centro de información europea de la diputación de huelva es acercar a la población onubense el trabajo que se lleva a cabo desde las instituciones de la unión, a través de actividades de formación, divulgación y comunicación, tanto de forma directa a todas aquellas personas que lo soliciten como con la realización de otras acciones que favorezcan su conocimiento entre la ciudadanía de huelva. así, junto a proyectos como el 'cuentacuentos sobre la unión europea', 'europa direct huelva' realiza actuaciones de divulgación, encuentros, concursos, acciones formativas, así como talleres de traducción de currículos al inglés. en esta nueva edición del 'cuentacuentos sobre la unión europea', está previsto celebrar 15 sesiones durante los meses de mayo y junio, que recorrerán los municipios de chucena, paterna, villarrasa, lucena del puerto, santa bárbara de casas, la redondela, puebla de guzmán, el campillo, nerva, puerto moral, los marines, cortelazor, punta umbría, cartaya y palos de la frontera. la actividad consiste en la realización de una sesión informativa para escolares con edades comprendidas entre los seis y los nueve años. cada sesión tiene una duración de aproximadamente 45 minutos y se realiza en una jornada por cada municipio. principalmente, en estos cuentacuentos se destacan los símbolos de la unión europea; moneda, bandera e himno y se dan a conocer diferentes aspectos de los países que la componen, así como los valores que la caracterizan como son la interculturalidad y la solidaridad. con este proyecto, que desde hace varios años viene desarrollando la diputación provincial de huelva, se fomenta la difusión y promoción de la unión europea entre los alumnos, desde su historia hasta sus capitales, desde sus primeros inicios hasta el día de hoy. la dinámica pretende ser eminentemente participativa, donde desde los más pequeños hasta los mayores tomen parte y sean ellos los verdaderos protagonistas del aprendizaje que, además tiene una parte bilingüe en inglés. desde 2006 el centro 'europe direct huelva' es miembro de la red de información europea de andalucía, que aglutina a los centros europeos de nuestra comunidad autónoma. anualmente se reciben en la oficina del centro unas mil consultas y a través de su web y perfil en redes sociales, planifica y desarrolla anualmente un plan de actividades encaminado a dar a conocer y difundir en la provincia, toda la actividad que se desarrolle en el entorno de la unión europea. & 860 & very low & Low & Socio-Economic & NA & NA & 2017-05-09 & 2017 & 2 & ECO
Frame & v.low & National & 500-1000 & 0.0141942 & -0.1980960 & -0.1536394 & -0.9382144 & 0.7927531 & 12.6 & 1.2981504 & 0.5794339 & Recipient & Domestic & Domestic & Domestic & Domestic|ECO & Positive\\
\addlinespace
Spain & http://www.eldiario.es/desalambre/Alemania-Austria-extraordinaria-crisis-refugiados\_0\_431157445.html & 636 & eldiario.es & Private/Non-Public & Online and Offline & National & medium = CP is important part of story & Political leverage & Negative & EU + Other country & No myth & NA & NA & NA & NA & NA & NA & NA & NA & Spain & alemania y austria piden una cumbre extraordinaria ante la crisis de los refugiados & 2015-09-15 & fondos estructurales & europa fracasa: los gobiernos no llegan a un acuerdo para el reparto de refugiados la canciller alemana, angela merkel, anunció hoy que berlín y viena han solicitado al presidente del consejo europeo, donald tusk, la celebración de una cumbre extraordinaria de la ue para buscar soluciones a la actual crisis de refugiados. en una rueda de prensa en berlín tras mantener un encuentro con el canciller austriaco, werner faymann, merkel subrayó la necesidad de que los líderes europeos, al margen de las cuotas que debaten los ministros de interior, aborden cuestiones como la ayuda a los países de origen de los refugiados o la creación de centros provisionales de acogida para el registro en las fronteras exteriores de la ue. este lunes, los gobiernos de la eu fracasaron en el reparto de los 120.000 refugiados sirios, eritreos e iraquíes que propuso distribuir la comisión europea en los próximos dos años. la canciller defendió la decisión de alemania de restablecer los controles fronterizos para "mejorar" el registro de los solicitantes de asil o que llegan al país y por razones de seguridad, una semana después de haber abierto sus fronteras de forma "excepcional" por motivos humanitarios. el titular alemán del interior, thomas de maizière, consideró este martes necesario comenzar a hablar de "medidas de presión" contra aquellos países que se niegan a un reparto equitativo de refugiados en la unión europea (ue). "el hecho es que estamos lejos de una cuota de reparto permanente", lamentó en unas declaraciones al programa matinal de la televisión pública zdf el ministro, quien criticó que a aquellos países que se niegan, "no les pasa nada", sino que los refugiados "sencillamente les pasan de largo". en este sentido se sumó a la propuesta del presidente de la comisión europea (ce), jean-claude juncker, quien se refirió a la posibilidad de responder a estos países con un recorte de los medios que reciben de los fondos estructurales comunitarios. de maizière recordó que los estados que se niegan a un reparto equitativo de refugiados son precisamente países que reciben grandes cantidades de los fondos estructurales. el ministro habló de un "comportamiento insolidario de una minoría" y lamentó que los ministros del interior de los 28 no pudieran alcanzar anoche "un resultado por unanimidad", sino sólo "por mayoría" en lo que se refiere al reparto de 160.000 refugiados, de los cuales sólo la cifra de 40.000 ha sido decidida formalmente. & 404 & medium & Medium & Power & NA & NA & 2015-09-15 & 2015 & 1 & POL
Frame & low-medium & National & <500 & 0.0141942 & -0.1980960 & -0.1536394 & -0.9382144 & 0.7927531 & 12.6 & 1.2981504 & 0.5794339 & Recipient & European & European & European & European|POL & Negative\\
Spain & http://www.elmundo.es/andalucia/2016/05/30/574c28e6e5fdeae03a8b4637.html & 617 & EL MUNDO & Private/Non-Public & Online and Offline & National & very low = CP mentioned once & Fraud/Corruption & Negative & Subnational & No myth & NA & NA & NA & NA & NA & NA & NA & NA & Spain & el ex número 2 del psoe-a cargó la luz de su casa a un curso de la junta & 2016-05-30 & fondo social europeo & antonia montilla, esposa de rafael velasco, llegó a cobrar 6.849,74 euros al mes y su hermana, auxiliar administrativa, 4.227 euros el escándalo que le costó la carrera política a rafael velasco antonio salvadorsevilla@ajsalvador70alfonso albacórdoba@alfonsoalba\_ 30/05/2016 14:01 la esposa del ex número dos del psoe andaluz rafael velasco cargó la luz del domicilio familiar a un curso de formación que la junta le concedió a su academia. la administración autonómica otorgó ayudas por 730.000 euros a aulacen cinco sl en cuatro años. el 3 de diciembre de 2007, el entonces director general de formación para el empleo, juan manuel fuentes doblado, dio su conformidad a la propuesta del servicio de gestión y programación de la fpo para que se le concediera a aulacen cinco una ayuda de 41.296,50 euros -cofinanciada al 80 por ciento por el fondo social europeo- para impartir un curso de formador ocupacional durante 2008. esta empresa se fundó en córdoba en diciembre de 2005 y tiene como administradora única a antonia montilla, esposa de rafael velasco. según se detalla en el expediente, codificado como 98/2007/j/373 y al que ha tenido acceso el mundo, la acción se impartió a 14 alumnos y se prolongó durante 414 horas entre el 10 de abril y el 15 de julio de 2008. en la liquidación final del curso presentada ante la junta, aulacen cinco imputó a gastos de 'energía y mantenimiento' una partida de 888,51 euros, de los que 595,47 euros corresponden a dos facturas de luz emitidas por sevillana endesa. en concreto, una importa 187,24 euros y detalla como titular a aulacen cinco, ubicada en un local de la cordobesa avenida arroyo del moro. el periodo de facturación abarca desde el 23 de abril al 24 de junio de 2008, meses en los que se impartió el curso subvencionado al 100 por cien por la junta. en la segunda factura, sin embargo, aparece como titular antonia montilla y el punto de sumistro coincide con el domicilio familiar que el matrimonio velasco montilla tiene en quintana, una aldea perteneciente a la localidad cordobesa de la carlota. se trata de un gasto de 408,23 euros fechado el 2 de julio de 2008 sin que se detalle ni el periodo de facturación ni la potencia contratada. a diferencia de la factura anterior, tampoco aparece ninguna referencia a la actividad económica (cnae), lo que confirma que es un cliente doméstico y no empresarial. aulacen cinco terminó cargando el 57,1 por ciento (233,11 euros) de ese importe a la ayuda de la junta, según se especifica en la relación de gastos que antonia montilla certificó ante la consejería de empleo el 1 de agosto de 2008 como representante legal de la empresa. cuando su esposa imputó el gasto de la luz de la casa familiar a la subvención de la junta, rafael velasco era diputado en el parlamento andaluz tras resultar elegido en los comicios celebrados el 9 de marzo de 2008. a nivel orgánico, desempeñaba el cargo de secretario de política institucional del psoe-a. documentos que implican a velasco en el expediente de este curso constan dos documentos que implican a rafael velasco, que dimitió el 26 de octubre de 2010 como parlamentario y como vicesecretario general del psoe-a después de que este periódico informara de que la empresa de su mujer recibió ayudas de la administración autonómica para impartir 24 cursos por un montante de 730.000 euros. uno es un documento firmado por antonia montilla en el que autoriza a su marido "a recoger y aceptar las propuestas de programas de los cursos de fpo 2007-2008". el otro es un escrito, con el nombre manuscrito de rafael velasco sierra y su condición de "persona autorizada", en la que éste "acepta la propuesta de programación relativa al expediente administrativo 98/2007/j/373 en base a la solicitud de participación en los programas de formación profesional ocupacional de 2007". el gasto de la luz no es el único que llama la atención de la ayuda para el curso de formador ocupacional que la junta otorgó a aulacen cinco. sorprende que la casi la mitad de los costes del curso lo representen la partida 'sueldos u honorarios de los profesores', concretamente 19.126,80 de los 41.296,50 euros concedidos. en el expediente constan diversas nóminas, con el sello de la consejería de empleo, que acreditan los sueldos astronómicos que percibían los profesores del curso. los cuatro pertenecían a las familias velasco y montilla. así, antonia montilla cobró 5.215,71 euros en abril de 2008 y 8.849,74 euros en junio de 2008 (netos). su hermana rafaela percibió 4.227, 92 euros en abril de 2008, a pesar de que en la nómina se detalla su categoría de "auxiliar administrativa". en concreto, se detalla un salario base de 647,79 euros y un complemento de 3.801 euros. los gastos de sueldos se completan con los 8.129,09 euros netos que el 30 de mayo de 2008 facturó pedro velasco sierra -hermano del ex dirigente del psoe-a- por los servicios prestados de "estrategias de orientación, seguimiento formativo, diseño de pruebas de evaluación de las acciones formativas, sensibilización ambiental, innovación y actualización docente e inserción laboral y técnicas de búsqueda de empleo". & 895 & very low & Low & Governance & NA & NA & 2016-05-30 & 2016 & 2 & POL
Frame & v.low & National & 500-1000 & 0.0141942 & -0.1980960 & -0.1536394 & -0.9382144 & 0.7927531 & 12.6 & 1.2981504 & 0.5794339 & Recipient & Domestic & Domestic & Domestic & Domestic|POL & Negative\\
Spain & http://www.abc.es/espana/abci-espana-saldo-netamente-favorable-201606232036\_noticia.html & 666 & ABC TU DIARIO EN ESPAÑOL & Private/Non-Public & Online and Offline & National & medium = CP is important part of story & Institutional bargaining over funding & Positive & EU + National & No myth & NA & NA & NA & NA & NA & NA & NA & NA & Spain & españa en la ue: un saldo netamente favorable & 2016-06-23 & fondo de cohesión & en plena campaña previa al referéndum en el que los británicos decidirán si desean permanecer en la unión europea o salir de ella, se daba a conocer hace diez días un estudio realizado por el pew research center, con sede en washington, sobre los niveles de euroescepticimismo en la población del viejo continente. entre los datos más llamativos se encontraba que españa es uno de los cinco países de la ue donde más han crecido, desde 2010, la visión negativa sobre europa. un 47\% de los ciudadanos españoles duda hoy de las ventajas de formarte parte de la ue que un 47 por ciento de ciudadanos españoles manifieste hoy sus dudas sobre las ventajas de pertenecer al club comunitario es algo que sorprende, si se tiene en cuenta el entusiasmo con que hace 31 años se acogió en nuestro país la firma del tratado de adhesión a la comunidad económica europea (cee), como se la denominaba entonces. e incluso, posteriormente, con el apoyo del 76,73 por ciento de los españoles a la constitución europea que se sometió a referéndum en 2005. aquella ceremonia del 12 de junio de 1985, vivida en el salón de columnas del palacio real simbolizaba el regreso de españa a una europa de la que había sido excluida durante el franquismo. la unanimidad con que los partidos políticos salidos de las primeras elecciones democráticas habían apoyado la petición formal de ingreso en la cee reflejaban bien cómo los españoles eran muy conscientes de que su futuro pasaba por europa, si querían que ese futuro estuviera marcado por la paz, la estabilidad, la democracia y el progreso económico. españa, el socio leal de la unión europea la entrada de españa en la cee fue, junto con la pertenencia a la alianza atlántica, un momento clave en el anclaje a un mundo de valores que hoy quizás no pase por sus mejores momentos, pero que sigue teniendo una gran capacidad de atracción. si españa no se hubiera incorporado a la cee, se ha dicho, estaría hoy -no sólo fíasicamente- mucho más cerca de áfrica que de europa. los beneficios que se han derivado para españa de la pertenencia a la unión europea han sido innumerables, aunque la entrada no fue un camino de rosas. hubo que superar difíciles obstáculos, desde el veto que durante tiempo mantuvo el presidente francés valery giscard estaing hasta las duras reconversiones en los sectores siderúrgico o naval, además de someternos a la normativa comunitaria en materia de pesca o aceptar limitaciones para algunos productos agrícolas. "si españa no se hubiera incorporado a la cee estaría hoy mucho más cerca de áfrica que de europa" desde entonces, españa se ha mostrado siempre en europa como un socio leal, comprometido con todos los proyectos de integración puestos en marcha -incluido el gran avance de la unión económica y monetaria-, frente a las actitudes más reticentes de países como el reino unido o dinamarca, o a la desgana mostrada por otros socios como grecia. la llegada de españa, junto con portugal, en 1985, llevó, además, a lograr un mayor equilibrio norte-sur, e hizo volver la mirada de europa hacia los vecinos de la otra orilla del mediterráneo. fue precisamente nuestro país el que, en 1995, impulso el proceso euromediterráneo con la conferencia de barcelona. igualmente, el reforzamiento de la dimensión eurolatinoamericana. los países de américa latina fueron pronto muy conscientes de que tenían en españa a su aliado más firme para mejorar su relación política y económica con europa, y el vínculo se concretó en las cumbre ue -- américa latina que se celebran cada dos años. un cambio de rumbo en política exterior españa se insertó en el mundo a través de la unión europea y desarrolló una política exterior que trataba de moverse en las coordenadas europeas, hasta el punto de que un español, javier solana, llegó a ser el "mister pesc", encargado de poner en marcha una política europea y de seguridad común. en la vida diaria, los españoles hemos asumido el concepto de ciudadanía europea y hemos disfrutado, por ejemplo, de las ventajas de moverse libremente y con una única moneda, por turismo o por motivos de trabajo, por un espacio mucho más amplio que el que acotan nuestras fronteras. desde el punto de vista económico, es obvio que el desarrollo de nuestro país no habría sido el mismo fuera de la ue. sirva como ejemplo, que de los 483 kilómetros de autovía con que contaba españa en 1986 se ha pasado a 14.000 kilómetros, al tiempo que se ha establecido una envidiable red ferroviaria de alta velocidad. españa ha sido uno de los principales beneficiarios de la política agrícola común (pac) y los fondos de cohesión que ha recibido de europa superan ampliamente los 150.000 millones de euros, a través de distintos instrumentos financieros, como los fondos estructurales y el llamado fondo de cohesión. tres décadas de continuo desarrollo a lo lo largo de más de tres décadas, nuestro país ha estado recibiendo fondos comunitarios, que han permitido un desarrollo espectacular hasta el punto de que, en ese tiempo, el producto interior bruto se ha duplicado. y si cuando ando españa ingresó en la ue, la renta per cápita era del 72 por ciento de la media europea de los 12 miembros que entonces componían la cee, hoy la situación ha cambiado radicalmente hoy. con una unión de con 28 miembros, españa está en el 94 por ciento de la media, una media que llegó a superar en los años anteriores a la crisis económica. hasta 2020, españa aún recibirá fondos europeos, pero hoy es ya un contribuyente neto en la unión europea, como consecuencia de la llegada de nuevos socios más pobres y de su progreso económico. a nadie se le escapa cuál es la situación real de españa, las exigencias que recibe de bruselas y el elevado índice de paro que padece, pero la perspectiva sería mucho peor si estuviera fuera de la ue. en conjunto, en estos 31 años, españa no sólo ha recorrido un sendero de éxito, sino que también se ha anclado de manera definitiva en europa, una realidad nada desdeñable, especialmente hoy cuando aumenta la intensidad de los cantos de sirenas que acompañan a las turbias corrientes populistas. & 1046 & medium & Medium & Power & NA & NA & 2016-06-23 & 2016 & 2 & POL
Frame & low-medium & National & +1000 & 0.0141942 & -0.1980960 & -0.1536394 & -0.9382144 & 0.7927531 & 12.6 & 1.2981504 & 0.5794339 & Recipient & Domestic & European & Mixed & Domestic|POL & Positive\\
Spain & http://ecodiario.eleconomista.es/sociedad/noticias/8173047/02/17/Discapacidad-inserta-imparte-en-malaga-un-curso-de-ofimatica-para-personas-con-discapacidad.html & 639 & El Economista (EcoDiario) & Private/Non-Public & Online and Offline & National & very low = CP mentioned once & Social awareness/inclusion & Positive & National + Subnational & No myth & NA & NA & NA & NA & NA & NA & NA & NA & Spain & discapacidad. inserta imparte en málaga un curso de ofimática para personas con discapacidad & 2017-02-22 & fondo social europeo & inserta empleo, la entidad de fundación once para la formación y el empleo, comenzó este miércoles en málaga un curso de ofimática para personas con discapacidad. un total de 15 personas con discapacidad asisten a esta acción formativa de 330 horas lectivas, que se impartirá hasta el próximo 2 de junio. el curso está dividido en los siguientes módulos: 'sistema operativo windows: técnicas de comunicación y archivo'; 'procesador de textos word 2000'; 'hoja de cálculo excel 2000'; 'base de datos access 2000'; 'programa de presentaciones powerpoint' e 'introducción a internet'. inserta proporciona a los alumnos las destrezas profesionales que les permitan un rendimiento competitivo en el mercado de trabajo y el desarrollo de aptitudes y habilidades personales para conseguir la plena participación en su entorno laboral y social. este curso se enmarca en los programas operativos de inclusión social y de la economía social (poises) y de empleo juvenil (poej), que está desarrollando fundación once a través de inserta, con la cofinanciación del fondo social europeo y la iniciativa de empleo juvenil, para incrementar la formación y el empleo de las personas con discapacidad. & 184 & very low & Low & Socio-Economic & NA & NA & 2017-02-22 & 2017 & 2 & ECO
Frame & v.low & National & <500 & 0.0141942 & -0.1980960 & -0.1536394 & -0.9382144 & 0.7927531 & 12.6 & 1.2981504 & 0.5794339 & Recipient & Domestic & Domestic & Domestic & Domestic|ECO & Positive\\
Spain & https://www.abc.es/espana/castilla-la-mancha/abci-junta-destina-41-millones-para-lograr-insercion-laboral-jovenes-desempleados-201904041825\_noticia.html & 689 & ABC TU DIARIO EN ESPAÑOL & Private/Non-Public & Online and Offline & National & very low = CP mentioned once & Social awareness/inclusion & Positive & Subnational & No myth & NA & NA & NA & NA & NA & NA & NA & NA & Spain & la junta destina 4,1 millones para lograr la inserción laboral de jóvenes desempleados & 2019-04-04 & fondo social europeo & estas ayudas están dirigidas a proyectos para jóvenes entre 16 y 30 años empadronados en castilla-la mancha el consejo de gobierno de castilla-la mancha ha aprobado esta semana una convocatoria de ayudas a entidades que desarrollen actuaciones para jóvenes inscritos en el plan de garantía juvenil, que deberán realizarse durante este año y contará con un presupuesto de 4.105.000 euros. estas ayudas, según explicó este jueves en rueda de prensa el consejero de educación, cultura y deportes, ángel felpeto, serán cofinanciadas por el fondo social europeo, la iniciativa de empleo juvenil y por la junta de comunidades de castilla-la mancha en el marco del programa operativo de empleo juvenil 2014-2020. en esta nueva convocatoria, que será publicada este viernes en el diario oficial de castilla-la mancha (docm), apuntó felpeto, "se han adaptado aspectos técnicos y de procedimiento con el objetivo de optimizar las ayudas ofrecidas". las entidades beneficiadas podrás ser privadas de iniciativa social y ciudadana sin ánimo de lucro, corporaciones de derecho público, organizaciones sindicales y patronales, fundaciones, los centros tecnológicos de la región y entidades declaradas de utilidad pública. la cuantía máxima por proyecto que se concederá por cada proyecto será de 500.000 euros, superables en casos excepcionales. además, cualquiera de las entidades beneficiadas por estas ayudas deberá desarrollar en los próximos meses, hasta el 31 de diciembre como fecha tope, un proyecto que comprenda un conjunto de actividades coordinadas para conseguir el objetivo final, que es la inserción laboral de los jóvenes desempleados de entre 16 y 30 años. como requisitos, deben estar empadronados en castilla-la mancha, estar desempleados y no ocupados ni integrados en sistemas de educación o formación, además de la ya citada de estar inscrito en el sistema nacional de garantía juvenil. los ámbitos de actuación de estas iniciativas son los siguientes: acciones de orientación profesional, información laboral y acompañamiento en la búsqueda de empleo; programas de segunda oportunidad; formación, especialmente en idiomas y tic; programas de movilidad para la mejora de competencias profesionales; formación para el emprendimiento y promoción de la cultura emprendedora; asesoramiento al autoempleo y creación de empresas para jóvenes universitarios así como fomento del empleo para jóvenes investigadores y prácticas no laborables en empresas. el consejero de educación, cultura y deportes recordó que, según los datos de la última encuesta de población activa, hay 48.700 jóvenes en la región que podrían beneficiarse. por ello, animó a todos estos jóvenes a que "se inscriban en el sistema nacional de garantía juvenil y participen en las ofertas de formación que realizarán las entidades beneficiarias". felpeto destacó el "interés despertado" en la primera de estas convocatorias, que contó con 27 proyectos realizados por 26 entidades, por lo que auguró una "excelente respuesta" en esta segunda. en este sentido, detalló que 13 proyectos fueron llevados a cabo por entidades sin ánimo de lucro, diez por ayuntamientos, dos por empresas y otros dos por la universidad de castilla-la mancha (uclm). además, resaltó que la oferta de esta primera convocatoria fue "muy variada" y abarcó iniciativas enfocadas desde la creación de aplicaciones móviles y reparación de estos dispositivos a cursos de fontanería, becas laborales a graduados y técnicos superiores combinadas con cursos de idiomas o cursos de consultores. de igual modo, hizo hincapié en que el 45,4 por ciento de los jóvenes participantes, un total de 1.363, han conseguido empleo en los seis meses siguientes, una cifra que felpeto espera que se amplíe ya que aún no han transcurrido estos seis meses desde la finalización de los últimos proyectos. & 596 & very low & Low & Socio-Economic & NA & NA & 2019-04-04 & 2019 & 3 & ECO
Frame & v.low & National & 500-1000 & 0.0141942 & -0.1980960 & -0.1536394 & -0.9382144 & 0.7927531 & 12.6 & 1.2981504 & 0.5794339 & Recipient & Domestic & Domestic & Domestic & Domestic|ECO & Positive\\
\addlinespace
Spain & http://www.libertaddigital.com/espana/2015-01-21/la-generalidad-trata-como-casos-aislados-el-medio-centenar-de-denuncias-por-la-inmersion-linguistica-1276538718/ & 621 & Libertad Digital & Private/Non-Public & Online only & Regional/Local & medium = CP is important part of story & Political capital/interests & Negative & EU + National + Subnational & No myth & NA & NA & NA & NA & NA & NA & NA & NA & Spain & la generalidad trata como "casos aislados" el medio centenar de denuncias por la inmersión lingüística & 2015-01-21 & fondo social europeo & el departamento de enseñanza de la generalidad de cataluña se ha especializado en incumplir las leyes educativas, desobedecer las sentencias sobre la enseñanza en español y en tirar por la borda millones de euros de financiación, sea europea o del ministerio de cultura. así lo ha puesto de relieve una pregunta de la diputada del pp maría josé garcía cuevas en el parlamento catalán sobre la aplicación en la comunidad autónoma del programa estatal de formación profesional básica, la propia lomce (ley orgánica para la mejora de la calidad educativa) y sobre las 53 reclamaciones judiciales que se han presentado, por el momento, por vulnerar los derechos lingüísticos de los alumnos. en este último caso, la generalidad se niega a que en los centros públicos se enseñe también en castellano y prefiere aplicar la inmersión forzosa, a pesar de fallos como el último del tribunal superior de justicia de cataluña (tsjc) en el que se condena a la administración autonómica al pago de una indemnización de tres mil euros tras la denuncia de unos padres. la sentencia estima que se perjudicó a una niña al retrasar su aprendizaje y someterla a un sobresfuerzo innecesario. sin embargo, la consejería de irene rigau (denunciada también por abrir los centros públicos para la votación del 9n) replica que los padres que solicitan la escolarización también en español de sus hijos son "casos aislados" y que la inmersión en las escuelas es "intocable". prueba de que a la generalidad sólo le interesa el mantenimiento a toda costa su procedimiento "educativo" y la erradicación del castellano en los colegios es su falta de atención a otros aspectos de la enseñanza como la formación profesional. así, la generalidad acaba de renunciar a 3,5 millones de euros por no ceñirse a los criterios mínimos fijados por el ministerio. así, se aprobó dicha cantidad para financiar la nueva fp a través del fondo social europeo. la partida, correspondiente al ejercicio pasado, no se abonó porque en la consejería de rigau se prefirió organizar un programa de formación alternativo saldado con un fracaso. los fondos europeos preveían que 680 alumnos se acogieran en 2014 a la fp básica diseñada por el ministerio. la generalidad, en cambio, optó por un modelo "experimental" en sólo cuatro colegios, con 54 alumnos y al margen de cualquier requisito legal, procediera de madrid o de bruselas, lo que provocó la anulación de la partida presupuestaria. sin embargo, eso no supone ningún problema para el gobierno de mas, empeñado en la construcción de su estado y quien segun la diputada garcía cuevas está dispuesto a perder más de cien millones de euros para la enseñanza por negarse a aplicar la lomce en cataluña. además y a su juicio, son muchos los padres que se sumarían a las denuncias por la inmersión lingüística pero no lo hacen por miedo a las represalias contra sus hijos. & 480 & medium & Medium & Power & NA & NA & 2015-01-21 & 2015 & 1 & POL
Frame & low-medium & Regional & <500 & 0.0141942 & -0.1980960 & -0.1536394 & -0.9382144 & 0.7927531 & 12.6 & 1.2981504 & 0.5794339 & Recipient & Domestic & European & Mixed & Domestic|POL & Negative\\
Spain & https://www.farodevigo.es/portada-pontevedra/2018/12/22/diputacion-recibe-cuatro-millones-cursos/2021644.html & 595 & Faro de Vigo & Private/Non-Public & Online and Offline & Regional/Local & very low = CP mentioned once & Jobs & Positive & Subnational & No myth & Social awareness/inclusion & Positive & Subnational & No myth & NA & NA & NA & NA & Spain & la diputación recibe cuatro millones para la cursos de formación de empleabilidad & 2018-12-22 & fondo social europeo & el equipo del gobierno provincial con los trabajadores de la diputación jubilados este año. // gustavo santos la diputación recibió una aportación de más de cuatro millones de euros del fondo social europeo que se destinará a impulsar promoción de empleo y de la formación en la provincial. será por medio de un plan, el "laboura 2020", con el que se pretende mejorar la empleabilidad de 855 personas para oficios que tienen mucha demanda. la presidenta provincial, carmela silva, anunciaba la aportación europea al programa, que supondrá una inversión del 80 por ciento, mientras que la diputación aportará cerca de otro millón de euros. "se trata de una actuación encamina a desarrollar once itinerarios integrados y personalizados dirigidos a la obtención de certificados de profesionalidad y mejorar la empleabilidad, respondiendo a las necesidades locales y generando empleo en los sectores con mayores oportunidades", según especificó silva. el proyecto tiene como objetivos la integración sostenible en el mercado de trabajo de las personas desempleadas que pertenezcan a colectivos vulnerables, desarrollar itinerarios formativos en diferentes localizaciones de la provincia y favorecer la participación de las mujeres, de parados de larga duración, menores de 30 años, mayores de 55 años y de personas con diversidad funcional y colectivos en riesgo de exclusión social. está previsto que se beneficien de "laboura 2020" 456 mujeres y 399 hombres de diferentes concellos de la provincia entre los que se encuentran pontevedra, barro, moraña, lalín, a guarda, cangas, vilagarcía de arousa, cambados, o porriño, mos, valga, ponteareas, o grove, tomiño, soutomaior y vigo. las 855 personas beneficiadas podrán participar en los itinerarios centrados en la formación de actividades auxiliares de almacén, atención sociosanitaria a personas dependientes en instituciones sociales y a domicilio, gestión de llamadas de teleasistencia, limpieza de superficies y mobiliario en edificios y locales, dinamización de actividades de tiempo libre educativo infantil y juvenil, socorrismo en instalaciones acuáticas, actividades auxiliares de conservación y mejora de montes, operaciones básicas de restaurante y bar, formación en competencias clave de nivel dos en lengua castellana y gallega y matemáticas. el proyecto incluye diez jornadas informativas de activación e intercambio de experiencias y buenas prácticas y acciones específicas en igualdad y no discriminación. empezarán el 1 de febrero. silva también salió al paso de las críticas del pp en el que denunciaba que la diputación pontevedresa gastaba poco en empleabilidad. la presidenta provincial indicó que pontevedra invierte un 4,25 por ciento del presupuesto, frente al 2,25 de la media estatal y muy por encima de otras provincias como ourense (2,02\%), málaga (3\%), palencia (3,31\%) o león (0,21), donde gobierna el pp. prácticas y prácticum depo silva también indicó que desde 2016, 300 alumnos y 32 profesores se vieron beneficiados del "prácticum depo" con el que pudieron viajar a otros países de la unión europea para completar su formación y tener experiencia labora en empresas e instituciones extranjeras y que según la presidenta provincial trata de compensar los recortes de la anterior ejecutiva estatal en el programa erasmus, a los que hay que sumar un centenar más de la universidad de vigo que se beneficiaron del convenio de la diputación para estudiar en el extranjero con una aportación de la administración provincial de 100.000 euros. desde el pp, en cambio, tildan de "fracaso estrepitoso" el plan de práctica laboral de la diputación. "hay una incompetencia absoluta en la gestión de este programa, que supondría una oportunidad real y muy importante para que los jóvenes pudiesen conseguir su primer puesto de trabajo según su formación", indicó el portavoz popular, ángel moldes que lamentaba que el bipartito dejase de invertir 8 millones de euros en 2017 y 2018. el popular criticó que con este plan no se consiguiese contratar a ninguna de las persona que ejercieron prácticas en las empresas. & 633 & very low & Low & Socio-Economic & Socio-Economic & NA & 2018-12-22 & 2018 & 3 & ECO
Frame & v.low & Regional & 500-1000 & 0.0141942 & -0.1980960 & -0.1536394 & -0.9382144 & 0.7927531 & 12.6 & 1.2981504 & 0.5794339 & Recipient & Domestic & Domestic & Domestic & Domestic|ECO & Positive\\
Spain & http://www.europapress.es/extremadura/noticia-adjudicada-asistencia-direccion-ambiental-obras-ave-madrid-extremadura-20160204121327.html & 684 & europa press & Private/Non-Public & Online only & National & very low = CP mentioned once & Infrastructure & Positive & Subnational & No myth & NA & NA & NA & NA & NA & NA & NA & NA & Spain & adjudicada la asistencia de la dirección ambiental de las obras del ave madrid-extremadura & 2016-02-04 & fondo europeo de desarrollo regional & actualizado 04/02/2016 12:17:43 cet mérida, 4 feb. (europa press) - adif alta velocidad ha adjudicado, por importe de 198.548 euros (iva incluido), el servicio de consultoría y asistencia técnica para la dirección ambiental de las obras sometidas a declaración de impacto ambiental (dia) pertenecientes a la línea de alta velocidad madrid-extremadura, en las provincias de cáceres y badajoz. el contrato contempla las labores de coordinación de los aspectos ambientales en el desarrollo de las obras, lo que incluye la elaboración de informes sobre el cumplimiento de las previsiones de las declaraciones de impacto ambiental y la supervisión de informes técnicos sobre la ejecución del programa de vigilancia ambiental. de igual forma, recoge el seguimiento ambiental de la obra, la sistematización de acciones correctivas y la unificación de criterios de actuación, entre otras tareas. los trabajos han sido adjudicados a la empresa alvartis asistencia técnica, según ha informado adif alta velocidad en nota de prensa. fondos europeos en el periodo 2007-2013 la línea madrid-extremadura ha recibido ayudas del fondo europeo de desarrollo regional (feder) a través del p.o. cohesión-feder y del p.o. de extremadura para las obras de plataforma, vía e instalaciones del tramo talayuela-cáceres-mérida y de las ayudas rte-t para los estudios y proyectos del tramo talayuela-frontera portuguesa y para las obras de plataforma del tramo mérida-badajoz-frontera portuguesa. en el periodo 2014-2020 está prevista su cofinanciación por el fondo europeo de desarrollo regional (feder) a través del p.o. crecimiento sostenible, objetivo temático 7: transporte sostenible. & 263 & very low & Low & Socio-Economic & NA & NA & 2016-02-04 & 2016 & 2 & ECO
Frame & v.low & National & <500 & 0.0141942 & -0.1980960 & -0.1536394 & -0.9382144 & 0.7927531 & 12.6 & 1.2981504 & 0.5794339 & Recipient & Domestic & Domestic & Domestic & Domestic|ECO & Positive\\
Spain & http://www.expansion.com/economia/2017/08/17/599551d4268e3e52238b4674.html & 653 & Expansión & Private/Non-Public & Online and Offline & National & very low = CP mentioned once & Jobs & Positive & National & No myth & NA & NA & NA & NA & NA & NA & NA & NA & Spain & el gobierno destinará 18 millones a ayudas para la formación digital y el empleo juvenil & 2017-08-17 & fondo social europeo & la ministra de empleo y seguridad social, fátima báñez, durante su intervención el 24 de julio en la uimp, en santander.efeexpansion el ministerio de energía, turismo y agenda digital, a través de la entidad pública red.es, destinará 18,36 millones de euros a ayudas para la formación digital y empleo juvenil de las que se beneficiarán 27 entidades, ha informado hoy el ministerio en un comunicado. en total, con la suma de la aportación de los beneficiarios, se movilizará un presupuesto de 24,1 millones de euros, que irá dirigido a la formación orientada a la industria digital y los nuevos modelos de negocio en 14 comunidades autónomas. ayudas de entre 10.000 y los 2 millones la cuantía de cada ayuda oscilará entre los 100.000 y los 2 millones de euros, que recibirán empresas, asociaciones y fundaciones con el fin de desarrollar 32 proyectos para realizar 74 actuaciones. la formación tiene el objetivo de facilitar el acceso de los jóvenes a puestos de trabajo que impulsen la transformación digital de las empresas. este proyecto está enmarcado en el "plan de inclusión digital y empleabilidad" de la agenda digital para españa, bajo el "programa operativo de empleo juvenil" cofinanciado por el fondo social europeo & 207 & very low & Low & Socio-Economic & NA & NA & 2017-08-17 & 2017 & 2 & ECO
Frame & v.low & National & <500 & 0.0141942 & -0.1980960 & -0.1536394 & -0.9382144 & 0.7927531 & 12.6 & 1.2981504 & 0.5794339 & Recipient & Domestic & Domestic & Domestic & Domestic|ECO & Positive\\
Spain & http://www.telecinco.es/informativos/sociedad/FSC-Inserta-Castellon-Empleo-discapacidad\_0\_1828050049.html & 690 & Telecinco & Private/Non-Public & Online only & National & low = CP mentioned more times but NOT important part of story (mainly about others issues) & Social awareness/inclusion & Positive & National & No myth & NA & NA & NA & NA & NA & NA & NA & NA & Spain & fsc inserta pone en marcha en castellón un taller de búsqueda activa de empleo para personas con discapacidad & 2014-07-15 & fondo social europeo & en concreto, este curso --cofinanciado por fundación once, el fondo social europeo y bankia-- se imparte hasta el próximo 30 de julio y tiene una duración de 60 horas lectivas.
 fsc inserta recuerda que los cursos y talleres proporcionan a los alumnos las destrezas profesionales que les permitan un rendimiento competitivo en el mercado de trabajo y el desarrollo de aptitudes y habilidades personales para conseguir la plena participación en su entorno laboral y social.
 todos los cursos de formación se enmarcan dentro del programa por talento, que está desarrollando fundación once a través de fsc inserta, con la cofinanciación del fondo social europeo, cuyo objetivo es incrementar la empleabilidad y la inserción laboral de las personas con discapacidad, según indica la entidad. & 123 & low & Low & Socio-Economic & NA & NA & 2014-07-15 & 2014 & 1 & ECO
Frame & low-medium & National & <500 & 0.0141942 & -0.1980960 & -0.1536394 & -0.9382144 & 0.7927531 & 12.6 & 1.2981504 & 0.5794339 & Recipient & Domestic & Domestic & Domestic & Domestic|ECO & Positive\\
\addlinespace
Spain & https://ecodiario.eleconomista.es/espana/noticias/9813233/04/19/Discapacidad-inserta-imparte-en-las-palmas-un-curso-de-experto-en-limpieza-de-inmuebles-para-personas-con-discapacidad.html & 648 & El Economista (EcoDiario) & Private/Non-Public & Online and Offline & National & very low = CP mentioned once & Social awareness/inclusion & Positive & National + Subnational & No myth & NA & NA & NA & NA & NA & NA & NA & NA & Spain & discapacidad. inserta imparte en las palmas un curso de experto en limpieza de inmuebles para personas con discapacidad & 2019-04-09 & fondo social europeo & inserta empleo, la entidad de fundación once para la formación y el empleo, ha comenzado a impartir en las palmas de gran canaria un curso de experto en limpieza de inmuebles al que asisten 15 personas con discapacidad. esta acción formativa, de 150 horas lectivas y que se imparte hasta el próximo 22 de mayo, está dividida en varios módulos formativos a través de los cuales los participantes abordarán contenidos relacionados con la limpieza diaria de inmuebles, manejo de equipos y maquinaria de limpieza. inserta proporciona a los alumnos las destrezas profesionales que les permitan un rendimiento competitivo en el mercado de trabajo y el desarrollo de aptitudes y habilidades personales para conseguir la plena participación en su entorno laboral y social. este curso se enmarca en los programas operativos de inclusión social y de la economía social (poises) y de empleo juvenil (poej), que está desarrollando fundación once a través de inserta, con la cofinanciación del fondo social europeo y la iniciativa de empleo juvenil, para incrementar la formación y el empleo de las personas con discapacidad. & 178 & very low & Low & Socio-Economic & NA & NA & 2019-04-09 & 2019 & 3 & ECO
Frame & v.low & National & <500 & 0.0141942 & -0.1980960 & -0.1536394 & -0.9382144 & 0.7927531 & 12.6 & 1.2981504 & 0.5794339 & Recipient & Domestic & Domestic & Domestic & Domestic|ECO & Positive\\
Spain & https://www.europapress.es/esandalucia/malaga/noticia-junta-destaca-potencial-uso-nuevas-tecnologias-aumentar-ventas-negocios-tradicionales-20181129150650.html & 594 & europa press & Private/Non-Public & Online only & National & very low = CP mentioned once & Research \& innovation & Positive & Subnational & No myth & Economic development & Positive & Subnational & No myth & NA & NA & NA & NA & Spain & la junta destaca el potencial del uso de las nuevas tecnologías para... & 2018-11-29 & fondo europeo de desarrollo regional & la viñuela (málaga), 29 nov. (europa press) - el consejero de empleo, empresa y comercio, javier carnero, ha destacado este jueves el potencial en términos de empleo y ventas que se deriva de la aplicación de las tecnologías de la información y la comunicación (tic) a los negocios tradicionales. así lo ha indicado en el transcurso de su visita a el mastrén, una panadería ecológica ubicada en la viñuela (málaga) cuyo principal rasgo distintivo, en opinión de carnero, "es su capacidad para hacer el pan con las técnicas de toda la vida, aunque aprovechando las oportunidades que ofrece una sociedad moderna y tecnologizada". pese a que la viñuela es un municipio de apenas 2.000 habitantes situado en la comarca de la axarquía-costa del sol, los clientes pueden adquirir los productos de esta panificadora local a través de internet, y recogerlos en puntos de distribución habilitados en diversas localidades de la provincia, entre ellas nerja, antequera, torre del mar (vélez-málaga) o la propia capital. "las tic han permitido multiplicar la distribución y con ella la producción y los ingresos, de un producto tradicional e innovador al mismo tiempo", ha asegurado el consejero. el promotor del proyecto, miguel ángel reina, transformó las circunstancias personales adversas que se derivaron de la crisis económica en una apuesta por el autoempleo "inyectando para ello aire fresco a la actividad tradicional de su familia". "una apuesta que, con mucho esfuerzo, le ha salido bien, y que incluso ha contribuido a generar puestos de trabajo", ha subrayado carnero. para la puesta en marcha de esta iniciativa, el mastrén ha recibido desde la agencia idea un incentivo de algo más de 24.300 euros en el marco de la subvención global competitividad-innovación-empleo de andalucía 2014-2020, integrada en el programa operativo feder andalucía 2014-2020, cofinanciado con el fondo europeo de desarrollo regional. "me alegra comprobar que la gestión del dinero puesto por los ciudadanos en manos de las administraciones permite desarrollar iniciativas productivas e innovadoras como esta", ha añadido el titular de la consejería. apoyos a empresas carnero ha recordado que con el objetivo de fomentar el emprendimiento y el desarrollo empresarial, se han puesto en funcionamiento una serie de iniciativas complementarias. "tenemos un compromiso con la industria, con todas las industrias", ha sostenido, exponiendo que se ha materializado fundamentalmente en tres medidas. ha citado la aprobación de la estrategia industrial de andalucía 2020, que sienta las bases para el desarrollo industrial presente y futuro de la comunidad autónoma, y que prevé movilizar unos 8.000 millones de inversión tanto pública como privada; de otro, el pacto por la industria, firmado en enero de 2017 y al que ya se han adherido unas 2.000 empresas, profesionales y entidades de todo tipo. en tercer lugar, ha continuado el consejero, se hallarían los 275 millones en ayudas e incentivos que se gestionan desde la agencia idea, donde a su vez se contemplan tres líneas: incentivos al desarrollo industrial, la competitividad empresarial, la transformación digital y la creación de empleo en andalucía, dotada de 145 millones; las ayudas a grandes empresas, con 12,5 millones; y los incentivos a la promoción de la investigación industrial, el desarrollo experimental y la innovación empresarial en andalucía, con otros 118 millones de euros. "a estas cantidades se sumarían los 278 millones que se han destinado al fomento del empleo para el bienio 2018-19 a través de las corporaciones locales. de ellos, 41,3 millones se invertirán en la provincia de málaga; a su vez, la viñuela recibirá una inversión de 68.800 euros, con los que se crearán siete puestos de trabajo", ha expuesto. & 607 & very low & Low & Socio-Economic & Socio-Economic & NA & 2018-11-29 & 2018 & 3 & ECO
Frame & v.low & National & 500-1000 & 0.0141942 & -0.1980960 & -0.1536394 & -0.9382144 & 0.7927531 & 12.6 & 1.2981504 & 0.5794339 & Recipient & Domestic & Domestic & Domestic & Domestic|ECO & Positive\\
Spain & http://www.antena3.com/noticias/economia/acusa-zapatero-defender-agricultores-ganadores\_20110408574702b06584a8f86268645f.html & 612 & Antena3.com & Private/Non-Public & Online only & National & very low = CP mentioned once & Institutional bargaining over funding & Factual & EU + National & No myth & NA & NA & NA & NA & NA & NA & NA & NA & Spain & el pp acusa a zapatero de no defender a los agricultores y ganadores & 2016-05-26 & política de cohesión & el líder del pp, mariano rajoy, ha planteado un decálogo de propuestas sobre la posición que españa ha de tener en la negociación de las próximas perspectivas financieras de la ue y de la política agraria común. en la clausura de un foro organizado por el pp sobre esa próxima negociación con la unión europea, rajoy ha criticado al gobierno por haber hecho que españa viva hoy su momento más débil en el exterior desde su adhesión a la ue. "siete años después (de la victoria del psoe), ni estamos ni se nos espera en algún debate importante", ha lamentado rajoy antes de precisar que no se alegra de ello porque españa afronta una negociación de enormes consecuencias para los agricultores y para las finanzas públicas. por ello, ha ofrecido consenso al gobierno para fijar de forma conjunta la posición de españa ante la negociación. la "inacción" del gobierno es lo que le ha llevado a hacer ese ofrecimiento y plantear un decálogo de propuestas sobre las que cree que debe girar la posición española en esa negociación. así, ha defendido que la financiación del presupuesto de la ue cuente con recursos suficientes y un sistema equilibrado de reparto de las cargas. a su juicio, el actual sistema de compensaciones distorsiona y desvirtúa la eficiencia y equilibrio del recurso relativo a la renta nacional bruta, y, por tanto, debería eliminarse. ha planteado también adaptar la política de cohesión a los objetivos de la estrategia 2020, dando más importancia al desempleo e introduciendo otros factores para el reparto de fondos como la brecha tecnológica, la innovación, los índices de abandono escolar temprano o la población inmigrante. asimismo, ha defendido que las regiones que abandonan el objetivo de convergencia (castilla-la mancha, andalucía y galicia) sigan teniendo unas dotaciones económicas para permitir que continúen su equiparación con las regiones más desarrolladas. ha propuesto también ayudas para las pymes y para la innovación, redes transeuropeas para garantizar la suficiencia energética de la ue y un refuerzo de los medios para atender a la inmigración. en el capítulo relativo a la pac ha apostado por mantener las dotaciones de esta política y velar por la aplicación del principio de reciprocidad en el ámbito de la actividad agraria. también ha exigido flexibilidad en el reparto de la pac para evitar distorsiones internas, la puesta en marcha de mecanismos eficaces de regulación de mercado y de gestión de crisis, e impulsar medidas para la incorporación jóvenes y mujeres a la actividad agraria. "a nosotros nos importa la agricultura, es fundamental que el gobierno dé la batalla y nosotros vamos apoyarle en defensa de nuestros intereses en europa", ha añadido antes de lamentar que exista la sensación de que el ejecutivo "no dedica ni un minuto" a estos asuntos. el presidente de los eurodiputados del pp, jaime mayor oreja, ha tenido una intervención en esa misma línea al asegurar que con el actual gobierno españa ha perdido peso específico en la ue. "españa ha abandonado el puente de mando de la ue y ya no es sólo un país periférico geográficamente, sino también en su peso político", ha criticado. para él, españa afronta una negociación muy difícil en la ue desde una posición débil y con un gobierno "interino". por ello, ha instado al ejecutivo a contar con el pp porque, convencido de que ganará las próximas elecciones generales, será este partido el que tendrá que concluir las negociaciones. no obstante, no ha confiado en que el gobierno atienda la llamada al consenso que realiza el pp. por su parte, la presidenta de la comunidad madrileña, esperanza aguirre, y el alcalde de madrid, alberto ruiz-gallardón, han exigido que españa vuelva "al lugar que le corresponde" en la ue y, por ello, han apostado por la libertad, la austeridad y la recuperación de la confianza perdida en europa. & 641 & very low & Low & Power & NA & NA & 2016-05-26 & 2016 & 2 & POL
Frame & v.low & National & 500-1000 & 0.0141942 & -0.1980960 & -0.1536394 & -0.9382144 & 0.7927531 & 12.6 & 1.2981504 & 0.5794339 & Recipient & Domestic & European & Mixed & Domestic|POL & Neutral\\
Spain & https://www.20minutos.es/noticia/3602345/0/junta-destina-12-4-millones-para-programas-mixtos-formacion-empleo-para-mas-1-100-trabajadores/ & 630 & 20 minutos & Private/Non-Public & Online and Offline & National & very low = CP mentioned once & Jobs & Positive & Subnational & No myth & NA & NA & NA & NA & NA & NA & NA & NA & Spain & la junta destina 12,4 millones para programas mixtos de formación y empleo para más de 1.100 trabajadores & 2019-03-31 & fondo social europeo & el primer programa invertirá 10 millones de euros para impartir formación en alternancia con un trabajo en el que participarán 866 desempleados. el segundo programa, con características similares y financiado por el fondo social europeo, destinará 2,4 millones a formación y empleo y participarán 240 jóvenes inscritos en el sistema nacional de garantía juvenil durante un periodo de seis meses, señala la junta a través de un comunicado remitido a europa press. los programas mixtos alternan acciones de formación con un trabajo efectivo a través de la ejecución de obras o servicios de utilidad pública o interés social y permiten la adquisición de competencias profesionales relacionadas con dichas obras o servicios. además, los participantes cuentan con un contrato de trabajo con una remuneración efectiva. los beneficiarios de estas ayudas serán entidades locales y entidades sin ánimo de lucro entre cuyos fines estén la formación y el empleo, y que desarrollen la ejecución de obras y servicios de utilidad pública o interés social, detallan desde la consejería. estas ayudas tienen como destinatarios preferentes a las personas pertenecientes a colectivos establecidos como prioritarios en la ii estrategia integrada de empleo, formación profesional, prevención de riesgos laborales e igualdad y conciliación en el empleo, 2016-2020: jóvenes menores de 35 años, preferentemente sin cualificación; mayores de 45 años, especialmente quienes carezcan de prestaciones y presenten cargas familiares; y parados de larga y muy larga duración, con especial atención a aquellos que hayan agotado sus prestaciones por desempleado y las personas en riesgo de exclusión social. la convocatoria específica destinada a jóvenes pertenecientes a la garantía juvenil tiene como objetivo concreto la reducción del desempleo juvenil, concluye el comunicado. & 276 & very low & Low & Socio-Economic & NA & NA & 2019-03-31 & 2019 & 3 & ECO
Frame & v.low & National & <500 & 0.0141942 & -0.1980960 & -0.1536394 & -0.9382144 & 0.7927531 & 12.6 & 1.2981504 & 0.5794339 & Recipient & Domestic & Domestic & Domestic & Domestic|ECO & Positive\\
Spain & http://www.diariosur.es/internacional/union-europea/201701/17/impacto-presupuestario-brexit-millones-20170117164712-rc.html & 606 & Sur & Private/Non-Public & Online and Offline & Regional/Local & very low = CP mentioned once & Institutional bargaining over funding & Balanced & EU + Other country & No myth & NA & NA & NA & NA & NA & NA & NA & NA & Spain & el & 2017-01-17 & política de cohesión & "esperamos una negociación difícil, visto que reforzaría probablemente las divisiones existentes", aseguran los autores del estudio la salida de reino unido de la unión europea (ue) representaría un déficit de unos 10.000 millones de euros anuales en el presupuesto comunitario, señala un estudio, que pronostica una negociación "difícil" entre los 27 para hacer frente a este "impacto". "nuestras estimaciones para el déficit ocasionado por el 'brexit' en el presupuesto de la ue varían de 5.000 a 17.000 millones de euros anuales", pero el monto de 10.000 millones (unos 10.700 millones de dólares) es el "caso más probable", según el estudio del centro de análisis instituto jacques delors, publicado el lunes. esta cantidad correspondería a la contribución neta de los británicos a la unión, es decir la diferencia entre lo que aportan y reciben, que "se eleva a unos 10.000 millones de media en los últimos cinco años", señala el informe firmado por los investigadores eulalia rubio y jörg haas. el déficit sería todavía mayor si la unión europea decide mantener su nivel de gasto, destinando a otro lugar las sumas transferidas actualmente a proyectos en reino unido. el impacto sería inferior, en cambio, si reino unido sigue participando en ciertos programas europeos y aportando una contribución, como noruega. incluso en caso de un 'brexit' "duro", el déficit podría ser "un poco más atenuado" por los ingresos en tasas de aduanas que reino unido pagaría a europa, apuntan los autores. a pesar de las incertidumbres, el estudio considera que "el 'brexit' representará un impacto para el presupuesto de la ue" que deberán afrontar los 27 "aumentando las contribuciones nacionales, reduciendo el gasto o combinando ambas opciones". "esperamos una negociación difícil, visto que el 'brexit' reforzaría probablemente las divisiones existentes entre los contribuyentes netos y los beneficiarios netos de la ue", apuntan. los eventuales recortes afectarían probablemente, según el estudio, a los dos apartados presupuestarios "más importantes": la política agrícola común (pac) y la política de cohesión, que busca armonizar el desarrollo de las diferentes regiones de la ue. la ue podría también recibir "un ingreso adicional mediante un nuevo recurso propio o una revisión completa de sus fuentes de ingresos", agregan los investigadores, en referencia a las propuestas en este sentido de un grupo de reflexión europeo presentadas por el ex primer ministro italiano, mario monti. según la actitud que adopten los 27, "el 'brexit' puede constituir una oportunidad o una amenaza para el presupuesto de la ue", concluyen los autores. & 416 & very low & Low & Power & NA & NA & 2017-01-17 & 2017 & 2 & POL
Frame & v.low & Regional & <500 & 0.0141942 & -0.1980960 & -0.1536394 & -0.9382144 & 0.7927531 & 12.6 & 1.2981504 & 0.5794339 & Recipient & European & European & European & European|POL & Neutral\\
\addlinespace
Spain & http://www.elmundo.es/andalucia/2017/11/13/5a089b1546163ffd118b45ed.html & 598 & EL MUNDO & Private/Non-Public & Online and Offline & National & medium = CP is important part of story & Fraud/Corruption & Negative & National + Subnational & 7.Fraud & NA & NA & NA & NA & NA & NA & NA & NA & Spain & por qué el fraude de la formación no es ningún 'bluf' & 2017-11-13 & fondo social europeo & la cantidad defraudada continúa creciendo en los recuentos de la junta y la investigación sigue viva la junta admite ya un agujero de 149 millones en subvenciones sin justificar "el supuesto fraude de la formación es un bluf que se desinfla en los tribunales y que fue aireado por el pp tras una investigación sin fundamento de la guardia civil". éste es más o menos el mantra que repiten una y otra vez desde el gobierno andaluz para crear un nuevo estado de opinión sobre uno de los agujeros negros en la gestión de los gobiernos de la junta. la tesis se sostiene fundamentalmente en la decisión de la juez maría núñez bolaños de archivar (tras la petición de la fiscalía) de la que se denomina pieza política del caso, la que investigaba una supuesta trama en la administración de la junta dirigida a desviar sistemáticamente los fondos de la formación a empresas afines o vinculadas a las redes clientelares del psoe andaluz. 39 empresas vinculadas al psoe la juez considera que no hay indicio alguno de que hubiera una red organizada para eludir los controles en el reparto y justificación de los fondos. aunque en el mismo auto con el que decreta el archivo señala que "entre las más de 18.000 subvenciones otorgadas" se han encontrado "39 empresas que tienen alguna vinculación con el psoe", pese a lo cual considera que el dato "no resulta significativo". entre esas empresas figura en un puesto destacado el entramado del ex consejero socialista ángel ojeda, propietario de hasta siete sociedades que recibieron 52 millones de euros en subvenciones, y que están bajo la lupa de la justicia en las muchas causas judiciales que, a diferencia de la causa política, siguen vivas. otras 38 empresas vinculadas a cargos del psoe y del gobierno andaluz se habrían repartido otros 48,39 millones de euros. por otro lado, el juzgado de núñez bolaños no es el único que investiga el fraude de la formación. hay aún casi medio centenar de causas relacionadas con las irregularidades detectadas en el reparto de las ayudas, distribuidas entre una veintena de jueces de sevilla, granada, almería, huelva, málaga y jaén, según los datos del tribunal superior de justicia de andalucía (tsja). devolución pero no sólo sigue habiendo procesos judiciales abiertos que investigan el uso fraudulento de los fondos de la formación. la propia junta de andalucía ha admitido que ha pedido ya la devolución de 149 millones de euros en ayudas pagadas a las entidades encargadas de impartir los cursos y que no han sido debidamente justificadas. esa cifra tendrá que ir creciendo inevitablemente en los próximos meses, pues el consejero de empleo, javier carnero, reconocía la pasada semana en el parlamento andaluz que, de los 8.505 expedientes investigados, sólo se han concluido 6.501. lo que significa que aún quedan por revisar 2.004 expedientes. y todo ello, sin olvidar que lo que se revisa es sólo un muestreo del total de las ayudas concedidas en el periodo investigado. perlas en la comisión la comisión parlamentaria que investigó a nivel político el fraude de la formación terminó sin un relato único acordado por mayoría parlamentaria. pero eso no impidió sacar a la luz, a través de testimonios de testigos y técnicos, un rosario de irregularidades cometidas con los fondos públicos aprovechando la barra libre sin control que se instaló en la junta. así, por ejemplo, se supo que el marido de susana díaz, josé maría moriche, fue contratado por una fundación de ugt como auxiliar administrativo y su sueldo se pagó con cargo a, nada menos, 102 cursos de formación diferentes (subvencionados por la junta), que abordaban temáticas tan dispares como informática, energías renovables o seguridad privada. también se pudo conocer que el que fuera número 2 del psoe andaluz, rafael velasco, cargó la luz del domicilio familiar a un curso de formación que la junta concedió a una academia de su propiedad. o que ex alcalde socialista de punta umbría, gonzalo rodríguez nevada, compró en su restaurante con dinero de la formación 80 kilos de pez espada, 20 muñecas de comunión y varias cajas de ron legendario. precisamente, el interventor provincial de la junta en huelva, miguel ángel garcía, llegó a hablar de un "100 por 100 de irregularidades" en los expedientes revisados, cuando lo normal en otras subvenciones fiscalizadas es que se hallen "casos aislados". fondos europeos el fondo social europeo suspendió las subvenciones a la junta de andalucía en 2014 por las irregularidades detectadas en la gestión de los fondos. y el propio gobierno de susana díaz ha tenido paralizado los cursos durante cinco años para poner orden en la gestión de los mismos y proceder a la revisión de los expedientes, un examen que se tendría que haber realizado a medida que se repartían los fondos pero que brilló por su ausencia durante años. curiosamente, los sindicatos ugt y ccoo, que han sido también objeto de las diferentes investigaciones como entidades receptoras de las subvenciones, han iniciado una campaña para reclamar los 1.000 millones de euros que los andaluces habrían pagado de sus nóminas (a través de una retención obligatoria) para formación durante los años en los que no se han celebrado cursos. y más curioso aún resulta que la junta, que tomó la decisión de suspender los cursos, se haya sumado a esa campaña, reclamando al gobierno de rajoy un plan extraordinario para recuperar el tiempo y los recursos perdidos. cortafuegos si algo ha intentado susana díaz en relación al escándalo de la formación es hacer de cortafuegos para que su gestión no se vea cuestionada. sin embargo, los contratos de su marido la salpican de lleno en las supuestas irregularidades cometidas por el sindicato ugt. es más, las medidas tomadas por su gobierno para evitar que un descontrol tan absoluto pueda repetirse en la distribución de los fondos han sido también cuestionadas por el actual interventor general de la junta, adolfo j. garcía, quien, en un informe de actuación fechado el 17 de diciembre de 2015, afirmó que, a fecha de 31 de diciembre de 2012, había 21.800 expedientes por un importe de más de 700 millones de euros que no se habían justificado dentro del plazo fijado, con lo que la junta corría el riesgo de perder cualquier derecho de reintegro de esos fondos si se superaban los cuatro años que establece la ley general de subvenciones para reclamar la devolución. & 1074 & medium & Medium & Governance & NA & NA & 2017-11-13 & 2017 & 2 & POL
Frame & low-medium & National & +1000 & 0.0141942 & -0.1980960 & -0.1536394 & -0.9382144 & 0.7927531 & 12.6 & 1.2981504 & 0.5794339 & Recipient & Domestic & Domestic & Domestic & Domestic|POL & Negative\\
Spain & http://www.cuatro.com/noticias/internacional/Bruselas-propondra-financiar-capacidades-investigacion\_0\_2283750883.html & 607 & Cuatro & Private/Non-Public & Online only & National & very low = CP mentioned once & Institutional bargaining over funding & Factual & EU & No myth & NA & NA & NA & NA & NA & NA & NA & NA & Spain & bruselas propondrá este miércoles un fondo de defensa para financiar capacidades conjuntas y ayudas a investigación & 2016-11-29 & fondos estructurales & el ejecutivo comunitario presentará este miércoles un plan de acción sobre defensa con medidas para incentivar una mayor cooperación en el ámbito militar y de defensa entre los estados miembro y que aúnen parte de sus propios recursos nacionales para aumentar la eficiencia y rendimiento del gasto en defensa, especialmente en los equipos y tecnologías estratégicas para la unión la comisaria de industria y mercado interior, elzbieta bienkowska, avanzó a principios de noviembre que plantearía la posibilidad de crear "un fondo europeo de defensa para apoyar la financiación de programas de capacidades de defensa conjuntamente acordados" a partir de "un modelo de financiación innovador y atractivo". "cooperar en las capacidades de defensa a nivel europeo no es una opción hoy. esto es una necesidad", justificó, insistiendo en la importancia de que los gobiernos decidan "juntos" las capacidades necesarias para hacer economías de escala, evitar duplicidades y promover la interoperabilidad entre los equipos. el fondo se podría financiar con eurobonos de defensa a modo de garantías públicas para respaldar las inversiones, una de las posibles opciones, aunque también se podría recurrir a la financiación del banco europeo de inversiones (bei) si se modifica su estatuto o a través de la reasignación de fondos estructurales, según fuentes diplomáticas, aunque fuentes comunitarias han evitado avanzar la fórmula para financiar el fondo y si planteará una cifra concreta para el fondo. el primer pilar del plan de acción, según avanzó la comisaria de industria, se centrará en "comenzar financiando la investigación en defensa a nivel de la ue", algo tabú hasta ahora. ello se hará a través de la acción preparatoria, que contará con 25 millones de presupuesto el primer año en 2017 y un total de 90 millones en los tres años que cubrirá y servirá para allanar el camino para un programa de investigación en defensa más amplio a partir de 2021, con unos 500 millones de euros al año de presupuesto comunitario. "tiene sentido empezar por la fase de investigación porque la investigación cooperativa es la primera fase para programas futuros, la cooperación industria y equipos comunes. con esta propuesta queremos incentivar la cooperación desde el principio", explicó la comisaria. el apoyo a la industria es clave para contribuir su base tecnológica y garantizar la autonomía "estratégica" de la unión tras años de caída en los presupuestos en defensa en los estados miembro por la crisis económica, que destinaron conjuntamente el doble menos que estados unidos a defensa solo en 2015. washington lleva años pidiendo a los países europeos que destinen más a defensa, incluido el presidente electo, donald trump. países como rusia y china han aumentado además fuertemente sus presupuestos de defensa en los últimos años. apoyo, también a las pymes el plan también contemplará explorar financiación del banco europeo de inversiones para las pequeñas y medianas empresas del sector en defensa con el objetivo de alentar la inversión en la cadena de suministro de defensa y reforzar el mercado interior de defensa con el objetivo de promover la adquisición conjunta -el 80\% de las contrataciones en defensa en la actualidad son a título nacional-- y la estandarización, evitando duplicidades de capacidades militares costosas. asimismo, contempla que los programas espaciales de la ue puedan contribuir a la seguridad y defensa. la falta de cooperación en defensa entre los estados miembro cuestan entre 25.000 millones y 100.000 millones de euros anualmente, según las estimaciones de bruselas. se espera que los jefes de estado y de gobierno de la ue decidan en la cumbre de diciembre sobre el plan de acción sobre defensa y el refuerzo de la defensa europea a la luz de la nueva estrategia de política exterior y de seguridad de la unión, así como sobre el refuerzo de la cooperación entre la ue y la otan. los ministros de exteriores y de defensa de los veintiocho acordaron en noviembre aumentar la autonomía estratégica europea y explorar la cooperación estructurada permanente en defensa para permitir a grupos de países avanzar por su cuenta en el desarrollo de capacidades o lanzar una operación, tal y como les planteó la alta representante de política exterior y de seguridad común de la ue en la estrategia global. & 697 & very low & Low & Power & NA & NA & 2016-11-29 & 2016 & 2 & POL
Frame & v.low & National & 500-1000 & 0.0141942 & -0.1980960 & -0.1536394 & -0.9382144 & 0.7927531 & 12.6 & 1.2981504 & 0.5794339 & Recipient & European & European & European & European|POL & Neutral\\
Spain & http://www.expansion.com/empresas/inmobiliario/2016/12/01/5840062922601d7e7b8b4590.html & 650 & Expansión & Private/Non-Public & Online and Offline & National & very low = CP mentioned once & Infrastructure & Positive & EU + National & No myth & NA & NA & NA & NA & NA & NA & NA & NA & Spain & ferrovial se adjudica cinco contratos en polonia por 175 millones & 2016-12-01 & fondo europeo de desarrollo regional & a través de su filial budimex, la empresa se ha hecho con tres proyectos de carreteras, la ampliación del hospital de torún y la construcción de un gasoducto. en concreto, budimex ha sido seleccionada por la oficina municipal de bialystok para llevar a cabo tres proyectos de carreteras que alcanzan los 82 millones de euros. el contrato comprende la ampliación de la segunda y tercera etapa de la conocida como carretera de la independencia, dentro de la carretera regional número 669 de la región, según ha indicado la compañía. los trabajos abarcan la construcción de rotondas de dos niveles, túneles, pasos elevados, pasarelas, intersecciones, vías de servicio, vías para bicicletas, aceras, andenes para autobuses y pantallas acústicas. la ejecución de la segunda etapa asciende a 31,5 millones de euros y la de la tercera a 32,5 millones de euros. asimismo, también se va a construir en la ciudad de bialystok la calle de tsiolkovsky por más de 18 millones de euros. por otro lado, budimex se ha adjudicado la reconstrucción y ampliación del hospital regional de la ciudad polaca de torún, en la región polaca de kujawsko-pomorskie, por 75 millones de euros. el centro hospitalario será objeto de una reforma integral y, como parte de la expansión del complejo, se crearán 6 nuevos edificios con una superficie total de 63.000 m2. el mayor de ellos, albergará las urgencias y tendrá 12 quirófanos, 318 camas para pacientes y helipuerto. los nuevos pabellones también proporcionan 196 camas para pacientes de la unidad de psiquiatría y 39 para aquellos con enfermedades infecciosas, así como consultas externas. el hospital tendrá además 303 plazas de aparcamiento. está previsto que los trabajos se prolonguen durante 36 meses. además, la compañía se ha adjudicado la construcción de un gasoducto de alta presión y la infraestructura necesaria para su funcionamiento por 17,6 millones de euros. está previsto que los trabajos se desarrollen durante 19 meses. el proyecto, cofinanciado por la unión europea con cargo al fondo europeo de desarrollo regional, forma parte de la expansión del sistema de transmisión de gas de baja silesia y del programa de inversión asociado con el lanzamiento de la terminal de gas natural licuado de swinoujscie y la creación de un corredor de gas norte-sur. & 379 & very low & Low & Socio-Economic & NA & NA & 2016-12-01 & 2016 & 2 & ECO
Frame & v.low & National & <500 & 0.0141942 & -0.1980960 & -0.1536394 & -0.9382144 & 0.7927531 & 12.6 & 1.2981504 & 0.5794339 & Recipient & Domestic & European & Mixed & Domestic|ECO & Positive\\
Spain & http://www.eleconomista.es/mercados-cotizaciones/noticias/9110759/05/18/Eurodiputados-espanoles-de-izquierda-critican-la-propuesta-presupuestaria-de-la-CE.html & 622 & El Economista & Private/Non-Public & Online and Offline & National & medium = CP is important part of story & Political leverage & Balanced & EU + National & No myth & Institutional bargaining over funding & Negative & EU + National & No myth & NA & NA & NA & NA & Spain & eurodiputados españoles de izquierda critican la propuesta presupuestaria de la ce & 2018-05-02 & política de cohesión & abre una cuenta en degiro.es y consigue 500 euros gratis en comisiones bruselas, 2 may (efe).- eurodiputados españoles de diferentes partidos políticos de izquierda criticaron hoy la propuesta de la comisión europea (ce) para el presupuesto plurianual que cubrirá el periodo 2021-2027, al considerarla falta de ambición. la eurodiputada socialista eider gardiazabal opinó que el ejecutivo comunitario ha perdido "una gran oportunidad política" para hacer una "propuesta valiente". "hace diez años que el proyecto se está tambaleando y esta era la ocasión de demostrar que la unión europea está viva", declaró la política en el pleno donde se presentó la propuesta de la ce. añadió que era la ocasión "para no tener que recortar en la pac (política agraria común) o la política de cohesión, que es donde se demuestra la solidaridad". su compañero de partido, josé blanco, lamentó la "falta de ambición" de bruselas y señaló que la sociedad exige "una política de cohesión fuerte y mantener la financiación para la política de agricultura y pesca, tan importantes para galicia". "necesitamos que los estados miembros se rasquen el bolsillo, necesitamos más recursos", dijo blanco. el eurodiputado de erc jordi solé denunció que la propuesta de presupuesto incluye "puntos preocupantes", en particular sobre "el volumen del presupuesto", y consideró que "para hacer frente a los retos se necesitan más recursos". el parlamentario de icv ernest urtasun coincidió en indicar que "este presupuesto es una nueva oportunidad perdida". "creemos que el nuevo marco financiero de la ue debería sufrir un cambio radical de orientación", resaltó urtasun, quien agregó que la unión económica y monetaria "necesita una profunda reforma" y precisó que ve con preocupación "que se cree un fondo de 25.000 millones para el apoyo de las reformas estructurales que han supuesto en muchos casos el empeoramiento de las políticas sociales y de bienestar". "nuestro grupo se opondrá a cualquier condicionalidad macroeconómica en los presupuestos. al mismo tiempo, el denominado fondo anticrisis de 30.000 millones nos parece que tiene una dotación insuficiente", señaló. si bien los eurodiputados españoles del pp no valoraron la propuesta presupuestaria en el pleno, en un comunicado el presidente del partido popular europeo, el alemán manfred weber, consideró que el proyecto de la comisión es "un buen punto de partida para las discusiones". el popular portugués josé manuel fernandes pidió, por su parte, "debatir" la propuesta de recortes de los pagos directos de la pac en un 4 \% o en el 7 \% en cohesión, si bien dio la bienvenida al aumento de los fondos para la investigación, el programa erasmus, el cambio climático y la seguridad y defensa. "pregunta a los estados miembros: si están de acuerdo en recortar fondos de agricultura y cohesión", avisó. el líder del grupo liberal en el parlamento europeo, guy verhofstadt, aplaudió en twitter la idea de vincular la recepción de los fondos europeos al respeto del estado de derecho y dijo que si el gobierno de un estado miembro no quiere "nuestros valores europeos", tampoco necesita "nuestro dinero europeo". la comisión europea presentó hoy su propuesta de presupuesto a largo plazo para el periodo 2021-2027, en la que pide elevarlo al 1,11 \% de la renta nacional bruta conjunta, frente al 1,03 \% actual. & 538 & medium & Medium & Power & Power & NA & 2018-05-02 & 2018 & 3 & POL
Frame & low-medium & National & 500-1000 & 0.0141942 & -0.1980960 & -0.1536394 & -0.9382144 & 0.7927531 & 12.6 & 1.2981504 & 0.5794339 & Recipient & Domestic & European & Mixed & Domestic|POL & Neutral\\
Spain & http://ecodiario.eleconomista.es/espana/noticias/9000664/03/18/El-SAS-invierte-150000-euros-en-la-mejora-del-area-de-Urgencias-de-EsteponaOeste.html & 680 & El Economista & Private/Non-Public & Online and Offline & National & low = CP mentioned more times but NOT important part of story (mainly about others issues) & Social justice & Positive & Subnational & No myth & NA & NA & NA & NA & NA & NA & NA & NA & Spain & el sas invierte 150.000 euros en la mejora del área de urgencias de estepona-oeste & 2018-03-13 & fondo europeo de desarrollo regional & la diputación invierte más de 2,2 millones de euros en llevar al pfea al interior de la provincia (6/03) el servicio andaluz de salud (sas) ha invertido 150.000 euros en la mejora del área de urgencias del centro de salud de estepona-oeste. las obras, que se están desarrollando intentando no distorsionar la actividad diaria del centro, han consistido en el modificación del área de críticos, la creación de una nueva sala de triaje, así como la adaptación de un baño para personas con discapacidad y la creación de dos entradas independientes para ambulancias y ciudadanía. estepona (málaga), 13 (europa press) los trabajos se han dividido en dos anualidades (2017 y 2018) y están sufragados, con fondos fondo europeo de desarrollo regional(feder), concretamente con 60.000 euros. los delegados de gobierno e igualdad, salud y políticas sociales, josé luis ruiz espejo y ana isabel gonzález, acompañados por la gerente del distrito sanitario costa del sol, maría dolores llamas, y el alcalde del estepona (málaga), josé maría garcía urbano, han visitado este martes las instalaciones. en este sentido, ruiz espejo ha explicado que uno de los objetivos de estos trabajos "es hacer unas urgencias más accesibles para toda la ciudadanía, remodelando y ampliando la sala de espera e instalando puertas automáticas". además, supone un cambio en la organización de las consultas de enfermería y de medicina del área de urgencias, con nuevos boxes y cambios en la zona destinada al personal. asimismo, se va a instalar un sistema de gestión de cita, según las prioridades establecidas por los profesionales para la atención en urgencias. estos trabajos también incluyen la renovación del mobiliario y la instalación del aire acondicionado en todo el centro de salud. "pero estas obras suponen, sobre todo, una apuesta por la mejora de la accesibilidad y de las medidas de protección de la intimidad de las personas que acuden a urgencias", ha valorado ruiz espejo, añadiendo, además, que "las obras terminarán este mes con la puesta en marcha de un nuevo sistema de gestión de citas que redundará en una mejor accesibilidad y seguridad del paciente". a cada persona que acuda a urgencias y tras dar sus datos en el mostrador de admisión, se le asignarán unas iniciales y un número por los que será llamado para su atención. de este modo, ninguno de los allí presentes tendrá que conocer datos personales del enfermo como nombre y apellidos, además de asegurar una identificación inequívoca previniendo posibles errores posteriores. el número aparecerá en una pantalla situada en la sala de espera para así asegurar la accesibilidad de las personas con problemas de audición. este nuevo sistema acompañará al paciente en su itinerario por las salas y servicios de urgencias que necesite, avisándole, desde la sala de espera, de cuándo le llega su turno y a qué sala debe acudir: consulta de triaje, sala de radiodiagnóstico o consulta de enfermería. mejoras de accesibilidad la remodelación de las urgencias del centro de salud forma parte de las obras de reforma y actuaciones puestas en marcha por el servicio andaluz de salud para mejorar la accesibilidad en un total de 16 centros de atención primaria de la provincia, priorizando los ubicados en zonas con necesidades de transformación social y mediante fondos feder. en total, han sido 16 actuaciones que se desarrollan en el marco de la estrategia de renovación de atención primaria que impulsa la consejería de salud y que suponen un total de 306.990,14 euros. las actuaciones han conllevado medidas de accesibilidad como reformas en accesos y puertas interiores y exteriores, adaptación de aseos, construcción o sustitución de rampas de acceso, entre otros. asimismo, también se han acometido reparaciones de fachadas o cubiertas, pintura, climatización, sustitución de puertas manuales por automáticas, reposición de grupos electrógenos, actualización de medidas contraincendios, luminarias o reordenación de espacios, además de renovación o incorporación de mobiliario clínico. en concreto, el distrito sanitario costa del sol esta llevando a cabo mejoras por un importe total de 105.000 euros, en 10 centros de salud de la comarca. en este sentido, la delegada de salud ha manifestado que con todo ello, "se da respuesta a un importante número de actuaciones de mejora identificadas por los equipos directivos de los centros de salud y distrito sanitario, atendiendo a las necesidades expresadas por los propios profesionales que trabajan en estos centros y los colectivos ciudadanos y de pacientes representados en las comisión de participación". nuevo equipamiento además, el sas ha licitado, tras la publicación en el boletín oficial del estado y el perfil del contratante, la adquisición de 98 nuevos ecógrafos, 49 salas de radiología digital y 48 retinógrafos --equipos diagnósticos para la detección de problemas oculares-- que se destinarán a centros de atención primaria de las ocho provincias andaluzas, priorizando los ubicados en zonas con necesidades de transformación social y mediante fondos feder, con un presupuesto de 8,5 millones de euros. en málaga, se instalarán 13 ecógrafos --uno para dispositivo de urgencias estepona-marbella--, cinco radiología digital --uno en el centro de salud la lobilla--, y seis retinógrafos. & 846 & low & Low & Socio-Economic & NA & NA & 2018-03-13 & 2018 & 3 & ECO
Frame & low-medium & National & 500-1000 & 0.0141942 & -0.1980960 & -0.1536394 & -0.9382144 & 0.7927531 & 12.6 & 1.2981504 & 0.5794339 & Recipient & Domestic & Domestic & Domestic & Domestic|ECO & Positive\\
\addlinespace
Spain & http://ecodiario.eleconomista.es/sociedad/noticias/8421693/06/17/Villalba-pone-en-marcha-una-campana-contra-la-brecha-salarial-femenina.html & 647 & El Economista (EcoDiario) & Private/Non-Public & Online and Offline & National & very low = CP mentioned once & Social justice & Positive & Subnational & No myth & NA & NA & NA & NA & NA & NA & NA & NA & Spain & villalba pone en marcha una campaña contra la brecha salarial femenina & 2017-06-11 & fondo social europeo & collado villalba, 11 jun (efe).- el ayuntamiento de collado villalba ha puesto en marcha una campaña de corresponsabilidad social frente a la brecha salarial para conciencias y promocionar la igualdad entre hombres y mujeres. el consistorio tratará de generar un mayor compromiso en la implantación de medidas de conciliación de la vida laboral y personal en las empresas del municipio, como detalla en una nota de prensa. la campaña se compone de acciones divulgativas y sensibilizadoras dirigidas especialmente al tejido empresarial, pero también a sindicatos y público en general. en este sentido, se realizará una muestra representativa del empresariado local sobre medidas conciliadoras aplicadas en sus empresas en la actualidad como buena práctica en igualdad empresarial. las empresas que participen verán sus logotipos impresos en el material divulgativo de la campaña. sobre este sentido, se llevarán a cabo conferencias y el diseño y difusión de un logotipo así como la elaboración y edición de un documento con imágenes y mensajes favorables a la igualdad salarial en las empresas y a la corresponsabilidad para facilitar la conciliación vida laboral-personal. el material recogerá aportaciones de las entidades colaboradoras y tiene como finalidad concienciar, dar voz y servir de herramienta divulgadora. también se pondrán a disposición de los ciudadanos una guía orientativa sobre información, detección y medidas para mejorar la conciliación y disminuir la discriminación salarial producida exclusivamente por condición de género. esta campaña está organizada por el área de mujer del ayuntamiento de collado villalba, y cofinanciada por la comunidad de madrid y el fondo social europeo. & 256 & very low & Low & Socio-Economic & NA & NA & 2017-06-11 & 2017 & 2 & ECO
Frame & v.low & National & <500 & 0.0141942 & -0.1980960 & -0.1536394 & -0.9382144 & 0.7927531 & 12.6 & 1.2981504 & 0.5794339 & Recipient & Domestic & Domestic & Domestic & Domestic|ECO & Positive\\
Spain & http://www.diariovasco.com/gipuzkoa/201604/22/plan-reactivacion-economica-gipuzkoa-20160422114141.html & 682 & El Diario Vasco & Private/Non-Public & Online and Offline & Regional/Local & very low = CP mentioned once & Social awareness/inclusion & Positive & Subnational & No myth & NA & NA & NA & NA & NA & NA & NA & NA & Spain & el plan de reactivación económica de gipuzkoa destinará ocho millones de euros en servicios sociales & 2016-04-22 & fondo social europeo & el plan de reactivación económica de gipuzkoa destinará 8 millones de euros, del total de 50 con los que está dotado el plan, a programas de políticas sociales. el diputado general, markel olano; la diputada del área, maite peña y el director de planificación e inversiones del departamento, carlos alfonso han detallado esta mañana las inversiones que se realizarán para los próximos cinco años. dentro de las medidas han destacado el proyecto aukerability, que permitirá la contratación de 554 personas con discapacidad en empresas del mercado laboral ordinario de aquí a 2020. el proyecto, que se desarrollará de la mano de gureak, recibirá 125.000 euros al año de la diputación y también contará con financiación del fondo social europeo. además, se facilitará el acceso al mercado laboral a 955 mujeres y 1.427 hombres a través de contratos de formación y prácticas entre otros. el diputado general ha resaltado la importancia del proyecto ya que "la inserción social de las personas pasa por lograr su inserción laboral. tener un empleo es la base", ha dicho. & 176 & very low & Low & Socio-Economic & NA & NA & 2016-04-22 & 2016 & 2 & ECO
Frame & v.low & Regional & <500 & 0.0141942 & -0.1980960 & -0.1536394 & -0.9382144 & 0.7927531 & 12.6 & 1.2981504 & 0.5794339 & Recipient & Domestic & Domestic & Domestic & Domestic|ECO & Positive\\
Spain & https://www.europapress.es/asturias/noticia-curso-gratuito-ilustracion-serigrafia-nuevas-tecnologias-arranca-lunes-20190224142335.html & 603 & europa press & Private/Non-Public & Online only & National & very low = CP mentioned once & Social awareness/inclusion & Positive & National & No myth & Jobs & Positive & National & No myth & NA & NA & NA & NA & Spain & el curso gratuito ilustración, serigrafía y nuevas tecnologías... & 2019-02-24 & fondo social europeo & oviedo, 24 feb. (europa press) - la factoría cultural acoge desde este lunes el curso gratuito denominado "ilustración, serigrafía y nuevas tecnologías", que se desarrollará en sus instalaciones hasta el 15 de mayo, de lunes a viernes en horario de 9.00 a 14.00 horas. el curso está dirigido a mujeres desempleadas de entre 18 y 35 años, atendiendo especialmente a aquellos perfiles que cuenten con mayores necesidades de apoyo para su inserción laboral, su autoconocimiento y para el refuerzo de sus capacidades personales y profesionales. el objetivo prioritario es facilitar el acceso laboral de las participantes a los sectores audiovisual, de la moda, de la publicidad o de cualquier otro sector relacionado con las artes gráficas. consta de una duración de 245 horas lectivas, que se dividen en cuatro bloques. el primero, de 165 horas, está dedicado a la formación técnica específica, abordando diferentes aspectos del dibujo, el color, la impresión en 3d, la ilustración en textiles y objetos, o el diseño y aplicación de vinilos, entre otros. el segundo bloque está dedicado a la motivación personal y para el empleo, y ocupará un total de 20 horas, las mismas que el tercero, centrado en la orientación a la búsqueda de empleo y el emprendimiento. por último, el cuarto bloque es el dedicado a la realización de prácticas laborales, con una duración de 40 horas, tutorías individualizadas y seguimiento durante el mes posterior a la finalización de las mismas. la factoría cultural desarrolla este curso en colaboración con la federación española de universidades populares a través del proyecto estatal sara, dirigido a la motivación y acompañamiento de mujeres jóvenes para su participación social y laboral. está financiado por el instituto de la mujer y para la igualdad de oportunidades, a través de la cofinanciación del fondo social europeo en el programa operativo de inclusión social y de la economía social (posies) 2014-2020. & 314 & very low & Low & Socio-Economic & Socio-Economic & NA & 2019-02-24 & 2019 & 3 & ECO
Frame & v.low & National & <500 & 0.0141942 & -0.1980960 & -0.1536394 & -0.9382144 & 0.7927531 & 12.6 & 1.2981504 & 0.5794339 & Recipient & Domestic & Domestic & Domestic & Domestic|ECO & Positive\\
Spain & http://www.farodevigo.es/comarcas/2016/01/19/ponteareas-opta-5-millones-fondos/1387827.html & 676 & Faro de Vigo & Private/Non-Public & Online and Offline & Regional/Local & very low = CP mentioned once & Infrastructure & Positive & National + Subnational & No myth & NA & NA & NA & NA & NA & NA & NA & NA & Spain & ponteareas opta a 5 millones de fondos europeos para ser una ciudad m�s saludable & 2016-01-19 & fondo europeo de desarrollo regional & la corporación municipal de ponteareas aprobó en pleno, con la única abstención del pp, presentar una candidatura para optar a las ayudas que el fondo europeo de desarrollo regional (feder) dispondrá para aplicar acciones relacionadas con el desarrollo urbano sostenible e integrado. la estrategia de desarrollo urbano sostenible e integrado(edusi) que presentará ponteareas necesita un presupuesto de 6,2 millones de euros la ayuda solicitada de fondos feder asciende a 5 millones de euros, el máximo al que se puede optar. mientras que el concello se compromete, si se le concede esa ayuda, a aportar los 1,2 millones de euros restantes en el período que va desde este año hasta 2020. la estrategia denominada "ponteareas, hábitat saludable" desarrollará, en caso de recibir esta subvención, cuatro grandes ejes: tecnologías de la información, economía baja en carbono, protección de en medio ambiente e inclusión social y reducción de la pobreza. se trata de una oportunidad para que "ponteareas logre una ciudad saludable, sostenible e inclusiva en el horizonte 2020. el objetivo es poner en marcha medidas relacionadas con el fomento de empleo, con la modernización de la administración, con el acceso a las nuevas tecnologías de la información, con la eficiencia energética, con la promoción del paisaje cultural y del patrimonio inmaterial y con la inclusión de los colectivos menos favorecidos" explica el alcalde, xosé represas. & 226 & very low & Low & Socio-Economic & NA & NA & 2016-01-19 & 2016 & 2 & ECO
Frame & v.low & Regional & <500 & 0.0141942 & -0.1980960 & -0.1536394 & -0.9382144 & 0.7927531 & 12.6 & 1.2981504 & 0.5794339 & Recipient & Domestic & Domestic & Domestic & Domestic|ECO & Positive\\
Spain & http://www.diariosur.es/costadelsol/201607/02/bloqueo-presupuesto-municipal-frenara-20160702005737-v.html & 604 & Sur & Private/Non-Public & Online and Offline & National & very low = CP mentioned once & Infrastructure & Factual & Subnational & No myth & NA & NA & NA & NA & NA & NA & NA & NA & Spain & el bloqueo del presupuesto municipal no frenará los proyectos urbanísticos & 2016-07-01 & fondo europeo de desarrollo regional & la prórroga de las cuentas de 2015 no afectará a las inversiones privadas, como el parque comercial, pero lastra los planes a corto plazo del psoe el bloqueo del presupuesto del ayuntamiento de torremolinos no afectará a las inversiones privadas ni a los proyectos urbanísticos previstos en el municipio costero, como el parque comercial y de ocio que la empresa intu prevé comenzar a construir en los próximos meses al norte del palacio de congresos. el alcalde, josé ortiz, optó por prorrogar las cuentas de 2015 después de que, en el pleno del pasado jueves, el partido popular y costa del sol sí puede tumbaran el presupuesto de este año, presentado con meses de retraso por el gobierno socialista. esta situación, insólita en las últimas dos décadas, tampoco alterará el proyecto de peatonalización de la plaza costa del sol, para el que ortiz solicitó fondos europeos. la prórroga, sin embargo, sí lastra los planes a corto plazo del ejecutivo local, que preveía aumentar las partidas destinadas a servicios sociales, cultura y educación, nuevas tecnologías, accesibilidad, juventud o empleo. el concejal de economía, pedro pérez, reconoció no estar "satisfecho" con el presupuesto municipal, que definió como "la mejor opción posible" en la actual situación financiera del ayuntamiento, que arrastra una deuda superior a los 210 millones de euros, según sus cálculos. en el caso de los grandes proyectos urbanísticos, la expedición de licencias queda supeditada a la aprobación definitiva del plan general de ordenación urbana (pgou), que la junta de andalucía podría anunciar en otoño. sería entonces cuando comenzarían a funcionar las máquinas de intu, que a comienzos de año ratificó el convenio para levantar el mayor parque comercial y de ocio del mediterráneo, una iniciativa que supondrá la mayor inversión privada en la historia del municipio, cercana a los 600 millones de euros. el bloqueo presupuestario tampoco repercutirá en las peticiones tramitadas para obtener subvenciones europeas, en concreto a la partida de quince millones de euros solicitada con cargo al fondo europeo de desarrollo regional (feder), con los que el gobierno local prevé poner en marcha su estrategia urbana sostenible. decisión "irresponsable" izquierda unida y ciudadanos se abstuvieron durante la votación. el portavoz de la coalición de izquierdas, david tejeiro, criticó ayer a costa del sol sí puede y al pp, cuya decisión tachó de "irresponsable". tejeiro sostiene que, aunque las cuentas del psoe "no eran todo lo rompedoras que deberían", su bloqueo implica que el ayuntamiento "deba seguir con los presupuestos aprobados por el pp, que eran electoralistas, propagandistas y falsos". desde el gobierno municipal también calificaron de "pinza" la posición de los populares y del partido instrumental de podemos, a quienes acusan de "frenar el progreso" de la localidad costasoleña. el rechazo al presupuesto municipal ha desatado una nueva tormenta política en torremolinos. el portavoz de costa del sol sí puede, josé piña, descartó la existencia de una pinza con el pp y justificó la negativa de su grupo municipal en redes sociales: "el psoe se equivoca con nosotros. si trabaja con honestidad, le ayudaremos. pero no vamos a votar a favor de unos presupuestos que ocultan la realidad económica del consistorio y que son partidistas como los del gobierno de pedro fernández montes". desde el pp, líder de la oposición, instan a ortiz a centrarse en las cuentas de 2017. la portavoz popular, margarita del cid, aseguró que el voto positivo de su grupo al presupuesto del próximo año estará condicionado a varias modificaciones sobre el documento bloqueado el pasado jueves, como la devolución de la paga extraordinaria de 2012 a los funcionarios, la inclusión de un plan de empleo local, el mantenimiento de bonificaciones en las tasas municipales o la creación de una comisión de transparencia. & 620 & very low & Low & Socio-Economic & NA & NA & 2016-07-01 & 2016 & 2 & ECO
Frame & v.low & National & 500-1000 & 0.0141942 & -0.1980960 & -0.1536394 & -0.9382144 & 0.7927531 & 12.6 & 1.2981504 & 0.5794339 & Recipient & Domestic & Domestic & Domestic & Domestic|ECO & Neutral\\
\addlinespace
Spain & https://www.europapress.es/extremadura/noticia-francisco-alcantara-cs-anuncia-oficina-municipal-captar-llegada-empresas-privadas-caceres-20190411135434.html & 671 & europa press & Private/Non-Public & Online only & National & very low = CP mentioned once & Mismanagement & Negative & Subnational & No myth & NA & NA & NA & NA & NA & NA & NA & NA & Spain & francisco alcántara (cs) anuncia una oficina municipal para captar... & 2019-04-11 & fondos estructurales & el candidato a la alcaldía propone un plan de barrios con cien medidas y modernizar la administración local cáceres, 11 abr. (europa press) - el candidato de ciudadanos (cs) a la alcaldía de cáceres, francisco alcántara, ha anunciado que en caso de ganar las elecciones municipales del próximo 26 de mayo, creará una oficina municipal para captar empresas privadas que se asienten en la ciudad, con el objetivo de crear empleo y garantizar que los jóvenes no tengan que marcharse fuera para desarrollar su proyecto profesional y vital. alcántara ha avanzado también que se elaborará un plan de barrios con cien medidas "urgentes" que se pondrán en marcha en los cien primeros días de gobierno, así como propuestas para modernizar la administración y eliminar trabas burocráticas tanto para los ciudadanos en general, como para empresas. en el acto de presentación oficial de su candidatura, que la formación naranja ha realizado en la calle costa rica en el popular barrio de llopis ivorra donde nació y se crió francisco alcántara, ha dicho que recibe este "reto" con "enorme ilusión", con "fuerte compromiso", "responsabilidad, humildad y con ambición" porque cree que "cáceres necesita un revulsivo" y un cambio "por necesidad" porque "en españa se vive muy bien pero en cáceres no tanto", ha dicho. "hago esto por mis hijos, para que tengan una oportunidad", ha resaltado, y "para ayudar al retorno a todos los que se fueron, en los años 60, y los que se están yendo ahora", ha indicado, al tiempo que ha incidido en que su trabajo consistirá en "sacar a cáceres de la situación de ostracismo en la que se encuentra" y para ello "hay que buscar alternativas", porque la población se estanca y los mayores de 64 años suponen más del 20\% de la población y "solo" el 14\% de la población tiene menos de 14 años. para alcántara la ciudad está falta de infraestructuras y pendiente de desarrollo, algo que reduce la competitividad empresarial, aparte de la presión fiscal porque, según ha dicho, los cacereños son los españoles que "más parte de su sueldo dedica al pago de impuestos". "las tasas de paro son insoportables y los salarios siguen sin alcanzar los niveles anteriores a la crisis", ha dicho en su presentación, en la que ha estado acompañado del coordinador regional de cs, cayetano polo, y parte del equipo municipal. el candidato también ha criticado que la "cantidad ingente" que ha llegado desde europa en fondos estructurales "no se ha empleado de forma correcta" porque la mayoría de los barrios tienen "grandes carencias" en servicios sociales y en equipamientos deportivos para jóvenes y mayores y algunos "siguen con tuberías que son un problema para la salud pública" o "con instalaciones sin acabar desde hace más de diez años", ha incidido. "el proyecto de ciudadanos no puede resignarse ni permanecer impasible ante eso", ha subrayado, al tiempo que ha asegurado que "el problema no son los técnicos" sino "unos políticos incompetentes que han hecho una gestión mediocre", ha espetado. así, ha pedido a la alcaldesa elena nevado que invierta en los barrios el superávit de 12 millones que han arrojado las cuentas de 2018 porque "la periferia está dejada". alcántara ha valorado la preparación de su equipo que "no tiene mochilas políticas" y "sabe lo que es trabajar fuera de la política", por lo que les ha conminado a "mejorar las cosas" y a "ofrecer resultados a la gente". "vamos a hacer política sencilla para gente sencilla y con problemas cotidianos", ha señalado. y como para cs el principal problema es el paro, una de sus primeras medidas será crear una oficina municipal para captar capital privado de empresas "grandes y medias" que inviertan en la ciudad, para lo que se hará una oferta de valor "seria", urbanística y de recursos humanos y financieros, de proveedores y de seguridad jurídica para que sea competitiva y "las grandes empresas vengan a cáceres". a la pregunta de si cs impulsaría la mina de litio si llegase al gobierno local, alcántara ha indicado que este asunto "es un ejemplo más de la fuerte incompetencia del gobierno actual municipal" porque en 2017 la alcaldesa dijo que esa mina "era una oportunidad para cáceres" pero un año después "sin tener más información", se posiciona en contra diciendo que va a generar menos empleo que una pizzería. alcántara ha criticado la "poca prudencia" que ha tenido el ejecutivo cacereño al no analizar la posible inversión que esto supondría para la ciudad. "lo que hay que hacer es recabar los informes y que los técnicos de las diferentes administraciones den toda la información necesaria porque no te puedes oponer a una inversión privada que puede generar muchos puestos de trabajo en la ciudad porque el litio es un mineral estratégico en la unión europea", ha resaltado. "nosotros decimos que se analice el proyecto desde un punto de vista medioambiental, económico y social pero que no se creen trabas y que encima se genere inseguridad jurídica porque eso impulsa otras inversiones", dicho, al tiempo que añade que "esto es un ejemplo más del capricho como se ha estado llevando cáceres, que no es un cortijo, sino una ciudad que tiene muchos problemas". en cuanto a los pactos postelectorales, alcántara ha indicado que cs "está preparado para gobernar" y espera que "otros" se acerquen a la formación naranja para formar gobierno. "vamos a dejar que los cacereños voten y después ya veremos como está el tema de pactos pero ahora lo importante es hablar de los programas", ha indicado. francisco alcántara nación en el barrio de llopis ivorra en 1967. es licenciado en ciencias actuariales y financieras, es empresario y profesor asociado de economía financiera y contabilidad de la universidad de extremadura (uex), aunque esta última actividad docente la ha abandonado al iniciar su vida política. también es doctor en ciencias económicas y empresariales y tiene un máster en gestión financiera. ha desempeñado su labor profesional en varias empresas hasta que en 2008 creó albroska, una sociedad que dirige desde cáceres y que cuenta con más de 70 delegaciones por toda españa y ha creado unos 200 puestos de trabajo. a alcántara le acompañarán de número dos el actual concejal antonio ibarra y también repetirá la edil maría del mar díaz. entre las incorporaciones a la candidatura de cs está la de ñete bohigas, que fue entrenador del primer equipo de baloncesto de la ciudad. & 1067 & very low & Low & Governance & NA & NA & 2019-04-11 & 2019 & 3 & POL
Frame & v.low & National & +1000 & 0.0141942 & -0.1980960 & -0.1536394 & -0.9382144 & 0.7927531 & 12.6 & 1.2981504 & 0.5794339 & Recipient & Domestic & Domestic & Domestic & Domestic|POL & Negative\\
Spain & http://www.farodevigo.es/portada-deza-tabeiros-montes/2016/11/19/cruz-roja-lalin-finaliza-curso/1572963.html & 663 & Faro de Vigo & Private/Non-Public & Online and Offline & Regional/Local & very low = CP mentioned once & Social awareness/inclusion & Positive & Subnational & No myth & NA & NA & NA & NA & NA & NA & NA & NA & Spain & cruz roja de lalín finaliza un curso orientado a la inclusión laboral de 20 personas & 2016-11-19 & fondo social europeo & un grupo de 20 personas, participantes en la estratexia de inclusión social 2014-2020, remató ayer en el concello de lalín un curso de primeros auxilios impartido por la cruz roja local y cofinanciado por la xunta y el fondo social europeo. el seminario, que se centró en conceptos elementales, estuvo organizado por el equipo de inclusión socio-laboral lalinense integrado en el consorcio galego de servizos de igualdade e benestar, que depende de la consellería de política social. el curso, que fue impartido en las instalaciones de la casa da xuventude, se desarrolló desde el pasado día 14 de noviembre y tuvo una duración definitiva de 20 horas lectivas. la finalidad de esta iniciativa de carácter público es que los asistentes al seminario de primeros auxilios adquieran los conocimientos básicos gracias a la formación que corrió a cargo de especialistas de la cruz roja de lalín. actuaciones como la que ayer en la cabecera comarcal dezana llegó a su finalización están incluidas en los itinerarios de inserción socio-laboral que los equipos de inclusión del consorcio galego de servizos de igualdade e benestar preparan para todos los participantes de la extratexia de inclusión social de galicia 2014-2020. & 199 & very low & Low & Socio-Economic & NA & NA & 2016-11-19 & 2016 & 2 & ECO
Frame & v.low & Regional & <500 & 0.0141942 & -0.1980960 & -0.1536394 & -0.9382144 & 0.7927531 & 12.6 & 1.2981504 & 0.5794339 & Recipient & Domestic & Domestic & Domestic & Domestic|ECO & Positive\\
Spain & http://www.laverdad.es/murcia/201601/23/bruselas-pide-explicaciones-ayudas-20160123003452-v.html & 609 & LA VERDAD & Private/Non-Public & Online and Offline & Regional/Local & high = CP is most important issue in story (can also cover other issues) & Fraud/Corruption & Negative & EU + National & No myth & NA & NA & NA & NA & NA & NA & NA & NA & Spain & bruselas pide explicaciones por las ayudas que otorgó a las desaladoras de acuamed & 2016-01-23 & fondos estructurales & la comisión europea (ce) avanzó ayer que ha pedido información a españa sobre el caso de la empresa pública acuamed, investigada por la audiencia nacional por presuntos fraudes en contratos, que podrían suponer varios millones de euros. en este sentido, la ce también quiso precisar que informará a la oficina europea de lucha contra el fraude (olaf) para que considere investigar el caso. "los servicios de la comisión informarán a la olaf sobre este asunto para que considere la posibilidad de abrir una investigación", señalaron fuentes comunitarias. la comisión fue alertada el pasado día 18 de la investigación abierta por supuesto fraude. acuamed es un "beneficiario tradicional" de fondos europeos, recordaron las mismas fuentes. especialmente del fondo europeo de desarrollo regional (feder) y de los fondos de cohesión para realizar obras públicas en el sector del agua en la cuenca mediterránea. torrevieja perdió los fondos entre 2007 y 2013, los fondos europeos cofinanciaron proyectos por valor de unos 660 millones de euros, entre los que se encuentran las desaladoras de torrevieja, valdelentisco y águilas. la ue retiró 55 millones de ayudas a la planta de torrevieja por los retrasos en la puesta en marcha de las instalaciones. el ministerio de hacienda intentó recuperar ese dinero a través de otros programas operativos. "tras las alegaciones aparecidas en la prensa española, la comisión europea pidió inmediatamente a la autoridad de gestión -al ministerio de hacienda y administraciones públicas- informar sobre la situación", indicaron las fuentes. a su vez, la ce "recordó su responsabilidad a la hora de tomar todas las medidas apropiadas para garantizar que el dinero de la unión europea no se utilice mal". bruselas aludió a que, bajo el sistema compartido de gestión que se aplica actualmente a los fondos estructurales, los estados miembros son "responsables de implementar y gestionar los programas que han sido acordados previamente con la comisión". eso quiere decir que "les corresponde seleccionar los proyectos para que sean cofinanciados de manera efectiva y verificar la regularidad del gasto que será enviado a la comisión para que lo reembolse, en total cumplimiento de la legislación nacional y de la ue", precisaron. águilas y las presas las citadas fuentes declararon que el ministerio de hacienda ya ha indicado al organismo europeo su decisión de suspender cualquier declaración de gasto para acuamed. además, "ha mostrado compromiso con tomar todas las medidas necesarias para evitar cualquier mal uso del dinero de la ue, una vez se hayan clarificado los proyectos concernidos y las supuestas prácticas". las impulsiones de la desaladora de águilas y los proyectos de las presas de lébor y la moreras, así como el recrecimimiento de camarillas, no están cofinanciados por la ue, indicaron fuentes oficiales, aunque podría afectarles la suspensión de los gastos a acuamed. & 458 & high & High & Governance & NA & NA & 2016-01-23 & 2016 & 2 & POL
Frame & high-very high & Regional & <500 & 0.0141942 & -0.1980960 & -0.1536394 & -0.9382144 & 0.7927531 & 12.6 & 1.2981504 & 0.5794339 & Recipient & Domestic & European & Mixed & Domestic|POL & Negative\\
Spain & https://elpais.com/internacional/2018/10/09/actualidad/1539081796\_424067.html & 644 & EL PAÍS & Private/Non-Public & Online and Offline & National & medium = CP is important part of story & Fraud/Corruption & Negative & EU + Other country & No myth & NA & NA & NA & NA & NA & NA & NA & NA & Spain & un detenido por el asesinato de la periodista de investigación búlgara & 2018-10-09 & fondos estructurales & viktoria marinova, de 30 años, fue hallada muerta con signos de haber sido violada, golpeada y estrangulada tras abordar casos de corrupción en su país la policía ha detenido a una persona sospechosa de estar implicada en el asesinato de la periodista de investigación búlgara viktoria marinova, según confirmó la oficina de prensa del ministerio interior de bulgaria. "un sospechoso ha sido llevado hoy a la comisaría de la policía en la ciudad de ruse. el ministro interior, mladen marinov, actualmente se encuentra en ruse y más tarde hará declaraciones a la prensa junto a la fiscalía" , declaró una portavoz. más información asesinado a tiros un periodista eslovaco que investigaba casos de corrupción asesinado a tiros el noveno periodista en méxico en lo que va del año el crimen que saca a la luz lo peor de malta en esa ciudad fue localizado el pasado sábado el cadáver de la periodista, directora administrativa y presentadora en el canal de televisión privado regional tvn. marinova fue hallada muerta en un parque de la ciudad búlgara de ruse, con signos de haber sido violada, golpeada y estrangulada, solo una semana después de que su programa detector de mentiras airease ciertos casos de presunta corrupción relacionados con el desembolso de fondos estructurales de la unión europea. el crimen de bulgaria se suma al asesinato de la periodista daphne caruana galiza en malta y del periodista jan kuciak en eslovaquia, ambos volcados en investigaciones de corruptelas en sus respectivos países. aunque el móvil del crimen búlgaro aún no se ha aclarado, la secuencia de tres asesinatos de periodistas en países de la ue en apenas un año ha disparado las alarmas en bruselas. & 279 & medium & Medium & Governance & NA & NA & 2018-10-09 & 2018 & 3 & POL
Frame & low-medium & National & <500 & 0.0141942 & -0.1980960 & -0.1536394 & -0.9382144 & 0.7927531 & 12.6 & 1.2981504 & 0.5794339 & Recipient & European & European & European & European|POL & Negative\\
Spain & http://www.heraldo.es/noticias/aragon/huesca-provincia/2017/01/08/continuaran-2017-los-trabajos-recuperacion-del-claustro-roda-isabena-1152273-1101026.html & 628 & HERALDO & Private/Non-Public & Online and Offline & Regional/Local & low = CP mentioned more times but NOT important part of story (mainly about others issues) & Cultural heritage & Positive & Subnational & No myth & NA & NA & NA & NA & NA & NA & NA & NA & Spain & continuarán en 2017 los trabajos de recuperación del claustro de roda de isábena & 2017-01-08 & fondo europeo de desarrollo regional & se trata de una de las partes más emblemáticas de este edificio, declarado monumento histórico-artístico en 1924 --hoy bic en la categoría de monumento-- y situado en una localidad ribagorzana que también goza de la declaración de conjunto de interés cultural en la categoría de conjunto histórico desde 1988. el estado de conservación del claustro es fruto del lógico deterioro sufrido con el paso del tiempo por este elemento expuesto a los agentes atmosféricos, pero también de las sucesivas intervenciones experimentadas por el mismo desde el siglo xvii y a partir de 1942 y no siempre acertadas. para detener este proceso de deterioro, el departamento de educación, cultura y deporte ha acometido en 2016 las primeras obras de restauración del lado norte del claustro, tanto de la arquería como de los paramentos interiores, y ahora pretende llevar a cabo la restauración de las arquerías de los lados este y oeste en el ejercicio 2017 y continuar así con la recuperación de este excepcional espacio. los trabajos se proponen para ser cofinanciados con el fondo europeo de desarrollo regional (feder), dentro del programa operativo 2014-2020. el presupuesto de licitación de los trabajos asciende a 148.404,96 euros (iva incluido). la excatedral de san vicente y san valero de roda de isábena es la catedral más antigua de aragón. conserva el esplendor del románico y del periodo en el que roda de isábena se convirtió en el centro religioso y político del condado de la ribagorza. se fundó en el siglo x, después de que ramón ii consiguiera que roda de isábena fuera nombrada sede episcopal. dada su importancia se declaró en el año 1924 monumento nacional. la catedral se levanta sobre los restos de un castillo y fue consagrada el 1 de diciembre del año 956 con la advocación de san vicente. las obras de construcción de la catedral se prolongaron durante dos siglos, debido a que en 1006 los árabes destruyen parte de su estructura. gracias al obispo san ramón se termina su reconstrucción en el siglo xii en estilo románico lombardo. el siglo xviii supone el declive de esta institución religiosa, perdiendo el rango de catedral por el de colegiata para no ser más que iglesia parroquial en la actualidad. se trata de un conjunto monumental, situado en la parte alta de la población, formado por la iglesia, tres criptas, una torre y un claustro trapezoidal, al que se añadieron posteriormente la hospedería y un pórtico. la fábrica original es de sillar mediano y presenta en el exterior de los ábsides decoración de tradición lombarda como lesenas, arquillos ciegos y bandas de esquinillas. la iglesia presenta planta basilical de tres naves que se dividen en tres tramos. la cabecera de tres ábsides está precedida por un presbiterio bajo el que se abren tres criptas. de igual forma se divide en tres naves la cripta alojada bajo el ábside central. las cubiertas y soportes son parte de la reconstrucción del templo en el siglo xii. así, las naves laterales se cubren con bóvedas de arista, de cañón apuntado en el presbiterio y en la central se sustituyó posteriormente una techumbre de madera por bóveda de cuarto de esfera. las naves de la iglesia se separan por pilares cruciformes y las de la cripta por columnas que sustentan pequeñas bóvedas de arista. en su lado norte tiene adosado un claustro románico abierto mediante arquería de medio punto sobre columnas con capiteles historiados e inscripciones funerarias. a través del claustro se accedía a las dependencias de la enfermería, la sala capitular, el refectorio, el dormitorio y a la capilla de san agustín. & 603 & low & Low & Socio-Economic & NA & NA & 2017-01-08 & 2017 & 2 & ECO
Frame & low-medium & Regional & 500-1000 & 0.0141942 & -0.1980960 & -0.1536394 & -0.9382144 & 0.7927531 & 12.6 & 1.2981504 & 0.5794339 & Recipient & Domestic & Domestic & Domestic & Domestic|ECO & Positive\\
\addlinespace
Spain & https://www.elmundo.es/internacional/2018/12/18/5c17ebb121efa0387a8b4623.html & 623 & EL MUNDO & Private/Non-Public & Online only & National & very low = CP mentioned once & Political leverage & Factual & EU + Other country & No myth & NA & NA & NA & NA & NA & NA & NA & NA & Spain & viktor orban, en control de los tiempos & 2018-12-18 & fondos estructurales & transmite ideas y extrae conclusiones basadas en la interpretación de hechos y datos por parte del autor. con merkel en retirada, macron ocupado en problemas internos y reino unido de salida, el halcón húngaro puede estar tranquilo viktor orban, el hombre más peligroso de la unión europea el primer ministro de hungría, viktor orban: "salvini es mi héroe" de viktor orban se pueden decir muchas cosas, pero que sea incapaz de interpretar los tiempos y los vientos no es una de ellas. el líder húngaro, antaño paladín liberal y defensor de la democracia y la ue, es hoy uno de los mayores desafíos y peligros para ambas. su deriva comenzó hace años por motivos de supervivencia política, con un giro hacia la derecha y el nacionalismo en busca de los votos fugados que le hicieron perder el poder. su posición, hoy, va sin embargo mucho más allá. su defensa del "iliberalismo", su cruzada contra george soros, el islam, la inmigración o el colectivo lgbt, contra bruselas y la prensa, su retirada de símbolos históricos y su reemplazo por los contrarios o la presión sobre el poder judicial son algo diferente, circunscrito en una ola y con un objetivo que va más allá de unas elecciones. lo que ocurre desde hace años es parte de un fenómeno más amplio, internacional, pero del que orban es la cara más visible y el jugador más consumado. hungría está siguiendo los pasos de polonia. o mejor dicho, las instituciones europeas están tratando de seguir con budapest el mismo sendero legal y punitivo que con varsovia, mediante el famoso (e imposible) artículo 7 del tratado de la ue, que en última instancia podría suponer la pérdida de voto y veto de un país en el consejo europeo. pero las cosas poco tienen que ver. aplicarlo hasta el final es un escenario tan polémico como impensable, para el que hace falta una unanimidad inviable en una unión sin los mecanismos ni recursos suficientes para actuar con la contundencia requerida en algunos casos. hay al menos tres grandes diferencias entre hungría o polonia. la primera, que los líderes y diplomáticos húngaros son muchísimo más finos. conocen mejor las instituciones, sus límites, sus debilidades, y las aprovechan. saben cómo retrasar las discusiones en las reuniones de embajadores, cómo posponer preguntas, cómo vincular dossieres. en definitiva, cómo usar sus cartas. el ejemplo menos conocido sea quizás cómo orban bloquea desde hace meses las iniciativas de la otan (y hasta cierto punto la ue) respecto a ucrania. no deja que el gobierno de kiev participe en reuniones o que se convoquen otras de la alianza atlántica como respuesta a una ley ucraniana de 2017 que, según fidesz, discrimina a la minoría húngara que reside en el país, al limitar el uso de las lenguas minoritarias en el sistema educativo. ni siquiera con lo sucedido en el mar de azov. el segundo punto es que fidesz, orban, está en el partido popular europeo. ha habido una polémica inmensa, y hasta el grupo votó a favor de arrancar los trámites del artículo 7 en la eurocámara. pero nadie aboga por su expulsión de los 'populares'. las próximas elecciones europeas de mayo van a ser muy apretadas y los escaños húngaros son indispensables para la actual dirección del ppe, muy temerosa de que acaben engrosando las filas de grupos escépticos y suponiendo fuera una amenaza mucho mayor de lo que es ahora desde dentro, donde esperan poder reconducir si deriva autoritaria. hay mucho ruido y pocas nueces entre los conservadores, que se llenan la boca de grandes palabras sobre democracia, valores y libertades mientras las mercadean por unos pocos diputados o apoyo en cuestiones propias (como la crisis catalana). el tercer elemento es que orban está interpretando mucho mejor los ritmos. no arremete contra macron como hacen los polacos. ataca cuando ve oportunidades, sin dudar, pero sin malgastar oportunidades. contra la open university, viendo que puede cortar de raíz un problema. contra otros parlamentarios, minorías o manifestantes. los difama, los menosprecia, llena de falsedades la arena política. y lo hace cuando la ue está más debilitada y sobrepasada. pendiente del brexit, con merkel ya empezando la retirada, con el elíseo asediado por problemas internos, con la comisión a destajo por la crisis económica italiana y la necesidad de cerrar decenas de asuntos a muy pocos meses de los comicios europeos. incluso aprovechando el hecho de que la figura clave desde bruselas en los dossieres que implican el estado de derecho, el holandés fran timmermans, está de campaña. ¿qué posibilidades hay de que pudiera llevar a presidente de la comisión europea con la oposición firme de varios países? orban no tiene prisa porque cree que el tiempo juega a su favor. en 2019 las instituciones comunitarias cambian de ciclo y de líderes. hacen falta nombres para la comisión, el parlamento, el consejo o la otan (además del bce). hay un baile de sillas, de nombres y de alianzas en la que cada voto y apoyo o bloqueo cuenta. además, el clima en el continente parece más bien alinearse con los intereses. ya no es sólo visegrado, sino austria, eslovenia o incluso rumanía, que el 1 de enero asume la presidencia temporal de la ue. hay quejas de ong, del parlamento, de las instituciones y la sociedad civil. pero no de los estados, porque todavía no hay receta buena para hacer frente a este discurso, al populismo. y porque, temen, cualquier intervención no haría sino empeorar la situación. orban, como putin, siempre busca los límites de la ue, empujando y empujando para ver hasta dónde puede llegar sin (excesivas) consecuencias. su gobierno sólo teme dos conceptos: el del marco financiero plurianual (es decir, los presupuestos de la ue) y la idea de que una ue a dos velocidades que los acabe aislando. mientras los fondos estructurales sigan llegando y no estén condicionados a solidaridad en temas como el de los refugiados, respira. y mientras las cosas se tengan que seguir haciendo en casi todo de forma unánime y no mediante cooperaciones reforzadas, poco cambiará. los votantes por el momento le apoyan. escarnio y estigma son vocablos que no maneja. y la presión entre sus socios es, todavía, muy justita. hay preocupación pero no urgencia. si el cambio llega, no parece que vaya a ser gracias a sus vecinos. más bien, se diría, que a pesar de su silencio. conforme a los criterios de saber más & 1070 & very low & Low & Power & NA & NA & 2018-12-18 & 2018 & 3 & POL
Frame & v.low & National & +1000 & 0.0141942 & -0.1980960 & -0.1536394 & -0.9382144 & 0.7927531 & 12.6 & 1.2981504 & 0.5794339 & Recipient & European & European & European & European|POL & Neutral\\
Spain & http://abcblogs.abc.es/luis-ayllon/public/post/espana-y-portugal-un-futuro-iberico-en-europa-16949.asp/ & 664 & abcblogs.abc.es & Private/Non-Public & Online and Offline & National & very low = CP mentioned once & Economic development & Positive & National + Other country & No myth & NA & NA & NA & NA & NA & NA & NA & NA & Spain & españa y portugal: un futuro ibérico en europa & 2018-04-18 & fondos estructurales & las selecciones de portugal y españa se enfrentarán en junio dentro de uno de los grupos del mundial de fútbol de rusia. son los dos grandes favoritos para pasar a la siguiente fase, con un fútbol muy por encima de los otros dos integrantes: marruecos e irán. pero, en cualquier caso, si uno llegará a ganar el mundial el otro no lo haría. esta exclusión que impone la competición futbolística no debe trasladarse al terreno de lo político o lo económico. la reciente visita de estado a españa del presidente portugués, marcelo rebelo de sousa, ha dejado un mensaje claro: los dos países pueden lograr mucho más trabajando juntos en europa y en otras partes del mundo que haciéndolo por separado. durante mucho tiempo, españa y portugal vivieron de espaldas o mirándose de reojo y con desconfianza. el hecho de que la adhesión a la unión europea se produjera a la vez fue importante, pero pasaron unos cuantos años hasta que los dos países decidieron intentar colaborar, entre otras cosas, a la hora de lograr los fondos estructurales y de cohesión comunitarios que tan útiles han sido para el desarrollo de la península ibérica. la crisis económica azotó fuertemente a españa y portugal y ambos, por caminos diferentes y con gobiernos de distintos signo, pero siempre introduciendo reformas estructurales, han conseguido unos resultados positivos, elogiados en europa. portugal, además, parece estar de moda. y no sólo porque su selección de fútbol con cristiano ronaldo al frente, se proclamara campeona de europa o porque salvador sobral, se alzara con el triunfo en eurovisión. además, un portugués, antonio guterres, es hoy el secretario general de naciones unidas, y otro, mario centeno, preside el eurogrupo, después de que durante diez años, josé manuel durao barroso, estuviera al frente de la comisión europea. se puede argumentar que el tamaño y el peso de portugal le hace ideal para convertir a sus candidatos en figuras de consenso, sin levantar recelos en los grandes, y que, años atrás, españa vivió también esa situación y logró puestos importantes en organismos internacionales, como el de javier solana, en la secretaria general de la otan o varios presidentes del parlamento europeo, pero lo cierto es que los portugueses se encuentran en uno de sus momentos de mayor visibilidad internacional. en cualquier caso, como dijo rebelo durante su estancia en españa, madrid y lisboa tienen que trabajar juntos, por encima de incomprensiones, suspicacias o diferencias, para lograr una europa que pueda hacer frente a los populismos, para contribuir con ideas al futuro de la construcción europea. los dos países que ya pusieron en marcha un mercado ibérico de la electricidad, quieren hacer lo mismo con el gas, para convertir la península en punto clave de suministro a europa, hoy dependiente en exceso del gas ruso. el asunto puede quedar cerrado en la cumbre bilateral que los dos gobierno celebrarán en el segundo semestre del año. españa cuenta con capacidad para exportar a europa el gas procedente de argelia y de otra quincena de países, pero no tiene suficientes infraestructuras de interconexión. la cuestión de las interconexiones, tanto energéticas como de infraestructuras, es, sin duda, una de las más importante para españa y portugal y estará también sobre la mesa en esa cumbre y en la que han de mantener en lisboa este mismo año los dos países, junto con francia y la comisión europea. la recuperación del proyecto de ave madrid-lisboa, paralizado en 2012 por portugal como consecuencia de la crisis está en la mente de los gobernantes españoles y portugueses y también en la comisión europea. españa y portugal están pues destinados a entenderse mucho más de lo que lo han hecho hasta ahora y como al parecer lo están haciendo a nivel transfronterizo las comunidades y municipios de galicia, castilla y león, extremadura y andalucía con sus contrapartes portuguesas. la visita de rebelo de sousa debe marcar un punto de inflexión en las relaciones bilaterales y con la unión europea. & 663 & very low & Low & Socio-Economic & NA & NA & 2018-04-18 & 2018 & 3 & ECO
Frame & v.low & National & 500-1000 & 0.0141942 & -0.1980960 & -0.1536394 & -0.9382144 & 0.7927531 & 12.6 & 1.2981504 & 0.5794339 & Recipient & Domestic & European & Mixed & Domestic|ECO & Positive\\
Spain & http://noticias.lainformacion.com/economia/ADM-Sevilla-empresas-inscritas-celebracion\_0\_906810005.html & 610 & La informacion & Private/Non-Public & Online only & Regional/Local & very low = CP mentioned once & Economic development & Positive & National + Other country & No myth & NA & NA & NA & NA & NA & NA & NA & NA & Spain & adm sevilla cuenta ya con 408 empresas inscritas de 24 países, un mes antes de su celebración & 2016-04-11 & fondo europeo de desarrollo regional & de esta forma, en su tercera edición, adm sevilla se consolida como el evento de negocios aeroespacial "más importante de españa", y, por tanto, como una cita clave para un sector estratégico para andalucía, que ha multiplicado por cuatro sus exportaciones en la última década, alcanzando los 1.518 millones de euros en 2015, un 9,2 por ciento más que en 2014. de esta forma, adm sevilla continúa con su carácter bienal, consolidando a andalucía como un punto clave en el circuito mundial de negocios del sector. adm sevilla está organizado por la consejería de economía y conocimiento, a través de la agencia andaluza de promoción exterior (extenda), quien ha remitido la nota de prensa, y por la empresa bci aerospace, especialista en reuniones internacionales del sector. en este evento de negocios de exclusivo carácter profesional se darán cita los principales fabricantes mundiales, contratistas y empresas de un sector industrial de máxima tecnología, que vienen a hacer negocio con las empresas del cluster andaluz. un sector que en andalucía facturó 2.244 millones de euros en 2014 (últimos datos disponibles) y que proporciona 12.688 empleos directos, a través de 114 empresas. esta actividad significa el 1,6 por ciento del pib andaluz, que ha triplicado su facturación en la última década y multiplicado por cuatro sus exportaciones, aportando más de 600 millones de euros en positivo a la balanza comercial de andalucía, así como una alta capacidad de diversificación de productos y destinos. adm sevilla 2016 cuenta con el patrocinio especial de airbus defence \& space, el mayor fabricante europeo del sector; de alestis, aernnova y aciturri, los tres únicos fabricantes de aeroestructuras de primer nivel de españa (tier 1), y de la empresa andaluza sofitec. igualmente, cuenta con el apoyo del sector, tanto en el ámbito nacional, con tedae (asociación española de empresas tecnológicas, de defensa, aeronáutica y espacio), como regional, con hélice, cluster aeroespacial andaluz. adm sevilla 2016 significa "un paso más en el desarrollo de la estrategia de apoyo a la internacionalización de un sector que la junta de andalucía viene impulsando a través de extenda de manera individualizada desde 2005". tras el éxito alcanzado en las dos primeras ediciones (2012 y 2014), adm volverá a proyectar al aeroespacial andaluz como una industria de referencia en el ámbito mundial y a andalucía en su conjunto como un enclave estratégico para la inversión de proyectos de industria compleja. exportaciones aeronáuticas andalucía cerró 2015 con 1.518 millones de euros en exportaciones aeronáuticas, lo que significa un crecimiento del 9,2 por ciento sobre 2014. el último dato publicado, de enero de 2016, sitúa el crecimiento sobre el mismo mes de 2015 en un 164 por ciento, llegando a 192 millones de euros en enero. de esta forma, andalucía ha incrementado en 2015 su significación como segunda comunidad autónoma exportadora en un sector de industria compleja, con más de uno de cada tres euros que exporta toda españa (36\% en 2015), sólo por detrás de la de madrid (2.193 millones), que no obstante, disminuyó su factura exportadora el último año un 1,6 por ciento. la tercera, ya a gran distancia, es aragón, con el 3,4 por ciento del total (144 millones), que adelanta a país vasco, que significa el 3,2 por ciento (137 millones). estas cifras, son reflejo de un sector claramente asentado e internacionalizado, que en el exterior está representado por 59 empresas exportadoras (más de la mitad del total de empresas del sector), y en el que ha crecido sólo en el último año un 4,3 por ciento el número de exportadoras regulares (4 años seguidos exportando), que son ya 24. origen y destino el eje sevilla-cádiz concentró la práctica totalidad (99,7\%) de las ventas aeronáuticas de andalucía en el exterior en 2015. sevilla lideró el crecimiento, con un once por ciento más que en 2014, y las ventas, superando por segunda vez en su historia los 1.000 millones de euros y marcando un nuevo récord, con 1.064 millones. por su parte, cádiz alcanzó los 449 millones, que significan el 30 por ciento del total y un crecimiento del seis por ciento interanual. con ventas menores pero por encima del millón de euros se sitúa córdoba, que creció un once por ciento, hasta 1,8 millones, y málaga, con 1,1 millones exportados. ya con ventas por debajo del millón están jaén, que llegó a los 877.000 euros; granada, con 120.000 euros, y huelva, con 55.000 euros. francia volvió a ser en 2014 el primer mercado para las ventas aeronáuticas andaluzas en el exterior, con 385 millones de euros (25,4\% del total y crecimiento del 19,3\%); seguido de malasia, con 253 millones de euros (16,6\% del total y multiplica por 107 la cifra del año anterior); méxico, con un incremento aún más espectacular, al multiplicar su cifra por 263, hasta los 129 millones (8,5\% del total); turquía, con 125 millones (8,2\% del total y descenso del 51\%); y alemania, con 121 millones (8\% del total y baja 11\%). más empresas inscritas en adm adm sevilla se plantea como objetivo alcanzar el millar de profesionales inscritos de 450 empresas y entidades del sector aeroespacial de 28 países de todo el mundo, para celebrar una agenda de más de 8.000 reuniones de negocio b2b concertadas. hasta la fecha, el ritmo de inscripción de empresas está siendo "muy superior" al de las dos ediciones anteriores. las 408 empresas ya inscritas provenientes de 24 países significan una participación ya asegurada un 47 por ciento superior a la que la edición anterior tenía un mes antes de su inicio. igualmente, se consolida el cariz internacional del evento, al ser ya 186 las empresas extranjeras inscritas (46\% del total), siendo francia, alemania, turquía y portugal los que más representación tienen hasta el momento, y contando también con compañás de sudáfrica; austria; bélgica; brasil; canadá; chequia; holanda; hungría; india; israel; italia; letonia; lituania; marruecos; polonia; reino unido; rusia; suecia y estados unidos. adm también refuerza en esta tercera edición su significación como mayor evento de negocios del sector en toda españa, con la participación de 222 compañías españolas, de las que 159 son de otras comunidades (72\%) y 63 son del cluster andaluz (28\%). concretamente, son las comunidades de madrid, con 65 empresas; la de cataluña, con 35, y la del país vasco, con 30, las que más empresas foráneas aportan. por categorías, se han inscrito ya en adm, además de los patrocinadores industriales mencionados, los otros tres grandes fabricantes mundiales, que son boeing, bombardier, y embraer, y hasta 41 grandes contratistas y tiers 1 (fabricantes de primer nivel). resultados anteriores ediciones la última edición de adm sevilla, celebrada en 2014, congregó a 943 profesionales de 420 empresas procedentes de 28 países, que mantuvieron 7.930 reuniones de negocio. estos resultados le valieron el calificativo del "mayor evento de negocios del sector celebrado nunca en españa". para alcanzar los objetivos de la tercera edición, extenda y bci aerospace están desarrollando en los últimos meses una intensa labor de promoción del evento en más de 20 de las principales ferias y eventos internacionales del sector. fibes, aerópolis y tecnobahía serán, de nuevo, los escenarios de este evento, que está financiado en un 80 por ciento con fondos procedentes de la unión europea, con cargo al programa operativo del fondo europeo de desarrollo regional (feder) de andalucía 2007-2013. & 1242 & very low & Low & Socio-Economic & NA & NA & 2016-04-11 & 2016 & 2 & ECO
Frame & v.low & Regional & +1000 & 0.0141942 & -0.1980960 & -0.1536394 & -0.9382144 & 0.7927531 & 12.6 & 1.2981504 & 0.5794339 & Recipient & Domestic & European & Mixed & Domestic|ECO & Positive\\
Spain & http://www.eleconomista.es/economia/noticias/7277633/01/16/Economia-Laboral-La-fundacion-del-escritor-y-dibujante-Peridis-ayudara-a-buscar-empleo-a-15000-parados-hasta-2019.html & 692 & El Economista & Private/Non-Public & Online and Offline & National & very low = CP mentioned once & Jobs & Positive & National & No myth & NA & NA & NA & NA & NA & NA & NA & NA & Spain & economía/laboral.- la fundación del escritor y dibujante 'peridis' ayudará a buscar empleo a 15.000 parados hasta 2019 & 2016-01-14 & fondo social europeo & economía/laboral.- el 13,5\% de los beneficiarios de la ayuda para parados con cargas y sin ingresos logran un empleo (22/10) la fundación santa maría la real, presidida por el escritor y dibujante josé maría pérez 'peridis', promoverá 548 lanzaderas de empleo entre 2016 y 2019 para ayudar a cerca de 15.000 desempleados a reforzar sus habilidades laborales y conocer nuevas herramientas para encontrar trabajo. el programa 'lanzaderas' contará con una inversión económica de 17,6 millones de euros, para lo que la fundación contará con la colaboración y cofinanciación del fondo social europeo, que aportará 12,8 millones de euros. asimismo, también están implicadas administraciones públicas locales y regionales y entidades privadas. "el paro no es sólo problema de las administraciones públicas, sino de la sociedad en su conjunto, por lo que necesitamos unir esfuerzos entre todos y fomentar una cultura colaborativa para afrontar esta lacra social", según ha subrayado 'peridis', presidente de esta fundación y promotor del programa 'lanzaderas'. este programa nació en 2013 y desde su estreno se han puesto en marcha más de 140 lanzaderas en 72 ciudades de 16 comunidades autónomas, llegando a más de 3.000 personas desempleadas y logrando una media de inserción laboral de entre el 60\% y el 70\%, según la fundación. "el apoyo que recibimos ahora del fondo social europeo supone un gran espaldarazo para situar a las lanzaderas entre una de las medidas eficientes en la lucha actual contra el desempleo", ha destacado 'peridis'. la fundación tiene previsto impulsar 170 lanzaderas de empleo este año, 164 en 2017, 124 en 2018 y 90 en 2019. estarán repartidas por todo el país y destinadas a diferentes colectivos: jóvenes con o sin formación; mayores de 45 años; parados de larga duración, parados residentes en zonas rurales; mujeres con dificultades de acceso al mercado laboral; inmigrantes, y personas con discapacidad, entre otros. cada lanzadera está formada por un equipo de 25 personas desempleadas, con diferentes perfiles formativos y trayectorias laborales. se reunirán semanalmente durante un periodo de cinco meses y contarán con la orientación de un 'coach' o coordinador, que les ayudará a emprender una búsqueda de empleo en equipo, ayudándose unos a otros a reforzar sus habilidades profesionales. & 370 & very low & Low & Socio-Economic & NA & NA & 2016-01-14 & 2016 & 2 & ECO
Frame & v.low & National & <500 & 0.0141942 & -0.1980960 & -0.1536394 & -0.9382144 & 0.7927531 & 12.6 & 1.2981504 & 0.5794339 & Recipient & Domestic & Domestic & Domestic & Domestic|ECO & Positive\\
Spain & http://www.elmundo.es/madrid/2016/09/19/57dfd38622601d716d8b4623.html & 616 & EL MUNDO & Private/Non-Public & Online and Offline & National & very low = CP mentioned once & Social awareness/inclusion & Positive & National + Subnational & No myth & NA & NA & NA & NA & NA & NA & NA & NA & Spain & ya se puede aprender a ser disc-jockey o maquillador en centros públicos de la comunidad de madrid & 2016-09-19 & fondo social europeo & noelia marin 19/09/2016 14:07 la comunidad de madrid ha ampliado este curso su oferta de formación profesional con dos nuevas titulaciones: grado de técnico en vídeo, disc-jockey y sonido; y de técnico superior en caracterización y maquillaje profesional. ambos se podrán estudiar en centros públicos de formación profesional, según ha avanzado ángel garrido, consejero de presidencia, justicia y portavoz del gobierno de la comunidad de madrid, en la rueda de prensa ofrecida tras la reunión del consejo de gobierno de este lunes. los dos ciclos formativos están compuestos por 2.000 horas de clase divididas en dos cursos lectivos y contarán con "una nueva asignatura de lengua extranjera profesional para que los alumnos adquieran conocimientos de idiomas que puedan aplicar en su vida profesional", tal y como ha recalcado garrido. sólo este curso se han matriculado 56.000 estudiantes para cursar alguna de las 110 titulaciones que se ofrecen en centros públicos donde se imparten titulaciones de formación profesional en la región. se duplica la inversión para la integración laboral garrido también ha informado durante la misma rueda de prensa de que este año se va a destinar 1,2 millones de euros para fomentar la integración laboral de personas en riesgo de exclusión. el 50\% de los fondos provienen del fondo social europeo y duplican la inversión del año pasado. los destinatarios de las subvenciones son las empresas de inserción y asociaciones sin ánimo de lucro que se ocupan de estos colectivos, así como empresas ordinarias que decidan contratar a personas en riesgo de exclusión social. "se calcula que unas 200 personas podrán beneficiarse de estos servicios", según garrido. entre los colectivos que pueden hacer uso de las ayudas están los perceptores de renta mínima de inserción, jóvenes mayores de 18 años o menores de 30 procedentes de instituciones de protección de menores, drogodependientes y personas en centros penitenciarios -siempre que tengan un régimen que les permita tener un empleo-. & 325 & very low & Low & Socio-Economic & NA & NA & 2016-09-19 & 2016 & 2 & ECO
Frame & v.low & National & <500 & 0.0141942 & -0.1980960 & -0.1536394 & -0.9382144 & 0.7927531 & 12.6 & 1.2981504 & 0.5794339 & Recipient & Domestic & Domestic & Domestic & Domestic|ECO & Positive\\
\addlinespace
Spain & http://www.rtve.es/noticias/20161116/bruselas-anima-invertir-tengan-margen-senala-ocho-posibles-incumplidores-deficit-2017/1444302.shtml & 611 & RTVE.es & Public & Online only & National & very low = CP mentioned once & Political leverage & Positive & EU + Other country & No myth & NA & NA & NA & NA & NA & NA & NA & NA & Spain & bruselas anima a invertir a los que tengan margen y señala ocho posibles incumplidores de déficit en 2017 - rtve.es & 2016-11-17 & fondos estructurales & italia pide un margen de un 0,4\%, entre otras cosas, por los terremotos "los que tienen margen para invertir, deben hacerlo", ha vuelto a repetir pierre moscovici este miércoles en bruselas, ahora en busca de un estímulo de más de 50.000 millones para la zona euro, aproximadamente el 0,5\% del pib. una llamada del comisario económico francés, que principalmente viaja hasta alemania y ahora también a países bajos, que denota un cierto relajamiento en las exigencias de austeridad ante el débil crecimiento de la eurozona explicitado este martes. en este caso, la comisión anima también al resto de socios de la eurozona, en la medida de sus posibilidades, a una política expansiva, dado que ven una "ventana de oportunidad". así recomienda adoptar un "enfoque colectivo", una composición del ajuste fiscal que distribuya esfuerzos entre los países y los tipos de gasto e impuestos. aunque a los que van con más problemas les reclama que se centren en cumplir con sus objetivos de déficit, especialmente a aquellos que, como españa, aún deben reducir sus desvíos por debajo del 3\%. la ventana no es nesariamente positiva porque parte de la constatación de "la lentitud de la recuperación y de los riesgos en el entorno macroeconómico". así los servicios de la comisión consideran, en una comunicación, "deseable" una expansión fiscal de hasta el 0,5\% del pib a nivel de la zona del euro en su conjunto". esta cifra es el resultado de una evaluación de la actividad económica, la capacidad no utilizada, el desempleo y la inflación. se trata de un "objetivo pragmático y prudente", aseguran, que apoyaría la política monetaria y el sobrecalentamiento innecesario de la economía. "es la primera vez que recomendamos una contribución presupuestaria global positiva para la zona euro" ha dicho el comisario de asuntos económicos poniendo el acento en el "paso" que supone abandonar la política fiscal restrictiva adoptada tras la crisis para abrazar una orientación expansiva. una política fiscal activa puede reducir el desempleo la comisión europea justifica su posición argumentando que la recuperación de la eurozona sigue siendo "lenta" y que las perspectivas de crecimiento "inciertas" fuera de la ue sugieren que un menor peso de las exportaciones. " la continuidad de la expansión en la zona euro necesita entonces sustentarse en una creciente demanda interna", defiende el texto del ejecutivo comunitario. además, bruselas denuncia que las normas actuales están diseñadas "esencialmente" para evitar los niveles "excesivos" de déficit y deuda públicos, mientras que sólo permiten "recomendar" y no "obligar" políticas expansivas a los países que no tienen problemas con sus cuentas. "las reglas pueden prohibir altos déficit pero solo pueden ordenar la reducción de superávit presupuestarios, no imponerla", subraya la comisión. de esta forma, critica que se da la "paradoja" de que "aquellos que no tienen espacio fiscal quieren usarlo, pero los que lo tienen no quieren utilizarlo". remarca que una política fiscal "más activa" y "bien diseñada" puede contribuir a "reducir más rápido el desempleo en el corto plazo" y a "elevar el crecimiento potencial en el medio plazo". la comisión no castiga las desviaciones además de estos estímulos, la "relajación" se advierte en no haber "castigado" las desviaciones y en que no va a congelar los fondos estructurales y de inversión para españa y portugal en 2017, tras haber evaluado las medidas presentadas por ambos países para atajar sus déficit. "@eu\_commission concludes that action by \#spain \& \#portugal is sufficient to avoid a suspension of eu funding." -- pierre moscovici (@pierremoscovici) 16 de noviembre de 2016 "no vamos a proponer suspender estos fondos. sé que es lo que se esperaba y es, obviamente, una buena noticia para dos países en los que los fondos europeos juegan un papel importante para sostener inversión y queremos que europa ayude a salir de la crisis", ha expresado moscovici. ocho países con riesgo de incumplimiento la comisión europea (ce) advirtió, sin embargo, de que seis países de la eurozona, además de españa y portugal, presentan riesgo de incumplir con sus compromisos de corrección del déficit público en 2017, después de evaluar los presupuestos, dentro de su denominado paquete de otoño del semestre europeo. son bélgica, italia, chipre, lituania, eslovenia y finlandia. moscovici ha señalado que en italia y chipre "los desvíos son grandes" y explicó que en el caso del primer estado "una parte importante del mismo se debe a los costes ligados a la actividad sísmica en el país, que ha sido dramática este año, y a la gestión de los flujos migratorios" y aseguró que la ce "tendrá esto en cuenta". por eso, italia ha solicitado a la comisión mayor flexibilidad fiscal y negocia con ella para obtener un margen adicional del 0,4 \% del pib en sus objetivos de déficit estructural para 2017 ya que calcula que esto es lo que le costará afrontar los gastos derivados de los recientes terremotos en el centro del país y de la gestión migratoria. también finlandia y lituania han pedido mayor flexibilidad para el año próximo, algo que la ce estudiará de aquí a la próxima primavera. francia cumple en general pero se duda de su consolidacion por su parte, francia, que ya recibió dos años extra para llevar su déficit por debajo de la cota del 3 \% considerada excesiva, ha presentado un presupuesto que a ojos de bruselas cumple "en general" con los objetivos pactados ya que prevé que el año próximo el desvío se sitúe en el 2,9 \%. sin embargo, bruselas señala que el esfuerzo fiscal es "significativamente menor" al recomendado y advierte de que si no hay cambios de política la corrección no será "duradera" en 2018, algo que moscovici advirtió "es condición sine qua non" para que el país salga del procedimiento por déficit excesivo. también cumplirán en grandes líneas irlanda, letonia, malta y austria, si bien la comisión cree que podría haber "alguna desviación" del ritmo de corrección fiscal previsto aunque solo a medio plazo. en el capítulo positivo, bruselas ha dado el visto bueno sin matices a los presupuesto de alemania, estonia, luxemburgo, eslovaquia y holanda. & 1007 & very low & Low & Power & NA & NA & 2016-11-17 & 2016 & 2 & POL
Frame & v.low & National & +1000 & 0.0141942 & -0.1980960 & -0.1536394 & -0.9382144 & 0.7927531 & 12.6 & 1.2981504 & 0.5794339 & Recipient & European & European & European & European|POL & Positive\\
Spain & http://www.elmundo.es/comunidad-valenciana/2018/03/21/5ab28e68e2704ea5478b48c3.html & 674 & EL MUNDO & Private/Non-Public & Online and Offline & National & medium = CP is important part of story & Institutional bargaining over funding & Balanced & EU + National + Subnational & No myth & NA & NA & NA & NA & NA & NA & NA & NA & Spain & ximo puig firma la adhesión de la comunidad a la alianza por la cohesión & 2018-03-21 & política de cohesión & el presidente de la generalitat valenciana, ximo puig, firmó este miércoles la adhesión de su comunidad a la alianza por la cohesión del comité europeo de las regiones (cdr), una iniciativa que reivindica que la política de cohesión mantenga su papel prominente en el presupuesto europeo tras 2020. puig formalizó la entrada de la comunidad en esta alianza durante un encuentro con el presidente del cdr, el belga karl-heinz lambertz, en la sede de este organismo en bruselas, tras el que explicó a efe que la región avala un fortalecimiento de esta partida de las cuentas comunitarias. "estamos en un momento decisivo para la configuración del próximo presupuesto comunitario y de la propia unión, y en este sentido queremos fortalecer el pilar de la cohesión que corre en estos momentos peligro ante la disminución de ingresos del conjunto europeo ante el 'brexit'", la salida del reino unido de la unión europea (ue), subrayó puig. la comisión europea (ce) presentará su primera propuesta para el futuro presupuesto multianual comunitario entre 2021 y 2027 el próximo 2 de mayo, y se prevé que la pérdida de la aportación británica y la llegada de nuevas prioridades de gasto, como la defensa o la seguridad, reduzca la dotación de partidas como la política de cohesión, que actualmente supone un tercio del total. "es importante que el cdr lidere la posición de las regiones europeas para que aquello que se ha logrado, de ir cohesionando al conjunto de regiones europeas, no se pierda en este momento", afirmó el presidente valenciano. puig señaló que el aumento de la desigualdad tras la crisis debe llevar a un aumento y no a una disminución de estos fondos, y pidió a los estados miembros que adopten "una posición mucho más proactiva" de cara a la negociación del presupuesto. "hay que aprovechar la penúltima oportunidad de europa, y eso debe hacerse con una mirada hacia los ciudadanos que en muchos momentos se han visto distanciados de los instituciones europeas", recalcó, al tiempo que aseguró que "la manera de suturar europa es avanzando hacia la igualdad". puig instó al gobierno español a posicionarse "claramente" en este debate y que "haga todos los esfuerzos necesarios para que exista un cuadro presupuestario que atienda los valores que decimos defender". "evidentemente hay nuevas necesidades en la agenda (...) que son también muy importantes, pero esto no puede ir en detrimento de la cohesión", señaló. la alianza por la cohesión, que comenzó su andadura el pasado otoño, cuenta ya con el respaldo de más de 4.000 entidades locales y regionales, asociaciones y empresas además de personalidades públicas, según datos del cdr. en españa, todas las comunidades autónomas salvo cataluña se han suscrito a la alianza, igual que lo han hecho decenas de autoridades locales, diputaciones, ciudades y otras entidades. & 465 & medium & Medium & Power & NA & NA & 2018-03-21 & 2018 & 3 & POL
Frame & low-medium & National & <500 & 0.0141942 & -0.1980960 & -0.1536394 & -0.9382144 & 0.7927531 & 12.6 & 1.2981504 & 0.5794339 & Recipient & Domestic & European & Mixed & Domestic|POL & Neutral\\
Spain & http://economia.elpais.com/economia/2016/07/07/actualidad/1467891193\_416391.html & 601 & EL PAÍS & Private/Non-Public & Online and Offline & National & high = CP is most important issue in story (can also cover other issues) & Political leverage & Negative & EU + National & No myth & NA & NA & NA & NA & NA & NA & NA & NA & Spain & bruselas da el primer paso para multar a españa y reclama ajustes & 2016-07-07 & fondos estructurales y de inversión europeos & moscovici: "colaboraremos para alcanzar un entendimiento común de los compromisos que deben asumirse" bruselas ha constatado hoy la "falta de medidas eficaces" de españa para cumplir con el déficit, tal como ha adelantado el país, y acerca un poco más las sanciones. la comisión europea cree asimismo "improbable" que españa cumpla sus metas en 2016. el brazo ejecutivo de la unión achaca ese incumplimiento a la expansión fiscal del año pasado, tanto por el incremento de gasto en las comunidades como por la rebaja de impuestos del gobierno de rajoy. bruselas aprobará en breve una nueva senda, con un año más para cumplir el déficit y objetivos del 3,7\% del pib para este año y del 2,5\% del pib el próximo, lo que supone un ajuste extra de unos 8.000 millones. en plena negociación para formar gobierno, el nuevo ejecutivo deberá lidiar con esa herencia y acordar unos presupuestos, de nuevo, relativamente austeros. de lo contrario, vienen curvas por el flanco de bruselas: "colaboraremos con españa y portugal para alcanzar un entendimiento común de los compromisos políticos que deben asumirse", ha asegurado el comisario pierre moscovici. más información bruselas abre el procedimiento de sanciones contra españa bruselas cree que españa volverá a incumplir las metas de déficit italia usa el brexit en bruselas para tratar de poner orden en sus bancos bruselas concluye que españa no tomó medidas efectivas para atajar el déficit en plata: la comisión quiere más ajustes, a los que por otra parte ya se ha comprometido mariano rajoy. el presidente en funciones aseguró en una carta a jean-claude juncker, justo antes de las elecciones, que españa tomará medidas adicionales si es necesario, a pesar de las promesas de más rebajas de impuestos que hizo en la primera página de la biblia del liberalismo, el financial times. ese momento de los recortes -- siempre "si es necesario" -- ha llegado: de lo contrario, españa está un poco más cerca de una sanción de hasta 2.100 millones de euros que incluiría la suspensión parcial de los compromisos de los fondos estructurales y de inversión europeos. bruselas admite que españa "ha aplicado grandes medidas de ajuste estructural", pero aun así concluye que esas medidas no han sido suficientes. la comisión ha advertido repetidamente en los últimos meses de la posibilidad de que el gobierno español incumpliera sus compromisos. ahora subraya que la "relajación de la política fiscal" el año pasado tuvo un "enorme impacto" en el agujero fiscal, que cerró 2015 en el 5,1\% del pib, casi 10.000 millones por encima de lo acordado. "las medidas de consolidación (...) fueron insuficientes para compensar el impacto de las medidas expansivas implementadas en 2015, como las rebajas de impuestos y la devolución de la paga extra a los funcionarios", dice el informe. bruselas admite que la caída de la inflación, que terminó el año claramente en negativo, "hace más difícil" cumplir con los objetivos. pero se queja de que madrid no aplicara con más dureza la ley de estabilidad a las comunidades autónomas. la inflación fue el argumento al que se agarró bruselas en 2015 para evitar sanciones a francia. bruselas templa gaitas hoy en público y asegura que está "dispuesta a colaborar con las autoridades españolas para definir el mejor camino a seguir", según el vicepresidente valdis dombrovskis. la realidad es otra: la comisión, tal vez a regañadientes, ha acabado forzando a españa y portugal a subir un peldaño en el procedimiento de infracción, que de esta manera se convierte en algo casi automático. el ministro de economía, luis de guindos, descarta la multa, pero el dictamen habla abiertamente de esa posibilidad. bruselas inicia con españa y portugal ese procedimiento sancionador, pese a que pudo hacerlo con francia. no lo hizo porque "francia es francia", explicó juncker hace unos días. la pelota está ahora en el tejado del ecofin: los ministros de finanzas deben confirmar el próximo martes esa recomendación de la comisión respecto a la falta de medidas efectivas, y a partir de ahí se abrirá un plazo de 10 días para que españa presente alegaciones, y bruselas propondrá sanciones en 20 días; probablemente, el 27 de julio. "la comisión podría recomendar que el consejo rebaje el importe de la multa o la anule", dice el texto. "esto podría ocurrir en el caso de circunstancias económicas excepcionales o previa solicitud motivada del estado miembro", añade. las únicas circunstancias excepcionales son las consecuencias del brexit, pero ese procedimiento se abre por lo ocurrido en 2015, y la salida de reino unido ha sido posterior. aun así, las reglas son lo suficientemente flexibles -- o volubles -- como para que todo pueda ocurrir si hay voluntad política. en bruselas "voluntad política" suele traducirse, en el caso español, por algo parecido a "recortes": la comisión quiere más ajustes para asegurarse de que, esta vez sí, españa reducirá su agujero fiscal por debajo del sacrosanto 3\% del pib en 2017. pero eso difícilmente puede suceder con un gobierno en funciones y con los partidos españoles aún pendientes de los pactos. españa alega que ha hecho grandes esfuerzos y que la inflación -en negativo- mermó su recaudación. pero "las reglas son las reglas", decía hoy una fuente diplomática en bruselas. el ecofin decidirá el martes cuál es el próximo paso. y todo es negociable, pero las fuentes consultadas en bruselas dan indicios de que nada va a ser fácil a partir de ahora: "la época de la inocencia se acabó: que cada palo aguante su vela", asegura a este diario una alta fuente europea, en relación al debate político que ahora debe empezar en el consejo, con alemania y holanda mostrando dureza, con italia y francia en contra de las sanciones, con una españa que busca aliados sin terminar de encontrarlos por el momento, ni en la comisión (pese a que juncker y rajoy militan en el mismo partido) ni en berlín (pese a que la política exterior española ha estado encaminada en la última legislatura a estrechar lazos con la canciller angela merkel, sin resultados visibles por ahora). en bruselas todo serán hoy palabras amables, pero más allá de las declaraciones está la decisión tomada, acompañada por un texto que acerca un paso más a españa al vértigo de estrenar las multas, en un país que a pesar de todo ha recortado su déficit del 11\% al 5\% con dos gobiernos distintos. frente al tono conciliador de una parte del discurso de dombrovskis y moscovici, el dictamen de bruselas es seco y rotundo: "es poco probable que españa logre una corrección oportuna y duradera de su déficit excesivo en 2016 a más tardar. además, el esfuerzo presupuestario acumulado a lo largo del período comprendido entre 2013 y 2015 fue muy inferior al establecido en la recomendación del consejo de junio de 2013. esto lleva a la conclusión de que españa no ha adoptado medidas eficaces en respuesta a la recomendación del consejo en el marco del procedimiento de déficit excesivo". & 1159 & high & High & Power & NA & NA & 2016-07-07 & 2016 & 2 & POL
Frame & high-very high & National & +1000 & 0.0141942 & -0.1980960 & -0.1536394 & -0.9382144 & 0.7927531 & 12.6 & 1.2981504 & 0.5794339 & Recipient & Domestic & European & Mixed & Domestic|POL & Negative\\
Spain & http://www.diarioinformacion.com/elda/2017/11/24/firma-textil-villena-eliminara-2500/1961284.html & 656 & Informacion & Private/Non-Public & Online and Offline & Regional/Local & very low = CP mentioned once & Environment/green/low-carbon & Positive & National + Subnational & No myth & NA & NA & NA & NA & NA & NA & NA & NA & Spain & una firma textil de villena eliminará 2.500 toneladas de co2 con biomasa & 2017-11-24 & fondo europeo de desarrollo regional & la directora general del ivace, júlia company, acompañada por la directora general de prevención de incendios, delia álvarez, visitó ayer dos proyectos de biomasa respaldados por el instituto valenciano de competitividad empresarial (ivace), en el marco del programa de fomento de las energías renovables que lleva a cabo para impulsar el uso de las energías renovables en la comunidad valenciana. durante la visita, company reiteró "el compromiso del ivace con las empresas que apuestan por incorporar las energías renovables para cubrir sus necesidades energéticas en sus procesos productivos e instalaciones". company recordó que ivace energía subvenciona hasta el 45\% del coste de este tipo de proyectos; un porcentaje que aumenta en 10 puntos en el caso de medianas empresas y en 20 puntos para pequeñas empresas. el ivace ha destinado en los últimos cuatro años un millón de euros para apoyar económicamente un total de 82 proyectos de biomasa en la provincia de alicante. así pues, visitaron la empresa textil athenea, ubicada en villena, donde se ha instalado una caldera de biomasa para dar suministro a las necesidades térmicas de su proceso productivo (generación de vapor y agua caliente de proceso) que ha recibido una subvención de 100.00 euros del ivace. para la directora del ivace, "esta instalación es una apuesta clara de la empresa por la sostenibilidad energética y medioambiental, así como por la reducción de sus costes energéticos, lo que permite aumentar su competitividad al emplear un combustible renovable más barato y con una menor sensibilidad a la variación de precios que los combustibles fósiles convencionales". la caldera tiene una potencia de tres megavatios, lo que la convierte en una de las calderas biomasa de mayor potencia de la comunidad valenciana. esta potencia, unida a la elevada intensidad de funcionamiento del proceso de fabricación, hacen que llegue a ahorrarse un consumo equivalente a más de 1.000 toneladas equivalentes de petróleo al año y evitar la emisión a la atmósfera de más de 2.500 toneladas de co2 al año. la biomasa empleada es astilla de origen forestal y es suministrada por la empresa valenciana forestal. procede de la limpieza de montes cercanos ubicados principalmente en el interior-sur de la provincia de valencia y en el interior de la provincia de alicante. el sistema de recogida en campo es realizado por un maquinaria móvil, también subvencionada con 212.000 euros por el ivace y que actualmente se encuentra en los monte de navalón (pedanía de enguera), a donde se han desplazado company y álvarez para conocer su funcionamiento. se trata de una maquinaria que se puede desplazar por diferentes puntos de la comunidad, para llevar a cabo labores relacionadas con las actividades de limpieza puntual de montes o en el marco de proyectos de gestión forestal planificada, siempre bajo la premisa de producir biomasa para su uso energético. hay que destacar que las ayudas a sendos proyectos están cofinanciadas por el fondo europeo de desarrollo regional (feder), a través del programa operativo de la comunidad valenciana 2014-2020. & 502 & very low & Low & Socio-Economic & NA & NA & 2017-11-24 & 2017 & 2 & ECO
Frame & v.low & Regional & 500-1000 & 0.0141942 & -0.1980960 & -0.1536394 & -0.9382144 & 0.7927531 & 12.6 & 1.2981504 & 0.5794339 & Recipient & Domestic & Domestic & Domestic & Domestic|ECO & Positive\\
Spain & https://www.diariovasco.com/bidasoa/irun/eurorregion-crea-portal-20181211003108-ntvo.html & 646 & El Diario Vasco & Private/Non-Public & Online and Offline & Regional/Local & very low = CP mentioned once & Territorial cooperation & Positive & National + Other country & No myth & NA & NA & NA & NA & NA & NA & NA & NA & Spain & la eurorregión crea un portal web transfronterizo sobre empleo & 2018-12-11 & fondo europeo de desarrollo regional & la eurorregión nueva aquitania, euskadi navarra (naen) ha dado un nuevo paso adelante en su objetivo de crear una verdadera cuenca de empleo transfronteriza, con la puesta en marcha de un nuevo portal web, que tiene por objetivo compartir información útil para desempleados y trabajadores (asalariados o autónomos) de los tres territorios. www.empleo-info.eu es un portal trilingüe que facilita información sobre alojamiento, requisitos de residencia y de contratación, aperturas de negocios o cuentas bancarias, transporte, formación en idiomas, becas y recursos de movilidad para estudiantes y convalidación de títulos universitarios, entre otros temas. la nueva web forma parte del proyecto empleo, cofinanciado al 65\% por el fondo europeo de desarrollo regional (feder), a través del programa poctefa, que ha realizado diversas acciones con el objetivo de hacer más fluidas las relaciones laborales entre las tres regiones. tres apartados la información que se ofrece en el portal empleo-info.eu está estructurada en tres secciones, según las características de cada usuario (demandante de empleo, empleado transfronterizo o autónomo) y para cada uno de ellos se abre un apartado, en función de la región a la que pertenece. con respecto a los demandantes de empleo, la nueva web ofrece información sobre cómo reconocer la cualificación laboral en cada territorio y dirige al usuario a los servicios públicos de empleo de cada región, además de añadir enlaces a entidades e instituciones de interés. por lo que se refiere a los trabajadores transfronterizos, el nuevo portal informa sobre los derechos laborales y sociales, así como sobre las obligaciones tributarias en cada territorio, con la posibilidad de descargarse modelos de contratos de trabajo. los trabajadores autónomos, por último, pueden encontrar información sobre los requisitos para montar un negocio en cualquiera de las tres regiones, sobre los desplazamientos de trabajadores y sobre los requisitos de contratación, entre otras cuestiones la web incluye, por otra parte, un apartado específico para formación, que orienta sobre dónde aprender o mejorar cualquiera de las tres lenguas de la eurorregión, así como sobre aspectos diversos relacionados con la movilidad estudiantil, becas y formación profesional con prácticas en empresas. agentes de empleo el nuevo portal funciona, además, como nexo de unión entre los actores de empleo de los tres territorios, ya que actúa como una intranet para la consulta e intercambio de información entre la treintena de agentes del tejido empresarial, servicios públicos de empleo, sindicatos y agencias de desarrollo que participan del proyecto empleo. fruto de este mismo proyecto, fue el primer estudio sobre empleo transfronterizo, realizado en 2017. dicho estudio puso en evidencia que no existe una verdadera cuenca de empleo transfronterizo en la eurorregión, ya que son apenas 3.900 las personas que cruzan diariamente la muga para trabajar, dentro del espacio conformado por la eurorregión nueva aquitania-euskadi-navarra, que suma 8,7 millones de habitantes. el estudio subrayó la multitud de barreras idiomáticas, fiscales y de transporte que encuentran los trabajadores, pero también localizó fortalezas, entre otras, que las tres regiones tienen una perspectiva de crecimiento del empleo para los próximos ocho años (entre el 5 y el 9\%) y que cuentan con estrategias regionales de innovación que podrían compartir. hay sectores que se pueden convertir en una apuesta común y un espacio de oportunidad, como la fabricación avanzada, las energías renovables, la salud, el sector agroalimentario y las industrias creativas. & 557 & very low & Low & Socio-Economic & NA & NA & 2018-12-11 & 2018 & 3 & ECO
Frame & v.low & Regional & 500-1000 & 0.0141942 & -0.1980960 & -0.1536394 & -0.9382144 & 0.7927531 & 12.6 & 1.2981504 & 0.5794339 & Recipient & Domestic & European & Mixed & Domestic|ECO & Positive\\
\addlinespace
Spain & http://www.elmundo.es/comunidad-valenciana/alicante/2018/01/23/5a67213d268e3ea1678b459b.html & 597 & EL MUNDO & Private/Non-Public & Online and Offline & National & very low = CP mentioned once & Jobs & Positive & Subnational & No myth & NA & NA & NA & NA & NA & NA & NA & NA & Spain & alicante contará con una nueva lanzadera de empleo para fomentar la inserción laboral de 20 personas & 2018-01-23 & fondo social europeo & fundación santa maría la real, fundación telefónica y la agencia local de desarrollo del ayuntamiento de alicante renuevan su colaboración contarán de nuevo con la cofinanciación del fondo social europeo, dentro del programa operativo poises la iii lanzadera de empleo de alicante comenzará a funcionar en marzo y servirá para ayudar a 20 personas en situación de desempleo a entrenar una nueva búsqueda de trabajo en equipo y contar con nuevas posibilidades de inserción laboral. para ello contarán con la ayuda y orientación de personal técnico experto que les facilitará nuevas herramientas de búsqueda y guiará para reforzar sus competencias y mejorar su empleabilidad. la nueva lanzadera estará destinada mayoritariamente a jóvenes menores de 35 años, debidamente inscritas como demandantes de empleo. si las 20 plazas no se cubren con perfiles de estas edades, se reserva la posibilidad de que cinco puedan destinarse a personas de mayor edad, hasta los 59 años. podrán apuntarse personas de cualquier nivel formativo (eso, formación profesional, bachillerato, así como diplomaturas o licenciaturas universitarias) y procedentes de cualquier sector laboral, tengan o no experiencia previa. las personas que resulten seleccionadas, se reunirán varios días a la semana en locales cedidos gratuitamente por el ayuntamiento. con la guía y orientación de un técnico especializado llevarán a cabo diversas actividades para optimizar su búsqueda de trabajo: talleres de autoconocimiento e inteligencia emocional, dinámicas de comunicación, marca personal y búsqueda de empleo 2.0; entrenamiento de entrevistas personales; elaboración de mapas de empleabilidad y visitas a empresas. "la lanzadera no es un mero curso para actualizar el curriculum. tampoco es una agencia de colocación donde damos trabajo a los participantes. es un programa intensivo de cinco meses para ayudar a personas desempleadas a activar y optimizar su búsqueda de trabajo en el mercado actual, con nuevas técnicas que les permiten ganar confianza y seguridad en las entrevistas, y con nuevas actividades que les acercan al tejido empresarial", explican desde fundación santa maría la real. "las lanzaderas son un proyecto ilusionante e innovador en su enfoque y en su apuesta por entender el trabajo colectivo como una manera de superar la situación de desempleo. desde fundación telefónica apoyamos esta iniciativa buscando ser un agente multiplicador que potencien sus buenos resultados, ayudando a personas desempleadas a encontrar su camino laboral", agregan desde fundación telefónica. las personas interesadas en participar en el programa, de carácter gratuito y voluntario, disponen hasta el 21 de febrero de 2018 para inscribirse a través de la web de lanzaderas o en formato presencial, solicitando y presentando el formulario en la agencia local de alicante (c/jorge juan, 21, 2ª planta), en horario de 9 a 14 horas. también podrán consultar información adicional en el teléfono 965 145 700. & 453 & very low & Low & Socio-Economic & NA & NA & 2018-01-23 & 2018 & 3 & ECO
Frame & v.low & National & <500 & 0.0141942 & -0.1980960 & -0.1536394 & -0.9382144 & 0.7927531 & 12.6 & 1.2981504 & 0.5794339 & Recipient & Domestic & Domestic & Domestic & Domestic|ECO & Positive\\
Spain & https://www.elmundo.es/andalucia/2018/11/18/5bf1c0b722601d544c8b45ab.html & 654 & EL MUNDO & Private/Non-Public & Online and Offline & National & very low = CP mentioned once & Economic development & Positive & EU + National + Subnational & No myth & NA & NA & NA & NA & NA & NA & NA & NA & Spain & de idiotikos y politikois & 2018-11-18 & fondos estructurales & en la antigua grecia los asuntos de estado concernían a todos los habitantes de la "polis" el 2 de diciembre los andaluces volveremos a pronunciarnos sobre la mejor opción de gobierno que deseamos para gestionar los asuntos de nuestra comunidad. abierta ya la campaña electoral es perfectamente constatable que el interés despertado en el electorado difiere mucho de aquellas donde la megafonía, la cartelería, la quincallería, el puerta puerta o los mítines interrumpían nuestra vida diaria para atender a candidatos y acompañantes, que con sonrisa electoral ad hoc y disculpas constantes intentábamos convencerles de las bondades de nuestras programáticas promesas. toda esta parafernalia ha dado paso a los escuetos mensajes y debates que mueven las poderosas redes sociales o los digitalizados encuentros sectoriales que suelen celebrarse en un ambiente de más proximidad entre los intervinientes y el reducido público asistente. del ruido callejero o las paredes tapizadas con las caras de los candidatos ya solo quedan los reportajes gráficos o televisivos de las visitas programadas a una explotación agraria o industrial y algunas entrevistas más o menos artificiosas en los periódicos, radios o televisiones, habitualmente poco ingeniosas y repetitivas nada que objetar a que las nuevas tecnologías, las exigencias medioambientales y las herramientas que han impuesto los nuevos canales de comunicación más sofisticados, más directos y más "limpios" para la convivencia ciudadana, pero es un hecho cierto que hoy el elector tiene un menor grado de interés y que ha disminuido notablemente la emotividad y la atención que rodeaban las anteriores campañas y los grandes mítines. . a nadie se le oculta que la inmoral y despreciable conducta de quienes se han enriquecido injustamente en estos últimos años, manchando de esta manera la trayectoria honesta y desinteresada de miles de hombres y mujeres que se han entregado al servicio de españa en el ejercicio de la política, ha contribuido al descrédito y la desconfianza en los partidos políticos. todos, los de la vieja y nueva política, están estigmatizados por un alejamiento del ciudadano hacia las palabras o promesas de los políticos. en la antigua grecia, los asuntos del estado concernían a todos los habitantes de la "polis". politikoi era todo lo que rodeaba al interés de los ciudadanos por los asuntos del estado y a esto se contraponían los idiotikos o asuntos de interés privado o personal. en el panorama político actual, da la impresión que han ganado más peso los "idiotikos o idiotes" que los "politikoi o políticos" porque la percepción del ciudadano es que los intereses privados o personales están muy por encima de los generales del estado. lo cierto y verdad es que si el cambio no se produce después del 2 de diciembre, los andaluces seguiremos sufriendo a los "idiotikos" que se afanan más en defender sus intereses personales o de partido que los generales de la comunidad, llegando al extremo de confundir andalucía con el propio partido socialista. una crítica al partido es una crítica a andalucía, una critica a susana díaz es un "desprecio" a andalucía cuando no una "canallada". frente a esto la "politikei" exige afrontar con realismo las carencias y deficiencias que hoy sufre andalucía: las de nuestra sanidad pública después de más de treinta años años de gobierno socialista o los graves problemas de seguridad ocasionados por el alarmante aumento de la criminalidad organizada por el narcotráfico o la trata de seres humanos. la "politikei" exige también ocuparse de la llegada masiva a nuestras costas y playas, de pateras repletas de inmigrantes ilegales, con final trágico para muchos de ellos ya que dejan sus esperanzas en un mediterráneo ensangrentado por quienes trafican con sus vidas, o de un sistema educativo que pretende hacer de la escuela un centro de adoctrinamiento estatalizado impidiendo a los padres y familias elegir libremente los centros de educación de sus hijos o poniendo trabas a la concertación de aquellos que optan por la educación diferenciada, cuya oportunidad y legalidad ha sentenciado el tc en repetidas ocasiones.. no es de recibo que, canal sur, nuestra televisión pública, haya renunciado a promocionar la cultura que se merece el pueblo andaluz repleto de prestigiosos profesionales del mundo del arte, las ciencias o las letras. ni nuestros mayores ni nuestros jóvenes se merecen el tratamiento que se le ofrecen en algunos de sus programas, sin mencionar la falta de objetividad y manipulación de sus informativos según le convenga a la actual regidora socialista. la "politikei" debe asumir y tratar de solucionar también las graves deficiencias que presentan hoy algunos de nuestros juzgados o tribunales o la importante presión fiscal, impuesto de sucesiones incluido, que son un relevante obstáculo para las inversiones y desarrollo de nuestro sistema productivo. es cierto que la andalucía de hoy no es ni la sombra de la de hace cuarenta años. los ingentes y millonarios fondos estructurales de la unión europea y la gestión y cooperación entre administraciones públicas, unido al impulso empresarial y al esfuerzo de millones de trabajadores han contribuido al desarrollo de sus infraestructuras, de su potencial turístico, del sector servicios o del desarrollo urbanístico, agrícola y rural. los andaluces nos sentimos orgullosos de nuestra tierra y de los logros conseguidos. pero necesitamos hombres y mujeres que desde la "politikei" sirvan con generosidad y entrega a una andalucía, un poco ya cansada, fatigada y desconfiada de los "idiotikos" que la han venido gobernando. ha llegado la oportunidad de iluminar una nueva andalucía, para no caer en la desesperanza que denunciaba con su especial finura poética federico garcía lorca: si la esperanza se apaga y la babel se comienza, ¿qué antorcha iluminará los caminos en la tierra? conforme a los criterios de saber más & 939 & very low & Low & Socio-Economic & NA & NA & 2018-11-18 & 2018 & 3 & ECO
Frame & v.low & National & 500-1000 & 0.0141942 & -0.1980960 & -0.1536394 & -0.9382144 & 0.7927531 & 12.6 & 1.2981504 & 0.5794339 & Recipient & Domestic & European & Mixed & Domestic|ECO & Positive\\
Spain & http://www.telecinco.es/informativos/sociedad/Comision-Europea-trasladar-UE-ONCE\_0\_2122050204.html & 608 & Telecinco & Private/Non-Public & Online only & National & low = CP mentioned more times but NOT important part of story (mainly about others issues) & Social awareness/inclusion & Positive & EU + National & No myth & NA & NA & NA & NA & NA & NA & NA & NA & Spain & la comisión europea ve positivo trasladar a otros países de la ue el modelo de gestión social de la once & 2016-01-26 & fondo social europeo & el presidente de la once y su fundación, miguel carballeda, acompañado del consejero general adjunto al presidente y responsable de la fundación once, alberto durán, y la consejera responsable de relaciones internacionales, ana peláez, acudió a bruselas este lunes 25 de enero para continuar con la ronda de reuniones iniciada el pasado mes de noviembre con comisarios europeos con el objetivo de dar a conocer la labor de esta entidad. carballeda indicó al vicepresidente de la comisión que la once es rentable económica y socialmente, crea empleo para personas con discapacidad, "incluso en momentos de crisis", y realiza una cogestión "eficaz" del fondo social europeo, logrando resultados en materia de formación y empleo "por encima de los que la propia ue establece para el conjunto de la unión". carbelleda puso como ejemplo a los estudiantes ciegos en españa que tienen un nivel de abandono escolar del 9,6\%, dos puntos por debajo de la media de la ue; los planes conjuntos entre fundación once y el fondo social europeo, dirigidos a los jóvenes con discapacidad, que han logrado ya más de 5.000 empleos; o la puesta en marcha de en el último año de 54 becas (31 mujeres y 23 hombres) para formación superior de estudiantes con discapacidad. el presidente de la once y su fundación recordó asimismo al comisario katainen cómo la once basa parte de su acción en la gestión del juego "bajo un estricto control público y un verdadero compromiso de gestión responsable", que se suma a la labor de ilunion, el grupo empresarial que actualmente cuenta con una plantilla de 32.000 personas, 11.000 con discapacidad. aambas partes confiaron en que el acta europea de accesibilidad, recién publicada por la comisión, sirva para alinear y armonizar la legislación de los estados miembro en materia de accesibilidad y evitar así la fragmentación de este sector que, de lo contrario, según la once, "será menos favorable e igualitario para todas las personas con discapacidad de la ue, extensible también a las personas mayores". la once y su fundación formularán sus recomendaciones al texto dentro de la consulta lanzada por la comisión europea. & 356 & low & Low & Socio-Economic & NA & NA & 2016-01-26 & 2016 & 2 & ECO
Frame & low-medium & National & <500 & 0.0141942 & -0.1980960 & -0.1536394 & -0.9382144 & 0.7927531 & 12.6 & 1.2981504 & 0.5794339 & Recipient & Domestic & European & Mixed & Domestic|ECO & Positive\\
Spain & http://www.farodevigo.es/galicia/2017/06/30/adif-licita-proyecto-montaje-via/1708364.html & 645 & Faro de Vigo & Private/Non-Public & Online and Offline & Regional/Local & very low = CP mentioned once & Infrastructure & Positive & National + Subnational & No myth & NA & NA & NA & NA & NA & NA & NA & NA & Spain & adif licita el proyecto para el montaje de vía de un tramo de 46 km del ave en ourense & 2017-06-30 & fondo europeo de desarrollo regional & se trata del trayecto que discurre entre campobecerros y taboadela adif alta velocidad acaba de licitar la redacción del proyecto constructivo de montaje de vía para el tramo campobecerros-taboadela, perteneciente a la línea de ave madrid-galicia, a su paso por la provincia de ourense. el tramo campobecerros-taboadela tiene 46,5 kilómetros de longitud y atraviesa los términos municipales de castrelo do val, laza, vilar de barrio, baños de molgas, xunqueira de ambía, allariz, paderne de allariz y taboadela. según explica adif, se ha diseñado para doble vía en ancho estándar, y cuenta con dos partes diferenciadas, cada una de unos 23 kilómetros: una doble plataforma (una para cada vía) entre campobecerros y porto y una única plataforma para vía doble entre porto y taboadela. el tramo cuenta con un puesto de adelantamiento y estacionamiento técnico en miamán y un puesto de banalización en taboadela, y el presupuesto de licitación del contrato es de casi 270.000 euros, con un plazo de ejecución de tres meses. el montaje de vía podrá ser cofinanciado por el fondo europeo de desarrollo regional a través del programa operativo crecimiento sostenible 2014-2020, según indica adif. & 194 & very low & Low & Socio-Economic & NA & NA & 2017-06-30 & 2017 & 2 & ECO
Frame & v.low & Regional & <500 & 0.0141942 & -0.1980960 & -0.1536394 & -0.9382144 & 0.7927531 & 12.6 & 1.2981504 & 0.5794339 & Recipient & Domestic & Domestic & Domestic & Domestic|ECO & Positive\\
Spain & http://www.abc.es/internacional/abci-bruselas-aprieta-cinturon-casi-todos-estados-primer-presupuesto-post-brexit-201804302224\_noticia.html & 631 & ABC TU DIARIO EN ESPAÑOL & Private/Non-Public & Online and Offline & National & low = CP mentioned more times but NOT important part of story (mainly about others issues) & Institutional bargaining over funding & Negative & EU + National & No myth & Political leverage & Negative & EU + Other country & NA & NA & NA & NA & NA & Spain & bruselas aprieta el cinturón a casi todos los estados en su primer presupuesto post brexit & 2018-05-01 & fondos estructurales & la salida del reino unido dejaría un "agujero" de 90.000 millones de euros menos de ingresos en siete años el comisario encargado del presupuesto europeo, el alemán gunther oettinger, suele decir que lo único que cuesta la ue es equivalente al precio de un café diario por cada ciudadano. sin embargo, el café será muy probablemente el ingrediente esencial en la discusión que empieza a partir de mañana y que probablemente se prolongará durante largas noches de amargas discusiones. este será además el primer presupuesto post brexit, es decir, que deberá gestionar el agujero financiero que deja la salida de un contribuyente neto y al mismo tiempo las ambiciones del presidente de la comisión, jean-claude juncker, para llevar a la ue hacia políticas nuevas de proyección de futuro, como defensa y seguridad, y que, naturalmente, cuestan dinero y por ello quiere aumentarlo un 10\%. lo que va a presentar la comisión mañana es su propuesta para lo que se llama técnicamente "marco financiero plurianual" y que no es más que una manera de llamar al presupuesto, con la salvedad de que comprende siete años. la razón es evitar que esta discusión y los desencuentros entre países se repitan cada doce meses. hasta ahora se cifraban en el 1\% del pib europeo. juncker quiere subirlos hasta el 1,1\% y aún con todo tendrás que hacer recortes para paliar el agujero dejado por la pérdida de la contribución británica. lo que ahora está sobre la mesa son los números para el periodo entre 2021 y 2027, cuando ni siquiera estará en funciones esta comisión. el luxemburgués está decidido a dejar su huella en las instituciones comunitarias a través de este presupuesto, en el que quiere cambiar las prioridades y las fórmulas para calcular los gastos. pero para eso tiene que convencer a los países, que son los que ponen el dinero.en el actual periodo presupuestario (2014- 2020) ascendió un billón 87.000 millones de euros. es decir, unos 155.000 millones de euros al año, equivalente a la mitad del presupuesto español. aún con los 13.000 millones anuales que aspiran a recaudar ahora, el café del comisario oettinguer es más bien hecho en casa, porque a precios de cafetería de bruselas las cuentas serían muy diferentes. las teorías que presentan a la ue como un monstruo del despilfarro no tienen ninguna base real. al contrario, los beneficios que obtienen los ciudadanos en su vida diaria salen muy baratos. sin embargo, esas teorías se han abierto paso en esta efervescencia nacional-populista que se ha propagado por muchos países y que está llevando a sus dirigentes a posiciones nunca vistas antes. las primeras víctimas han sido los británicos, que al irse dejan un agujero que según los distintos cálculos puede llegar a los 90.000 millones en todo el periodo de siete años. en los términos del acuerdo de divorcio que aún se está negociando, queda claro que tendrán que pagar todos sus compromisos hasta diciembre de 2020, cuando concluye el actual periodo presupuestario plurianual, pero no se sabe todavía nada del futuro, salvo que se entiende que tendrán que pagar solo por las políticas en las que decidan participar. en todo caso, la idea de la comisión es que después del brexit se acabarán todos los "cheques" y descuentos que se habían instaurado sucesivamente debido a las reticencias de londres. pero aún así, la comisión ha necesitado de mucha imaginación para establecer unos presupuestos en los que tendrá menos ingresos y aspira a introducir nuevos gastos de forma que se mantengan al mismo tiempo los capítulos más emblemáticos de la acción de la ue como la política agrícola común (pac) y los fondos de cohesión que hasta ahora se llevan más de dos tercios del dinero. la primera propuesta que va a introducir la comisión para resolver este enigma será cambiar el criterio por el que se distribuyen los fondos estructurales y de cohesión de modo que ya no se calcularán directamente en función de la renta de los ciudadanos de cada región, sino que se introducirán otros elementos como el nivel de desempleo o su participación en la política de acogida de refugiados. se trata de una fórmula que parece razonable, pero que parece hecha a medida para atacar a los países del este, como polonia o hungría, que son los principales beneficiarios del dinero europeo, pero que responden con actitudes políticas que no se compadecen con las políticas del ejecutivo comunitario. pero los recortes no serán solo para los polacos o los húngaros. hasta ahora la comisión ha gastado el presupuestos en pura inversión, en proyectos concretos que se desarrollan en los países. a partir de ahora la comisión quiere tener también la posibilidad de gastar en acciones propias dedicadas a nuevas políticas. por ejemplo, en protección de fronteras exteriores, que se ha revelado como uno de los asuntos que pone en peligro la solidez del área de libre circulación, o la financiación de proyectos en la industria de la defensa, campos nuevos para la ue. la cuestión del calendario es también muy importante. los expertos de la comisión temen que si la negociación se envenena, sería sin duda una sobredosis de combustible para los partidos antieuropeos y populistas. normalmente las negociaciones suelen durar dos años, y se considera que será imposible haber llegado a un acuerdo antes de las elecciones, en mayo de 2019, como quisiera juncker. por ello se prevé que harán una presentación escalonada de los distintos capítulos, para que aquellos países que reciban con simpatía una parte no tengan tantos argumentos después para rechazar las otras. se necesita un acuerdo por unanimidad y hasta la minúscula malta puede bloquear. naturalmente, los países que más tienen que hablar son los que más pagan. y ahí está la razón por la que el encargado de hacer los cálculos es el comisario alemán. cuando se anunció que oettinger dejaba la muy estratégica cartera de energía solo podía ser por otra aún más interesante como decidir cuánto costará un café en europa. & 1007 & low & Low & Power & Power & NA & 2018-05-01 & 2018 & 3 & POL
Frame & low-medium & National & +1000 & 0.0141942 & -0.1980960 & -0.1536394 & -0.9382144 & 0.7927531 & 12.6 & 1.2981504 & 0.5794339 & Recipient & Domestic & European & Mixed & Domestic|POL & Negative\\
\addlinespace
Spain & http://www.lainformacion.com/economia/Gobierno-IBI-mantendra-Impuesto-Patrimonio-presuspuestos-bruselas-2017-espana-deficit\_0\_963204165.html & 640 & lainformacion & Private/Non-Public & Online only & Regional/Local & medium = CP is important part of story & Political leverage & Negative & EU + National & No myth & NA & NA & NA & NA & NA & NA & NA & NA & Spain & hacienda mantendrá el ibi y patrimonio para intentar convencer a bruselas & 2016-10-16 & fondos estructurales & el presupuesto prorrogado fija en el 3,6\% el déficit en 2017, medio punto por encima de lo estipulado por bruselas, por lo que el próximo gobierno deberá recortar unos 5.000 millones. relacionados guindos advierte que habrá que recortar 5.000 millones en el presupuesto de 2017 una cuarta parte de los empleados de la banca se han ido al paro durante la crisis bruselas tiene ya a su disposición los presupuestos generales -prorrogados- de españa. y habrá recortes y subidas de impuestos para el próximo ejercicio. el documento remitido por el partido popular al seno de las instituciones europeas no sería el definitivo, ya que, en caso de abstenerse el psoe, ciertos detalles como la cifra del déficit deberán ser ajustadas a lo exigido por la comisión europea. el 3,6\% de déficit que supondría acabar 2017 con esos presupuestos -que son una prórroga de los 2016- está medio punto por encima del 3,1\% que pidió bruselas para el siguiente ejercicio. españa deberá recaudar unos 5.000 millones para cumplir con la petición, y para ello tendrá que apretar las tuercas en materia fiscal. si este año se acaba con un 4,6\% de déficit significará que la reforma en el tipo de sociedades cumplirá su objetivo. también significará que la bajada de impuestos realizada por el partido popular en vísperas electorales ha dado unos cuantos sustos innecesarios. se calcula que debido a ese recorte se 'perdieron' unos 4.000 millones. esto no solo significó estar a punto de recibir una penosa multa por incumplir en enésima ocasión lo dictado por bruselas, también que se hayan tenido que adelantar 8.600 millones del cobro del impuesto de sociedades. consulta el documento completo aquí para legar a ese 3,1\% el gobierno en funciones se ha comprometido con bruselas a prorrogar el impuesto de patrimonio, lo que supondrá 1.300 millones de euros para las arcas a partir de 2017 -estaba previsto que desapareciera en el próximo ejercicio-, y mantendrá la posibilidad de aplicar coeficientes incrementados del impuesto sobre bienes inmuebles (ibi). aunque el ibi dependa de los ayuntamientos, el ejecutivo dicta la línea a seguir. estaba previsto que la medida caducase en 2013, pero el gobierno lo ha ido alargando ya que es el único impuesto que no ha caído nunca y recauda unos 13.000 millones al año. mantener los tipos incrementados supondrá 24 millones de euros. en cuanto al empleo, el documento asegura que españa pasará de un paro del 19,7\% en 2016 al 17,8\% en 2017, unos 400.000 empleos, algo menos que en 2016. aun así, españa tiene desde hace varios años el índice más elevado de desempleo en la zona euro, sólo por detrás de grecia. el punto fuerte de la economía española sigue siendo su crecimiento: al 2,3\% en 2017, algo menos de lo que se crecerá en 2016 -un 3\% o superior-, siendo la economía desarrollada que más crezca en los próximos dos años. juntando este crecimiento estable al recorte que se deberá acometer en 2017 el ejecutivo de rajoy espera ganarse la confianza de bruselas y que castigos como la congelación de los fondos estructurales se queden en agua de borrajas. jean-claude juncker, presidente de la comisión europea, avisó de que no le temblaría el pulso a la hora de congelar los fondos estructurales, a ambos países, si estos no corregían sus déficit en la segunda mitad del año y en el borrador presupuestario de 2017. además, el gobierno calcula que la presión fiscal -porcentaje de impuestos obtenidos por los impuestos- alcanzará el 37,8\% del pib en 2016 y un 37,7\% en 2017, por debajo de la media europea. por último, las pensiones subirán un 0,25\% en 2017, el mínimo establecido por la ley, aumentando el gasto total en pensiones un 3\%. & 640 & medium & Medium & Power & NA & NA & 2016-10-16 & 2016 & 2 & POL
Frame & low-medium & Regional & 500-1000 & 0.0141942 & -0.1980960 & -0.1536394 & -0.9382144 & 0.7927531 & 12.6 & 1.2981504 & 0.5794339 & Recipient & Domestic & European & Mixed & Domestic|POL & Negative\\
Spain & http://ecodiario.eleconomista.es/espana/noticias/5938796/07/14/FSC-Inserta-imparte-en-Jaen-un-taller-de-alfabetizacion-digital-para-15-personas-con-discapacidad.html & 643 & El Economista (EcoDiario) & Private/Non-Public & Online and Offline & National & low = CP mentioned more times but NOT important part of story (mainly about others issues) & Social awareness/inclusion & Positive & National + Subnational & No myth & NA & NA & NA & NA & NA & NA & NA & NA & Spain & fsc inserta imparte en jaén un taller de alfabetización digital para 15 personas con discapacidad & 2014-07-14 & fondo social europeo & fsc inserta imparte en camarena (toledo) un curso de conserje ordenanza para personas con discapacidad (11/07)
fsc inserta, la entidad para la formación y el empleo de la fundación once, ha puesto en marcha en jaén un taller de alfabetización digital al que asisten 15 personas con discapacidad y que se impartirá hasta este próximo viernes día 18.
 jaén, 14 (europa press)
según ha informado este lunes la entidad en una nota, esta acción formativa tiene una duración de 25 horas lectivas divididas en tres módulos denominados 'profundizamos en el sistema operativo', 'aplicaciones ofimáticas' y 'avanzamos en internet'.
 fsc inserta proporciona a los alumnos "las destrezas profesionales que les permitan un rendimiento competitivo en el mercado de trabajo y el desarrollo de aptitudes y habilidades personales para conseguir la plena participación en su entorno laboral y social".
 todos los cursos de formación se enmarcan dentro del programa 'por talento', que está desarrollando la fundación once a través de fsc inserta con la cofinanciación del fondo social europeo, cuyo objetivo es "incrementar la empleabilidad y la inserción laboral de las personas con discapacidad". & 183 & low & Low & Socio-Economic & NA & NA & 2014-07-14 & 2014 & 1 & ECO
Frame & low-medium & National & <500 & 0.0141942 & -0.1980960 & -0.1536394 & -0.9382144 & 0.7927531 & 12.6 & 1.2981504 & 0.5794339 & Recipient & Domestic & Domestic & Domestic & Domestic|ECO & Positive\\
Spain & http://ecodiario.eleconomista.es/espana/noticias/8281034/04/17/La-Junta-financia-la-convocatoria-de-universidades-para-contratar-1300-tecnicos-de-apoyo-a-la-investigacion.html & 683 & El Economista (EcoDiario) & Private/Non-Public & Online and Offline & National & very low = CP mentioned once & Research \& innovation & Positive & Subnational & No myth & NA & NA & NA & NA & NA & NA & NA & NA & Spain & la junta financia la convocatoria de universidades para contratar 1.300 técnicos de apoyo a la investigación & 2017-04-07 & fondo social europeo & la consejería de economía y conocimiento de la junta de andalucía financia la convocatoria que en estos días están poniendo en marcha las universidades públicas andaluzas para la contratación de 1.300 técnicos y técnicas de apoyo a la investigación. sevilla, 7 (europa press) con esta iniciativa se movilizarán casi 37 millones de euros de los 61 que la consejería contempla en su presupuesto de 2017 en la financiación básica de las universidades para la contratación de jóvenes investigadores y personal técnico de apoyo a la i+d+i, con cargo al fondo social europeo, según ha precisado en una nota el departamento que encabeza antonio ramírez de arellano. esta primera actuación, enmarcada en el programa operativo de empleo juvenil (objetivo temático 08), se dirige a personas jóvenes --mayores de 16 y menores de 30 años--, titulados universitarios o en formación profesional, a los que se contratará como personal técnico de apoyo y de gestión de la i+d, reforzando el trabajo de los institutos, grupos, equipos, centros propios y mixtos, instalaciones y unidades científicas de apoyo, servicios generales y servicios de gestión de investigación en las universidades andaluzas. para ello, economía y conocimiento ha transferido a las instituciones universitarias un montante de 36,9 millones de euros en esta primera fase, siendo éstas las beneficiarias directas del programa, y quienes convocan sus propios procedimientos competitivos de selección, en función de las cantidades y plazas consignadas a cada una de ellas. el reparto de este paquete económico por universidad, habiendo atendido en su totalidad a la demanda de necesidades de cada una, queda con 8,4 millones para la universidad de sevilla; 7,7 para granada; 4,7 para cádiz; 3,7 para jaén; 3,6 para almería, tres millones para málaga; 2,9 para córdoba; 2,3 millones para la universidad pablo de olavide de sevilla; y 360.000 euros para huelva. las convocatorias que las universidades públicas andaluzas ponen en marcha contarán con un plazo de diez días hábiles a partir del día siguiente a su publicación. las personas destinatarias de estas acciones deberán estar inscritas en el sistema nacional de garantía juvenil a la fecha de finalización del plazo de presentación de solicitudes. asimismo, deben estar en posesión de una titulación universitaria o de una titulación en el marco del sistema de la formación profesional, según las características de las plazas ofertadas en la correspondiente convocatoria. el criterio de selección a utilizar será la nota media del expediente académico. los contratos se formalizarán en la modalidad de contrato en prácticas y tendrán una duración mínima de seis meses, prorrogables por otros seis. con la puesta en marcha de la primera fase de esta iniciativa se persigue impulsar trayectorias profesionales de apoyo a la actividad investigadora en cualquiera de los ámbitos de conocimiento, de forma que el personal beneficiario de estos contratos adquiera nuevas habilidades y mejora su capacitación y competitividad para oportunidades futuras en el mercado laboral. la segunda fase, a través de la cual se incorporarán a jóvenes investigadores al sistema científico, se pondrá en marcha en otoño de 2017. este programa se enmarca dentro del plan andaluz de investigación, desarrollo e innovación (paidi) horizonte 2020, un plan estratégico para fortalecer el sistema científico andaluz, que va a movilizar en 2017 casi 340 millones de euros a través de una serie de mecanismos de apoyo a la i+d+i. el pasado mes de febrero, el consejo andaluz de universidades ratificaba el acuerdo para el reparto de diez millones de euros, destinados a fortalecer a los grupos de investigación y consignados en el presupuesto de 2017. esta medida suponía la puesta en marcha del paidi 2020, que continúa ahora con este programa para la contratación de técnicos de apoyo a la i+d. el paidi activa, así, sus primeras líneas de actuación, y lo hace reforzando las bases del sistema: incorporando a jóvenes talentos y apoyando la actividad de los grupos, unidad básica en ciencia. igualmente, la consejería de economía y conocimiento está ultimando una orden de bases específica que contemplará incentivos de carácter competitivo para todo el sistema de conocimiento de la comunidad autónom, que incluye además de a las universidades a los centros públicos de investigación, parques y centros tecnológicos, centros de impulso de la transferencia del conocimiento o instituciones de divulgación. & 718 & very low & Low & Socio-Economic & NA & NA & 2017-04-07 & 2017 & 2 & ECO
Frame & v.low & National & 500-1000 & 0.0141942 & -0.1980960 & -0.1536394 & -0.9382144 & 0.7927531 & 12.6 & 1.2981504 & 0.5794339 & Recipient & Domestic & Domestic & Domestic & Domestic|ECO & Positive\\
Spain & http://www.europapress.es/cantabria/noticia-gobierno-prc-psoe-incrementara-plan-empleo-corporaciones-locales-modificara-criterios-20150902115427.html & 618 & europa press & Private/Non-Public & Online only & National & very low = CP mentioned once & Jobs & Positive & Subnational & No myth & NA & NA & NA & NA & NA & NA & NA & NA & Spain & se incrementará el plan de empleo de corporaciones locales & 2015-09-02 & fondo social europeo & actualizado 02/09/2015 13:47:19 cet quiere que el plan abarque más parte del año y que el servicio cántabro de empleo sea el que seleccione a los candidatos santander, 2 sep. (europa press) - el gobierno de cantabria prc-psoe mantendrá de cara a 2016 las ayudas a las corporaciones locales para la contratación de parados, con previsión de incrementar, incluso, la partida destinada a ello y el número de beneficiarios, pero modificará los criterios fijados en la convocatoria del año pasado por parte del anterior ejecutivo regional (pp) para que este plan de empleo abarque una mayor parte del año y para que sea el servicio cántabro de empleo el que seleccione a los candidatos. según estableció el anterior ejecutivo del pp, en 2015 se han destinado 28 millones a la orden de ayudas a las corporaciones locales, que ha permitido contratar este año a 4.000 parados. ahora el bipartito prc-psoe intentará incrementar la partida --sota no ha especificado la cuantía-- y los beneficiarios, que podrían llegar a cerca de 5.000 en 2016. así lo ha anunciado este miércoles el consejero de economía, hacienda y empleo, juan josé sota (psoe), en declaraciones realizadas tras reunirse con el alcalde de santander, íñigo de la serna (pp). el regidor ha calificado como una "excelente noticia para todos" la reedición de este programa de ayudas y se ha mostrado "absolutamente a favor" de que se realicen los cambios "que haya que hacer" para garantizar la seguridad jurídica de las bases y los procesos de selección. con estos cambios de criterios anunciados por sota, el gobierno de cantabria confía, por un lado, en evitar el "problema" que, según ha apuntado, ha supuesto este año la pérdida del 50\% de la financiación del fondo social europeo (fse) de los fondos para estas contrataciones por "incumplimientos" en los procesos de selección. de esta manera, el ejecutivo regional intentará beneficiarse en próximos ejercicios de esa financiación. por otra parte, el gobierno regional quiere que las contrataciones se desarrollen a lo largo de un periodo mayor del año para que los ayuntamientos tengan "disponibilidad plena" de este personal "desde la primavera al otoño", que es el periodo de "más demanda", para que puedan programar las actuaciones que consideren necesarias y que serán desempeñadas por los seleccionados. sota ha explicado que la intención del gobierno es tener ya "a primeros" de 2016 la orden de ayudas ya publicada para poder fijar con los ayuntamientos las necesidades de contratación. ((habrá ampliación)) & 417 & very low & Low & Socio-Economic & NA & NA & 2015-09-02 & 2015 & 1 & ECO
Frame & v.low & National & <500 & 0.0141942 & -0.1980960 & -0.1536394 & -0.9382144 & 0.7927531 & 12.6 & 1.2981504 & 0.5794339 & Recipient & Domestic & Domestic & Domestic & Domestic|ECO & Positive\\
Spain & http://www.20minutos.es/noticia/2274187/0/fsc-inserta-participara-este-jueves-con-stand-feria-enterate-2014-hotel-gran-domine-bilbao/ & 625 & 20 minutos & Private/Non-Public & Online and Offline & National & very low = CP mentioned once & Jobs & Positive & National + Subnational & No myth & NA & NA & NA & NA & NA & NA & NA & NA & Spain & fsc inserta participará este jueves con un stand en la feria 'entérate 2014' en el hotel gran domine de bilbao & 2014-10-22 & fondo social europeo & gracias al programa por talento, cofinanciado por fundación once y el fondo social europeo fsc inserta, la entidad para la formación y el empleo de fundación once, participará este jueves en bilbao con un stand en la feria 'entérate 2014', organizada por b4work, empresa dedicada a fomentar la relación de los jóvenes con las empresas y entidades. según ha informado fsc inserta, el objetivo principal de este encuentro, que se celebrará en el hotel silken gran domine bilbao, es ayudar a los jóvenes a conectar con las entidades expositoras y los participantes de un modo interactivo y dinámico, además de resolver las inquietudes y necesidades de éstos, planteándoles en un mismo espacio un amplio abanico de opciones de futuro. la feria se divide en zonas de exposición distribuidas por áreas: masters, reclutamiento, internacionalización (idiomas y salidas al extranjero), emprendimiento, información de becas y ayudas, voluntariado y servicio para jóvenes. en el stand de fsc inserta, técnicos expertos informarán a los visitantes sobre la actividad que desarrolla esta entidad de fundación once en favor del empleo y la formación de las personas con discapacidad. su participación en este evento se enmarca en el programa por talento, cofinanciado por el fondo social europeo. & 201 & very low & Low & Socio-Economic & NA & NA & 2014-10-22 & 2014 & 1 & ECO
Frame & v.low & National & <500 & 0.0141942 & -0.1980960 & -0.1536394 & -0.9382144 & 0.7927531 & 12.6 & 1.2981504 & 0.5794339 & Recipient & Domestic & Domestic & Domestic & Domestic|ECO & Positive\\
\addlinespace
Spain & https://ecodiario.eleconomista.es/espana/noticias/9394028/09/18/Discapacidad-mas-de-500-jovenes-con-discapacidad-participaran-en-el-game-de-empleo-de-fundacion-once-en-madrid.html & 691 & El Economista & Private/Non-Public & Online and Offline & National & very low = CP mentioned once & Social awareness/inclusion & Positive & National & No myth & NA & NA & NA & NA & NA & NA & NA & NA & Spain & discapacidad. más de 500 jóvenes con discapacidad participarán en el game de empleo de fundación once en madrid & 2018-09-18 & fondo social europeo & el complejo deportivo y cultural de la once, situado en el madrileño paseo de la habana 208, acogerá el próximo 26 de septiembre un game de empleo organizado por fundación once en colaboración con la cámara de comercio de madrid, en el que está previsto que participen más de medio millar de jóvenes con discapacidad inscritos en el sistema nacional de garantía juvenil. con esta iniciativa, que se desarrolla en el marco del programa operativo de empleo juvenil 2014-2020 (poej) del fondo social europeo, se quiere mejorar las capacidades de jóvenes con discapacidad en situación de desempleo. el evento, que anteriormente se ha celebrado en valladolid, gijón, sevilla y alicante, se desarrollará bajo el formato de 'gamificación', con el objetivo de motivar a los jóvenes a marcarse unas metas viables, visualizar la importancia de la formación, mostrar oportunidades laborales, despertar nuevas vocaciones y dotar a los asistentes de herramientas efectivas, innovadoras y diferenciadoras, así como trabajar habilidades sociales. las actividades comenzarán con una conferencia a cargo de 'trainers paralímpicos' y continuarán con distintos talleres en los que se trabajarán los objetivos antes mencionados. además, los participantes podrán contactar con responsables de distintas empresas cuyos responsables de recursos humanos harán propuestas de mejora y darán consejos para presentar un curriculum de forma efectiva. & 213 & very low & Low & Socio-Economic & NA & NA & 2018-09-18 & 2018 & 3 & ECO
Frame & v.low & National & <500 & 0.0141942 & -0.1980960 & -0.1536394 & -0.9382144 & 0.7927531 & 12.6 & 1.2981504 & 0.5794339 & Recipient & Domestic & Domestic & Domestic & Domestic|ECO & Positive\\
Spain & http://ecodiario.eleconomista.es/espana/noticias/8336399/05/17/Una-decena-de-jovenes-se-forman-en-actividades-auxiliares-en-viveros-y-jardines-con-el-programa-de-Garantia-Juvenil.html & 626 & El Economista (EcoDiario) & Private/Non-Public & Online and Offline & National & very low = CP mentioned once & Jobs & Positive & National + Subnational & No myth & NA & NA & NA & NA & NA & NA & NA & NA & Spain & una decena de jóvenes se forman en actividades auxiliares en viveros y jardines con el programa de garantía juvenil & 2017-05-04 & fondo social europeo & un total de diez jóvenes se están formando en el ayuntamiento de palma para obtener el certificado de profesionalidad de actividades auxiliares en viveros, jardines y centros de jardinería, a través del programa garantía juvenil de palmaactiva impulsado por el servicio de ocupación de las islas baleares (soib) y cofinanciado por el fondo social europeo. palma de mallorca, 4 (europa press) el programa 'palma verd' consta de 1.860 horas, de las cuales el 26 por ciento son de formación y el 74 por ciento, trabajo efectivo sobre la especialidad, según han indicado el ayuntamiento de palma en un comunicado. también dentro del programa de garantía juvenil, el ayuntamiento de palma realiza la acción formativa 'palma web', que se realiza en colaboración con el instituto municipal de informática (imi) y en la que participan 16 personas. durante la visita, el alcalde de palma, josé hila, ha afirmado que con estos programas -que combinan formación y experiencia práctica- se quiere ayudar a los jóvenes que tienen más dificultades para acceder al mercado laboral, "como los que no han acabado la eso o los que no tienen ninguna experiencia profesional". los alumnos están trabajando en el parc de son quint, en el parc de la mar, en es carnatge, en el parc de jacint verdaguer, en el castillo de bellver y en los viveros municipales. una vez finalizado, los diez participantes obtendrán el certificado de profesionalidad de actividades auxiliares en viveros, jardines y centros de jardinería de nivel 1 y una unidad de competencia del certificado de profesionalidad de nivel 2 de agricultura ecológica. & 262 & very low & Low & Socio-Economic & NA & NA & 2017-05-04 & 2017 & 2 & ECO
Frame & v.low & National & <500 & 0.0141942 & -0.1980960 & -0.1536394 & -0.9382144 & 0.7927531 & 12.6 & 1.2981504 & 0.5794339 & Recipient & Domestic & Domestic & Domestic & Domestic|ECO & Positive\\
Spain & http://cadenaser.com/ser/2017/04/07/sociedad/1491583251\_975926.html & 652 & Cadena SER & Private/Non-Public & Online only & National & very low = CP mentioned once & Social justice & Negative & National & No myth & NA & NA & NA & NA & NA & NA & NA & NA & Spain & protestas por diez despidos en el instituto de la mujer & 2017-04-08 & fondo social europeo & las trabajadoras creen que es una represalia tras haber demandado al organismo por contratos temporales encadenados la ugt denuncia que el gobierno ha vaciado de contenido al organismo referente en las políticas de igualdad el psoe pide explicaciones en el congreso por los despidos y el desmantelamento del instituto el instituto de la mujer prescindió la semana pasada de diez trabajadoras técnicas cuyo contrato estaba vinculado a los programas financiados con el fondo social europeo. se da la circunstancia de que ocho de ellas presentaron el año pasado una demanda conjunta contra el organismo para reclamar sus derechos laborales ya que las ha tenido desde su incorporación con contratos temporales encadenados. "claro que nuestro despido ha sido una represalia" señala isabel garcía calvo, una de las perjudicadas. tras la demanda "a todas nos cambiaron de sitio, nos aislaron de las demás, nos cortaron el acceso a información en red, en el ordenador, algo que nunca me había pasado" señala la ya extrabajadora que estuvo empleada en el instituto de la mujer más de once años. las trabajadoras recibieron la carta de despido el pasado 3 de marzo de este año y tuvieron que recoger sus pertenencias y abandonar el instituto el viernes 31. también corrió la misma suerte otra empleada (que denunció por su cuenta al organismo) a pesar de que un juzgado le dio la razón. isabel garcía afirma que el panorama de "desmantelamiento del instituto es desolador". dice que cada vez tiene menos presupuesto, menos trabajo y que la plantilla está envejecida o se marcha de allí. "si lo ahogas económicamente con menos dinero; si invisibilizas su trabajo; si pones al frente a esta directora y a las otras dos anteriores cuyo concepto de igualdad es el que es; el instituto se muere" lamenta isabel garcía que denuncia además que "ya no es un organismo impulsor de las políticas de igualdad sino que se ha convertido en un mero gestor alejado del objetivo principal para el que fue creado". el grupo parlamentario socialista solicita la comparecencia urgente en el congreso de la actual directora del instituto de la mujer lucía del carmen cerón. el psoe pide que explique la causa de los despidos adelantados por el diario público y que "profundizan en el ataque del pp a las políticas de igualdad". en la misma línea, el sindicato ugt censura que las trabajadoras fueran despedidas por denunciar su situación laboral, que hayan sido represaliadas y condena la estrategia del gobierno de vaciar de contenido al instituto de la mujer. el ministerio de sanidad, servicios sociales e igualdad, del que depende el instituto asegura en un comunicado que "no se ha producido ningún despido" sino que "con fecha de 31 de marzo de 2017 han finalizado los contratos de trabajo de varias trabajadoras que se encontraban prestando sus servicios en virtud de contratos de obra o servicio determinado". según el departamento que dirige dolors montserrat, eran plazas cofinanciadas por el fondo social europeo dentro del programa de "lucha contra la discriminación" que se prorrogaron hasta el mes pasado y que una vez finalizada esa prórroga se les comunicó a las trabajadoras la extinción de su relación laboral. seguir leyendo igualdad instituto de la mujer dolors montserrat ministerio de sanidad desigualdad social relaciones género mujeres ministerios administración estado administración pública sociedad + & 550 & very low & Low & Socio-Economic & NA & NA & 2017-04-08 & 2017 & 2 & ECO
Frame & v.low & National & 500-1000 & 0.0141942 & -0.1980960 & -0.1536394 & -0.9382144 & 0.7927531 & 12.6 & 1.2981504 & 0.5794339 & Recipient & Domestic & Domestic & Domestic & Domestic|ECO & Negative\\
Spain & https://www.lavozdegalicia.es/noticia/galicia/2018/02/16/galicia-abocada-dar-salto-vacio/0003\_201802G16P7993.htm & 686 & La Voz de Galicia & Private/Non-Public & Online and Offline & Regional/Local & medium = CP is important part of story & Institutional bargaining over funding & Negative & EU + National & No myth & NA & NA & NA & NA & NA & NA & NA & NA & Spain & galicia, abocada a dar un salto al vacío & 2018-02-16 & fondos estructurales & la ue pretende eliminar las ayudas a las regiones en transición bruselas ha puesto en marcha la cuenta atrás para la desconexión. no solo la del reino unido con la ue, también la de las eternas regiones en transición, como galicia, de los fondos estructurales y agrícolas. quiere recortes y reorientación de las ayudas hacia sectores que pueden aportar "más valor añadido". eso es lo que deja entrever la hoja de ruta que presentó la comisión europea esta semana para los futuros presupuestos comunitarios 2020-2027. en ella hay tres vías. la de mantener el rumbo actual, que de inicio ya rechaza: "el estatus quo no es un opción", aseguró ayer la comisaria de política regional, corina cretu. si no lo es, quedan otros dos escenarios y en ninguno galicia salvaría un solo euro de los fondos europeos. claro está que se trata de un punto de inicio para arrancar el debate sobre qué hacer con la política de cohesión y cómo financiarla, pero sienta un mal precedente. los fondos estructurales y la pac serán los rehenes con los que acudirá la comisión europea a negociar con los líderes de la ue a partir del mes de mayo. si quieren financiar nuevas prioridades (seguridad, defensa y economía digital), además de evitar un escenario apocalíptico para regiones como galicia, deberán rascarse los bolsillos. "espero que en este mes de febrero, cuando se reúnan nuestros líderes, se enfrenten a la realidad y se planteen la posibilidad de aumentar su contribuciones a la ue tras el brexit", deslizó ayer cretu. pero no hay hambre en la mesa del consejo europeo para seguir financiando otros siete años más los programas de convergencia. hay muchas razones que explican este recelo: el repliegue nacionalista, la desafección ciudadana hacia las instituciones de la ue, largamente estigmatizadas desde las capitales, y las presiones euroescépticas. según los estudios de la comisión europea, galicia es una de esas regiones en transición que, a pesar de los fondos inyectados a lo largo de los últimos 20 años, no ha conseguido despegar para acoplarse al vagón de cabeza de la ue. las regiones menos desarrolladas y las más ricas progresan más rápido que ella. galicia se estanca y su situación puede agravarse si se cronifica el alto desempleo y persiste la sangría demográfica, dos riesgos que bruselas no ha puesto en la balanza a pesar de la insistencia del gobierno gallego. conserve o no los fondos, los recortes de ayudas están asegurados. llegó el momento de dar el salto al vacío. & 419 & medium & Medium & Power & NA & NA & 2018-02-16 & 2018 & 3 & POL
Frame & low-medium & Regional & <500 & 0.0141942 & -0.1980960 & -0.1536394 & -0.9382144 & 0.7927531 & 12.6 & 1.2981504 & 0.5794339 & Recipient & Domestic & European & Mixed & Domestic|POL & Negative\\
Spain & http://www.eleconomista.es/espana/noticias/7692076/07/16/El-autoempleo-femenino-alcanza-el-52-en-la-Region.html & 681 & El Economista & Private/Non-Public & Online and Offline & National & very low = CP mentioned once & Social justice & Positive & Subnational & No myth & NA & NA & NA & NA & NA & NA & NA & NA & Spain & el autoempleo femenino alcanza el 52\% en la región & 2016-07-08 & fondo social europeo & las mujeres empresarias piden financiación para seguir luchando contra la brecha entre hombres y mujeres y más datos sobre empresarias las mujeres empresarias piden financiación para seguir luchando contra la brecha entre hombres y mujeres y más datos sobre empresarias cartagena (murcia), 8 (europa press) el autoempleo femenino en la región ya alcanza el 52 por ciento. estos son los datos que de momento baraja la organización murciana de mujeres empresarias y profesionales (omep) que echa en falta que desde la administración se diferencie por sexo cada empresario que solicita información para montar una empresa porque así "podríamos tener datos más fiables". la presidenta de omep, manuela marín, ha acudido esta mañana a la asamblea regional acompañada de varias empresarias para presentar a la presidenta del parlamento autonómico, rosa peñalver, y a diputados de os distintos grupos parlamentarios las actuaciones que están llevando a cabo para poner en valor la actividad de la mujer en la empresa. en ese sentido, han presentado el observatorio de la igualdad, que se creó en 2015, y que, según dicen, "necesita un apoyo y una financiación". marín ha informado que han solicitado 72.000 euros para realizar acciones de sensibilización dentro de la empresa, formación a empresarias, revitalización de viveros de empresa y mantenimiento del observatorio para dos años. actualmente, colaboran con la consejería de familia, con la que han elaborado un proyecto con cargo al fondo social europeo. "al ser una subvención, están muy encorsetados; nuestra actividad necesita de una planificación estratégica con varios años de duración", ha indicado antes de añadir que durante la reunión pedirán a los grupos parlamentarios que financien la continuidad del observatorio. las empresarias aseguran que en la última década la situación de la mujer en la región "ha mejorado, pero va poco a poco". para ello aluden al estudio que presentaron hace unos días sobre el diagnóstico de la igualdad en las pymes, en el que se desprende que el número de directivas en las empresas está por debajo de la media nacional. no obstante, alertan de que "las empresas están muy concienciadas de que hacen falta medidas de igualdad, la conciliación y trabajar por la brecha salarial", aunque necesitan más formación. "por lo que he visto, las pymes cumplen flexibilidad en los horarios, pero siguen necesitando asesoramiento y ver el reflejo de otras empresas que ya lo han hecho. para tomar medidas también necesitamos los datos de mujeres que crean empresas en la región", concluye. & 409 & very low & Low & Socio-Economic & NA & NA & 2016-07-08 & 2016 & 2 & ECO
Frame & v.low & National & <500 & 0.0141942 & -0.1980960 & -0.1536394 & -0.9382144 & 0.7927531 & 12.6 & 1.2981504 & 0.5794339 & Recipient & Domestic & Domestic & Domestic & Domestic|ECO & Positive\\
\addlinespace
Spain & http://www.farodevigo.es/portada-pontevedra/2018/04/12/frena-reforma-puente-o-burgo/1871314.html & 667 & Faro de Vigo & Private/Non-Public & Online and Offline & Regional/Local & low = CP mentioned more times but NOT important part of story (mainly about others issues) & Infrastructure & Positive & National + Subnational & No myth & NA & NA & NA & NA & NA & NA & NA & NA & Spain & el estado frena la reforma del puente de o burgo y 4 millones más en obras al no liberar los fondos & 2018-04-12 & fondo europeo de desarrollo regional & las obras en el puente ya se podrían adjudicar, pero sin la orden de hacienda no es posible. // rafa vázquez en octubre de 2016 pontevedra recibía la confirmación de que su candidatura a los fondos europeos de desarrollo urbano sostenible e integrado (dusi) era seleccionado por el ministerio de hacienda para recibir diez millones de euros hasta 2020. año y medio después, el gobierno central no ha logrado liberar la asignación definitiva de un solo euro, pese a que el concello de pontevedra (junto con otros beneficiados en la provincia como marín o vigo) sí han elaborado y licitado proyectos vinculados a este plan. a día de hoy, cuatro iniciativas municipales en el casco urbano, presupuestadas en más de 6,5 millones de euros, no pueden ser adjudicadas ni ejecutadas por el retraso de hacienda y fuentes municipales temen que el proceso aún se retraso "bastante más". esas cuatro obras son la rehabilitación y peatonalización del puente de o burgo (2,6 millones), la reforma integral del pabellón multiusos de a xunqueira (2,1 millones), la reurbanización de la rúa de o gorgullón (1,5 millones) y la adecuación de la planta alta del mercado a sus nuevos usos (280.000 euros). de todos esos fondos, europa aportaría alrededor de cuatro millones, pero son por los que aún espera el concello, año y medio después. a mediados del pasado año, casi doce meses después de la concesión oficial de los diez millones de euros, el concello inició la tramitación de los proyectos con el fin de poner activarlos tan pronto como hacienda diera el visto bueno definitivo, una luz verde que se anunció para el último trimestre de 2017 y después para enero de 2018. sin embargo, en abril aún no hay respuesta. mientras tanto, el gobierno local sacó a licitación esas cuatro obras y a día de hoy solo resta su adjudicación. decenas de empresas han presentado sus ofertas, (quince para el mercado y otras tantas para o gorgullón y alrededor de una docena en o burgo) pero el proceso estará paralizado hasta obtener una respuesta oficial del estado. el concello no sabe cuándo podrá adjudicar los trabajos ni dispone de un calendario de ejecución, ya que no es posible dar esos pasos sin el certificado oficial del estado. esto no significa que pontevedra pueda perder los fondos, pero necesita un documento que certifique que los diez millones concedidos en su día están "en caja" y se pueden usar para el fin elegido. en el seno del gobierno local no se ocultaba ayer la "contrariedad" que supone este retraso, que ya es de varios meses, e incluso algunas voces apuntan que "este año puede que no haya obra alguna", si bien se trata de una estimación poco optimista. se da la circunstancia de que esta paralización coincide con un nuevo paso para la ejecución del proyecto estrella del concello con los fondos dusi: la restauración del puente de o burgo. los técnicos municipales analizan desde ayer una docena de ofertas para ejecutar esa reforma y peatonalización. el proyecto tiene un presupuesto de licitación de 2.561.390,91 euros, dividido en dos lotes: el de la obra, por valor de 1.700.598,75 euros (con diez ofertas), y el de la iluminación por valor de 860.792,16 euro y dos empresas aspirantes. la ejecución del lote uno es de 10 meses y la del lote dos es de tres meses. el proyecto estará financiado, cuando se certifique la ayuda, en un 80\% por el fondo europeo de desarrollo regional (feder) y la actuación pretende recuperar la concepción del puente como monumento central de la historia de pontevedra. la obra se divide en tres ámbitos de actuación: el propio puente, cuyo tablero será una plataforma única de losas de piedra para continuar la vía peatonal de 11 metros de ancho (de lado a lado del viaducto) con una iluminación rasante siguiendo el camino de santiago y otra en la nueva barandilla lateral que también será rasante; el entronque con la plaza valentín garcía escudero (cabecera sur), donde se cambiará el paso de cebra con el fin de darle continuidad al trazado peatonal; y el cruce con el paseo domingo fontán y calle juan manuel pintos (cabecera norte), donde se eliminará la rotonda de tráfico rodado, para continuar el carril en los dos sentidos de circulación por ese punto uniendo juan manuel pintos con domingo fontán. se creará una gran plaza de más de 4.000 metros cuadrados con más de 40 árboles, mobiliario, bancos y papeleras. en cuanto a la iluminación se hará en tres fases: una superficial rasante, con puntos azules que marcarán el camino de santiago; otra rasante empotrada en la baranda dirigida hacia el tablero para que los peatones puedan ver por donde pasan, y uno tercera que alumbrará los arcos, tanto por el exterior como por el interior. & 814 & low & Low & Socio-Economic & NA & NA & 2018-04-12 & 2018 & 3 & ECO
Frame & low-medium & Regional & 500-1000 & 0.0141942 & -0.1980960 & -0.1536394 & -0.9382144 & 0.7927531 & 12.6 & 1.2981504 & 0.5794339 & Recipient & Domestic & Domestic & Domestic & Domestic|ECO & Positive\\
Spain & http://www.vozpopuli.com/economia-y-finanzas/recaudacion-Sociedades-objetivo-aprobado-Congreso\_0\_964403891.html & 659 & vozpopuli.com & Private/Non-Public & Online only & National & low = CP mentioned more times but NOT important part of story (mainly about others issues) & Political leverage & Factual & EU + National & No myth & NA & NA & NA & NA & NA & NA & NA & NA & Spain & la recaudación de sociedades no llegará a su objetivo anual a pesar del cambio aprobado en el congreso & 2016-10-20 & fondos estructurales & el pp ha conseguido el apoyo de gran parte de la cámara baja para sacar el cambio en el impuesto de sociedades con el que pretende recaudar 8.300 millones extra este año para cuadrar las cuentas. sin embargo, la recaudación del tributo va tan mal que el cambio no permitirá llegar al objetivo anual que se incluyó en los presupuestos de 2016: casi 25.000 millones. los ingresos de sociedades se quedarán en 21.000 millones. así lo ha señalado el ministro de hacienda, cristóbal montoro en el congreso de los diputados, donde ha reconocido que el cambio supone una corrección de la política que él puso en marcha y que el objetivo es devolver al impuesto a los parámetros de la décima legislatura y a una recaudación próxima a los 21.000 millones, como en 2015 y 2014. en 2017, ejercicio en el que el cambio seguirá en vigor, el impuesto volverá a recaudar unos 21.000 millones. ¿en qué consiste el cambio? en recuperar el mínimo obligatorio en los pagos fraccionados que hacen las empresas suprimido en 2015. la medida afectará a todas las empresas que facturen más de 10 millones de euros, unas 9.000, que tendrán que pagar un mínimo del 23\% en cada pago fraccionado. la banca, en cambio, pagará un 25\%. ésta es una de las medidas estrella que presentó el ministro luis de guindos a bruselas para evitar así multa por el déficit y la congelación de los fondos estructurales. tras un cambio de última hora aprovechando la modificación de la ley de estabilidad, la medida no afectará a canarias, ceuta y melilla, las tres regiones de españa que cuentan con un régimen fiscal especial. esta enmienda ha sido presentada por pp, psoe, ciudadanos y coalición canarias para salvaguardar los excepcionales tipos tributarios existentes en las tres regiones, que no estaban contemplados en la reforma del impuesto. todos los grupos menos podemos, erc y bildu han votado a favor de la enmienda. la modificación de sociedades, según el ministro, no altera la naturaleza del impuesto ni perturba de forma importante el funcionamiento a las empresas, a pesar de que la ceoe se ha quejado enérgicamente de la medida. permanecerá vigente, al menos, hasta que el déficit público baje del 3\% y españa salga del procedimiento de déficit excesivo, aunque en el plan presupuestario enviado a bruselas el gobierno recalcaba que la medida no tenía fecha de caducidad, lo que hace pensar que es posible que venga para quedarse. con esta medida, que el gobierno considera "necesaria" para fortalecer la recuperación que "tanto necesitan" los españoles, el gobierno garantiza la reducción del déficit desde el desfase del 5\% del año pasado al 4,6\% que exige bruselas. y espera además evitar la congelación de los fondos estructurales que actualmente estudian la comisión y el parlamento europeo. las instituciones comunitarias aún no han valorado el plan presupuestario de españa ni el informe de acción efectiva, pero parecen satisfechas tras el endurecimiento del gobierno en el último momento. & 503 & low & Low & Power & NA & NA & 2016-10-20 & 2016 & 2 & POL
Frame & low-medium & National & 500-1000 & 0.0141942 & -0.1980960 & -0.1536394 & -0.9382144 & 0.7927531 & 12.6 & 1.2981504 & 0.5794339 & Recipient & Domestic & European & Mixed & Domestic|POL & Neutral\\
Spain & http://www.farodevigo.es/economia/2017/07/13/empresas-recibiran-3000-euros-ano/1715504.html & 634 & Faro de Vigo & Private/Non-Public & Online and Offline & Regional/Local & low = CP mentioned more times but NOT important part of story (mainly about others issues) & Social awareness/inclusion & Positive & National + Subnational & No myth & NA & NA & NA & NA & NA & NA & NA & NA & Spain & las empresas recibirán 3.000 euros al año durante un trienio por hacer fijos a "ninis" & 2017-07-13 & fondo social europeo & empleo calcula que hasta 600.000 jóvenes podrían beneficiarse del plan - la medida se suma a las bonificaciones para convertir en indefinidos los contratos de formación el gobierno dará 250 euros al mes (3.000 euros anuales) durante tres años a aquellas empresas -incluidos los trabajadores autónomos- que conviertan en indefinidos a los jóvenes beneficiarios del contrato de formación o aprendizaje (una vez hayan cumplido con la duración inicial de un año o la prórroga de seis meses), que hayan percibido la ayuda de acompañamiento de 430 euros mensuales. según fuentes del ministerio de empleo y seguridad social, este incentivo será adicional al incentivo que ya existe por convertir a contratos en indefinidos. en la actualidad, las compañías reciben un incentivo por hacer indefinido a un trabajador de 1.500 euros al año y de 1.800 euros anuales en caso de mujeres. esta medida se integra dentro del sistema nacional de garantía juvenil dirigida a facilitar la inserción laboral de los jóvenes con un bajo nivel formativo y está relacionada con la ayuda directa de activación de 430 euros mensuales vinculada a un contrato de formación para jóvenes que ni estudian ni trabajan y que no tienen un título superior a bachillerato. la ministra de empleo y seguridad social, fátima báñez, anunció ayer durante la sesión de control al gobierno en el congreso que el gobierno se encontraba en negociaciones con la mesa de diálogo social y las comunidades para impulsar esta bonificación "potente", que será de 3.000 euros anuales, durante tres años a favor de las empresas que contraten de manera indefinida a los ninis después de los 18 meses de formación. haz click para ampliar el gráfico según báñez, esta medida persigue que los jóvenes desempleados consigan un trabajo "estable y de calidad". la ministra resaltó que el gobierno elige este contrato de formación y aprendizaje porque los que tienen más baja formación son los más vulnerables. "la diferencia de tasa de paro entre menores de 30 años se duplica entre los que tienen formación y los que no la tienen", indicó. por otro lado, recordó que los 430 euros, financiados por el fondo social europeo, que recibirán los ninis durante su contratación de formación y aprendizaje "no es una ayuda salarial", sino "una ayuda directa" que no se restará de su salario. "es una ayuda directa a la formación. se trata de 430 euros que se sumarán al sueldo que les correspondan según convenio", subrayó. fuentes consultadas calculan que en españa hay entre 500.000 y 600.000 jóvenes con escasa formación (sin la eso terminada) y en situación de desempleo, a quienes la ministra recientemente se refirió como un colectivo al que es urgente formar y capacitar para volver a engancharlo al mercado laboral. en galicia, según datos de la encuesta de población activa, hay más de 26.000 gallegos de entre 16 y 24 años que ni estudian ni trabajan. por otra parte, báñez defendió la reforma laboral llevada a cabo por el partido popular en 2012, la cual "permitió, por primera vez en españa, que la recuperación económica tuviera presente el empleo juvenil". "la novedad de esta recuperación, gracias al compromiso de la sociedad española, es que los jóvenes han encontrado empleo desde el primer momento que se inicia. esto es una muestra del compromiso de empresarios, autónomos e interlocutores sociales", concluyó. & 560 & low & Low & Socio-Economic & NA & NA & 2017-07-13 & 2017 & 2 & ECO
Frame & low-medium & Regional & 500-1000 & 0.0141942 & -0.1980960 & -0.1536394 & -0.9382144 & 0.7927531 & 12.6 & 1.2981504 & 0.5794339 & Recipient & Domestic & Domestic & Domestic & Domestic|ECO & Positive\\
Spain & http://www.abc.es/espana/galicia/abci-varela-marca-despoblacion-como-reto-presente-y-futuro-201602161046\_noticia.html & 633 & ABC TU DIARIO EN ESPAÑOL & Private/Non-Public & Online and Offline & National & high = CP is most important issue in story (can also cover other issues) & Institutional bargaining over funding & Factual & EU + National + Subnational & No myth & NA & NA & NA & NA & NA & NA & NA & NA & Spain & rey varela marca la despoblación como un reto de presente y futuro & 2016-02-16 & política de cohesión & el foro de comunidades por el cambio demográfico, que ayer reunía en castilla y león a las ocho autonomías españolas con balances negativos, sentó este lunes las bases del futuro dictamen sobre este ámbito de actuación que presentarán ante el comité de las regiones de la ue. así lo confirmaba ayer el conselleiro de política social, josé manuel rey varela, quien marcó la recuperación demográfica como "un reto de presente, pero sobre todo, de futuro". rey varela, quien calificó de "muy productiva" la reunión celebrada este lunes en torno a un tema que, dijo, "debe formar parte de la agenda española y europea y que debe estar por encima de las diferencias políticas, advirtió de que "la política de cohesión europea debe estar mucho más estructurada para dar respuesta a los desafíos demográficos". en este sentido, destacó los logros alcanzados en el foro de regiones españolas con desafíos demográficos, (fredd) que galicia liderará a partir de junio y anunció que "la próxima reunión tendrá lugar en este mismo mes en nuestra tierra". especial hincapié hizo en la "inclusión del objetivo transversal del cambio demográfico en el acuerdo de asociación de españa 2014-2020 remitido a la comisión europea o a la comparecencia en el senado, en la comisión de entidades sociales". el responsable autonómico de política social explicó que con este trabajo, iniciado en 2013, se pretende "trasladar a la comisión europea la necesidad de tener en cuenta en el diseño de las políticas las especificidades derivadas del cambio demográfico, demandar al gobierno central un gran pacto de estado en torno a los desafíos demográficos y potenciar la participación en el foro". el conselleiro situó el papel de la xunta en "sensibilizar a la población y dar pasos en la dirección adecuada para construir una galicia mejor y más joven". "tarxeta benvida" según indicó respecto a las actuaciones desarrolladas desde la administración autonómica, "están comenzando a dar sus primeros resultados". entre ellas citó el programa de apoyo a la natalidad (pan) y en este marco la "tarxeta benvida", con la que las familias con niños nacidos en 2016 contarán con una ayuda directa de 100 euros al mes para gastos básicos del bebé. una iniciativa que, según remarcó, cuenta ya con casi 1.000 solicitudes que la administración autonómica está resolviendo. & 381 & high & High & Power & NA & NA & 2016-02-16 & 2016 & 2 & POL
Frame & high-very high & National & <500 & 0.0141942 & -0.1980960 & -0.1536394 & -0.9382144 & 0.7927531 & 12.6 & 1.2981504 & 0.5794339 & Recipient & Domestic & European & Mixed & Domestic|POL & Neutral\\
Spain & https://www.lavozdegalicia.es/noticia/galicia/2018/12/10/ideas-contra-niebla-a-8-pagaran-fondos-id-galicia/0003\_201812G10P6991.htm & 678 & La Voz de Galicia & Private/Non-Public & Online and Offline & Regional/Local & low = CP mentioned more times but NOT important part of story (mainly about others issues) & Research \& innovation & Positive & National + Subnational & No myth & NA & NA & NA & NA & NA & NA & NA & NA & Spain & las ideas contra la niebla en la a-8 se pagarán con fondos i+d para galicia & 2018-12-10 & fondo europeo de desarrollo regional & son 5 millones de partidas feder que el anterior gobierno decidió desviar a un proyecto que es de competencia estatal la letra pequeña del contrato para comenzar a seleccionar las ideas para solventar el problema de la niebla en la a-8, en el tramo cercano a mondoñedo que hay que cerrar al tráfico periódicamente cuando los umbrales de visibilidad son mínimos, refleja que buena parte de la financiación se basará en fondos europeos destinados a la investigación y el desarrollo en galicia. se trata de una decisión del anterior gobierno que se comunicó a la xunta en una de las reuniones que el anterior ministro de fomento, íñigo de la serna, mantuvo con el ejecutivo gallego. en total, la aportación de fondos a este proyecto de competencia estatal que en principio iban a destinarse a programas de investigación gallegos asciende a 4,9 millones de euros. esta es la cantidad que sería imputable a la comunidad autónoma de una licitación que asciende, iva incluido, a 7,26 millones. en realidad, la partida procedente del fondo europeo de desarrollo regional 2014-2020, en el marco del programa operativo fondo tecnológico, servirá para sufragar el lote más importante del contrato, el que se utilizará para construir los prototipos de las ideas destinadas a aislar, eliminar o desplazar las brumas marítimas que se quedan estancadas en el alto de o fiouco, principalmente en los meses de mayo, junio, julio y en menor medida agosto. estos fondos, según ha podido saber la voz, se articularán a través de galicia innovación (gain), encargada de gestionar los fondos feder destinados a la investigación y al conocimiento y dependiente de la consellería de industria. el resto del presupuesto de licitación del contrato de compra pública innovadora (unos 1,8 millones de euros) servirá para la adquisición de sistemas de ayuda a la conducción en condiciones de baja visibilidad. el margen de maniobra de la xunta en este asunto era escaso, porque, aunque los fondos son regionales y se reparten por territorios, son gestionados por la administración del estado, que en vez de destinarlos a proyectos propios de la comunidad autónoma decidió desviarlos a una infraestructura que es de competencia estatal y cuyo cuestionado diseño -sin anticipar los graves problemas que causaría la niebla- es responsabilidad del ministerio de fomento. la xunta utiliza este tipo de financiación feder para conceder ayudas relacionadas con la innovación y la investigación o financiar proyectos innovadores. así, durante el 2018 se financiaron a través de gain ayudas a centros de investigación o a la realización de proyectos de i+d+i en áreas estratégicas para galicia, entre otros asuntos. retraso en el proyecto la búsqueda de una solución para el problema de la niebla en la autovía del cantábrico arrastra un importante retraso, en buena medida debido al parón de la administración durante el año que el gobierno estuvo en funciones. esta demora se refleja en la tardanza que sufrió el proyecto de compra pública innovadora que está ahora en marcha. la abogacía del estado aprobó los pliegos en enero del 2017 y la secretaría de estado del ministerio de fomento no dio vía libre a la contratación hasta marzo de este año. por el momento parece que se están seleccionando seis proyectos de entre los 26 que se presentaron en el ministerio, y alguno más que se pudo aportar de forma sobrevenida dentro de la licitación abierta en primavera. más adelante, los seleccionados deberán demostrar la viabilidad de sus ideas. de esta segunda selección saldrán los proyectos que podrán desarrollar los prototipos que serán probados in situ, en dos emplazamientos próximos a la zona más afectada, probablemente en el verano del 2019, cuando la niebla suele afectar más a la circulación. muy cerca del tramo afectado por la niebla, inaugurado por la ministra ana pastor en febrero del 2014, el ministerio de fomento ha detectado errores graves en la cimentación de uno de los viaductos que se construyeron en esta zona, donde la autovía circula a una altura máxima de 698 metros en el alto de o fiouco. recientemente, el ministerio de fomento sacó a información pública el proyecto de trazado para la construcción de una bajante escalonada bajo el viaducto de vedrós, con el objeto de evitar que el agua de escorrentía socave las pilas del viaducto y genere con el tiempo un problema en las cimentaciones que podría provocar complicaciones en la estabilidad estructural del puente. el viaducto salva un valle en el que no hay ninguna corriente permanente de agua, pero debido a la configuración del relieve en esta zona el agua de lluvia se canaliza bajo el viaducto. la bajante proyectada se ha ideado para canalizar el agua, impidiendo su contacto con el terreno y la acción erosiva que se ha venido produciendo desde la puesta en servicio de los dos últimos tramos de la autovía del cantábrico en galicia. en estos 16 kilómetros de autovía fomento se gastó 192 millones de euros, pero aún tendrá que seguir inyectando dinero público para solucionar el problema de la niebla y estos defectos en la construcción de una de las estructuras del trazado. no obstante, si se logra un sistema antiniebla que funcione correctamente, podrá ser comercializado en otros puntos del mundo en los que también sufren el efecto de la baja visibilidad en el tráfico. & 888 & low & Low & Socio-Economic & NA & NA & 2018-12-10 & 2018 & 3 & ECO
Frame & low-medium & Regional & 500-1000 & 0.0141942 & -0.1980960 & -0.1536394 & -0.9382144 & 0.7927531 & 12.6 & 1.2981504 & 0.5794339 & Recipient & Domestic & Domestic & Domestic & Domestic|ECO & Positive\\
\addlinespace
UK & https://www.chroniclelive.co.uk/business/business-news/county-durham-firm-raises-1m-16186735 & 777 & Chronicle Live & Private/Non-Public & Online and Offline & Regional/Local & very low = CP mentioned once & Economic development & Positive & EU + Subnational & NA & NA & NA & NA & NA & NA & NA & NA & NA & UK & county durham firm raises £1m to aid breast cancer detection & 2019-04-28 & european regional development fund & ibex innovations will create eight new jobs as it develops a new product for use in detecting breast cancer get the biggest business stories by emailsubscribesee our privacy noticemore newslettersthank you for subscribingwe have more newslettersshow mesee our privacy noticecould not subscribe, try again laterinvalid email an innovative north east business which has developed new x-ray technology has secured a further £1m of funding to help it develop a new product for use in detecting breast cancer. new jobs will also be created by ibex innovations, which is based at sedgfield's netpark, following the second funding round in six months which has taken the total raised by the company to more than £5m. this new funding round comes from existing investors ip group plc and the north east venture fund (nevf), supported by the european regional development fund and managed by mercia fund managers. the latest investment, which will create eight new jobs, will help the company to commercialise its existing technology and develop new hardware for use in mammography tests, following its recent success in winning a £700,000 grant from innovate uk for the project. ibex's systems fit to existing x-ray machines and can improve the quality of images while reducing the dose of radiation which the patient receives. the technology has been proven in tests with major multinational x-ray companies and in trials at newcastle hospitals and south tees nhs trusts. neil loxley, the company's chief executive officer, said: "2019 is set to be an exciting year for ibex as we move from the research and development phase to the full commercialisation of our trueview medical imaging product. "we are grateful for the continued support of ip group and the north east venture fund in the latest funding round. we expect that patients will begin to see the benefits of the improved x-ray image quality delivered by trueview as early as 2020." ian wilson, who leads mercia's team in the north east, added: "ibex's technology is ideal for mammography tests as fatty deposits in the body can compromise the image quality of x-rays. the new investment will allow it to develop a new mammography application, and support the wider commercialisation of trueview to secure adoption by major medical imaging companies." jason hobbs, ceo of the north east fund, said: "ibex is a great example of an organisation with huge potential, and one that the north east fund is very pleased to support. the technologies which they are developing will benefit many people and the investment secured will now enable the company to drive forward growth, create new jobs and achieve continued success here in the region." the nevf can invest up to £1m for firms in northumberland, durham and tyne \& wear, particularly those which are engaged in innovation or developing disruptive business models. funding is available to all companies with high growth potential and also pre-start enterprises. & 491 & very low & Low & Socio-Economic & NA & NA & 2019-04-28 & 2019 & 3 & ECO
Frame & v.low & Regional & <500 & -0.7997063 & -1.2431481 & 1.1980777 & 1.4613788 & -0.9363498 & 1.1 & -0.8854249 & -0.3297660 & Payer & Domestic & European & Mixed & Domestic|ECO & Positive\\
UK & http://www.bbc.co.uk/news/uk-wales-politics-37517959 & 742 & BBC & Public & Online only & National & low = CP mentioned more times but NOT important part of story (mainly about others issues) & Institutional bargaining over funding & Positive & EU + National + Subnational & No myth & NA & NA & NA & NA & NA & NA & NA & NA & UK & conservative conference: lifting the lid on brexit thinking - bbc news & 2016-10-01 & structural funds & it is difficult to see past brexit negotiations as the main point of discussion for the conservatives as they gather in birmingham for their annual conference. organisers are not beating around the bush either with the subject top of the agenda on the opening day. among the many questions we would all like the answer to is when the official negotiations will begin, firing the starting gun on the uk's withdrawal. another is whether tariff-free trade is something the prime minister would ultimately be prepared to sacrifice if it meant it was the only way to limit the unrestricted free movement of people. theresa may will inevitably respond to these with the familiar line that she is not prepared to give a running commentary on the negotiations, but conference is bound to lift the lid on some of the cabinet-level thinking. it may also give a sense of the early thinking on what should be done with funding for agriculture and economically-deprived communities, both particularly relevant for wales and both set to be changed fundamentally by brexit. westminster will in future provide the money for agriculture but we do not know whether the cash will be handed over to the welsh government to be distributed via a system designed in cardiff, or whether whitehall would like to have more of a say. if it is the latter then expect an almighty tussle with the welsh government who will resist anything other than complete control. and then there is the replacement for eu structural funds for areas such as the south wales valleys. it is too early to get any indication of whether the uk government would replace this - never mind how - but there have been some strong comments from welsh secretary alun cairns about the need for this kind of aid to undergo major change. elsewhere on the policy front, the interesting question for the party is whether it tries to move into the centre ground in response to jeremy corbyn's dominance of labour, or whether it looks to maintain its traditional ground to the right with policies like the reintroduction of grammar schools in england. the welsh tories are gathering after a disappointing year which saw them go backwards in the assembly elections and lose their status as the main opposition. what made it worse was the fact it followed the party's best result in wales in 30 years at the 2015 general election. there has been an internal review carried out into what went wrong but i am told it is not the kind of report that could put the leader in wales, andrew rt davies, in any difficulty. in fact, it appears he has nothing to worry about having found himself on the right side in the referendum campaign as a high-profile brexiteer. he is also being helped by the lack of any obvious candidates to challenge him for the top job in wales. but this is a party, and a current leadership, that will be defined by brexit - and applies as much in leave-supporting wales as anywhere else. & 520 & low & Low & Power & NA & NA & 2016-10-01 & 2016 & 2 & POL
Frame & low-medium & National & 500-1000 & -0.7997063 & -1.2431481 & 1.1980777 & 1.4613788 & -0.9363498 & 1.1 & -0.8854249 & -0.3297660 & Payer & Domestic & European & Mixed & Domestic|POL & Positive\\
UK & http://www.belfasttelegraph.co.uk/business/news/5m-funding-boost-to-promote-growth-of-small-firms-in-ni-31034736.html & 775 & Belfast Telegraph & Private/Non-Public & Online and Offline & Regional/Local & very low = CP mentioned once & Economic development & Positive & Subnational & No myth & NA & NA & NA & NA & NA & NA & NA & NA & UK & 5m funding boost to promote growth of small... a fund for investing in smes in northern ireland has received an extra 5m to help the... & 2015-03-03 & european regional development fund & a fund for investing in smes in northern ireland has received an extra £5m to help the subject businesses grow. the boost for the co-fund ni has already led to cash injections for autism testing start-up autism biotech, interactive maths firm komodo learning and hardware engines developer titan ic systems. co-fund ni - said to be the most active source of equity funding in northern ireland - was created by invest ni and is managed by clarendon fund managers. the latest boost, which is from invest ni part-financed by the european regional development fund, means its funding has grown to £12.5m. co-fund also announced the appointment of brian cummings as an investment director following the growth in its portfolio. clarendon fund managers finance director neil simms said: "it has been an exceptional year for co-fund ni, with the top-up in funding coming at the end of a busy 12-month period when we invested £2.75m alongside £3.45m of private investment over 18 investment rounds. "this means we have been the most active local source of equity funding in the last year, with a great pipeline of investment opportunities coming in to 2015. "given this backdrop the increase in funding is very timely and, as we expected, brian has hit the ground running, having led our participation in two of the last three investments made in to new portfolio companies." the additional £5.3m can be invested along with private investors such as business angels or their syndicates into eligible smes. co-fund ni provides up to 50\% of equity investment, alongside 50\% from private investors on a deal-by-deal basis. william mcculla, director of corporate finance at invest ni, said: "co-fund ni is a key component of invest ni's wider access to finance strategy, with the objective of stimulating the availability of risk capital to smes across northern ireland. "as co-fund ni has successfully demonstrated, business angels and indeed other sources of private investment, are playing an increasingly important role in the development of young, high potential companies that are seeking equity finance to pursue growth in export markets. "the uplift in funding will allow co-fund ni to continue matching private sector investors and stimulate investment activity, and we are pleased that the local delivery team is expanding." mr cummings said co-fund ni had invested £7.1m in 26 companies, alongside £10.1m in match funding from private investors. "i have been pleasantly surprised with the strength of the existing portfolio, sourced alongside our co-investment partners, which should translate into economic and employment growth for the region." & 442 & very low & Low & Socio-Economic & NA & NA & 2015-03-03 & 2015 & 1 & ECO
Frame & v.low & Regional & <500 & -0.7997063 & -1.2431481 & 1.1980777 & 1.4613788 & -0.9363498 & 1.1 & -0.8854249 & -0.3297660 & Payer & Domestic & Domestic & Domestic & Domestic|ECO & Positive\\
UK & http://www.huffingtonpost.co.uk/entry/nigel-farage-ireland\_uk\_5a75ce1ce4b0905433b49116 & 701 & The Huffington Post UK & Private/Non-Public & Online only & National & low = CP mentioned more times but NOT important part of story (mainly about others issues) & Infrastructure & Positive & EU + National & No myth & Institutional bargaining over funding & Negative & EU + National & 2.Rich countries pay & NA & NA & NA & NA & UK & nigel farage urges ireland to rebel against brussels after eu 'humiliation' & 2018-02-03 & structural funds & he told a sizeable conference of supporters: "you are paying into the european budget and your taoiseach said in strasbourg the other week he is happy for ireland to pay even more into the european budget. leo varadkar said he was "open" to contributing more for things which advance the "european ideal" such as structural funds for central and eastern europe to help them unlock economic potential. farage said: "the perception of the media across europe is that ireland is very pro-european, very servile to the demands of brussels. "i don't think ireland is a pro-eu country, i think the political, media and big businesses in dublin, they are the ones." he suggested it was irish civil servants lining their pockets in brussels who supported the eu. to cheers he added: "they love it. they love it." the crowd almost filled a hall at the rds conference centre and gave farage a standing ovation. he railed against the depiction of brexiteers as xenophobes and bigots. he said: "they go for the man and not for the ball." farage said the european project was not going to work and the euro was unsuited to ireland. he added: "and yet i get told that ireland is a very proud eu country, michael o'leary (ryanair chief executive) told me so, it must be right. "the euro has been bad for ireland but a total catastrophe for countries like greece." he said there was an east/west split, with eurosceptic countries like hungary leading the way. "the eu will not work, it has not worked, it is increasingly unloved by the people of europe," he said. & 276 & low & Low & Socio-Economic & Power & NA & 2018-02-03 & 2018 & 3 & ECO
Frame & low-medium & National & <500 & -0.7997063 & -1.2431481 & 1.1980777 & 1.4613788 & -0.9363498 & 1.1 & -0.8854249 & -0.3297660 & Payer & Domestic & European & Mixed & Domestic|ECO & Positive\\
UK & http://www.mirror.co.uk/news/uk-news/student-branded-filthy-inbreds-vile-9002319 & 773 & Mirror & Private/Non-Public & Online and Offline & National & very low = CP mentioned once & Economic development & Positive & Subnational & No myth & Environment/green/low-carbon & Positive & Subnational & No myth & NA & NA & NA & NA & UK & shameless trolls attack student for setting up sanitary towels business & 2016-10-07 & european regional development fund & "i was shocked at the level of disgust and horror that i received in response" a student who set up a business selling reusable sanitary towels has revealed the shocking level of abuse dished out by faceless internet trolls . sarah callaway has been branded "sick and disgusting" by anonymous online attackers after launching a business selling colourful pads she makes by hand. the 20-year-old says the pads can be washed in a machine and used again but when she first started to advertise her product sarah says she received a torrent of "vile and ignorant" abuse. sarah, from cardiff, told wales online : "i uploaded an ad to a facebook selling page hoping to connect with some new customers and to answer some questions. "i was shocked at the level of disgust and horror that i received in response. "i had so many vile comments saying i was disgusting and unclean. "one particularly nasty comment called me and my partner filthy inbreds." but despite the abuse, sarah and her partner mike pitman, 23, are determined to make their business, house of callaway, a success. to try and remove some of the stigma surrounding reusable sanitary wear sarah has designed her pads in bold colours with patterns and prints to try and minimise obvious staining. the towels are made from terry toweling for absorbency and poppers to hold the pad in place and come in floral, geometric patterns and even zebra print. there are two absorbency levels to accommodate a light or heavy flow and prices vary, starting at £3.50 with the most expensive pad costing £8. sarah said she was inspired to start her business after initially making the sanitary towels for her own use. she said: "i switched from disposable pads to a menstrual cup and cloth pads over a year ago after suffering terribly with rashes from conventional sanitary pads. "straight away i noticed that my skin immediately became less irritated. "no matter what, cloth pads will always be more comfortable for me now so i would never switch back to disposables." sarah hopes that more women will switch to sustainable sanitary wear in the future which could have huge environmental benefits. she said: "the average woman uses 9,600 tampons in her lifetime. i now wash and reuse a relatively small number of cloth pads alongside using a menstrual cup, both of which can last for years. "if more women switched we could make a real dent in the feminine waste that ends up in landfills. "people say that my way is old fashioned but to me it is a sensible, comfortable option that has the potential to impact the future of our planet." sarah and mike received funding from big ideas wales to start their business, the project is part of the welsh government's youth entrepreneurship service and part funded by the european regional development fund which helps young people aged between five and 25 to develop entrepreneurial skills. & 496 & very low & Low & Socio-Economic & Socio-Economic & NA & 2016-10-07 & 2016 & 2 & ECO
Frame & v.low & National & <500 & -0.7997063 & -1.2431481 & 1.1980777 & 1.4613788 & -0.9363498 & 1.1 & -0.8854249 & -0.3297660 & Payer & Domestic & Domestic & Domestic & Domestic|ECO & Positive\\
\addlinespace
UK & http://www.manchestereveningnews.co.uk/in-your-area/stockport-business-innovation-centre-celebrates-8161947 & 772 & Manchester Evening News & Private/Non-Public & Online and Offline & Regional/Local & very low = CP mentioned once & Economic development & Positive & Subnational & No myth & NA & NA & NA & NA & NA & NA & NA & NA & UK & stockport business and innovation centre celebrates first anniversary & 2014-11-24 & european regional development fund & the centre based in broadstone mill began with just 14 businesses and is now home to 34 from a range of industries a centre for start-up businesses is celebrating its first anniversary this week. stockport business and innovation centre (sbic) is home to 34 small businesses. the centre, based in broadstone mill in reddish , began with just 14 businesses a year ago. the council-owned firm, which is part-funded by the european regional development fund, provides flexible office space aimed at start-up businesses looking for their first professional offices. it offers mentoring and 'business incubation' services to help and support businesses as they grow. companies from a range of industries, including human resources, software, web design, creative and professional services, with 132 employees between them, work out of the centre, which is run by oxford innovation on behalf of the council. it is also home to a business lounge which has attracted a large number of visitors. most recent new arrivals include copson ltd, trading as inspired goodbyes, safety smart, greenlight bookkeeping, ultimedia and sav uk. centre manager john booth said: "we knew we had something special to offer fledgling business owners looking for the right environment to build their business. "there is a vibrancy and buzz which is not only helping businesses to flourish, but actually driving trade between them, which is hugely satisfying to see." councillor patrick mcauley, stockport council's executive member for economic development and regeneration, said: "twelve months on sbic is still proving to be a popular and successful destination for a range of businesses. "we want to continue to develop its very particular brand of hosting and mentoring which means so much to these start-up organisations." for more information about sbic contact john booth on 443 4100 or email stockport@oxin.co.uk & 303 & very low & Low & Socio-Economic & NA & NA & 2014-11-24 & 2014 & 1 & ECO
Frame & v.low & Regional & <500 & -0.7997063 & -1.2431481 & 1.1980777 & 1.4613788 & -0.9363498 & 1.1 & -0.8854249 & -0.3297660 & Payer & Domestic & Domestic & Domestic & Domestic|ECO & Positive\\
UK & http://www.walesonline.co.uk/news/local-news/skillscymru-one-month-go-wales-7801236 & 708 & WalesOnline & Private/Non-Public & Online only & Regional/Local & very low = CP mentioned once & Jobs & Positive & Subnational & No myth & NA & NA & NA & NA & NA & NA & NA & NA & UK & skillscymru: one month to go to wales' largest career and skills event & 2014-09-19 & european social fund & young people across wales will have the opportunity to explore a huge range of career options available to them at the skillscymru 2014 event which is to be held at two venues in wales in october with just under one month to go until skillscymru, wales' largest career and skills event takes place, exhibitors are gearing up to inspire people across wales with a huge selection of interactive and engaging skill-based games and activities. from learning how to look like an extra from casualty to coming face to face with a dalek and taking part in a pit-stop tyre-changing challenge, learners will have the opportunity to speak to real-life people working in every industry from manufacturing to medicine, teaching to tourism and catering to caring. representatives from a wide range of exhibitors including the army, royal navy, royal air force, south wales fire and rescue service, british horse-racing authority, nhs, the prince's trust, bbc wales, s4c, admiral, mcdonalds and visit wales, will be challenging learners to a series of engaging and fun time-trials, taste-tests, cooking challenges and aptitude experiments. there will also be photo booths, workshops and competitions for attendees to take part in to help them make the most of their visit. young people across wales will have the opportunity to explore academic and vocational career options available to them while having the chance to try their hand at a new skill, get first-hand expert careers advice and explore fresh approaches to education, work, learning, skills and careers. latest research from the uk commission's employer skills survey show that while there has been a 14 per cent increase in job vacancies in wales over the past two years, skills shortages remain a key issue for employers in wales and the rest of the uk. similarly unemployment figures in wales for over 16's are continuing to fall with just 6.6\% of the population now unemployed. julie james, the deputy minister for skills and technology said: "against a backdrop of falling unemployment rates, skillscymru is a large-scale careers event to help thousands of people of all ages realise the range of work and study options available to them. "events like skillscymru bring a cross section of organisations together from across wales to demonstrate and communicate the breadth of careers and industries that people can try their hand at, while also highlighting the skills that they will need to be successful in the career they choose. "skillscymru promises to be exciting, interesting, interactive and inspirational. investing in skillscymru is just one of the ways in which the welsh government is closing the skills gap and helping people secure a job for life." skillscymru 2014 will be held over four days at two venues - venue cymru in llandudno on 8-9 october and cardiff's motorpoint arena on 22-23 october - giving over 10,000 people the opportunity to find out how to gain new skills and train for the future. these events will be part funded by the european social fund. skillscymru last took place in 2010 at the millennium stadium. over 20,000 visitors attended, from school children and school leavers to university students and adults looking for a career change, and met with 150 exhibitors and organisations. this year's events are being organised by prospects and cazbah, and supported by the welsh government and careers wales. for further information on skillscymru 2014, visit www.skillscymru.co.uk. schools, colleges and other community groups from across wales who would like to visit the events can get in touch with the skillscymru team on 01823 362800 or email georga.blair@prospects.co.uk. any organisations wishing to exhibit can contact gabrielle.mcevans@prospects.co.uk or 01823 362811 . & 630 & very low & Low & Socio-Economic & NA & NA & 2014-09-19 & 2014 & 1 & ECO
Frame & v.low & Regional & 500-1000 & -0.7997063 & -1.2431481 & 1.1980777 & 1.4613788 & -0.9363498 & 1.1 & -0.8854249 & -0.3297660 & Payer & Domestic & Domestic & Domestic & Domestic|ECO & Positive\\
UK & http://www.bbc.co.uk/news/uk-wales-south-east-wales-39029868 & 720 & BBC & Public & Online and Offline & National & very high = CP is most important issue + CP is mentioned in title/headline & Cultural development & Positive & EU + Subnational & No myth & NA & NA & NA & NA & NA & NA & NA & NA & UK & porthcawl maritime centre project gets £2m in eu cash - bbc news & 2017-02-21 & european regional development fund & a new maritime centre on the south wales coast has secured £2.1m in eu funding. science and skills secretary julie james will announce the cash on a visit to porthcawl in bridgend county on tuesday. porthcawl maritime centre will include water sports and exercise facilities, a coastal science centre and an outdoor theatre. it is hoped work will begin in the summer and it will open by 2019. the £2.1m for the development, which is being built on land owned by bridgend council, has come from the european regional development fund. it is expected to cost in excess of £7m and has already secured lottery funding. & 108 & very high & High & Socio-Economic & NA & NA & 2017-02-21 & 2017 & 2 & ECO
Frame & high-very high & National & <500 & -0.7997063 & -1.2431481 & 1.1980777 & 1.4613788 & -0.9363498 & 1.1 & -0.8854249 & -0.3297660 & Payer & Domestic & European & Mixed & Domestic|ECO & Positive\\
UK & http://www.dailyrecord.co.uk/news/local-news/south-lanarkshire-council-clyde-gateway-4889633 & 705 & Daily Record & Private/Non-Public & Online and Offline & Regional/Local & very low = CP mentioned once & Infrastructure & Positive & Subnational & No myth & NA & NA & NA & NA & NA & NA & NA & NA & UK & transport partnership set to improve rutherglen bus stops & 2014-12-31 & european regional development fund & councillors on this month's executive committee agree to the partnership with clyde gateway, strathclyde passenger transport and glasgow city council. south lanarkshire council has agreed to enter into a partnership agreement that will see improvements made to bus stops in rutherglen. councillors on this month's executive committee agree to the partnership with clyde gateway, strathclyde passenger transport and glasgow city council. the arrangement will be part of a sustainable transport network which the council is already part off. it will see the council improve 11 bus stops in rutherglen as well as create five new stops to serve the clyde gateway area. funding of £67,125 has come from the scottish government air quality fund and a further £44,751 will be provided by clyde gateway through the european regional development fund. six stops on glasgow road will be upgraded, along with three on dalmarnock road, two on rutherglen road and one on glasgow road. shawfield road will benefit from two new stops while there will be other new stops on dalmarnock road, queen street and potters way. gordon mackay, head of roads and transportation services, said: "we are delighted to be entering into this agreement with clyde gateway, glasgow city council and strathclyde partnership for transport. "as well as improving 11 bus stops across the rutherglen area, five completely new bus stops will be introduced. "this will give passengers greater choice when planning their journeys. "the work will take place over the coming months and into the next financial year, 2015/16." pat doherty, transport co-ordinator for clyde gateway, said: "the aim of the clyde gateway sustainable transport project is to raise awareness of the benefits of sustainable transport and make people more aware of travel options in the area. "the project involves introducing a range of measures including upgrading existing bus stops and shelters, transport information and cycle facilities. connections between different modes of travel have and are being introduced including linking areas of economic disadvantage with areas of existing and emerging employment. "bus priority measures have also been introduced in and around the clyde gateway area. the above transport infrastructure improvements are already underway and further bus stop and shelter upgrades are due to be completed by early 2015." & 375 & very low & Low & Socio-Economic & NA & NA & 2014-12-31 & 2014 & 1 & ECO
Frame & v.low & Regional & <500 & -0.7997063 & -1.2431481 & 1.1980777 & 1.4613788 & -0.9363498 & 1.1 & -0.8854249 & -0.3297660 & Payer & Domestic & Domestic & Domestic & Domestic|ECO & Positive\\
UK & http://www.walesonline.co.uk/business/farming/new-park-promises-families-chance-13204737 & 763 & WalesOnline & Private/Non-Public & Online only & Regional/Local & very low = CP mentioned once & Economic development & Positive & Subnational & No myth & NA & NA & NA & NA & NA & NA & NA & NA & UK & new farm park opens & 2017-06-19 & european regional development fund & founders geraint robson and caitlyn corless are keen to make education a key element of every visit to the farm get business updates directly to your inbox+ subscribethank you for subscribing!could not subscribe, try again laterinvalid email a couple of animal loving young entrepreneurs have opened the gates to a new educational farm park near bridgend. geraint robson and caitlyn corless, both from cardiff, founded ddraig valley farm park in pencoed, an animal centre designed to allow the public to get close to and learn more about both traditional and exotic species. following the completion of their studies in environmental management and animal studies at bridgend college's pencoed campus, the pair used their entrepreneurial talents to set up ddraig valley farm on the grounds of the college campus, their former stomping ground. ddraig valley farm is home to traditional farm animals such as goats, sheep, chickens, ducks, donkeys and geese, as well as more unusual species such as tortoises, iguanas, terrapins, quails and alpacas. video loading video unavailable click to play tap to play the video will start in 8cancel play now the venue also houses an aviary, which is home to species including diamond doves, java sparrows and canaries, as well as accommodating smaller animals such as rabbits and guinea pigs. geraint and caitlyn are keen to make education a key element of every visit to the farm. caitlyn explained: "we wanted to set up ddraig valley farm park because we were passionate about creating somewhere for families in south wales and beyond to experience animals up close in an environment that is safe, friendly and knowledgeable." while caitlyn tends to the farming side of the business, putting her diploma in animal studies to good use, geraint, who studied environmental management, runs tours of the farm park, showcasing to visitors everything that ddraig valley has to offer. growth plans geraint and caitlyn attribute much of their success to the advice and support they have received along the way, both from big ideas wales but also the support provided by bridgend college's enterprise team, specifically from enterprise champion ruth rowe, who has worked with the couple in achieving their ambition. geraint said: "we could not have got to this point without the amount of guidance we have had over the last four years. at our first big ideas wales bootcamp event in 2014, we received intensive support over the course of a weekend and were able to meet likeminded young entrepreneurs. "from there, we've been lucky enough to have the constant backing and encouragement of individuals like ruth, who has been with us every step of the way since." geraint and caitlyn impressed ruth at the carnegie trust test town event in bridgend, where young entrepreneurs got to put their trading skills to the test. after winning the test town competition, geraint and caitlyn signed the lease on the farm land owned by bridgend college and saw their dreams of running a farm park turned into a reality. ruth rowe, bridgend college's enterprise champion, said: "caitlyn and geraint have seized every opportunity available to them and their hard work has really paid off now they've opened the gates of ddraig valley farm to the public. "they've shown great determination and tenacity throughout and i hope their business continues to grow in the future." and despite only opening the gates to the farm park within the past month, the pair already have expansion plans in their sights and are hoping to increase not only the range of animals they home, but also the size of the farm site too. during their business journey, caitlyn and geraint received guidance and support from big ideas wales, which forms part of the welsh government's business wales service, and is part funded by the european regional development fund. their links with the support service even resulted in award success for the couple, with their venture winning overall best business at big ideas celebrated - a showcase event for big ideas wales which took place at cardiff city stadium recently honouring achievements in youth entrepreneurship. & 686 & very low & Low & Socio-Economic & NA & NA & 2017-06-19 & 2017 & 2 & ECO
Frame & v.low & Regional & 500-1000 & -0.7997063 & -1.2431481 & 1.1980777 & 1.4613788 & -0.9363498 & 1.1 & -0.8854249 & -0.3297660 & Payer & Domestic & Domestic & Domestic & Domestic|ECO & Positive\\
\addlinespace
UK & http://www.belfasttelegraph.co.uk/news/local-national/republic-of-ireland/eu-boost-for-crossborder-economy-30663871.html & 723 & Belfast Telegraph & Private/Non-Public & Online and Offline & Regional/Local & medium = CP is important part of story (alongside other issues) & Territorial cooperation & Positive & EU + National & No myth & NA & NA & NA & NA & NA & NA & NA & NA & UK & eu boost for cross-border economy & 2014-10-14 & structural funds & more than half a billion euro is to be pumped into counties on either side of the irish border, dublin's public spending minister has announced. as part of the republic's budget measures, labour's brendan howlin said new funding has been secured from europe for the next seven years. the money will come from the peace and interreg schemes. the funds will be used to help boost the economy on both sides of the border, increase tourism as well as improve cross-border and cross-community relations. mr howlin, delivering the first tax-cut and spending increase budget in the republic for seven years, said : "the government attaches a high priority to these cross-border programmes, and i am pleased to say that draft programmes have now been submitted to the commission by the government in partnership with the northern ireland executive." mr howlin said a special allocation of 100 million euro for the border, midlands and western region would also form part of an overall 1.2 billion euro package of structural funds up to 2020. "i am committed to ensuring that the benefits of our economic revival are shared across the whole country," he said. & 199 & medium & Medium & Socio-Economic & NA & NA & 2014-10-14 & 2014 & 1 & ECO
Frame & low-medium & Regional & <500 & -0.7997063 & -1.2431481 & 1.1980777 & 1.4613788 & -0.9363498 & 1.1 & -0.8854249 & -0.3297660 & Payer & Domestic & European & Mixed & Domestic|ECO & Positive\\
UK & http://www.derbytelegraph.co.uk/burton/grant-for-burton-businesses-1756294 & 776 & Derby Telegraph & Private/Non-Public & Online and Offline & Regional/Local & low = CP mentioned more times but NOT important part of story (mainly about others issues) & Economic development & Positive & EU + Subnational & No myth & NA & NA & NA & NA & NA & NA & NA & NA & UK & burton businesses urged to take advantage of grants worth up to £167k & 2018-07-10 & european regional development fund & get daily updates directly to your inbox+ subscribesee our privacy noticethank you for subscribing!could not subscribe, try again laterinvalid email a business chief is urging firms struggling during the partial closure of burton bridge to take advantage of grants worth up to £167,000. businesses across east staffordshire are being urged to seek help with the business growth programme, which has £4 million left and will end in september 2018. chris plant, divisional director of burton and district chamber of commerce, is urging small and medium-sized companies to use the money and help create jobs as they grow. a total of 395 applications to the scheme have been approved to date with more than £11 million of grants, of up to £167,000, awarded to businesses across the west midlands, including just under £500,000 to east staffordshire based businesses. mr plant said: "this is a great opportunity for small to medium-sized companies in the region who are looking to grow their business and create jobs to receive a funding boost. "one of the programme's priorities is to secure investment to create jobs and help our businesses to thrive and grow, and this funding is one example of how the council is doing that. "i would strongly urge eligible businesses to get in touch and make the most of this funding opportunity. "at a time when many are affected by the bridge closure, there has never been a better time to make the most of the support on offer." one lane of burton bridge is currently open to traffic accessing burton while strengthening work, lasting 10 weeks, is taking place. video loading video unavailable click to play tap to play the video will start in 8cancel play now some businesses have said footfall is declining with shoppers staying away from the town discouraged by local traffic queues out of the town trying to use the other main exit route of st peter's bridge. birmingham-based dmn logistics is one business that has benefited from the scheme. it provides a wide range of vehicle logistics, from enclosed trailer transportation, vehicle storage to vehicle services and vehicle logistics recruitment. one of the main challenges for dmn logistics was accessing funding and finance to grow the business. through the business growth programme, the firm has received grant funding to the value of £27,000, which supported the growth of the company and helped create new full-time roles. nick chadaway, managing director of dmn logistics, said: "as a result of this funding, we achieved rapid growth that created four full-time roles for us in the business." part-funded by the european regional development fund and supported by east staffordshire borough council, a key partner in the scheme, the business growth programme is designed to strengthen the local economy. the erdf business growth programme is available to businesses in east staffordshire, with grants of up to £167,000. for further information contact the greater birmingham and solihull growth hub by calling 0800 032 3488 or emailing info@gbslepgrowthhub.co.uk & 512 & low & Low & Socio-Economic & NA & NA & 2018-07-10 & 2018 & 3 & ECO
Frame & low-medium & Regional & 500-1000 & -0.7997063 & -1.2431481 & 1.1980777 & 1.4613788 & -0.9363498 & 1.1 & -0.8854249 & -0.3297660 & Payer & Domestic & European & Mixed & Domestic|ECO & Positive\\
UK & http://www.walesonline.co.uk/news/wales-news/cardigan-castle-open-doors-april-8789395 & 714 & WalesOnline & Private/Non-Public & Online only & Regional/Local & very low = CP mentioned once & Cultural heritage & Positive & National + Subnational & No myth & NA & NA & NA & NA & NA & NA & NA & NA & UK & transformed cardigan castle to open its doors on april 15, over a decade after regeneration plans were first drawn up & 2015-03-06 & european regional development fund & regenerated cardigan castle will open its doors to the public on april 15 it has been announced, exactly 12 years after the historic site came into public ownership. after more than a decade of lobbying and fundraising, led by the cadwgan building preservation trust (cbpt) and its devoted volunteers, the 900-year-old site has undergone a £12m restoration project, which began in 2011. the project has seen the site transformed from a derelict building into a high-end heritage attraction, luxury accommodation, restaurant and wedding and events venue. 'exciting programme of events' castle director cris tomos said: "we're busy putting the finishing touches in place and are looking forward to opening ahead of the may bank holidays and summer school break, which are, of course, key trading times. "and there's plenty to attract visitors to the castle over the summer months, with an exciting programme of events, from open-air theatre to music concerts, and fascinating exhibitions which tell the story of cardigan, the castle, the people who lived here, and its role as the birthplace of the eisteddfod." rich history the castle was built in the 12th century by the norman gilbert de clare, and passed to his son. it was captured by rhys ap gruffydd in 1136 and passed back and forth between welsh and norman hands over the next century. the site also famously hosted wales' first eisteddfod in 1176. it was claimed by edward i at the end of the 13th century, suffered damage during the english civil war and lay uninhabited until the early 19th century when a private property was built on site. 'a top tourist destination' a spokesman for the cadwgan building preservation trust said: "it is hoped the castle will become one of west wales' top tourist destinations, attracting at least 33,000 visitors in the first year and bringing significant economic benefit to cardigan and the wider area." related: 43 pictures that prove welsh castles are the coolest thing history ever did the trust received more than £6m from the heritage lottery fund (hlf) and £4.3m from the european regional development fund (erdf) through welsh government, to restore the grade i listed building, while creating new, sustainable uses for the historic site. further funding was received through a communities asset transfer grant with support from the welsh government, cadw, big lottery fund, the uk association of preservation trusts, the architectural heritage fund, ceredigion county council, cardigan town council and the prince's regeneration trust, as well as through community fundraising. 3,000 daffodils the redevelopment of the site has included the recreation of the paths and lawns of the regency gardens, fitting of a floor-to-ceiling glass restaurant with panoramic views over the river teifi and the restoration of the whalebone arch - a "must have" feature for early 19th century gardens. related: hay castle to undergo major £500,000 restoration project a total of 3,000 daffodils have been planted and more than 1,700 rolls of turf laid as part of the work, which is also aimed at making the ancient building a spectacular wedding and events venue. & 523 & very low & Low & Socio-Economic & NA & NA & 2015-03-06 & 2015 & 1 & ECO
Frame & v.low & Regional & 500-1000 & -0.7997063 & -1.2431481 & 1.1980777 & 1.4613788 & -0.9363498 & 1.1 & -0.8854249 & -0.3297660 & Payer & Domestic & Domestic & Domestic & Domestic|ECO & Positive\\
UK & http://www.manchestereveningnews.co.uk/news/greater-manchester-news/latvian-workers-discovered-living-portable-7610187 & 789 & Manchester Evening News & Private/Non-Public & Online and Offline & Regional/Local & very low = CP mentioned once & Public services & Negative & National + Other country & No myth & NA & NA & NA & NA & NA & NA & NA & NA & UK & latvian workers discovered living in portable cabins on car park of £5m government regeneration site & 2014-08-14 & european regional development fund & the homes and communities agency is considering legal action against developers northern group over the situation at jactin house in ancoats building workers living in portable cabins on a flagship £5m redevelopment project are to be removed after government officials slammed developers for breaching the terms of their contract. the homes and communities agency (hca) said it is considering legal action against manchester developers northern group after the m.e.n made it aware of the situation at jactin house in ancoats. local mp lucy powell has also slammed the project, saying 'workers shouldn't have to live in these conditions'. six latvian workers have been observed living in portable cabins in the car park of the derelict site on hood street over the last four weeks. the hca-owned former mill workers' hostel is undergoing a multi-million pound transformation into office space for new businesses - funded by the european regional development fund. a hca spokesman said: "the homes and communities agency strongly opposes the unsanctioned use of jactin house as a temporary place of residency during the construction period, which breaches the terms of northern group's lease. "we are taking action to put a stop to this, and are reviewing our legal options." after being told of the hca's response northern group said it would take immediate action and be removing the workers by monday. residents have complained of late-night drinking and anti-social behaviour on the site. manchester central mp lucy powell said she would be seeking answers as to how the situation had been allowed on a government site. she said: "workers shouldn't have to live in these conditions and residents shouldn't have to put up with this disturbance after hours. "this shouldn't happen on any site, let alone one funded through a government agency. "i'll be getting in touch with the agencies involved to get to the bottom of this. action needs to be taken to rectify this situation for all concerned." coun bernard priest said: "the city council does not encourage, nor does it condone developers allowing their contractors to live on-site." work on the site is expected to be completed next summer. once completed the five-storey building will house office space as well as community facilities, including a gym and a nursery. 'we'll find new accommodation for workers' nathan ezair, director of ancoats-based developer northern group, said the company would find new accommodation for workers living on the jactin house site. he said: "one of the subcontractor teams working on this site are living in health and safety executive approved accommodation supplied by oldham-based company bunkabin. this kind of accommodation is used across europe and very similar to that used by workers on london's olympic park. "our subcontractors have expressed no dissatisfaction with the standard of accommodation whatsoever. "however, it has been brought to our attention that the hca believes the use of these cabins to contravene the terms of our lease and we will act on this immediately and find new accommodation for the six workers affected. "we would like to add that late night noise on our site is unacceptable to us and all complaints are dealt with immediately." & 541 & very low & Low & Socio-Economic & NA & NA & 2014-08-14 & 2014 & 1 & ECO
Frame & v.low & Regional & 500-1000 & -0.7997063 & -1.2431481 & 1.1980777 & 1.4613788 & -0.9363498 & 1.1 & -0.8854249 & -0.3297660 & Payer & Domestic & European & Mixed & Domestic|ECO & Negative\\
UK & http://www.bbc.co.uk/news/uk-wales-politics-35708734 & 746 & BBC & Public & Online only & National & medium = CP is important part of story (alongside other issues) & Institutional bargaining over funding & Balanced & EU + National + Subnational & No myth & NA & NA & NA & NA & NA & NA & NA & NA & UK & andrew rt davies insists aid cash would be available after eu exit - bbc news & 2016-03-02 & structural funds & wales would still get economic aid for its poorest areas even if britain votes to leave the european union, welsh tory leader andrew rt davies has said. the eu is providing £1.8bn to wales between 2014-2020 to help economic growth. speaking on the wales report with huw edwards, mr davies said: "i can guarantee that a uk government would make sure that money would be re-distributed around the regions of the uk, otherwise it would be failing in its remit to deliver help and support to the nation it is elected to govern. "frankly we cannot continue with operation fear, driving people in to the ballot box because you are scaring them into voting one way." a referendum will be held on 23 june on whether britain should leave the eu, or remain a member on new terms negotiated by mr cameron. economic aid is likely to feature heavily in the campaign. because west wales and the valleys' economic output is much lower than the eu average, wales receives far more in so-called "structural funds" than scotland or northern ireland. supporters say the money helps bring in investment in communities which may otherwise see low levels of government support. but opponents say the money is effectively re-cycled from britain's annual contribution to the eu budget, and would be better administered direct from london or cardiff. mr davies' comments come after the prime minister, asked on bbc wales today if he could make up the shortfall in eu aid for places like wales after an exit, said: "i think you can't be certain about that. "we know, between 2014 and 2020, in the european budget is £1.8bn for wales, vital money for economic development and important projects." he said: "in those circumstances, of course, the united kingdom government would always want to do everything it could for all the different parts of the united kingdom, but you can't guarantee these things, because we might be in quite difficult economic circumstances." mr cameron, asked about mr davies's decision to campaign on the opposite side of the referendum campaign, said it was "always disappointing when someone doesn't back your view". & 368 & medium & Medium & Power & NA & NA & 2016-03-02 & 2016 & 2 & POL
Frame & low-medium & National & <500 & -0.7997063 & -1.2431481 & 1.1980777 & 1.4613788 & -0.9363498 & 1.1 & -0.8854249 & -0.3297660 & Payer & Domestic & European & Mixed & Domestic|POL & Neutral\\
\addlinespace
UK & http://www.telegraph.co.uk/news/worldnews/europe/germany/11746707/No-bailout-can-save-Greece-from-itself-or-from-the-German-voters.html & 726 & The Telegraph & Private/Non-Public & Online and Offline & National & very low = CP mentioned once & Financial burden & Negative & EU + Other country & NA & NA & NA & NA & NA & NA & NA & NA & NA & UK & no bailout can save greece from itself - or from the german voters & 2015-07-17 & cohesion fund & in 1991, the kohl government introduced the "solidarity tax", effectively an earmarked tax on incomes, capital gains and profits to finance fiscal transfers from west to east germany. the president of the mannheim-based centre for european economic research (zew), clemens fuest, has recently proposed to raise the solidarity tax rate from 5.5\% to 8\%, and to use the additional revenue to fund fiscal transfers to greece, in lieu of the recurring bailouts and rescue packages. fuest's logic is simple: emergency 'loans' to greece are not really loans. much of it will be never be repaid; it will not be officially written off, but it will be decimated through all kinds of accountancy tricks. so either way, there will be a net transfer of resources from germany to greece, and according to fuest's back-of-the-envelope calculation, a 2.5 percentage point hike in the solidarity tax would cost the german taxpayer no more than the policies currently pursued. it would merely replace a roundabout, highly complex transfer mechanism with a direct and simple one. it is the sort of proposal only an economist could make. it would be transparent, it would be honest, and it would be far more efficient than what is currently going on. and that is exactly why it has exactly zero chance of becoming a reality. instead, the current course of muddling through will continue. today, the bundestag has ratified the third greek bailout package with an overwhelming majority. the social democrats are in favour of it, the greens are in favour, and a few grumbling backbenchers aside, the christian democrats are in favour of it too. but the general population is far less enthusiastic. according to the latest survey, germans are roughly evenly split on the issue, with 46\% supporting the third bailout package, and 49\% opposing it. now imagine fuest's proposal were implemented. the tax would immediately become known as the "greece tax", or something to that effect. month after month, people would see on their payslip how much of their income is being transferred to greece, euro by euro, cent by cent. that brutal transparency would tip the balance towards germany's own "oxi" camp. the solidarity tax was never hugely popular in west germany, but it did not cause much discontent either. most west germans are prepared to subsidise east germany, just as most people in london and the south east of england are prepared to subsidise the less prosperous parts of the uk. but unless very special circumstances apply (say, a country has been hit by a natural disaster), that sense of solidarity does not automatically extend to other countries, and this is one of the fundamental design flaws of the eurozone. pushing transfer levels above what most people feel comfortable with is a surefire way to create resentment and bitterness. before the euro crisis, greece was already a long-term net recipient of transfers from the european regional development fund, the european social fund and the european cohesion fund. this seemed to be roughly the level of transfers that taxpayers in the net contributor countries were prepared to pay (although the funding streams were never particularly transparent, so we cannot even know that for sure). but a currency union of countries that are so economically far apart requires transfer levels an order of magnitude above that. the founders of the euro project hoped that a common currency would help to foster a pan-european identity: west germans would be as naturally prepared to pool resources and share sovereignty with greeks as they are with east germans. that, frankly, has not happened, but in germany and elsewhere, political elites are not willing to admit that to themselves. so they insist on dragging on, whilst trying to hide the true cost of their policies from their electorates. * aep: greece should seize germany's botched offer of a velvet grexit * halle institute's reint gropp: the greek deal is in germany's best interest when german politicians try to present themselves as hardliners in negotiations with greece, don't take it at face value. the tough-on-greece rhetoric is really just a signalling exercise to assuage voters back home. yet the greek crisis will not be solved by this this package, nor by the next, because the country's structural problems are far deeper than that. the greek economy is simply not productive enough to sustain incomes and government transfers anywhere near the pre-crisis level, and that cannot be solved by a few parametric changes. greece's economy needs a fundamental overhaul, and this overhaul will never happen as long as it is perceived as an imposition from outside. greece needs to find its own way towards reform, which means that northern european politicians need to stop meddling - but they cannot stop meddling as long as greece is a recipient of transfers that most northern european voters are not really prepared to pay for. that is the fundamental dilemma behind this never-ending greek tragedy, and as long as greece remains within the eurozone, it will continue in one way or another. dr kristian niemietz is senior research fellow at the institute of economic affairs & 870 & very low & Low & Values & NA & NA & 2015-07-17 & 2015 & 1 & ECO
Frame & v.low & National & 500-1000 & -0.7997063 & -1.2431481 & 1.1980777 & 1.4613788 & -0.9363498 & 1.1 & -0.8854249 & -0.3297660 & Payer & European & European & European & European|ECO & Negative\\
UK & http://www.theguardian.com/society/2016/jun/21/brexit-widen-north-south-divide-poorest-areas-lose-most-eu & 699 & The Guardian & Private/Non-Public & Online and Offline & National & high = CP is most important issue in story (can also cover other issues) & Economic development & Positive & EU + National + Subnational & No myth & Institutional bargaining over funding & Positive & EU + National & No myth & NA & NA & NA & NA & UK & brexit will widen the north-south divide as poorest areas stand to lose most | peter hetherington & 2016-06-21 & structural funds & the ultimate irony at the heart of the debate is that the most eurosceptic regions - the north, the midlands and cornwall - are the biggest recipients of eu funds in a uk with such glaring social and economic disparities europe has become a vital source of funding for our poorest regions and nations. you wouldn't know it in the current, fevered climate, where evidence, like caution, has been thrown to the wind. brexit campaigners, in their crazed world of self-deception, airily dismiss a looming black hole in regional economies caused by eu withdrawal as yet another scare story dreamt up by the "experts" so loathed by justice secretary michael gove. yet, from cornwall to north-east england, multi-billion pound support packages for new jobs, and employment and social programmes, have proved a lifeline to areas beset by the collapse of traditional industries. question the outers on how - say - the current £8.5bn, seven-year programme largely from the eu's regional development and social funds will be maintained, let alone another £3bn annually in agricultural and rural support, and their response is as predictable as it is incredible: why, from the mythical £350m we give weekly to the eu, stupid. as the university of sheffield's political economy research institute recently reported, that £8.5bn - which doubles with match funding from the government and other sources - has to be seen in the context of severe budget cuts hammering local government through "austerity politics". thus it has achieved greater importance because seven-year funding rounds can allow councils to plan with greater certainty. while the uk's financial contribution to the eu has been widely debated, considerably less attention has been paid to funding that the home nations receive from the eu's regional and social programmes - so-called "structural funds" - and how it is distributed for improving, say, employment and educational opportunities up to 2020 (in the current round). areas with incomes per head below 75\% of the eu average - west wales, the welsh valleys and cornwall, for instance - benefit the most, while 11 so-called "transition regions" with incomes between 75\% and 90\% of the eu average (including cumbria, devon, the scottish highlands and islands, northern ireland, merseyside, south yorkshire and durham) also get a lower level of eu support. the sheffield researchers calculate that from 2007, wales has gained almost 37,000 eu-financed jobs; scotland 44,000 and the north of england 70,000. their report warns: "if the uk were to leave the eu ... evidence suggests that the loss of structural funds would disproportionately affect wales, northern ireland, the south-west and the north-east (with) a significant impact on job creation and business activity." as craig berry, the deputy director of the sheffield research institute, says, the economic status quo works neither in favour of the uk's poorest regions - which face the most uncertainty in the event of brexit - nor for the government's much-vaunted "northern powerhouse" agenda. this, he says, would be incompatible with eu withdrawal through the loss of vital european funding and restricted access to continental export markets. but there's an ultimate irony at the heart of the eu debate, underlined last week in further research by the leading economic geography professor philip mccann, of the university of groningen in the netherlands. while support for continued membership is probably strongest in london and its rich commuter belt, the capital gains the least from the eu because it has a wider global trading base. conversely, the most eurosceptic counties and regions, such as the north, midlands and cornwall- are both larger recipients of eu funds and often far more reliant on exports to the single market. the north-east, for instance, is the only region with a balance of payments surplus, partly through car exports. mccann is clear on one point: a likely "asymmetric regional shock" from brexit would make the london and south-east versus the rest of the country divide even greater than it isn now "which even now is close to being an all-time high", in rural as well as urban areas. in short, because of the unequal social and economic makeup of the uk, , weaker regions will suffer not only from loss of eu structural funding but also from the loss of vital export markets. and with no "equalisation formula" agreed to address regional inequalities, he believes that the government's much-trumpeted devolution agenda - handing power to city-regions such as greater manchester to help address those disparities - will be undermined before it takes off. & 763 & high & High & Socio-Economic & Power & NA & 2016-06-21 & 2016 & 2 & ECO
Frame & high-very high & National & 500-1000 & -0.7997063 & -1.2431481 & 1.1980777 & 1.4613788 & -0.9363498 & 1.1 & -0.8854249 & -0.3297660 & Payer & Domestic & European & Mixed & Domestic|ECO & Positive\\
UK & http://www.devonlive.com/news/devon-news/history-forgotten-island-prepares-demolished-973567 & 700 & Exeter Express and Echo & Private/Non-Public & Online only & Regional/Local & very low = CP mentioned once & Public services & Positive & National + Subnational & No myth & Civic participation/collaboration & Positive & National + Subnational & No myth & NA & NA & NA & NA & UK & the history of a 'forgotten island' as it prepares to be demolished & 2017-12-24 & european regional development fund & get daily updates directly to your inbox+ subscribethank you for subscribing!could not subscribe, try again laterinvalid email it was once home to britain's biggest naval community, but now residents of the barne barton estate can't wait for it to be torn down so they can leave. the estate in plymouth has had its fair share of trials and tribulations over the years since its construction in the 1960s. locally referred to as the 'forgotten island', the former mod estate used to provide accommodation for those serving in hmnb devonport and their families, and other warships stationed there. but as numbers in the royal navy reduced, sections of the estate were sold off one-by-one and now it is predominantly owned by different housing associations, the plymouth herald reports. in november plans went on display to showcase the £22million transformation of part of the estate, which will be torn down and completely rebuilt. clarion is currently the main landlord in barne barton and manages 228 two-bedroom flats for affordable rent, as well as a number of houses. but the flats in and around wilkinson road now suffer from structural issues, damp, and plumbing issues, and as a result have high energy bill for residents - many of who are single parents with young children. residents say they are living in "unbelievable" conditions - and they can't wait to leave. but how did it ever come to this? the forgotten island: its construction and demise 1960s the construction of barne barton begins by the mod. barne barton in the sixteenth century was a farm (the words 'barne barton' actually mean 'barne farm') and produced essentials such as meat, wheat, grain and potatoes. its situation by the tamar enabled its produce to be transported up stream to old settlements such as tavistock. after the estate's completion it provided accommodation for those serving in hmnb devonport and their families, and other warships stationed there. donna howard once lived on the estate as a naval wife and has "good and bad memories" of living there. she said: "in the january of 1987, the mains in the roof space froze and burst. i was living in one of the middle floor flats with a three-month-old baby. the navy said i could use a vacant flat during the day but had to return home to sleep!" 1990s fotonow, which is running a social history project called bridging barne barton: the island stories, said during a crisis in the supply of social housing in the 1990s, the mod sold off its married quarters (military housing) nationwide. they said: "in barne barton this created an opportunity for a new pool of affordable places to live at a time when local authorities around the country were effectively barred from building good-quality public housing." the island stories is focused on exploring and capturing the history of the community and is supported by heritage lottery fund, north yard community trust and affinity sutton homes (part of clarion housing group). you can find out more here . the belief is held by long standing residents in the area that this transition from military ownership to civilian ownership was the start in the decline of the area. 1999 a 15 ft-hgih mural was created at the top of poole park road and savage road. the mural was one of the many action 2000 projects which got under way across the city. 1998 having little to no amenities there, work got underway to create a park at kit hill crescent, which saw the installation of a shelter, kickabout area and play equipment. 2002 major plans to transform a derelict area of barne barton and create an urban village run entirely by local residents took a huge leap forward. proposals to revamp kinterbury square in barne barton had been delayed for more than seven years as legal wrangling and discussions over the scheme raged on. now work to regenerate the mod land and create shops, community facilities and small businesses looked finally to get under way with the appointment of a project manager. the first phase of the ambitious scheme got under way in spring that year with £185,000 from the single regeneration budget and the european regional development fund to pay for work over the next two years. under phase one, ownership of the land around kinterbury square was passed from the mod to residents themselves, land would then be developed for building. 2002 members of barne barton community action trust struck a deal to buy the community centre they had been based in, for just £1. it was planned the community centre, in kit hill crescent, would become a one-stop-shop scheme offering local people advice on a range of matters, a credit union, an employment point, ict courses, luncheon clubs and children's summer play schemes. 2004 the tamar view community centre officially opened new facilities to the public, ensuring it serves almost everyone in the community. the next projects planned were a shop and a nursery, and work was due to begin in january 2005. 2005 builders started work on a new £410,000 convenience store. the venture was expected to create 14 part-time jobs and would be run by the plymouth and south west co-operative society, but it would operate on a profit-share basis. the co-op store remains today. 2010 construction work on a barne barton housing scheme began after it received a £5million funding boost. the homes and communities agency grant supported the development of around 90 family homes in kinterbury square and foulston avenue. the project, delivered by devon and cornwall housing association (dcha), would be a mixture of two-, three- and four- bedroom houses. 2011 plans for a controversial waste incinerator were given the go-ahead. councillors voted 7-5 in favour of the plans for devonport dockyard's north yard following a heated six-hour debate. mvv environment devonport limited was now poised to begin building the plant in 2012 - to the dismay of residents living in its shadow. 2013 the final section of the chimney at plymouth's new incinerator was put in place. 2017 clarion announced it would be transforming barne barton in a £22 million scheme. the ambitious plans have been developed over a number of years through close working with residents, the local community and plymouth city council. plans for the future of the estate include replacing the existing flats with a combination of flats and houses and introducing opportunities for low cost home ownership in addition to homes for affordable rent. the masterplan, which is being created by exeter based clifton emery design, proposes to completely change the existing layout of the estate to create improved views and people friendly streets. the detailed design work will be undertaken by rh partnership architects. & 1144 & very low & Low & Socio-Economic & Socio-Economic & NA & 2017-12-24 & 2017 & 2 & ECO
Frame & v.low & Regional & +1000 & -0.7997063 & -1.2431481 & 1.1980777 & 1.4613788 & -0.9363498 & 1.1 & -0.8854249 & -0.3297660 & Payer & Domestic & Domestic & Domestic & Domestic|ECO & Positive\\
UK & https://www.chroniclelive.co.uk/business/business-news/film-company-second-draft-puts-15090821 & 697 & Chronicle Live & Private/Non-Public & Online and Offline & Regional/Local & very low = CP mentioned once & Cultural development & Positive & National + Subnational & No myth & Jobs & Positive & National + Subnational & No myth & NA & NA & NA & NA & UK & film company second draft puts the region in the frame to attract investment & 2018-08-30 & european regional development fund & the creators of sunderland's official film for the city of culture 2021 bid met thousands of miles away but are now based in the region get business updates directly to your inboxsubscribesee our privacy noticethank you for subscribing!could not subscribe, try again laterinvalid email a pair of entrepreneurs who met thousands of miles away from the north east have been tasked with showcasing the region in a bid to attract investment and jobs. mark stuart bell and glen colledge, 26, first connected when they met 8,000 miles away on the falkland islands. mr bell, who studied for a ma in journalism at the university of sunderland, had been researching a documentary and needed a collaborator and a guide. at the time, mr colledge was a cameraman and editor on the island's tv station. six months later the filmmakers were working together back in the uk, before taking the big leap to launch their video production company, second draft. now second draft - which received widespread acclaim for producing the official film for sunderland's 2021 city of culture bid - has moved into office space at the north east business and innovation centre (bic) in sunderland. the two-year-old business, which was supported with its business growth by sunderland city council, has also won contracts with invest north east england and the north east local enterprise partnership. the expansion and contract wins are both hugely significant for second draft according to the firm's co-founders and they are now set to produce six short films in a bid to provide a boost to the region's fortunes. mr bell said: "after building various strong relationships in sunderland, we wanted to stay in the area and continue to work with the city's business talent and showcase what's great about sunderland and the wider north east region. "and the move to the bic presents us with new opportunities to collaborate due to the variety of businesses it has on-site and the regular networking events it hosts. "having the permanent office also gives us the opportunity to expand as a business as it will enable us to take on additional projects and recruit staff." mr colledge said: "as filmmakers, our eternal goal is to share things that we find interesting or inspiring, and in both of these projects we get to find new creative and exciting ways to showcase what makes the north east special. "our goal with every film is to help our clients make meaningful connections with their audience to reflect their passion and to help people understand the emotion that drives them so we can tell their unique story. "when videos are made in this way they often have a high level of shareability - particularly on social media which has contributed massively to the recent rise of video marketing." coun graeme miller, leader of sunderland city council, said: "second draft is a great business, and the video they made for our city of culture bid is testimony to how passionate they are about sunderland. "moving to the bic has proved to be the perfect business environment for them to network with fellow businesses in the city and grow even further. "they're a talented group of people and we're delighted to be able to offer our support." the pair moved to the bic from the enterprise place at the university's hope street xchange building. the enterprise place is funded through the european regional development fund (erdf) and available to all students and alumni of the university of sunderland. the enterprise place offers office space and facilities, start-up and growth courses, and specialist advisers offering support and guidance all aimed at helping people turn a business idea into reality. glen added: "the support we received from the enterprise place was fantastic, it gave us the foundation we needed as we were starting out on our venture. & 655 & very low & Low & Socio-Economic & Socio-Economic & NA & 2018-08-30 & 2018 & 3 & ECO
Frame & v.low & Regional & 500-1000 & -0.7997063 & -1.2431481 & 1.1980777 & 1.4613788 & -0.9363498 & 1.1 & -0.8854249 & -0.3297660 & Payer & Domestic & Domestic & Domestic & Domestic|ECO & Positive\\
UK & http://www.bbc.co.uk/news/uk-wales-31418603\#sa-ns\_mchannel\%3Drss\%26ns\_source\%3DPublicRSS20-sa & 791 & BBC & Public & Online only & National & very low = CP mentioned once & Infrastructure & Positive & EU + Subnational & No myth & NA & NA & NA & NA & NA & NA & NA & NA & UK & transport, broadband key to region & 2015-02-12 & structural funds & a metro transport network and faster broadband are top of the shopping list for the new cardiff capital region. it has laid out its vision for the next 15 years in driving south wales forward. public transport is the priority according to the "powering the welsh economy" report. it needs to be "high-performing, seamless and efficient" with too much reliance at the moment on cars. the region needs to build on the planned electrification of great western and valleys rail lines. the network - which could be built by 2030 - would involve trains, buses and city trams and could cost over £2bn if it goes ahead. board chairman roger lewis says the project could attract hundreds of millions of pounds worth of european union structural funds, which can be used to develop the metro relatively quickly. another key area is faster, reliable and cheap broadband. "this is now a 'must have' for any city region with global aspirations and is critical for economic growth," says the report. it points to the infrastructure in place and being developed and says the region now needs to capitalise on its "significant fibre backbone". as for mobile coverage it points to black spots and says continued investment is needed to eliminate these and ensure 3g and new 4g coverage is available from all the major operators across the region. in its 41-page report the region's advisory board sets out what it thinks needs to happen next. mr lewis says the city region needs strong leadership, a clear and agreed vision, power, the backing of key stakeholders and importantly cross-party support. with cardiff also part of the great western cities and cores cities groups, he said it was now about bringing those strands together. economy minister edwina hart, who set up the cardiff capital region and swansea bay city region boards, argues all of wales will benefit if they succeed in attracting jobs and investment. she denies that the plethora of city organisations are "talking shops" . when the welsh government set up advisory boards for the two city regions they were relatively new concepts. it's incredible how quickly the landscape has changed in the uk. the city regions in the north of england, and the wider partnership under the heading one north have already benefited from many millions of extra cash from the treasury, for the projects they have identified as being key to economic growth. wales needs to act fast to not be left behind. mrs hart will now consider cardiff's capital regions strategy and will decide what will happen next - including whether it gets the powers and budget to match its ambitions. & 444 & very low & Low & Socio-Economic & NA & NA & 2015-02-12 & 2015 & 1 & ECO
Frame & v.low & National & <500 & -0.7997063 & -1.2431481 & 1.1980777 & 1.4613788 & -0.9363498 & 1.1 & -0.8854249 & -0.3297660 & Payer & Domestic & European & Mixed & Domestic|ECO & Positive\\
\addlinespace
UK & http://metro.co.uk/2016/06/23/leaving-the-eu-could-put-game-of-thrones-in-jeopardy-5961815/ & 717 & Metro & Private/Non-Public & Online and Offline & National & very low = CP mentioned once & Cultural development & Positive & EU + National & No myth & NA & NA & NA & NA & NA & NA & NA & NA & UK & leaving the eu could put game of thrones in jeopardy & 2016-06-23 & european regional development fund & this site uses cookies. by continuing, your consent is assumed. learn more leaving the eu could put game of thrones in jeopardy hanna flint for metro.co.ukthursday 23 jun 2016 9:52 am if you're a game of thrones and still undecided how to vote in the eu referendum then this might sway you into the remain camp. because if brexit gets the go ahead it could threaten the production of future series of the hbo show. winter would definitely be coming if that happened. more: the actor who plays wun wun played these game of thrones characters too as you may know, key filming takes place on location in northern ireland, which at the moment receives money from the eu's european regional development fund. if we leave the eu, northern ireland wouldn't be entitled to that dosh anymore, causing the loss of jobs for thousands of people as well as potentially affecting the quality of episodes when it comes to those epic scenes like in the battle of the bastards. seeing wun wun bash down winterfell's door don't come cheap. it's not just game of thrones that could be affected but the uk's film and tv industry in general, with foreign productions looking to places with programs set up to fund the making of shows and films. hasn't jon snow been through enough? more: from david beckham to elizabeth hurley: this is how 86 celebrities are voting in the eu referendum & 251 & very low & Low & Socio-Economic & NA & NA & 2016-06-23 & 2016 & 2 & ECO
Frame & v.low & National & <500 & -0.7997063 & -1.2431481 & 1.1980777 & 1.4613788 & -0.9363498 & 1.1 & -0.8854249 & -0.3297660 & Payer & Domestic & European & Mixed & Domestic|ECO & Positive\\
UK & http://martini.heraldscotland.com/opinion/13953205.Developing\_creative\_talent\_in\_the\_Highlands\_and\_Islands/?ref=rss & 718 & Herald Scotland & Private/Non-Public & Online and Offline & Regional/Local & very low = CP mentioned once & Cultural development & Positive & Subnational & No myth & NA & NA & NA & NA & NA & NA & NA & NA & UK & developing creative talent in the highlands and islands & 2015-11-09 & european regional development fund & the organisation's aim back in 1965 was clear; to help breathe life back into communities that had witnessed a downward spiral of depopulation. from its humble beginnings with just six employees, hidb worked doggedly with partner organisations to invest in businesses, industry sectors, skills and infrastructure. this approach has since enabled the region to make the most of opportunities in a host of business sectors, including; energy, food and drink, tourism, life sciences, business services and the creative industries. creativity is part of who we are in the highlands and islands. for creative people living in the region looking to pursue their own chosen path and make a living; be it in the screen and broadcast sector, craft or fashion, the music industry or writing and publishing, there has never been a better time to harvest the opportunities which are there to be pursued. take bafta award-winning cbeebies series katie morag, based on the books by highland author mairi hedderwick. originated by cromarty-based indie, move on up, the series was made entirely in the highlands using local talent - particularly on lewis, where a set was built to accommodate the production. supported by xponorth, katie morag even offered two budding screenwriters, jan storie from plockton and louise wyllie from boat of garten, the opportunity to work with the development team resulting in an all-important credit on two episodes. both had been mentored through the xponorth network, which offers guidance, practical hands-on advice and networking opportunities in the screen, broadcast and digital sector throughout the region. according to lindy cameron of move on up, 62 per cent of the budget was left behind in the region; i.e. it was used and consumed in the highlands and islands, so the hie investment increased by more than tenfold. the support offered by the xponorth creative industries network, which is funded by hie and the european regional development fund (erdf), is behind many of the creative success stories in the highlands and islands. the jewel in the xponorth crown is a two-day long festival held every june in inverness which this year attracted more than 1,400 delegates and key international decision-makers in the fields of screen, broadcast, craft, fashion, writing, publishing and music. our short film showcase attracted an unprecedented number of submissions at just under 3,000 from home and around the globe. there were 104 films screened, including a short shot on a gopro in the waters around orkney by local filmmaker mark jenkins. next year, we will be offering free placements to anyone interested in working on our live xponorth tv station. we're also planning a host of workshops and hands-on training aimed at under-25s in digital media, lyric and song writing, as well as photography. this will run alongside our 30 industry conference panels, film and music showcasing, receptions, parties and networking. we're also introducing a brand new writing showcase. xponorth is year-round commitment. just last week we launched a new creative cluster called northport studio in elgin as part of moray business week. on friday, i did a recce in my old school, lochaber high in fort william ahead of a gaming playground and hands-on camera workshop we're running there as part of lochaber ideas week. we're also holding six networking sessions over the coming fortnight to promote two animation training programmes we're running with king rollo films in skye in december and january, not to mention formulating our offering for next year's festival of architecture. xponorth is the delivery partner for this exciting festival in the highlands. to top the week off, we're at the bafta scotland awards on sunday keeping our fingers crossed for various highland success stories - including katie morag. xponorth may be known to many as a two-day festival in june but the hard work continues throughout the year. & 654 & very low & Low & Socio-Economic & NA & NA & 2015-11-09 & 2015 & 1 & ECO
Frame & v.low & Regional & 500-1000 & -0.7997063 & -1.2431481 & 1.1980777 & 1.4613788 & -0.9363498 & 1.1 & -0.8854249 & -0.3297660 & Payer & Domestic & Domestic & Domestic & Domestic|ECO & Positive\\
UK & http://www.walesonline.co.uk/news/politics/four-reasons-aneurin-bevans-biographer-11509044 & 725 & WalesOnline & Private/Non-Public & Online only & Regional/Local & very low = CP mentioned once & Solidarity to poor countries/regions & Positive & EU & No myth & Institutional bargaining over funding & Negative & National & No myth & NA & NA & NA & NA & UK & aneurin bevan's biographer says the nhs pioneer woudd vote to stay in eu & 2016-06-22 & european social fund & bevan was a eurosceptic but nick thomas-symonds is sure that in 2016 he would back eu membership labour giant aneurin bevan was a eurosceptic but today he would vote to stay in the european union and protect the nhs, according to his biographer. author nick thomas-symonds - who is now labour's torfaen mp - argues that bevan would be alarmed by the prospect of a post-brexit right-wing conservative government and would also want to preserve the eu's potential to redistribute resources and tackle cross-border issues such as climate change. he said: "despite his hostility in the 1950s in a very different context i believe that today he would be very firmly backing remain." here are three reasons why he thinks bevan would vote to remain: 1. keeping right-wing hands off the nhs noting that the ebbw vale mp resigned in 1951 in protest at prescription charges for glasses and dental care, the biographer argued that he would not want to see individuals who have made positive noises about charging getting to influence government policy. he said: "i think [bevan] would be very concerned about a very right-wing brexit tory government taking office... clearly, that's complete anathema to aneurin bevan's view." 2. the eu redistributes wealth in his biography, nye: the political life of aneurin bevan, mr thomas-symonds acknowledges that the mp saw the "british parliament as the vehicle for the advance of socialism" and feared the "eec would undermine its authority". but the author argues that he would take a very different view in 2016, saying: "the european union is now redistributive in terms of wealth from the richest to the poorest. a very good example of that is european social funding, of course. "i simply don't believe that, again, a right-wing tory government [would] make up the shortfall in the funding. and the point of the european social fund across europe is that it does help certain areas; i mean, my own constituency of torfaen benefits enormously from it." 3. we can only solve our biggest problems together mr thomas-symonds also reckons that bevan would appreciate the eu's potential to solve common problems. he said: "there's a lovely quote from bevan in the late 1950s where he talks about the world leaders of the age and he used the phrase about them strutting on smaller and smaller stages. behind that is the idea that the world is and was then dramatically changing, so what are the levers are available to world leaders to change the situations around them? "if you think about some of the challenges we face today like climate change and environmental protection, for example, [it] is a hugely cross-border issue - you can't deal with that as one country." arguing that tackling the challenges of globalisation also requires cooperation, he said: "if we really want to influence that i think we stand a far better chance of doing that as part of a grouping of 500 million people than we do on our own." 4. the eu defends workers' rights he added: "if you have global multinationals, [what] the european union does is prevent them playing one country off against another in terms of workers' rights... there is no doubt at all that european law has enhanced and deepened [rights] and has also protected i believe workers in periods when we've had right-of-centre governments in this country. "that floor of workers' rights is something that would have been extraordinarily attractive to [the] whole attlee cabinet because workers' rights are one of the raison d'êtres of the labour party." eu referendum latest news the financial markets are worried academics condemn 'falsehoods' brexit would devastate farming businesses need free movement 1 of 4 & 635 & very low & Low & Values & Power & NA & 2016-06-22 & 2016 & 2 & ECO
Frame & v.low & Regional & 500-1000 & -0.7997063 & -1.2431481 & 1.1980777 & 1.4613788 & -0.9363498 & 1.1 & -0.8854249 & -0.3297660 & Payer & European & European & European & European|ECO & Positive\\
UK & https://www.walesonline.co.uk/business/business-news/flight-simulation-company-thousands-customers-15030380 & 704 & WalesOnline & Private/Non-Public & Online only & Regional/Local & very low = CP mentioned once & Economic development & Positive & Subnational & No myth & NA & NA & NA & NA & NA & NA & NA & NA & UK & flight simulation company was started in swansea by a 16 year old & 2018-08-15 & european regional development fund & joe charman has built the business up over the past four years get business updates directly to your inboxsubscribesee our privacy noticethank you for subscribing!could not subscribe, try again laterinvalid email at 16 years old and with just £200 joe charman started a business that in four years has thousands of customers around the world. the now 20-year-old business student from swansea university is hoping his airport and flight simulation software will really take off. mr charman built on his passion for aviation to develop his business, which now boasts more than 3,000 customers worldwide. mr charman launched pilot plus at the age of 16 and now manages a team of three employees. along with his team mr charman replicates real-world airports in a digital 3d format, using imagery they capture of airport scenes. the digital 3d replicas of the exterior of airport environments, including gatwick, bristol and geneva, create a mock-up environment for users, as if navigating a plane through that particular airport in real life. the software has been sold to flight schools for pilot training purposes, commercial pilots, and aviation enthusiasts, with sales that have covered nearly every corner of the world. at present there are seven products in the pilot plus repertoire, and the team are currently working on their next product for london city airport. speaking about the creation of the company, mr charman said: "when i was growing up i wanted to be an airline pilot and was desperate to explore the industry. "before long i realised that airport simulation didn't really exist as a product for the mass market, so with about £200 to my name i started to develop my first product for southampton airport. "i had my first few customers, which then turned into more, and the customer base has kept growing since." it's an intricate process developing the software, which individual members of the team take responsibility for. the first step for joe's team is defining what they want the end product to look like, determining what assets they will need to create it and how they will get there. if possible, the team will visit each individual airport to take as many reference pictures as possible. those images are then collated with further research, such as building plans and environmental data. once all the necessary data is collected the pilot plus team begin the process of turning everything into a virtual world. joe uses a wide range of programmes, some of which are industry leading, while others are more niche. the team make product assets which build a realistic environment, replicating buildings and designing their own environment assets. all components are then brought together to create the most realistic environment possible, before being packaged up and the software product published. mr charman said: "when i first started i had a basic understanding of the tools available, limiting how realistic the environments looked. "in the past few years, as the team has grown, we've adopted new technology and skills. "we focus on constantly innovating, bringing new technology to our products to further enhance the realism of the environments, and we're extremely proud of the recent in-house advances we've made and are yet to publicise. "our current project, london city airport, will look phenomenal when it's complete. "the plan after is to filter the recent tech advances we've made into older products, so we have the best and most up-to-date products possible on sale." since its instigation, pilot plus has seen an average 26\% growth in profit year on year. by the end of 2019, mr charman hopes to expand his team to seven employees, which will give them the capacity to run multiple projects at one time. mr charman manages his business alongside his studies at swansea university, where he is going into his third year of business management. with his entrepreneurial ambitions in mind, joe was introduced to big ideas wales, part of the business wales service to encourage youth entrepreneurship in wales. big ideas wales is funded by welsh government and the european regional development fund aimed at anyone between five and 25 who wants to develop a business idea. seeing an opportunity to expand his connections further, joe set up swansea university's first entrepreneurs network with a fellow student, a group which encourages students to follow their business ambitions. he continued: "i have always been business minded, but big ideas wales has helped my personal development skills and linked me with many influential entrepreneurs, further fuelling my entrepreneurial drive." in his third year mr charman has plans to take up office space in swansea. he said: "i have always found it easy enough to manage my university work and business life, knowing when to give one more attention than the other when up against deadlines. "in my final year i plan to take on more people to help expand our product range and continue to push to be the main market leader, competing with others that have launched in the last few years. "i believe the future of pilot plus is bright, we are creating products that lead innovation in the market. "as we focus on our current project, london city airport, the roadmap is then to bring all past products to the same standard. "we're also planning to invest in different markets where our skills and tech can be applied, such as interior design with vr, as well as architecture visualisation." julie walters, development manager for big ideas wales, said: "joe is a shining example of a driven young person who big ideas wales is proud to support. "his entrepreneurial flair from such a young age is so impressive and we wish him success." rhodri evans, enterprise support officer and enterprise champion at swansea university, said: "at swansea we encourage all our students to act upon their entrepreneurial flair to enhance their own experience at university and increase their employability when they graduate. "thanks to his natural aptitude for business, we know joe has a bright future ahead of him." & 1023 & very low & Low & Socio-Economic & NA & NA & 2018-08-15 & 2018 & 3 & ECO
Frame & v.low & Regional & +1000 & -0.7997063 & -1.2431481 & 1.1980777 & 1.4613788 & -0.9363498 & 1.1 & -0.8854249 & -0.3297660 & Payer & Domestic & Domestic & Domestic & Domestic|ECO & Positive\\
UK & http://www.belfasttelegraph.co.uk/news/northern-ireland/heritage-body-hits-out-at-extension-to-waterfront-hall-in-belfast-31480936.html & 715 & Belfast Telegraph & Private/Non-Public & Online and Offline & Regional/Local & very low = CP mentioned once & Cultural heritage & Negative & National + Subnational & 5.Useless projects & NA & NA & NA & NA & NA & NA & NA & NA & UK & heritage body hits out at extension to waterfront hall in belfast & 2015-08-27 & european regional development fund & the controversial design of a £30m extension to belfast's waterfront hall has been blasted by one of northern ireland's architectural heritage groups. the new conference centre construction on the river lagan has already sparked anger, with some labelling it "cultural vandalism" by blocking what was one of the most impressive views in the city. now nikki mcveigh, chief executive of the ulster architectural heritage society, has said it is "unfortunate that approved plans for its extension are not seen to give due reference to the waterfront hall's architectural qualities, particularly its distinctly recognisable circular form". "this is particularly disappointing when more sympathetic proposals may have been presented for consideration," she added. she said while still a relatively new building, which is "not eligible for protection of listing due to its age", it has become "a key aspect of the city's heritage and identity". "the value it represents to local people is evident in the attention this alteration has received. the value of our architecture, our heritage, socially, environmentally and culturally, must not be overlooked." work began on the extension to the waterfront last october and is due to be completed this december. the extension won't officially open for business until next may. as construction work continues, many have taken to social media to express their disapproval at how it looks. todd architects - the company behind the design of the extension - defended it, calling it a "contemporary, active and dynamic frontage to the waterfront". the firm's managing director paul crowe said the extension was being unfairly and prematurely judged. he said the firm would discuss the approach to the design once construction was completed. the new building has attracted several unflattering nicknames , including the box on the docks, the waterfront wall and the sore on the shore. the extension will double the size of the waterfront hall's conference and exhibition facilities, offering an additional 4,000sq metres of space, including a 2,000sq metre hall and a 750sq metre smaller hall. belfast city council has put £11m of ratepayers' money towards the work, and estimates the new facilities could bring up to £39m a year into the economy. the rest of the funding is coming from the tourist board - £4m - and the european regional development fund, which is contributing £14.5m. construction of the waterfront hall for belfast city council began in 1993, with the circular concept created by architect robinson mcilwaine. designed to catch the eye with its shape and domed roof, it was initially set back from neighbouring buildings and the riverfront. it soon became belfast's premier concert and conference venue, before losing out to gigs when the much bigger odyssey arena - now the sse arena - opened for business in 2000. & 460 & very low & Low & Socio-Economic & NA & NA & 2015-08-27 & 2015 & 1 & ECO
Frame & v.low & Regional & <500 & -0.7997063 & -1.2431481 & 1.1980777 & 1.4613788 & -0.9363498 & 1.1 & -0.8854249 & -0.3297660 & Payer & Domestic & Domestic & Domestic & Domestic|ECO & Negative\\
\addlinespace
UK & http://www.liverpoolecho.co.uk/news/business/datalaw-expansion-trail-after-north-8149978 & 766 & Liverpool Echo & Private/Non-Public & Online and Offline & Regional/Local & very low = CP mentioned once & Economic development & Positive & EU + Subnational & No myth & NA & NA & NA & NA & NA & NA & NA & NA & UK & datalaw on expansion trail after north west fund investment & 2014-11-24 & european regional development fund & online legal training provider datalaw has received financial backing as it ramps up its expansion plans. the north west fund for venture capital, managed by enterprise ventures, has invested a six-figure sum for a minority stake in the liverpool business. datalaw managing director charles peter said the money would enable the firm to greatly increase the number of courses it offers and to fund a new technology platform. the investment has already helped datalaw to make key appointments in research and development, marketing and digital development to support its growth strategy. datalaw delivers more than 350 continuing professional development courses via webinars and podcasts to thousands of solicitors nationwide, covering more than a dozen areas of law. it is also a provider of mandatory courses and exams required for solicitors to represent clients in police stations and magistrates' courts. mr peter said: "we were the first company to move from training solicitors in conventional physical venues to offering online courses, allowing them to comply with the profession's cpd requirements far more cost-effectively. "having grown to become one of the uk's largest and most innovative providers of online legal training, with particular strengths in criminal and family law, we have been planning the next phase of our expansion. "over the next 12 months, we aim to quadruple the number of courses we deliver, reaching a wider range of lawyers. we will also be launching a new technology platform. "this investment from the north west fund for venture capital is a major endorsement of our strategy." the north west fund for venture capital supports entrepreneurs building high-growth companies with finance ranging from £50,000 to £2m. the north west fund for venture capital is part of the north west fund which is financed by the european regional development fund and the european investment bank. will clark, of enterprise ventures, said: "datalaw is a business with a proven track record of innovation and growth. this new investment will allow it to further develop its offering." & 337 & very low & Low & Socio-Economic & NA & NA & 2014-11-24 & 2014 & 1 & ECO
Frame & v.low & Regional & <500 & -0.7997063 & -1.2431481 & 1.1980777 & 1.4613788 & -0.9363498 & 1.1 & -0.8854249 & -0.3297660 & Payer & Domestic & European & Mixed & Domestic|ECO & Positive\\
UK & http://theconversation.com/brexit-extension-how-much-will-it-actually-cost-the-uk-to-leave-the-eu-115486 & 749 & The Conversation & Private/Non-Public & Online only & National & low = CP mentioned more times but NOT important part of story (mainly about others issues) & Institutional bargaining over funding & Factual & EU + National & No myth & NA & NA & NA & NA & NA & NA & NA & NA & UK & brexit extension: how much will it actually cost the uk to leave the eu? & 2019-04-17 & structural funds & eu leaders have agreed a short extension to brexit until october 31 at the latest, in order to give the british parliament time to agree a deal. this date avoids the uk still being a member when the next eu budget cycle starts - but what does it mean for the "divorce bill", the money the uk will pay the eu after its departure? the withdrawal agreement, which provides the legal framework for the uk's exit from the european union, includes arrangements to determine its financial settlement with the eu. the essential principles of the financial settlement have been set out for months and the eu has made it clear that it regards the withdrawal agreement as settled. if and when the british parliament does agree a brexit deal, this financial settlement is likely to apply - but it won't change directly as a result of the extension from the original march 29 brexit date to october 31. read more: brexit extended to october 31: why the eu chose a six-month reprieve for its awkward partner the final figure is yet to be determined, both because of uncertainties over future developments - including of course the date when the uk actually leaves the eu - and also because the bill is to be settled in euros and therefore depends on the future exchange rate between the pound and the euro. under the financial settlement, the uk will continue to participate in the eu's budgets over 2019 and 2020, which takes us up to the end of the current eu budget cycle. this can be seen as the uk simply honouring the commitments it made as an eu member state at the start of the current eu budget round, and continuing to receive agreed funds through payments from the common agricultural policy, structural funds and other eu programmes. a key principle is that the uk should not pay more than it would have done as an eu member, nor should it be expected to make any payments earlier than it would have been required to do as a member. the uk's net payments include its long-negotiated budget rebate. the uk is the third-largest contributor to the eu budget in total. it is a net contributor to the eu budget payments, when the funds the uk receives back are taken into account, although it is only sixth in terms of net payments relative to its gdp. the key uncertainties over the future payments arise from future contingencies and the final settlement of accounts with the eu. a key principle here is that of the uk not making payments any sooner than it would have done as a member. the annual eu budget makes future expenditure commitments under what are called reste à liquider arrangements, outstanding commitments which have been made but not yet translated into payments. under the final settlement, the uk continues to be liable for spending commitments undertaken during the current eu budget round while it has still been a member - although it will also receive back its share of assets including capital in the european investment bank. the largest item in the outstanding liabilities is for eu staff pensions and other benefits. eu staff pension commitments are unfunded, meaning the eu has not set aside provision for these future payments, as it operates on a "pay as you go" system. this means there is uncertainty over the size of future eu pension liabilities and therefore the uk's future commitments to paying them. estimating the final bill hm treasury has estimated the total payments will be around £35-£39 billion. the independent office of budget responsibility (obr) has also produced similar estimates. crucially, the most recent obr estimates from march 2019 indicate that most of the final settlement will be made up of the reste à liquider commitments. settlement of the outstanding net liabilities - including future eu pension liabilities - would account for less than 10\% of the total. payments from the uk to the eu will continue for many years into the future (up to the 2060s), although the obr estimates that these would be small with around three-quarters of the payments completed by 2022. there is also uncertainty over the euro-sterling exchange rate and uk economic growth, which determines the relative size of the uk economy and its contribution to the eu budget until 2020 from "own resources", the payments determined by a member's gross national income. in june 2018, parliament's public accounts committee questioned the treasury's estimates, indicating that payments could be as much as £10 billion higher. in particular, they note potential commitments under the european development fund of overseas expenditure on aid programmes. the next multi-annual financial framework for the eu budget is scheduled for 2021-27. should the withdrawal agreement eventually be agreed by parliament, a short extension of brexit beyond october 31 would not lead to large changes in the financial settlement, but a longer extension into 2020 could potentially lead to greater commitments if the uk remains a member into the next budgetary period. & 851 & low & Low & Power & NA & NA & 2019-04-17 & 2019 & 3 & POL
Frame & low-medium & National & 500-1000 & -0.7997063 & -1.2431481 & 1.1980777 & 1.4613788 & -0.9363498 & 1.1 & -0.8854249 & -0.3297660 & Payer & Domestic & European & Mixed & Domestic|POL & Neutral\\
UK & http://www.bbc.co.uk/news/uk-wales-politics-40860948 & 759 & BBC & Public & Online only & National & high = CP is most important issue in story (can also cover other issues) & Political capital/interests & Negative & EU + National & No myth & Institutional bargaining over funding & Positive & EU + National & No myth & NA & NA & NA & NA & UK & colwyn bay eu flag paving slab 'propaganda' criticised - bbc news & 2017-08-08 & european regional development fund & a european flag etched into a paving slab on colwyn bay promenade has sparked accusations of eu propaganda. the slab, part of the waterfront project which is part-funded by eu cash, was spotted by a conservative town councillor. ukip mep nathan gill said it was "insensitive" of conwy to set into concrete the eu flag. conwy council said it was obliged by the terms of the grant to acknowledge the source of the funding. the uk is due to leave the european union in march 2019, following the 2016 referendum when a majority of those voting in wales backed brexit. the slab is part of phase two of conwy council's waterfront project, which includes coastal defence work and environmental improvements. the £4.5m scheme is supported by welsh government and £1.6m of eu funds. mr gill, who is also an independent am for north wales, said: "i understand that until we leave the eu in may 2019 it is still 'technically' business as usual - we still pay into the pot and get some of our money back. "but i think this is very insensitive and short-sighted of conwy council to set into concrete the eu flag, with the date 2017, a year after wales and specifically conwy voted to leave. "this flag will be embedded on our sea front for decades after we have left, and made a huge success of brexit." adrian mason, a conservative town councillor in colwyn bay, said: "i'm disappointed on the basis that it is eu propaganda. "why they choose to do that now, while we are leaving, is beyond me." a conwy council spokeswoman said that in accepting eu funds for the project, which was approved prior to the referendum, it had an obligation to acknowledge it was part-financed by the european regional development fund. remain-supporting llyr gruffydd, plaid cymru am for north wales, said: "regardless of whoever paid for it, it's only fair that people that provide funding are recognised." he added: "it may remind many people of what we potentially might lose when we leave the eu." & 353 & high & High & Power & Power & NA & 2017-08-08 & 2017 & 2 & POL
Frame & high-very high & National & <500 & -0.7997063 & -1.2431481 & 1.1980777 & 1.4613788 & -0.9363498 & 1.1 & -0.8854249 & -0.3297660 & Payer & Domestic & European & Mixed & Domestic|POL & Negative\\
UK & http://www.bbc.co.uk/news/in-pictures-39254734 & 730 & BBC & Public & Online only & National & very low = CP mentioned once & Social justice & Positive & EU & No myth & NA & NA & NA & NA & NA & NA & NA & NA & UK & in pictures: the faces of brexit - bbc news & 2017-03-14 & european regional development fund & photographer steve franck has set out to photograph 100 people, 52 who voted leave and 48 who wanted to remain in the eu. the project continues, but here we present 10 of those photographed, five from each side of the argument. "i listened to both sides of the argument just to make sure i knew what i was doing was the right thing for me. i wanted independence for our country, not to stop immigration or anything like that, but something so we could have our set of rules because i felt we'd been gathered up into europe when our country's quite different to the europeans." "i voted to remain because i felt that it was the best way forward for the economy and politics. if you think back over the last 100 years and the number of wars we've had in europe, i think that the eu is a means to prevent that ever happening again. i think if you resurrected any of the people killed in world war one and asked them is this a good idea, they'd all say yes." "the referendum for brexit seemed very binary and the reason to stay or leave was based on two decisions. the benefits of staying or leaving were clouded over by the headline campaigns - immigration and bleak outlook for the economy versus regaining our sovereignty. "looking at the wider debate and the many reasons to leave or stay, the decision for me to vote leave was based on my opinion that the eurozone is not a level playing field. having relations in the farming industry, they have struggled with competition from the eurozone. "although they receive subsidies, they are not equal to those afforded to their counterparts across europe. bringing prices down is not always bad, except here it is at the cost of the british farming industry, similarly with the uk fishing industry." "i voted to remain because despite the eu's faults it achieved a lot in terms of protecting workers' rights and our environment. it was also very effective at redistributing wealth across eu nations - something i'd seen first hand in malta where my family originate from. "i also voted to remain because of migration. as someone under 30 and a second generation immigrant, i understand how important freedom of movement is - from a human rights perspective, for work and education and for economic benefit. and i am a direct beneficiary of migration." "born and bred in liverpool and growing up with a fairly 'left wing' view of life i was too young to vote on joining the eec, but i do remember it being a very exciting time. i have a great number of friends, family and work friends who are from european countries (eu and non-eu) and have always felt 'part' of europe. "however i have also always been unhappy with the bureaucracy and wastage that appears to come with the eu. my largest concern is probably the process where the eu remit evolved from a single trade agreement/market into an organisation that has grown its remit to even include making laws that member countries have to uphold - and all this change made with very little public visibility. "i voted leave and strongly believe it was the right choice - and i still do. there has been much pain on both sides since the vote and my biggest personal regret is that i have lost some friends over the vote, simply because they believed a vote to leave the eu was a vote 'against them'." "i voted remain. i voted from the heart, i didn't delve too much in the political statements made by either side to be honest. instinctively i felt that we should not leave the eu. we gain a lot from being part of it, freedom and mobility for instance. "the ability to explore countries and cultures that are on our doorstep with such ease can only help our understanding of the world and the intricacies of its people. every one of us could do with a bit more understanding." "the ruling classes always needed a large pool of cheap labour, in the past they've had it from the irish, then they had it from the west indians, now they've got it from the eu and that's what they want to keep. "there are now three million eu citizens in the uk. as far as i'm concerned they should stay, at least temporarily until they can do something about their circumstance. you can't just boot them out. there are too many people working for the working class in this country instead of with them and i think that is a major point." "well for me it's quite simple. first of all i don't have enough information about what exactly they're talking about and the devil you know is better than the angel you don't. "i don't know what's going to happen if we're out, what's the best thing for me, so i voted in. i think we've managed to live like this for so long, why change the way it is and what are the benefits, what are the pros and cons? no-one's able to tell you. " "for a long time i was remain and then as it got closer to the event, i thought i should actually ditch all the press and do my own research so i spent days looking for information, finding out how the eu works. i was absolutely horrified. "there are more than 10,000 members of staff employed by the eu who earn more than our pm. and the fact that the meps have absolutely no power whatsoever, they're just there as a token gesture, everything is decided behind closed doors, the minutes of all the meetings never have to be published, so we are told essentially how to live our lives by someone who we have no power to unelect. i do not like to be dictated to." "my instincts throughout my life have been what i would call just left of centre. the eu has been one of the greatest redistributors of wealth downwards, but a lot of the people who benefit from that don't understand that there are many projects and improvements in their areas with money from the european regional development fund. "i was open to listening to the leave campaign, but i think to change the status quo they've got to make a strong case for change and i don't believe they did. their main plank was reducing immigration but they didn't explain how they would do it, when they would do it and i'm not convinced they will do it to any great degree." & 1144 & very low & Low & Socio-Economic & NA & NA & 2017-03-14 & 2017 & 2 & ECO
Frame & v.low & National & +1000 & -0.7997063 & -1.2431481 & 1.1980777 & 1.4613788 & -0.9363498 & 1.1 & -0.8854249 & -0.3297660 & Payer & European & European & European & European|ECO & Positive\\
UK & http://www.liverpoolecho.co.uk/news/firm-becomes-first-tenant-flagship-11222879 & 783 & Liverpool Echo & Private/Non-Public & Online and Offline & Regional/Local & very low = CP mentioned once & Research \& innovation & Positive & Subnational & No myth & NA & NA & NA & NA & NA & NA & NA & NA & UK & firm becomes first tenant at flagship daresbury enterprise zone development & 2016-04-23 & european regional development fund & conveyor networks to relocate to techspace at sci-tech daresbury a software and automation firm has become the first tenant in a landmark development at sci-tech daresbury enterprise zone. conveyor networks will relocate from lymm to the techspace office and laboratory complex at sci-tech daresbury. established in 2009, conveyor networks works with retail and e-commerce businesses in the uk. the company has signed terms to become the first tenant in techspace two, one of two sister buildings being built on the site. the project was supported by the european regional development fund programme 2007-2013. together with techspace one, it will provide 56,000 sqft of speculative grade a office and laboratory space at the sci-tech daresbury enterprise zone. work will be completed this summer. david carroll, conveyor networks managing director, said the combination of engineering and high-tech expertise at sci-tech daresbury offered the perfect fit for his business, which was seeing increased demand for software and technology-led logistics solutions as well as growth in its core automation business. a company spokeswoman said the firm's expertise has driven the successful development of its imio warehouse software suite, which is in place at e-retailers such as qvc, whsmith and debenhams. she said: "conveyor networks' business goals for the next five years focus on growth driven by new product development, excellent customer care and the ability to understand and react to developments in logistics and e-commerce. "conveyor networks will occupy the ground floor of techspace two when it takes up its tenancy, locating 16 staff at its new home. "the new space means the business can take advantage of the opportunities for networking, developing its partner base and adding more value to its offering." mr carroll said: "the pace of change in logistics is fast and it's crucial for us to stay ahead of this and offer customers long-term partnership and support. "we've already seen the benefits of the connections at sci-tech daresbury which have helped us develop initiatives quickly and cost effectively." & 345 & very low & Low & Socio-Economic & NA & NA & 2016-04-23 & 2016 & 2 & ECO
Frame & v.low & Regional & <500 & -0.7997063 & -1.2431481 & 1.1980777 & 1.4613788 & -0.9363498 & 1.1 & -0.8854249 & -0.3297660 & Payer & Domestic & Domestic & Domestic & Domestic|ECO & Positive\\
\addlinespace
UK & http://www.belfasttelegraph.co.uk/business/news/lisburn-firm-leckey-creates-50-jobs-as-part-of-32m-investment-34528185.html & 712 & Belfast Telegraph & Private/Non-Public & Online and Offline & Regional/Local & very low = CP mentioned once & Jobs & Positive & National + Subnational & No myth & NA & NA & NA & NA & NA & NA & NA & NA & UK & lisburn firm leckey creates 50 jobs as part of £3.2m investment & 2016-03-10 & european regional development fund & fifty new jobs have been announced as part of a £3.2m investment in lisburn company leckey. northern ireland's first and deputy first ministers announced the new jobs as part of an investment to expand its product offering and boost its export sales. leckey is market leader and a global player in the manufacture and supply of supportive equipment for children with special needs and successfully exports its positioning equipment worldwide. recruitment of the 50 new jobs has already started and 18 of the roles. the 50 jobs will be across a range of roles including manufacturing, sales, design, engineering and marketing. first minister, arlene foster said: "this £3.2million investment is a significant commitment by leckey which makes a valuable contribution to northern ireland's vibrant life sciences sector. "this export focused development will help the company respond to the demands of its customers as well as help shape the future technology available within the mobility market." the deputy first minister, martin mcguinness, said: "this investment, resulting in 50 new jobs, will generate over £1million annually in additional salaries and offer valuable employment opportunities. support for this expansion demonstrates the executive's commitment to help indigenous manufacturers strengthen their competitive position in export markets. ceo james leckey, added: "this investment is an exciting milestone in the strategic growth of our business which will focus on building our core sales and introducing a new range of products. "innovation has been central to our success to date and working with invest ni we are positioning ourselves to move with key trends and developments driving growth in our target markets." invest northern ireland has offered the company £715,366 towards its growth plans which involves r\&d to design and prototype new products and grow its customer base outside northern ireland. invest ni's support includes a r\&d grant of £315,366 part funded by the european regional development fund under the eu investment for growth and jobs programme 2014-2020. & 331 & very low & Low & Socio-Economic & NA & NA & 2016-03-10 & 2016 & 2 & ECO
Frame & v.low & Regional & <500 & -0.7997063 & -1.2431481 & 1.1980777 & 1.4613788 & -0.9363498 & 1.1 & -0.8854249 & -0.3297660 & Payer & Domestic & Domestic & Domestic & Domestic|ECO & Positive\\
UK & http://www.walesonline.co.uk/news/politics/dafydd-wigley-lays-out-positive-11494108 & 737 & WalesOnline & Private/Non-Public & Online only & Regional/Local & low = CP mentioned more times but NOT important part of story (mainly about others issues) & Institutional bargaining over funding & Positive & EU + National + Subnational & No myth & NA & NA & NA & NA & NA & NA & NA & NA & UK & dafydd wigley lays out the positive case for wales in europe & 2016-06-19 & structural funds & our world is deeply influenced by our common european heritage, writes the plaid cymru peer last thursday's tragic events add a brutal backdrop to the eu referendum. jo cox's political assassination will have reverberations well beyond thursday's vote. it may well effect the outcome, though the extent will never be known. we are asked to make a momentous decision which will determine our future - economic, social, environmental and cultural - for generations. the outcome will impact on every family throughout wales. there will be no turning back. each of us has a duty to cast our vote. if there's a remain outcome, it can't be a matter of everything carrying on as before. serious issues have emerged from this campaign. the uk must co-operate with like-minded eu partners, to make the union more relevant to its diverse peoples; including how wales' needs can be better projected in brussels without always being attenuated by a london filter. if the vote is to quit europe, many more questions will arise - ones which haven't been answered in this campaign. the case for change has to be made by its proponents. the onus is on brexit to justify a huge step into the unknown. they have failed to make that case. this doesn't mean that those frustrated with the status-quo don't have valid arguments; it is just that brexit is the wrong answer to the wrong question. the new york times article entitled "from great britain to little england" , highlighted this: "for many, their angry sense of powerlessness is aimed at the european union. but the truth is that it's from bloated privileged london, not brussels, that the english need to take back control. "the brexit campaign orators, themselves members of that metropolitan elite, have carefully diverted english fury into empty foreigner baiting..." should we stay in the eu? what the polls show 19 things eu funding has done for wales what the rest of the world is saying the referendum has create a monster 1 of 4 let's therefore emphasize the positive case for europe. the eu came into existence, after a war fought to eradicate fascist racism. people came together determined that never again should such a fate strike our continent. since then we have enjoyed seventy years of peace - the longest such period in four centuries. we should never take peace for granted. nor should we take for granted our freedom to travel on holiday, to work or to retire in eu countries, with healthcare and pension cover; and for students to study in these countries the eu market of 500 million people is vital to wales. we have over 200 american companies and 50 japanese companies, who came here to manufacture for european markets - companies like ford, toyota and siemens. if such companies decline, many small businesses who supply them, will be hit. if we leave, what access to these markets does brexit propose? initially, they cited norway or switzerland as models. but to secure market access, both have accepted eu regulations, and free movement of people. now they say uk companies could export under world trade rules. yes, subject to tariff barriers, on manufactured goods and between 14\% and 70\% for agricultural produce. over 90\% of sheep-meat and beef exported by welsh farmers, is sold to europe. eu agricultural payments underpin 80\% of welsh farms. rural wales would be devastated if such aid ends. both the fuw and nfu support remaining in the eu. wales benefits from european structural funds. in total wales gets £245m more each year from the eu, than we pay in, which we stand to lose if we leave. brexiteers claim that westminster would replace such money from the net £9bn per annum britain pays to brussels. but their spokespersons have committed this money to fund dozens of programmes - adding up to £112bn. they've spent the same money ten times over! having lost the economic arguments, brexiteers now focus on immigration. they warn of five million, mainly turkish, immigrants. but britain has a veto, allowing us to block turkey's eu accession. over half britain's immigration comes from non=eu countries, over which westminster already has control.. there are 130,000 immigrants from europe undertaking vital nhs jobs. what they're saying about the eu referendum stopping brexit is vital - guto harri brexit would be an 'absolute disaster' we need free movement, dylan jones-evans carwyn's furious attack on brexit leave to protect jobs - gisela stuart whetherspoon's tim martin backs brexit sir emyr jones parry geraint talfan davies 1 of 8 absolute control over migration is impossible.. northern ireland would need customs and passport control along its 160-mile land border with the eu. do brexiteers envisage a wall across ireland, like trump wants between america and mexico? i find the latent racism which underpins the anti-immigrant agenda, utterly repugnant. brexiteers complain of government by "unelected bureaucrats in brussels". the process of making regulations is just as undemocratic at westminster! orders are drawn up by civil servants, which mps can't amend. repatriating regulations to westminster doesn't solve the issue. most eu regulations have a valid purpose. the eu pioneered health, hygiene and safety provisions; statutory parental leave; disability rights; and the working hours directive. these provide social justice, preventing employers from gaining competitive advantage, by exploiting their workforce. decisions should be taken close to those they affect. if decisions can be taken in cardiff, they shouldn't be centralized to london or brussels. but many decisions must be taken on an eu basis, such as controlling atmospheric or maritime pollution. what sort of country do we want for our grand-children? in wales we have deep european roots: in language, culture, music and christian values; our world is deeply influenced by our common european heritage. europe represents an essential dimension of our identity. this is our continent. let's stop sulking on the fringes; let's work with the grain of europe, to build a future of which all can be proud and in which wales plays a positive role. dafydd wigley was the plaid mp for caernarfon from 1974 until 2001 and an am from 1999 until 2003. he now sits in the house of lords eu referendum latest news the financial markets are worried academics condemn 'falsehoods' brexit would devastate farming businesses need free movement 1 of 4 & 1075 & low & Low & Power & NA & NA & 2016-06-19 & 2016 & 2 & POL
Frame & low-medium & Regional & +1000 & -0.7997063 & -1.2431481 & 1.1980777 & 1.4613788 & -0.9363498 & 1.1 & -0.8854249 & -0.3297660 & Payer & Domestic & European & Mixed & Domestic|POL & Positive\\
UK & http://www.liverpoolecho.co.uk/news/business/chambers-councils-lep-join-forces-10134085 & 767 & Liverpool Echo & Private/Non-Public & Online and Offline & Regional/Local & very low = CP mentioned once & Economic development & Positive & National + Subnational & No myth & NA & NA & NA & NA & NA & NA & NA & NA & UK & chambers, councils and the lep join forces for liverpool city region exports push & 2015-09-28 & european regional development fund & chambers of commerce and local authorities in the liverpool city region are joining forces to launch a new exporters' network to help firms gain a foothold in overseas markets. backed by the liverpool city region local enterprise partnership (lep, the network will be launched at liverpool john lennon airport ljla) later this week. the exporters' network will provide a forum for potential exporters to take part in an export masterclass, get advice from support organisations and learn how next year's international festival (ifb) for business can assist in developing global networks. ifb 2016 is being held in liverpool for three weeks next summer and will see business people converge on liverpool from across the world. this week's event at ljla on thursday is being supported by liverpool gin which will be offering gin cocktails to guests. confirmed speakers so far include executives from the leather satchel company, 4040 media, sigmatex, clarke energy and charge point. mark basnett, executive director of the liverpool city region lep, said: "accelerating export growth is vital for the future of the city region economy which is why we are supporting this event and the establishment of this network. "we want all businesses in the city region involved or interested in exporting, from experienced exporters to first timers, to come together to learn from each other and take advantage of the wide range of expertise and support available to increase exporters. "liverpool was built on global trade and this is a great opportunity to further our global ambitions as a city region" exportsefton is a partnership between local businesses, investsefton/sefton council, liverpool and sefton chamber of commerce and uk trade \& investment with the aim of encouraging and supporting businesses to export. the export initiative is funded by sefton council and the european regional development fund. & 302 & very low & Low & Socio-Economic & NA & NA & 2015-09-28 & 2015 & 1 & ECO
Frame & v.low & Regional & <500 & -0.7997063 & -1.2431481 & 1.1980777 & 1.4613788 & -0.9363498 & 1.1 & -0.8854249 & -0.3297660 & Payer & Domestic & Domestic & Domestic & Domestic|ECO & Positive\\
UK & http://www.theguardian.com/public-leaders-network/2016/jul/29/jobs-businesses-education-at-risk-birmingham-without-eu-money & 758 & The Guardian & Private/Non-Public & Online and Offline & National & very high = CP is most important issue + CP is mentioned in title/headline & Institutional bargaining over funding & Positive & EU + National + Subnational & No myth & Economic development & Positive & EU + National + Subnational & No myth & NA & NA & NA & NA & UK & jobs, businesses and education are all at risk in birmingham without eu money & 2016-07-29 & european regional development fund & it's no exaggeration to say that eu funding has been critical to the rebirth and regeneration of birmingham in the past 30 years. today's vibrant, dynamic, confident city has only been made possible with european money. so with the aftershocks of the 23 june vote to leave the eu still clearly being felt across the world, our city needs reassurances that previous eu spending commitments will be honoured. the local government association, which represents councils in england, is seeking urgent guarantees that local areas will receive every penny of eu funding they are expecting by the end of the decade. don't get me wrong. we're still open for business and we will build on deals that have brought the likes of hsbc and deutsche bank to our city. but our task will be made much easier if the government can put an end to uncertainty over outstanding eu funding. since the mid-1980s more than £1bn of eu funding has been a catalyst for this change in birmingham. the renaissance of the city has seen eu investments in the national exhibition centre, the international convention centre, millennium point and the town hall, to name just a few notable beneficiaries. but it has also helped reshape the city centre infrastructure by providing funding to remove the old inner-city ring road. local businesses have benefitted, too. in the 2007-13 programme, 24,000 west midlands-based businesses received support through the european regional development fund, which also supported more than 7,000 people to start up a new business. in the same period more than £300m was invested in jobs and skills, which moved around 60,000 people into work, 48,000 into further education and training, and supported 50,000 in gaining new qualifications. eu funds also helped to respond to crisis such as the closure of the mg rover plant in 2005, where almost 6,000 people were made redundant and thousands more in the supply chain suffered. with more than £65m of eu funding support, a rover taskforce was created and by april 2007 90\% of the redundant workers were back in work. greater birmingham and solihull have established an eu funding strategy to outline how £237m of new eu resources for 2014-20 would support a wider strategic economic plan for our area. a quarter of this resource has already been committed to 10 key investments, including a £50m youth employment initiative supporting 16,000 young people into work and a £33m business growth programme providing financial support for more than 500 small and medium businesses, creating 1,300 jobs. the referendum result has now placed these and the remaining 75\% of uncommitted potential investments at risk. five weeks have passed since the vote to leave areas right across the country are still waiting for the government to reveal its plans for these programmes. in the meantime key investments have stalled. we are still waiting to hear about the fate of funds earmarked for a £250m midlands engine loan fund and £22m for green infrastructure developments, with a further £14m of new skills investments are in the pipeline. if cities such as birmingham are to continue to lead and grow the regional economy then the uncertainty over eu funding needs to be removed. we urgently need a guarantee that these resources can continue to stimulate and help transform our city regions. talk to us on twitter via @guardianpublic and sign up for your free weekly guardian public leaders newsletter with news and analysis sent direct to you every thursday. & 596 & very high & High & Power & Socio-Economic & NA & 2016-07-29 & 2016 & 2 & POL
Frame & high-very high & National & 500-1000 & -0.7997063 & -1.2431481 & 1.1980777 & 1.4613788 & -0.9363498 & 1.1 & -0.8854249 & -0.3297660 & Payer & Domestic & European & Mixed & Domestic|POL & Positive\\
UK & http://www.bbc.co.uk/news/uk-scotland-highlands-islands-28570371\#sa-ns\_mchannel\%3Drss\%26ns\_source\%3DPublicRSS20-sa & 790 & BBC & Public & Online only & National & very low = CP mentioned once & Infrastructure & Positive & National + Subnational & No myth & Economic development & Positive & National + Subnational & No myth & Jobs & Positive & National + Subnational & No myth & UK & highland town sees broadband upgrade & 2014-07-30 & european regional development fund & super-fast broadband services have officially launched in fort william. more than 3,240 homes and businesses in the lochaber town now have access to the high-speed technology, installed as part of the bt openreach programme. this figure will increase to around 3,880 as engineers complete the local upgrade in the weeks ahead. highland council leader drew hendry said the upgrade would help stimulate local businesses and create jobs. mr hendry said: "super-fast fibre broadband in fort william offers huge benefits to local residents and businesses and will help our local economy to flourish. better, faster communications help businesses to grow and stimulate job creation. "the arrival of fibre broadband means local people and firms can do more online at faster speeds and on multiple devices. "this is great news for many people in fort william and i look forward to fibre broadband being rolled out across the rest of the highlands." in scotland, bt said it was investing about £126m in fibre broadband partnerships with the scottish government, highlands and islands enterprise, the department for culture, media and sport (broadband delivery uk), european regional development fund and scotland's local authorities. bt scotland director brendan dick said: "businesses tell us it's helping them in a wealth of ways, from day to day activities like downloading software, collaborating with clients and moving large data files around to big business decisions like expanding the workforce or introducing better quality it services at less cost." & 247 & very low & Low & Socio-Economic & Socio-Economic & Socio-Economic & 2014-07-30 & 2014 & 1 & ECO
Frame & v.low & National & <500 & -0.7997063 & -1.2431481 & 1.1980777 & 1.4613788 & -0.9363498 & 1.1 & -0.8854249 & -0.3297660 & Payer & Domestic & Domestic & Domestic & Domestic|ECO & Positive\\
\addlinespace
UK & http://www.theguardian.com/politics/2016/apr/07/brexit-northern-ireland-progress-risk-alan-johnson-theresa-villiers & 741 & The Guardian & Private/Non-Public & Online and Offline & National & low = CP mentioned more times but NOT important part of story (mainly about others issues) & Institutional bargaining over funding & Balanced & EU + National + Subnational & No myth & NA & NA & NA & NA & NA & NA & NA & NA & UK & brexit would put northern ireland progress at risk, says alan johnson & 2016-04-07 & european regional development fund & conservative cabinet ministers campaigning for britain to leave the eu are putting northern ireland's political progress at risk, labour's alan johnson will warn. johnson, a former home secretary, will single out theresa villiers, the northern ireland secretary, who is pushing for brexit, saying she ought to understand that the eu has been a steadfast source of political and financial support for the peace process. during a trip to northern ireland next week to campaign for the uk to remain in the eu, johnson will say: "northern ireland's ability to access the single market, coupled with the success of the good friday agreement, has brought about economic development for northern ireland as well as enhanced economic cooperation between north and south in ireland. "a withdrawal from the european union could risk reversing that trend and undermining the economic and political progress made. by siding with the leave faction within the conservative party, the northern ireland secretary theresa villiers is putting that progress at risk." remain campaigners point to comments from the irish ambassador, who told a parliamentary committee that the eu had helped improve relations between the uk and ireland. they claim brexit would cost northern ireland about £100m of funding from the peace programme, as well as money from the european regional development fund. villiers said it was "highly irresponsible to suggest that people in northern ireland will abandon their commitment to peaceful and democratic politics because of this referendum". she said: "there is steadfast support in northern ireland for the political settlement established by the good friday agreement and, as the irish ambassador has made clear, a vote to leave will not affect the peace process. "the peace process was delivered by the hard work of northern ireland's leaders and successive uk and irish governments, supported by the us. there is strong commitment in both the uk and ireland to continue to work together for a peaceful and prosperous northern ireland, and leaving the eu will not change that." villiers added: "with the money we would save from leaving, we could afford to fund the programmes the eu has supported in northern ireland and have money left over for other important priorities like the nhs." remain campaigners in both conservatives and labour have repeatedly warned about the possibility of scotland holding a second independence referendum if the uk votes to leave the eu, but there has been less scrutiny of the possible ramifications for northern ireland. this year enda kenny, ireland's outgoing taoiseach, said northern ireland would face a "serious difficulty" if the uk voted to leave the eu. at a downing street press conference, kenny suggested that the success of the northern ireland peace process was in part linked to the uk's and the irish republic's membership of the eu. "the guns are silent. this has taken a great deal of work from so many people over so many years," he said, when asked whether a uk exit would damage the peace process. "it is important to say that the road out of inequality, the path out of that unfairness, is employment and opportunity. that is why we have shared trade missions to a number of locations, there is a great deal of cooperation with respect of issues of economics in europe. "we should not put anything like that at risk. from our perspective it would be a serious difficulty for northern ireland. i don't want to see that happen. we work on the positive end of this - future benefits and potential coming from a strong britain being part of a strong europe and ireland associated with that north and south." some sinn féin politicians have recently called for a border poll on irish reunification if there is a brexit vote. martin mcguinness, the deputy first minister of northern ireland and former ira chief of staff, said that if britain voted to leave the eu there would be a "democratic imperative" to allow people on the island of ireland to vote on reunification. a border poll is unlikely to be granted by london even if the uk leaves the eu, and successive opinion polls in northern ireland have shown a substantial majority in the region favour remaining within the uk. the largest unionist force in northern ireland, the democratic unionist party, is backing the campaign to leave the eu. however, the ulster unionist party's ruling executive voted to support david cameron's campaign to remain in the eu. & 749 & low & Low & Power & NA & NA & 2016-04-07 & 2016 & 2 & POL
Frame & low-medium & National & 500-1000 & -0.7997063 & -1.2431481 & 1.1980777 & 1.4613788 & -0.9363498 & 1.1 & -0.8854249 & -0.3297660 & Payer & Domestic & European & Mixed & Domestic|POL & Neutral\\
UK & http://www.chroniclelive.co.uk/business/business-news/jesmond-retail-office-development-track-9787904 & 780 & Chronicle Live & Private/Non-Public & Online and Offline & Regional/Local & very low = CP mentioned once & Economic development & Positive & Subnational & No myth & NA & NA & NA & NA & NA & NA & NA & NA & UK & the jesmond retail and office development on track for autumn opening & 2015-08-04 & european regional development fund & the regeneration project, on the site of the former jesmond picture house, could create 100 jobs according to developers the redevelopment of the former jesmond picture house site in newcastle is on course for completion this autumn, as developers report solid progress in its construction. the £6.4m regeneration project was unveiled last year, with developers detailing how the scheme - called the jesmond - could create 100 jobs through a mix of retail and office space. supermarket giant sainsbury's plans to take on a large chunk of the space, which has room for at least nine businesses, through both office and retail. the three-story scheme is being led by property investment firm mk partnership, with support from the department for communities and local government as well as a european regional development fund grant, a result of its pledge to create job opportunities for north east smes. building work, led by contractor metnor construction, began last september and the steel structure has now been completed. sunil mehra of the mk partnership, said: "work at the jesmond has been progressing to plan since last year meaning we are expecting to complete the project on schedule. "the sight of the steel structure is a visual milestone for everyone involved with the project and for the nearby local community that the site is coming back to life. "updating such a historic site has come with many challenges but ultimately the investment is worth it. the jesmond will help regenerate the area, provide a new focal point for this vibrant tyneside suburb and deliver a tremendous amount of value to local businesses and residents alike. "the jesmond will be a landmark building which is ideally suited to a growing business as it offers occupiers a high profile that is normally only associated with trophy buildings in the city centre, but at a significantly lower cost." since the official unveiling of the building plans in november, the development has received significant interest from both potential office and retail occupiers and enquiries are still being considered for the cafe unit, alongside the new sainsbury's. as well as having in excess of 2,000sqm of workspace, the jesmond will have outside terraces on two of its three floors. construction has involved complicated sequencing due to the tight site as the building takes up the entire footprint. the jesmond concept, which is being project managed by knight frank, was penned by award-winning architect kevin owens, the design principal for the london 2012 olympic games and is being delivered by local architect stuart palmer of studio sp. & 430 & very low & Low & Socio-Economic & NA & NA & 2015-08-04 & 2015 & 1 & ECO
Frame & v.low & Regional & <500 & -0.7997063 & -1.2431481 & 1.1980777 & 1.4613788 & -0.9363498 & 1.1 & -0.8854249 & -0.3297660 & Payer & Domestic & Domestic & Domestic & Domestic|ECO & Positive\\
UK & http://www.bbc.co.uk/news/uk-northern-ireland-40799475 & 743 & BBC & Public & Online only & National & very low = CP mentioned once & Institutional bargaining over funding & Factual & Other country & No myth & NA & NA & NA & NA & NA & NA & NA & NA & UK & irish brexit report sets out united ireland proposal - bbc news & 2017-08-02 & structural funds & an irish parliamentary committee tasked with looking at the impact of brexit on ireland has set out a proposal to find a way towards peaceful re-unification. the committee will publish its report later on wednesday. the report outlines 17 recommendations on what the republic of ireland should seek to have in the final agreement between the eu and the uk. fianna fáil senator mark daly, who compiled the report, said that unity could only come through active consent. "this is the first report by a committee of the dail or the senate on how to achieve the peaceful unity of ireland," he said. "last year our former taoiseach enda kenny said the eu needs to prepare for a united ireland. "and it's clear from the 17 recommendations by the committee that a lot of work needs to be done in advance of a referendum."' he added: "from talking to people in both communities in the north it is clear that everybody believes that at some stage there will be a referendum. "but we must learn the lesson from brexit and the lesson from brexit is that you don't have a referendum and then tell people what the future will look like, what you do is you lay out the future in great detail, you talk about the issues of great concern to all communities. the report - entitled brexit and the future of ireland: uniting ireland and its people in peace and prosperity - was ratified by the all-party committee on 13 july. it outlines in detail the options for the island of ireland in the wake of brexit. it recommends the need for special status for northern ireland, the need to protect structural funds and that there should be no new passport controls. in september 2016, bbc northern ireland commissioned an opinion poll in the wake of the brexit referendum. this suggested that 63\% of people in northern ireland supported staying in the uk, whilst only 22\% said they would vote to join a united ireland. & 339 & very low & Low & Power & NA & NA & 2017-08-02 & 2017 & 2 & POL
Frame & v.low & National & <500 & -0.7997063 & -1.2431481 & 1.1980777 & 1.4613788 & -0.9363498 & 1.1 & -0.8854249 & -0.3297660 & Payer & European & European & European & European|POL & Neutral\\
UK & http://www.bbc.co.uk/news/uk-england-cornwall-43405636 & 696 & BBC & Public & Online only & National & very low = CP mentioned once & Cultural heritage & Positive & EU & No myth & Economic development & Positive & EU & No myth & NA & NA & NA & NA & UK & first 'hot rocks' lido to open in uk & 2018-03-15 & european regional development fund & work has begun to create the first lido in the uk to be heated by geothermal energy. drilling has started at jubilee pool, penzance, cornwall to make a geothermal well which will draw up water that has been heated deep underground. the heat will then be transferred to water in adjacent pipes which carry water to the pool. referring to iceland's geothermal spa, pool user jenna fisher said: "penzance is going to get its own blue lagoon!" more on this story and others in cornwall and devon engineers are drilling down to 1.4km (0.8 miles) below the surface where temperatures reach up to about 35c (95f). a giant drilling rig has gone up and the project is costing about £1.5m. the eu has paid for much of the start-up costs through the european regional development fund. the pool's directors hope once engineers can tap into the renewable energy source, the pool's running costs will fall. one of them, martin nixon said: "in iceland, the blue lagoon is the most popular tourist attraction. "over 350,000 visitors a year go directly to iceland to visit the geothermal falls there. "to a degree, we hope a little bit of that factor will happen in penzance." on jubilee pool's facebook page, the idea of a heated outdoor pool in cornwall is getting some positive reactions. annette johns said: "wow, how exciting! spent hours in this pool as a child! strangely then didn't really notice the cold. it's an amazing pool. how fabulous to have it heated!" julie stewart added that the move was "forward-thinking". "how fantastic. it should bring a lot of people down to penzance," she added. the new heated section of jubilee pool will open to the public next summer. & 300 & very low & Low & Socio-Economic & Socio-Economic & NA & 2018-03-15 & 2018 & 3 & ECO
Frame & v.low & National & <500 & -0.7997063 & -1.2431481 & 1.1980777 & 1.4613788 & -0.9363498 & 1.1 & -0.8854249 & -0.3297660 & Payer & European & European & European & European|ECO & Positive\\
UK & http://www.belfasttelegraph.co.uk/news/uk/charities-expected-to-lose-200m-in-eu-funding-after-brexit-committee-warns-35565824.html & 745 & Belfast Telegraph & Private/Non-Public & Online and Offline & Regional/Local & high = CP is most important issue in story (can also cover other issues) & Institutional bargaining over funding & Positive & EU + National + Subnational & No myth & NA & NA & NA & NA & NA & NA & NA & NA & UK & charities expected to lose £200m in eu funding after brexit, committee warns & 2017-03-26 & european social fund & http://www.belfasttelegraph.co.uk/news/uk/article35565823.ece/cff88/autocrop/h342/panews\%20bt\_p-e6911624-c63f-4b1d-96a1-da9754a276a4\_i1.jpg charities which are "the lifeblood of society" are expected to lose £200 million in european union funding after brexit, a parliamentary committee has warned. the house of lords committee on charities said innovative small charities could be the worst hit as they are already under financial pressure. it said public bodies must recognise their importance with grants to test new ideas and by avoiding overly-prescriptive contracts for services. charities receive the eu money from structural and investment funds, in particular the european social fund (esf), and must get more support with funding to meet core costs, the peers said. in evidence to the committee, the royal mencap society described the esf as a "major source of revenue" to help raise living standards and provide job opportunities for young people and the long-term unemployed. medical research charities raised concerns about losing eu funding and opportunities for collaboration after brexit, such as through the horizon 2020 innovation programme. civil society minister rob wilson told the committee the government was listening to charities' concerns and said if they can show "strong value for money" they will continue to be funded. but the peers called on the office for civil society to audit the potential impact of brexit on charities, including the loss of funding and research collaboration, and publish a report by the end of the year. elsewhere, the committee said it had "grave concerns" about the charity commission's proposal to charge charities an annual registration fee. peers said any such plan must have evidence to back it up and the commission must show how any resulting improvements to its regulation of the sector would directly benefit charities. the committee also urged ministers to commission charitable services based on impact and social value rather than simply going for those that cost the least. and it criticised the government for causing "unnecessary concern" with an "anti-advocacy" clause in grant awards and with 2014 laws which "threatened the vital advocacy role of charities", and urged ministers to improve the way it consults the sector when bringing forward policies. committee chairman baroness pitkeathley said: "charities are the lifeblood of society. they play a fundamental role in our civil life and do so despite facing a multitude of challenges. yet for them to continue to flourish, it is clear that they must be supported and promoted. "we found that charities lead the way with innovation, but that this is at risk of being stifled by the 'contract culture'. and while advocacy is a sign of a healthy democracy, and is a central part of charities' role, this role has been threatened by government. "we hope that charities will be encouraged by this report; that the government will respect their role; and that in addition it will value the connections charities have with all sections of society, and encourage the vital scrutiny they provide." sir stuart etherington, chief executive of the national council for voluntary organisations, seized on the committee's concerns over the commission's plans to charge charities a fee. "we know the charity commission is under financial pressure, and that it is in charities' interests to have a well-funded regulator," he said. "but the lords have been unambiguous in saying that they do not think the charity commission have properly considered their plans to levy fees on charities. "the commission must now consider the questions of principle that the committee have raised before it launches into consultation on the detail of a charging scheme." he added: "given concerns over partisan appointments to the charity commission board, there is a clear case that any move to charging should go hand in hand with an overhaul of the commission's governance to ensure it is properly independent of the government of the day." the local government association urged charities to engage with devolution of powers to councils as it will give them "considerable" opportunities. a spokesman said: "while most of the deals agreed so far have focused on driving economic growth, councils have been consistently clear that devolution needs to go further and support local public service reform to better tackle issues such as long term unemployment, youth offending and homelessness. "charities and the voluntary sector often play a vital role in providing services that address the needs of specific communities and in helping drive increased levels of participation. "as devolution provides local areas with greater freedom over the design of public services we expect they will continue to develop their role in shaping effective local public services." & 776 & high & High & Power & NA & NA & 2017-03-26 & 2017 & 2 & POL
Frame & high-very high & Regional & 500-1000 & -0.7997063 & -1.2431481 & 1.1980777 & 1.4613788 & -0.9363498 & 1.1 & -0.8854249 & -0.3297660 & Payer & Domestic & European & Mixed & Domestic|POL & Positive\\
\addlinespace
UK & http://www.bbc.co.uk/news/uk-politics-eu-referendum-36054645 & 770 & BBC & Public & Online only & National & low = CP mentioned more times but NOT important part of story (mainly about others issues) & Economic development & Positive & EU + Subnational & No myth & Jobs & Positive & EU & No myth & Institutional bargaining over funding & Positive & EU + Subnational & No myth & UK & cornwall: in or out? - bbc news & 2016-04-16 & european social fund & over the last 15 years, cornwall has received more than £1bn of eu structural investment. don't worry if that fact has passed you by. plenty of the people i met in cornwall appear to be unaware of it too. i'm testing out how the eu impacts on people's lives for a newsnight series, referendum road. first stop, five hours by train from westminster, even further by plane from brussels, was penzance. from the europhiles' perspective, cornwall has lots to thank europe for. the council of europe gave cornish protected minority language status back in 2002 and cornish clotted cream and pasties are among the local delicacies given special eu protection. and then there's the small matter of millions of pounds. but matthew lobb, a 26-year-old volunteer, told me: "living down here, it's just a dead end... i haven't heard of no money being put into cornwall." lynn watson, a resident on penzance's treneere estate, said: "i don't know where the money from the eu has gone, but it hasn't come here." she then added that she doesn't want to leave the eu. last year, treneere was designated as the most deprived area of cornwall. i visited the whole again communities project, running a soup kitchen for anyone in need, and met brian collett, who was on his way back from a local food bank. life's been tough since he left the navy and he told me he's homeless and now living in a tent. "but i suppose a tent is better than a doorway," he said. "and they do keep telling me the sun is going to come out." on the deprivation index, parts of cornwall are poorer than regions of poland, lithuania and hungary and, because of that, it's eligible for eu funds. the aim is to have a long-term impact on the economy. between 2000 and 2014, £888m was invested in cornwall from the european regional development fund (erdf) and the european social fund (esf). the money has financed infrastructure projects, airports, universities, road widening schemes, superfast broadband and local businesses. another £486m has been earmarked between 2014 and 2020. but when it comes to the referendum, it appears hard cash doesn't automatically buy loyalty. as stephen cummins, a penzance resident buying his fish from a local door-to-door salesman, put it: "my initial reaction was to get out, even though cornwall has had all these millions - because i'm thinking nationally. but i'm still not certain, there are too many unknowns." the unknowns of a brexit (there is no word for that in cornish) were one explanation given by folk who told me they'll vote to stay in. but i also found plenty threatening to vote to leave. as in so many coastal areas, the fishing fraternity are keen "outers". they talk of the good old days before eu quotas, but it's not only about fish. martin clive, who runs newlyn fish co, told me: "i don't like the way europe doesn't feel locally democratic." darren shah, busy filleting the fish that had been bought at dawn at newlyn market to send to the more lucrative london market, said: "it's time we looked after ourselves." a yougov survey suggested cornwall is veering eurosceptic as we head towards 23 june. part of the issue may be that the eu doesn't obviously appear to trumpet its investment. unlike in some other parts of europe, i barely saw that blue eu sign with its familiar yellow stars. one exception is carley's organic foods. based just outside truro, it is one of more than 25,000 cornish businesses to have benefited from erdf investment. on the wall outside its new sustainable premises, part-funded by the eu, there's a little eu logo. owner john carley explained that the £300,000 investment boosted turnover significantly and meant the firm could expand from three to nine employees in a place where every job counts. he's worried that the cornish - and the british in general - will vote to leave the eu: "i suppose there's a degree of contrariness in cornwall. we don't like to be told what to do. i sincerely hope the people of cornwall in the majority will realise the benefits we get from being in the eu," he said. cornwall is the only english area that has access to this extra eu structural investment, because its gdp is less than 75\% of the eu member average. the county has less than 1\% of the uk population, but the funds earmarked in the six years up to 2020 account for 5\% of the eu's uk investment. it's a substantial sum and some here are worried that the county - and the nation - could be heading for brexit without properly considering the consequences. susan stuart spent the best part of two years renovating penzance's chapel house, which she's turned into a hotel. she's backing local regeneration plans, which include hopes for new flood defences, cycle paths and a digital hub. she told me they couldn't fund this without going to europe and an out vote means "we wouldn't get the funding". something for the cornish to mull over as they get closer to referendum day. & 898 & low & Low & Socio-Economic & Socio-Economic & Power & 2016-04-16 & 2016 & 2 & ECO
Frame & low-medium & National & 500-1000 & -0.7997063 & -1.2431481 & 1.1980777 & 1.4613788 & -0.9363498 & 1.1 & -0.8854249 & -0.3297660 & Payer & Domestic & European & Mixed & Domestic|ECO & Positive\\
UK & http://www.walesonline.co.uk/news/local-news/rhondda-wru-apprentice-jess-hancock-11729755 & 707 & WalesOnline & Private/Non-Public & Online only & Regional/Local & very low = CP mentioned once & Jobs & Positive & Subnational & No myth & NA & NA & NA & NA & NA & NA & NA & NA & UK & rhondda wru apprentice wants others to follow in her footsteps & 2016-08-10 & european social fund & the 19-year-old from porth says the apprenticeship was a 'perfect fit' for her, combining her love for teaching and sport a welsh rugby union apprentice from the rhondda has urged other people to consider the route into employment. jess hancock, 19, from porth , started her wru coach core apprenticeship in october 2015, and now works with young children in schools and clubs across rhondda cynon taff . the teenager said applying for an apprenticeship was the "perfect fit" for her, combining her love for teaching and sport. 'getting paid for doing something i love' she said: "sport has played an important role in my life as it helped me to lose a lot of weight when i was younger. "i wanted to be a teacher growing up, but knew i wouldn't get the grades required to get into university, so the coach core apprenticeship was the perfect fit because it combined my love for both areas. "first i tried a catering apprenticeship, but it wasn't the right fit for me. i'm so glad i went for the apprenticeship with the wru, because now i get to work with children and get paid for doing something that i love. "i'm based in the rhondda cynon taff area, and i coach children at local schools and rugby clubs." and, the rhondda apprentice said the work with young people is so rewarding. "it's such a privilege to work with them," she said. "their enthusiasm and energy always makes me smile." carl scales, the apprenticeship manager at the wru, is overseeing the coach core apprenticeship which recruited 12 apprentices in september 2015, and he is backing the welsh government's campaign to encourage young school leavers and employers to consider apprenticeships. 'a head start in the world of work' carl said: "entering the workplace can be a shock to the system for some young people, but it's surprising how working alongside established employees can have a real effect upon their attitude and motivation. "already our apprentices have come on in leaps and bounds and are proving to be motivated and proactive individuals. "it's great to be giving them a head start in the world of work." the apprenticeship programme is funded by the welsh government with the support of the european social fund. for more information visit www.careerswales.com or call 0800 0284844. to apply to be a wru coach core apprentice, visit www.wru.wales/vacancies . & 412 & very low & Low & Socio-Economic & NA & NA & 2016-08-10 & 2016 & 2 & ECO
Frame & v.low & Regional & <500 & -0.7997063 & -1.2431481 & 1.1980777 & 1.4613788 & -0.9363498 & 1.1 & -0.8854249 & -0.3297660 & Payer & Domestic & Domestic & Domestic & Domestic|ECO & Positive\\
UK & http://www.walesonline.co.uk/business/business-opinion/40m-project-create-400-jobs-11063721 & 724 & WalesOnline & Private/Non-Public & Online only & Regional/Local & very low = CP mentioned once & Ineffective goal achievement & Negative & Subnational & 4.No added value & NA & NA & NA & NA & NA & NA & NA & NA & UK & why did a £40m project to create 400 jobs end up creating just 170 at the cost of £200,000 per job? & 2016-03-19 & structural funds & high performance computing wales (hpc wales) was a £40m project established in 2011 to provide a world-class supercomputer facility in high performance computing. funded by £19m from european structural funds, £10m from the uk government, and the balance from the welsh government, the private sector and universities, it seemed a laudable project given that analysis of large amounts of data is becoming more critical to scientific and commercial research. unfortunately, whilst hpc wales did help some firms speed up innovation for commercial success, a report from bbc wales last week showed it that had failed to reach key targets for job creation, supporting welsh businesses and bringing in additional investment to wales. whilst hpc wales had set itself the aim of creating more than 400 jobs and supporting 550 firms, it only created 170 jobs and assisted 247 businesses. despite eventually spending £33m, it generated a paltry £3.7m into the welsh economy. a case of déjà vu unfortunately, hpc wales seems to be a case of déjà vu all over again. like the failed technium programme of incubator buildings that were funded to the tune of more than £100m by the public purse between 2002 and 2006, hpc wales seems to be another grandiose project that did not reflect the real needs of business and delivered little to the welsh economy. yes, technium and hpc wales looked good on paper and even better on powerpoint presentations to government officials. however, they merely continue the welsh obsession with what professor kevin morgan of cardiff university memorably described as "cathedrals in the desert", namely large publicly funded projects that singularly fail to achieve even modest targets to boost economic development and innovation. many will question why such funding is not being awarded directly to businesses, and entrepreneurs in wales will wonder how a cost of £200,000 per job for both techniums and hpc wales could ever be justified in the real world. as pointed out in my access to finance review for the welsh government, given the dearth of money available to businesses during 2011-15, one can only imagine what the impact could have been if the £33m spent by hpc wales had instead been given as either grants or loans to high growth innovative firms in wales with a track record of creating jobs and prosperity. other great columns from prof dylan jones-evans opportunities in hungary welsh entrepreneurial zeal how we can learn from scotland inward investment skills for the future swansea bay city deal lessons from asia women in the workplace 1 of 8 given that the average cost per job for most grant schemes is around £12,000 per job, then over 3,000 jobs could have been supported if this scheme had been used more effectively elsewhere. i am sure many of those firms in the wales fast growth 50 that i work with every week would have been delighted to receive such financial support to help expand their businesses. that is not to say that universities shouldn't be involved in supporting innovative firms, if business needs drive the aims of the project and full advantage is taken of the knowledge and expertise within academic institutions. for example, at the university of the west of england, we are currently delivering the innovation 4 growth programme that is supporting businesses in the south west of england to develop innovative products, technologies, processes and services. since 2014, £6.7m has been awarded to 81 businesses, resulting in 511 new jobs - three times as many jobs as hpc wales for a fifth of the cost. triple helix approach i have always been a strong advocate of the so-called triple helix approach where government, business and academia work together to support innovative businesses. indeed, there are some examples in wales, such as the recent tie up between techhub, the dvla and the university of wales trinity st davids, that could bear real fruit in the future. yet it would seem that in the case of hpc wales, government has supplied the money, academia has spent it and businesses have received little benefit. following the multi-million pound failure of techniums and now hpc wales to suitably help welsh firms, this simply cannot be allowed to happen again at a time when public funding needs to be allocated efficiently to support the welsh economy. that is not to say wales shouldn't have invested in a high performance computer to support academic research, and there should be support for this in the future. however, the main aim of hpc wales was to support welsh business and this it failed to do properly. it was essentially like buying a ferrari but keeping it locked up in the garage for the majority of the year. implausible excuse the management of the project suggested that this failure to hit targets was due to the economic downturn, although this excuse seems implausible given that hpc wales operated during a period of sustained economic growth when record numbers of jobs were created elsewhere in the welsh economy. so was it because there was not enough demand from the pool of 230,000 firms in wales, or did the hpc wales team of advisers and consultants simply fail to persuade enough businesses that this amazing resource could be of value to their commercial interest? to date, no proper explanation has been forthcoming and whilst the failure of hpc wales to reach its targets may not be a matter for the wales audit office, there must be a thorough appraisal by the welsh government into why, following the failure of the techniums, another high profile project has spent tens of millions of pounds with little return to the welsh economy. more importantly, politicians at both ends of the m4 corridor must ensure that this never happens again if public funds are to be used wisely to support innovation and economic growth. & 988 & very low & Low & Socio-Economic & NA & NA & 2016-03-19 & 2016 & 2 & ECO
Frame & v.low & Regional & 500-1000 & -0.7997063 & -1.2431481 & 1.1980777 & 1.4613788 & -0.9363498 & 1.1 & -0.8854249 & -0.3297660 & Payer & Domestic & Domestic & Domestic & Domestic|ECO & Negative\\
UK & http://www.walesonline.co.uk/news/politics/cardiff-mans-home-made-anti-11495203 & 738 & WalesOnline & Private/Non-Public & Online only & Regional/Local & very high = CP is most important issue + CP is mentioned in title/headline & Institutional bargaining over funding & Balanced & EU + National + Subnational & 2.Rich countries pay & Economic development & Positive & EU & NA & NA & NA & NA & NA & UK & this man's anti-brexit poster has become the talk of the uk & 2016-06-19 & structural funds & this retired teacher's home-made eu referendum poster has become the talk of the uk after spreading widely on twitter and being picked up by national newspapers . eugene nowakowski from pontcanna got so fed up with hearing what he describes as the "brexit nonsense" that he decided he wanted to get his point across. the 66-year-old retired teacher, who has lived in cardiff for nearly 30 years, created a poster which read: "i remember what cardiff looked like before we joined the eu. let's not go back to that sh*t hole. stay in!" mr nowakowski put the poster in the front window of his home on wyndham road, and after it was spotted and tweeted by passers-by it has received a huge response online - including being picked up by the independent and the metro . mr nowakowski sent us this picture of himself mr nowakowski said: "i heard billy connolly say something similar when he was touring the uk. i remember watching him on television when he came to cardiff and he said: "last time i was here in the 60s, this place was a sh*t hole. "so i was thinking about that when i done the poster and i thought it was quite appropriate. "when i was a teacher, a lot of students used to say this or that area of cardiff is rubbish and i would just stand there and think you want to see what it looked like before. \#cardiff's best \#remain poster @ilovesthediff @pethepontcanna. gwych pic.twitter.com/pwc0ohunsl -- david clubb (@davidoclubb) june 18, 2016 "when we were lads, cardiff was a different place. it was just a really rough area in parts and you look at it now and it's just beautiful. "i can't say for definite that this is because we joined the eu, but i can't say that it will get any better if we leave. "to be honest i'm fed up of this brexit nonsense. the debate surrounding it is very much impinging on peoples consciousness, especially when race, religion and immigration are being brought into it. "i just wanted to say something about it and put my point across. i didn't expect it to get the response that it did - i've had people stopping to take pictures of it all day." in the past 20 years, welsh urban centres have benefited from £4bn from brussels, particularly in the valleys and cardiff. the poorest areas of wales qualify for eu regional funding, with £1.8bn of european structural funds investment expected for the 2014-2020 period. according to the welsh government: "since 2007, eu projects have created 11,925 enterprises and 36,970 (gross) jobs, assisted 72,700 people into work, 229,110 to gain qualifications, and 56,055 into further learning." the wales governance centre claims that "wales' net benefit from the eu equated to around £79 per head in 2014." what they're saying about the eu referendum stopping brexit is vital - guto harri brexit would be an 'absolute disaster' we need free movement, dylan jones-evans carwyn's furious attack on brexit leave to protect jobs - gisela stuart whetherspoon's tim martin backs brexit sir emyr jones parry geraint talfan davies 1 of 8 & 548 & very high & High & Power & Socio-Economic & NA & 2016-06-19 & 2016 & 2 & POL
Frame & high-very high & Regional & 500-1000 & -0.7997063 & -1.2431481 & 1.1980777 & 1.4613788 & -0.9363498 & 1.1 & -0.8854249 & -0.3297660 & Payer & Domestic & European & Mixed & Domestic|POL & Neutral\\
UK & http://www.chroniclelive.co.uk/news/north-east-news/beamish-museum-project-could-lose-8843749 & 755 & Chronicle Live & Private/Non-Public & Online and Offline & Regional/Local & medium = CP is important part of story (alongside other issues) & Institutional bargaining over funding & Factual & EU + National + Subnational & No myth & NA & NA & NA & NA & NA & NA & NA & NA & UK & beamish museum project could lose out on european development cash & 2015-03-16 & european regional development fund & pleas ring out for power-holders to intervene after government removed county durham attraction from list of erdf beneficiaries chiefs at beamish museum may be forced to downsize an ambitious expansion plan as doubt is cast over european funding. the much-loved heritage site was relying on £4m from the european regional development fund (erdf) to push ahead with 'remaking beamish', a bold 12-year plan that would create 100 jobs. the county durham museum was listed in the north east strategic economic plan as a priority for a share of the money later in the year. however, north east labour mep jude kirton says the government has overruled the north east local enterprise partnership. it means the museum will not get the cash because it is a not-for-profit venture and not a business. a spokesman for the museum said it would be a "terrible missed opportunity" for the region's tourism industry to miss out on the vital fund. the mep is now making an urgent plea to power-holders to intervene and said beamish and other not-for-profit companies are critical to the north east economy. she told the european parliament in brussels this week: "beamish is one of europe's largest open air museums which annually entertains 650,000 visitors. it is a key local employer and a unique selling point for my constituency." she added: "the north east cultural partnership is trying to build on existing economic successes, in part using the eu's erdf funding. but, our uk government has blocked access to eu funding for not-for-profit cultural enterprises. "this brutal blow ignores the social and economic benefits offered to europe by cultural industries, at a time when our constituencies can ill afford to turn down the opportunities for jobs and growth available." the remaking beamish proposal includes a replica of a 1950s town and is forecast to create almost 100 jobs, as well as 50 apprenticeships, and bring in an extra 100,000 visitors to the north east. around £10m for the £17m project has come via a grant from the heritage lottery fund, but the european cash is crucial for the remaking beamish vision to be realised as the museum cannot raise the remainder in full. a museum spokesman said: "beamish is a great example of how investment in heritage and culture can create jobs and growth. "more than half our visitors are tourists from outside the north east - that's around 300,000 people every year spending money in our region's hotels, b\&amp;bs, restaurants and shops. "over the next five years beamish plans to invest £17m to build on our current success - creating 95 new jobs, sustaining an existing workforce of 370, training 50 new apprentices and attracting 100,000 more tourists to our region every year. "it's an ambitious and exciting plan. we know we need to raise the vast majority of this money ourselves - but we are looking to the erdf to help close a shortfall of about £4m and make sure we can start this project - called remaking beamish - next year." the museum is still hopeful the uk government could divert other european cash for the project, based on the nelep's original plan. the spokesman added: "now is the time to all work together to ensure we make the case for investment in our heritage and culture to create jobs and growth. it is absolutely vital that this sector is eligible for eu support once funding is available later this year. it is not yet very clear to us whether it will be eligible or not. "heritage and culture are critical to creating the conditions for growth and innovation - developing a vibrant, distinctive region with a fantastic quality of life. if we failed to support the further development of this sector it would be a terrible missed opportunity." & 649 & medium & Medium & Power & NA & NA & 2015-03-16 & 2015 & 1 & POL
Frame & low-medium & Regional & 500-1000 & -0.7997063 & -1.2431481 & 1.1980777 & 1.4613788 & -0.9363498 & 1.1 & -0.8854249 & -0.3297660 & Payer & Domestic & European & Mixed & Domestic|POL & Neutral\\
\addlinespace
UK & http://www.walesonline.co.uk/news/local-news/grass-roofs-pedestrian-boulevards-feature-13746560 & 706 & WalesOnline & Private/Non-Public & Online only & Regional/Local & very low = CP mentioned once & Jobs & Positive & EU + National + Subnational & No myth & NA & NA & NA & NA & NA & NA & NA & NA & UK & grass roofs and pedestrian boulevards for swansea's the kingsway? & 2017-10-11 & european regional development fund & get daily updates directly to your inbox+ subscribethank you for subscribing!could not subscribe, try again laterinvalid email hundreds of pedestrians every day now have the chance to take a glimpse into the future of swansea's the kingsway's future. hoardings outside the demolished former oceana site include conceptual images that show how the street could look once its transformation into an employment district is finished. also included on the hoardings is artwork related to the council's shortlisted bid for uk city of culture 2021 status. funded by swansea council, a european regional development fund grant and city deal money from the uk government, £12.7m regeneration plans for kinsgway include office developments that it is hoped would open up thousands of jobs and boost spending in city centre businesses. a digital district benefitting from world class digital infrastructure is also planned, as well as better links with areas including oxford street and castle square. councillor rob stewart, swansea council leader, said: "these images show the kind of green, high-tech, pedestrian-friendly environment we have in mind for kingsway in future, although no final decisions have been made on the scheme's shape and form. "we know that far fewer people now live and work in swansea city centre when compared to other cities of a similar size, so we need to redress that balance not just to open up more jobs for local people, but also to generate the footfall our city centre businesses need to thrive. "with the city centre's nightlife having migrated to wind street and uplands, there's an opportunity to transform kingsway into a living and working area to help realise our goals for the benefit of local people. this is why an office development for tech businesses is planned for the oceana site, complementing other building refurbishments in the area." preparations to improve kingsway's landscape will soon start on site according to the council, with the main works programme starting in the spring. this will include the return of two-way traffic on kingsway later in 2018. "we know we have to considerably improve kingsway's look and feel if we're to attract major employers there and create an environment where enterprising young business talent can flourish," added mr stewart. "that's why the metro track will soon be removed as we pave the way for the return of two-way traffic, with many other environmental improvements to follow over the next 18 months or so as well." & 420 & very low & Low & Socio-Economic & NA & NA & 2017-10-11 & 2017 & 2 & ECO
Frame & v.low & Regional & <500 & -0.7997063 & -1.2431481 & 1.1980777 & 1.4613788 & -0.9363498 & 1.1 & -0.8854249 & -0.3297660 & Payer & Domestic & European & Mixed & Domestic|ECO & Positive\\
UK & http://www.theguardian.com/public-leaders-network/2016/may/06/brexit-british-voters-unaware-eu-benefits-public-services & 693 & The Guardian & Private/Non-Public & Online and Offline & National & very low = CP mentioned once & Empowerment of institutions & Positive & Subnational & No myth & Economic development & Positive & No actor & No myth & Jobs & Positive & EU & No myth & UK & british voters unaware of eu benefits for public services, report finds & 2016-05-06 & structural funds & many british voters believe eu membership has no impact on public services, according to a new poll. the poll of 1,002 british adults, commissioned by the chartered institute of public finance and accounting (cipfa) and conducted by comres, found that 40\% of respondents believed that being in the eu is detrimental to the delivery of public services, while 78\% said that uk membership puts pressure on public services. but a high proportion of respondents said it would make no difference to health and social care (46\%), the quality of higher education (60\%) and regional economic development (40\%) whether britain voted to leave or to remain. this is in stark contrast to the views of staff working in health and social care, local government, higher education and elsewhere in the public sector, according to cipfa's new report named treuble and strife, on the impact of eu membership on uk public services. the report is based on 20 in-depth, anonymised interviews with public sector leaders, 19 of whom said that public services would be better off if the uk voted to stay in the eu. uk and eu legislation, policy and economic activity is intertwined, leaving a complicated picture. one leader from the higher education sector is quoted in the report: "because britain has been a part of the eu for 40 years it's pretty complex to work out what are the influences separate from everything else. it's difficult to work out what's eu-related and what's uk government-related." eurosceptics argue that the eu's free movement policy places a financial strain on public services, that eu regulations create unnecessary costs, and that the uk's contribution to the eu budget - a net amount of £5.7bn in 2014 (pdf) - could be spent on public services instead. eu benefits but according to the cipfa report, the public sector has benefitted from access to a wider pool of skills and talent because of the free movement of workers within the eu. it states that 10\% of nhs health and social care professionals (pdf) are from countries within the european economic area (eea), while around 15\% of academic staff in uk universities are from the eu (pdf). european legislation is credited with improving working conditions, and eu structural funds have helped create more than 50,000 jobs in poorer regions of the uk. the report also argues that the uk balance sheet relies heavily on economic stability, so a vote to leave could see a downturn in public spending. "if we had an exit and it had a severe negative impact on the british economy then you would expect that it would filter down through austerity measures to local government," said one survey respondent. there is also a risk that the value of public sector pensions would decrease following a vote to leave the eu. health there's a fear among staff interviewed by cipfa that a leave vote could lead to staff shortages in the health sector. in social care, for example, migrant workers are more willing to work the unsocial hours required to provide 24/7 care, oxford university research suggests. one health professional quoted in the report said: "the ability for the travel of the labour across into our country is critical and they add a significant added value to the expertise, not just at a clinical level but across the pool." cipfa's report suggests that eu formal and informal networks have had a positive impact, allowing staff to learn from other countries and work together on key health issues such as obesity. the eu working time directive, which regulates work time and rest periods, has had a significant impact on the health sector workforce - though it has been criticised for leading to reduced hours and inflexibility. local government local government leaders in many of the uk's biggest cities have defended the influence the eu has on local services, but that hasn't stopped havering council becoming the first to vote to leave the european union, in january this year. many in the sector are worried about the impact of having to accommodate more migrants, and some believe that eu migration is part of the problem. some are also annoyed that local government is subject to eu procurement rules, which ensure free access on competition across member states, meaning that uk suppliers do not get priority for uk tenders. but, according to the report, others argue that eu membership cultivates growth - the uk will receive £5.3bn in structural funds for 2014-2020 - which in turn encourages devolution. environmental targets are often set at eu level and carried out by local government, and local areas have seen the benefits of policies on pollution, clean beaches and protected nature zones. a complex picture rob whiteman, chief executive of cipfa, said: "jobs, healthcare, defence and all the issues we care about rely on public services that are deeply interlinked with eu membership. this message has not got through strongly enough. "our research shows an extremely complex picture. overall, in the research amongst public service leaders, respondents considered the benefits of eu membership outweigh the drawbacks. what is abundantly clear, is that decoupling the british state from the eu will cause tremendous upheaval for public services for many years." the report concludes that it is crucial people are provided with detailed and balanced information. "it is of the utmost importance then that the voters get to hear better informed arguments that clearly demonstrate and consider the impact the eu has on public services rather than emotionally driven arguments that would seem to be founded on, often, false perceptions rather than reality," it said. talk to us on twitter via @guardianpublic and sign up for your free weekly guardian public leaders newsletter with news and analysis sent direct to you every thursday. & 977 & very low & Low & Power & Socio-Economic & Socio-Economic & 2016-05-06 & 2016 & 2 & POL
Frame & v.low & National & 500-1000 & -0.7997063 & -1.2431481 & 1.1980777 & 1.4613788 & -0.9363498 & 1.1 & -0.8854249 & -0.3297660 & Payer & Domestic & Domestic & Domestic & Domestic|POL & Positive\\
UK & https://www.dailymail.co.uk/sciencetech/article-6358685/Drilling-begin-UK-s-geothermal-power-plant.html & 793 & Daily Mail Online & Private/Non-Public & Online and Offline & National & very low = CP mentioned once & Environment/green/low-carbon & Positive & National + Subnational & No myth & NA & NA & NA & NA & NA & NA & NA & NA & UK & drilling starts at the uk's first deep geothermal electricity plant & 2018-11-06 & european regional development fund & drilling will start this week at what could become the uk's first deep geothermal electricity plant in cornwall. two wells will be drilled through hot granite rock near st day, the deepest of which will reach 2.8 miles (4.5km) deep and could power 3,000 homes. the firm running the project, geothermal engineering, says the aim is to demonstrate the potential of geothermal technology to produce electricity and renewable heat in the uk. drilling will start this week at what could become the uk's first deep geothermal electricity plant in cornwall (pictured). two wells will be drilled through granite rock near st day, the deepest of which will reach 2.8 miles (4.5km) deep it is believed that the plant at the united downs industrial estate has the potential to supply up to 3mwe (mega watt electrical) of electricity. once drilling at the site is complete, water will be pumped from the deepest well at a temperature of approximately 190c (374f). this water will be fed through a heat exchanger at the surface and re-injected into the ground to pick up more heat from the rocks in a continuous cycle. the extracted heat will be converted into electricity and supplied to the national grid. geothermal technology is described as a 'continuous' energy source because it does not suffer from peaks and troughs experienced by other sustainable power sources. this water will be fed through a heat exchanger at the surface and re-injected into the ground to pick up more heat from the rocks in a continuous cycle. the extracted heat will be converted into electricity and supplied to the national grid developers hope the technology used at the facility could be used in other locations in cornwall and devon which are home to 'hot rocks' (pictured) developers hope the technology used at the facility could be used in other locations in cornwall and devon. similar plants have been developed at insheim and landau in germany. dr ryan law, managing director of geothermal engineering, said geothermal resources have the potential to deliver up to 20 per cent of the uk's electricity and heat energy needs. he said: 'it is incredibly exciting to see this pioneering project getting off the ground in what we hope will be the start of many similar initiatives across the uk.' the £18 million (\$24m) project has received £10.6 million (\$13.8m) funding from the european regional development fund, £2.4 million (\$3.1m) from cornwall council and £5 million from private investors. cut your energy bills if you have been stuck with the same provider for some time, chances are you could shave hundreds off your energy bills. millions of households are sat on their provider's most expensive, out-of-contract deals. but switching to a better deal can instantly save you money. according to this is money and mailonline's expert partner service, energyhelpline, one in ten families could cut their annual dual fuel bill by £537 a year by ditching an switching. you can do your own postcode comparison in minutes using the tool above - or here - to find the best price. & 528 & very low & Low & Socio-Economic & NA & NA & 2018-11-06 & 2018 & 3 & ECO
Frame & v.low & National & 500-1000 & -0.7997063 & -1.2431481 & 1.1980777 & 1.4613788 & -0.9363498 & 1.1 & -0.8854249 & -0.3297660 & Payer & Domestic & Domestic & Domestic & Domestic|ECO & Positive\\
UK & https://www.chroniclelive.co.uk/business/business-news/nels-busiest-ever-quarter-takes-15641184 & 774 & Chronicle Live & Private/Non-Public & Online and Offline & Regional/Local & very low = CP mentioned once & Economic development & Positive & Subnational & No myth & NA & NA & NA & NA & NA & NA & NA & NA & UK & nel's busiest ever quarter takes investment values past £2m mark & 2019-01-07 & european regional development fund & nel fund managers made 19 investments in the last three months of the year get business updates directly to your inboxsubscribesee our privacy noticemore newslettersthank you for subscribingwe have more newslettersshow mesee our privacy noticecould not subscribe, try again laterinvalid email regional fund management firm nel fund managers is striding into the new year on the back of its busiest ever quarter. nineteen investments worth a total of £1,185,000 were completed by nel in companies in the last three months of 2018 from the two funds it manages which form part of the overall £120m north east fund supported by the european regional development fund. the fourth quarter investments also took the total amount invested so far by nel across tyne and wear, durham and northumberland from the north east small loan fund and the north east growth capital fund past the £2m mark. firms which received investments at the end of 2018 include north tyneside technology firm cleanily, northumberland interior design business troynorth and rowlands gill-based commercial fit out firm rosebirch. nel is now continuing to encourage regional firms that want to grow and create new jobs to put their business plans and investment ideas forward to gain a slice of the two funds. the £9m north east small loan fund typically offers loans to businesses of between £10,000 and £50,000, and is designed to assist with the creation of over 1,200 new regional jobs in more than 320 smes over the life of the fund. the £18m north east growth capital fund, meanwhile, has been designed to create over 800 jobs in more than 70 regional firms over life of the fund, and offers unsecured investments of up to £500,000 to established businesses looking to realise their growth potential. yvonne gale, chief executive at nel fund managers, said: "the response to both funds has been encouraging since they went live, but we saw a real increase in demand in the final quarter of 2018 and our team has been working flat out to get as many investments finalised as possible. "enquiries have been coming in from right across our investment area, and applications to the small loan fund in particular have reflected the wide diversity of the north east business community. "our investments are already having a direct impact on the capacity of the businesses with which we work to grow and create new jobs, but there's still a lot more we can do. "north east entrepreneurs are clearly committed to putting development plans into action for their businesses by accessing the capital they need to take them forward, and we're expecting to continue building on the momentum created at the back end of last year during 2019 and beyond." & 461 & very low & Low & Socio-Economic & NA & NA & 2019-01-07 & 2019 & 3 & ECO
Frame & v.low & Regional & <500 & -0.7997063 & -1.2431481 & 1.1980777 & 1.4613788 & -0.9363498 & 1.1 & -0.8854249 & -0.3297660 & Payer & Domestic & Domestic & Domestic & Domestic|ECO & Positive\\
UK & http://www.leicestermercury.co.uk/news/business/leicestershires-hotels-zoos-farm-parks-395503 & 768 & Leicester Mercury & Private/Non-Public & Online only & Regional/Local & very low = CP mentioned once & Economic development & Positive & Subnational & No myth & NA & NA & NA & NA & NA & NA & NA & NA & UK & leicestershire's hotels, zoos and farm parks to get extra help & 2017-08-29 & european regional development fund & tourism already major employer in leicester and leicestershire - providing 17,000 full-time jobs tourism businesses are being offered free specialist support to help them grow. small and medium-sized attractions and accommodation and hospitality businesses will soon be able to take advantage of the advice, designed to help them attract more business and contribute even more to the local economy. the project launches next month and aims to support 130 businesses across leicester and leicestershire. it will offer one-to-one support on top of workshops focusing on topics such as making the most of social media and consumer review websites. delivered by birmingham tourism consultancy winning moves, in partnership with leicester shire promotions, it is funded by leicestershire county council, leicester city council and the european regional development fund (erdf). county council leader nick rushton said: "tourism is a hugely important sector in terms of attracting visitors, businesses and providing a great place to live. "this project will provide valuable support to the many small and medium sized enterprises that make up this growing sector." leicester mayor sir peter soulsby said: "this latest contract to help and support the tourism and hospitality sectors will help us to build on the recent momentum which has helped make leicester such a visitor destination. "i hope this work gives a real helping hand to small and medium businesses across the city and county." dr sharon redrobe, chief executive at twycross zoo, said: "as chair of the tourism advisory board, and board member of the leicester and leicestershire enterprise partnership (llep), i am more than aware of the importance of this sector to the local economy. "tourism is a major employer in leicester and leicestershire, providing over 17,000 full-time equivalent jobs, with a further 4,749 supported by indirect tourism revenue. "i am absolutely delighted to see this investment in supporting our sector. "in the last six years, the value of the sector has grown by 33 per cent to £1.762 billion and with support from the llep, city and county i'm sure we will see that successful growth continue." the project is part of a wider business support programme called collaborate for business growth which has been awarded £3.1 million through the erdf. the collaborate programme provides a range of sector support, grants for small and medium-sized businesses and inward investment. it is led by leicester city council in partnership with leicestershire county council, east midlands food and drink forum and east midlands chamber of commerce. businesses can find out more by clicking here. & 429 & very low & Low & Socio-Economic & NA & NA & 2017-08-29 & 2017 & 2 & ECO
Frame & v.low & Regional & <500 & -0.7997063 & -1.2431481 & 1.1980777 & 1.4613788 & -0.9363498 & 1.1 & -0.8854249 & -0.3297660 & Payer & Domestic & Domestic & Domestic & Domestic|ECO & Positive\\
\addlinespace
UK & http://www.independent.co.uk/news/uk/politics/brexit-latest-charities-funding-medical-research-house-of-lords-lose-a7650021.html & 753 & The Independent & Private/Non-Public & Online and Offline & National & medium = CP is important part of story (alongside other issues) & Institutional bargaining over funding & Positive & EU + National & No myth & NA & NA & NA & NA & NA & NA & NA & NA & UK & brexit: charities will lose £200m a year unless the government steps in, a parliamentary committee warns & 2017-03-26 & european social fund & brexit will deliver a £200m a year hit to charities unless the government steps in to plug the gap, a parliamentary committee warns today. small charities will be punished hardest by the loss of vital grants, mostly from the european social fund (esf), peers suggest. their report highlights fears that medical research charities - for conditions including cancer, heart disease, bone disease and obesity - will lose "considerable research funding and opportunities for collaboration". read more tens of thousands take to streets to demand brexit be reversed and it calls on ministers to publish a full assessment of the likely financial blow by the end of this year. "this should include the impact of loss of funding as well as on research collaboration," the house of lords committee on charities said. "the anticipated loss of funding when the uk leaves the european union will impact on smaller charities the most, yet it is these that are already under the most significant funding pressures," it added. however, the government has given no guarantee that grants from the esf and other eu investment programmes will be picked up by the british taxpayer after 2020. even before then, only schemes which ministers believe offer proper value for money will continue to receive funding after brexit, the treasury said. in evidence to the committee, the brain tumour charity told of its "concern that the uk government would not be able to guarantee the level of funding currently leveraged from the eu". the royal mencap society described the esf as a "major source of revenue" to help raise living standards and provide job opportunities for young people and the long-term unemployed. and the british heart foundation pointed out that the uk received the second highest sum of all eu states from horizon 2020, the research and innovation programme. when questioned by the peers, rob wilson, the civil society minister, gave no guarantee to charities but insisted he was "listening to their concerns". "we recognise that charities will be affected by exiting the eu and there are a broad range of implications," he admitted. projects showing "strong value for money" would still be funded. elsewhere, the committee also said it had "grave concerns" about the charity commission's proposal to charge charities an annual registration fee. and it criticised the government for causing "unnecessary concern" with an "anti-advocacy" clause in 2014 laws which "threatened the vital advocacy role of charities". baroness pitkeathley, the committee's chairwoman, said: "charities are the lifeblood of society. they play a fundamental role in our civil life and do so despite facing a multitude of challenges. "yet for them to continue to flourish, it is clear that they must be supported and promoted. we found that charities lead the way with innovation, but that this is at risk of being stifled by the 'contract culture'. "and, while advocacy is a sign of a healthy democracy, and is a central part of charities' role, this role has been threatened by government." sir stuart etherington, chief executive of the national council for voluntary organisations, seized on the committee's concerns over the commission's plans to charge charities a fee. "but the lords have been unambiguous in saying that they do not think the charity commission have properly considered their plans to levy fees on charities," he said. more about: brexit european union house of lords charities & 564 & medium & Medium & Power & NA & NA & 2017-03-26 & 2017 & 2 & POL
Frame & low-medium & National & 500-1000 & -0.7997063 & -1.2431481 & 1.1980777 & 1.4613788 & -0.9363498 & 1.1 & -0.8854249 & -0.3297660 & Payer & Domestic & European & Mixed & Domestic|POL & Positive\\
UK & http://www.theguardian.com/lifeandstyle/2014/sep/03/how-to-set-up-a-community-garden & 721 & The Guardian & Private/Non-Public & Online and Offline & National & very low = CP mentioned once & Civic participation/collaboration & Positive & Subnational & No myth & NA & NA & NA & NA & NA & NA & NA & NA & UK & how to set up a community garden & 2014-09-03 & european social fund & in the past, the areas of moss side and hulme in manchester haven't had a good reputation. the initial founder and current chair of hulme community garden, richard lockwood, realised that something positive could be done to turn an area of deprivation into somewhere to encourage healthy living through gardening and food growing. what happens? the centre is open to the public every day, running sessions with a variety of people, from service users (who pay a small fee to come and be involved in a range of activities to learn life skills), to mothers and toddlers, and to probationers. mondays and tuesdays are solely for service users, but wednesdays are open to everyone: "lots of men who used to work as labourers come along because they miss the graft," says rachel summerscales, manager of the garden. on thursdays, there is a group for mums and toddlers; around 20 to 30 go along and do a range of suitable activities, such as pond dipping and bug hunting. "there is a day when we have all three groups - site users, probationers and the mum and toddler group - on site at the same time working on different projects and it does work," says rachel. "it's not just a garden with a garden centre. it's hard to explain without seeing it, and i know it sounds a bit naff, but we really do get people coming here saying: 'wow! this has changed my life.'" how many people involved? hulme community garden has had 5,000 people involved over the years, including volunteers in training or those running workshops. there are about 150 regular volunteers, but 50 is the norm for a single month. does the group get funding? at present, they are working with siemens to build build two underground rainwater reserves which will pump water around to parts of the garden that don't get enough. laing o'rourke, hewden, and the university of sheffield are also helping with a shipping container conversion which will provide space for volunteers to have a rest. as part of a corporate challenge, rbs are helping to develop a composting area in the garden. "as we have expanded, we have had to get more people on board and this has come at a cost," explains summerscales. "we had two funds; one lasted for five years and another lasted for another five, but we have just come to the end of those time periods and fallen off the edge a little. the recession has really had an impact on us because so much money has been cut on social services and nhs health provisions that previously would have offered services to those using the garden." the garden is hoping to get some more funding through applications to the european social fund and awards for all. what would they like to do next? "we would like to start doing more with after-school groups and there are some bushcraft-type workshops we would like to run," says rachel. the garden is also interested in developing a way to measure the impact of the garden. rachel: "we need to invest in a decent database. we have received a couple of grants for this, but money has to be spent on so many things like computers. we are getting there with business partners but it would be great if, rather than having 50 people down from an accountancy firm to repot plants, they could come and help us go through our accounts!" what can you do to help? "we are looking for someone who can act as a coordinator between us and the younger demographic; someone around 25 years old who can help us make hcgc a place where young people can come and do some really solid and constructive work experience." aside from signing up to the garden's newsletter, and following them on facebook and twitter, donations are always really appreciated. "in 2015, we're going to start a kickstarter crowdfunding campaign to help towards the costs," says summerscales. "we have these ethics that we like to stick to in every aspect of what we do - all our tea is fairtrade, all our loo roll is recycled - but all these things are incredibly expensive, so the kickstarter campaign is going to be about 'money for loo roll'. can i set something like this up in my area? the royal horticultural society runs the campaign for school gardening, working with more than two-thirds of all uk schools to help them make the most of their outdoor space and engage the local community. the scheme is free and comes with numerous resources and support, including a team of advisers who work directly with school and communities. you can sign up to the scheme here. the wild network's wild time app (available on android and ios) suggests timed activities for children and their carers to get outdoors and make the most of their time in nature. how can people get involved? the federation of city farms and gardens also offers advice on all aspects of community-managed gardening. they can put you in touch with a community garden within travelling distance from you and can give advice over the phone or on site. you can also download one of their starter packs (or order a hard copy for £6). "my advice would be to establish a solid team of really dedicated staff and allow a lot of time for administrative things," says summerscales. "you have to safeguard against many variables, then there's health and safety, and collecting information so you can report back to your funders ... there is a lot of work that needs to get done that isn't just about being out digging in the sun." & 963 & very low & Low & Socio-Economic & NA & NA & 2014-09-03 & 2014 & 1 & ECO
Frame & v.low & National & 500-1000 & -0.7997063 & -1.2431481 & 1.1980777 & 1.4613788 & -0.9363498 & 1.1 & -0.8854249 & -0.3297660 & Payer & Domestic & Domestic & Domestic & Domestic|ECO & Positive\\
UK & http://www.chroniclelive.co.uk/business/business-news/northumbria-university-best-uk-businesses-12953072 & 788 & Chronicle Live & Private/Non-Public & Online and Offline & Regional/Local & very low = CP mentioned once & Research \& innovation & Positive & Subnational & No myth & NA & NA & NA & NA & NA & NA & NA & NA & UK & northumbria university best in uk for businesses started by its graduates & 2017-04-27 & european regional development fund & companies started by northumbria graduates turned over £69m and employed more than 1,000 peopleget business updates directly to your inbox+ subscribethank you for subscribing!could not subscribe, try again laterinvalid email a north east university has come top of uk rankings for the turnover of businesses started by its graduates. northumbria university has risen to the top slot in the higher education business and community interaction survey after turnover from its graduates in 2015-6 reached £69.2m, more than £25m ahead of the nearest rival. newcastle university was also in the national top ten, with the £20.1m turnover of its graduates' businesses putting it in eighth position. the survey also showed that northumbria's graduates' businesses employed 1,050 people last year, the second highest figure in the country. the university has hailed the figures as a sign that its northumbria enterprise business support project scheme (nebs), which recently secured a £1.2m european regional development fund grant, is helping its students launch new ventures, grow their businesses and provide employment for them and others. lucy winskell, pro vice-chancellor at northumbria, said: "our track record for graduate enterprise is second to none. entrepreneurial talent, energy and ambition run deep among our students and graduates, and we aim to support this by providing the best possible learning experience and opportunities for them to succeed as global graduates. "this latest report by hebcis bears this out and shows that through our student and graduate enterprise service and initiatives like nebs we are leading the way. the fact that so many of our start-ups are not only surviving but continuing to prosper beyond the crucial first three years is also testament to the quality of our support." graham baty, enterprise manager at northumbria, said: "the emphasis is on helping students and graduates create sustainable businesses and not on the number of new start-ups. "through the nebs project, the university is able to provide mentoring from industry experts to get start-ups to the point of trading. perhaps more importantly, the support continues even when businesses surpass the 12 months stage as this is where problems can arise." one of the successful businesses set up by a northumbria graduate was best student halls, which is run by owen dixon, who completed a business studies degree in 2014. the business, which is a student accommodation comparison service, has over 100,000 rooms listed on its site and plans to add international properties in the near future. mr dixon said: "the support i've received from northumbria has really helped me to focus on growing the business at a much earlier stage than if i was doing this alone. i've been able to benefit from their free hatchery space in the city centre as well as bespoke mentoring that is really helping my business reach the next level." & 479 & very low & Low & Socio-Economic & NA & NA & 2017-04-27 & 2017 & 2 & ECO
Frame & v.low & Regional & <500 & -0.7997063 & -1.2431481 & 1.1980777 & 1.4613788 & -0.9363498 & 1.1 & -0.8854249 & -0.3297660 & Payer & Domestic & Domestic & Domestic & Domestic|ECO & Positive\\
UK & http://www.walesonline.co.uk/news/politics/chancellor-told-deliver-880m-avoid-12179987 & 740 & WalesOnline & Private/Non-Public & Online only & Regional/Local & very low = CP mentioned once & Institutional bargaining over funding & Factual & EU + National + Subnational & No myth & NA & NA & NA & NA & NA & NA & NA & NA & UK & chancellor told to deliver £880m to avoid a betrayal of wales & 2016-11-15 & european structural and investment funds & five welsh mps have challenged the chancellor to honour pledges made during brexit campaignget daily updates directly to your inbox+ subscribe thank you for subscribing!could not subscribe, try again laterinvalid email chancellor philip hammond has been challenged to boost the cash that comes to wales by £880m a year in the wake of brexit referendum. five welsh mps have warned that anything less would be a "betrayal of the wishes of the people of wales". they argue that the vote leave campaign encouraged voters to believe that brexit would result in an extra £350m a week spent on health - which they say would result in an extra £880m a year going to the welsh government. welsh mps say: 'show us the money' the labour mps want mr hammond to commit to the increase when he delivers next week's autumn statement. during the referendum campaign the vote leave battle bus bore the slogan: "we send the eu £350m a week. let's fund our nhs instead." in april, liam fox - who is now secretary of state for international trade - made a direct appeal to welsh voters, saying: "instead of handing over £350m a week to brussels we should be spending that money on local priorities like the welsh nhs, which has faced a billion pounds of cuts by the welsh labour government since 2011." labour mps chris evans (islwyn), david hanson (delyn), carolyn harris (swansea east), stephen kinnock (aberavon) and owen smith (pontypridd) have called for the chancellor to deliver the cash. they write: "we believe in a wales with an excellent, well-funded public sector that provides a world-class service to the people of wales, pays its hard-working staff well, and treats them with respect. "this was the vision promised by your cabinet colleagues who campaigned for a leave vote in the eu referendum. vote leave promised that, if britain left the eu, £350m a week extra would be spent on the nhs... "we accept the verdict of the british people. yet if this mandate is to mean anything, it must include the single most visible promise of the leave campaign - spending £350m more a week on the nhs." 'anything else will be a betrayal of the wishes of the people of wales' urging him to use his first autumn statement to make a major announcement, they state: "we are calling on you to commit to giving wales its share, which the house of commons library has calculated to be £880m, as soon as this money becomes available by leaving the european union. this additional funding must be over and above the amount that is currently planned to be spent on the national health service. "anything else will be a betrayal of the wishes of the people of wales. we challenge you, when you stand up in the house of commons on november 23, to show us the money and commit to vote leave's promise; or explain why you cannot, and why your cabinet colleagues so cynically misled the people of wales." according to vote leave watch, the commons library estimates show that an extra £350m a week of health spending would boost funding in scotland by £1.58bn, in wales by £0.88bn, and in northern ireland by £0.59bn. a welsh government spokesman said: "the leave campaign made clear wales would not lose a penny as a result of brexit. indeed, promises were made to the people of wales that significant extra investment would be made in our public services if the uk voted to leave the eu. "following the vote to leave, we, and the people of wales, expect those promises to be kept and delivered." autumn statement analysis by david williamson the autumn statement is chancellor philip hammond's biggest opportunity to set the agenda for the government since he came to the post in july. he has already signalled a change in course from his predecessor, george osborne. gone is the goal of balancing the nation's books by the end of this parliament, which in turn will give him the freedom to borrow more. mr hammond is also not expected to pursue mr osborne's vision of cutting corporation tax to 15\% - a 17\% rate by 2020 seems more likely. instead, he is strongly expected to announce investments in rail and roads. his core challenge is to reassure his audience across britain and the rest of the world that the country can weather the financial challenges posed by brexit. this is his chance to put his stamp on the exit process. he is regularly portrayed as favouring a "softer" version of brexit than the vote leave veterans who now occupy senior positions around the cabinet table. but for about an hour on november 23 he will command the limelight. tory mps have been doing their best to remain united, even though a majority wanted to stay in the eu. but as the prospect of formal exit talks looms, mps who fear the consequences of a clean break with the eu could find mr hammond a figure to rally around. he has no shortage of people offering advice about measures he should include in the autumn statement. the joseph rowntree foundation wants a rebalancing fund to boost britain's regions by supplying "at least an equivalent level of funding to that committed to esif (european structural and investment funds)". however, he will not have much financial elbow room. the institute for fiscal studies has warned that "lower growth forecasts will push up forecast government borrowing and debt". it cautioned that by 2019-20, "lower growth could result in tax revenues being £31bn lower than forecast in the budget." if the growth forecasts from the office for budget responsibility are grim and it looks as if austerity will stretch well into the next parliament, mr hammond will find it hard to paint himself as the architect of a booming economy. plaid cymru is pushing for a change of course jonathan edwards, the party's treasury spokesman, said: "it is now almost seven years since the tories first promised to eliminate the deficit through cutting spending. we were told that the deficit would be eliminated by the end of the previous parliament, which was almost two years ago, but of course, the deficit is nowhere near eliminated... "the chancellor needs to accept that the tories' so-called 'long-term economic plan' has failed, and change course. he should use his first autumn statement to make that bold decision." a treasury spokesman said: "sustainable public finances are necessary to build an economy that works for the whole of the uk, including wales, and the chancellor has been clear that we will return the budget to balance in a way that allows us the space to support the economy as needed. "the fundamentals of the uk economy are strong, and we are well-placed to deal with the challenges and take advantage of opportunities ahead." & 1161 & very low & Low & Power & NA & NA & 2016-11-15 & 2016 & 2 & POL
Frame & v.low & Regional & +1000 & -0.7997063 & -1.2431481 & 1.1980777 & 1.4613788 & -0.9363498 & 1.1 & -0.8854249 & -0.3297660 & Payer & Domestic & European & Mixed & Domestic|POL & Neutral\\
UK & http://www.theguardian.com/uk-news/2016/jun/25/view-wales-town-showered-eu-cash-votes-leave-ebbw-vale & 698 & The Guardian & Private/Non-Public & Online and Offline & National & high = CP is most important issue in story (can also cover other issues) & Institutional bargaining over funding & Negative & EU + National + Subnational & 2.Rich countries pay & Financial burden & Negative & EU + National & 2.Rich countries pay & Economic development & Positive & EU + National + Subnational & No myth & UK & view from wales: 'but we put in more money than we get out, don't we?' & 2016-06-25 & european social fund & the eu has lavished millions on regenerating the deprived former steel town of ebbw vale, but 62\% still voted to leave "what's the eu ever done for us?" zak kelly, 21, asks me this standing next to a brand new complex of buildings and facilities that wouldn't look out of place in canary wharf. it's not canary wharf, though, it's ebbw vale, a former steel town of 18,000 people in the heart of the welsh valleys, where 62\% of the population - the highest proportion in wales - voted leave. to go there - along a new dual carriageway - and stand next to the town's new sixth form and training college, a glass and steel architectural showpiece next to its new leisure centre, a few hundred yards away from a new train station, is to stare into the abyss of the uk's failed remain campaign. even kelly, who has just finished a training session on a brand new football pitch, backtracks slightly after asking that question. "well, i know ... they built all this," he says, and motions his head at the impressive facilities that are all around us. "but we put in more money than we get out, don't we?" we're standing on the site of the old steelworks, a toxic industrial wasteland left rotting when the plant, once the biggest in europe, finally closed in 2002. it's now "the works" - a flagship £350m regeneration project funded by the eu redevelopment fund and home to the £33.5m coleg gwent, where some of the 29,000 welsh apprenticeships the european social fund pays for help young people learn a trade. add in a new £30m railway line and £80m improvement to the heads of the valley road from other pots of eu money, and the town centre has just received £12.2m for various upgrades and improvements. ebbw vale, left devastated when the steelworks closed, has had more european money poured into it than perhaps any other small town in britain. but according to the figures kelly heard, "we get out £7m a year from the eu and we put in £19m". anyway, he says, "it was time for a change". and change is now coming. but what it will mean for an area dependent on inward investment and with the highest unemployment in wales - nearly 40\% of people are either unemployed or not available for work - has yet to be seen. in the local fish and chip shop, deborah basini says that she voted remain. "all my family did. i'm very worried about what's going to happen to inward investment. i'm 60 - this isn't going to affect me. it'll be my grandchildren who are not yet born." her customers, however, thought differently. "there was only one word people had on their mind: immigration. they didn't look at the facts at all." are there any immigrants in ebbw vale? "no! hardly any. and the ones there are are all working, all contributing. it's just ... illogical. i just don't think people looked at the facts at all." it's a town with almost no immigrants that voted to get the immigrants out. a town that has been showered with eu cash that no longer wants to be part of the eu. a town that holds some of the clues, perhaps, in understanding quite how spectacularly the remain message failed to land. there's a sense of injustice that is far greater than the sum of the facts, and the political landscape has fractured and split. zak kelly says that many of his friends, in what is nye bevan's old constituency, voted ukip. wales isn't just a net eu beneficiary, eu capital funding has been an essential part of attracting firms to come here. all around town are signs marked with the eu flag for the ebbw vale enterprise zone. the website notes that as an eu tier 1 area, "companies can benefit from the highest level of grant aid in the uk". earlier this year the sports car company tvr announced it would build a factory and create 150 jobs there. will it still come? will the circuit of wales, a multimillion-pound motor racing circuit a private company has been proposing to build on the town's outskirts creating 6,000 jobs? will the £1.8bn of eu cash promised to wales for projects until 2020 still arrive? and what happens after? will central government really give more money to ebbw vale than the eu has? even kelly looks like he could be doubtful on this point. "david cameron got a good kicking," he says. so, what about boris johnson? do you want him? "no way. he's london through and through. he'll just forget about wales." or as michael sheen, the welsh-born actor from port talbot, tweeted: "wales votes to trust a new and more rightwing tory leadership to invest as much money into its poorer areas as eu has been doing." "it is what it is," says kelly. "we'll see, won't we?" & 855 & high & High & Power & Values & Socio-Economic & 2016-06-25 & 2016 & 2 & POL
Frame & high-very high & National & 500-1000 & -0.7997063 & -1.2431481 & 1.1980777 & 1.4613788 & -0.9363498 & 1.1 & -0.8854249 & -0.3297660 & Payer & Domestic & European & Mixed & Domestic|POL & Negative\\
\addlinespace
UK & http://www.dailyrecord.co.uk/news/politics/snp-accounting-blunder-loses-45million-6297557 & 727 & Daily Record & Private/Non-Public & Online and Offline & Regional/Local & medium = CP is important part of story (alongside other issues) & Mismanagement & Negative & EU + National + Subnational & No myth & NA & NA & NA & NA & NA & NA & NA & NA & UK & snp's accounting blunders lose £45million for scotland's poorest & 2015-08-22 & european social fund & euro bosses have frozen social fund cash handouts after "irregularities" and lack of proof the scottish government will spend it properly. millions of pounds of euro cash to help scotland's poor has been blocked because the scottish government can't prove it will be spent properly. the european commission suspended £45million of payments because of accounting "irregularities", it emerged yesterday. the snp ­government have had seven months to sort out the mess with the european social fund. the fund is used to train unemployed people with the aim of alleviating poverty. payments of £41.4million for projects in the lowlands and uplands and £4.8million for the highlands and islands have been suspended. that's nearly a quarter of the £193million awarded to scotland in the 2007-13 spending round. the european commission said: "the commission has taken the decision to suspend payments from the european social fund for scotland. "member states have the ­obligation to ensure that eu money is spent properly and that all procedures and documents respect the rules set out under the structural funds." the ­statement added that problems were first highlighted in december and have not been resolved despite "extensive dialogue". it added: "the commission decided that there is insufficient assurance that all the measures to rectify the problems have been taken. the commission has therefore adopted a suspension of payments for these two programmes." scottish tory enterprise spokesman murdo fraser said it was another "blunder" by the scottish government. he added: "it's yet another example of the snp getting their sums wrong. "not only will social projects now lose out on vital funding, but many will now see scotland as a country which cannot handle its own financial matters." the scottish government blamed other public bodies for the problem. a spokesman said: "no money will be lost to scotland as a result of this process. "current interruptions to programme payments were prompted by some public bodies failing to comply with their audit obligations. this is unacceptable and is being addressed." try your hand at our quick news quiz: & 346 & medium & Medium & Governance & NA & NA & 2015-08-22 & 2015 & 1 & POL
Frame & low-medium & Regional & <500 & -0.7997063 & -1.2431481 & 1.1980777 & 1.4613788 & -0.9363498 & 1.1 & -0.8854249 & -0.3297660 & Payer & Domestic & European & Mixed & Domestic|POL & Negative\\
UK & http://www.bbc.co.uk/news/uk-wales-politics-31921381\#sa-ns\_mchannel\%3Drss\%26ns\_source\%3DPublicRSS20-sa & 771 & BBC & Public & Online only & National & very low = CP mentioned once & Economic development & Positive & Subnational & No myth & NA & NA & NA & NA & NA & NA & NA & NA & UK & small business scheme in wales gets £44m funding - bbc news & 2015-03-18 & european regional development fund & small firms in wales will be encouraged to get bigger with an extra £44m for a scheme advising them on expansion. ministers said the small and medium sized enterprises support service will run for another five years as part of the business wales scheme. the advice service, launched two years ago, offers help on issues like getting finance and developing markets abroad. £26m of the funding comes from the european regional development fund and £18m from welsh ministers. & 78 & very low & Low & Socio-Economic & NA & NA & 2015-03-18 & 2015 & 1 & ECO
Frame & v.low & National & <500 & -0.7997063 & -1.2431481 & 1.1980777 & 1.4613788 & -0.9363498 & 1.1 & -0.8854249 & -0.3297660 & Payer & Domestic & Domestic & Domestic & Domestic|ECO & Positive\\
UK & https://www.chroniclelive.co.uk/business/business-news/ignite-accelerator-programme-returns-newcastle-14961398 & 695 & Chronicle Live & Private/Non-Public & Online and Offline & Regional/Local & very low = CP mentioned once & Jobs & Positive & Subnational & No myth & Research \& innovation & Positive & Subnational & No myth & NA & NA & NA & NA & UK & ignite accelerator programme returns to newcastle after three years & 2018-07-29 & european regional development fund & up to 20 tech firms will receive up to £15,000 in investment after joining the scheme get business updates directly to your inbox+ subscribesee our privacy noticethank you for subscribing!could not subscribe, try again laterinvalid email a popular start-up accelerator programme is returning to newcastle for the first time in three years, and is looking for tech firms to invest in. ignite was the first accelerator programme to be launched in newcastle when it originally opened in 2011. the scheme proved successful with the local tech scene and helped support many of the entrepreneurs that now run businesses around the city. ignite's next accelerator is returning to the north east this october and will offer three months' support and office space to those who join the scheme. the tech firms involved will also receive £15,000 of investment to help develop their business. tristan watson, founder of ignite, said: "we are really excited to be returning home to newcastle where we launched in 2011. since then we've delivered 16 programmes all around the uk, we've supported hundreds of companies and invested in over 140 of them. "it's three years since we last ran a programme in newcastle, we know there are so many great entrepreneurs and innovators in the region and we can't wait to meet them." the programme is looking for entrepreneurs that are working to solve "real-world problems with innovative technology-led ideas". the organisers are particularly keen to work with founders looking to work on ageing-related issues. up to 20 entrepreneurs will be given a chance to join ignite, which will work out of tuspark newcastle, at maybrook house on grainger street. the best business will then be invited back to join ignite's later-stage six-month programme and will qualify from further investment. the investment is being delivered by northstar ventures and the north east innovation fund, which is supported by the european regional development fund. successful companies that have grown out of ignite in the past include music platform leaf music, ecommerce start-up moltin, and baby boomer and dementia platform mindmate. the newcastle programme will be run by mindmate founder gabriela matic and will benton, who launched music streaming site chew before exiting the business in 2017. ms matic said: "we're really looking forward to delivering this in newcastle. we have a unique perspective in both having been on the programme before and also having been in the position the entrepreneurs we will work with are, as founders of our own start-ups. so along with the investment and network of experts, we will be able to offer the right kind of support to those on the programme as we know exactly what they are going through. "the pre-accelerator is about developing and testing out your business idea, and getting your first customers - so whatever stage you're at, even if you've just got a great idea and you're not sure what to do next, get in touch and see if we can help." alasdair greig, director at northstar ventures, said: "we are delighted to be able to support ignite's return to newcastle. "we want to find the north east's best innovators and start-ups and we know the ignite team are experts in doing this and running programmes that support developing, testing out and growing innovative businesses. we're very pleased to be working with them and i am sure the new ignite pre-accelerator will be widely welcomed and attract the best, most exciting ideas and entrepreneurs in the region." & 605 & very low & Low & Socio-Economic & Socio-Economic & NA & 2018-07-29 & 2018 & 3 & ECO
Frame & v.low & Regional & 500-1000 & -0.7997063 & -1.2431481 & 1.1980777 & 1.4613788 & -0.9363498 & 1.1 & -0.8854249 & -0.3297660 & Payer & Domestic & Domestic & Domestic & Domestic|ECO & Positive\\
UK & http://www.dailymail.co.uk/wires/reuters/article-5744815/Economic-proposals-Italys-5-Star-League-government-pact.html & 731 & Daily Mail Online & Private/Non-Public & Online and Offline & National & very low = CP mentioned once & Social justice & Factual & EU + Other country & No myth & NA & NA & NA & NA & NA & NA & NA & NA & UK & economic proposals in italy's 5-star, league government pact & 2018-05-18 & european social fund & rome, may 18 (reuters) - italy's anti-establishment 5-star movement published on friday a programme it agreed with the far-right league, proposing a review of the european union's fiscal rules and "limited" deficit spending to fund tax cuts and increased welfare spending. some of the more radical proposals -- including seeking to exclude state securities held by the european central bank from countries' debt calculations -- were dropped or watered down compared with previous drafts. there are no measures calling into question italy's membership of the euro zone, and the measures have no time-frame given for their adoption. following are highlights of the key economic segments. debt and deficit - the accord says the government will cut debt, not by raising taxes or imposing austerity but by fuelling growth - will seek a change in eu rules so that expenditure on public investments does not count in budget deficit calculations - europe's economic governance, including the stability and growth pact and the fiscal compact, needs to be revised in cooperation with eu partners - the government's policies will be funded, after the intended revisions of eu treaties, by a multi-year plan "to cut wasteful spending, manage debt and an appropriate and limited recourse to deficit spending" - proposes to issue short-term treasury bills as payment to individuals and companies who are owed money by the state, and, in this context, calls for official statistics agencies to "consider the definition of public debt" - makes no reference to revisions to italy's current deficit and debt targets taxes - next year's scheduled sales and excise tax increases, worth 12.5 bln euros, will be cancelled. "anachronistic" gasoline excise tax components also to be eliminated - there will be just two tax rates, set at 15 pct and 20 pct, for individuals and companies, instead of the current rates ranging from 23 pct to 43 pct. families to receive 3,000-euro annual tax deduction based on household income. no timing for this given - increase in sanctions, including prison, for tax evasion - amnesty for people struggling to pay tax arrears pensions - abolish the 'fornero' pension reform which raised the retirement ages and required future hikes. the pact says 5 billion euros will be needed to cover the cost - introduce a new points system allowing people to combine their age with the number of years they have paid social security contributions. this must equal 100, with the idea that those who have paid into the system for 41 years can retire once they are at least 59 years old. no timing given - cut in so-called "golden pensions" of more than 5,000 euros per month which have not been fully funded by contributions into the system universal income - ensure income of 780 euros a month for the poor by providing universal income support. recipients are obliged to look for work and accept one of the first three job offers received - calls for a 2 billion euro investment in employment agencies to better help job seekers - open dialogue with european union to guarantee the use of 20 pct of the european social fund allocation to help italy establish the universal income support. no timing given banking - creation of a public investment bank to help the economy, using existing resources - calls on the eu to "radically reform" its bail-in regime, to ensure greater protection for savers - tougher penalties for both bosses and regulators in cases of bank failures and possibility of compensation for retail shareholders of resolved banks - calls for a review of the basel accords, saying parameters are seriously damaging small businesses in italy - calls for state to remain a shareholder in bank monte dei paschi, recently bailed out by the government - calls for a move towards separating investment banking from retail banking - plans to make it obligatory for banks to obtain court authorisation before trying to recover money from debtors labour - introduce a minimum wage - reduce labour taxes - re-introduce voucher payment scheme intended to simplify paperwork and reduce taxes for workers without regular contracts - a ban on unpaid internships industry, transportation - alitalia must not simply be saved, but also relaunched because the country needs a competitive national airline. league economic chief claudio borghi told reuters that the programme includes stopping the sale of alitalia - a high-speed rail line under construction between turin and lyon should be re-considered with france ilva steel plant - the pact says the health of citizens and the environment around the ilva steel factory in taranto needs protecting, calls for an economic "re-conversion" to promote industry in the south based on closing "sources of pollution". the pact does not say if the whole ilva site should be shuttered gambling - phase out slot machines and video lotteries, adopt greater restrictions on gambling industry (reporting by crispian balmer, steve scherer, giuseppe fonte and gavin jones editing by hugh lawsonk, jon boyle and peter graff) & 818 & very low & Low & Socio-Economic & NA & NA & 2018-05-18 & 2018 & 3 & ECO
Frame & v.low & National & 500-1000 & -0.7997063 & -1.2431481 & 1.1980777 & 1.4613788 & -0.9363498 & 1.1 & -0.8854249 & -0.3297660 & Payer & European & European & European & European|ECO & Neutral\\
UK & http://www.chroniclelive.co.uk/news/north-east-news/no-promise-north-easts-eu-11850153 & 748 & Chronicle Live & Private/Non-Public & Online and Offline & Regional/Local & high = CP is most important issue in story (can also cover other issues) & Institutional bargaining over funding & Positive & EU + National + Subnational & No myth & NA & NA & NA & NA & NA & NA & NA & NA & UK & no promise on north east's eu funding after 2020 & 2016-09-06 & structural funds & brexit secretary david davis says he can't promise what will happen after the next election to cash the region gets from the eu there is no way of guaranteeing eu funding currently going to the north east will continue after 2020, the secretary of state for brexit has warned. david davis told the house of commons that he could not make any promises about what would happen to funding after 2020, once the uk leaves the european union. he was speaking in response to questions from north durham mp kevan jones, who said there was continuing uncertainty in the region about the future of eu funds used for economic development. the north east has received twice as much european cash per person than any other part of england, with the region getting £187 per head compared to a national average of £82. cash has typically gone to councils or economic development agencies, and projects which benefited from eu money included gateshead college, which has received £10,753,750, and new college durham, which received £8,557,950. in august, the government attempted to end uncertainty about existing projects by issuing a statement guaranteeing funding for most projects which have already been promised eu funding, even after the uk leaves the eu. the treasury also said it would fund some new projects if applications for eu funding are made and approved before brexit takes place. projects which will benefit include the tees valley compass and business fund which will run until the end of 2019 and is expected to deliver 800 new jobs and around £15m in private sector investment, the treasury said. but the guarantee only applies to the current round of eu funding, which ends in 2020. a general election must also be held by 2020 at the latest, and prime minister theresa may has indicated she is likely to wait until then before going to the polls. mr jones told mr davis: "there is uncertainty, certainly in the north east of england, about the future of eu structural funds. can he give a guarantee that, once we come out of the eu, those funds will be replaced by the government?" mr davis said: "i cannot speak for a future government - that will be beyond the next election - but ... we will put in the library the chancellor's letter underwriting many of the structural funds, research grants and common agricultural policy funds that are already in place." newcastle north mp catherine mckinnell told the government the north east needs answers on brexit if the region is to work together "to make the best of this mess". and she said the north east combined authority should be involved in the brexit negotiations. there are a series of "unanswered questions" about what brexit means for the north east, she told mps. mrs mckinnell said: "under what circumstances will north east firms be able to trade their goods and services to eu and non-eu countries in the future? "that knowledge is crucial to the only region in the uk that consistently exports more ​than it imports, with some 58\% of north east exports currently going to the eu. "what will happen to eu nationals who have made a life for themselves in the north east, such as the 1,000 people who work in the nhs and the 600 university staff? those are just two examples." she added: "to what extent will the north east be involved in the brexit negotiations? britain leaving the eu will clearly have a profound effect on my region and i share the determination of the north east combined authority that our voice is heard as loudly as anybody else's throughout the process. "how will the government be held accountable for any of this?" more on the eu referendum could it be overturned? what happens now? can we rejoin? how the north east voted in full live reaction and analysis how brexit will affect your money britain votes to leave the eu what is article 50? 1 of 8 & 675 & high & High & Power & NA & NA & 2016-09-06 & 2016 & 2 & POL
Frame & high-very high & Regional & 500-1000 & -0.7997063 & -1.2431481 & 1.1980777 & 1.4613788 & -0.9363498 & 1.1 & -0.8854249 & -0.3297660 & Payer & Domestic & European & Mixed & Domestic|POL & Positive\\
\addlinespace
UK & https://www.walesonline.co.uk/comm-part-test/new-premises-job-creation-export-14994490 & 764 & WalesOnline & Private/Non-Public & Online only & Regional/Local & very low = CP mentioned once & Economic development & Positive & Subnational & No myth & NA & NA & NA & NA & NA & NA & NA & NA & UK & the skip company increase revenue thanks to business support service & 2018-08-06 & european regional development fund & business wales aims to build capacity and innovation within welsh businesses get daily updates directly to your inbox+ subscribesee our privacy noticethank you for subscribing!could not subscribe, try again laterinvalid email an online supplier of skip hire and waste management solutions has increased revenue by 30\% thanks to the welsh government's business wales service. nationwide waste solutions, trading as the skip company, offer a waste-disposal brokerage service offering complete waste management and recycling solutions to all throughout the uk and the republic of ireland. the skip company has traded for four years and during this time has achieved 100\% growth. the business has recently moved to new, larger premises in pontypool and now employs eight members of staff. managing director matthew davies first contacted business wales in 2016 looking for help and advice on how to grow and develop his business. he was unsure how this could be achieved, whether his growth plan was viable and what, if any, help was available. matthew commented: "over the 12 months we've worked with business wales and thanks to their help, our revenue grew circa 30\% and we have achieved significant milestones - new premises, job creation, international trade and iso accreditation. i have also made use of other services available including mentoring, training, hr and environmental support. "the team at business wales are very knowledgeable, helpful and always available to pass on their expertise and contacts. i would recommend them to any sme and cannot rate or thank them enough for the support and help that i have received." phil summers, relationship manager at business wales, said: "business wales always aims to build capacity and innovation within welsh businesses, creating the right environment for them to grow and thrive. "we have helped matthew and his team to work more efficiently and sustainably, while creating jobs and investment into the local community. i look forward to seeing their future plans as they continue to expand in wales and beyond." the skip company work with several large european companies, have recently rolled out their services in ireland and are looking to expand into the southern hemisphere and beyond. nationwide waste solutions' top tips for anyone looking to start or grow their own business would be: create a business plan - this will outline where you are now, where you want to go with the business and how you are going to do it! having the right staff is crucial as they can make or break the business short and long-term. spend some time recruiting and training the right people. cashflow is critical in the early days. there may be tendency to focus on growth, but you should spend as much time and effort getting the money in as you may come unstuck. be persistent - network, cold call, up sell, cross sell. don't wait for it to happen - make it happen. work hard and play hard - for all your hard work and effort, find time for yourself and do something that you enjoy. business wales, which is funded by the european regional development fund through the welsh government, supports the sustainable growth of small and medium-size enterprises across the country by offering access to information, guidance and business support. to find out how business wales can help start or develop your business, call 03000 6 03000, follow @\_businesswales or @\_busnescymru or visit www.businesswales.gov.wales or www.busnescymru.llyw.cymru for further information. & 574 & very low & Low & Socio-Economic & NA & NA & 2018-08-06 & 2018 & 3 & ECO
Frame & v.low & Regional & 500-1000 & -0.7997063 & -1.2431481 & 1.1980777 & 1.4613788 & -0.9363498 & 1.1 & -0.8854249 & -0.3297660 & Payer & Domestic & Domestic & Domestic & Domestic|ECO & Positive\\
UK & http://www.theguardian.com/uk-news/2016/nov/28/wales-urged-to-do-deal-with-ireland-to-secure-eu-funds-post-brexit & 752 & The Guardian & Private/Non-Public & Online and Offline & National & medium = CP is important part of story (alongside other issues) & Institutional bargaining over funding & Positive & National + Other country & No myth & NA & NA & NA & NA & NA & NA & NA & NA & UK & wales urged to do deal with ireland to secure eu funds post-brexit & 2016-11-28 & european social fund & the welsh government is being urged to open talks with ireland in an attempt to secure continued access to european funding after brexit, under a plan by plaid cymru. the party believes it can take a leaf out of norway's book. the country gets eu funding despite not being a member, thanks to a partnership with sweden. "the welsh government cannot afford to play 'wait and see' with wales's future relations with our nearest neighbours and must think creatively in order to further wales's interests post-brexit," said the party's external affairs spokesman, steffan lewis. he made his call for a new celtic sea alliance on the first of two days of biannual talks at the british-irish parliamentary assembly in cardiff. it says that the belfast agreement provides for bilateral deals even if the uk quits the eu. lewis believes such an alliance could mitigate the economic impact of brexit on wales and ireland, which is already being hit by the fall of the pound. the exit from the eu could have devastating consequences on the economy of wales, where the vast majority voted for brexit. it is the single biggest beneficiary of eu funds in britain, with £2.7bn earmarked in the present round of funding. this includes allocations to farms, the european social fund, the horizon science and research fund and the erdf, which accounts for the lion's share of money at £1.8bn. ireland is already feeling the brunt of the brexit vote following the drop of sterling. britain is the country's single biggest trading partner and exports are already hit, particularly in food and agriculture. several mushroom farms in ireland have already taken a hammering because of the exchange rate and last month political leaders warned of the "incalculable consequences" brexit would have on the irish economy. brexit also poses an opportunity for wales, with the devolved government expected to redouble efforts to lure irish business. there are already strong trade links between the two countries, with several major irish food processing plants in north and south wales, and it is widely felt that the only way irish exporters can soften the brexit blow is to have manufacturing arms in britain. lewis: "it is necessary for that to be formalised between the governments of wales and ireland through an agreement that could establish a celtic sea alliance, focused on collaboration between the two nations but especially the western regions of wales and the eastern regions of ireland." there may also be an opportunity for ireland and wales to dip into eu inter-region funds, known as interreg. norway and sweden share €73m (£62m) from interreg funds for programmes designed to protect vulnerable border regions, in terms of environment, employment and social cohesion. the notion that wales could continue to benefit from eu funds comes as similar attempts are made in sectors such as education and science, which are heavily dependent on eu funds. british universities are considering plans to open branches inside the european union to soften the blow of britain's exit. university leaders fear brexit will make student and staff recruitment much more difficult, cutting off eu research and funding and probably constricting the flow of eu students, who have been the fastest growing proportion of young undergraduates. funding in scientific research often flows from collaboration with institutes across europe, and elite universities feel they will be excluded from cross-border consortia long before britain quits the eu. & 581 & medium & Medium & Power & NA & NA & 2016-11-28 & 2016 & 2 & POL
Frame & low-medium & National & 500-1000 & -0.7997063 & -1.2431481 & 1.1980777 & 1.4613788 & -0.9363498 & 1.1 & -0.8854249 & -0.3297660 & Payer & Domestic & European & Mixed & Domestic|POL & Positive\\
UK & http://www.dailyrecord.co.uk/news/local-news/north-ayrshire-council-launch-multi-11370711 & 760 & Daily Record & Private/Non-Public & Online and Offline & Regional/Local & very low = CP mentioned once & Social awareness/inclusion & Positive & Subnational & No myth & NA & NA & NA & NA & NA & NA & NA & NA & UK & north ayrshire council launch multi-million pound anti-poverty campaign & 2017-10-19 & european social fund & the project coincides with challenge poverty week and is funded jointly by the european social fund and big lottery fund scotland. get daily updates directly to your inbox+ subscribethank you for subscribing!could not subscribe, try again laterinvalid email a multi-million pound project geared towards helping north ayrshire's poverty-stricken residents was launched this week. better off north ayrshire was unveiled on monday at pennyburn's playz with the aim of improving the financial circumstances of people out of work or on low incomes. the project - which coincides with challenge poverty week - is funded jointly by the european social fund and big lottery fund scotland with £3m awarded for a three year period to 2020. the initiative is a collaboration between the council and a range of third sector delivery partners as the local authority ramps up its bid to reduce inequality and poverty in north ayrshire. the better off service is a one-stop shop that will provide a range of advice and support to help ease the financial worries for those on lower incomes, out of work, or who are lone parents. councillor joe cullinane, leader of north ayrshire council, said: "we are incredibly proud to be launching better off north ayrshire. it is apt that we are doing so at the very beginning of challenge poverty week. "we can't hide from the fact that poverty affects communities across north ayrshire. but we want to make a difference. we want to help people and we want to support people. "we are bringing services and expertise together in what will be a united effort by all of us. a particular focus will be on loans and household good services. "our research estimates that 10,000 people in north ayrshire borrow a total of £10million from payday lenders, home credit companies, rent-to- buy outlets and pawnbrokers each year, paying around £5million in interest in the process. "so we are looking to see how we can provide more ethical and affordable options to some of those that are out there just now. "we want to really engage with people, find out what kind of financial issues they are experiencing and help them. our team at better off north ayrshire will provide ongoing support to help them towards financial sustainability." the playz is one of the hub venues alongside the michael lynch centre in ardrossan, ceis in stevenston and the fullarton centre in irvine. an additional hub will open in kilbirnie in november. maureen mcginn, big lottery fund scotland chair, said: "i am delighted that the people of north ayrshire will benefit from this funding, thanks in part to money raised by national lottery players. the better off north ayrshire service will help people facing disadvantage in the area to improve their money management skills and lower their debt burden, which often acts as a barrier to social inclusion." expert teams from a variety of partners will be available to offer a range of support from managing your money and debts, digital access and training, finding the best credit options, helping residents explore cheaper fuel options and sourcing low-cost furniture. the council put in an additional £400,000 to support the project. this included the funding for the online tool and to support financial inclusion work within the health and social care partnership. the better off north ayrshire website ( northayrshire.betteroff.org.uk ) will also be stacked full of helpful information to help with benefits applications, budgeting and money advice. it will also help users access childcare and apply for jobs. challenge poverty week begins the start of challenge poverty week was officially marked with a range of passionate speeches from community figures. the 'aye we can' conference launched a week of activities across the region as the council joined the poverty alliance's campaign to raise awareness of the existence of poverty in communities. and the event at fullarton community hub promoted the message that north ayrshire council, partners and a range of organisations are fighting hard to help and support local communities. the event was opened by chief executive elma murray, who highlighted the work being carried out by the council and, in particularly, the recently- approved fair for all strategy. the strategy looks at ways we can empower communities, encourage local innovation and create more equity in our communities. peter kelly from the poverty alliance spoke about how they are finding solutions to tackle poverty, while zoe ferguson from carnegie uk trust provided a fascinating insight into how promoting kindness is essential to the well-being of people and communities. donna fitzpatrick, from the fullarton community association, detailed her inspiring journey which ultimately led to development of the purpose-built fullarton community hub. closing the event was the councillor joe cullinane, leader of north ayrshire council, and he said: "challenge poverty week is hugely important to all of us. it's a week where we can try and raise awareness and try to build connections. "we aren't going to solve the issue of poverty overnight but we want to take steps, even if they are quite small ones, in trying to make a difference. "the conference was a great opener. it allowed us to hear from people with different perspectives of poverty and share experiences, problems we've encountered and, most importantly, lessons that we've all learned." read more news from irvine and kilwinning & 902 & very low & Low & Socio-Economic & NA & NA & 2017-10-19 & 2017 & 2 & ECO
Frame & v.low & Regional & 500-1000 & -0.7997063 & -1.2431481 & 1.1980777 & 1.4613788 & -0.9363498 & 1.1 & -0.8854249 & -0.3297660 & Payer & Domestic & Domestic & Domestic & Domestic|ECO & Positive\\
UK & https://www.independent.co.uk/news/uk/politics/brexit-northern-ireland-eu-funding-money-peace-process-a8532496.html & 722 & The Independent & Private/Non-Public & Online and Offline & National & medium = CP is important part of story (alongside other issues) & Territorial cooperation & Positive & EU + National + Subnational & No myth & NA & NA & NA & NA & NA & NA & NA & NA & UK & northern ireland should keep eu funding after brexit to preserve peace process, european parliament says & 2018-09-11 & cohesion policy & northern ireland should be allowed to keep hundreds of millions of pounds in eu funding after brexit to keep the region's peace process on track, the european parliament has said. a report drawn up by the parliament's influential committee on regional development recommended that funding for two schemes - the interreg and peace programmes - should continue whether "deal or no-deal" because of the invaluable role they have played in reducing community tensions. the programmes, whose total funding is £470 million, are currently 85 per cent funded by the eu and focus on building trust between the unionist and nationalist communities in northern ireland. the report recommends that "post-2020, without prejudice to the ongoing eu-uk negotiations, eu support for territorial cooperation, especially regarding cross-border and cross-community projects, should be continued". it says that there are legitimate "fears that an end to these programmes would endanger cross-border and inter- and cross-community trust-building activities and, as a consequence, the peace process". the report was overwhelmingly backed by a vote of meps in plenary by 565 votes to 51. conservatives meps notably abstained, on the grounds that aspects of the report raised questions about northern ireland's sovereignty after brexit. derek vaughan, the committee's rapporteur told meps: "whatever happens in the future, deal or no-deal, the british government, the irish government, and the european union, should commit to continuing to operate and fund interreg and the peace programme. "it has been huge economic improvement and tensions have been reduced. but we do know that those tensions still bubble away. when we were there on the delegation one of the things we were told is that some survey work had been done and since the referendum attitudes have hardened amongst the communities." he added: "i hope nobody gets the impression, and they shouldn't get the view from the report, that we're saying that 'no eu funds means a return to the troubles' - the report isn't saying that, and i wouldn't say that. but i am saying, and what the report says, is that eu funds have made a valuable contribution to reducing those tensions and conflict." corina crețu, the eu commissioner for regional policy, said that northern ireland was "undoubtedly one of the most telling examples of what cohesion policy has helped achieved in border regions and beyond". meps from both unionist and nationalist communities in northern ireland welcomed the report in a debate in the european parliament. the eu legislature's gentle approach to northern ireland over brexit appears to contrast with that of some british brexiteers. a poll from june found that most leave voters would press ahead with brexit even if it meant sacrificing the good friday agreement, while prominent leave-supporting tories have complained the border issue has been given too much prominence. the question of how to avoid a hard border in northern ireland after the uk leaves has been the main stumbling block on signing a brexit withdrawal agreement. & 505 & medium & Medium & Socio-Economic & NA & NA & 2018-09-11 & 2018 & 3 & ECO
Frame & low-medium & National & 500-1000 & -0.7997063 & -1.2431481 & 1.1980777 & 1.4613788 & -0.9363498 & 1.1 & -0.8854249 & -0.3297660 & Payer & Domestic & European & Mixed & Domestic|ECO & Positive\\
UK & http://www.walesonline.co.uk/business/business-news/wave-power-device-promises-cut-8407633 & 792 & WalesOnline & Private/Non-Public & Online only & Regional/Local & very low = CP mentioned once & Environment/green/low-carbon & Positive & EU + Subnational & No myth & Research \& innovation & Positive & Subnational & No myth & NA & NA & NA & NA & UK & wave power device to be tested in milford haven & 2015-01-09 & structural funds & swansea-based marine power systems has developed a technology it claims could cut the costs of energy generation from waves a revolutionary device for capturing the energy potential of ocean waves is due to be tested in pembrokeshire later this year. wave power technology is so far less commercially developed than tidal technology despite many devices having been designed and tested over the years. swansea-based marine power systems, founded by swansea university engineering postgraduates dr gareth stockman and dr graham foster, has developed a technology called wavesub which it claims could significantly "reduce the costs associated with energy generation from waves." the project has now entered its third phase which should see the company construct and test a quarter scale prototype of wavesub in the latter part of this year. which energy project would you welcome in your neighbourhood? the device will be tested in the water at milford haven for between six and 12 months and the results will inform the development of a full-scale version, which the company hopes will be ready for testing in 2017. the full-scale version will be between 35 and 40 metres long and have an output of 1.5mw. marine power systems hopes to deploy its first small farm of devices in the south pembrokeshire wave demonstration zone in 2019. managing director dr stockman said that farms of up to 100 devices could potentially be deployed in the future. wavesub's promised efficiency and low cost generation owes much to the company's sea state tuning technology, which allows the device to adjust to different sea states and continue to produce electricity in a much wider range of wave heights and sea conditions than other devices. it is also not limited in regards to sea depth. the company announced in december that it had raised more than £1m from a combination of local investors and the welsh government, the latter in the form of a £359,000 smart cymru research and development grant. it is in discussions for further funding from money set aside for marine renewables in the latest round of eu structural funds. the south pembrokeshire demonstration zone is one of three sea areas set aside for the testing of wave and tidal energy devices. lying between eight and 13 miles off the coast south of pembroke dock, it covers 35 square miles of seabed and offers connection to a coastal substation via a high voltage cable. it is run by wave hub, a not-for-profit company owned and funded by the uk government. the other areas lie off the coast of cornwall and devon. wavesub is one of two innovative marine energy devices that will be tested in the waters off pembrokeshire this year. deltastream, a tidal current turbine that has been several years in development, is due to be deployed in ramsey sound shortly. the 150-tonne demonstration device, which has been named ysbryd y mor or spirit of the sea, was ready to be installed last november but its deployment has been delayed by bad weather. cardiff-based tidal energy, which has developed the 400kw device, is waiting for the next combination of good weather and the right tidal conditions to allow it to deploy the device, which generates electricity from tidal currents while sitting on the seabed. & 554 & very low & Low & Socio-Economic & Socio-Economic & NA & 2015-01-09 & 2015 & 1 & ECO
Frame & v.low & Regional & 500-1000 & -0.7997063 & -1.2431481 & 1.1980777 & 1.4613788 & -0.9363498 & 1.1 & -0.8854249 & -0.3297660 & Payer & Domestic & European & Mixed & Domestic|ECO & Positive\\
\addlinespace
UK & https://www.walesonline.co.uk/special-features/business-wales-best-practice-campaign-16001242 & 702 & WalesOnline & Private/Non-Public & Online only & Regional/Local & very low = CP mentioned once & Jobs & Positive & Subnational & No myth & NA & NA & NA & No myth & NA & NA & NA & NA & UK & business wales launches new hr campaign & 2019-03-21 & european regional development fund & the business aims to help welsh enterprises develop effective strategies and practices get the biggest daily stories by emailsubscribesee our privacy noticemore newslettersthank you for subscribingwe have more newslettersshow mesee our privacy noticecould not subscribe, try again laterinvalid email it is generally accepted that people are the greatest asset of any business, large or small. establishing a robust human resource strategy as an integral part of a business can support effective recruitment, performance management, aid staff retention, engage and motivate staff and, ultimately, benefit the bottom line. business wales is running a campaign to help more welsh smes develop these strategies, implement good practice, comply with current legislation and put effective policies and procedures in place. cathryn edwards, a human resources adviser at business wales, explains: "good practice in human resource management is fundamental to running an efficient business with well-motivated staff and a satisfied client base. "our specialist hr team consists of experienced professionals and employment lawyers who can provide a wide range of fully-funded support. we are running this campaign to let more welsh smes know that they can access our service free of charge. "we can help start-ups to recruit their first employee or apprentice, highlight key issues relating to contracts of employment, the legal aspects of employing staff, company handbooks and policies, while also supporting growing businesses with employee engagement, performance management, discipline, grievance and capability issues." whether it's helping with recruitment, legal compliance or staff engagement, hundreds of businesses across wales have already benefited from the business wales specialist hr support. eifion thomas, manager at the spectacular st tewdrics house in chepstow, now an exclusive wedding venue, received advice on the various routes to recruitment, including information on jobs growth wales, apprenticeships and training schemes, as well as contracts of employment, hr policies and procedures. eifion said: "working with the right people is crucial for every business and business wales has been a key partner for us. their advice and experience in sourcing valuable professional resources is a service that would benefit every business." helping businesses improve the business wales hr campaign also sees the introduction of the business wales equality pledge - a practical way for businesses to take action towards improving their equality performance and more importantly, communicate this commitment to their community. organisations will be encouraged to commit to a number of activities which will benefit their customers, employees or help them achieve equality best practice among suppliers, ultimately supporting businesses to become employers of choice. business wales, which is funded by the european regional development fund through the welsh government, supports the sustainable growth of small and medium-sized enterprises across the country by offering access to information, guidance and business support. for more information on how business wales can help start or develop your business, visit their website, call 03000 603 000, or follow them on facebook and twitter @\_businesswales or @\_busnescymru. & 485 & very low & Low & Socio-Economic & NA & NA & 2019-03-21 & 2019 & 3 & ECO
Frame & v.low & Regional & <500 & -0.7997063 & -1.2431481 & 1.1980777 & 1.4613788 & -0.9363498 & 1.1 & -0.8854249 & -0.3297660 & Payer & Domestic & Domestic & Domestic & Domestic|ECO & Positive\\
UK & http://www.walesonline.co.uk/business/business-news/wales-cannot-lose-single-penny-11538896 & 739 & WalesOnline & Private/Non-Public & Online only & Regional/Local & very high = CP is most important issue + CP is mentioned in title/headline & Institutional bargaining over funding & Positive & EU + National + Subnational & No myth & Economic development & Positive & EU & No myth & NA & NA & NA & NA & UK & wales must not lose a single penny in eu funding says mark drakeford & 2016-06-29 & structural funds & the cabinet secretary for finance and local government says the barnett formula needs urgent replacing the welsh government said that "every penny" lost to the welsh economy in european union funding will need to be replaced by the uk government. cabinet secretary for finance and local government mark drakeford has said he will do all he can to ensure that there is no negative impact following the result of last week's eu referendum. wales's is currently in receipt of structural funding for the valleys and west wales up to 2020. of the £1.8bn committed, some 40\% has yet to be allocated. campaign promise during the referendum campaign leader of the welsh conservatives, andrew rt davies claimed wales would be better off financially outside the eu. mr davies said: "wales could be as much as half a billion pounds a year better off if the uk votes to leave the european union. "the uk is a massive net contributor to the eu and wales would be entitled, under the barnett formula, to its share of that £9.98n 'brexit dividend.' no ifs, no buts." speaking ahead of oral questions in the senedd today, mr drakeford said that eu structural funds play a vital role in supporting growth and jobs across wales and any loss of funds will have a impact on people, businesses and communities. he added:"wales is a net beneficiary from the european union. the millions of pounds wales receives from the eu help people into work and training; support businesses; drive innovation and help to regenerate communities. "the uk government must guarantee that wales will not lose out from an eu withdrawal.""funding cuts will have a real impact on our budgets and strategies for growth and jobs. "the first minister has written to the uk prime minister about this very issue. while arrangements for a brexit are being made for the longer term, we will continue to deliver existing eu programmes in wales, investing in projects to ensure continuity for citizens, communities, farmers and businesses." poll loading ... as part of any eu exit, the uk government must give notification under article 50 of the lisbon treaty which would give up to two years to negotiate its withdrawal. that is not expected until a new prime minister succeeding david cameron is appointed - possibly in september mr drakeford said: "it is a priority for us to seek clarification from the uk government about the nature, timing and outcome of negotiations for a uk exit and how significant eu funds for wales will be replaced, particularly for those parts of wales which need these the most. "wales must be involved in the negotiations so we get the best deal for wales. this includes the need for urgent reform of the out-dated barnett formula by the uk government to ensure a fairer funding system for wales, taking into account needs arising from eu withdrawal." welsh secretary secretary of state for wales, alun cairns will today meet business leaders to assure them the uk government stands ready to support welsh businesses as the country prepares to leave the eu. mr cairns will setting out the uk government's plans for the creation of a specialist eu unit to tackle complex issues like trade treaties and legal agreements when he meets representatives from welsh businesses as well as cbi wales and the institute of directors. mr cairns will also be meeting the welsh local government association and executives from higher education. mr cairns said: "we are in the early days of a momentous move for the uk. i am in wales to reassure business and council leaders that we are determined to address their concerns and make the new settlement work. "the welsh economy is now a thriving and dynamic one, with more people than ever in work and many coming from overseas to our high performing universities. my priority is to ensure we maintain this success and manage our transition to the new arrangements in a calm and measured way. "today's meetings are the first of a series of events i will be holding across wales. while we remain eu members for now, i am starting work straight away to ensure we get the best deal for wales." & 711 & very high & High & Power & Socio-Economic & NA & 2016-06-29 & 2016 & 2 & POL
Frame & high-very high & Regional & 500-1000 & -0.7997063 & -1.2431481 & 1.1980777 & 1.4613788 & -0.9363498 & 1.1 & -0.8854249 & -0.3297660 & Payer & Domestic & European & Mixed & Domestic|POL & Positive\\
UK & https://www.thesun.co.uk/news/3605672/theresa-may-to-plough-eu-money-into-scotland-and-the-north-in-post-brexit-spending-spree/ & 733 & The Sun & Private/Non-Public & Online and Offline & National & medium = CP is important part of story & Lose sovereignty & Negative & EU + National & 2.Rich countries pay & Institutional bargaining over funding & Negative & EU + National & 2.Rich countries pay & NA & NA & NA & NA & UK & theresa may to plough eu money into scotland and the north in £9bn brexit spree & 2017-05-19 & structural funds & theresa may yesterday unveiled plans for a post-brexit spending blitz in scotland as part of a new £9 billion fund to rebuild britain using money that currently goes to the eu. vowing to keep our "precious" union together, the pm said scotland would share in the proceeds of a new prosperity fund alongside forgotten communities in wales and north as the campaign entered its final three weeks. speaking at the launch of the scottish tory manifesto in edinburgh mrs may insisted the uk may be four nations but "one people". and she pledged "to take back control" of the £9.3billion of taxpayers' cash that brussels spends on investments in the uk and use it to "strengthen" the union. eu structural spending plans were last drawn up in 2014, with the uk pencilled in for £9.3 billion until 2020. now mrs may says this cash will continue and be used to set up a "united kingdom shared prosperity fund" after brexit. yesterday the pm said the new targeted scheme's "sole purpose will be to reduce the inequalities which exist within and between the four nations of our united kingdom." she added: "we will take back control of structural funds and use them to strengthen our union." buried in the tory manifesto small print is a pledge that "the money that is spent will help deliver sustainable, inclusive growth based on our modern industrial strategy." in 2016 the uk government paid £13.1 billion to the eu budget, and eu spending on the uk was forecast to be £4.5 billion on subsidies and investments. last year the chancellor promised to honour all of these spending commitments until 2020, and the new fund will see around half of that money continue to be spent for the rest of the next parliament. the pm also told scottish voters that "leaving the eu will also enable us to build a better future for our fishermen." mrs may said: "after brexit, we will be responsible for the access and management of the waters where we have historically exercised sovereign control." but angry fishermen last night said that as "a total backslide and fudged deal." alan hastings of the fishing for leave campaign said "british fishermen have only 'historically exercised sovereign control' of waters 12 miles off the uk coast." he added: "but an international law change in 1976 means the uk's economic exclusion zones should stretch to 200 miles if britain's coastline." however as the uk was already governed by eu sharing rules by then, "historically" our fishermen have never enjoyed their full right to the waters -- sparking the cries of betrayal. & 443 & medium & Medium & Power & Power & NA & 2017-05-19 & 2017 & 2 & POL
Frame & low-medium & National & <500 & -0.7997063 & -1.2431481 & 1.1980777 & 1.4613788 & -0.9363498 & 1.1 & -0.8854249 & -0.3297660 & Payer & Domestic & European & Mixed & Domestic|POL & Negative\\
UK & http://www.chroniclelive.co.uk/business/business-news/north-east-universities-aim-help-13561820 & 785 & Chronicle Live & Private/Non-Public & Online and Offline & Regional/Local & very low = CP mentioned once & Research \& innovation & Positive & National + Subnational & No myth & Economic development & Positive & National + Subnational & No myth & NA & NA & NA & NA & UK & north east universities aim to help innovative businesses with grants of up to £25,000 & 2017-09-03 & european regional development fund & the creative fuse north east project is backing new firms in the creative, digital and it sector get business updates directly to your inbox+ subscribethank you for subscribing!could not subscribe, try again laterinvalid email a consortium of the north east's five universities is aiming to support innovative new businesses in the region with grants for smes in the creative, digital and it sectors. the creative fuse north east project - which brings together newcastle, northumbria, durham, sunderland and teesside universities - is offering grants for up to £25,000 to companies in the region through its inovation pilot and its innovation development award scheme. grants are available for smes and freelancers in the region's creative, digital and it (cdit) sector to help them find ways to grow and innovate in their business. to qualify for the funding, they will need to work with staff from two of the universities involved in the scheme. prof eric cross from newcastle university, who is leading the £4m project, said: "we want to hear ideas from businesses and freelancers which will help them to overcome challenges facing them, such as barriers to growth or developing new methods of working. "we're particularly interested in applications which are 'fused': that is, combine ideas from creative design and technology. "the five university partners know just how important the cdit sector is to our region and we are all looking forward to using our knowledge and expertise to help organisations achieve their goals." creative fuse north east will be supporting a range of £5,000 innovation pilot awards, as well as a small number of innovation development awards for projects up to £25,000. the grants come after creative fuse north east carried out a survey of more than 500 firms earlier this year to better understand the state of the cdit sector in the north east. the survey found the turnover of more than half of businesses surveyed grew in 2015/16. it also highlighted that employment in the region's creative industries grew faster than in the uk as a whole, with a 22.6\% increase from 2001 and 2015, compared to 19.5\% nationally. funding for creative fuse north east comes from the universities, plus the arts and humanities research council (ahrc), arts council england (ace) and the european regional development fund (erdf). the deadline for applications is october 16. for more information about the application process, visit http://www.creativefusene.org.uk/project-overview/content/innovation/ . & 414 & very low & Low & Socio-Economic & Socio-Economic & NA & 2017-09-03 & 2017 & 2 & ECO
Frame & v.low & Regional & <500 & -0.7997063 & -1.2431481 & 1.1980777 & 1.4613788 & -0.9363498 & 1.1 & -0.8854249 & -0.3297660 & Payer & Domestic & Domestic & Domestic & Domestic|ECO & Positive\\
UK & http://www.liverpoolecho.co.uk/incoming/big-feature-liverpool-chamber-leads-7825031 & 769 & Liverpool Echo & Private/Non-Public & Online and Offline & Regional/Local & very low = CP mentioned once & Economic development & Positive & EU + Subnational & No myth & NA & NA & NA & NA & NA & NA & NA & NA & UK & big feature: liverpool chamber leads drive to encourage mersey firms to trade across the world & 2014-09-25 & european regional development fund & the chamber has set up a series of overseas trade missions and is providing funding and other advice to city region businesses looking to break into new territories liverpool chamber of commerce wants to encourage more local firms to export more goods and services. the chamber has set up a series of overseas trade missions and is providing funding and other advice to merseyside businesses looking to break into new territories. traditional barriers that often deter small firms from venturing overseas include a lack of contacts with retailers and distributors, language barriers, currency issues, difficulties granting credit to customers based in overseas jurisdictions, as well as anxieties about laws and legal systems that they don't understand. however, liverpool chamber links into established overseas networks that can help firms overcome these challenges. the first target destination for liverpool chamber's forthcoming trade missions is china. it takes place between october 18-25 and costs £2,500 to take part. however, there is a grant available to cover 35\% of the cost. a second trade mission between december 6-12 flies out to the united arab emirates dubai and qatar, costing £2,030. other destinations planned for 2015 include turkey, latin america and south east asia. andy snell, director of international trade at liverpool and seftons chambers of commerce said: "the prospect of trading internationally is often feared for perceived lack of resource, finance or opportunity. our goal is to make the export journey as clear and enticing as possible." the chamber is tapping into a european union pot of cash, known as the european regional development fund, to help finance its new markets programme. the money can be used to subsidise the cost of trade missions and independent visits to new export markets that merseyside firms wish to make to build valuable relationships with potential customers. as well as facilitating overseas trips and linking in to chambers' and ukti's overseas networks, video conferencing can be arranged with potential customers around the world. one local firm taking part in the new markets programme is distillery liverpool gin. the firm's distiller, john o'dowd, told the echo that his firm will be taking part in trade missions to mexico and to other events connected with the europa league and champions league travel of everton and liverpool football clubs, including liverpool's forthcoming away game in madrid. mr o'dowd said: "drinking gin is an olympic sport in spain." however, he warns that considerable preparations have to be put in advance of any trip. mr o'dowd explained: "it can take a couple of years. you have to plan what you're going to say and do when you get there." preparations can include developing suitable websites and social media initiatives as well as forming and understanding of regulatory issues in other countries. he said: "you need a properly worked out story. we are working hard to make sure we have the right website, the right packaging the right distributor and that doesn't happen overnight. "gin is on a roll at the moment. vodka ruled the roost for a long time but gin is on the up because it's got a history. it was the foundation of all the cocktails in the roaring 20s and the big times in the savoy in london was based on gin. "we want to get the exports lined up and ready to roll. "you've got to do your homework. you've got to do it properly otherwise it could go very wrong. you've got to form the right relationships. that's fundamental in most aspects of business. people can be thousands of miles away and you've got to have a good rapport with them. "if you get the wrong distributor for a period of time and it goes wrong then you have lost that territory." another trade mission is being planned by ukti, the government agency responsible for promoting british trade overseas. this mission will be visiting mexico and columbia between november 24 and december 3. mexico is the world's 11th largest economy and is the largest in latin america after brazil. the country offers particular promise for firms in aerospace, energy, infrastructure and retail. colombia is the third largest country in latin america, and has a tradition of stable economic growth. over 100 british businesses already operate in colombia, including well-known companies like virgin, bt and shell as well as many smes. paul eadie, ukti north west latin america advisor said: "this trade mission is an excellent opportunity for the region's businesses to make the most of their potential in these important markets. "according to the world bank, colombia and mexico are in the top 50 in the world, having made reforms that facilitate business activity. "mexico is a country of huge potential which offers uk businesses well-established and integrated supply-chains, an increasingly skilled workforce, and free trade agreements with 44 countries. there are great opportunities within the construction, manufacturing, transportation, telecommunications, retail and automotive sector. "in addition colombia's has enormous opportunities for uk businesses." & 852 & very low & Low & Socio-Economic & NA & NA & 2014-09-25 & 2014 & 1 & ECO
Frame & v.low & Regional & 500-1000 & -0.7997063 & -1.2431481 & 1.1980777 & 1.4613788 & -0.9363498 & 1.1 & -0.8854249 & -0.3297660 & Payer & Domestic & European & Mixed & Domestic|ECO & Positive\\
\addlinespace
UK & http://www.dailymail.co.uk/wires/pa/article-3738650/Funding-pledge-EU-backed-projects-not-end-post-Brexit-uncertainty.html & 750 & Daily Mail Online & Private/Non-Public & Online and Offline & National & medium = CP is important part of story (alongside other issues) & Institutional bargaining over funding & Positive & EU + National + Subnational & No myth & NA & NA & NA & NA & NA & NA & NA & NA & UK & funding pledge for eu-backed projects 'will not end post-brexit unc... & 2016-08-13 & structural funds & funding pledge for eu-backed projects 'will not end post-brexit uncertainty' by press association published: 08:37 gmt, 13 august 2016 | updated: 08:37 gmt, 13 august 2016 the scottish government has said the chancellor's protection for european union-backed projects will not end the post-brexit uncertainty for scottish communities. philip hammond will guarantee government funding for projects backed by the eu structural and investment fund, including agri-environment schemes, which are signed off before this year's autumn statement at a cost of £4.5 billion a year to uk taxpayers. scottish finance secretary derek mackay said the "limited" funding is hundreds of millions of pounds short of what scotland would receive as a member of the eu. derek mackay said the pledge falls far short of what scotland needs mr mackay said: "we will study the detail but what is already clear is the chancellor's approach falls far short of what fishermen, farmers and communities across scotland need. "a limited guarantee for some schemes for a few short years leaves scotland hundreds of millions of pounds short of what we would receive as members of the eu. "major funding streams such as contracts for eu structural funds and european maritime fisheries projects beginning after the autumn statement have no guarantee of continuation at all. that simply isn't good enough. "it puts at risk significant investment and jobs, revealing the reality of brexit. "scotland didn't back brexit and doesn't want brexit. we certainly should not now see funding and investment in communities hammered as a result of brexit. "since the outcome of the eu referendum, we have urged the uk government to provide clarity and certainty on these vital funds. "yet, all that is clear with this announcement is that the uncertainty will continue. "we will of course engage urgently with the treasury to seek a way forward, and provide what reassurance we can to proposed beneficiaries. "but the best way to guarantee the jobs, investment, services and projects all over the country which depend on this funding beyond 2020 is by maintaining scotland's relationship with the eu." first minister nicola sturgeon has pledged to "explore all options" to keep scotland in the eu and said another independence referendum is "highly likely". mr hammond said: "we recognise that many organisations across the uk which are in receipt of eu funding, or expect to start receiving funding, want reassurance about the flow of funding they will receive. "that's why i am confirming that structural and investment funds projects signed before the autumn statement and horizon research funding granted before we leave the eu will be guaranteed by the treasury after we leave. "the government will also match the current level of agricultural funding until 2020, providing certainty to our agricultural community, who play a vital role in our country." & 478 & medium & Medium & Power & NA & NA & 2016-08-13 & 2016 & 2 & POL
Frame & low-medium & National & <500 & -0.7997063 & -1.2431481 & 1.1980777 & 1.4613788 & -0.9363498 & 1.1 & -0.8854249 & -0.3297660 & Payer & Domestic & European & Mixed & Domestic|POL & Positive\\
UK & http://www.scotsman.com/news/scotland-set-to-lose-200m-a-year-in-funding-after-britain-leaves-eu-1-4167989 & 756 & The Scotsman & Private/Non-Public & Online and Offline & Regional/Local & low = CP mentioned more times but NOT important part of story (mainly about others issues) & Institutional bargaining over funding & Positive & EU + National + Subnational & No myth & NA & NA & NA & NA & NA & NA & NA & NA & UK & scotland set to lose £200m a year in funding after britain leaves eu & 2016-07-02 & european social fund & scotland faces losing more than £200 million a year if the country's threatened eu exit pulls the plug on european funding. vital cash to help create jobs, boost training, and develop transport projects and green energy are among areas at risk. scientists also fear key research grants will dry up. there are major question marks over future support to farmers and the country's fishing industry. future loans from the european investment bank which have helped fund roads, railways and hospitals are also in doubt. scotland pays more to the eu than it receives in european funding. however, scottish ministers fear this will not be passed back from westminster as part of a brexit deal. scotland contributed some £6.5 billion (€7.8bn) to the eu and got back £5bn (€6bn) between 2007 and 2013, according to the scottish parliament information centre (spice). over the current european funding period, from 2014-20, scotland has been allocated the equivalent of more than £100m a year in "structural" funds to offset economic deficiencies. the country is also expected to win around £120m a year in scientific grants under the eu's horizon 2020 research programme. however, if scotland were to leave the eu along with the uk, it would be up to the uk government to decide whether to plug the gap. transport minister humza yousaf, who is close to first minister nicola sturgeon, told scotland on sunday: "there are massive implications for scotland and there is no guarantee the funds will come back to scotland. "the conservative leadership candidates [and likely next prime minister] are not people sympathetic to scottish wishes." there is further uncertainty because scotland's future relationship with the eu - if it leaves - is unknown, and thus it could determine whether the country remains eligible for assistance. non-eu countries such as norway can apply for some grants. scotland was due to receive £787m (€941m) in direct european funding between 2014 and 2020, or more than £110m a year. these come from "structural" funds to areas such as employment, and are paid out under the european social fund and european regional development fund. the hundreds of recipients from the previous seven-year funding round, totalling £686m (€820m), have included arts centres, bridges, wind farms, conservation projects and training for jobseekers. the scottish government said these had created more than 40,000 jobs, helped nearly 100,000 people find work and assisted a similar number of new businesses and enterprises. the projects usually also involved funding from other bodies. they included upgrading roads, such as in harris and skye, improving dalmarnock station in glasgow for the commonwealth games, refurbishing the city's maryhill burgh halls and constructing the loch carnan wind farm in south uist. past funding has included the falkirk wheel canal boat lift. the scottish government, which manages the funds for the european commission, has so far published only five projects under the current funding round, worth a total of £146m, which are centred on job creation and boosting manufacturing. these include a low carbon infrastructure transition fund, to help the development of low-carbon products, which has £33m of european funding. the scottish government said those projects, running to 2018-19, were "legally committed" and would be unaffected if scotland left the eu. however, organisations funded in the past were less sure. enable scotland campaigns director jan savage said: "we would be concerned that, if this funding were to disappear, people would lose out on the opportunity to live as part of an equal society. "we have to make sure that people in scotland who have learning disabilities do not lose out as a result of the uk breaking away from the european union." andy kerr, executive director of the edinburgh centre for carbon innovation, said: "we are concerned about longer-term funding prospects. we anticipate that eu funding streams will be open to us for at least the next couple of years, but there is huge uncertainty thereafter." robert farquharson, chief executive of the action group, for people with support needs, said: "uk employability programmes are being devolved to scotland, unfortunately with seriously reduced budgets. "if, in the future, there was also the loss of eu funding it would be a huge backward step." children's charity barnardo's scotland said it had a "significant level of concern" over funding. the eu also funds scientific research through the horizon 2020 programme, for 2014-20, with scottish universities and other research centres winning at least £182m (€217m) so far from successful applications. academics fear scotland could lose both funding and equally crucial collaboration with eu colleagues. professor anne glover of aberdeen university, a former chief scientific adviser to the european commission, said scotland had been expected to secure around £840m (€1bn) from horizon 2020 - the equivalent of £120m a year - based on its success in the previous funding programme. however, glover said that funding was at risk if the country left the eu because she doubted the uk government would make up the shortfall. she said: "my confidence is not high. uk funding has not kept pace with inflation and they have not got a track record for increasing funding." but glover said if scotland could still apply for european grants after brexit, it would be unable to influence what research was eligible. the uk fought off attempts to remove stem cells from the programme - an area where scotland excels. equally uncertain is scotland's biggest source of european funding: from the common agricultural policy (cap). scotland received £560m in agricultural support payments in 2014, and it is not known what will replace the cap should scotland find itself outside the eu. the country is also due to get £90m under the european maritime and fisheries fund from 2014-20, whose future has also to be decided. the national farmers union of scotland said: "while the outcome of the vote brings a period of significant uncertainty, it also presents an opportunity to negotiate the best possible deal to support our farming and food sectors." but the national trust for scotland said the loss of cap support from its farmland, which it used for conservation, would be "a great blow". the scottish government sought to put a brave face on the prospects for funding. a spokeswoman said: "we are intent on pursuing all options to maintain scotland's eu status so that these benefits can be preserved. "european structural funds will clearly be a key part of those negotiations." however, the european movement in scotland, which campaigned for the uk to remain in the eu, feared the country would lose out. policy adviser alex orr said: "in the event of brexit, scotland would be at the whim of westminster as to how any such funding was allocated. "we are conscious of the desire of the government at westminster to focus on delivering the northern powerhouse [in england]. "there is therefore clearly a concern that funding which has benefited key areas, such as the highlands and islands and south-west scotland, may not continue at the sort of levels currently enjoyed, with obvious implications for those in these communities." the uk government pledged to look after scotland's interests once the brexit process begins. a spokesman said: "these negotiations will involve all the devolved administrations to make sure the interests of all parts of the uk are protected." & 1231 & low & Low & Power & NA & NA & 2016-07-02 & 2016 & 2 & POL
Frame & low-medium & Regional & +1000 & -0.7997063 & -1.2431481 & 1.1980777 & 1.4613788 & -0.9363498 & 1.1 & -0.8854249 & -0.3297660 & Payer & Domestic & European & Mixed & Domestic|POL & Positive\\
UK & http://www.chroniclelive.co.uk/news/north-east-news/eu-referendum-north-east-buildings-11127331 & 694 & Chronicle Live & Private/Non-Public & Online and Offline & Regional/Local & medium = CP is important part of story (alongside other issues) & Jobs & Positive & EU & No myth & Economic development & Positive & EU & No myth & NA & NA & NA & NA & UK & eu referendum: these are the north east buildings that got euro cash & 2016-04-01 & european regional development fund & some of the north east's best known landmarks and buildings have received eu funding, generating thousands of jobs. here we list a few of them and how much euro cash they got. the core, newcastle - a landmark building in science central, providing serviced office space for high growth technology and science-based businesses - £5.6m european regional development fund. the sage gateshead - the internationally-acclaimed venue was built more than a decade ago with a european contribution of £5.6m. there was also another £2.5m from the eu for "quays visitor infrastructure", including land reclamation and site preparation. the gateshead millennium bridge - an iconic monument over the river tyne received £2m in eu cash. the toffee factory in newcastle - refurbishment of the former 'maynard toffee factory' into a brand new building providing high quality and contemporary serviced office space for a range of digital and creative businesses - £2.8m. live theatre liveworks - construction of a new four-floor business centre providing premium, large office space within the iconic landscape of the newcastle's quayside -£2.5m. north bank of the tyne - a programme enabling infrastructure works to develop this key riverside site which will enhance the competitiveness of the area in the sub-seamarine offshore sectors - £2.48m. the beacon - an enterprise hub created to address the social and economic needs in the west end of newcastle - £2.4m. newcastle science city company's 'newcastle innovation machine' - identifying technology and science applications that are available or open to development - £2.28m. newcastle university translational research building - a new building located on the campus of ageing and vitality, providing space for clinicians, academics and commercial companies jointly engaged to bring together industry projects and academic research - £1.8m. the river tyne energy and innovation centre has received £1m. a low carbon enterprise zone for businesses to settle at swan hunter ship yard in north tyneside - £7.8m. tyne and wear metro's simonside station opened in 2008 and cost £3.2 million. it was part funded by the erdf the angel of the north - cost £800,000, of which £150,000 was euro cash. portobello trade park in durham - £2.7m. consett business park in durham - £1.1m. sunderland software centre - £4.4m the european union debate the case for opting out the case for staying in how local mp's plan to vote eu debate at hardwick hall business group campaigning against eu stronger in europe group launch campaign north east employers depend on the eu alan johnson's view on eu 1 of 8 & 427 & medium & Medium & Socio-Economic & Socio-Economic & NA & 2016-04-01 & 2016 & 2 & ECO
Frame & low-medium & Regional & <500 & -0.7997063 & -1.2431481 & 1.1980777 & 1.4613788 & -0.9363498 & 1.1 & -0.8854249 & -0.3297660 & Payer & European & European & European & European|ECO & Positive\\
UK & http://www.walesonline.co.uk/business/business-news/development-bank-wales-could-up-11913714 & 762 & WalesOnline & Private/Non-Public & Online only & Regional/Local & very low = CP mentioned once & Economic development & Positive & EU + Subnational & No myth & NA & NA & NA & NA & NA & NA & NA & NA & UK & a development bank for wales could be up and running next spring & 2016-09-21 & european regional development fund & minister ken skates committed to a development bank for wales as finance wales launches new £136m fund a development bank for wales could be operational by the spring of next year economy secretary ken skates has confirmed as finance wales has formally launched its new £136m fund to back growth-focused smes. the new wales business fund, from the wholly-owned investment bank subsidiary of the welsh government, will be financed via a combination of non repayable finance for the eu's european regional development fund (£75m) the welsh government (£30m) and realisations from finance wales equity investments made from its now fully invested jeremie fund (£30m). with no element in the new fund repayable finance - as was the case with the jeremie fund with a £75m loan from the european investment bank - the new fund will have a lower interest rate range on debt from 4\% to 12\%. under jeremie rates on debt ranged from 8\& to 12\%. fund breakdown nearly £50m earmarked for equity investments over the seven years of the fund - with £27.3m backing early stage businesses. targeting to back 415 firms - compared to 600 under the £150m jeremie fund. the fund's debt element will be just over £86m. the fund will also look for significant leveraged investment from banks and other private funders on both debt and equity deals. the welsh government is now considering how finance wales effectively evolves to take on a wider remit of the development bank for wales. the business case providing the roadmap for its evolution was penned by finance wales itself. however, last year an independent panel, commissioned by then business minister edwina hart and chaired by professor dylan jones-evans recommended that finance wales should not evolve into a development bank and that a new organisation should focus much more on providing long-term and affordable finance to smes and in particular micro-businesses -those employing 10 or less. minister's view mr skates said the development bank was progressing well and that the welsh government had received and was evaluating the detailed business case (from finance wales) with a view to the development bank being up and running by the second quarter of 2017/18. he said: "i am committed to creating a fair and prosperous economy that benefits everyone in wales - and that means increased support for our businesses. "the wales business fund will mean welsh smes can collectively access up to £136m of financial support over the next seven years, ensuring they are well placed not only to develop their home markets but also to compete internationally. "this support will be complemented by the establishment of the development bank for wales which will further support welsh businesses to access finance and ultimately help us to safeguard and create jobs right across wales. "our work to establish the development bank is progressing well and we have now received a fully costed business plan from finance wales which would see them evolve into the development bank for wales. work is now ongoing to fully evaluate the costs and benefits of that plan with a view to the bank being up and running in the second half of next year. the minister said the key aim of the development bank is to increase investment levels to welsh business and "enable the economy to thrive." he added: "this is more important than ever as we seek to increase business confidence in the wake of the eu referendum and i will be looking to the development bank to drive up investment levels to £80m per annum within 5 years." finance wales chief executive of finance wales, giles thorley, said: "the launch of the £136m wales business fund and the exciting progress that's being made with plans for the development bank for wales is a significant boost to the welsh economy. our new fund, part of more than £700m of funds managed by the finance wales group, offers flexible investments to welsh small and medium-sized enterprises (smes). smes are the lifeblood of the welsh economy and we're proud to support them. "funding remains available for welsh businesses , despite the recent brexit vote. eu structural and investment fund projects - like the wales business fund - which have been approved prior to this year's autumn statement are guaranteed funding. "with this new fund and our evolution into a development bank for wales in the coming financial year we'll be able to build on our existing experience, offering flexible finance solutions to a range of businesses. the new package outlined by the welsh government will attract both international companies and help home-grown entrepreneurs. wales is a great place to do business and we're committed to supporting a strong business environment." & 793 & very low & Low & Socio-Economic & NA & NA & 2016-09-21 & 2016 & 2 & ECO
Frame & v.low & Regional & 500-1000 & -0.7997063 & -1.2431481 & 1.1980777 & 1.4613788 & -0.9363498 & 1.1 & -0.8854249 & -0.3297660 & Payer & Domestic & European & Mixed & Domestic|ECO & Positive\\
UK & http://www.belfasttelegraph.co.uk/news/local-national/northern-ireland/expansion-creates-dozens-of-jobs-30687633.html & 711 & Belfast Telegraph & Private/Non-Public & Online and Offline & Regional/Local & very low = CP mentioned once & Jobs & Positive & Subnational & No myth & Research \& innovation & Positive & Subnational & No myth & NA & NA & NA & NA & UK & expansion creates dozens of jobs & 2014-10-23 & european regional development fund & a global technology firm is creating 35 new jobs in londonderry as part of a multi-million pound expansion. seagate technology, which makes computer hard-drives, said the posts at its springtown operation would have an average salary of about £35,000-a-year. invest northern ireland has contributed £7.8 million, part funded by the european regional development fund, towards the new jobs. first minister peter robinson said the announcement reinforced the company's commitment to northern ireland. he said: " seagate has made an enormous contribution to the northern ireland economy over the last 21 years and this investment of almost £35 million underpins the company's commitment to northern ireland and its confidence in our high quality of staff and ability to deliver the highest quality services ." deputy first minister martin mcguinness described it as a tremendous day for his native city, derry. he said: "seagate operates in an environment where the pace of innovation is relentless and this investment signals the company is confident derry and the north west is capable of meeting the considerable challenges. "whether it is advances in cloud computing, big data or the next big thing in technology this site at springtown will be an important element of seagate's future. the technological decisions and developments made in derry have the potential to impact on future generations globally." seagate has been operating in northern ireland since 1993 and is one of the north west's biggest employers with 1,300 people on its books. the jobs are part of a £34.7 million investment in a research and development project which involves the manufacture of cutting-edge, heat assisted magnetic recording (hamr) technology and requires a team of 178 highly skilled research and development engineers. david mosley, seagate's president of operations and technology, said: "the explosive increase in demand for storage from mobile applications, cloud-computing infrastructures, social media, business applications and consumer markets is driving the need to develop new technology that can deliver greater storage capacity." & 336 & very low & Low & Socio-Economic & Socio-Economic & NA & 2014-10-23 & 2014 & 1 & ECO
Frame & v.low & Regional & <500 & -0.7997063 & -1.2431481 & 1.1980777 & 1.4613788 & -0.9363498 & 1.1 & -0.8854249 & -0.3297660 & Payer & Domestic & Domestic & Domestic & Domestic|ECO & Positive\\
\addlinespace
UK & http://www.chroniclelive.co.uk/news/north-east-news/multi-million-pound-plans-jesmond-picture-8049029 & 781 & Chronicle Live & Private/Non-Public & Online and Offline & Regional/Local & very low = CP mentioned once & Economic development & Positive & Subnational & No myth & NA & NA & NA & NA & NA & NA & NA & NA & UK & multi-million pound plans for jesmond picture house site are unveiled & 2014-11-05 & european regional development fund & developers have revealed a £6.4m regeneration scheme in jesmond multi-million pound plans for the site of a former tyneside picture house have been unveiled. developers said a £6.4m regeneration scheme to create a retail and office complex on the site of the former jesmond picture house could create 100 jobs. plans for the jesmond have been unveiled to the community, outlining supermarket giant sainsbury's plans to take on a large chunk of the space, which has room for at least nine businesses, through both office and retail. the three-story scheme is being led by property investment firm mk partnership, with support from the department for communities and local government as well as a european regional development fund grant, a result of its pledge to create job opportunities for north east smes. as well as providing in excess of 2,000sqm of workspace, the jesmond will have outside terrace areas on two of the three floors and the building facade will be crafted from reconstituted stone rather than concrete. sainsbury's have long awaited the redevelopment on the former cinema site, but their plans have been met with concern by other small businesses, as the new store will be just a stone's throw away from existing waitrose and tesco supermarkets. however, the firm and developers said the new store aims to provide local shoppers with greater choice and convenience. building work will get under way soon and designs were conceived by architect kevin owens, the man behind the design for venues in the london 2012 olympic games - both at the olympic park and outside of the main park. enlisting mr owens represents a major coup for the project, with the award-winning architect's current work list including england's rugby world cup in 2015 and a new urban masterplan in the middle east. sunil mehra of the mk partnership said: "we are delighted to unveil the updated plan for this historic site that will help regenerate the area and deliver a huge amount of value to the local economy. "we are privileged to have such a talented and passionate architectural team behind the jesmond. "stuart palmer at studio-sp has maintained the design integrity of kevin owens' original concept to deliver a building with contemporary and exciting design that will become a landmark not just in jesmond but newcastle as well. "alongside the significant economic benefits that this scheme will deliver, we firmly believe the jesmond is a fitting tribute to the original building and this site." the project is now being managed by knight frank in newcastle, the selected agent for the project. although plans for the site formerly occupied by the jesmond picture house were delayed due to the recession, mk partnership said the launch of the jesmond marks the end of that period as well an improving economic climate. mr mehra added said: "we have invested a huge amount of time, effort and resource over the last few years to ensure we can restore this site into a top quality, state-of-the art and environmentally-friendly building, whilst also making it work for the local community. "we have maintained a high quality scheme without compromise by taking a measured approach to the project. "once our grant application was signed by the secretary of state and additional erdf support was assured, the time had come to deliver the jesmond." & 568 & very low & Low & Socio-Economic & NA & NA & 2014-11-05 & 2014 & 1 & ECO
Frame & v.low & Regional & 500-1000 & -0.7997063 & -1.2431481 & 1.1980777 & 1.4613788 & -0.9363498 & 1.1 & -0.8854249 & -0.3297660 & Payer & Domestic & Domestic & Domestic & Domestic|ECO & Positive\\
UK & http://www.dailymail.co.uk/\textasciitilde{}/article-3785648/index.html & 735 & Daily Mail Online & Private/Non-Public & Online and Offline & National & medium = CP is important part of story & Political leverage & Negative & EU + Other country & No myth & NA & NA & NA & NA & NA & NA & NA & NA & UK & eu soft touch on portugal deficit risks credibility - finnish finmin & 2016-09-12 & structural funds & eu soft touch on portugal deficit risks credibility - finnish finmin by reuters published: 15:09 gmt, 12 september 2016 | updated: 15:09 gmt, 12 september 2016 e-mail by jussi rosendahl and tuomas forsell helsinki, sept 12 (reuters) - portugal has got off too easy for not reducing its budget deficit, and the european union's soft line poses a risk to its credibility, finland's new finance minister petteri orpo said in an interview. with signs of euro scepticism growing in the bloc, spain and portugal in july escaped being fined by the european union for not reducing their budget gaps to below 3 percent of their gross domestic product. "spain does not have a government in place, so action against spain with no responsible administration would not perhaps be very wise," orpo, who replaced alexander stubb as the finance minister in june, told reuters in his office. "but regarding portugal, they do have a government, so i think it would be necessary, according to the growth and stability pact, to demand measures that will balance public finances." he said a suitable sanction could be to withhold money from the eu's structural funds until sufficient reforms are under way. "if there is no sanction at all, we will lose the credibility of the monetary union." finland itself is struggling to get back to economic growth after a long period of stagnation, and the eu last year warned helsinki about its rising debt and budget deficits. "there must be some principles to hold on to, regardless of the fact that we are on the same path ourselves," orpo noted. he said the nordic country, known for its stiff opposition to bailouts during the euro zone debt crisis, must practice as it preaches and curb public debt growth along with the government's 10 billion euro long-term savings plan. orpo, returning from his first european finance ministers' meeting in bratislava, also said he opposed proposed new fiscal stabilization tools for the monetary union. "this is not the time to invent new tools which will surely face opposition in the member countries." (editing by hugh lawson) & 355 & medium & Medium & Power & NA & NA & 2016-09-12 & 2016 & 2 & POL
Frame & low-medium & National & <500 & -0.7997063 & -1.2431481 & 1.1980777 & 1.4613788 & -0.9363498 & 1.1 & -0.8854249 & -0.3297660 & Payer & European & European & European & European|POL & Negative\\
UK & https://www.dailypost.co.uk/news/north-wales-news/lorries-queue-holyhead-bangor-theres-15203856 & 757 & northwales & Private/Non-Public & Online and Offline & National & very low = CP mentioned once & Institutional bargaining over funding & Factual & EU + National & No myth & NA & NA & NA & NA & NA & NA & NA & NA & UK & lorries will queue from holyhead to bangor if there's a 'no deal' brexit, says councillor & 2018-09-26 & structural funds & get daily updates directly to your inboxsubscribesee our privacy noticethank you for subscribingsee our privacy noticecould not subscribe, try again laterinvalid email queues of lorries could stretch from holyhead to bangor if there's a "no deal" brexit, says a councillor amid uncertainty over the impact on the anglesey port. london and brussels hope to agree a deal by the end of the year to avoid tariffs and trade barriers, but theresa may's chequers proposals have been criticised by both brexiteers, who want a full clean break, and the european union , who say it would undermine the single market. in the eu referendum in 2016, anglesey narrowly voted to leave the union, despite the local mp and am campaigning for a remain vote. this was despite warnings that some of the 1,000 jobs at holyhead could be at risk if northern irish ports continued to enjoy a 'soft' border with the republic of ireland while more stringent checks were introduced at ports on the british mainland. with holyhead handling around 320,000 trucks a year, drivers could face long delays and tailbacks on the roads unless an agreement is reached with the european union. this is a view shared by the first minister carwyn jones, who's already expressed fears that full border controls between the uk and ireland would be "chaos" for north wales, resulting in "queues all across the island". at a full council meeting on tuesday, caergybi councillor robert llewellyn jones tabled a motion to set up a panel to ensure a contingency plan is in place for a "no deal" brexit. "the next few months will be a very worrying time for our electorate," said cllr jones. "on ynys mon, as we follow the discussions in brussels, will we or won't we be given a deal where we can continue exporting our lambs and cattle, manufacturing goods and our banking services to the rest of the eec in favourable conditions? "there are real fears we won't get a deal, meaning we could be looking at a very difficult time as our exporters face barriers and tariffs, and we could face shortages of basic goods - something we haven't seen since the end of the second world war. "it's always best to be prepared for the worst while hoping for the best. "in holyhead, we depend on the port for so many jobs. even now, there are queues of traffic building up on the roads leading into the port. "but imagine if they brought in checks on the border. those queues will be as far back as bangor, so there are huge challenges we need to address and we need to give out the message that we as a council are prepared." in july, a delegation of anglesey councillors visited dublin for talks with the lord mayor of dublin and other city officials at the mansion house. part of that delegation was economic development lead member cllr carwyn jones, who said that work was already underway to maintain the strong links between anglesey - specifically holyhead - and dublin, which it is hoped will also lead to enhanced mutual trade, tourism and cultural benefits. responding to cllr robert llewellyn jones, he recommended that the council should stop short of setting up a panel due to a lack of influence, but proposed alternative measures including regular briefing sessions with members. "time is moving very quickly and we are approaching march 29 very quickly, but the only information forthcoming is what everyone else is hearing," he said. "i propose to write to theresa may regarding the same concerns that we've discussed this morning, including the port, the agricultural sector and objective one structural funds." what do you think? leave a comment below. & 627 & very low & Low & Power & NA & NA & 2018-09-26 & 2018 & 3 & POL
Frame & v.low & National & 500-1000 & -0.7997063 & -1.2431481 & 1.1980777 & 1.4613788 & -0.9363498 & 1.1 & -0.8854249 & -0.3297660 & Payer & Domestic & European & Mixed & Domestic|POL & Neutral\\
UK & http://www.bbc.com/news/world-europe-43220275 & 732 & BBC & Public & Online only & National & very low = CP mentioned once & Fraud/Corruption & Negative & Other country & 7.Fraud & NA & NA & NA & NA & NA & NA & NA & NA & UK & slovak pm offers €1m over slain journalist & 2018-03-04 & structural funds & slovak prime minister robert fico has offered a one million euro (\$1.2m) reward for anyone who comes forward with information about the murder of an investigative journalist. jan kuciak, 27, and his partner martina kusnirova were shot dead in their home. both were found with single gunshot wounds on sunday. several newspapers in slovakia have printed kuciak's last article, which links the italian mafia to high-level political corruption. mr fico held a press conference on tuesday, where he stood by piles of banknotes that the government is promising in exchange for information about the killings. press speculation about the motive for the mysterious murders has been rife in slovakia. kuciak's colleagues in the media say that authorities should look no further than his latest piece, published by slovak outlets overnight. in the unfinished article, he alleges that businessmen in eastern slovakia - with links to calabria's notorious 'ndrangheta mafia - are embezzling eu structural funds. he also claims that they have political ties in the country. according to earlier media reports, maria toroskova, a senior advisor to pm fico, was among those being probed. "do not link innocent people without any evidence to a double homicide," mr fico told journalists on tuesday. "it's crossing the line. it's no longer funny." during the tuesday press conference mr fico complained about "political abuse of a tragedy" after opposition politicians held a news conference making accusations linking the ruling smer party to the killings. opposition groups have called for fresh anti-corruption protests in the capital bratislava on wednesday. police president tibor gaspar said authorities had questioned 20 people since monday, and had contacted the czech republic and italy about the investigation. mr gaspar also confirmed the eu's police agency europol had offered their specialist assistance with the investigation. the chief has said the motive was "most likely" related to kuciak's investigative journalism, and has warned reporters about publishing details of the case. "how can we do our work effectively if you are alerting some people who may be involved?" he said. like other ex-communist countries, slovakia enacted far-reaching law and justice reforms in order to qualify for eu membership. kuciak had been working for aktuality.sk, an online unit of swiss and german-owned publisher ringier axel springer, for three years. he, like fellow murdered maltese journalist daphne caruana galizia, worked on the panama papers scandal. the slovak prime minister has said he would meet media outlets to assure them "that the protection of freedom of speech and the safety of journalists is our common priority and that it is extremely important to my government". & 442 & very low & Low & Governance & NA & NA & 2018-03-04 & 2018 & 3 & POL
Frame & v.low & National & <500 & -0.7997063 & -1.2431481 & 1.1980777 & 1.4613788 & -0.9363498 & 1.1 & -0.8854249 & -0.3297660 & Payer & European & European & European & European|POL & Negative\\
UK & http://www.bbc.co.uk/news/uk-england-nottinghamshire-30056009\#sa-ns\_mchannel\%3Drss\%26ns\_source\%3DPublicRSS20-sa & 716 & BBC & Public & Online only & National & very low = CP mentioned once & Cultural development & Positive & EU + National + Subnational & No myth & NA & NA & NA & NA & NA & NA & NA & NA & UK & art gallery marks fifth birthday & 2014-11-15 & european regional development fund & it has been five years since the nottingham contemporary art gallery first opened its doors to the public. the £20m gallery, which has faced both praise and criticism for its design, aimed to link old and new nottingham when it opened in november 2009. the green and gold building, designed by architect caruso st john, was built into a sandstone cliff and decorated with lace embossed concrete panels. the lace design was taken from a print of an 1847 pattern which was found in a time capsule by architects near the site. the building took four years to complete and saw the project run almost £6m over budget. the centre, a registered charity, is supported using public funding by arts council england and nottingham city council. it also received help from the european regional development fund and national lottery. it contains four galleries, performance and film space and a learning room. on its opening day hundreds of people queued to see its first exhibitions which included 60 paintings from david hockney and work by american artist frances stark. centre director, alex farquharson, said: "when we opened five years ago we aimed to be a leading international art gallery with a strong local sense of purpose. "we believe that contemporary artists offer extraordinary perspectives on society - and we wanted to share their work with as many people as possible." since it opened the gallery has shown 31 international art exhibitions and has clocked up 930,000 visitors. exhibitions have included star city, which looked at art under communism, huang yong ping's art work made from the fuselage of an american spy plane, and aquatopia, an exhibition inspired by the ocean. peter knott, from arts council england, said: "our ambition is for as many people as possible to benefit from art and culture, and it's great to be able to look back over the last five years and see how public investment in nottingham contemporary has had an impact." the building won a royal institute of british architects (riba) award for architectural excellence in 2010 and has had several nominations for other design and architecture awards since opening. & 358 & very low & Low & Socio-Economic & NA & NA & 2014-11-15 & 2014 & 1 & ECO
Frame & v.low & National & <500 & -0.7997063 & -1.2431481 & 1.1980777 & 1.4613788 & -0.9363498 & 1.1 & -0.8854249 & -0.3297660 & Payer & Domestic & European & Mixed & Domestic|ECO & Positive\\
\addlinespace
UK & http://www.liverpoolecho.co.uk/news/business/developer-instructs-agents-find-tenants-8952978 & 778 & Liverpool Echo & Private/Non-Public & Online and Offline & Regional/Local & very low = CP mentioned once & Economic development & Positive & Subnational & No myth & NA & NA & NA & NA & NA & NA & NA & NA & UK & developer instructs agents to find tenants for 74,000 sq ft office scheme in liverpool & 2015-04-01 & european regional development fund & west nile developments is renovating the grade ii-listed watson building in renshaw street a developer has instructed commercial agents to find tenants for its £16.5m office scheme in liverpool city centre. west nile developments is renovating the grade ii-listed watson building in renshaw street to create 74,153 sq ft of office space. the project is part funded by the european regional development fund and has received a £5.6m loan from merseyside's chrysalis fund, which was created to support economic regeneration in liverpool. it forms part of the wider central village scheme which also incorporates the former lewis's department store. central village includes a 455 space multi-storey car park, an adagio aparthotel - both of which are completed - 75,000 sq ft of offices within the adjacent department building and leisure accommodation which is part pre-let to occupiers including prezzo, cosmo, harvester and pure gym. agents from cbre and mason owen \& partners have been tasked buy west nile with finding tenants for the site. neil kirkham, associate director at cbre, said: "the watson building is being brought back to life, linking in with the broader central village and ropewalks area, which is fast becoming the city's most popular quarter. "its location is unrivalled, as from an employer's perspective - it's hard to surpass the amenity offering on the doorstep. we expect high interest from occupiers who are keen to attract and retain staff through occupying one of the most strategically positioned buildings within the city centre." contractors bowmer \& kirkland were the successful bidders for the construction work which is due for completion by late 2015. & 274 & very low & Low & Socio-Economic & NA & NA & 2015-04-01 & 2015 & 1 & ECO
Frame & v.low & Regional & <500 & -0.7997063 & -1.2431481 & 1.1980777 & 1.4613788 & -0.9363498 & 1.1 & -0.8854249 & -0.3297660 & Payer & Domestic & Domestic & Domestic & Domestic|ECO & Positive\\
UK & http://www.dailymail.co.uk/wires/reuters/article-5878033/Italy-says-arrogant-France-main-enemy-migration.html & 734 & Daily Mail Online & Private/Non-Public & Online and Offline & National & medium = CP is important part of story (alongside other issues) & Political leverage & Negative & EU + Other country & No myth & NA & NA & NA & NA & NA & NA & NA & NA & UK & italy says 'arrogant' france could become main enemy on migration & 2018-06-23 & structural funds & paris, june 23 (reuters) - italy on saturday said "arrogant" france risked becoming its "no.1 enemy" on migration issues, a day before european leaders convene in brussels for a hastily arranged meeting on the divisive topic. in answer to comments by french president emmanuel macron, who said migration flows towards europe had reduced compared with a few years ago, italy's deputy prime minister luigi di maio said macron's words showed he was out of touch. "italy indeed faces a migration emergency and it's partly because france keeps pushing back people at the border. macron risks making his country italy's no.1 enemy on this emergency," di maio wrote on his facebook page. macron said european cooperation had managed to cut migration flows by close to 80 percent and problems stemmed from "secondary" movements of migrants within europe. "the reality is that europe is not experiencing a migration crisis of the same magnitude as the one it experienced in 2015," the french president said. "a country like italy has not at all the same migratory pressure as last year. ... the crisis we are experiencing today in europe is a political crisis." but italy's interior minister and deputy prime minister matteo salvini said his country had faced 650,000 arrivals by sea over the past four years, 430,000 asylum requests and the hosting of 170,000 "alleged refugees" for an overall cost of more than 5 billion euros (\$5.8 billion). "if for the arrogant president macron this is not a problem, we invite him to stop insulting and to show instead some concrete generosity by opening up france's many ports and letting children, men and women through at ventimiglia," he said in a statement, referring to the northeastern italian town at the border with france. macron also said france favoured financial sanctions for eu countries that refuse migrants with proven asylum status. "you can't have countries that massively benefit from the solidarity of the european union and that massively voice their national selfishness when it comes to migrant issues," he added, in a clear hint to hungary, poland and the czech republic, which oppose the eu relocation scheme for asylum seekers. heated debate the fate of the aquarius rescue ship and its more than 600 migrants sparked a heated debate this month over eu states' responsibilities. italy and malta refused to take in the ship which was stranded at sea for days before being offered safe haven in spain. on saturday, another migrant rescue ship, the mv lifeline, was waiting in the mediterranean having been refused harbour by italy and malta. the french president suggested that additional conditions should be attached to the granting of the eu's structural funds to guarantee the recipients take their share of asylum migrants. he made the comments after his first meeting with the new spanish prime minister, pedro sanchez, at the elysee palace. both leaders agreed that additional detention centres should be opened in europe to review asylum seekers' applications. this would come on top of so-called "disembarkation platforms" outside the eu where asylum requests could be assessed before claimants reach europe. migrants seeking to reach europe were picked up in their hundreds in the mediterranean on friday and saturday. spanish authorities said they had rescued nearly 600 migrants trying to make the perilous crossing from africa, while off the coast of libya coastguards recovered bodies of five migrants and picked up 210 survivors, the coastguard said. the container ship alexander maersk picked up 113 migrants from a boat off southern italy on friday, the ship owner said. the ship was south of sicily awaiting further instructions from the authorities. (reporting by valentina za, marine pennetier and mathieu rosemain; editing by ros russell, mark potter and andrew bolton) & 633 & medium & Medium & Power & NA & NA & 2018-06-23 & 2018 & 3 & POL
Frame & low-medium & National & 500-1000 & -0.7997063 & -1.2431481 & 1.1980777 & 1.4613788 & -0.9363498 & 1.1 & -0.8854249 & -0.3297660 & Payer & European & European & European & European|POL & Negative\\
UK & https://www.dailyrecord.co.uk/lifestyle/money/how-scottish-council-helped-residents-14032552 & 728 & Daily Record & Private/Non-Public & Online and Offline & Regional/Local & very low = CP mentioned once & Social justice & Positive & Subnational & No myth & NA & NA & NA & NA & NA & NA & NA & NA & UK & how a scottish council helped its residents save an average of £1000 per year & 2019-02-28 & european social fund & get the biggest daily stories by emailsubscribesee our privacy noticemore newslettersthank you for subscribingwe have more newslettersshow mesee our privacy noticecould not subscribe, try again laterinvalid email a scottish council tackling poverty signed up 1000 residents to help make the most of their money - and each saved on average £1000 a year. north ayrshire's money experts offered financial support to help residents make the most of their money. the area is known to have some of the worst levels of economic and social deprivation in the country with the highest levels of poverty. by providing money advice the poorest residents made over £1 million of savings since the scheme started in 2017. the council was selected as one of five areas in scotland to receive funding from the national lottery community fund and the european social fund to deliver a programme offering advice to residents to help them save money. the better off north ayrshire initiative selected residents who were aged 16-plus, who are lone parents, or living in low income or non-working households. one resident, a single mother of five, and a tenant with a local housing association received help with unmanageable debts. they also spent time helping her learn vital budgeting skills. after this help she is now £90 a week better off and said: "i'm much less stressed and feeling positive about the future for myself and my family." an advocate of the scheme was council leader councillor joe cullinane, who said: "this was absolutely incredible and a huge congratulations to the whole team for the way they embraced the challenge of helping local people save money. "poverty is an issue that affects too many people in our area. the better off north ayrshire team was set up to make a tangible difference and they clearly delivered and got to the heart of the daily financial struggles local people faced. "this initiative helped them to overcome financial barriers and provided advice and support to help them with their money worries." here's some of the main steps that helped residents save up to £1000 each a year, that anyone can consider: budgeting advice write down how much money you have coming in and how much you have going out each month. then work out where reductions could be made to expenditure. this helps you take back control and fully understand your circumstances. through budgeting assistance, one individual in north ayrshire set themselves a goal of reducing the amount of money being spent each month on takeaways, resulting in a saving of £360 a year. energy bills paying for energy costs a considerable amount of money but people can save hundreds of pounds every year by switching energy suppliers. installing a free smart meter can also help you see how much you are wasting and help you cut back on bills. advisors can also help people access energy grants and work out if there are ways to write off energy debts or move to a cheaper supplier. affordable loans residents were directed to credit unions for low cost loans, allowing them to build up savings, with higher interest rates on deposits. one person in north ayrshire who was struggling to provide food for their family due to having a high cost loan had the loan paid off after approaching the credit union. this saved the family £30 per week and over £200 in interest which has assisted the family immensely. the individual is now a credit union member, with access to all member benefits. benefits advice many people are not receiving the benefits they are entitled to but advisors can help determine what is due and help you negotiate bureaucracy to claim what's owed. in north ayrshire examples included: one woman was given help to claim maternity allowance. she received £3000, helping with the costs of a new baby. an individual was given help to apply for carer's allowance of £1924 a year which was granted due to being their son's carer. one person was helped to apply for pension credit and was subsequently awarded £2724 a year and a backdated amount of £576. help was given to an individual to apply for personal independence payment who was then awarded £4157 a year. one person struggling to pay their rent received £75 discretionary housing payment each month and is no longer being threatened with eviction. digital access skills help was given to people to enable them to apply for benefit applications online and some of the success stories include: one person was given help to apply for universal credit and was awarded £7380 a year, which included a personal element and the housing element. an individual was assisted to apply for a council tax reduction which resulted in a discount of £827 being applied to their annual council tax bill. people were also introduced to comparison sites and shown how they can assist with budgeting and sourcing the best deals. managing debt practical debt advice can work out ways to clear debts, freeze interest rates, and stop your creditors from chasing you. a range of options are available from an informal agreement with creditors, working out plans for dealing with council tax and rent arrears, free debt management plans, or the debt arrangement scheme. in north ayrshire, examples included: one person was assisted with setting up a payment plan for a debt where court action was being threatened. an individual suffering from physical and mental health issues was helped to challenge their debts and subsequently had two credit card debts written off, totaling £7545. various people were assisted with applying for sequestration where this was the most appropriate option for them. where to get help advice direct scotland - free, confidential advice on a wide range of matters, including dealing with debt. call 0808 800 9060, 9am-6pm, monday to friday, or access help 24/7 on www.advice.scot including live chat. stepchange - free, confidential expert debt advice, tailored to meet your needs. call 0800 138 1111, mon-fri, 8am-8pm, sat 8am-4pm or visit their website, www.stepchange.org. national debtline - free, confidential debt advice, including practical steps to deal with your debts. call 0808 808 4000 or online on www.nationaldebtline.org & 1044 & very low & Low & Socio-Economic & NA & NA & 2019-02-28 & 2019 & 3 & ECO
Frame & v.low & Regional & +1000 & -0.7997063 & -1.2431481 & 1.1980777 & 1.4613788 & -0.9363498 & 1.1 & -0.8854249 & -0.3297660 & Payer & Domestic & Domestic & Domestic & Domestic|ECO & Positive\\
UK & http://www.walesonline.co.uk/special-features/how-digital-marketing-campaign-made-14171777 & 709 & WalesOnline & Private/Non-Public & Online only & Regional/Local & very low = CP mentioned once & Jobs & Positive & Subnational & No myth & NA & NA & NA & NA & NA & NA & NA & NA & UK & how a digital marketing campaign made this welsh butcher sizzle & 2018-01-18 & european regional development fund & a conwy butcher's integrated digital marketing campaign helped create jobs and increase sales to become the number one premium sausage and burger brand in wales in the process. edwards of conwy and its sister brand, the traditional welsh sausage company, are now planning to expand their production space and add new products to their range as part of ambitious growth plans. the company, which employs more than 60 people and supplies major uk supermarkets as well as selling directly to consumers via its high street shop, combined social media marketing, a bilingual tv advert, bus and billboard advertising, digital advertising, and traditional media relations for the campaign. the award-winning edwards of conwy opened in 1984, with the welsh sausage company, which specialises in supplying supermarkets and food services, launching in 2004. it has won contracts with the co-op and the dylan's restaurant chain in north wales in the last 18 months. now the business is aiming for 20 per cent growth in the next 12 months, with social media set to play a key role. 'we need bigger premises to cope with the demand' marketing manager laura baker said: "we have ambitious plans to grow, and following on from our marketing campaign we have delivered growth in categories which are flat or in decline. for example, in one leading welsh retailer, we have demonstrated growth within the declining sausage category. "in total, our product range has won more than 200 awards, most notably 11 great taste awards. we only use the highest quality ingredients and all our meat is fully traceable. we combine our traditional values with a forward-thinking approach to business and marketing. "we are so busy that we need the bigger premises sooner rather than later. we recently sold a record 130,000 welsh beef burgers in a week and are selling between 1.5 and 2 million sausages a month. 'social media is the way ahead for us' "the investment in the new product development and expansion in the production facility means we're looking for a more cost-effective way of continuing the momentum and achieving our growth target of 20 per cent. and we believe social media is going to be key for us. "on social media, more than half (58 per cent) of 18 to 34 year olds and almost a third (31 per cent) of over 55s are more likely to buy food and drink after seeing pictures on the likes of facebook. "we attended a superfast business wales social media workshop, as well as receiving one-to-one advice to help with our online marketing strategy. the workshop was brilliant and very informative. we received lots of tips and tricks to implement, for example how to use twitter lists, and have noticed an increase in engagement. "the knowledge gained has allowed us to use social media in a more structured and meaningful way, which will continue to improve our brand awareness and drive more traffic to the site. the course has helped me understand how to curate social media content in a more professional way. we will also be sharing more on linkedin to build relationships with key business people on the platform. "we're also hopeful that our digital activity will help increase footfall into the shop on the high street." more about superfast business wales it is a fully-funded welsh government and european regional development fund-backed scheme offering free ict advice to help small and medium sized businesses grow through digital technology. to find out more visit businesswales.gov.wales/superfastbusinesswales to find out more about edwards of conwy, visit www.edwardsofconwy.co.uk & 611 & very low & Low & Socio-Economic & NA & NA & 2018-01-18 & 2018 & 3 & ECO
Frame & v.low & Regional & 500-1000 & -0.7997063 & -1.2431481 & 1.1980777 & 1.4613788 & -0.9363498 & 1.1 & -0.8854249 & -0.3297660 & Payer & Domestic & Domestic & Domestic & Domestic|ECO & Positive\\
UK & http://www.walesonline.co.uk/news/local-news/porth-man-scoops-apprentice-year-8168718 & 710 & WalesOnline & Private/Non-Public & Online only & Regional/Local & very low = CP mentioned once & Jobs & Positive & Subnational & No myth & NA & NA & NA & NA & NA & NA & NA & NA & UK & porth man scoops apprentice of the year award & 2014-11-25 & european social fund & a rhondda cynon taff man is on the road to a successful career after scooping a prestigious award. luke parry has been named apprentice of the year at the vsp awards - set up to celebrate the success of vocational and work-based learning in wales. luke, who is from porth, secured a job as an apprentice toolmaker after completing a six-week internship at fsg tool and die ltd in llantrisant. as a result of his success, he was awarded the accolade at a ceremony hosted by vocational skills partnership, a training consortium that offers work-based learning qualifications and placements. the first of its kind, the vsp awards recognised the work of apprentices, jobs growth wales participants and employers from across wales who have demonstrated a commitment to providing work-based learning placements. the 22-year-old is undertaking coleg morgannwg's four year pathways to apprenticeships programme, which is part financed by the welsh government's european social fund to get more young people into the workplace. the apprenticeship, which is being delivered by bridgend-based training provider tsw training, has given luke a broad experience in engineering including learning about tool design and building. more than 50 apprentices and employers from across wales were nominated for awards at the ceremony, which took place at cardiff city football club. due to the number of people taking part in apprenticeships and work schemes, the vsp hope to run the awards ceremony every year. luke said: "now that i have achieved my apprenticeship, i'm looking to start a hnd and take on any opportunity that i get to continue learning. "i hope to develop my career at fsg and believe that my apprenticeship has really helped me with this." ian slaughter, apprentice training officer at fsg tool and die ltd, said: "we took our first apprentices on at fsg in 1967. over 50\% of our current senior management team are ex-apprentices, some of which have gone on to be directors. we're very proud of luke's achievement in winning this award. "we believe it not only recognises luke's achievement, but also that of everyone that has supported luke over the past four years." & 367 & very low & Low & Socio-Economic & NA & NA & 2014-11-25 & 2014 & 1 & ECO
Frame & v.low & Regional & <500 & -0.7997063 & -1.2431481 & 1.1980777 & 1.4613788 & -0.9363498 & 1.1 & -0.8854249 & -0.3297660 & Payer & Domestic & Domestic & Domestic & Domestic|ECO & Positive\\
\addlinespace
UK & http://www.bbc.co.uk/news/uk-england-leicestershire-30703520 & 782 & BBC & Public & Online only & National & very low = CP mentioned once & Economic development & Positive & Subnational & No myth & NA & NA & NA & NA & NA & NA & NA & NA & UK & £26m makeover for 'neglected' area & 2015-01-06 & european regional development fund & a "neglected" part of leicester is set to receive a £26.5m makeover, including housing, businesses and leisure activities. city mayor sir peter soulsby has unveiled a 15-year plan for the 150-acre waterside area around the river soar. work has already begun to convert the old friar's mill into offices and a cafe. the proposals go to a six-week public consultation next week. the waterside area was once a thriving area of industry in the city but has become "neglected", he said. "we have the opportunity here to bring jobs, investment and housing into this part of the city and see it totally transformed and become an asset not a liability," he said. the plans include student accommodation, restoration of some of the area's former industrial character and new leisure activities around the river and canal. funding has come from the government and the city council but the authority has not revealed how much it has contributed. a £6m project to convert friar's mill on the site is under way, which was mostly funded by a european regional development fund grant. a public consultation starts on friday 16 january, the council said. & 198 & very low & Low & Socio-Economic & NA & NA & 2015-01-06 & 2015 & 1 & ECO
Frame & v.low & National & <500 & -0.7997063 & -1.2431481 & 1.1980777 & 1.4613788 & -0.9363498 & 1.1 & -0.8854249 & -0.3297660 & Payer & Domestic & Domestic & Domestic & Domestic|ECO & Positive\\
UK & http://www.medicalnewstoday.com/releases/314607.php & 786 & Medical News Today & Private/Non-Public & Online only & National & very low = CP mentioned once & Research \& innovation & Positive & EU + Other country & No myth & NA & NA & NA & NA & NA & NA & NA & NA & UK & how the tuberculosis vaccine may protect against other diseases & 2016-12-08 & european regional development fund & the tuberculosis vaccine is well known to help protect against other infectious diseases, as well as cancer, but the exact mechanisms have not been clear. a study published in cell reports now shows that the broad-spectrum effects of the bacillus calmette-guerin (bcg) vaccine - the most widely used vaccine in the world - could be mediated by metabolic and epigenetic changes in white blood cells called monocytes through a process called trained immunity. this discovery could pave the way for strategies that combine immunological and metabolic stimulation to boost the effectiveness of vaccines and anti-cancer therapies. "the implications of these findings are double: on the one hand, we have uncovered new biological interactions that link cellular metabolism with immune responses, and on the other hand, we have opened the door for new therapeutic approaches in which metabolism modulators modulate innate immune responses and can serve as potential novel immunotherapies," says senior study author mihai netea of radboud university medical center. "however, what it is important to realize is that this is the beginning of the process to bring this to clinical practice, and more studies are needed for that." many epidemiological studies have demonstrated bcg's capacity to protect against infections other than tuberculosis. for example, early administration of the bcg vaccine reduces child mortality, mainly due to a reduction in lower respiratory infections and harmful immune responses triggered by infections. bcg is also used to treat bladder cancer and appears to be beneficial in several other conditions, including asthma and parasitic diseases. however, it has not been clear exactly how bcg exerts its wide-ranging effects. to address this question, netea and his team examined bcg-induced metabolic changes in innate immune cells called monocytes. they found that vaccination induced a strong, long-lasting increase in glycolysis and, to a lesser extent, glutamine metabolism in mice and humans. this shift in glucose metabolism toward glycolysis was necessary to trigger trained immunity. this process relies on epigenetic changes, which affect gene activity without altering the dna sequence, to enhance the ability of innate immune cells to recognize and mount more effective responses against previously encountered pathogens. specifically, bcg-induced metabolic changes were required to induce modifications to proteins called histones, which act as scaffolds around which dna wraps. in the human cohorts, single-nucleotide variations in genes encoding glycolysis enzymes affected the induction of trained immunity in monocytes. taken together, the results show that cellular metabolism reprogramming is a central process involved in bcg-induced trained immunity. "these findings change the concept that the innate immune system cannot adapt in the long-term after an infection or vaccination," netea says. "the whole concept that the function of innate immune cells can change in a stable way, for example, being improved by certain vaccines such as bcg, is a paradigm shift in immunology, as until not too long ago it was assumed that only the adaptive immune system can adapt to previous infections or vaccinations." host immune responses are classically divided into innate immune responses, which react rapidly and nonspecifically upon encountering a pathogen, and adaptive immune responses, which are slower to develop but are specific and build up immunological memory, netea explains. the discovery of trained immunity has challenged the dogma that only adaptive immunity can build immunological memory. according to netea, the next step is to conduct a bigger, broader analysis of circulating monocytes in bcg-vaccinated individuals at risk for infections. "in the future, bigger studies should assess inter-individual variation in these responses, in order to be able to identify which factors influence vaccination responses at the level of a person," netea says. "in the end, a better understanding of bcg-induced trained immunity could lead to the development of strategies that alter cellular metabolism pathways to improve human host defense mechanisms and boost the effectiveness of vaccines and immunotherapy in patients." the researchers were supported by the european research council; the netherlands organization for scientific research; the northern portugal regional operational programme, under the portugal 2020 partnership agreement, through the european regional development fund; and the fundação para a ciência e tecnologia. article: immunometabolic pathways in bcg-induced trained immunity, rob j.w. arts, agostinho carvalho, claudia la rocca, carla palma, fernando rodrigues, ricardo silvestre, johanneke kleinnijenhuis, ekta lachmandas, luís g. gonçalves, ana belinha, cristina cunha, marije oosting, leo a.b. joosten, giuseppe matarese, reinout van crevel, mihai g. netea, cell reports, doi: 10.1016/j.celrep.2016.11.011, published 6 december 2016. & 749 & very low & Low & Socio-Economic & NA & NA & 2016-12-08 & 2016 & 2 & ECO
Frame & v.low & National & 500-1000 & -0.7997063 & -1.2431481 & 1.1980777 & 1.4613788 & -0.9363498 & 1.1 & -0.8854249 & -0.3297660 & Payer & European & European & European & European|ECO & Positive\\
UK & http://www.chroniclelive.co.uk/business/business-news/jobs-created-generator-launches-17m-11674012 & 779 & Chronicle Live & Private/Non-Public & Online and Offline & Regional/Local & very low = CP mentioned once & Economic development & Positive & EU + Subnational & No myth & NA & NA & NA & NA & NA & NA & NA & NA & UK & jobs to be created as generator launches £1.7m growth programme & 2016-07-28 & european regional development fund & the digital futures programme is expected to create 75 jobs and help develop 112 companies a £1.7m programme to trigger growth in the north east's expanding creative and digital sector is expected to create scores of new jobs and help develop more than 100 firms. creative development agency generator has launched the seven-figure programme - its latest award confirmed from the european regional development fund (erdf) - pledging to use the investment to drive and develop the wider commercial creative and digital industries across the region. the 30-month digital futures programme will assist new and existing businesses as well as potential entrepreneurs within the creative and digital industries based in the north east local economic partnership area. generator, which helped create over 200 jobs and safeguard 151 thanks to its last erdf backed initiative ending in 2015, said the future has never been brighter for the region's growing sector. the agency's chief executive officer, jim mawdsley, said: "generator is extremely excited about the latest erdf award and the prospect of building upon our huge success in the area of business support. "having earned a national reputation for our unique approach to flexible sector specialist business support we will be implementing our techniques and mechanisms sector wide. "the digital futures programme will enable us to help businesses achieve significant growth as well as provide effective interventions at critical points in a business's life cycle. "this programme, alongside our acquisition and rebuilding of digital union, will help close the productivity gap that exists within the north east and the rest of the country by ensuring growth of businesses within the sector and therefore lead to the creation of jobs in the creative and digital industries. we're looking forward to working with our business constituency and partners, including sunderland software city and campus north to achieve this goal." digital futures has set itself some bold targets and will deliver a minimum of 600 development opportunities to a minimum of 262 businesses ensuring that they are equipped with the information, skills and confidence they will need to achieve growth. it is expected that the new programme will create at least 75 jobs in the region and develop 112 new firms helping to set new ambitions and adopt emerging business models. "digital futures will create the right environment for those individuals and businesses with innovative, profitable ideas so that they can develop, succeed and grow," said mr mawdsley. "it has been reported that uk cities with the highest concentrations of 'new work' businesses are those involved in the creative, professional and digital sectors. "we will work on developing the strategy, business development, vision and performance of those businesses based in the north east to meet the region's growth priorities." to strengthen the agency's position in the creative and digital industries, generator recently acquired digital union the north east's creative and digital networking organisation. this has allowed generator access to a wide range of businesses in the sector and is currently rebuilding the proposition to one that is truly representative and fit for purpose for this industry that continues to adapt and evolve. & 526 & very low & Low & Socio-Economic & NA & NA & 2016-07-28 & 2016 & 2 & ECO
Frame & v.low & Regional & 500-1000 & -0.7997063 & -1.2431481 & 1.1980777 & 1.4613788 & -0.9363498 & 1.1 & -0.8854249 & -0.3297660 & Payer & Domestic & European & Mixed & Domestic|ECO & Positive\\
UK & http://www.medicalnewstoday.com/releases/293929.php & 787 & Medical News Today & Private/Non-Public & Online only & National & very low = CP mentioned once & Research \& innovation & Positive & EU + Other country & No myth & NA & NA & NA & NA & NA & NA & NA & NA & UK & study: medical marijuana pill may not be effective in treating behavioral symptoms of dementia & 2015-05-23 & european regional development fund & a new study suggests that medical marijuana pills may not help treat behavioral symptoms of dementia, such as aggression, pacing and wandering. the research is published online issue of neurology, the medical journal of the american academy of neurology. however, researchers did find that the drug dosage used in the clinical trial was safe and well-tolerated. "our study results are valuable since any firm evidence of the effectiveness and safety of medical marijuana in this disease area is scarce," said study author geke a.h. van den elsen, md, with radboud university medical center in nijmegen, the netherlands. "ours is the largest study carried out so far on evaluating this drug for behavioral symptoms of dementia." for the study, researchers randomly selected 50 participants with dementia and behavioral symptoms to receive 1.5 milligrams of medical marijuana or a placebo pill three times per day for three weeks. the medical marijuana pill contained tetrahydrocannabinol (thc), which is the main chemical involved in marijuana's psychoactive effects. the main study measurement was change in scores on a test of behavioral symptoms called the neuropsychiatric inventory, assessed at the start of the study and after two and three weeks. the test scores improved for both the medical marijuana and the placebo groups, but there was no significant difference between the scores for the two groups. there was also no difference between the two groups for participants' quality of life, daily living activities or pain-related behavior and pain intensity. van den elsen said improvements in the placebo group could be due to several factors, including attention and support from the study personnel, expectations of patients and caregivers and training of nursing home personnel. people in the two groups had a similar number of mild and moderate side effects. there were no serious side effects in either group. "since the side effects were mild to moderate, it's possible that a higher dose could be tolerated and could possibly be beneficial," said van den elsen. "future studies are needed to test this. a drug that can treat the behavioral symptoms of dementia is much needed, as about 62 percent of dementia patients in the general community and up to 80 percent of nursing home residents experience these symptoms." the study was supported by the european regional development fund and the province of gelderland. the drug was provided by echo pharmaceuticals, weesp, the netherlands. to learn more about dementia, please visit www.aan.com/patients. & 412 & very low & Low & Socio-Economic & NA & NA & 2015-05-23 & 2015 & 1 & ECO
Frame & v.low & National & <500 & -0.7997063 & -1.2431481 & 1.1980777 & 1.4613788 & -0.9363498 & 1.1 & -0.8854249 & -0.3297660 & Payer & European & European & European & European|ECO & Positive\\
UK & http://www.dailymail.co.uk/wires/pa/article-3737761/Post-Brexit-funding-gap-key-EU-backed-projects-set-cost-4-5bn-year.html & 751 & Daily Mail Online & Private/Non-Public & Online and Offline & National & low = CP mentioned more times but NOT important part of story (mainly about others issues) & Institutional bargaining over funding & Factual & EU + National & No myth & NA & NA & NA & NA & NA & NA & NA & NA & UK & post-brexit funding gap for key eu-backed projects set to cost £4.5... & 2016-08-12 & european social fund & post-brexit funding gap for key eu-backed projects set to cost £4.5bn a year by press association published: 22:54 gmt, 12 august 2016 | updated: 22:54 gmt, 12 august 2016 british taxpayers will pay around £4.5 billion a year to plug the post-brexit funding gap for key european union-backed projects that support scientists, farmers and infrastructure, the chancellor will announce. philip hammond will guarantee government funding for projects backed by the eu structural and investment fund, including agri-environment schemes, which are signed off before this year's autumn statement. the treasury will assess whether other similar projects that are signed after the mini budget - expected in november or december - should also get a guarantee, in a bid to reassure uk organisations. philip hammond will guarantee government funding for projects backed by the eu structural and investment fund which are signed off before this year's autumn statement and if organisations bid directly to the european commission on a competitive basis, such as universities seeking horizon 2020 research funding, the treasury will underwrite the payments even if the projects continue after brexit. on saturday, mr hammond will also promise that the current level of common agricultural policy (cap) pillar one funding for farmers will be matched by the uk after it comes out of the eu until 2020, as a domestic system is put in place. ahead of the announcement, mr hammond said: "the uk will continue to have all of the rights, obligations and benefits that membership brings, including receiving european funding, up until the point we leave the eu. "we recognise that many organisations across the uk which are in receipt of eu funding, or expect to start receiving funding, want reassurance about the flow of funding they will receive. "that's why i am confirming that structural and investment funds projects signed before the autumn statement and horizon research funding granted before we leave the eu will be guaranteed by the treasury after we leave. "the government will also match the current level of agricultural funding until 2020, providing certainty to our agricultural community, who play a vital role in our country. "we are determined to ensure that people have stability and certainty in the period leading up to our departure from the eu and that we use the opportunities that departure presents to determine our own priorities." asked how much he expected the move to cost, mr hammond said: "well, it will depend on when we leave the eu and it will depend on the level of applications for these structural and investment funds and for the research funds horizon, but around probably, around about £4.5 billion a year would be the level we would expect." the structural and investment funds that will be guaranteed include cap pillar two, t he european social fund, t he european maritime and fisheries fund and t he european regional development fund, including european territorial cooperation. examples of projects that have received or are due to receive regional development fund money include: :: £5 million for the graphene engineering innovation centre at the university of manchester. :: £9 million for the manufacturing growth programme to support areas in the midlands, yorkshire and the humber and the east of england. :: £3 million for a new life sciences incubation and innovation centre at porton down in wiltshire. chief secretary to the treasury david gauke has also written to the devolved administrations to confirm the same level of assurances offered to uk government departments in relation to programmes they administer but for which they are expected to rely on eu funding. brexit secretary david davis said: "this announcement shows that the government is ensuring that those people and organisations currently supported by eu funding can continue to benefit from a measure of continuity. this will offer reassurance to them, and help ease the transition to our new relationship with the eu." business and energy secretary greg clark said: "by underwriting the significant horizon 2020 grants we are showing the extent of our commitment, standing squarely behind our researchers and scientists as they continue working with their european partners to develop new technologies, discover life-saving medicines and pioneer everyday innovations that will benefit all hard-working britons." work and pensions secretary damian green said: "the fund supports hundreds of initiatives across england like the ways to work programme in liverpool, which is helping disadvantaged people in the area to gain new skills and turn their lives around for the better. with this decision, providers can plan with certainty and help more people gain the security and dignity of work." local government secretary sajid javid said: "local enterprise partnerships are a vital part of our efforts to rebalance the economy, and have helped create thousands of jobs over the past five years. "guaranteeing eu funding will further support this work by enabling them to plan ahead with certainty so businesses, universities and local authorities across the country can enable economic growth." the national trust called for backing for farming and environment projects which get under way after the autumn statement. a spokesman said: "we welcome the government's decision to continue with funding for newly-agreed and existing agri-environment schemes. but set against this is the continued uncertainty should new applications be restricted beyond this autumn. this would put at serious risk decades of effort by farmers and organisations like ours to protect and enhance our countryside. "farmers need certainty as we move to a more sustainable model of support for farming which puts the recovery and resilience of the natural environment at its heart. "better funded and more widely available agri-environment schemes would offer a 'bridge' between the old and the new, helping farmers prepare for a future of new market opportunities whilst delivering greater public benefits from public money." & 974 & low & Low & Power & NA & NA & 2016-08-12 & 2016 & 2 & POL
Frame & low-medium & National & 500-1000 & -0.7997063 & -1.2431481 & 1.1980777 & 1.4613788 & -0.9363498 & 1.1 & -0.8854249 & -0.3297660 & Payer & Domestic & European & Mixed & Domestic|POL & Neutral\\
\addlinespace
UK & http://www.telegraph.co.uk/education/2017/04/25/universities-will-see-funding-increase-brexit-education-select/ & 747 & The Telegraph & Private/Non-Public & Online and Offline & National & medium = CP is important part of story (alongside other issues) & Institutional bargaining over funding & Negative & EU + National & 2.Rich countries pay & NA & NA & NA & NA & NA & NA & NA & NA & UK & universities will see their funding increase after brexit, education select committee says & 2017-04-25 & european regional development fund & universities will see their funding increase after brexit, the education select committee has told academics. in a report published on tuesday, mps noted that once britain has left the eu, universities will no longer be able to draw funds from the european structural and investment funds. but since the uk is a "net contributor" to these funds, the government could set up its own regional growth fund after brexit which "could easily exceed" the amount of investment that universities traditionally received from the eu. the report, titled exiting the eu: challenges and opportunities for higher education, noted: "it is likely structural funding will be a casualty of the uk's withdrawal from the eu, as these funds are distributed only to eu members." it said that the uk received £1.67 billion from european regional development fund (erdf) and european social fun (esf) in 2014/15, which is only 29 per cent of the amount the british government pays in. in a submission to the select committee, the university alliance said that the erdf and esf hand out around £100 million each year in british universities to fund projects promoting enterprise and growth in their local area. but if the government set up a new fund, far more could be invested in universities. mps also identified the government's immigration policy and european staff as priorities for universities during brexit negotiations. alistair jarvis, deputy chief executive of universities uk, said: "the government should seek to secure continued close collaboration with eu research partners and also provide certainty for eu staff currently working in uk universities in terms of work and residency rights. "changes to our immigration system are also needed to ensure that the uk remains a destination of choice for international talent and students. "as large and complex organisations, universities plan for years down the line, so it important that we receive clarity of the government's positions on these crucial issues as soon as possible." the report also said that the department for education, along with the home office and the department for business, energy and industrial strategy should publish a contingency plan for higher education, so that they are prepared for a "no deal" situation. & 368 & medium & Medium & Power & NA & NA & 2017-04-25 & 2017 & 2 & POL
Frame & low-medium & National & <500 & -0.7997063 & -1.2431481 & 1.1980777 & 1.4613788 & -0.9363498 & 1.1 & -0.8854249 & -0.3297660 & Payer & Domestic & European & Mixed & Domestic|POL & Negative\\
UK & http://www.leicestermercury.co.uk/news/business/midlands-engine-investment-fund-launches-1253317 & 765 & Leicester Mercury & Private/Non-Public & Online only & Regional/Local & very low = CP mentioned once & Economic development & Positive & National + Subnational & No myth & NA & NA & NA & NA & NA & NA & NA & NA & UK & midlands engine investment fund launches £100m support for smes & 2018-02-23 & european regional development fund & british business bank releases further wave of its £250 million midlands engine investment fund get business updates directly to your inbox+ subscribethank you for subscribing!could not subscribe, try again laterinvalid email tens of millions of pounds of funding to power the so-called midlands engine has been released. the british business bank has released a further wave of its £250 million midlands engine investment fund - with up to £100 million of equity finance available to small businesses in the east, west and south-east midlands. the money comes from the european regional development fund, and has been made available to create new jobs and harness business opportunities. the leicester and leicestershire enterprise partnership is one of 10 leps around the region working with the bank to promote the funding. llep chairman nick pulley attended the launch of the money with sajid javid mp, housing secretary and ministerial champion for the midlands engine. mr javid said: "the midlands engine already boasts over 14 per cent of the uk's high-growth businesses and its economy is worth more than £230 billion - larger than countries like denmark. "we want to harness its huge potential and help give small firms across the midlands that much needed boost to grow their business. "ensuring every part of the uk can play to its strengths and spread prosperity is central to our industrial strategy, and this fund will not only generate jobs to benefit the entire midlands but, ultimately, the british economy." patrick magee, chief commercial officer of the british business bank, said: "on tuesday the british business bank's small business finance market report revealed significant regional imbalances in the supply of equity finance across the uk. "we believe there is a major opportunity to increase the supply of equity finance in the midlands, and to enhance the growth ecosystem. "the british business bank already supplies £1.2 billion of finance to midlands businesses and the midlands engine investment fund will provide more in the most needed areas. "the launch of the midlands engine investment fund's equity finance will help to grow the region's access to funds to support its economic potential and collective talent. "i encourage midlands businesses looking to take the next step in their growth journey to connect with our three fund managers to see how equity finance could benefit them. to find out how you can apply for funding click here & 403 & very low & Low & Socio-Economic & NA & NA & 2018-02-23 & 2018 & 3 & ECO
Frame & v.low & Regional & <500 & -0.7997063 & -1.2431481 & 1.1980777 & 1.4613788 & -0.9363498 & 1.1 & -0.8854249 & -0.3297660 & Payer & Domestic & Domestic & Domestic & Domestic|ECO & Positive\\
UK & http://www.belfasttelegraph.co.uk/news/northern-ireland/northern-ireland-would-lose-billions-in-euro-exit-warns-martin-mcguinness-34414426.html & 754 & Belfast Telegraph & Private/Non-Public & Online and Offline & Regional/Local & medium = CP is important part of story (alongside other issues) & Institutional bargaining over funding & Positive & EU + Other country & No myth & NA & NA & NA & NA & NA & NA & NA & NA & UK & northern ireland would lose billions in euro exit, warns... leaving the eu would have a devastating financial impact on northern ireland, with billions of... & 2016-02-02 & european social fund & leaving the eu would have a devastating financial impact on northern ireland, with billions of euros worth of funding lost, martin mcguinness has warned. the deputy first minister told the assembly that loss of access to various eu money streams, including structural and regional development funds, would be severely detrimental. "the financial implications are absolutely massive for us, not least for our farming community. the north of ireland is a net beneficiary of the european union," said mr mcguinness. "we have received significant support from the eu through a number of different funding programmes, which, in the event of a brexit, we would no longer have access to." mr mcguinness said structural and regional development funds, comprising the european regional development fund, the european social fund, interreg and the peace iv programme are worth £742m in the six years 2014-2020. "we would also lose access to funding under the common agricultural policy, which is worth approximately €2.5bn (£1.9bn)in the 2014-2020 period. that represents a massive investment in the sector," he said. & 176 & medium & Medium & Power & NA & NA & 2016-02-02 & 2016 & 2 & POL
Frame & low-medium & Regional & <500 & -0.7997063 & -1.2431481 & 1.1980777 & 1.4613788 & -0.9363498 & 1.1 & -0.8854249 & -0.3297660 & Payer & European & European & European & European|POL & Positive\\
UK & http://www.belfasttelegraph.co.uk/business/news/major-new-push-to-aid-women-in-their-business-careers-31552935.html & 713 & Belfast Telegraph & Private/Non-Public & Online and Offline & Regional/Local & very low = CP mentioned once & Jobs & Positive & Subnational & No myth & NA & NA & NA & NA & NA & NA & NA & NA & UK & major new push to aid women in their business careers & 2015-09-24 & european social fund & women in business has launched its new programme for supporting women through every stage of their careers, including self-employment. the connect programme is funded by the european social fund and delivered by east belfast enterprise. it's also supported by the department for employment and learning and belfast city council. women in business chief executive roseann kelly said the programme would help women "through every stage of their career, from exploring their options, considering self-employment, to growing and developing professionally, becoming leaders and role models." frances magee and linzi conway said they benefited from earlier programmes. ms magee has set up interior design business dress this house, while ms conway started key to success consultants (kts). & 118 & very low & Low & Socio-Economic & NA & NA & 2015-09-24 & 2015 & 1 & ECO
Frame & v.low & Regional & <500 & -0.7997063 & -1.2431481 & 1.1980777 & 1.4613788 & -0.9363498 & 1.1 & -0.8854249 & -0.3297660 & Payer & Domestic & Domestic & Domestic & Domestic|ECO & Positive\\
UK & http://www.mirror.co.uk/news/world-news/desperate-refugees-vow-starve-themselves-6450560 & 736 & Mirror & Private/Non-Public & Online and Offline & National & very low = CP mentioned once & Political leverage & Negative & EU + Other country & No myth & NA & NA & NA & NA & NA & NA & NA & NA & UK & desperate refugees vow to starve themselves to death in no-man's land between hungary and serbia & 2015-09-15 & structural funds & desperate refugees stuck in serbia behind a ­towering razor-wire fence have vowed to starve to death if they cannot pass into western europe, reports russell myers on the border between hungary and serbia. hundreds of migrants staged a mass sit-in, blocking a major crossing point, as hungary declared a state of emergency after closing its border. it came as german chancellor angela merkel piled pressure on david cameron and other eu leaders by demanding an emergency brussels summit next week over the crisis. hungary's government has rushed through tough anti-migration laws, meaning those trying to enter illegally will be jailed. more than 60 refugees were arrested today. groups pouring into a no-man's land area between hungary and serbia refused to move, making threats to starve themselves to death if they were not allowed to pass. dozens angrily threw down bread, tubs of chocolate spread and bottles of water given to them by charities. aasif abdullah, from afghanistan, said: "we will not eat this food or drink their water until we can cross this border. "we will die here on the road, we will starve ourselves to death and then they will have to pick up our bodies from the streets. "is that what europe wants? we do not want to stay here, we want to pass and will not move from here until we are allowed to." a young boy was lifted up to hang a t-shirt, daubed with the words "europe shame", on the three-metre-high wire gates. while iihan sadi, from syria, shouted: "remove these borders. remove these police otherwise we will not move ourselves. "we have travelled hundreds of miles to be here and suddenly now we cannot pass. where is the humanity in this? show me. "they offer food and water but we will not take it." women wailed and children cried as the crowd surged forward, banging on the gates. hundreds chanted "we want to pass" and "no food, no water, until you open this border" and "open the door, open the door" late into the night. men held young children aloft holding signs with the words "mama merkel help us". behind the raucous crowds in the hungarian side stood around 200 armed police in riot gear as the authorities moved to bolster their defences. watch this video again watch next in further harrowing scenes, 22 migrants drowned and 211 were saved today when a boat capsized off turkey's coast while trying to reach greek island kos. it came as more syrian refugees arrived at piraeus, near athens, on a passenger ship. hungary's prime minister viktor orban vowed to shut out tens of thousands of migrants by securing the country's southern frontier. the crackdown means police can detain anyone who tries to breach the 110-mile wire fence put up on monday. those caught face an automatic jail term of 12 months to three years. a record of nearly 10,000 migrants crossed the border before the curfew, and around 4,000 were believed to have tried the same route today. on monday night, the mirror watched soldiers secure a rail track across the border using a locomotive trailer surrounded by razor wire. and as dozens of migrants began laying tents on the road, hungary announced it would extend its fence to neighbouring romania. some were optimistic about claiming asylum, having been ushered to reception points by hungarian officials. but tonight it was confirmed all 16 cases ruled on had been denied. save the children's justin forsyth blasted hungary's "extreme response". he said: "hundreds of children are among those who have fled bombs and bullets in syria, risked their lives crossing the sea, and then trekked for days to hungary, only to have the doors closed in their face, literally." serbia's social affairs minister, aleksandar vulin, agreed the crossing should remain open, adding: "the hungarians never told us they will close the border." meanwhile, erno simon, of the un's refugee agency unhcr, warned of an "impending humanitarian crisis". sources said hungary had also angered western european nations for its handling of the crisis. yet eu states including germany, austria, the czech republic, slovakia and finland have introduced new border controls too, effectively ending the 20-year schengen agreement of free movement across much of europe. german interior minister thomas de maiziere today made a veiled threat that the uk and other eu nations reluctant to take in thousands of refugees could face penalties from brussels. he said: "we need to talk about ways of exerting pressure. these are often countries that receive a lot of structural funds from the eu. "european commission president jean-claude juncker has suggested we should look at whether these countries should get less structural funds, which i agree with." and mrs merkel called for support from the whole eu, with an emergency council being held next tuesday to reach an agreement on relocating 120,000 asylum seekers from frontier states. a no10 spokeswoman said david cameron today had "a good discussion" with polish president andrzej duda about tackling "the root causes of the problem". meanwhile austria's interior ministry warned it was running out of shelter for migrants arriving from hungary. it had 20,000 places as of monday for 19,700 refugees. a further 4,000 arrived today. & 894 & very low & Low & Power & NA & NA & 2015-09-15 & 2015 & 1 & POL
Frame & v.low & National & 500-1000 & -0.7997063 & -1.2431481 & 1.1980777 & 1.4613788 & -0.9363498 & 1.1 & -0.8854249 & -0.3297660 & Payer & European & European & European & European|POL & Negative\\
\addlinespace
UK & http://www.chroniclelive.co.uk/business/business-news/generator-launches-420000-development-programme-11735020 & 719 & Chronicle Live & Private/Non-Public & Online and Offline & Regional/Local & very low = CP mentioned once & Cultural development & Positive & Subnational & No myth & NA & NA & NA & NA & NA & NA & NA & NA & UK & generator launches £420,000 development programme for creative young people & 2016-08-11 & european regional development fund & new ladder programme will focus on skills development for the music, television, film and digital media industries more than 100 north east young people looking to enter the creative and digital industries are set to benefit from a six-figure fund aimed at supporting budding entrepreneurs. the newcastle-based music and creative development agency generator has set up the £420,000 ladders programme in association with local organisations, this is creative enterprise (tice), youth focus and foundation futures. over the next three years, it is hoped at least 100 people aged 18 to 24 can be helped by the initiative, which will consist of creative courses, industry workshops and other support. the programme comes as research from the north east england chamber of commerce shows 49\% of young people are either not very informed or not informed at all about opportunities available to them. in the fast-growing creative sector, it is thought 1.2m new employees will be required by 2022, but generator says it is difficult for young people to get a break in the industry. louise henry, its enterprise development coordinator, said: "the creative and digital industries typically recruit from a narrow section of society and ladders will provide an opportunity to meet with industry professionals, learn practical and creative skills, visit potential workplaces, get interview experience and learn about enterprise. "the focus is very much on the practical." the programme forms part of generator's £1.7m digital futures project, supported by the european regional development fund (erdf), with additional assistance from the big lottery fund. it will see tice deliver creative courses and industry workshops focused on developing skills for the music, television, film and digital media industries. employability training will also be offered alongside wraparound support to ensure young people progress and that their ambition continues to grow. to help boost participants' confidence, foundation futures will likewise provide tailored one-to-one sessions. tice managing director jennifer barret said: "tice is delighted to be a delivery partner on ladders; the project has been carefully designed to fulfil the growing needs of the creative and digital sectors. "more importantly, it is about giving young people the chance to take ownership and manage their own career progression, assessing their skills and being proactive in finding career building opportunities through self-employment and employment options." neil burke, regional development officer at youth focus added: "ladders will really support young people to enter the creative sector through its rounded approach which will not only provide participants with creative skills and sector knowledge, but it will also provide young people with the generic support which they need to enter the world of work." places on the free ladders creative courses, which begin in september, are extremely limited. for more information, email louise@generator.org.uk or call 0191 255 4467. & 471 & very low & Low & Socio-Economic & NA & NA & 2016-08-11 & 2016 & 2 & ECO
Frame & v.low & Regional & <500 & -0.7997063 & -1.2431481 & 1.1980777 & 1.4613788 & -0.9363498 & 1.1 & -0.8854249 & -0.3297660 & Payer & Domestic & Domestic & Domestic & Domestic|ECO & Positive\\
UK & http://www.manchestereveningnews.co.uk/business/business-growth-hub-secures-172k-11159670 & 784 & Manchester Evening News & Private/Non-Public & Online and Offline & Regional/Local & very low = CP mentioned once & Research \& innovation & Positive & EU + National + Subnational & No myth & NA & NA & NA & NA & NA & NA & NA & NA & UK & business growth hub secures £172k funding & 2016-04-10 & european regional development fund & the business growth hub, part of manchester growth company, has secured £172,000 in funding to support local life science companies. this will help them to develop new health products and access european markets. to be delivered over the next four years the funding is part of boost 4 health, a european union project developing a northern europe life science network. forty local businesses will benefit from one-to-one guidance from a life science specialist and will be able to travel to other european boost 4 health centres to meet industry experts, tap into local knowledge and learn how to enter international markets. boost 4 health aims to generate £20m in sales in the life sciences, a key priority growth sector for greater manchester. the fund will be delivered by the hub's recently expanded innovation service and new sectors team, which provides business support and advice to companies working in, or diversifying into, the life science sector. read more: growth: new support programmes are stimulating growing firms the project will also develop and enhance links between the hub's teams and local partners including the greater manchester academic health science network, the nhs's innovation experts trustech, and other bioscience and life science organisations. the project is funded by the interreg europe programme, part of the european regional development fund. the hub's head of innovation and programme development, chris greenhalgh said: "greater manchester is leading the way in life sciences. we have europe's largest clinical academic campus, europe's biggest cancer treatment centre and the largest clinical trials unit in the world. "this funding will help innovative local life science companies bring their own ideas to market. by offering mentoring and a chance to meet international experts it will help them grow, generate jobs and attract new investment." & 302 & very low & Low & Socio-Economic & NA & NA & 2016-04-10 & 2016 & 2 & ECO
Frame & v.low & Regional & <500 & -0.7997063 & -1.2431481 & 1.1980777 & 1.4613788 & -0.9363498 & 1.1 & -0.8854249 & -0.3297660 & Payer & Domestic & European & Mixed & Domestic|ECO & Positive\\
UK & http://www.chroniclelive.co.uk/business/business-news/journalist-launches-digital-magazine-promote-10027320 & 729 & Chronicle Live & Private/Non-Public & Online and Offline & Regional/Local & very low = CP mentioned once & Social justice & Positive & Subnational & No myth & NA & NA & NA & NA & NA & NA & NA & NA & UK & journalist launches digital magazine to promote the achievements of the region's women & 2015-09-10 & european regional development fund & a fellowship scheme run by teesside university has helped a former newspaper journalist launch a digital magazine to promote the achievements of the region's women. express north was set up by lucy richardson, previously a senior reporter at the northern echo and hexham courant. the magazine, which will generate revenue from local advertisers keen to target a niche market, will have fresh content added five days a week, including both written articles and videos, and will feature career profiles, real-life stories, opinion pieces, and fashion and beauty sections. ms richardson set up her new venture after gaining support from teesside university's digitalcity fellowship scheme. she said: "i felt that there was a big gap in the market for something new. "the internet is buzzing with local female facebook networking groups and the federation of small businesses held its first north east networking lunch for women recently. "i want to create something that's interactive, regularly updated and can be read as easily on a phone as a laptop. "i also want to present stories in different ways - a fashion store's top autumn/winter picks would work well as a youtube video as would a tour of festival of thrift in darlington later this month." the tagline for the new website is: 'no beach bodies, no kiss and tells but lots of revealing women'. "i'm fed up of standing in line in a supermarket or chemist faced with magazine headlines that shame celebrities if they are not a size zero and praise them for pinging back into a bikini days after having a baby," ms richardson explained. "i want express north to be the antithesis to those publications." through the digitalcity fellowship, ms richardson received a start-up grant as well as four days of help and advice from industry experts. as part of the mentoring programme, she also had the fully-responsive website designed by former digitalcity fellow malik jani, of laughing beards. she added: "when i applied for the fellowship scheme i just had mock-ups of what i wanted to create but now that dream has been turned into a reality. "the feedback i've had so far about express north has been really positive. "i'm excited about what it could achieve and want to create my own success story." digitalcity is part-financed by the european regional development fund. laura woods, director of the forge at teesside university, which provides a single-point of contact for companies looking to access business services, said: "it's good to see express north, led by a female entrepreneur, championing the talent and achievements of the region's women. "we're delighted that through digitalcity we've been able to help launch another great business in the north east." & 464 & very low & Low & Socio-Economic & NA & NA & 2015-09-10 & 2015 & 1 & ECO
Frame & v.low & Regional & <500 & -0.7997063 & -1.2431481 & 1.1980777 & 1.4613788 & -0.9363498 & 1.1 & -0.8854249 & -0.3297660 & Payer & Domestic & Domestic & Domestic & Domestic|ECO & Positive\\
UK & http://www.dailymail.co.uk/wires/reuters/article-5422029/Merkel-eyes-overhaul-EU-finances-post-Brexit-bloc.html & 744 & Daily Mail Online & Private/Non-Public & Online and Offline & National & medium = CP is important part of story (alongside other issues) & Institutional bargaining over funding & Negative & EU + Other country & No myth & Political leverage & Negative & Other country & No myth & NA & NA & NA & NA & UK & merkel eyes overhaul of eu finances for post-brexit bloc & 2018-02-22 & structural funds & by madeline chambers and paul carrel berlin, feb 22 (reuters) - brexit offers the european union an opportunity for a broad rethink of its financial set-up, german chancellor angela merkel said on thursday before an eu summit that will tackle the bloc's future budget. addressing the bundestag (lower house of parliament), merkel made clear that the future of the eu would be a priority in her fourth term, provided a coalition deal between her conservatives and the pro-eu social democrats is approved by spd members. "we need a new start for europe," said merkel, adding the looming debate on a new budget for the 27-member bloc after britain's withdrawal in 2019 could lead to some major changes. "the debate about the future financial framework is also a chance to look at the finances of the eu as a whole," she said. leaders at friday's summit will discuss whether to boost the next seven-year budget to pay for common policies on security, defence and migration between 2021 to 2027. germany, europe's biggest economy, is the biggest net contributor and has said it is ready to pay more although some others, including the netherlands, argue that a smaller bloc - without britain - implies a smaller budget. merkel sought to reassure those in her conservative bloc who are more sceptical about deeper eu integration and paying out for poorer, debt-ridden member states. "solidarity cannot be a one-way street," she said. "the stability and growth pact will remain the compass for europe," she added, reiterating that liabilities and supervision must go hand in hand as part of any euro zone reform. taking aim at some eastern european members, merkel also said the bloc's structural funds should in future be linked to taking in foreign migrants. eastern members with right-wing nationalist governments have rejected an eu burden-sharing plan for migrants coming from the middle east, africa and asia. "with the distribution of structural funds, we must ensure that the allocation criteria in future reflect the engagement of many regions and municipalities in absorbing and integrating migrants," she said. "blackmail" that idea is "political blackmail," said the far-right, eurosceptic alternative for germany (afd), which will be the main opposition party if a "grand coalition" goes ahead. the afd made clear it rejects bigger payments to the eu. "germany is going from teacher to taskmaster to paymaster," alexander gauland, head of the afd's parliamentary group, said to applause from his lawmakers in a riposte to merkel's address. the chancellor, at risk of being supplanted as europe's most influential leader by french president emmanuel macron, said the eu should be more ambitious in developing a response to growing global political and economic pressures. "more than ever we need european answers to the pressing, big questions of our time," she told the bundestag, urging the eu to adjust to digital advances in technology to be able to compete economically in the future. macron has offered an ambitious vision for europe, but has had to wait for a response from germany, due in part to merkel's protracted attempts to cement a governing coalition. she emphasised that european policy would be front and centre in a new grand coalition with the social democrats, which would be a repeat of her last coalition that prevailed up to the sept. 24 election. "germany can only do well in the long run if europe does well," she said. "it is no coincidence that the first chapter of the coalition agreement is on europe". the coalition parties have agreed on a deal but spd rank-and-file are voting to approve it. the result is due on march 4. (writing by madeline chambers; editing by mark heinrich) & 628 & medium & Medium & Power & Power & NA & 2018-02-22 & 2018 & 3 & POL
Frame & low-medium & National & 500-1000 & -0.7997063 & -1.2431481 & 1.1980777 & 1.4613788 & -0.9363498 & 1.1 & -0.8854249 & -0.3297660 & Payer & European & European & European & European|POL & Negative\\
UK & https://www.walesonline.co.uk/news/wales-news/projects-wales-eu-funded-15147507 & 761 & WalesOnline & Private/Non-Public & Online only & Regional/Local & very high = CP is most important issue + CP is mentioned in title/headline & Economic development & Positive & EU + National + Subnational & No myth & Institutional bargaining over funding & Factual & EU + National & No myth & NA & NA & NA & NA & UK & the projects in wales the eu has funded & 2018-09-14 & european regional development fund & get daily updates directly to your inboxsubscribesee our privacy noticethank you for subscribingsee our privacy noticecould not subscribe, try again laterinvalid email if you've ever wondered what projects the eu has funded in your area, then a new interactive map has the answers for you. the new map gives a detailed breakdown of the eu funded projects in wales and the rest of the uk. according to the map's developers myeu, the european union invests about £5bn a year in the uk. this cash is mainly invested in farming, research, jobs, growth and culture initiatives, ranging from small amounts in the hundreds of pounds right up to millions of pounds worth of funding. the data was pulled together from a range of eu institutions and government bodies to create an interactive map that allows the user to find details of funding provided in their local area. but researchers stress the projects listed are not completely exhaustive and there could be many more to be included. scroll through the map or enter in your postcode: the team started working on myeu.uk at a hackathon event in july, which was organised by tech for uk, in partnership with campaign group best for britain. included in the map are details of the £4,524,999 the eu provided swansea council to fund 36\% of the regeneration of the kingsway into a city park . there is also a breakdown of farming subsidies provided in different areas of wales. how much money does wales get from the european union? according to the welsh government, wales currently receives around £680m of eu funds every year. this money comes in three broad categories: european structural funds: this covers everything from supporting people into work and training to urban development. common agricultural policy: a £200m a year scheme providing payments to more than 16,000 farms in wales "to help protect and enhance the countryside". one pillar of this is the welsh government rural communities rural development programme, a £957m programme running from 2014-20 supporting businesses and farms in rural areas. other funds: there are several other funds covering everything from the arts to biodiversity. advocates of remaining in the eu also point to the economic benefits that accrue from being able to export without tariffs and barriers (a welsh government white paper points out that welsh international exports are dominated by a small number of large exporters). what is the uk's contribution to the eu budget and what does it currently get back? according to research from the office for national statistics (ons), the uk's gross contribution to the eu in 2016 amounted to £19bn. however, a large proportion of that money was never actually transferred to the eu. before the uk government transfers money to the eu, a rebate is applied. in 2016, this rebate amounted to £5bn, meaning £13.9bn was transferred from the government to the eu. but this only accounts for the money that the uk pays to the eu. some of this £13.9bn is credited back to the uk public sector, of which a proportion is then paid to the private sector. the ons reports that £4.4bn came back to the uk public sector and private sector in credits in 2016. this included £359m that came back through the european regional development fund and £2.4bn through the agricultural guarantee fund. given these figures, ons reported that the uk government's net contribution to the eu - that is the difference between the money it paid to the eu and the money it received - was £9.4bn in 2016 as compared with the £18.9bn gross contribution. uk contributions to the eu budget from 2010 to 2016: what do the myeu map developers say? eloise todd, best for britain chief executive, said: "the eu invests around £5bn a year in the uk and with myeu.uk every community can see how they benefit from this money. "this support means more jobs and greater prosperity in places up and down the uk. "people across the country need to know they have the right to the deal we currently have - we must be able to compare the deal we currently have to the deal the government comes back with, because any kind of brexit would leave us worse off, and a soft brexit deal would leave us with no say at all over the majority of laws that govern our trade. "technology can help us and we'll be running these hackathons over the coming months. "we hope that people will use the app and contact their mp to make sure the people and not politicians have the final say on whether we leave the eu." hope thomas, one of the app's designers, said: "the number and range of things the eu has funded around europe really surprised us when we started to look at it, so we wanted to share it with other people. "the reaction to the site has shown us how many people are affected by the work the eu has been doing." & 850 & very high & High & Socio-Economic & Power & NA & 2018-09-14 & 2018 & 3 & ECO
Frame & high-very high & Regional & 500-1000 & -0.7997063 & -1.2431481 & 1.1980777 & 1.4613788 & -0.9363498 & 1.1 & -0.8854249 & -0.3297660 & Payer & Domestic & European & Mixed & Domestic|ECO & Positive\\
\addlinespace
Greece & http://e-thessalia.gr/thessalia-11-ekat-evro-gia-ekpedeftiki-ypostirixi-ke-tin-entaxi-mathiton-anapiria/ & 309 & e-thessalia.gr & Private/Non-Public & Online only & Regional/Local & low = CP mentioned more times but NOT important part of story (mainly about others issues) & Social justice & Positive & EU + Subnational & No myth & NA & NA & NA & NA & NA & NA & NA & NA & Greece & θεσσαλία: 1,1 εκατ. ευρώ για εκπαιδευτική υποστήριξη και την ένταξη μαθητών με αναπηρία - e-thessalia.gr & 2016-07-26 & ευρωπαϊκό κοινωνικό ταμείο & εκδόθηκε με απόφαση του περιφερειάρχη θεσσαλίας κ. κώστα αγοραστού, μια νέα πρόσκληση του πεπ θεσσαλίας 2014-2020 που αφορά δράση που συγχρηματοδοτείται από το ευρωπαϊκό κοινωνικό ταμείο (ε.κ.τ.) για την υποστήριξη ευάλωτων ομάδων πολιτών. ειδικότερα, προκηρύχθηκε η δράση "εξειδικευμένη εκπαιδευτική υποστήριξη για την ένταξη μαθητών με αναπηρία ή/και ειδικές εκπαιδευτικές ανάγκες για τα έτη 2016-2017 και 2017-2018", συνολικού προϋπολογισμού 1.124.000,00 €. η δράση αφορά σε υλοποίηση της εξειδικευμένης εκπαιδευτικής υποστήριξης για την ένταξη μαθητών με αναπηρία ή και ειδικές εκπαιδευτικές ανάγκες που φοιτούν σε σχολικές μονάδες της πρωτοβάθμιας και δευτεροβάθμιας εκπαίδευσης της περιφέρειας θεσσαλίας για τα σχολικά έτη 2016-2017 και 2017-2018 από ειδικό βοηθητικό και ειδικό εκπαιδευτικό προσωπικό. στόχος της είναι η ενσωμάτωση και η εξασφάλιση της προσβασιμότητας όλων των μαθητών στο κοινωνικό γίγνεσθαι με αποτέλεσμα την αντιμετώπιση του κοινωνικού αποκλεισμού, την αντιμετώπιση της σχολικής αποτυχίας και της πρόωρης εγκατάλειψης του σχολείου. υπενθυμίζεται ότι αντίστοιχη δράση χρηματοδοτήθηκε από το πεπ θεσσαλίας και για το σχολικό έτος 2015-2016, με ποσό 562.000 ευρώ και για 46 ωφελούμενα άτομα. ο κ. κ. αγοραστός τόνισε τα εξής: "παρά τις μεγάλες δυσκολίες που αντιμετωπίζει η χώρα, είναι ανάγκη να δοθεί ιδιαίτερο βάρος στη στήριξη των ευάλωτων κοινωνικών ομάδων με στόχο την όσο το δυνατόν καλύτερη ενσωμάτωσή τους στην κοινωνία. από πλευράς της αιρετής περιφέρειας η αναβάθμιση του συστήματος κοινωνικής προστασίας αποτελεί κεντρική επιλογή. μέσω του νέου εσπα 2014 - 2020, στέλνουμε ως περιφέρεια θεσσαλίας ένα ξεκάθαρο και έμπρακτο μήνυμα στήριξης σε ανθρώπους που έχουν ανάγκη". το ειδικό βοηθητικό προσωπικό (εβπ) προσλαμβάνεται για θέματα αυτοεξυπηρέτησης, καθημερινής διαβίωσης και λειτουργικών διευκολύνσεων των μαθητών με αναπηρία ή και με ειδικές εκπαιδευτικές ανάγκες που φοιτούν σε σχολικές μονάδες της πρωτοβάθμιας και δευτεροβάθμιας εκπαίδευσης της περιφέρειας θεσσαλίας για τα σχολικά έτη 2016-2017 και 2017-2018. το ειδικό εκπαιδευτικό προσωπικό (εεπ) αφορά σε σχολικούς νοσηλευτές που θα προσφέρουν τις υπηρεσίες τους σε σχολικές μονάδες, για μαθητές με αναπηρία ή και ειδικές εκπαιδευτικές ανάγκες, όπως μαθητές με χρόνια προβλήματα υγείας που φοιτούν σε σχολικές μονάδες γενικής αγωγής και απαιτείται να έχουν νοσηλευτική υποστήριξη για τη χορήγηση φαρμάκων σε ενέσιμη ή μη μορφή, χρήση συσκευών κ.λπ.) κατά τα σχολικά έτη 2016 - 2017 και 2017 - 2018. η πρόσκληση, απευθύνεται στον "δικαιούχο" φορέα που είναι η ειδική υπηρεσία εφαρμογής εκπαιδευτικών δράσεων του υπουργείου πολιτισμού, παιδείας και θρησκευμάτων η οποία πρόκειται να υποβάλει πρόταση για αξιολόγηση ένταξης του έργου στο πεπ θεσσαλίας, ώστε να ξεκινήσει η υλοποίησή του. με τον προϋπολογισμό του πεπ θεσσαλίας, αναμένεται να ωφεληθούν ανά σχολικό έτος, 46 άτομα. η παρούσα δράση θα λειτουργήσει και για τα επόμενα έτη σε συνέργεια με αντίστοιχη δράση του τομεακού προγράμματος εκτ στο εσπα, στο πλαίσιο του οποίου σχεδιάζεται να υλοποιηθούν οι υποδράσεις : α)παράλληλη στήριξη μαθητών με αναπηρία ή και ειδικές εκπαιδευτικές ανάγκες β) επιμόρφωση εκπαιδευτικών παράλληλης στήριξης προερχόμενους από τη γενική εκπαίδευση. η πρόσκληση με τα συνημμένα έγγραφα είναι αναρτημένη στο δικτυακό τόπο της ειδικής υπηρεσίας διαχείρισης ε.π. περιφέρειας θεσσαλίας www.thessalia-espa.gr. & 492 & low & Low & Socio-Economic & NA & NA & 2016-07-26 & 2016 & 2 & ECO
Frame & low-medium & Regional & <500 & 1.8924259 & 1.3590316 & -1.2391092 & 0.9814602 & -0.7922579 & 0.0 & -0.9049211 & -1.0736569 & Recipient & Domestic & European & Mixed & Domestic|ECO & Positive\\
Greece & http://www.newsbeast.gr/greece/arthro/2591507/katartisi-6-339-ofeloumenon-apo-tin-kinofeli-ergasia & 339 & Newsbeast.gr & Private/Non-Public & Online only & National & medium = CP is important part of story & Jobs & Factual & EU + Subnational & No myth & Research \& innovation & Factual & EU + Subnational & No myth & NA & NA & NA & NA & Greece & κατάρτιση 6.339 ωφελούμενων από την κοινωφελή εργασία & 2017-02-23 & ευρωπαϊκό κοινωνικό ταμείο & τη κατάρτιση 6.339 ωφελούμενων περιλαμβάνει η δεύτερη φάση του προγράμματος κοινωφελούς εργασίας που υλοποιείται σε 34 δήμους. η δράση χρηματοδοτείται από το ε.π. "ανάπτυξη ανθρώπινου δυναμικού, εκπαίδευση \& δια βίου μάθηση", με πόρους της ελλάδας και της ευρωπαϊκής ένωσης (ευρωπαϊκό κοινωνικό ταμείο), με συνολικό προϋπολογισμό 5.616.646 ευρώ. αντικείμενο της δράσης, όπως γράφει το dimosio.gr αποτελεί η παροχή προγραμμάτων κατάρτισης στην τεχνολογία πληροφοριών και επικοινωνίας (τπε) και στην κοινωνική οικονομία και επιχειρηματικότητα και η πιστοποίηση των γνώσεων και δεξιοτήτων τπε που θα αποκτηθούν, στους ωφελούμενους συμμετέχοντες στο πρόγραμμα κοινωφελούς χαρακτήρα σε 34 δήμους - θύλακες ανεργίας οι οποίοι έχουν τοποθετηθεί σε θέσεις απασχόλησης και έχουν επιλέξει να παρακολουθήσουν τα προγράμματα κατάρτισης. σκοπός της δράσης είναι η ενδυνάμωση και αναβάθμιση των προσόντων και των δεξιοτήτων των συμμετεχόντων μέσω συνδυασμένων δράσεων κατάρτισης και πιστοποίησης, και η διευκόλυνση της ένταξης/επανένταξής τους στην αγορά εργασίας. πιο συγκεκριμένα, στο πλαίσιο της δράσης θα υλοποιηθεί μία δέσμη παρεμβάσεων η οποία θα περιλαμβάνει: οι ωφελούμενοι της δράσης, δικαιούνται μια προσωπική επιταγή κατάρτισης, με την οποία αποκτούν τη δυνατότητα να καταρτιστούν στη χρήση τεχνολογιών πληροφορικής και επικοινωνίας και κοινωνικής οικονομίας και επιχειρηματικότητας και να συμμετάσχουν σε εξετάσεις πιστοποίησης σε πάροχο κατάρτισης της επιλογής τους, από αυτούς που περιλαμβάνονται στο μητρώο παρόχων. για την πιστοποίηση των γνώσεων και δεξιοτήτων πληροφορικής που θα αποκτηθούν μέσω της κατάρτισης οι πάροχοι θα συμβληθούν με αναγνωρισμένους φορείς πιστοποίησης. ο ωφελούμενος θα λαμβάνει τις υπηρεσίες κατάρτισης μία ημέρα την εβδομάδα, ενώ τις υπόλοιπες ημέρες θα εργάζεται στη θέση απασχόλησης στην οποία έχει επιτύχει και τοποθετηθεί, καθώς η δράση της κατάρτισης αποτελεί μέρος της ευρύτερης δράσης απασχόλησης στο πλαίσιο των προγραμμάτων κοινωφελούς χαρακτήρα. η δράση θα πραγματοποιηθεί στις περιφερειακές ενότητες των 34 δήμων που συμμετέχουν στο πρόγραμμα και θα πρέπει να έχει ολοκληρωθεί μέσα στο 8μηνο απασχόλησης του ωφελούμενου. η ενημέρωση των ενδιαφερομένων για τη δράση αλλά και για θέματα που θα προκύψουν κατά την υλοποίησή της, θα πραγματοποιείται, σύμφωνα με τους όρους της παρούσας, μέσα από την ειδική ιστοσελίδα http://www.voucher.gov.gr. & 333 & medium & Medium & Socio-Economic & Socio-Economic & NA & 2017-02-23 & 2017 & 2 & ECO
Frame & low-medium & National & <500 & 1.8924259 & 1.3590316 & -1.2391092 & 0.9814602 & -0.7922579 & 0.0 & -0.9049211 & -1.0736569 & Recipient & Domestic & European & Mixed & Domestic|ECO & Neutral\\
Greece & http://www.rodiaki.gr/article/378251/entos-toy-2018-oloklhrwnetai-to-neo-geniko-nosokomeio-karpathoy-poy-episkefthhke-shmera-o-perifereiarxhs & 389 & Rodiaki.gr & Private/Non-Public & Online and Offline & Regional/Local & medium = CP is important part of story & Infrastructure & Positive & EU + Subnational & No myth & Public services & Positive & EU + Subnational & No myth & NA & NA & NA & NA & Greece & εντός του 2018 ολοκληρώνεται το νέο γενικό νοσοκομείο καρπάθου, που επισκέφθηκε σήμερα ο περιφερειάρχης | η ροδιακη & 2017-11-18 & ευρωπαϊκό ταμείο περιφερειακής ανάπτυξης & το νέο γενικό νοσοκομείο της καρπάθου, το οποίο αναμένεται να ολοκληρωθεί εντός του 2018, επισκέφθηκε σήμερα το μεσημέρι ο περιφερειάρχης νοτίου αιγαίου, γιώργος χατζημάρκος, προκειμένου να διαπιστώσει την πρόοδο των εργασιών. τον περιφερειάρχη, που για τρίτη μέρα σήμερα βρίσκεται στην κάρπαθο, για να επιληφθεί των θεμάτων του νησιού, συνοδεύουν ο έπαρχος καρπάθου - κάσου, γιάννης μηνατσής, ο αντιπεριφερειάρχης πρωτογενούς τομέα και γαστρονομίας, φιλήμονας ζαννετίδης, η αντιπεριφερειάρχης τουρισμού, πολιτισμού και αθλητισμού, μαριέττα παπαβασιλείου, ο πρόεδρος της φυτώριο αε. μιχάλης μπαριανάκης και ο στέργος στάγκας, ειδικός συνεργάτης του γραφείου περιφερειάρχη για τεχνικά θέματα. στην διάρκεια της επίσκεψης διαπιστώθηκε ότι οι εργασίες κατασκευής του νέου νοσοκομείου εξελίσσονται κανονικά και αναμένεται η αποπεράτωσή του εντός του 2018. το δυναμικότητας 22 κλινών, γενικό νοσοκομείο καρπάθου υλοποιείται με πόρους του επιχειρησιακού προγράμματος "νότιο αιγαίο 2014 - 2020" στο οποίο εντάχθηκε, συμπεριλαμβανομένου και του εξοπλισμού του, τον δεκέμβριο του 2015, με απόφαση του περιφερειάρχη, γιώργου χατζημάρκου και με επιλέξιμη δαπάνη 6,1 εκατ. ευρώ, με συγχρηματοδότηση από το ευρωπαϊκό ταμείο περιφερειακής ανάπτυξης, στο πλαίσιο του εσπα 2014 - 2020. το νέο νοσοκομείο θα αντικαταστήσει το κέντρο υγείας του νησιού και θα παρέχει πρωτοβάθμια και δευτεροβάθμια φροντίδα υγείας, βραχεία νοσηλεία και παρηγορητική ιατρική στον πληθυσμό της καρπάθου και των γειτονικών νησιών, που μέχρι τώρα στερούνται δευτεροβάθμιας περίθαλψης. ακολούθησε επίσκεψη του περιφερειάρχη και του κλιμακίου που τον συνοδεύει, στον πυροσβεστικό σταθμό καρπάθου, όπου προ ημερών παραλήφθηκαν τα τρία υπερσύγχρονα πυροσβεστικά οχήματα, τα οποία χρηματοδοτήθηκαν επίσης από πόρους του επιχειρησιακού προγράμματος "νότιο αιγαίο 2014 - 2020", μετά από απόφαση του κ. χατζημάρκου να χρηματοδοτήσει από τους ευρωπαϊκούς πόρους που διαχειρίζεται η περιφέρεια νοτίου αιγαίου, την ενίσχυση και την αναβάθμιση του εξοπλισμού του πυροσβεστικού σώματος με το συνολικό ποσό των 2,9 εκατ. ευρώ. βάσει αυτής της απόφασης, ο πυροσβεστικός σταθμός της καρπάθου ενισχύθηκε με τρία υδροφόρα πυροσβεστικά οχήματα (4χ4), εκτός δρόμου, χωρητικότητας 2.500, 4.000 και 5.000 λίτρων νερού, αντίστοιχα. & 308 & medium & Medium & Socio-Economic & Socio-Economic & NA & 2017-11-18 & 2017 & 2 & ECO
Frame & low-medium & Regional & <500 & 1.8924259 & 1.3590316 & -1.2391092 & 0.9814602 & -0.7922579 & 0.0 & -0.9049211 & -1.0736569 & Recipient & Domestic & European & Mixed & Domestic|ECO & Positive\\
Greece & http://www.newsbeast.gr/politiki/arthro/2155693/i-komision-epitrepi-tin-enischisi-tou-misthologikou-kostous-se-anatoliki-makedonia-ke-thraki & 393 & Newsbeast.gr & Private/Non-Public & Online only & National & low = CP mentioned more times but NOT important part of story (mainly about others issues) & Economic development & Factual & EU + National & No myth & NA & NA & NA & NA & NA & NA & NA & NA & Greece & η κομισιόν επιτρέπει την ενίσχυση του μισθολογικού κόστους σε ανατολική μακεδονία και θράκη & 2016-03-02 & διαρθρωτικά ταμεία & τι απάντησε η επίτροπος ανταγωνισμού μ. βεστάγκερ στη μαρία σπυράκη η ευρωπαϊκή ένωση δεν απαγορεύει στις εθνικές κυβερνήσεις να χορηγούν ενίσχυση με εθνικούς πόρους σε επιχειρήσεις που δραστηριοποιούνται σε οικονομικά μειονεκτούσες περιοχές ακόμη και αν έχουν τη μορφή της επιδότησης μισθών, εφόσον αυτές πληρούν τις προϋποθέσεις του κοινοτικού κανονισμού απαλλαγής. h κ. σπυράκη σε ερώτηση της προς την ευρωπαϊκή επιτροπή διευκρίνισε ότι η ενίσχυση επιχειρήσεων σε ανατολική μακεδονία και θράκη παρέχεται μέσω επιδότησης ποσοστού του μισθολογικού κόστους από τον οαεδ, με στόχο την άρση αντικειμενικών ανισοτήτων και την ενθάρρυνση της δημιουργίας και διατήρησης θέσεων εργασίας καθώς η έδρα των συγκεκριμένων επιχειρήσεων βρίσκεται σε μειονεκτούσες περιοχές. η ευρωβουλευτής επεσήμανε τον κίνδυνο οι καθυστερήσεις στην καταβολή της ενίσχυσης από το ελληνικό κράτος να οδηγήσουν σε παραγραφή της οφειλής προς τις επιχειρήσεις που δικαιούνται την συγκεκριμένη ενίσχυση. η ευρωπαϊκή επιτροπή από την πλευρά της απαντά ότι: "το μισθολογικό κόστος μπορεί να επιδοτηθεί για έργα που υλοποιούνται σε αυτές τις περιοχές, εφόσον συνδέεται με αρχικές επενδύσεις. επιπλέον, υπάρχουν δυνατότητες για χορήγηση ενισχύσεων στις μμε (συμπεριλαμβανομένων των δαπανών μισθοδοσίας), καθώς και για την απασχόληση εργαζομένων σε μειονεκτική θέση. τα εν λόγω μέτρα ενίσχυσης μπορούν να εφαρμοστούν από το κράτος μέλος χωρίς προηγούμενη κοινοποίηση στην επιτροπή, υπό τον όρο ότι πληρούνται όλες οι προϋποθέσεις που προβλέπονται στον γενικό κανονισμό απαλλαγής κατά κατηγορία, αριθ. 651/2014" και καταλήγει ότι: " εάν ένα μέτρο κρατικής ενίσχυσης δεν πληροί όλες τις προϋποθέσεις του εν λόγω κανονισμού, μπορεί μολοντούτο να εγκριθεί ως συμβατή ενίσχυση από την επιτροπή, κατόπιν κοινοποίησης του μέτρου από το ενδιαφερόμενο κράτος μέλος και με την προϋπόθεση ότι πληρούνται οι σχετικές προϋποθέσεις συμβατότητας. σε κάθε περίπτωση, η απόφαση για το αν πράγματι χορηγηθεί κρατική ενίσχυση στο πλαίσιο των κανόνων αυτών, εναπόκειται, φυσικά, σε κάθε κράτος μέλος." ερώτηση: "το ελληνικό κράτος χορηγεί με εθνικούς πόρους ενίσχυση προς επιχειρήσεις που δραστηριοποιούνται σε οικονομικά μειονεκτούσες περιοχές της χώρας. η ενίσχυση παρέχεται μέσω επιδότησης ποσοστού του μισθολογικού κόστους από τον οργανισμό απασχόλησης εργατικού δυναμικού. στόχος της ενίσχυσης είναι να άρει αντικειμενικές ανισότητες που στρεβλώνουν τον ανταγωνισμό και ενθαρρύνουν τη δημιουργία και διατήρηση θέσεων εργασίας σε μειονεκτούσες περιοχές. σύμφωνα με τον κανονισμό (εκ) αριθ. 800/2008 της επιτροπής, οι εθνικές ενισχύσεις περιφερειακού χαρακτήρα προωθούν την οικονομική, κοινωνική και εδαφική συνοχή των κρατών μελών και της κοινότητας στο σύνολό της. οι εθνικές ενισχύσεις περιφερειακού χαρακτήρα αποσκοπούν στην υποβοήθηση της ανάπτυξης των πλέον μειονεκτουσών περιοχών με την υποστήριξη των επενδύσεων και της δημιουργίας θέσεων εργασίας σε ένα βιώσιμο πλαίσιο. ωστόσο, από τo 2010 η ενίσχυση καθυστερεί να χορηγηθεί στους δικαιούχους με κίνδυνο παραγραφής της οφειλής, ενώ πρόσφατα ανακοινώθηκε η ακύρωσή της από τον υπουργό οικονομίας με το επιχείρημα ότι υπάρχουν εμπόδια από την πλευρά της ευρωπαϊκής ένωσης. ερωτάται η επιτροπή: 1. υπάρχει συγκεκριμένη ευρωπαϊκή νομοθεσία ή σχετική απόφαση που υποχρεώνει την ελλάδα να άρει τη χορήγηση της παραπάνω εθνικής ενίσχυσης περιφερειακού χαρακτήρα; 2. υπάρχει η δυνατότητα ενίσχυσης των οικονομικά ευαίσθητων περιοχών μέσα από την επιδότηση του μισθολογικού κόστους μέσα από τα διαρθρωτικά ταμεία για την περίοδο 2014-2020;" απάντηση: "βάσει των διαθέσιμων πληροφοριών, δεν είναι δυνατό να εντοπιστεί το μέτρο στο οποίο αναφέρεται η κα βουλευτής. ως εκ τούτου, η επιτροπή δεν μπορεί να σχολιάσει τα ιδιαίτερα χαρακτηριστικά του εν λόγω μέτρου. για στήριξη που εμπίπτει στο πεδίο της εφαρμογής των κανόνων περί κρατικών ενισχύσεων, επιτρέπεται η χορήγηση ενισχύσεων περιφερειακού χαρακτήρα για αρχικές επενδύσεις σε περιοχές που έχουν αναγνωριστεί ως μειονεκτούσες. το μισθολογικό κόστος μπορεί να επιδοτηθεί για έργα που υλοποιούνται σε αυτές τις περιοχές, εφόσον συνδέεται με αρχικές επενδύσεις. επιπλέον, υπάρχουν δυνατότητες για χορήγηση ενισχύσεων στις μμε (συμπεριλαμβανομένων των δαπανών μισθοδοσίας), καθώς και για την απασχόληση εργαζομένων σε μειονεκτική θέση. τα εν λόγω μέτρα ενίσχυσης μπορούν να εφαρμοστούν από το κράτος μέλος χωρίς προηγούμενη κοινοποίηση στην επιτροπή, υπό τον όρο ότι πληρούνται όλες οι προϋποθέσεις που προβλέπονται στον γενικό κανονισμό απαλλαγής κατά κατηγορία, αριθ. 651/2014. εάν ένα μέτρο κρατικής ενίσχυσης δεν πληροί όλες τις προϋποθέσεις του εν λόγω κανονισμού, μπορεί μολοντούτο να εγκριθεί ως συμβατή ενίσχυση από την επιτροπή, κατόπιν κοινοποίησης του μέτρου από το ενδιαφερόμενο κράτος μέλος και με την προϋπόθεση ότι πληρούνται οι σχετικές προϋποθέσεις συμβατότητας. σε κάθε περίπτωση, η απόφαση για το αν πράγματι χορηγηθεί κρατική ενίσχυση στο πλαίσιο των κανόνων αυτών, εναπόκειται, φυσικά, σε κάθε κράτος μέλος. καμία εταιρεία δεν δικαιούται να λάβει κρατική ενίσχυση, εκτός αν κράτος μέλος αποφασίσει να χορηγήσει ενίσχυση σύμφωνα με τους αντίστοιχους κανόνες της εε". & 721 & low & Low & Socio-Economic & NA & NA & 2016-03-02 & 2016 & 2 & ECO
Frame & low-medium & National & 500-1000 & 1.8924259 & 1.3590316 & -1.2391092 & 0.9814602 & -0.7922579 & 0.0 & -0.9049211 & -1.0736569 & Recipient & Domestic & European & Mixed & Domestic|ECO & Neutral\\
Greece & http://www.tovima.gr/politics/article/?aid=985836 & 349 & TO BHMA International & Private/Non-Public & Online and Offline & National & very low = CP mentioned once & Institutional bargaining over funding & Positive & EU + Other country & No myth & NA & NA & NA & NA & NA & NA & NA & NA & Greece & tovima.gr - η αντιπρόταση μέρκελ σε αυτή του μακρόν για την ευρωζώνη & 2018-06-03 & διαρθρωτικά ταμεία & η καγκελάριος άνγκελα μέρκελ για πρώτη φορά τοποθετήθηκε συνολικά στο θέμα της μεταρρύθμιση της ευρωζώνης σε συνέντευξή της στην κυριακάτικη έκδοση της "frankfurter allgemeine" (fas) και απαντά έτσι στον γάλλο πρόεδρο εμμανουέλ μακρόν, ο οποίος τάσσσεται υπέρ της ενίσχυσης της ευρωζώνης. η κ. μέρκελ τάσσεται υπέρ της σταδιακής εισαγωγής ένας επενδυτικού προϋπολογισμού στην ζώνη του ευρώ. αυτός θα πρέπει αρχικά να είναι "στην κατώτερη διψήφια κλίμακα δισεκατομμυρίων ευρώ" και στη συνέχεια να αξιολογείται η επίδρασή του. άφησε ανοικτό αν αυτός ο προϋπολογισμός θα πρέπει να είναι μέρος του τακτικού προϋπολογισμού της εε, ή - σύμφωνα με τις ιδέες μακρόν- αν θα πρέπει να βρίσκεται στην αρμοδιότητα των υπουργών οικονομικών της ζώνης του ευρώ. επίσης, η καγκελάριος παρουσίασε την άποψη της για το ευρωπαϊκό νομισματικό ταμείο, το οποίο θα πρέπει να προκύψει από τον δημιουργηθέντα κατά τη διάρκεια της κρίσης ευρωπαϊκό μηχανισμό στήριξης (εμς). μεταξύ άλλων, πρότεινε οι χώρες που αντιμετωπίζουν δυσκολίες εξαιτίας εξωτερικών περιστάσεων, να βοηθούνται με βραχυπρόθεσμα δάνεια. "πάντα υπό τον όρο, φυσικά, ότι θα είναι περιορισμένου ύψους και θα αποπληρωθούν πλήρως", όπως τόνισε. η γερμανίδα καγκελάριος μέρκελ τάχθηκε επίσης υπέρ της εξισορρόπησης των οικονομικών διαφορών στην ευρωζώνη μέσω ενός τέτοιου προϋπολογισμού. ένας τέτοιος επενδυτικός προϋπολογισμός είχε ήδη συμπεριληφθεί στην συμφωνία συνασπισμού μεταξύ της χριστιανοδημοκρατικής ένωσης (cdu/csu) και του σοσιαλδημοκρατικού κόμματος της γερμανίας (spd). το ευρωπαϊκό νομισματικό ταμείο θα πρέπει να αξιολογεί την οικονομική κατάσταση σε όλα τα κράτη μέλη με δική του αρμοδιότητα, να αξιολογεί την βιωσιμότητα του χρέους των κρατών μελών και "να διαθέτει τα απαραίτητα εργαλεία, τα οποία εν ανάγκη θα την αποακαθιστούν". "το ταμείο θα πρέπει να είναι οργανωμένο διακρατικά και από κοινού με την ευρωπαϊκή επιτροπή, τους δύο πυλώνες σταθερότητας της ζώνης του ευρώ". η καγκελάριος τάχθηκε υπέρ της ολοκλήρωσης των διαπραγματεύσεων για τον προϋπολογισμό της ε.ε. για την περίοδο 2021-2027 πριν από τις ευρωπαϊκές εκλογές του μαΐου του 2019. "στους σημερινούς αβέβαιους καιρούς, η ευρώπη πρέπει να είναι σε θέση να δράσει ανά πάσα στιγμή", είπε. "αν αναβάλουμε τις συζητήσεις, θα μπορούσε να συμβεί να μην μπορούν να χορηγούνται για έναν ολόκληρο χρόνο υποτροφίες erasmus ή να καθυστερήσει η αναδιοργάνωση της frontex ή η υλοποίηση σημαντικών σχεδίων για την καταπολέμηση των λόγων της φυγής (των προσφύγων), για να μην αναφέρουμε τα διαρθρωτικά ταμεία και τα σημαντικά ερευνητικά προγράμματα", πρόσθεσε η κ. μέρκελ στη συνέντευξή της στην fas. επίσης, η κ. μέρκελ συμφωνεί με τον πρόεδρο μακρόν για μια στενότερη ευρωπαϊκή συνεργασία στην αμυντική πολτική: "διάκειμαι θετικά στην πρόταση του προέδρου mακρόν για μια πρωτοβουλία επεμβασης", είπε. ωστόσο, μια τέτοια δύναμη επέμβασης με μια κοινή στρατιωτική και στρατηγική κουλτούρα θα πρέπει να ενταχθεί στη δομή της αμυντικής συνεργασίας, πρόσθεσε ενόψει της συμφωνηθείσας στενότερης συνεργασίας των κρατών μελών της εε (pesco). ωστόσο, η καγκελάριος έθεσε τον περιoρισμό ότι ο γερμανικός στρατός (bundeswehr) πρέπει να παραμείνει ένας "κοινοβουλευτικός στρατός", όπως αποκαλείται , έτσι ώστε κάθε στρατιωτική αποστολή στο εξωτερικό θα πρέπει πάντα να εγκρίνεται από το γερμανικό κοινοβούλιο (bundestag) εκ των προτέρων. μια τέτοια πρωτοβουλία παρέμβασης δεν σημαίνει επίσης "ότι θα συμμετέχουμε σε κάθε στρατιωτική αποστολή", πρόσθεσε η καγκελάριος & 504 & very low & Low & Power & NA & NA & 2018-06-03 & 2018 & 3 & POL
Frame & v.low & National & 500-1000 & 1.8924259 & 1.3590316 & -1.2391092 & 0.9814602 & -0.7922579 & 0.0 & -0.9049211 & -1.0736569 & Recipient & European & European & European & European|POL & Positive\\
\addlinespace
Greece & http://e-thessalia.gr/o-alexandros-meikopoulos-gia-tin-pagkosmia-imera-kata-ton-narkotikon/ & 307 & e-thessalia.gr & Private/Non-Public & Online only & Regional/Local & very low = CP mentioned once & Public services & Factual & EU + National & No myth & NA & NA & NA & NA & NA & NA & NA & NA & Greece & ο αλέξανδρος μεϊκόπουλος για την παγκόσμια ημέρα κατά των ναρκωτικών - e-thessalia.gr & 2016-06-25 & ευρωπαϊκό κοινωνικό ταμείο & δήλωση του αλέξανδρου μεϊκόπουλου, βουλευτή μαγνησίας του συριζα, με αφορμή την παγκόσμια ημέρα κατά των ναρκωτικών αναφέρει σε ανακοίνωσή του: "η 26η ιουνίου, παγκόσμια ημέρα κατά των ναρκωτικών και της παράνομης διακίνησής τους, είναι μια μέρα σκέψης, αυτοκριτικής και ανασυγκρότησης, μια μέρα ευαισθητοποίησης της παγκόσμιας κοινής γνώμης για τις επιπτώσεις από τη χρήση των ναρκωτικών ουσιών. οι κοινωνικές αλλαγές, ως απόρροια της οικονομικής κρίσης που βιώνει η χώρα μας, δίνουν μια επιπλέον βαρύτητα στη σημερινή ημέρα, καθώς η ανάγκη να μοιραστούμε τα προβλήματα, να προσφέρουμε και να βοηθήσουμε είναι πιο επιτακτική από ποτέ. στην προσπάθεια αυτή η εμπειρία και η βοήθεια όλων είναι πολύτιμη. η αντιμετώπιση και η πρόληψη της μάστιγας των ναρκωτικών ουσιών αποτελούν μείζον ζήτημα τόσο στην χώρα μας όσο και στην παγκόσμια κοινότητα. η μάχη ενάντια στα ναρκωτικά φυσικά δεν είναι εύκολη υπόθεση. η κυβέρνηση δίνει ιδιαίτερη βαρύτητα στην άσκηση μιας κοινωνικής πολιτικής για την αντιμετώπιση των ναρκωτικών βασισμένη πρωτίστως στον άνθρωπο. αντιστρέφει την τάση αποδόμησης και συρρίκνωσης του κοινωνικού κράτους κατά τα προηγούμενα χρόνια, εγκαινιάζει μια νέα, ανοιχτή και συμμετοχική, προσέγγιση, και δρομολογεί δέσμη ενεργειών για την ενίσχυση των δομών που ασχολούνται με τις εξαρτήσεις, στοχεύοντας στην βελτίωση της πρόσβασης των εξαρτημένων στις υπηρεσίες, την διασφάλιση της συνέχειας στη φροντίδα των εξαρτημένων και την συνέργεια των αρμόδιων φορέων με βέλτιστη αξιοποίηση των διαθέσιμων πόρων. στο πλαίσιο αυτό της ενίσχυσης των αρμόδιων δομών, ο προϋπολογισμός του οκανα αυξήθηκε κατά 5 εκατομμύρια ευρώ (από 21 σε 26), ενώ γίνονται συζητήσεις για τη διασφάλιση 25 εκατομμυρίων ευρώ από το ευρωπαϊκό κοινωνικό ταμείο προκειμένου να δημιουργηθούν υπηρεσίες πρόληψης και αντιμετώπισης των εξαρτήσεων. τέλος, εγκρίθηκαν ήδη, μετά από 10 χρόνια, προσλήψεις 80 ατόμων για τη στελέχωση των δομών απεξάρτησης σε κεθεα, οκανα, 18 άνω-ψνα και ψνθ, ενώ παράλληλα θα πραγματοποιηθούν προσλήψεις ορισμένου χρόνου μέσω των πόρων του εσπα. στις 23 ιουνίου, το κυβερνητικό συμβούλιο κοινωνικής πολιτικής (κυ.σ.κοι.π.) πραγματοποίησε ειδική θεματική συνεδρίαση υπό τον αντιπρόεδρο της κυβέρνησης, γιάννη δραγασάκη, με θέμα "δημόσιες πολιτικές αντιμετώπισης των εξαρτήσεων". η σύνθεσή του διευρυμένη με τη συμμετοχή του υπουργού δικαιοσύνης και θεσμοθετημένων φορέων που δραστηριοποιούνται στον τομέα των εξαρτήσεων, οι οποίοι παρουσίασαν τα σχέδια δράσης τους. το κυ.σ.κοι.π. εξουσιοδότησε τους υπουργούς υγείας και δικαιοσύνης να καταθέσουν εντός του ιουλίου εισήγηση για τις αναγκαίες θεσμικές αλλαγές στο νόμο περί εξαρτησιογόνων ουσιών (ν.4139/13) όσο και για τη διαδικασία εκπόνησης εθνικού σχεδίου δράσης για την αντιμετώπιση των εξαρτήσεων, με σκοπό τον καλύτερο συντονισμό της εθνικής πολιτικής και την αποτελεσματικότερη λειτουργία των αρμόδιων οργάνων. επιπλέον, εισηγήθηκε τη συμπερίληψη των φορέων που δραστηριοποιούνται για την αντιμετώπιση των εξαρτήσεων στο μεσοπρόθεσμο πλαίσιο δημοσιονομικής στρατηγικής τριετούς διάρκειας πλάνο χρηματοδότησης και ανέθεσε στα συναρμόδια υπουργεία τον εντοπισμό και την εφαρμογή συνεργειών με άλλες δράσεις κοινωνικής πολιτικής που ήδη υλοποιούνται όπως τα κέντρα κοινότητας, την κοινωνική οικονομία, τα σχολεία δεύτερης ευκαιρίας, για την κοινωνική υποστήριξη και την επανένταξη των εξαρτημένων ατόμων. οι παρεμβάσεις για την καταπολέμηση αλλά και την πρόληψη των εξαρτήσεων είναι συνεχείς, ο αγώνας όλων μας ενάντια στις εξαρτήσεις πρέπει να είναι καθημερινός. η πρόληψη είναι τρόπος ζωής. είμαστε δίπλα στους χρήστες ουσιών και στις οικογένειές τους και μαχόμαστε μαζί με τους νέους μας για ένα καλύτερο αύριο. γιατί η εξάρτηση δεν είναι ανίκητη". & 530 & very low & Low & Socio-Economic & NA & NA & 2016-06-25 & 2016 & 2 & ECO
Frame & v.low & Regional & 500-1000 & 1.8924259 & 1.3590316 & -1.2391092 & 0.9814602 & -0.7922579 & 0.0 & -0.9049211 & -1.0736569 & Recipient & Domestic & European & Mixed & Domestic|ECO & Neutral\\
Greece & http://news247.gr/eidiseis/oikonomia/ergasia/evep-me-taxutatoys-rythmous-h-katartish-2-500-epapeiloumenwn-me-anergia-ergazomenwn.3797179.html & 313 & news247.gr & Private/Non-Public & Online only & National & low = CP mentioned more times but NOT important part of story (mainly about others issues) & Jobs & Positive & National & No myth & NA & NA & NA & NA & NA & NA & NA & NA & Greece & εβεπ: με ταχύτατους ρυθμούς η κατάρτιση 2.500 επαπειλούμενων με ανεργία εργαζομένων - news247.gr & 2015-11-30 & ευρωπαϊκό κοινωνικό ταμείο & στους ωφελούμενους που πληρούσαν τα κριτήρια, εκταμιεύθηκε το ποσό των 4.235.000 ευρώ, 1.750 ευρώ κατά άτομο, ήτοι το 36\% της χρηματοδότησης ημερίδα, που αφορά την υλοποίηση του προγράμματος για την κατάρτιση 2.500 επαπειλούμενων με ανεργία εργαζομένων και αυτοαπασχολουμένων στους βιομηχανικούς κλάδους της ευρύτερης περιοχής του πειραιά και του θριασίου πεδίου, πραγματοποιήθηκε απόψε στο εμπορικό και βιομηχανικό επιμελητήριο πειραιά (εβεπ), παρουσία του προέδρου της νέας δημοκρατίας γιάννη πλακιωτάκη και του υπουργού ναυτιλίας και νησιωτικής πολιτικής θοδωρή δρίτσα. πρόκειται για την παρουσίαση της πράξης του εσπα 2007 - 2013, που υλοποίησε το εβεπ σε πολύ γρήγορους ρυθμούς και με την οποία δόθηκε η δυνατότητα σε επαπειλούμενους με ανεργία εργαζόμενους και αυτοαπασχολούμενους να λάβουν πιστοποίηση και συμβουλευτική υποστήριξη σε 11 ειδικότητες, με κοινοτική χρηματοδότηση από το ευρωπαϊκό κοινωνικό ταμείο. σε 2.420 ωφελούμενους που πληρούσαν τα κριτήρια, εκταμιεύθηκε το ποσό των 4.235.000 ευρώ, 1.750 ευρώ κατά άτομο, ήτοι το 36\% της χρηματοδότησης. η κατάρτιση έγινε με 107 τμήματα των 20 με 25 ατόμων κάθε τμήμα και τα μαθήματα ξεκίνησαν το μάρτιο του 2015 και ολοκληρώθηκαν στις 07/08/2015. στα 107 τμήματα κατάρτισης, 27 είχαν εκπαιδευτικό αντικείμενο συγκολλήσεις, 9 αυτοματισμούς, σε 19 εκπαιδεύτηκαν ελασματουργοί, σε 3 μονωτές, σε 6 ξυλουργοί, σε 5 χειριστές εργαλειομηχανών, σε 5 αμμοβολιστές, σε 10 σωληνουργοί, σε 5 ψυκτικοί, σε 3 τεχνίτες ικριωμάτων και σε 15 χειριστές κινητών μηχανημάτων έργου. η πρακτική πραγματοποιήθηκε στις 12 ακόλουθες εταιρείες ολπ, ναυπηγεία ελευσίνας, durostic αβεε, ανδρέας φίλης αεβε, ασφαλτερ αε, σταύρος κασιδιάρης, ζιτακατ ατεβε, κοραλ επε, ενωμένη ψυκτική αε, νικοπαλ αε, λάζαρης επε, μμ ναβεπ αε. "οι καταρτιζόμενοι αυτοί, είχαν τόσο πολύ υψηλό επίπεδο ικανοτήτων που μπορεί να ήταν και εκπαιδευτές αντί για εκπαιδευόμενοι, αλλά τους δώσαμε την ευκαιρία για πρώτη φορά να μπορέσουν να πιστοποιήσουν τις εξαιρετικές ικανότητες που διαθέτουν, αλλά και να διατηρήσουν ή να βρουν εργασία" τόνισε στην ομιλία του ο πρόεδρος του εβεπ βασίλης κορκίδης. πρόσθεσε ότι η πιστοποίηση με σύγχρονα πρότυπα των τεχνικών ικανοτήτων αποτελούν μια προστιθέμενη αξία για τον κόσμο της εργασίας και της επιχειρηματικότητας. ο βασίλης κορκίδης επεσήμανε επίσης, ότι το εβεπ ολοκλήρωσε το πρόγραμμα με ταχείς ρυθμούς, καθώς τα χρήματα ήταν έτοιμα να επιστραφούν πίσω στην ευρωπαϊκή ένωση. "καταφέραμε" πρόσθεσε "να μη χαθεί ούτε ένα ευρώ". από την πλευρά του ο πρόεδρος της νδ γιάννης πλακιωτάκης τόνισε ότι πρόκειται για ένα πρόγραμμα που επεξεργάστηκε ο ίδιος πριν από ένα χρόνο ως υφυπουργός εργασίας, ενώ χαρακτήρισε άκρως επιτυχημένες τις προσπάθειες της ομάδας του βασίλη κορκίδη και του εβεπ για την υλοποίηση του και συμπλήρωσε πως η κυβέρνηση του συριζα αποδεικνύεται ότι αδυνατεί να αξιοποιήσει και τους διασφαλισμένους πόρους του εσπα που κερδίσαμε με σκληρή διαπραγμάτευση, ώστε να διατεθούν εμπροσθοβαρώς (πάνω από 750 εκατ. ευρώ από το εσπα 2014 - 2020). "η εποχή των επιδομάτων πρέπει να τελειώσει" δήλωσε ο γιάννης πλακιωτάκης και τόνισε ότι η νδ σε πολύ σύντομο χρόνο κατάφερε να εισάγει μια νέα "κουλτούρα" στα προγράμματα απασχόλησης που δημιουργούν συνθήκες μείωσης της ανεργίας και όχι παραγωγή "επαγγελματιών" ανέργων, δίνοντας έμφαση στην πιστοποίηση των δεξιοτήτων και των επαγγελματικών προσόντων. ο υπουργός ναυτιλίας και νησιωτικής πολιτικής θοδωρής δρίτσας συνεχάρη το εβεπ και τον πρόεδρό του για την υλοποίηση του σχετικού προγράμματος και είπε ότι όταν ανέλαβε την ηγεσία του υπουργείου κατάφερε να διασώσει δύο με τρία προγράμματα για τη ναυτική εκπαίδευση. & 541 & low & Low & Socio-Economic & NA & NA & 2015-11-30 & 2015 & 1 & ECO
Frame & low-medium & National & 500-1000 & 1.8924259 & 1.3590316 & -1.2391092 & 0.9814602 & -0.7922579 & 0.0 & -0.9049211 & -1.0736569 & Recipient & Domestic & Domestic & Domestic & Domestic|ECO & Positive\\
Greece & https://e-thessalia.gr/prosvasimes-paralies-se-amea-apokta-i-alonnisos/ & 352 & e-thessalia.gr & Private/Non-Public & Online only & Regional/Local & very low = CP mentioned once & Social justice & Positive & Subnational & No myth & Infrastructure & Positive & Subnational & No myth & Environment/green/low-carbon & Positive & Subnational & No myth & Greece & προσβάσιμες παραλίες σε αμεα αποκτά η αλόννησος - e-thessalia.gr & 2018-08-07 & ευρωπαϊκό ταμείο περιφερειακής ανάπτυξης & νέο σημείο αναφοράς στις συνθήκες διαβίωσης των κατοίκων και των τουριστών της αλοννήσου, δημιουργεί η πρόσφατη ένταξη του δήμου αλοννήσου σε πρόγραμμα διαμόρφωσης δύο παραλιών ώστε να είναι προσβάσιμες σε αμεα, μέσα από την πράξη "δημιουργία ολοκληρωμένων τουριστικών προσβάσιμων θαλασσίων προορισμών στις παραλίες μηλιά και γλύφα του δήμου αλοννήσου" στον άξονα προτεραιότητας "ανάπτυξη μηχανισμών στήριξης της επιχειρηματικότητας" του επιχειρησιακού προγράμματος "ανταγωνιστικότητα, επιχειρηματικότητα και καινοτομία" που συγχρηματοδοτείται από το ευρωπαϊκό ταμείο περιφερειακής ανάπτυξης (ετπα). ο δήμος αλοννήσου διεκδίκησε και διασφάλισε τη συμμετοχή του στο πρωτοποριακό πρόγραμμα με προϋπολογισμό και συνολική δημόσια δαπάνη ύψους 119.536,00 ευρώ, του οποίου, πέραν της καταλληλότητας των παραλιών ή άλλων παραγόντων, κριτήριο επιλογής ήταν και η σειρά προτεραιότητας των αιτήσεων όπως προέκυπτε από την ημερομηνία υποβολής φακέλου με αριθμό πρωτοκόλλου για τη χρηματοδότηση του έργου που αφορά σειρά εξοπλισμού, εγκαταστάσεων και δράσεων, όπως περιγράφονται παρακάτω για τις δύο παραλίες του νησιού: (α) δύο μη μόνιμες συναρμολογούμενες διατάξεις για την αυτόνομη πρόσβαση αμεα στη θάλασσα (μία διάταξη για κάθε προτεινόμενη παραλία). (β) δημιουργία δύο (τουλάχιστον) χώρων στάθμευσης αμεα (ένας χώρος για κάθε παραλία) με τις απαιτούμενες διαστάσεις. (γ) ξύλινοι διάδρομοι κατάλληλου μήκους, οι οποίοι θα ενώνουν τις υποδομές μεταξύ τους και θα έχουν το απαραίτητο πλάτος ώστε να εξασφαλίζεται η ανεμπόδιστη μετακίνηση των χρηστών αναπηρικών αμαξιδίων και θα αποτελούνται από λυόμενα τμήματα με μέγιστο μήκος 2,50μ. για την εύκολη μεταφορά και αποθήκευσή τους. (δ) δύο μη μόνιμα αποδυτήρια αμεα (με ντουζιέρες), ένα για καθεμιά από τις δύο παραλίες. (ε) δύο χώρους σκίασης (ένας για κάθε παραλία) με εμβαδόν τουλάχιστον 6,50τ.μ. ο καθένας. (στ) δύο χημικές/οικολογικές τουαλέτες για αμεα (μία διάταξη για κάθε προτεινόμενη παραλία). (ζ) δύο συστήματα ασφάλειας - συναγερμού (ένα για κάθε προτεινόμενη παραλία). (η) πληροφοριακές πινακίδες και πινακίδες σήμανσης. (θ) ψηφιακή εφαρμογή για την προσβασιμότητα αμεα. (ι) δράσεις δημοσιότητας και ενημέρωσης του κοινού για την πράξη. "με υψηλό αίσθημα κοινωνικής ευθύνης και σεβασμό στις ανάγκες του συνανθρώπου μας, δημιουργούμε τις προϋποθέσεις για βελτίωση των όρων διαβίωσης των ατόμων με ειδικές ανάγκες, ώστε τόσο οι κάτοικοι του όμορφου νησιού μας, όσο και οι τουρίστες να απολαμβάνουν απρόσκοπτα τα δώρα της φύσης", δηλώνει ο δήμαρχος αλοννήσου κ. πέτρος βαφίνης για το σημαντικό αυτό έργο που εντάσσεται στη δέσμη πρωτοποριακών δράσεων και μέτρων του δήμου με στόχο την αναβάθμιση του περιβάλλοντος και του βιοτικού επιπέδου και που περιλαμβάνει μεταξύ άλλων και άλλες ενέργειες που κατά καιρούς έχουν ευοδωθεί, όπως η συρρίκνωση της χρήσης της πλαστικής σακούλας. εμβαθύνοντας, η προέκταση των ωφελειών της συγκεκριμένης δράσης αφορούν και στη δημιουργία προϋποθέσεων για ένα νέο τουριστικό "προϊόν", όπως άλλωστε τονίζεται και στην περιγραφή του αντικειμένου του προγράμματος εσπα που καταλήγει πως στόχος είναι "η άρση περιθωριοποίησης των αμεα αλλά και η ενίσχυση των επισκέψεων σε περιοχές που αποτελούν πόλο έλξης στο δήμο αλοννήσου". σύμφωνα μάλιστα και με το σύνδεσμο βρετανών ταξιδιωτικών πρακτόρων, η αγορά του προσβάσιμου τουρισμού αναπτύσσεται ραγδαίως, καθώς εκτιμάται πως μόνο στην μεγάλη βρετανία, 11 εκατομμύρια βρετανοί με κινησιακά πρoβλήματα και αναπηρίες ταξιδεύουν με υψηλή συχνότητα και διαθέτουν αγοραστική δύναμη, άνω των 250 δισεκατομμυρίων λιρών. & 500 & very low & Low & Socio-Economic & Socio-Economic & Socio-Economic & 2018-08-07 & 2018 & 3 & ECO
Frame & v.low & Regional & <500 & 1.8924259 & 1.3590316 & -1.2391092 & 0.9814602 & -0.7922579 & 0.0 & -0.9049211 & -1.0736569 & Recipient & Domestic & Domestic & Domestic & Domestic|ECO & Positive\\
Greece & http://news247.gr/eidiseis/kosmos/news/germania-isxyrh-kai-xwris-antipalo-h-merkel-sth-souper-kyriakh-para-thn-amfisvhthsh-gia-to-prosfygiko.3953279.html & 312 & news247.gr & Private/Non-Public & Online only & National & very low = CP mentioned once & Research \& innovation & Factual & Subnational & No myth & NA & NA & NA & NA & NA & NA & NA & NA & Greece & γερμανία: ισχυρή και χωρίς αντίπαλο η μέρκελ στη σούπερ κυριακή , παρά την αμφισβήτηση για το προσφυγικό & 2016-03-13 & περιφερειακή πολιτική & στις κάλπες 14 εκατομμύρια γερμανοί, εν μέσω αμφιβολιών για τη διαχείριση της προσφυγικής κρίσης από την καγκελάριο μέρκελ. το μεγάλο τεστ πριν τις βουλευτικές εκλογές, σε μία χώρα σε αναβρασμό, μετά το άνοιγμα των πυλών για περισσότερους από ένα εκατομμύριο αιτούντες άσυλο περίπου 13 εκατομμύρια γερμανοί καλούνται την κυριακή στις κάλπες για τρεις περιφερειακές εκλογές σε μια χώρα που ταλανίζεται από τις αμφιβολίες για τη διαχείριση της προσφυγικής ροής, εκλογές που μπορεί να καταβαραθρώσουν το στρατόπεδο της μέρκελ και να αναδείξουν στα ύψη τη λαϊκιστική δεξιά. για τις εκλογές αυτές που αποτελούν ένα τεστ δεκαοκτώ μήνες πριν από τις βουλευτικές εκλογές, οι πρώτες εκτιμήσεις αναμένεται να γίνουν γνωστές απόψε μετά τις 19.00 ώρα ελλάδας και το κλείσιμο των εκλογικών κέντρων στα περιφερειακά κρατίδια της βάδης-βυρτεμβέργης (νοτιοδυτικό), ρηνανίας-παλατινάτου (δυτικό και σαξονίας-άνχαλτ (ανατολικό). για να αποτρέψει μια οδυνηρή αποτυχία της χριστιανοδημοκρατικής ένωσης (cdu), η μέρκελ δεν σταμάτησε να πηγαίνει στα περιφερειακά κρατίδια, πολλαπλασιάζοντας τις εκλογικές συγκεντρώσεις, κυρίως στη βάδη-βυρτεμβέργη, προπύργιο των συντηρητικών όπου όμως το αποτέλεσμα δείχνει αμφίρροπο και στη ρηνανία-παλατινάτο, όπου η cdu δίνει μάχη στήθος με στήθος με τους σοσιαλδημοκράτες (spd). και το σάββατο, η καγκελάριος αφιέρωσε το μεγαλύτερο μέρος της τελευταίας δημόσιας παρέμβασης στην προσφυγική κρίση που απειλεί το βάθρο της για πρώτη φορά στα δέκα χρόνια που είναι στην εξουσία. η μέρκελ επέμεινε στο "δικαίωμα" ένταξης των προσφύγων, επαναλαμβάνοντας ότι οι υπεσχημένες ευρωπαϊκές λύσεις εδώ και μήνες θα μειώσουν τον αριθμό των μεταναστών που αρχίζουν μια επικίνδυνη οδύσσεια για τη βόρεια ευρώπη. επίδοση ρεκόρ; η γερμανία είναι σε αναβρασμό από τότε που άνοιξε τις πύλες της το 2015 σε πάνω από ένα εκατομμύριο αιτούντες άσυλο, κυρίως σύρους, οι οποίοι για να γλιτώσουν από την κόλαση του πολέμου επιβιβάζονται σε φουσκωτές βάρκες προκειμένου να φθάσουν στην ευρωπαϊκή ένωση. εμπρησμοί σε εστίες όπου φιλοξενούνται αιτούντες άσυλο, σοκαρισμένοι πολίτες από σεξουαλικές επιθέσεις με δράστες μετανάστες στην κολωνία: οι γερμανοί, που αρχικά είχαν καλωσορίσει τους πρόσφυγες με λιχουδιές και αρκουδάκια τώρα δείχνουν να έχουν χάσει τον μπούσουλα. και είναι όλο και περισσότεροι αυτοί που στρέφονται στη λαϊκιστική δεξιά, η οποία ελπίζει σήμερα σε ένα ιστορικό ρεκόρ. η εναλλακτική για τη γερμανία (afd), το νεόκοπο ξενοφοβικό κόμμα που ιδρύθηκε πριν από τρία χρόνια για να αμφισβητήσει το ευρώ, ίσως αναδειχθεί νικητής των εκλογών αυτών, λαμβάνοντας ποσοστό μεταξύ 9\% και 19\% των προθέσεων ψήφου ανάλογα με την περιοχή. μετά τη σημερινή ψηφοφορία, το afd, που έχει εξασφαλίσει έδρες στο ευρωπαϊκό κοινοβούλιο, μπορεί να εκπροσωπείται στα μισά από τα 16 περιφερειακά κοινοβούλια. στη σαξονία-άνχαλτ, με 19\% στις δημοσκοπήσεις, το afd διεκδικεί ακόμη και τη θέση της δεύτερης πολιτικής δύναμης του κρατιδίου από το κόμμα της ριζοσπαστικής αριστεράς, die linke. και παρότι αυτό το μη προνομιούχο κρατίδιο της πρώην ανατολικής γερμανίας φιλοξενεί λίγους πρόσφυγες. η άνοδος αυτού του κόμματος αποτελεί πρωτοφανές σενάριο από το 1945 σε μια χώρα που διαρκώς αναζητεί ηθικό παράδειγμα, μετά τη ναζιστική φρίκη. παρά τα λεκτικά ολισθήματα, το afd, που τάσσεται κυρίως υπέρ των κλειστών συνόρων, συνεχίζει τη λαϊκιστική του πορεία δημαγωγώντας στα πλήθη κατά των παραδοσιακών κομμάτων. "ντροπή" τα παραδοσιακά κόμματα απορρίπτουν οποιαδήποτε συνεργασία με το afd, που χαρακτηρίζεται "ντροπή για τη γερμανία", σύμφωνα με τον υπουργό οικονομικών. για τα κόμματα αυτά, το cdu και το spd κυριαρχούν στην πολιτική εδώ και 70 χρόνια, το καθήκον να σχηματίσουν βιώσιμους περιφερειακούς συνασπισμούς μπορεί να καταστεί ακόμη πιο δύσκολο. σίγουρη, η άγγελα μέρκελ ορκίζεται τουλάχιστον ότι το afd θα χάσει τη δύναμή του μόλις επιλυθεί η προσφυγική κρίση. στη βάδη-βυρτεμβέργη, ένα εύπορο κρατίδιο όπου το ποσοστό ανεργίας είναι το χαμηλότερο της γερμανίας, 4\%, το cdu μπορεί να χάσει έως και δέκα μονάδες σε σχέση με τις προηγούμενες εκλογές και να εκθρονιστεί από πρώτη περιφερειακή πολιτική δύναμη από τους πράσινους. οι λαϊκιστές λαμβάνουν 11\%. ο υποψήφιος πρωθυπουργός σε αυτό το κρατίδιο, ο γκίντο βολφ, πήρε αποστάσεις από την μέρκελ κατά τη διάρκεια της προεκλογικής εκστρατείας σε μια προσπάθεια να πείσει ένα εκλογικό σώμα πολύ προσηλωμένο στις χριστιανικές αξίες που ανησυχεί για τη ροή μουσουλμάνων μεταναστών. στο γειτονικό κρατίδιο της ρηνανίας-παλατινάτου, η υποψήφια του cdu τζούλια κλέκνερ επιχείρησε επίσης να κρατήσει αποστάσεις, συνηγορώντας υπέρ μιας αυστηρότερης μεταναστευτικής πολιτικής. δίνει πλέον μάχη στήθος με στήθος με την υποψήφια των σοσιαλδημοκρατών, μαλού ντρέγερ, ενώ το afd μπορεί να ελπίζει σε ένα ποσοστό 9\%. ορισμένοι πολιτικοί αναλυτές σημειώνουν ωστόσο ότι όσο κι αν αμφισβητείται η άγγελα μέρκελ παραμένει πολύ ισχυρή και χωρίς αντίπαλο ικανό να την νικήσει. η δημοτικότητά της εξάλλου αυξήθηκε τις τελευταίες εβδομάδες. & 736 & very low & Low & Socio-Economic & NA & NA & 2016-03-13 & 2016 & 2 & ECO
Frame & v.low & National & 500-1000 & 1.8924259 & 1.3590316 & -1.2391092 & 0.9814602 & -0.7922579 & 0.0 & -0.9049211 & -1.0736569 & Recipient & Domestic & Domestic & Domestic & Domestic|ECO & Neutral\\
Greece & http://www.902.gr/eidisi/oikonomia/111361/zita-na-perasei-ston-esm-o-eleghos-tis-dimosionomikis-peitharhias-stin & 321 & 902.gr & Private/Non-Public & Online and Offline & National & low = CP mentioned more times but NOT important part of story (mainly about others issues) & Political leverage & Negative & EU + Other country & No myth & NA & NA & NA & NA & NA & NA & NA & NA & Greece & ζητά να περάσει στον esm ο έλεγχος της & 2016-10-15 & διαρθρωτικά ταμεία & σε μια ακόμα παρέμβαση που αντανακλά τους εντεινόμενους ενδοκαπιταλιστικούς ανταγωνισμούς στο εσωτερικό της ευρωζώνης προέβη ο υπουργός οικονομικών της γερμανίας, β. σόιμπλε, προτείνοντας ο έλεγχος της δημοσιονομικής πειθαρχίας στη ζώνη του ευρώ να περάσει από την ευρωπαϊκή επιτροπή στον ευρωπαϊκό μηχανισμό σταθερότητας (esm). συγκεκριμένα, ο γερμανός υπουργός οικονομικών, σε συνέντευξή του στην εφημερίδα "stuttgarter zeitung", επαναφέρει την κριτική του απέναντι στον πρόεδρο της κομισιόν, ζ. κ. γιούνκερ, λέγοντας ότι έχει αναλάβει έναν "πολιτικό ρόλο" που δεν συνάδει "με το ρόλο του ως ουδέτερος θεματοφύλακας των συνθηκών" της ευρώπης. με βάση αυτόν τον ισχυρισμό, αναφέρει ότι θα πρέπει να εξευρεθεί ένας άλλος τρόπος ώστε να εφαρμοστεί πραγματικά το σύμφωνο σταθερότητας που ορίζει τους κανόνες για τα ελλείμματα και προχωρά στην πρότασή του να αναλάβει ο esm αυτόν το ρόλο. "αυτός είναι ο λόγος για τον οποίο λέω ότι μπορούμε να οδηγήσουμε το ευρωπαϊκό ταμείο για τις κρίσεις, τον esm, να αναπτυχθεί σε τέτοια κατεύθυνση σε κάθε περίπτωση για τις χώρες της νομισματικής ένωσης. ο esm αποφαινόταν για τις προβλέψεις των προϋπολογισμών όχι από πολιτική άποψη, αλλά ανάλογα με τους αυστηρούς κανόνες", ανέφερε ο β. σόιμπλε. σύμφωνα με το σύμφωνο σταθερότητας, τη διαδικασία του ευρωπαϊκού εξαμήνου και τους υπόλοιπους μηχανισμούς των μνημονίων διαρκείας που ισχύουν για όλα τα κράτη μέλη της εε, όλες οι χώρες της ευρωζώνης πρέπει να υποβάλλουν κάθε χρόνο στην κομισιόν τα σχέδια προϋπολογισμού τους, διατηρώντας το ποσοστό του κρατικού χρέους κάτω από το 60\% του αεπ και το δημοσιονομικό έλλειμμα κάτω από το 3\% του αεπ. μέχρι τώρα, ωστόσο, οι προβλεπόμενες κυρώσεις για τη μη τήρηση των εν λόγω κανόνων (πρόστιμα και αναστολή χρηματοδότησης από τα διαρθρωτικά ταμεία της εε) δεν έχουν ενεργοποιηθεί ποτέ, με πιο πρόσφατο τέτοιο παράδειγμα την απόρριψη της επιβολής προστίμων σε ισπανία και πορτογαλία για τα "υπερβολικά ελλείμματά" τους. να σημειωθεί, βέβαια, ότι το θέμα των δύο ιβηρικών χωρών επανέφερε την περασμένη βδομάδα ο αντιπρόεδρος της κομισιόν, β. ντομπρόβσκις, δηλώνοντας ότι παραμένει ανοιχτό το ενδεχόμενο της αναστολής χρηματοδότησης από τα ευρωενωσιακά ταμεία. σε κάθε περίπτωση, με αφορμή και τη συζήτηση περί κυρώσεων, αλλά και με την τωρινή παρέμβαση σόιμπλε, "στον αφρό" επανέρχονται οι ενδοκαπιταλιστικές κόντρες, με ισχυρές καπιταλιστικές χώρες με μεγάλα κρατικά χρέη (κυρίως γαλλία και ιταλία) να διεκδικούν ακόμα μεγαλύτερα περιθώρια "ευελιξίας" έναντι των ευρωενωσιακών κανόνων, προκειμένου να διασφαλίσουν μεγαλύτερη κρατική στήριξη των μονοπωλιακών τους ομίλων, σε μια κόντρα κατά κύριο λόγο με τη γερμανία και με την ενεργό παρέμβαση των ηπα... καμία... "ευελιξία", βέβαια, δεν επιδεικνύει καμία από όλες τις παραπάνω πλευρές στην κλιμάκωση της αντιλαϊκής επίθεσης, σε όλα τα κράτη μέλη της εε, η οποία αποτελεί προϋπόθεση για τη στήριξη των κερδών και της ανταγωνιστικότητας του κεφαλαίου. & 436 & low & Low & Power & NA & NA & 2016-10-15 & 2016 & 2 & POL
Frame & low-medium & National & <500 & 1.8924259 & 1.3590316 & -1.2391092 & 0.9814602 & -0.7922579 & 0.0 & -0.9049211 & -1.0736569 & Recipient & European & European & European & European|POL & Negative\\
\addlinespace
Greece & https://www.naftemporiki.gr/story/1386352 & 386 & naftemporiki.gr & Private/Non-Public & Online and Offline & National & very low = CP mentioned once & Public services & Positive & Subnational & No myth & NA & NA & NA & NA & NA & NA & NA & NA & Greece & περιφέρεια αττικής: χρηματοδότηση 13,35 εκατ. ευρώ σε 21 νοσοκομεία & 2018-08-30 & ευρωπαϊκό ταμείο περιφερειακής ανάπτυξης & δείτε ακόμα ειναπ: μεγάλες ελλείψεις ειδικευόμενων γιατρών στα νοσοκομεία 30/08 12:17 σε νέα χρηματοδότηση, ύψους 13,35 εκατ. ευρώ, προς 21 νοσοκομεία του λεκανοπεδίου, για την προμήθεια νοσοκομειακού εξοπλισμού, προχωρά η περιφέρεια αττικής, πρωτοβουλία με την οποία ολοκληρώνεται η πρώτη φάση ολοκληρωμένου σχεδίου ενίσχυσης της πρωτοβάθμιας, δευτεροβάθμιας και τριτοβάθμιας φροντίδας υγείας με ένα μίγμα εθνικής και κοινοτικής χρηματοδότησης, ύψους περίπου 90 εκατομμυρίων. ήδη η περιφερειάρχης, ρένα δούρου, υπέγραψε νέα πρόσκληση για τη συμμετοχή 21 νοσοκομείων της αττικής σε σχετική δράση του επιχειρησιακού προγράμματος "αττική 2014-2020". η δράση αυτή, συνολικού προϋπολογισμού 13.356.880 ευρώ, εντάσσεται στον άξονα προτεραιότητας 10 "ανάπτυξη-αναβάθμιση στοχευμένων κοινωνικών υποδομών και υποδομών υγείας" και συγχρηματοδοτείται από το ευρωπαϊκό ταμείο περιφερειακής ανάπτυξης (ετπα), με στόχο τη διασφάλιση της εύρυθμης λειτουργίας των νοσοκομείων και της δημόσιας υγείας, την αναβάθμιση των παρεχόμενων υπηρεσιών, αλλά και την ισότιμη πρόσβαση σε υπηρεσίες υγείας υψηλού επιπέδου. με την πρόσκληση καλούνται να υποβάλουν προτάσεις στο πεπ "αττική 2014-2020" τα εξής νοσηλευτικά ιδρύματα: γ.ν.α. σισμανόγλειο, γ.ν.α. κατ, νοσοκομείο θείας πρόνοιας "η παμμακάριστος", γ.ν. παίδων πεντέλης, γ.ν. ν.ιωνίας "κωνσταντοπούλειο"-πατησίων, γενικό ογκολογικό νοσοκομείο κηφισιάς "οι 'αγιοι ανάργυροι", γ.ν.α. "γ.γεννηματάς", γ.ν.α. "ο ευαγγελισμός", γ.ν.ν.θ.α. "η σωτηρία", γ.ν.α. αλεξάνδρα, α.ο.ν.α. "ο 'αγιος σάββας", γ.ν.α. ιπποκράτειο, γ.ν.α. κοργιαλένειο - μπενάκειο ε.ε.ς., γ.ν.α. "η ελπίς", ν.α. \& δ.ν.α. "ανδρέας συγγρός", γ.ν.α. "λαϊκό", γ.ν. παίδων "αγλαΐα κυριακού", νοσοκομείο "έλενα βενιζέλου", γ.ν.π. "τζάνειο", π.γ.ν. "αττικόν", ψυχιατρικό νοσοκομείο αττικής. οι χρηματοδοτήσεις αυτές της περιφέρειας αττικής αφορούν: - την προμήθεια υπερσύγχρονου και απαραίτητου εξοπλισμού για τα νοσοκομεία της αττικής, κόστους 40,6 εκατ. ευρώ (απόφαση για την οποία χρειάστηκε και νομοθετική ρύθμιση). - την προμήθεια νέων ασθενοφόρων για το εκαβ, κόστους 3,6 εκατ. ευρώ. - την προμήθεια εξοπλισμού πρωτοβάθμιας υγείας και κέντρου υγείας κερατσινίου, κόστους 9,5 εκατ. ευρώ. - την υπογραφή προγραμματικής σύμβασης με το υπουργείο εθνικής άμυνας για την κάλυψη επειγουσών ελλείψεων ιατροτεχνολογικού και μηχανολογικού εξοπλισμού των στρατιωτικών νοσοκομείων με προϋπολογισμό 20 εκατ. ευρώ. & 351 & very low & Low & Socio-Economic & NA & NA & 2018-08-30 & 2018 & 3 & ECO
Frame & v.low & National & <500 & 1.8924259 & 1.3590316 & -1.2391092 & 0.9814602 & -0.7922579 & 0.0 & -0.9049211 & -1.0736569 & Recipient & Domestic & Domestic & Domestic & Domestic|ECO & Positive\\
Greece & http://www.newsbeast.gr/greece/arthro/2382554/metro-aerodromio-ke-paralia-thessalonikis-sto-epikentro-epafon & 381 & Newsbeast.gr & Private/Non-Public & Online only & National & very high = CP is most important issue + CP is mentioned in title/headline & Infrastructure & Positive & EU + National & No myth & Bureaucracy and/or delays & Negative & National & No myth & NA & NA & NA & NA & Greece & μετρό, αεροδρόμιο και παραλία θεσσαλονίκης στο επίκεντρο επαφών & 2016-09-19 & διαρθρωτικά ταμεία & στην πόλη ευρωβουλευτές της επιτροπής περιφερειακής ανάπτυξης του ευρωκοινοβουλίου τα έργα του μετρό, του αεροδρομίου και της ανάπλασης της παραλίας βρέθηκαν στο επίκεντρο των επαφών των ευρωβουλευτών της επιτροπής περιφερειακής ανάπτυξης του ευρωπαϊκού κοινοβουλίου, που επισκέφτηκαν τη θεσσαλονίκη, προκειμένου να ενημερωθούν για έργα, τα οποία υλοποιήθηκαν ή υλοποιούνται με χρηματοδότηση από τα ευρωπαϊκά ταμεία. ικανοποιημένη για την ανάπλαση του θαλάσσιου μετώπου της θεσσαλονίκης δήλωσε η ίσκρα μιχαήλοβα, πρόεδρος της επιτροπής περιφερειακής ανάπτυξης του ευρωπαϊκού κοινοβουλίου, η οποία τόνισε ότι το εν λόγω έργο άλλαξε την εικόνα της πόλης. σχετικά με το μετρό, ανέφερε ότι η αντιπροσωπεία είχε μια εμπεριστατωμένη και λεπτομερή ενημέρωση για όλα τα στάδια υλοποίησης του προγράμματος, συμπεριλαμβανομένων και των οικονομικών ζητημάτων που προέκυψαν και του πλάνου ολοκλήρωσής του. έκανε λόγο για ένα από τα πιο περίπλοκα έργα που υλοποιούνται στην ελλάδα και λόγω των αρχαιολογικών ανασκαφών και από χρηματοδοτικής άποψης καθώς άρχισε το 2006, συνεχίστηκε στη δεύτερη προγραμματική περίοδο 2007-2013, και αναμένεται να ολοκληρωθεί στο τέλος της τρίτης χρηματοδοτικής περιόδου, το 2020. παρ΄ όλα αυτά, εκτίμησε ότι "σήμερα το χρονοδιάγραμμα για ολοκλήρωση του έργου το 2020 είναι πιο ρεαλιστικό από ό,τι ήταν όταν ξεκίνησε το έργο, γιατί τώρα όσοι είναι υπεύθυνοι γι' αυτό γνωρίζουν ήδη τι θα αντιμετωπίσουν και ήδη έλυσαν ορισμένα ζητήματα". "είμαστε εδώ για να εντοπίσουμε προβλήματα, να δούμε αν τα ειδικά μέτρα που ελήφθησαν από την ευρωπαϊκή ένωση για την υποστήριξη της ελλάδας, υλοποιούνται στη βάση τους και αυτά τα ειδικά μέτρα υποστηρίζουν έργα όπως το μετρό. είμαστε η επιτροπή που προετοίμασε την απόφαση του ευρωπαϊκού κοινοβουλίου και αισθανόμαστε υπεύθυνοι, αλλά και μέλη της ομάδας της θεσσαλονίκης επίσης", τόνισε χαρακτηριστικά η κ. μιχαήλοβα. ο αντιπρόεδρος του ευρωπαϊκού κοινοβουλίου και ευρωβουλευτής του συριζα, δημήτρης παπαδημούλης, τόνισε ότι "το ευρωκοινοβούλιο έχει εγκρίνει με πολύ μεγάλη πλειοψηφία αυξημένη χρηματοδότηση προς την ελλάδα μέσα από τα διαρθρωτικά ταμεία, σε μια περίοδο που η χώρα μας διανύει μεγάλη κρίση. κι αυτό φαίνεται ότι εδώ στη θεσσαλονίκη έχει αρχίσει και πιάνει τόπο". με αφορμή την ενημέρωση της αντιπροσωπείας των ευρωβουλευτών, σχολίασε ότι "μεγάλα έργα που είχαν κολλήσει για χρόνια όπως το μετρό προχωράνε, το έργο στην παραλία ήδη λειτουργεί και προσφέρει οφέλη και στην πόλη και στους πολίτες". ο κ. παπαδημούλης επισήμανε ότι σε σύσκεψη που έγινε στο δημαρχείο της θεσσαλονίκης ενημερώθηκε από τον δήμαρχο γιάννη μπουτάρη και για την πρόοδο των έργων και για τους καινούριους στόχους και τις δυσκολίες που υπάρχουν και απαιτούν τη συνεργασία του δήμου, της περιφέρειας και της κυβέρνησης. ωστόσο εξέφρασε τη δυσαρέσκεια, όπως είπε, όλων των μελών της αντιπροσωπείας του ευρωκοινοβουλίου για την απουσία της περιφέρειας και του περιφερειάρχη από αυτή τη συνάντηση. "είναι κρίμα σε τόσο σημαντικές συναντήσεις να λείπει ο εκλεγμένος περιφερειάρχης γιατί χωρίς τη συνεργασία του δήμου, της περιφέρειας, της κυβέρνησης και το ευρωκοινοβουλίου που θέλει να βοηθήσει δεν μπορούμε να έχουμε τα αποτελέσματα που πρέπει και που μπορούμε να έχουμε", πρόσθεσε. "η περιφέρεια προσεκλήθη αλλά δεν ήρθε", είπε ο κ. μπουτάρης. για το θέμα ρωτήθηκαν από το αθηναϊκό-μακεδονικό πρακτορείο ειδήσεων και οι υπηρεσίες της περιφέρειας που απάντησαν ότι δεν υπήρξε καμία πρόσκληση από την επιτροπή του ευρωπαϊκού κοινοβουλίου για ενημερωτική συνάντηση και ότι η περιφέρεια κεντρικής μακεδονίας είναι διαθέσιμη για κάθε ενημέρωση για τα προγράμματα του εσπα σε όλη την περιφέρεια, καθώς έχει την ευθύνη γι' αυτά. την εκτίμηση ότι το μετρό έχει ξεμπλοκαριστεί διατύπωσε η ευρωβουλευτής της νέας δημοκρατίας και του ελκ μαρία σπυράκη, η οποία πρόσθεσε: "είναι ευτύχημα που το μετρό ξεμπλοκάρεται, το ότι τα χρήματα των ευρωπαίων φορολογουμένων δείχνουν πλέον να πιάνουν τόπο και οι θεσσαλονικείς να επωφελούνται από αυτό. το μετρό δεν είναι όχημα για να κάνει κανένας καριέρα, ούτε για να θριαμβολογεί για το ξεμπλοκάρισμα, ούτε για να μπαίνει σε άγονες αντιπαραθέσεις σχετικά με ευρήματα μεγάλης ιστορικής αξίας. είναι ανάγκη για την πόλη και είναι πάρα πολύ σημαντικό το ότι προχωρά με ευρωπαϊκά χρήματα". κατά τη διάρκεια της σύσκεψης που πραγματοποιήθηκε στην αίθουσα συνεδριάσεων του δημοτικού συμβουλίου θεσσαλονίκης, ο δήμαρχος θεσσαλονίκης ενημέρωσε την αντιπροσωπεία των ευρωβουλευτών για τα μελλοντικά σχέδια του δήμου, που, όπως είπε, εντάσσονται στο πλαίσιο του στρατηγικού του σχεδιασμού για την περίοδο 2020 - 2030. "η συνεννόηση που είχαμε με την περιφέρεια κεντρικής μακεδονίας ήταν αρνητική", είπε και τόνισε: "παρά την εκπόνηση του στρατηγικού σχεδιασμού υπήρξε αρνητικό αποτέλεσμα για το εργαλείο που δημιουργήσαμε σχετικά με τις ολοκληρωμένες χωρικές επενδύσεις για τη βιώσιμη αστική ανάπτυξη". για το λόγο αυτό ανέφερε ότι ο δήμος έχει ζητήσει από το αρμόδιο υπουργείο να γίνει ο ίδιος ενδιάμεση διαχειριστική αρχή και ήρθε σε επαφή με τη μονάδα οργάνωσης της διαχείρισης αναπτυξιακών προγραμμάτων (μοδ) για να δημιουργηθεί μια ομάδα εργασίας ώστε η μοδ να βοηθήσει το δήμο να διεκπεραιώνει πιο γρήγορα όποια προγράμματα επιθυμεί. ο κ. μπουτάρης μίλησε και για τα προβλήματα γραφειοκρατίας που αντιμετωπίζει ο δήμος με το ελληνικό δημόσιο, τα ζητήματα αρμοδιοτήτων διαφορετικών φορέων και τους περιορισμένους πόρους που υπάρχουν ενώ σημείωσε ότι επεξεργάζεται σχέδιο για να αναπτυχθεί το παραλιακό μέτωπο. "είμαστε η μοναδική μεσογειακή πόλη στον κόσμο που δεν έχει ούτε μια βάρκα στην παραλία της, το έχω πει εκατό εκατομμύρια φορές", είπε χαρακτηριστικά. αναφέρθηκε, μεταξύ άλλων, στα σχέδια του δήμου για τη δημιουργία δύο μεγάλων πάρκινγκ στην περιοχή του μακεδονία παλλάς, πάνω από το δρόμο, τη συνεργασία που επιδιώκει με την πγδμ για τον καθαρισμό του ποταμού αξιού από λύματα και τις σκέψεις για τη σύνδεση του θερμαϊκού κόλπου με τον ποταμό δούναβη, μέσω του αξιού και του μοράβα. μετά την ολοκλήρωση της σύσκεψης στο δημαρχείο, οι ευρωβουλευτές επισκέφθηκαν το αεροδρόμιο προκειμένου να ενημερωθούν για τα έργα που υλοποιούνται εκεί. & 908 & very high & High & Socio-Economic & Governance & NA & 2016-09-19 & 2016 & 2 & ECO
Frame & high-very high & National & 500-1000 & 1.8924259 & 1.3590316 & -1.2391092 & 0.9814602 & -0.7922579 & 0.0 & -0.9049211 & -1.0736569 & Recipient & Domestic & European & Mixed & Domestic|ECO & Positive\\
Greece & https://www.naftemporiki.gr/\_story/1377697 & 374 & naftemporiki.gr & Private/Non-Public & Online and Offline & National & low = CP mentioned more times but NOT important part of story (mainly about others issues) & Political capital/interests & Negative & National & No myth & NA & NA & NA & NA & NA & NA & NA & NA & Greece & υπ. οικονομίας: κυνική διασπορά fake news από τον κυριάκο μητσοτάκη & 2018-07-31 & διαρθρωτικά ταμεία & δείτε ακόμα κυρ. μητσοτάκης: αδυνατώ να αντιληφθώ τι σημαίνει πολιτική ευθύνη δίχως παραίτηση 31/07 12:12 συνέντευξη τύπου του κυρ. μητσοτάκη για τις πυρκαγιές 30/07 17:35 "ο κ. μητσοτάκης και η νέα δημοκρατία ζουν μάλλον σε μία δική τους, εικονική πραγματικότητα. δεν εξηγείται διαφορετικά η σπουδή τους να εμφανίσουν ως επιτυχία τους τη θετική ανταπόκριση της ευρωπαϊκής επιτροπής στο αίτημα της ελληνικής κυβέρνησης για οικονομική βοήθεια προς τους πυρόπληκτους" αναφέρει το υπουργείο οικονομίας και ανάπτυξης, με αφορμή τις σημερινές δηλώσεις του κυριάκου μητσοτάκη. σύμφωνα με το υπουργείο, "τα γεγονότα τούς διαψεύδουν κατηγορηματικά", καθώς "ο πρωθυπουργός επικοινώνησε από την πρώτη στιγμή της κρίσης με τον κ. γιουνκέρ, ζητώντας τη στήριξη της ευρωπαϊκής ένωσης. παράλληλα, ο αναπληρωτής υπουργός οικονομίας, αλέξης χαρίτσης, βρίσκεται όλες αυτές τις ημέρες σε διαρκή επικοινωνία με την αρμόδια επίτροπο κορίνα κρέτσου και με επιστολή του από την 25η ιουλίου ζήτησε να ενεργοποιηθεί γρήγορα το ταμείο αλληλεγγύης της ε.ε. και να εξασφαλισθεί πρόσθετη ενίσχυση για τις πληγείσες περιοχές από τα ευρωπαϊκά διαρθρωτικά ταμεία, με την κα. κρέτσου να απαντά θετικά την ίδια ημέρα (επισυνάπτονται οι επιστολές). είναι αυτές οι άμεσες πρωτοβουλίες της κυβέρνησης που κινητοποίησαν την ευρωπαϊκή επιτροπή, όπως άλλωστε επισημαίνει και η κα. κρέτσου σε επίσημη δήλωσή της την παρασκευή 27 ιουλίου. παρόλα αυτά, η ν.δ. επέλεξε την ίδια μέρα να βγάλει επίσημη ανακοίνωση, συγχαίροντας τον εαυτό της για τη θετική ανταπόκριση της επιτροπής. αναλογιζόμενοι την κρισιμότητα της στιγμής και μη θέλοντας να εμπλακούμε στην προσφιλή για το κόμμα της αξιωματικής αντιπολίτευσης, μικροπολιτική αντιπαράθεση, δεν απαντήσαμε. σήμερα, όμως, ο κ. μητσοτάκης επανήλθε εμφατικά στους ίδιους ανυπόστατους ισχυρισμούς και την κυνική διασπορά fake news. είναι λοιπόν χρέος μας να αποκαταστήσουμε την αλήθεια. οι στιγμές είναι κρίσιμες και απαιτούν από όλους μας ευθύνη και σοβαρότητα και όχι επικοινωνιακά πυροτεχνήματα και μικροκομματικές κινήσεις εντυπωσιασμού, σαν και αυτές που επιδίδονται η νδ και ο αρχηγός της" καταλήγει το υπουργείο οικονομίας και ανάπτυξης. & 315 & low & Low & Power & NA & NA & 2018-07-31 & 2018 & 3 & POL
Frame & low-medium & National & <500 & 1.8924259 & 1.3590316 & -1.2391092 & 0.9814602 & -0.7922579 & 0.0 & -0.9049211 & -1.0736569 & Recipient & Domestic & Domestic & Domestic & Domestic|POL & Negative\\
Greece & https://e-thessalia.gr/neos-exoplismos-gia-tin-neyrologiki-kai-psychiatriki-kliniki-toy-panepistimiakoy-nosokomeioy-larisas/ & 351 & e-thessalia.gr & Private/Non-Public & Online only & Regional/Local & low = CP mentioned more times but NOT important part of story (mainly about others issues) & Infrastructure & Positive & EU + National + Subnational & No myth & Public services & Positive & EU + National + Subnational & NA & NA & NA & NA & NA & Greece & νέος εξοπλισμός για την νευρολογική και ψυχιατρική κλινική του πανεπιστημιακού νοσοκομείου λάρισας - e-thessalia.gr & 2018-10-26 & ευρωπαϊκό ταμείο περιφερειακής ανάπτυξης & στη φάση της υλοποίησης μπαίνει η προμήθεια και εγκατάσταση του νέου εξοπλισμού για την νευρολογική κλινική και την ψυχιατρική κλινική του π.γ.ν.λ, καθώς ο περιφερειάρχης θεσσαλίας έδωσε την έγκριση δημοπράτησης του έργου προϋπολογισμού 221.225,00 ευρώ € (με φπα). ειδικότερα ολοκληρώθηκαν οι προβλεπόμενες διαδικασίες σύνταξης - υποβολής και ελέγχου από τη διαχειριστική αρχή περιφέρειας θεσσαλίας στο έργο του π.ε.π. θεσσαλίας 2014-2020, με δικαιούχο, κύριο και φορέα υλοποίησης το πανεπιστημιακό γενικό νοσοκομείο λάρισας - γενικό νοσοκομείο λάρισας . το έργο συγχρηματοδοτείται από το ευρωπαϊκό ταμείο περιφερειακής ανάπτυξης και εθνικούς πόρους. ακολουθεί πλέον η έγκριση και δημοσιοποίηση της προκήρυξης από το πανεπιστημιακό νοσοκομείο λάρισας, για την ανάδειξη αναδόχου και υλοποίηση του έργου. ο περιφερειάρχης θεσσαλίας και πρόεδρος της ένωσης περιφερειών ελλάδος τόνισε τα εξής: "στηρίζουμε έμπρακτα τη δημόσια υγεία. αποτελεί προτεραιότητά μας καθώς πιστεύουμε ότι μια χώρα είναι σύγχρονη και ανεπτυγμένη όταν οι πολίτες της προστρέχουν στις δημόσιες υπηρεσίες υγείας. για το λόγο αυτό δώσαμε τη δυνατότητα στο πανεπιστημιακό νοσοκομείο λάρισας τη δυνατότητα να εξοπλίσει με νέα μηχανήματα για την νευρολογική κλινική και την ψυχιατρική κλινική, προϋπολογισμού 221.225,00 ευρώ.. έχουμε χρηματοδοτήσει έργα συνολικού προϋπολογισμού πάνω 80 εκ. ευρώ στηρίζοντας στην πράξη τα δημόσια νοσοκομεία μας και τη δημόσια υγεία στη θεσσαλία". αντικειμενο εργου: ο διαγωνισμός αφορά στην προμήθεια ιατροτεχνολογικού εξοπλισμού για την νευρολογική κλινική και την ψυχιατρική κλινική του πανεπιστημιακού νοσοκομείου. η νευρολογική κλινική θα εξοπλιστεί με έναν ηλεκτρομυογράφο 6 καναλιών, με έναν ηλεκτροεγκεφαλογράφο 24ωρης καταγραφής τουλάχιστον 25 καναλιών ηλεκτροεγκεφαλογραφήματος και 14 διπολικών καναλιών και με δύο μόνιτορ μεγέθους τουλάχιστον 10 ιντσών και τουλάχιστον 4 κυματομορφών. η ψυχιατρική κλινική θα εξοπλιστεί με έναν ηλεκτροεγκεφαλογράφο τουλάχιστον 32 καναλίων από τα οποία τουλάχιστον 9 θα είναι διπολικές και βοηθητικές είσοδοι, συμβατό με διακρανιακό μαγνητικό ερεθιστή, ένα σύστημα νευροπλοήγησης, συμβατό με μαγνητικό ερεθιστή και ένα μαγνητικό ερεθιστή με δυνατότητα παραγωγής μονοφασικής, διφασικής κυματομορφής και κυματομορφής μισού ημιτόνου. με την προμήθεια του του ανωτέρω εξοπλισμού θα βελτιωθεί σημαντικά το έργο της νευρολογικής κλινικής και της ψυχιατρικής κλινικής. συγκεκριμένα η νευρολογική κλινική ετησίως εξυπηρετεί περίπου 2.500 περιστατικά. η προμήθεια του αιτούμενου εξοπλισμού θα επιτρέψει τη διενέργεια εξετάσεων, οι οποίες μέχρι τώρα γίνονται σε νοσοκομειακές μονάδες των δύο μεγάλων αστικών κέντρων. ως εκ τούτου ο πληθυσμός της θεσσαλίας θα είναι δυνατό να εξυπηρετείται στο πανεπιστημιακό γενικό νοσοκομείο λάρισας, εξασφαλίζοντας βέλτιστες υπηρεσίες υγείας και αποφεύγοντας περιττά έξοδα και ταλαιπωρία κατά την μετακίνησή τους στα δύο μεγάλα αστικά κέντρα. & 395 & low & Low & Socio-Economic & Socio-Economic & NA & 2018-10-26 & 2018 & 3 & ECO
Frame & low-medium & Regional & <500 & 1.8924259 & 1.3590316 & -1.2391092 & 0.9814602 & -0.7922579 & 0.0 & -0.9049211 & -1.0736569 & Recipient & Domestic & European & Mixed & Domestic|ECO & Positive\\
Greece & http://www.newsbeast.gr/woman/arthro/785173/kastro-ton-andron-i-ereuna-stin-ellada/ & 390 & Newsbeast.gr & Private/Non-Public & Online only & National & very low = CP mentioned once & Research \& innovation & Factual & National & No myth & NA & NA & NA & NA & NA & NA & NA & NA & Greece & "κάστρο" των ανδρών η έρευνα στην ελλάδα & 2015-02-02 & ευρωπαϊκό ταμείο περιφερειακής ανάπτυξης & ένατη μεταξύ των χωρών της ε.ε. σε γυναικείο δυναμικό έρευνας και ανάπτυξης κυριαρχία των ανδρών έναντι των γυναικών στον τομέα της έρευνας και ανάπτυξης στην ελλάδα, παρουσιάζει νέα έκδοση του εθνικού κέντρου τεκμηρίωσης (εκτ). από την ανάλυση των στατιστικών στοιχείων προκύπτει πως οι άντρες ερευνητές υπερισχύουν με ποσοστό 63,3\% έναντι του 36,7\% των γυναικών. μάλιστα συγκρίνοντας τους ελληνικούς δείκτες των γυναικών που απασχολούνται στον τομέα της έρευνας με ευρωπαϊκούς η έκδοση φιλοδοξεί να συνεισφέρει στοιχεία στο σχεδιασμό εθνικών πολιτικών για την επίτευξη της ισότητας των δύο φύλων στην έρευνα. όπως προκύπτει από την ανάλυση των επίσημων στατιστικών στοιχείων, που παράγει το εκτ για την έρευνα \& ανάπτυξη, το 2011, στην ελλάδα απασχολούνται συνολικά 29.879 γυναίκες σε δραστηριότητες έρευνας \& ανάπτυξης (ως ερευνήτριες, τεχνικό ή λοιπό προσωπικό υποστήριξης), αριθμός που αντιστοιχεί σε ποσοστό 42,5\% του συνολικού δυναμικού ε\&α, ενώ ο μέσος όρος στην εε 28 είναι 34,8\% και η ελλάδα κατατάσσεται 9η μεταξύ των χωρών της εε. οι γυναίκες ερευνήτριες έχουν μεγαλύτερη συμμετοχή από τους άνδρες στις "ανθρωπιστικές επιστήμες" (54,1\%), ενώ υψηλό ποσοστό (43\%) καταγράφεται και στον τομέα "ιατρική και επιστήμες υγείας". το φαινόμενο της "γυάλινης οροφής" (glass ceiling), της συσσώρευσης, δηλαδή, των γυναικών στις χαμηλές βαθμίδες της ιεραρχίας, είναι πραγματικότητα και στην ελλάδα: τα ποσοστά απασχόλησης των γυναικών υπερτερούν των ανδρών στις χαμηλότερες ακαδημαϊκές βαθμίδες, ενώ στις υψηλότερες βαθμίδες η εικόνα αντιστρέφεται. η υποστήριξη της διαμόρφωσης πολιτικών στη βάση στοιχείων αποτελεί στρατηγική προτεραιότητα του εκτ, που εφαρμόζει την επιστήμη για την ανάπτυξη πολιτικών έρευνας και καινοτομίας. η έκδοση "h συμμετοχή των γυναικών στην έρευνα και ανάπτυξη στην ελλάδα το 2011" διατίθεται online στη διεύθυνση: http://metrics.ekt.gr/el/node/170. η παραγωγή των στατιστικών έρευνας \& ανάπτυξης (ε\&α) υλοποιείται από το εκτ, μετά από ανάθεση από τη γενική γραμματεία έρευνας και τεχνολογίας και σε συνεργασία με την ελληνική στατιστική αρχή (ελστατ). η ανάλυση των στοιχείων και η εξαγωγή των σχετικών δεικτών για το 2011 έχουν ήδη δημοσιευθεί στην έκδοση του εκτ "δείκτες έρευνας \& ανάπτυξης για δαπάνες και προσωπικό το 2011 στην ελλάδα" η οποία διατίθεται online. η υλοποίηση των σχετικών δράσεων πραγματοποιείται στο πλαίσιο της πράξης "εθνικό πληροφοριακό σύστημα έρευνας και τεχνολογίας/κοινωνικά δίκτυα-περιεχόμενο παραγόμενο από χρήστες" που υλοποιείται από το εθνικό κέντρο τεκμηρίωσης, στο πλαίσιο του επιχειρησιακού προγράμματος "ψηφιακή σύγκλιση" (εσπα), με τη συγχρηματοδότηση της ελλάδας και της ευρωπαϊκής ένωσης-ευρωπαϊκό ταμείο περιφερειακής ανάπτυξης. η συμμετοχή των γυναικών στο προσωπικό ε\&α στην ελλάδα απασχολούνται συνολικά 29.879 γυναίκες σε ε\&α (ερευνητές, τεχνικό προσωπικό, άλλο προσωπικό υποστήριξης), αριθμός που αντιστοιχεί σε ποσοστό 42,5\% του συνολικού δυναμικού ε\&α (70.229 άτομα), ενώ ο μέσος όρος στην εε 28 είναι 34,8\%. με βάση το ποσοστό αυτό απασχόλησης γυναικών σε ε\&α, η ελλάδα κατατάσσεται 9η μεταξύ των χωρών της εε. στην κατηγορία των ερευνητών κυριαρχούν οι άνδρες, με 63,3\% έναντι 36,7\% των γυναικών. στο τεχνικό προσωπικό και το προσωπικό υποστήριξης οι γυναίκες είναι περισσότερες από τους άνδρες, με ποσοστά 51,6\% και 54,5\%, αντίστοιχα. στους τρεις μεγαλύτερους τομείς εκτέλεσης ε\&α, τον τομέα τριτοβάθμιας και μεταδευτεροβάθμιας εκπαίδευσης (τομέας hes), τον κρατικό τομέα (τομέας gov) και τον τομέα επιχειρήσεων (τομέας bes), ο αριθμός των γυναικών που απασχολούνται ως προσωπικό ε\&α είναι μικρότερος των ανδρών. οι γυναίκες υπερτερούν ελαφρά των ανδρών στον τομέα των ιδιωτικών μη κερδοσκοπικών ιδρυμάτων (τομέας pnp). αναλυτικότερα, στον τομέα τριτοβάθμιας και μεταδευτεροβάθμιας εκπαίδευσης απασχολούνται συνολικά 20.687 γυναίκες σε δραστηριότητες ε\&α (ερευνήτριες, τεχνικό και άλλο προσωπικό υποστήριξης) και αποτελούν το 44,6\% του συνολικού προσωπικού ε\&α του τομέα. στον κρατικό τομέα (gov) απασχολούνται 5.730 γυναίκες (43,2\% του συνολικού προσωπικού ε\&α του τομέα), ενώ στον τομέα των επιχειρήσεων (bes) απασχολούνται 3.139 γυναίκες (31,4\% του συνολικού προσωπικού ε\&α του τομέα). ο αριθμός των γυναικών που απασχολείται στον τομέα των ιδιωτικών μη κερδοσκοπικών ιδρυμάτων (pnp) είναι 323 (50,7\% του συνολικού προσωπικού ε\&α του τομέα). με βάση αυτά τα ποσοστά απασχόλησης, η ελλάδα βρίσκεται στην 19η θέση στον τομέα της τριτοβάθμιας και μεταδευτεροβάθμιας εκπαίδευσης, στην 20η θέση στον κρατικό τομέα και στην 7η θέση στον τομέα των επιχειρήσεων. στην κατηγορία "ερευνητές" η υψηλότερη συμμετοχή γυναικών καταγράφεται στον κρατικό τομέα (48,1\% γυναίκες ερευνήτριες), ποσοστό που κατατάσσει την ελλάδα στην 9η θέση, πάνω από τον κοινοτικό μέσο όρο (40,9\%). υψηλότερη θέση από τον κοινοτικό μέσο όρο έχει η ελλάδα και στον τομέα των επιχειρήσεων, καταλαμβάνοντας την 7η θέση, αν και το ποσοστό των γυναικών ερευνητριών στον τομέα είναι μόνο 30,8\%. στον τομέα της τριτοβάθμιας εκπαίδευσης, η ελλάδα υπολείπεται του κοινοτικού μέσου όρου και με ποσοστό 35,6\% γυναίκες στους ερευνητές κατατάσσεται 24η μεταξύ των χωρών μελών της εε. η συμμετοχή των γυναικών στα επιστημονικά πεδία οι γυναίκες ερευνήτριες έχουν μεγαλύτερη συμμετοχή από τους άνδρες στις "ανθρωπιστικές επιστήμες" (54,1\%), ενώ υψηλό ποσοστό (43\%) καταγράφεται και στον τομέα "ιατρική και επιστήμες υγείας". η μικρότερη συμμετοχή γυναικών (29,5\%) αφορά το επιστημονικό πεδίο "επιστήμες μηχανικού \& τεχνολογία". στον τομέα της τριτοβάθμιας και μεταδευτεροβάθμιας εκπαίδευσης, στην ελλάδα και στις υπόλοιπες χώρες της εε, οι γυναίκες ερευνήτριες απασχολούνται περισσότερο από ό,τι οι άνδρες στα επιστημονικά πεδία "ιατρική και επιστήμες υγείας", "κοινωνικές επιστήμες" και "ανθρωπιστικές επιστήμες". αντίστροφη είναι η εικόνα για τα επιστημονικά πεδία "επιστήμες μηχανικού \& τεχνολογία" και "φυσικές επιστήμες". στον κρατικό τομέα, στην ελλάδα και τις περισσότερες χώρες της εε28, η απασχόληση των γυναικών ερευνητριών συγκεντρώνεται, επίσης, στα επιστημονικά πεδία "ιατρική και επιστήμες υγείας", "κοινωνικές επιστήμες" και "ανθρωπιστικές επιστήμες". στον τομέα των επιχειρήσεων, το επιστημονικό πεδίο "ιατρική και επιστήμες υγείας" είναι, επίσης, αυτό στο οποίο καταγράφονται υψηλότερα μερίδια απασχόλησης γυναικών. χαρακτηριστικά απασχόλησης: επίπεδο σπουδών και ηλικία οι περισσότερες γυναίκες που απασχολούνται σε ε\&α είναι κάτοχοι μεταπτυχιακού τίτλου σπουδών ή πτυχίου τριτοβάθμιας εκπαίδευσης (isced 5a \& 5b) και ακολουθούν οι κάτοχοι διδακτορικού τίτλου σπουδών (isced 6). η ίδια εικόνα ισχύει και για τους άνδρες, με λίγο υψηλότερα ποσοστά διδακτόρων έναντι των γυναικών. οι γυναίκες παρουσιάζουν μεγαλύτερο μερίδιο στις χαμηλότερες ηλικιακές ομάδες, ενώ η τάση αυτή αντιστρέφεται καθώς αυξάνεται η ηλικία. για τους άνδρες η κατανομή στις ηλικιακές ομάδες είναι περισσότερο ισορροπημένη, ιδιαίτερα στον κρατικό τομέα. οι γυναίκες στην τριτοβάθμια εκπαίδευση: το φαινόμενο της "γυάλινης οροφής" το φαινόμενο της "γυάλινης οροφής" (glass ceiling), της συσσώρευσης, δηλαδή, των γυναικών στις χαμηλές βαθμίδες της ιεραρχίας, είναι πραγματικότητα και στην ελλάδα. τα ποσοστά απασχόλησης των γυναικών υπερτερούν των ανδρών στις χαμηλότερες ακαδημαϊκές βαθμίδες (γ και δ), ενώ σταδιακά στις υψηλότερες βαθμίδες η εικόνα αντιστρέφεται: στη βαθμίδα β τα ποσοστά ανδρών και γυναικών είναι περίπου ίσα, ενώ στη βαθμίδα α οι άνδρες υπερτερούν σαφώς των γυναικών. επιπλέον, το εκτ εξέτασε την εκπροσώπηση των γυναικών το 2012 στα υψηλότερα όργανα διοίκησης των φορέων της τριτοβάθμιας εκπαίδευσης (πρυτανικές αρχές και διοικητικά συμβούλια των πανεπιστημίων και τει). η συμμετοχή των γυναικών σε ανώτερες θέσεις και σε όργανα λήψης αποφάσεων είναι πολύ χαμηλή, κάτω από τα χαμηλότερα όρια ποσόστωσης που επιχείρησε να εισάγει ο νόμος 2839/2000. οι γυναίκες προάγονται με βραδύτερους ρυθμούς συγκριτικά με τους άνδρες, γεγονός που οφείλεται στη δυσκολία συμφιλίωσης της οικογενειακής και της επαγγελματικής ζωής, σε συνδυασμό με την ανάγκη πολλών ωρών απασχόλησης στις θέσεις υψηλής ευθύνης, όπως φανερώνουν και τα αποτελέσματα της έρευνας του εκτ στο πλαίσιο του έργου "χαρτογράφηση του γυναικείου ερευνητικού δυναμικού στην ελλάδα". το εθνικό κέντρο τεκμηρίωσης υποστηρίζει ενεργά την ενίσχυση της συμμετοχής των γυναικών στον ευρωπαϊκό χώρο έρευνας από το 2007. μετά την εκτεταμένη καταγραφή των ελληνίδων ερευνητριών με το έργο "χαρτογράφηση του επιστημονικού χώρου του ελληνικού γυναικείου ερευνητικού δυναμικού" και τη συμμετοχή στα ευρωπαϊκά έργα gendera και shemera, το εκτ συνεισφέρει στην καθιερωμένη πλέον πανευρωπαϊκή έκδοση "she figures" με αναλυτικά στατιστικά στοιχεία και δείκτες για τις γυναίκες ερευνήτριες. παράλληλα, το εκτ υποστηρίζει την έρευνα στον τομέα της ισότητας των φύλων, μέσω των δράσεων του ως εθνικό σημείο επαφής στο πρόγραμμα "επιστήμη στην κοινωνία και μαζί με την κοινωνία" του ορίζοντα 2020 που είναι το νέο χρηματοδοτικό πρόγραμμα της εε για την έρευνα και καινοτομία. & 1304 & very low & Low & Socio-Economic & NA & NA & 2015-02-02 & 2015 & 1 & ECO
Frame & v.low & National & +1000 & 1.8924259 & 1.3590316 & -1.2391092 & 0.9814602 & -0.7922579 & 0.0 & -0.9049211 & -1.0736569 & Recipient & Domestic & Domestic & Domestic & Domestic|ECO & Neutral\\
\addlinespace
Greece & http://www.rodiaki.gr/article/379681/me-to-poso-twn-7-8-ek-eyrw-h-perifereia-enisxyei-tis-limenikes-ypodomes-twn-nhsiwn & 345 & Rodiaki.gr & Private/Non-Public & Online and Offline & Regional/Local & low = CP mentioned more times but NOT important part of story (mainly about others issues) & Infrastructure & Positive & Subnational & No myth & NA & NA & NA & NA & NA & NA & NA & NA & Greece & με το ποσό των 7,8 εκ. ευρώ, η περιφέρεια ενισχύει τις λιμενικές υποδομές των νησιών | η ροδιακη & 2017-12-07 & ευρωπαϊκό ταμείο περιφερειακής ανάπτυξης & το ποσό των 7,8 εκ. ευρώ για παρεμβάσεις βελτίωσης των λιμενικών υποδομών στα νησιά κυκλάδων και δωδεκανήσου, διαθέτει η περιφέρεια νοτίου αιγαίου, από το επιχειρησιακό πρόγραμμα "νότιο αιγαίο 2014-2020", με σκοπό την περαιτέρω βελτίωση των λιμενικών εγκαταστάσεων, την άρση της απομόνωσης των νησιών και την αναβάθμιση και τον εκσυγχρονισμό των παρεχομένων υπηρεσιών. στόχος είναι η ενίσχυση των λιμενικών πυλών εισόδου - εξόδου, ο σχεδιασμός και η υλοποίηση κατάλληλων δράσεων που θα συμβάλουν στην αντιμετώπιση των προβλημάτων σύνδεσης των νησιών με την υπόλοιπη περιφέρεια, καθώς και με τις πύλες εισόδου - εξόδου της περιφέρειας, με σκοπό την προώθηση της ισόρροπης χωρικής ανάπτυξης και την εξασφάλιση ισοδύναμης μεταφορικής εξυπηρέτησης των πολιτών. συγκεκριμένα, ο περιφερειάρχης νοτίου αιγαίου, γιώργος χατζημάρκος απευθύνει πρόσκληση με τίτλο "δράσεις αναβάθμισης των λιμενικών υποδομών δεδ-μ" συγχρηματοδοτούμενης δημόσιας δαπάνης ύψους 7.824.591 ευρώ, για την υποβολή προτάσεων έργων, προκειμένου να ενταχθούν και να χρηματοδοτηθούν στο πλαίσιο του άξονα προτεραιότητας "βελτίωση βασικών υποδομών", του επιχειρησιακού προγράμματος "νότιο αιγαίο 2014 - 2020", ο οποίος συγχρηματοδοτείται από το ευρωπαϊκό ταμείο περιφερειακής ανάπτυξης. η πρόσκληση απευθύνεται στους δυνητικούς δικαιούχους: 1. περιφέρεια νοτίου αιγαίου (διεύθυνση τεχνικών έργων κυκλάδων, διεύθυνση τεχνικών έργων δωδεκανήσου) 2. δήμους ρόδου, θήρας, μυκόνου, νάξου, πάρου και σύρου - ερμούπολης και 3. τα δημοτικά λιμενικά ταμεία νότιας δωδεκανήσου, θήρας, μυκόνου, νάξου, πάρου και σύρου οι δράσεις που θα χρηματοδοτηθούν στοχεύουν στην ανάπτυξη υφιστάμενων λιμενικών υποδομών της περιφέρειας που περιλαμβάνονται στο διευρωπαϊκό δίκτυο μεταφορών (δεδ-μ). συγκεκριμένα, για την ενίσχυση των λιμενικών πυλών εισόδου - εξόδου της περιφέρειας, θα χρηματοδοτηθεί η: · βελτίωση, επέκταση λιμενικών υποδομών, όπως κατασκευή προβλητών, κρηπιδωμάτων, κυματοθραυστών, προσαρμοσμένων στις απαιτήσεις των σύγχρονων μεταφορικών μέσων · υλοποίηση κτιριακών έργων λιμενικής ζώνης, όπως κατασκευή επιβατικών σταθμών, στεγάστρων και βοηθητικών κτιρίων · εγκατάσταση συστημάτων ασφαλείας λιμενικών εγκαταστάσεων η ημερομηνία έναρξης της υποβολής προτάσεων είναι η 1η/12/2017 και ημερομηνία λήξης η 31/01/2018. ως ελάχιστος προϋπολογισμός των υποβαλλομένων πράξεων ορίζεται το ποσό των 300.000 ευρώ. & 312 & low & Low & Socio-Economic & NA & NA & 2017-12-07 & 2017 & 2 & ECO
Frame & low-medium & Regional & <500 & 1.8924259 & 1.3590316 & -1.2391092 & 0.9814602 & -0.7922579 & 0.0 & -0.9049211 & -1.0736569 & Recipient & Domestic & Domestic & Domestic & Domestic|ECO & Positive\\
Greece & http://voria.gr/article/louketo-sta-psichodiagnostika-kentra-tou-kethea---se-stasi-ergasias-avrio & 319 & Ernst\&Young: Φρένο στις δημόσιες εγγραφές παγκοσμίως & Private/Non-Public & Online only & Regional/Local & medium = CP is important part of story & Ineffective goal achievement & Negative & National & No myth & NA & NA & NA & NA & NA & NA & NA & NA & Greece & λουκέτο στα ψυχοδιαγνωστικά κέντρα του κεθεα - σε στάση εργασίας αύριο & 2016-09-13 & ευρωπαϊκό κοινωνικό ταμείο & οι εργαζόμενοι διαμαρτύρονται καθώς μέχρι στιγμής δεν έχει βρεθεί τρόπος για τη συνέχιση λειτουργίας των ψυχοδιαγνωστικών κέντρων. σε στάση εργασίας προχωρούν αύριο, τετάρτη, οι εργαζόμενοι στο κεθεα, διαμαρτυρόμενοι για τη διακοπή λειτουργίας των ψυχοδιαγνωστικών κέντρων. όπως αναφέρουν σε σχετική ανακοίνωσή τους, "η παράταση λειτουργίας των ψυχοδιαγνωστικών κέντρων του οργανισμού φτάνει στο τέλος της στις 14 σεπτεμβρίου και μέχρι στιγμής δεν έχει βρεθεί, κάποιος τρόπος για τη συνέχιση της λειτουργίας τους και η όποια ένταξή τους σε νέο ευρωπαϊκό πρόγραμμα δεν είναι δυνατή σε αυτή τη χρονιά". η στάση εργασίας θα διαρκέσει από τις 12:00 έως τις 16:00. τα πολυδύναμα ψυχοδιαγνωστικά κέντρα κεθεα κλείνουν 3 χρόνια προσφοράς και εμπειρίας και ο σύλλογος εργαζομένων θεωρεί ότι τέτοιου είδους υπηρεσίες δεν μπορεί να διακόπτουν τη λειτουργία τους, όχι μόνο εξαιτίας των θέσεων εργασίας που χάνονται, αλλά πρωτίστως εξαιτίας των υπηρεσιών προς την κοινωνία που εξαλείφονται. ζητά "σταθερή υποστήριξη των ατόμων με διπλή διάγνωση. συνεχή λειτουργία των ψυχοδιαγνωστικών κέντρων του κεθεα. σεβασμό στο έργο των εργαζομένων". αναλυτικά η ανακοίνωση των εργαζομένων: "στις 15 σεπτεμβρίου 2016, τα πολυδύναμα ψυχοδιαγνωστικά κέντρα του κεθεα δεν θα λειτουργήσουν. τι γίνεται με τους πολλούς ανθρώπους που αντιμετωπίζουν προβλήματα εξάρτησης από παράνομες ουσίες, αλκοόλ, τζόγο ή διαδίκτυο και ταυτόχρονα με την εξάρτησή τους αντιμετωπίζουν σοβαρά προβλήματα ψυχικής υγείας; τι γίνεται όταν δεν υπάρχουν καν δομές και υπηρεσίες που να μπορούν να υποδεχτούν αυτές τις περιπτώσεις; πολύ περισσότερο, τι γίνεται όταν καλούνται οργανισμοί που έχουν ήδη τεράστια κενά, να καλύψουν και καινούργιες υπηρεσίες; θα μπορούσε κανείς να υποστηρίξει ότι σε μια οργανωμένη πολιτεία όλοι οι εμπλεκόμενοι ορίζουν μια διαδικασία για να υπάρχει μια απάντηση στα ερωτήματα και στις ανάγκες. επίσης, ότι καινούργιες δομές και πρωτοπόρες ιδέες ανοίγουν εναλλακτικούς δρόμους ως δοκιμασίες που θα καταλήγουν σε τελικές επιλογές για το ποιες υπηρεσίες ανταποκρίνονται στις πολλαπλές και εξειδικευμένες ανάγκες των ανθρώπων. σε κάθε περίπτωση, θα μπορούσε να διαμορφωθεί ένα σχέδιο δράσης και ένα σαφές πλαίσιο για το σχεδιασμό, την υλοποίηση, τον απολογισμό τέτοιων δράσεων. εκεί βρίσκεται και το βασικό πρόβλημα: μέχρι πού φτάνει η ματιά, όταν εντοπίζονται ανάγκες και αρχίζουν δράσεις. η "οργανωμένη" μας πολιτεία, παρουσιάζεται διαχρονικά ανίκανη να αλλάξει το εύρος ενός σχεδιασμού πέρα από τις βραχυπρόθεσμες "λύσεις" του προσωρινού και της υπόσχεσης. το keθεα αρχίζει από το 2013 να υλοποιεί τις εγκεκριμένες πράξεις "δημιουργία δικτύου πολυδύναμων ψυχοδιαγνωστικών κέντρων στα 2 μεγάλα αστικά κέντρα (αθήνα - θεσσαλονίκη) και στην περιφέρεια" (ηπείρου, θεσσαλίας, κρήτης, πελοποννήσου και β. αιγαίου), οι οποίες εντάσσονται στο θεματικό άξονα "εδραίωση της μεταρρύθμισης στον τομέα ψυχικής υγείας, ανάπτυξη της α' μιας φροντίδας υγείας και προάσπιση της δημόσιας υγείας του πληθυσμού στις 3 περιφέρειες σταδιακής εξόδου" του ε.π. "ανάπτυξη ανθρώπινου δυναμικού" 2007-2013 και συγχρηματοδοτούνται από το ευρωπαϊκό κοινωνικό ταμείο. εσπα, λοιπόν, με δυνατότητα απασχόλησης, 1 + 1 έτη για το προσωπικό. ως προς την ουσία, το κεθεα διευρύνει το δίκτυο υπηρεσιών του με νέες υπηρεσίες σε επτά πόλεις. περίπου 2.000 άτομα βρίσκουν ένα περιβάλλον στο οποίο μπορούν να έχουν υπηρεσίες που δεν τις βρίσκουν πουθενά αλλού: εξειδικευμένη υποστήριξη ατόμων με διπλή διάγνωση, που είναι η συνύπαρξη εξάρτησης από ουσίες, αλκοόλ, τζόγο ή διαδίκτυο με σοβαρή ψυχική διαταραχή. κέντρα που προσφέρουν ολιστική θεραπευτική αντιμετώπιση και για τις δύο διαταραχές στον ίδιο χώρο, σε περιβάλλον ασφάλειας και εμπιστευτικότητας. τα κέντρα στοχεύουν στην κινητοποίηση για θεραπεία, τη διακοπή της χρήσης ουσιών και τη σταθεροποίηση της αποχής, την πρόληψη της υποτροπής στη χρήση παράλληλα με την αντιμετώπιση ή σταθεροποίηση της ψυχικής διαταραχής και την πρόληψη της υποτροπής της. εντέλει, την ένταξη στο κοινωνικό περιβάλλον. ταυτόχρονα ένα καινούργιο, φρέσκο ανθρώπινο δυναμικό, στελεχώνει αυτές τις υπηρεσίες. εκπαιδεύεται, πασχίζει να ανταποκριθεί στα καθήκοντα του για τη φροντίδα των εξυπηρετούμενων, μαθαίνει τη δουλειά πλάι και μαζί με τον καθημερινό φόρτο και τις ελλείψεις του οργανισμού σε προσωπικό για τον βασικό κορμό των υπηρεσιών του. ορισμένου χρόνου, όμως, και το προσωπικό και το project, ενώ η δίχρονη λειτουργία των νέων, εξειδικευμένων δομών σε αθήνα, θεσσαλονίκη, ηράκλειο, λάρισα, γιάννενα, καλαμάτα και μυτιλήνη, ανέδειξε άμεσα την ανάγκη για μόνιμη και σταθερή υποστήριξη των ατόμων με διπλή διάγνωση. πού έφτανε, άρα, η περιβόητη "εδραίωση της μεταρρύθμισης στον τομέα ψυχικής υγείας"; πουθενά μα πουθενά. το καλοκαίρι του 2015 σταματάει το πρόγραμμα. όχι όμως και το φιλότιμο των εργαζομένων που προσπάθησαν, στο βαθμό που ήταν εφικτό, να συνεχίζουν κάποιες υπηρεσίες υπολειτουργώντας ορισμένα κέντρα. τον γενάρη του 2016 το κεθεα προχωρά στην επαναλειτουργία των 7 πολυδύναμων ψυχοδιαγνωστικών κέντρων με την πρόσληψη σαράντα δύο (42) ατόμων, θεραπευτικού προσωπικού, σύμφωνα με την από 17.7.2015 α.π.: διπαααδ/φ.εγκρ./63/21225 απόφαση της επιτροπής της πυς 33/2006, με σχέση εργασίας ιδιωτικού δικαίου ορισμένου χρόνου, για οκτώ μήνες και με δική του χρηματοδότηση. σε ανακοίνωση του κεθεα αναφέρεται ότι "πριν εκπνεύσει η 8μηνη παράταση της λειτουργίας τους είναι αναγκαίο και επείγον να διασφαλιστεί η συνέχιση της λειτουργίας τους, με την ένταξή τους στο σταθερό δίκτυο υπηρεσιών του κεθεα, ώστε να μην βρεθούν εκ νέου στο δρόμο οι άνθρωποι που λαμβάνουν υπηρεσίες." οκτώ μήνες μετά. οκτώ μήνες αβεβαιότητας και παράλληλα νέος θεραπευτικός σχεδιασμός με όρια συγκεκριμένα, με τη ματιά απορροφημένη στα υποκατάστατα λύσεων. η μόνιμη και σταθερή υποστήριξη των ανθρώπων όλο και απομακρύνεται από το οπτικό πεδίο. η χωρίς διακοπή παροχή υπηρεσιών φαντάζει αδύνατη. τη θέση της παίρνουν τα εσπα, τα μπλοκάκια, οι παρατάσεις συμβάσεων, η χρηματοδότηση από "άλλους" πόρους. αλλά ακόμα και σε αυτά παρουσιάζονται όρια και αδυναμία εξεύρεσης λύσεων. η παράταση των συμβάσεων δεν είναι δυνατή σύμφωνα με τους νομικούς. η εκ νέου χρηματοδότηση από εσπα απομακρύνθηκε για το επόμενο έτος, αφού περιφέρειες και υπουργείο δεν φαίνεται να τα βρίσκουν. ως σύλλογος εργαζομένων, καλούμε το υπουργείο υγείας και την διοίκηση του κεθεα να ανταποκριθούν στις ανάγκες των μελών που έχουν ανάγκη τις νέες υπηρεσίες και των εργαζομένων που δουλεύουν σε αυτές τις δομές, σύμφωνα με τις αρχές που έχουν οι ίδιοι επικαλεστεί, οι οποίες δεν είναι άλλες από την εξασφάλιση σταθερών υπηρεσιών για τους εξυπηρετούμενους και την επένδυση στο ανθρώπινο δυναμικό στο χώρο των εξαρτήσεων. καλούμε, δηλαδή, σε κατανόηση του βασικού ζητήματος, ότι, σε έναν χώρο θεραπείας, το προσωπικό είναι ο μοναδικός παραγωγικός συντελεστής και η ποιότητα δεν είναι προϊόν τεχνολογικών αυτοματισμών αλλά προσωπικής εμπλοκής των εργαζομένων. ατομικές συμβάσεις, μπλοκάκια, εργαζόμενοι πολλών ταχυτήτων, και υπηρεσίες με δόσεις δεν αποτελούν κατά τη γνώμη μας παρά μια "λύση ανάγκης" αναδυόμενη από τη λογική του προσωρινού και της υπόσχεσης που τρώει τα σωθικά μιας σπαραγμένης κοινωνίας. έχουν όμως εξαντληθεί τα περιθώρια ουσιαστικών λύσεων; ως σύλλογος εργαζομένων οφείλουμε να ενώσουμε τη φωνή μας και τις δυνάμεις μας με όλους αυτούς που θεωρούν ότι τέτοιου είδους υπηρεσίες δεν μπορεί να διακόπτουν τη λειτουργία τους όχι μόνο εξαιτίας των θέσεων εργασίας που χάνονται αλλά πρωτίστως εξαιτίας των υπηρεσιών προς την κοινωνία που εξαλείφονται. οι υπηρεσίες που προσφέρουν τα ψυχοδιαγνωστικά κέντρα δεν μπορούν να υποκατασταθούν από άλλες μονάδες του ήδη οριακά λειτουργούντος οργανισμού. οι δε εξυπηρετούμενοι από τα κέντρα δεν μπορούν επίσης να παραπεμφθούν σε άλλες δομές που να παρέχουν ίδιες ή έστω παρεμφερείς υπηρεσίες. ως σύλλογος εργαζομένων του κεθεα διεκδικούμε το αυτονόητο: να συνεχιστεί, μετά τις 14 σεπτέμβρη, χωρίς καμία διακοπή, η λειτουργία των πολυδύναμων ψυχοδιαγνωστικών κέντρων, να προσληφθεί με σταθερούς όρους εργασίας το απαραίτητο ανθρώπινο δυναμικό για τη διατήρηση υψηλού επιπέδου υπηρεσιών και να πραγματοποιηθεί η ένταξή τους στο οργανόγραμμα των υπηρεσιών του κεθεα. για να υπογραμμίσουμε τις θέσεις μας προκηρύσσουμε τετράωρη στάση εργασίας την τετάρτη 14 σεπτέμβρη 12:00 με 16:00 και καλούμε όλους τους συναδέλφους του κεθεα να λάβουν μέρος σε αυτή αγωνιστικά και μαζικά. ταυτόχρονα θα δοθεί συνέντευξη τύπου για το θέμα, στο χώρο του ψυχοδιαγνωστικού κέντρου αθήνας στην οδό θεμιστοκλέους 39, στις 13:00 και καλούμε τους εργαζόμενους να συμμετέχουν. ζητάμε σταθερή υποστήριξη των ατόμων με διπλή διάγνωση. συνεχή λειτουργία των ψυχοδιαγνωστικών κέντρων του κεθεα. σεβασμό στο έργο των εργαζομένων". & 1269 & medium & Medium & Socio-Economic & NA & NA & 2016-09-13 & 2016 & 2 & ECO
Frame & low-medium & Regional & +1000 & 1.8924259 & 1.3590316 & -1.2391092 & 0.9814602 & -0.7922579 & 0.0 & -0.9049211 & -1.0736569 & Recipient & Domestic & Domestic & Domestic & Domestic|ECO & Negative\\
Greece & http://www.newsit.gr/topikes-eidhseis/PSifiakes-ypiresies-sto-nosokomeio-Papageorgioy/648879 & 342 & newsit.gr & Private/Non-Public & Online only & National & very low = CP mentioned once & Infrastructure & Positive & Subnational & No myth & Public services & Positive & Subnational & No myth & NA & NA & NA & NA & Greece & ψηφιακές υπηρεσίες στο νοσοκομείο παπαγεωργίου - κόστισαν πάνω από ένα εκατομμύριο ευρώ & 2016-09-13 & ευρωπαϊκό ταμείο περιφερειακής ανάπτυξης & νοσηλευόμενοι ασθενείς βλέπουν σε οθόνες που υπάρχουν επάνω από τα κρεβάτια τους όχι μόνο ταινίες, αλλά και πληροφορίες για τις υπηρεσίες του νοσοκομείου, ενώ νοσηλευτές και γιατροί μπορούν από τις ίδιες οθόνες να βλέπουν τους φακέλους των ασθενών και επιπλέον, χάρη στην τεχνολογία παρακολούθησης έχουν τη δυνατότητα να εντοπίζουν κινητά ιατρικά μηχανήματα, που βρίσκονται σε οποιονδήποτε χώρο του νοσοκομείου. όλα αυτά συμβαίνουν σήμερα σε ελληνικό δημόσιο νοσοκομείο. πρόκειται για το "παπαγεωργίου", στη θεσσαλονίκη, το οποίο -όπως επισημαίνει στο απε-μπε ο διευθυντής της τεχνικής διεύθυνσης του, γρηγόρης σοφιαλίδης- "ίσως να είναι και το μόνο δημόσιο νοσοκομείο σε όλη την ελλάδα που διαθέτει αυτές τις πρωτοποριακές ψηφιακές υπηρεσίες, οι οποίες στοχεύουν στη βελτίωση της ποιότητας της ζωής των ασθενών και του προσωπικού". "σήμερα το "παπαγεωργίου" με τους 250 αναμεταδότες (access points) που έχουν τοποθετηθεί σε όλο το νοσοκομείο, παρέχει τη δυνατότητα wi-fi σύνδεσης σε περίπου 80\% της έκτασης των κτιρίων με άπειρες εφαρμογές. επίσης, χάρη στα rfid -σημεία παρακολούθησης πάγιου εξοπλισμού- παρέχεται στο προσωπικό η δυνατότητα παρακολούθησης και ελέγχου μηχανημάτων, έτσι ώστε ανά πάσα στιγμή να είναι γνωστό πού βρίσκεται ένα κινητό μηχάνημα π.χ. ένας καρδιογράφος. με τις 165 οθόνες που έχουν τοποθετηθεί επάνω από κρεβάτια νοσηλείας οι ασθενείς έχουν τη δυνατότητα να βλέπουν τηλεόραση ή βίντεο κατ' επιλογή, να ακούν μουσική και ραδιόφωνο και να αντλούν χρηστικές πληροφορίες. ταυτόχρονα, οι οθόνες μπορούν να χρησιμοποιούνται από το εξουσιοδοτημένο προσωπικό για άντληση/καταχώρηση πληροφοριών στον ιατρικό φάκελο. επιπλέον, έχουμε προχωρήσει στον εκσυγχρονισμό των τηλεπικοινωνιών του νοσοκομείου με τη χρήση αρχικά 100 ασύρματων ip τηλεφώνων, που χρησιμοποιούνται από το προσωπικό που -λόγω της φύσης της εργασίας του- πρέπει να μετακινείται σε διάφορους χώρους" αναφέρει ο κ. σοφιαλίδης. οι υπηρεσίες αυτές παρέχονται χάρη στο έργο "ψηφιακές υπηρεσίες προστιθέμενης αξίας προς τους νοσηλευόμενους και το προσωπικό του γ.ν.θ. παπαγεωργίου", το οποίο άρχισε στα τέλη του 2013 και η πιλοτική του εφαρμογή ολοκληρώθηκε το 2015. "επειδή οι ασθενείς που νοσηλεύονται στο νοσοκομείο περνούν λίγες μόνο ώρες με τους γιατρούς και τους νοσηλευτές και τις υπόλοιπες ώρες δεν έχουν τι να κάνουν, είχαμε την ιδέα να βάλουμε στο νοσοκομείο wi-fi ώστε να συνδέονται με το ίντερνετ και να περνάει ευχάριστα ο χρόνος τους. απ΄ αυτή την ιδέα ξεκίνησε το έργο "ψηφιακές υπηρεσίες προστιθέμενης αξίας προς τους νοσηλευόμενους και το προσωπικό του γνθ παπαγεωργίου", το οποίο χρηματοδοτήθηκε από πόρους του επιχειρησιακού προγράμματος "ψηφιακή σύγκλιση" και συγχρηματοδοτήθηκε από το ευρωπαϊκό ταμείο περιφερειακής ανάπτυξης της εε με συνολικό προϋπολογισμό 1.197.751 ευρώ. η σύμβαση υπογράφηκε το 2013, στα τέλη του 2014 ολοκληρώθηκε το φυσικό αντικείμενο του έργου και στα τέλη του 2015 η πιλοτική λειτουργία του" ανέφερε ο κ σοφιαλίδης. με αφορμή την ολοκλήρωση του έργου και στο πλαίσιο των συμβατικών υποχρεώσεων του νοσοκομείου διοργανώνεται στις 15 σεπτεμβρίου ενημερωτική ημερίδα, η οποία θα πραγματοποιηθεί στο αμφιθέατρο του παπαγεωργίου από τις 09.00 έως τις 15.00. & 472 & very low & Low & Socio-Economic & Socio-Economic & NA & 2016-09-13 & 2016 & 2 & ECO
Frame & v.low & National & <500 & 1.8924259 & 1.3590316 & -1.2391092 & 0.9814602 & -0.7922579 & 0.0 & -0.9049211 & -1.0736569 & Recipient & Domestic & Domestic & Domestic & Domestic|ECO & Positive\\
Greece & http://www.rodiaki.gr/article/361923/h-perifereia-notioy-aigaioy-exoplizei-ta-kentra-ygeias-kai-ta-polydynama-perifereiaka-iatreia-twn-mikrwn-nhsiwn & 392 & Rodiaki.gr & Private/Non-Public & Online and Offline & Regional/Local & very low = CP mentioned once & Infrastructure & Positive & Subnational & No myth & NA & NA & NA & NA & NA & NA & NA & NA & Greece & η περιφέρεια νοτίου αιγαίου εξοπλίζει τα κέντρα υγείας και τα πολυδύναμα περιφερειακά ιατρεία των μικρών νησιών | η ροδιακη & 2017-04-04 & ευρωπαϊκό ταμείο περιφερειακής ανάπτυξης & με το ποσό του 1.757.171 ευρώ, από το επιχειρησιακό πρόγραμμα "νότιο αιγαίο 2014 - 2020", η περιφέρεια νοτίου αιγαίου ενισχύει τον νευραλγικό όσο και κρίσιμο για τους νησιώτες τομέα της πρωτοβάθμιας φροντίδας υγείας, εξοπλίζοντας με σύγχρονα ιατροτεχνολογικά μέσα, 3 κέντρα υγείας και 21 πολυδύναμα περιφερειακά ιατρεία, στα μικρά νησιά της περιφέρειας. με απόφαση του περιφερειάρχη κ. γιώργου χατζημάρκου, η πράξη "προμήθεια εξοπλισμού για τις δομές πρωτοβάθμιας φροντίδας υγείας των μικρών νησιών της περιφέρειας νοτίου αιγαίου", εντάσσεται στον άξονα προτεραιότητας "ενίσχυση της περιφερειακής συνοχής", του επιχειρησιακού προγράμματος "νότιο αιγαίο 2014 - 2020". οι μονάδες υγείας που θα ωφεληθούν είναι τα κέντρα υγείας αμοργού, ίου και πάτμου καθώς και τα πολυδύναμα περιφερειακά ιατρεία ανάφης, αντιπάρου, αστυπάλαιας, δονούσας, ηρακλειάς, θηρασιάς, κάσου, κέας, κιμώλου, κουφονησίων, κύθνου, λειψών, μεγίστης, νισύρου, σερίφου, σικίνου, σίφνου, σύμης, σχοινούσας, φολεγάνδρου και χάλκης. η προμήθεια αφορά σε ιατροτεχνολογικό εξοπλισμό (όπως απινιδωτές, πιεσόμετρα, doppler αγγείων, καρδιοτοκογράφους, ψηφιοποίηση ακτινολογικών εργαστηρίων κλπ) λοιπό εξοπλισμό (όπως φορητούς και σταθερούς υπολογιστές, εκτυπωτές κλπ), απαραίτητο για την εξασφάλιση της εύρυθμης λειτουργίας των δομών πρωτοβάθμιας φροντίδας υγείας και σε έξι (6) ασθενοφόρα μικρού όγκου, κατάλληλα να επιχειρούν σε νησιά με οδικά δίκτυα και οικισμούς, που παρουσιάζουν ιδιαιτερότητες. το έργο συγχρηματοδοτείται από το ευρωπαϊκό ταμείο περιφερειακής ανάπτυξης, στο πλαίσιο του εσπα 2014 - 2020 και η επιλέξιμη δημόσια δαπάνη ανέρχεται σε 1757.171,72 ευρώ. ημερομηνία έναρξης της πράξης είναι η 10/07/2017 και λήξης η 31/12/2018. & 231 & very low & Low & Socio-Economic & NA & NA & 2017-04-04 & 2017 & 2 & ECO
Frame & v.low & Regional & <500 & 1.8924259 & 1.3590316 & -1.2391092 & 0.9814602 & -0.7922579 & 0.0 & -0.9049211 & -1.0736569 & Recipient & Domestic & Domestic & Domestic & Domestic|ECO & Positive\\
Greece & http://www.rodiaki.gr/article/389058/diplasioi-poroi-synolikoy-ypsoys-3-ekat-eyrw-diatithentai-apo-to-espa-gia-thn-prostasia-twn-aktwn-apo-thn-diabrwsh & 346 & Rodiaki.gr & Private/Non-Public & Online and Offline & Regional/Local & low = CP mentioned more times but NOT important part of story (mainly about others issues) & Infrastructure & Positive & Subnational & No myth & Environment/green/low-carbon & Positive & Subnational & No myth & Economic development & Positive & Subnational & No myth & Greece & διπλάσιοι πόροι, συνολικού ύψους 3 εκατ. ευρώ, διατίθενται από το εσπα, για την προστασία των ακτών από την διάβρωση | η ροδιακη & 2018-04-25 & ευρωπαϊκό ταμείο περιφερειακής ανάπτυξης & με απόφαση του περιφερειάρχη νοτίου αιγαίου, γιώργου χατζημάρκου, διπλασιάζονται οι πόροι που διατίθενται από το επιχειρησιακό πρόγραμμα "νότιο αιγαίο" 2014 - 2020" για παρεμβάσεις που σκοπό έχουν την προστασία των ακτών από την διάβρωση. πιο συγκεκριμένα, ο περιφερειάρχης υπέγραψε σήμερα την τροποποίηση της αρχικής πρόσκλησης, ο προϋπολογισμός της οποίας, ανερχόταν σε 1.500.000 ευρώ. η νέα πρόσκληση έχει συνολικό προϋπολογισμό 3.000.000 ευρώ, ενώ παράλληλα, παρατείνεται και η προθεσμία υποβολής προτάσεων για ένταξη, έως τις 29/06/2018. η απόφαση της περιφερειακής αρχής να διπλασιάσει το ποσό που διατίθεται μέσω του εσπα 2014 - 2020, αποτελεί καίριας σημασίας παρέμβαση για την προστασία των ακτών των νησιών του νοτίου αιγαίου, από το διαρκώς εντεινόμενο φαινόμενο της διάβρωσης και τις σοβαρές επιπτώσεις του, σε επίπεδο όχι μόνο περιβαλλοντικό, αλλά και κοινωνικοοικονομικό. δυνητικοί δικαιούχοι της πρόσκλησης "προστασία των ακτών της περιφέρειας από τη διάβρωση", είναι: * περιφέρεια νοτίου αιγαίου * δήμοι της περιφέρειας νοτίου αιγαίου * δημοτικά λιμενικά ταμεία περιφέρειας νοτίου αιγαίου για την υποβολή προτάσεων έργων, προκειμένου να ενταχθούν και να χρηματοδοτηθούν από το επιχειρησιακό πρόγραμμα "νότιο αιγαίο 2014 - 2020", στον άξονα προτεραιότητας "αειφορική ανάπτυξη και διαχείριση των πόρων", ο οποίος συγχρηματοδοτείται από το ευρωπαϊκό ταμείο περιφερειακής ανάπτυξης (ετπα). η επιλέξιμη συγχρηματοδοτούμενη δημόσια δαπάνη ανέρχεται σε 3.000.000 ευρώ. η διάβρωση των ακτών, αν και συνιστά φυσικό φαινόμενο, επιταχύνεται από την ανθρώπινη παρέμβαση. προκαλεί προβλήματα στις παράκτιες υποδομές και περιουσίες, εγκυμονώντας παράλληλα σοβαρούς κινδύνους για την ασφάλεια των παραλιακών δρόμων, των οικημάτων, αλλά και των ανθρώπων που προσεγγίζουν την ακτή, με συνέπεια σημαντικό κοινωνικοοικονομικό και περιβαλλοντικό κόστος, αλλά και επιπτώσεις στο τουριστικό προϊόν εν γένει, εξαιτίας της υποβάθμισης των παραλιών και των τουριστικών εγκαταστάσεων και καταλυμάτων. για τους λόγους αυτούς είναι ιδιαίτερα κρίσιμη η ανάληψη πρωτοβουλιών για την αντιμετώπιση του φαινόμενου της διάβρωσης ώστε να αποφευχθούν οι οποιεσδήποτε αρνητικές συνέπειες στο περιβαλλοντικό και οικονομικό περιβάλλον των διαφόρων περιοχών της περιφέρειας. η δράση αφορά στην υλοποίηση ήπιων έργων προστασίας ή και αναστροφής των φαινομένων διάβρωσης σε παραλίες των νησιών του νοτίου αιγαίου, λαμβάνοντας υπόψη και τις απαιτήσεις της ευρωπαϊκής οδηγίας πλαίσιο για τη θαλάσσια στρατηγική. ενδεικτικά μέτρα που μπορεί να ληφθούν είναι: εφαρμογή μεθόδων τεχνητής συμπλήρωσης της παραλίας τεχνικά έργα συγκράτησης υλικού ή εγκλωβισμού της άμμου και ελαχιστοποίησης των απωλειών, όπως κατασκευή κυματοθραυστών ή προβόλων ή τοποθέτηση ογκολίθων θωράκισης. στις προτεινόμενες δράσεις μπορεί να περιλαμβάνεται η εκπόνηση των οριστικών μελετών των τεχνικών έργων, στις περιπτώσεις που αυτές δεν έχουν εκπονηθεί. & 395 & low & Low & Socio-Economic & Socio-Economic & Socio-Economic & 2018-04-25 & 2018 & 3 & ECO
Frame & low-medium & Regional & <500 & 1.8924259 & 1.3590316 & -1.2391092 & 0.9814602 & -0.7922579 & 0.0 & -0.9049211 & -1.0736569 & Recipient & Domestic & Domestic & Domestic & Domestic|ECO & Positive\\
\addlinespace
Greece & https://www.rodiaki.gr/article/402832/ena-ekat-eyrw-gia-thn-ananewsh-toy-stoloy-twn-peripolikwn-ths-astynomias-sto-notio-aigaio & 364 & Rodiaki.gr & Private/Non-Public & Online and Offline & Regional/Local & medium = CP is important part of story & Infrastructure & Positive & EU + National + Subnational & No myth & NA & NA & NA & NA & NA & NA & NA & NA & Greece & ένα εκατ. ευρώ για την ανανέωση του στόλου των περιπολικών της αστυνομίας στο νότιο αιγαίο | η ροδιακη & 2018-11-12 & ευρωπαϊκό ταμείο περιφερειακής ανάπτυξης & με το ποσό του ενός εκατομμυρίου ευρώ (1.000.000 €), από το επιχειρησιακό πρόγραμμα "νότιο αιγαίο", στο πλαίσιο του εσπα 2014 - 2020, θα χρηματοδοτηθεί το υπουργείο εσωτερικών, για την ανανέωση του στόλου των περιπολικών της ελληνικής αστυνομίας, καθώς και για την προμήθεια ειδικού εξοπλισμού της, σε κυκλάδες και δωδεκάνησα. για το σκοπό αυτό, ο περιφερειάρχης νοτίου αιγαίου, γιώργος χατζημάρκος απευθύνει πρόσκληση προς το υπουργείο εσωτερικών, ως δυνητικό δικαιούχο, με τίτλο "έργα ενεργητικής οδικής ασφάλειας", για την υποβολή προτάσεων έργων, προκειμένου να ενταχθούν και να χρηματοδοτηθούν από το επιχειρησιακό πρόγραμμα "νότιο αιγαίο 2014 - 2020", στο πλαίσιο του άξονα προτεραιότητας 3 "βελτίωση βασικών υποδομών", ο οποίος συγχρηματοδοτείται από το ευρωπαϊκό ταμείο περιφερειακής ανάπτυξης (ετπα). το περιεχόμενο της πρόσκλησης έχει στόχο την βελτίωση της ενεργητικής ασφάλειας στο οδικό δίκτυο της περιφέρειας νοτίου αιγαίου, τη μείωση των τροχαίων ατυχημάτων, την αύξηση των ελέγχων και την εμπέδωση του αισθήματος της οδικής ασφάλειας, προβλέποντας δράσεις αναβάθμισης της ποιότητας και ασφάλειας υπηρεσιών μετακίνησης προσώπων και αγαθών και δράσεις προμήθειας εξοπλισμού για την τροχονομική αστυνόμευση του οδικού δικτύου. ενδεικτικά περιλαμβάνεται: ειδικός εξοπλισμός, έγχρωμα και συμβατικά επιβατικά οχήματα και συστήματα για συλλογή και διαχείριση δεδομένων τροχαίων ατυχημάτων, με στόχο την αναλυτική διερεύνηση των αιτιών που τα προκάλεσαν, καθώς και την αυξημένη αποδοτικότητα των υπηρεσιών τροχονομικής αστυνόμευσης της α΄ και β΄ διεύθυνσης αστυνομίας δωδεκανήσου και της διεύθυνσης αστυνομίας κυκλάδων. η συγχρηματοδοτούμενη δημόσια δαπάνη που θα διατεθεί για την ένταξη των πράξεων με την παρούσα πρόσκληση, ανέρχεται σε 1.000.000 €. οι προτάσεις υποβάλλονται μέσω του οπς εσπα, στην ηλεκτρονική διεύθυνση http://logon.ops.gr/ από την 12.11.2018 έως και την 31.12.2018 & 265 & medium & Medium & Socio-Economic & NA & NA & 2018-11-12 & 2018 & 3 & ECO
Frame & low-medium & Regional & <500 & 1.8924259 & 1.3590316 & -1.2391092 & 0.9814602 & -0.7922579 & 0.0 & -0.9049211 & -1.0736569 & Recipient & Domestic & European & Mixed & Domestic|ECO & Positive\\
Greece & https://e-thessalia.gr/ti-dimioyrgia-choroy-anapsychis-sti-nea-ionia-chrimatodotei-i-perifereia-thessalias/ & 298 & e-thessalia.gr & Private/Non-Public & Online only & Regional/Local & very low = CP mentioned once & Jobs & Positive & EU + Subnational & No myth & Economic development & Positive & EU + Subnational & No myth & NA & NA & NA & NA & Greece & tη δημιουργία χώρου αναψυχής στη νέα ιωνία χρηματοδοτεί η περιφέρεια θεσσαλίας - e-thessalia.gr & 2018-12-22 & ευρωπαϊκό ταμείο περιφερειακής ανάπτυξης & στη φάση της δημοπράτησης μπαίνει το έργο "διαμόρφωση δημοτικού χώρου στη συμβολή των οδών μαιάνδρου και καισαρείας στον βόλο" προϋπολογισμού 108.000,00 ευρώ μετά την έγκριση δημοπράτησης του έργου που δόθηκε από τον περιφερειάρχη θεσσαλίας κώστα αγοραστό. σε δηλώσεις του ο περιφερειάρχης θεσσαλίας κώστας αγοραστός τόνισε τα εξής: "σε συνεργασία με δήμο βόλου χρηματοδοτούμε και υλοποιούμε αναπλάσεις χώρων στις οποίες διεισδύουν οι επιθυμίες των κατοίκων της πόλης. πρόκειται για σημαντικά έργα, έργα ουσίας, που αλλάζουν την όψη των γειτονιών του βόλου και δημιουργούν καλύτερες συνθήκες ζωής. ενώνουμε τις δυνάμεις μας για να προσφέρουμε όλοι μαζί προς τους συνανθρώπους μας. παράλληλα με τα έργα αυτά διαχέονται κοινοτικά κονδύλια στην τοπική οικονομία και δημιουργούνται θέσεις εργασίας". το έργο συγχρηματοδοτείται από το ευρωπαϊκό ταμείο περιφερειακής ανάπτυξης και εθνικούς πόρους, μέσω του προγράμματος δημοσίων επενδύσεων. είναι ένα από τα έργα που προβλέπονται στην εγκεκριμένη στρατηγική β.α.α. δήμου βόλου. ακολουθεί πλέον η έγκριση και δημοσιοποίηση του διαγωνισμού από το δήμο βόλου. αντικειμενο εργου το έργο αφορά σε εργασίες για τη διαμόρφωση ακίνητου εμβαδού 490,20 m2, σε υπαίθριο δημόσιο χώρο στη νέα ιωνία και σε ακτίνα 500 μέτρων από το κέντρο της νέας ιωνίας και τον άξονα σύνδεσής της με το κέντρο του βόλου. ο προς διαμόρφωση χώρος περικλείεται από τις οδούς μαιάνδρου, καισαρείας και ρωμανού, ενώ η ανατολική του πλευρά σε όλο το μήκος της συνορεύει με μεσοτοιχία άλλης ιδιοκτησίας. εντός του οικοπέδου, βρίσκονται δύο κτίσματα: ένα κτίριο κατοικίας 77,73 τ.μ. και ένα κτίριο όπου στεγαζόταν παλαιότερα ο χώρος αναψυχής "καφενείο του πέτρου" εμβαδού 113,80τμ. τα κτίρια αυτά μαζί με την υφιστάμενη περίφραξη θα κατεδαφιστούν και στο κενό οικόπεδο θα διαμορφωθεί ένας υπαίθριος χώρος αστικού χαρακτήρα. οι εργασίες που θα γίνουν αφορούν σε καθαιρέσεις- αποξηλώσεις - κατεδαφίσεις - εκσκαφές, όπου απαιτείται, κατασκευή ενιαίας πλάκας υπόβασης και πλακόστρωση σύμφωνα με τη μελέτη. επίσης θα γίνουν εργασίες φύτευσης και άρδευσης και θα τοποθετηθεί ο απαραίτητος αστικός εξοπλισμός. τέλος, για το φωτισμό του χώρου θα τοποθετηθούν νέα φωτιστικά σώματα τύπου led. το έργο αφορά σε εργασίες που θα γίνουν για τη διαμόρφωση χώρου που βρίσκεται στη δημοτική ενότητα νέας ιωνίας του δήμου βόλου, στη συμβολή των οδών μαιάνδρου και καισαρείας, όπου στεγαζόταν παλαιότερα ο χώρος αναψυχής "καφενείο του πέτρου". στην πρόταση διαμόρφωσης περιλαμβάνεται η διαμόρφωση πλατείας, η φύτευση δέντρων σκίασης και η τοποθέτηση αστικού εξοπλισμού. με την κατασκευή του έργου αποδίδεται σε κοινή χρήση ένας κεντρικός χώρος, εντός του πυκνοκατοικημένου τμήματος της περιοχής της νέας ιωνίας με ταυτόχρονη αύξηση του πρασίνου στην περιοχή. & 410 & very low & Low & Socio-Economic & Socio-Economic & NA & 2018-12-22 & 2018 & 3 & ECO
Frame & v.low & Regional & <500 & 1.8924259 & 1.3590316 & -1.2391092 & 0.9814602 & -0.7922579 & 0.0 & -0.9049211 & -1.0736569 & Recipient & Domestic & European & Mixed & Domestic|ECO & Positive\\
Greece & https://www.in.gr/2019/01/16/greece/xyty-grammatikou-gia-dynamikes-antidraseis-proeidopoiei-dimotiki-koinotita/ & 375 & in.gr & Private/Non-Public & Online only & National & medium = CP is important part of story & Ineffective goal achievement & Balanced & EU + National + Subnational & NA & NA & NA & NA & NA & NA & NA & NA & NA & Greece & χυτυ γραμματικού: για δυναμικές αντιδράσεις προειδοποιεί η δημοτική κοινότητα | in.gr & 2019-01-16 & ταμείο συνοχής & με ψήφους 3 υπέρ και 1 κατά (του πρώην δημάρχου λιοσίων και μέλος της εκτελεστικής επιτροπής του εδσνα δημήτρη μπουραΐμη) μεταξύ των 4 παρόντων από τα 7 μέλη της εκτελεστικής επιτροπής, ψηφίσθηκε χθες από τον εδσνα (ειδικός διαβαθμιδικός σύνδεσμος νομού αττικής) η πρόταση για ετήσια δυναμικότητα του χυτυ γραμματικού 23.250 τόνων. σημειώνεται ότι σύμφωνα με το πεσδα (περιφερειακό σχέδιο διαχείρισης αποβλήτων) αττικής του 2016, η οεδα (ολοκληρωμένη εγκατάσταση διαχείρισης απορριμμάτων) γραμματικού θα διαχειρίζεται στη μεα (μονάδα επεξεργασίας απορριμμάτων) 60.000 τόνους ετησίως, η μεβα (μονάδα επεξεργασίας βιο-αποδομήσιμων) 20.000 τόνους ετησίως. στη συνέχεια, σύμφωνα με τη χθεσινή απόφαση, θα οδηγείται για διαχείριση και ταφή ποσότητα 23.000 τόνων στον χυτυ (χώρος υγειονομικής ταφής υπολειμμάτων) γραμματικού -- ποσότητα η οποία, όπως επισημάνθηκε αντιστοιχεί σε τέσσερεις ημέρες εναπόθεσης απορριμμάτων στον χυτα (χώρος υγειονομικής ταφής απορριμμάτων) φυλής. εξοργισμένη η τοπική κοινωνία το θέμα συζητήθηκε παρόντων εκπροσώπων της τοπικής κοινωνίας της δημοτικής κοινότητας γραμματικού και του δημάρχου μαραθώνα ηλία ψινάκη. η συζήτηση, στη διάρκεια της οποίας εκφράσθηκαν όλες οι απόψεις και αποσαφηνίσθηκαν θέματα που απασχολούν την τοπική κοινωνία, έγινε σε έντονο κλίμα. αυτό που κυριάρχησε ως κατακλείδα στις εισηγήσεις των δημοτικών συμβούλων και της δημοτικής αρχής μαραθώνα από όλες τις παρατάξεις που παραβρέθηκαν στη συνεδρίαση και μίλησαν ήταν ότι η απόφαση του εδσνα για την ετήσια δυναμικότητα ενός εργοστασίου που ακόμη δεν έχει κατασκευαστεί συνιστά για τους κατοίκους "αιτία πολέμου". μάλιστα, ο πρόεδρος της ένωσης συλλόγου γονέων δήμου μαραθώνα βαγγέλης κυπαρίσσης, τόνισε ότι οι 4.000 μαθητές των σχολείων του μαραθώνα έχουν ήδη από χθες προχωρήσει σε κατάληψη των σχολείων. οι κάτοικοι επισήμαναν ότι επιθυμούν να αναλάβουν τη διαχείριση των απορριμμάτων τους, αλλά ζητούν χρόνο για να αξιολογηθεί η πρόταση που έχουν ήδη καταθέσει ως δήμος μαραθώνα και για να προχωρήσουν σε περαιτέρω προτάσεις. χαρακτηριστική ήταν η αντίδραση του δημάρχου μαραθώνα ηλ. ψινάκη, ο οποίος μετά τη λήψη της απόφασης, σχολίασε: "δεν καταλαβαίνω την απόφασή σας. βγάζετε απόφαση για άδεια τραπεζοκαθισμάτων, χωρίς να έχει βγει ακόμη άδεια λειτουργίας καταστήματος". την ίδια στιγμή, κατά τη διάρκεια της συνεδρίασης, στο γραμματικό οι κάτοικοι, αίτημα των οποίων ήταν να μη ληφθεί απόφαση για ετήσια δυναμικότητα χυτυ, απέκλεισαν το χώρο και μπλόκαραν μέσα στον χυτα το απορριμματοφόρο που είχε το πρωί εναποθέσει υπολείμματα από το κδαυ (κέντρο διαλογής ανακυκλώσιμων υλικών) κορωπίου. ο οδηγός του φορτηγού, που μετέφερε τα υπολείμματα, κρατείτο έως χθες αργά το βράδυ στο αστυνομικό τμήμα μαραθώνα, ύστερα από παρέμβαση των κατοίκων, διότι μετέφερε υπολείμματα απορριμμάτων χωρίς παραστατικά. η ρίψη απορριμμάτων χωρίς παραστατικά "είναι νόμιμη σε δοκιμαστική λειτουργία" όπως εξήγησε κατά τη συνεδρίαση ο γενικός γραμματέας του εδσνα γιάννης δρίβας, η ρίψη απορριμμάτων δεν είναι παράνομη στο πλαίσιο δοκιμαστικής λειτουργίας, καθώς εμπεριέχεται στη σύμβαση του εργολάβου και προβλέπεται στο πεσδα. επίσης, εξήγησε ότι τα υπολείμματα που μεταφέρονται στο γραμματικό είναι του δήμου μαραθώνα και επισήμανε ότι κατά την κανονική λειτουργία της οεδα θα εξυπηρετείται μόνο μέρος των δήμων της ανατολικής αττικής. συγχρόνως, επισήμανε ο κ. δρίβας στο απε-μπε πριν από την έναρξη της συνεδρίασης, ο εδσνα έχει υπογράψει προγραμματική σύμβαση με τους δήμους μαρκοπούλου-μεσογαίας, ραφήνας-πικερμίου, σπάτων-αρτέμιδος και ραφήνας-πικερμίου για τη δημιουργία κοινής για τους τέσσερεις δήμους μεβα (μονάδας επεξεργασίας βιο-αποδομήσιμων), από την οποία θα παράγεται κομπόστ για την κάλυψη των τοπικών αγροτικών αναγκών και ίσως προς πώληση. πάντως, πηγές της περιφέρειας αττικής στηλίτευσαν το γεγονός ότι οι αρμόδιοι του εδσνα αυθαίρετα επέλεξαν να μεταφερθούν υπολείμματα από ιδιωτικό κδαυ (του κορωπίου) κατά τη δοκιμαστική φάση λειτουργίας και μάλιστα χωρίς να έχουν αρμοδιότητα για κάτι τέτοιο. ειδικότερα, στηλιτεύθηκε το γεγονός ότι ύστερα από παρέμβαση στελέχους του εδσνα μεταφέρθηκε υπόλειμμα και μάλιστα με ιδιωτικές συμβάσεις που πληρώνει ο εδσνα. και αυτό τη στιγμή που κάθε κδαυ υποχρεούται, με δικά του έξοδα και μέσα, να μεταφέρει υπολείμματα στον όποιο τελικό χώρο απόθεσης. οι αποφάσεις του στε και άλλων οργάνων της διοίκησης και οι επικρίσεις της ευρωπαϊκής επιτροπής άλλο θέμα αντιπαράθεσης: η θέση των ευρωπαϊκών οργάνων και της ελληνικής δικαιοσύνης απέναντι στο θέμα της καταλληλότητας του χώρου που έχει επιλεγεί για την εγκατάσταση της οεδα γραμματικού. όπως επισήμανε ο κ. δρίβας, έχουν υπάρξει πάμπολλες αποφάσεις που απορρίπτουν τις αιτιάσεις όσων επιχειρηματολογούν για δυο βασικά θέματα: τη σεισμικότητα της περιοχής και το γεγονός ότι η οεδα επιλέχθηκε να δημιουργηθεί σε περιοχές με ρέματα και άμεση γειτνίαση με περιοχές natura 2000. μάλιστα, τόνισε την ύπαρξη δυο αποφάσεων του συμβουλίου της επικρατείας (2007, 2011) που κρίνουν ως αβάσιμες τις περιβαλλοντικές ανησυχίες. από την άλλη πλευρά - των κατοίκων - τονίσθηκε ότι οι αποφάσεις του στε βασίζονται σε έκθεση του τότε γγ υδάτων (το 2012) κώστα τριάντη, ο οποίος αρνήθηκε την ύπαρξη ρέματος στην περιοχή. λίγο καιρό αργότερα, οι επιθεωρητές περιβάλλοντος στην έκθεσή τους τόνιζαν ότι η επιλογή της συγκεκριμένης περιοχής είναι περιβαλλοντικά επαχθής λόγω των ρεμάτων και επισημαίνουν την ακαταλληλότητά της. η έκθεση των επιθεωρητών περιβάλλοντος στάλθηκε στη συνέχεια (το 2014) στην εισαγγελία, η οποία επέβαλε πρόστιμο στην περιφέρεια αττικής. σε απάντηση που απέστειλε η επιτροπή, στην αναφορά (0573/2011) του κωνσταντίνου παπαδιγενόπουλου, που συνοδευόταν από 2.220 υπογραφές για τα σοβαρά σφάλματα και παραλείψεις στην έκθεση περιβαλλοντικών επιπτώσεων και τη συνεπαγόμενη παραβίαση της περιβαλλοντικής νομοθεσίας της εε, αναφέρεται μεταξύ άλλων: "όσον αφορά τη συμμόρφωση με την περιβαλλοντική νομοθεσία της εε, η επιτροπή θα ήθελε να υπενθυμίσει ότι, όπως αναφέρθηκε στις προηγούμενες ανακοινώσεις της, το ελληνικό συμβούλιο της επικρατείας έχει επανειλημμένα αποφανθεί υπέρ της συνέχισης του έργου, καθώς θεώρησε ότι το σχέδιο είναι σύμφωνο με το σύνολο της σχετικής νομοθεσίας της εε. μια από τις αποφάσεις αυτές εκδόθηκε μετά τη διερευνητική των πραγμάτων αποστολή που πραγματοποίησε στην ελλάδα η επιτροπή αναφορών τον σεπτέμβριο το 2013. η επιτροπή δεν έχει λάβει νέα αποδεικτικά στοιχεία μετά τις αποφάσεις του ελληνικού συμβουλίου της επικρατείας που θα έδειχναν παραβίαση της περιβαλλοντικής νομοθεσίας της εε". λίγο αργότερα, το 2015 και το 2018, η επιτροπή αναφορών του ευρωκοινοβουλίου, ύστερα από επιτόπια έρευνα και λήψη νέων στοιχείων, με επιστολές της προτείνει στις ελληνικές αρχές "δημοσιονομική διόρθωση 100\% της συνεισφορά του ταμείου συνοχής λόγων έλλειψης πληρότητας και λειτουργικότητας του σχεδίου". εν ολίγοις γνωστοποιεί στην ελλάδα ότι προτείνει την αναστολή της συγχρηματοδότησης από το ταμείο συνοχής που είχε γίνει για την περίοδο προγραμματισμού 2000-2006. τον μάρτιο 2015 οι ελληνικές αρχές είχαν απαντήσει ότι δεν αποδέχονται την πρόταση κλεισίματος από τις υπηρεσίες της επιτροπής. καθώς, παρά τις επανειλημμένες διαβεβαιώσεις το έργο δεν ολοκληρωνόταν, στις 29/05/2018 η επιτροπή απέστειλε στις ελληνικές αρχές επιστολή προδιόρθωσης, η οποία πρότεινε και πάλι δημοσιονομική διόρθωση 100\%. στις 30/04/2018 η επιτροπή αναφορών απέστειλε προς την περιφέρεια αττικής, τον δήμο μαραθώνα και το υπουργείο περιβάλλοντος, επιστολή στην οποία τονίζεται η ανησυχία της επιτροπής αναφορών για: την έλλειψη συμμόρφωσης προς τη νομοθεσία εε, την ανάγκη για δημόσια διαβούλευση (με την τοπική κοινωνία), τον χαρακτηρισμό κοντινού σημείου ως περιοχής natura 2000 (σέσι και σχινιάς) και την ανάγκη για νέα μελέτη περιβαλλοντικών επιπτώσεων (που ζητούσαν μετ' επιτάσεως να γίνει οι εκπρόσωποι του δήμου μαραθώνα, πριν ληφθεί απόφαση, κατά τη χθεσινή συνεδρίαση του εδσνα, χωρίς ανταπόκριση). στην επιστολή, όπως αναφέρει το απε, θίγονται θέματα που είχαν συζητηθεί κατά τη συνεδρίαση της επιτροπής στις 21/03/2018, κατά την οποία αποφασίσθηκε: να παραμείνει ανοικτή η αναφορά για τον χυτα (σ.σ. και όχι χυτυ) γραμματικού, να ζητηθεί η ενημερωμένη γνώμη της ευρωπαϊκής επιτροπής όσον αφορά στη μελέτη του ιγμε, το χαρακτηρισμό κοντινού σημείου ως natura 2000 και την ανάγκη για νέα μελέτη περιβαλλοντικών επιπτώσεων. σχόλια της επιτροπής αγώνα φυλής σχόλια για τη λήψη απόφασης έγιναν και από μέλη της επιτροπής αγώνα φυλής, που βρίσκονταν σε παρακείμενο σημείο του κτιρίου συνεδρίασης του εδσνα. τα σχόλια επικεντρώθηκαν στην απουσία της προέδρου του εδσνα και περιφερειάρχη αττικής ρ. δούρου και στην παρουσία του κ. μπουραΐμη, ο οποίος εξασφάλισε την απαρτία, που επέτρεψε τη λήψη απόφασης. τα μέλη της συντονιστικής επιτροπής αγώνα σχολίασαν ότι ο κ. μπουραΐμης καταψήφισε την έναρξη λειτουργίας του γραμματικού, ενώ ζητά να κλείσει ο χυτα φυλής και ότι σε περίπτωση που απουσίαζε, δεν θα μπορούσε να ληφθεί η απόφαση που,απ' ό,τι φαίνεται, ενδέχεται να προκαλεί σοβαρά προβλήματα στη δημοτική ενότητα γραμματικού. & 1317 & medium & Medium & Socio-Economic & NA & NA & 2019-01-16 & 2019 & 3 & ECO
Frame & low-medium & National & +1000 & 1.8924259 & 1.3590316 & -1.2391092 & 0.9814602 & -0.7922579 & 0.0 & -0.9049211 & -1.0736569 & Recipient & Domestic & European & Mixed & Domestic|ECO & Neutral\\
Greece & https://www.rodiaki.gr/article/413956/aporrifthhke-h-protash-gia-to-anoikto-kentro-emporioy & 302 & Rodiaki.gr & Private/Non-Public & Online and Offline & Regional/Local & high = CP is most important issue in story (can also cover other issues) & Bureaucracy and/or delays & Negative & EU + Subnational & No myth & NA & NA & NA & NA & NA & NA & NA & NA & Greece & απορρίφθηκε η πρόταση για το ανοικτό κέντρο εμπορίου | η ροδιακη & 2019-05-02 & ευρωπαϊκό ταμείο περιφερειακής ανάπτυξης & απορρίφθηκε η πρόταση που είχε καταθέσει ο δήμος ρόδου, με συνδικαιούχο το επιμελητήριο δωδεκανήσου, για το "ανοικτό κέντρο εμπορίου", με στόχο την ανάπλαση και την αισθητική αναβάθμιση του εμπορικού κέντρου της ρόδου. συγκεκριμένα, απορρίφθηκε η πράξη "ανοικτό κέντρο εμπορίου της ρόδου" στο επιχειρησιακό πρόγραμμα "ανταγωνιστικότητα επιχειρηματικότητα και καινοτομία 2014-2020" που χρηματοδοτείται από το ευρωπαϊκό ταμείο περιφερειακής ανάπτυξης (ετπα) και το ταμείο συνοχής. υπενθυμίζεται ότι στόχος του δήμου ήταν οι αναπλάσεις σε προσδιορισμένο σημείο του εμπορικού κέντρου, εγκατάσταση και ανάπτυξη συστημάτων έξυπνης πόλης καθώς και ενέργειες και δράσεις προβολής και προώθησης. η χρηματοδότηση ήταν της τάξεως του 1.614.777,75 ευρώ και μεταξύ άλλων η δράση θα περιελάμβανε: αναβάθμιση των πεζοδρόμων, ανάπλαση πλατειών, φωτισμό μνημείων και όψεων δημοσίων κτηρίων, παρεμβάσεις υψηλής σκίασης, κατασκευή και τοποθέτηση πινακίδων σήμανσης μνημείων και πολιτιστικών χώρων, συστήματα οδοφωτισμού, προμήθεια ηλεκτροκίνητων οχημάτων, προμήθεια και εγκατάσταση ομοιόμορφων στεγάστρων και τεντών κ.α. όπως αναφέρεται στο σκεπτικό της απόρριψης της πρότασης, που αναλύει η ειδική γραμματέας διαχείρισης τομεακών επ ετπα και τς, ευγενία φωτονιάτα, η πρόταση αξιολογήθηκε αρνητικά στο στάδιο α' που είναι "έλεγχος πληρότητας και επιλεξιμότητας πρότασης". ειδικότερα, με σχετικό έγγραφο ζητήθηκαν συμπληρωματικά στοιχεία και διευκρινίσεις τα οποία και υπέβαλε εμπρόθεσμα ο δήμος, δηλαδή υποβλήθηκε το σύνολο των απαιτούμενων δικαιολογητικών της πρότασης. ο δικαιούχος δήμος ρόδου επρόκειτο να υλοποιήσει τις υποχρεωτικές ενέργειες στις εξής κατηγορίες: * "αναβάθμιση δημόσιου χώρου" και * "προμήθεια και εγκατάσταση συστημάτων έξυπνης πόλης" ο συνδικαιούχος επιμελητήριο δωδεκανήσου επρόκειτο να υλοποιήσει υποχρεωτικά ενέργειες στην κατηγορία: * "ανάδειξη ταυτότητας εμπορικής περιοχής" και * "προβολή και προώθηση της εμπορικής περιοχής" από κοινού επρόκειτο να υλοποιηθούν ενέργειες στην κατηγορία: * "συμβουλευτικές υπηρεσίες για την υλοποίηση της πράξης για τον δικαιούχο/δαπάνες για αμοιβές προσωπικού για τον συνδικαιούχο". όπως αναφέρει η ειδική γραμματέας για το σκεπτικό της απόρριψης, σύμφωνα με την πρόσκληση, οι δαπάνες στην κατηγορία για την αναβάθμιση του δημόσιου χώρου δεν θα πρέπει να ξεπερνούν το 85\% του προϋπολογισμού του δικαιούχου, δηλαδή του δήμου ρόδου. όπως προκύπτει, όμως, από τα συμπληρωματικά στοιχεία που υποβλήθηκαν, ο προτεινόμενος προϋπολογισμός της κατηγορίας αυτής ανέρχεται στα 1.454.863,75 ευρώ, δηλαδή το ποσοστό επί του προϋπολογισμού που έχει στη διάθεσή του ο δήμος ανέχεται στο 96,99\% και ως εκ τούτου απορρίφθηκε. & 359 & high & High & Governance & NA & NA & 2019-05-02 & 2019 & 3 & POL
Frame & high-very high & Regional & <500 & 1.8924259 & 1.3590316 & -1.2391092 & 0.9814602 & -0.7922579 & 0.0 & -0.9049211 & -1.0736569 & Recipient & Domestic & European & Mixed & Domestic|POL & Negative\\
Greece & http://www.flashnews.gr/page.ashx?pid=3\&aid=187161\&cid=3 & 378 & flashnews.gr & Private/Non-Public & Online only & Regional/Local & very low = CP mentioned once & Infrastructure & Positive & EU + Subnational & NA & NA & NA & NA & NA & NA & NA & NA & NA & Greece & flashnews.gr - παράταση διακοπής κυκλοφορίας λόγω έργων της δευαχ & 2014-08-06 & ευρωπαϊκό ταμείο περιφερειακής ανάπτυξης & παρατείνεται η διάρκεια της διακοπής κυκλοφορίας και απαγόρευσης της στάθμευσης των οχημάτων λόγω της συνέχισης των εργασιών για την υλοποίηση του έργου "ολοκληρωση δικτυων ακαθαρτων δ.ε. νεας κυδωνιας δ. χανιων, υποεργο: δικτυα ακαθαρτων". πιο συγκεκριμένα, η διακοπή θα ισχύει μέχρι 31-08-2014, με σκοπό την ολοκλήρωση των εργασιών για την κατασκευή του δευτερεύοντος δικτύου αποχέτευσης ακαθάρτων (λυμάτων) στην περιοχή της παρηγοριάς και συγκεκριμένα στις οδούς ι. μπρεδάκη, α. καραφά, ιουστινιανού, δελφών και νότιος κόμβος οδού δελφών, καθώς επίσης και στην περιοχή του δαράτσου και συγκεκριμένα στις οδούς καντάνου, κουστογέρακου, πρεβελάκη, ισαάκ \& σολωμού. η κυκλοφορία θα διεξάγεται από τους παραπλεύρους δρόμους και θα υπάρχει υποχρεωτική σήμανση των οδών σύμφωνα με την ισχύουσα νομοθεσία. η δημοτική επιχείρηση ύδρευσης αποχέτευσης χανίων ζητά την κατανόηση τόσο των κατοίκων και επαγγελματιών της περιοχής, όσο και των οδηγών για τις οχλήσεις που θα υπάρξουν το αντίστοιχο διάστημα, όμως για λόγους ασφαλείας των διερχομένων αλλά και των εργαζομένων κατά τη διάρκεια εκτέλεσης του έργου, επιβάλλεται η τήρηση της σχετικής απαγόρευσης κυκλοφορίας. σημειώνεται ότι το έργο έχει συνολικό προϋπολογισμό σύμβασης 5.299.570,96 ευρώ (με φ.π.α.) και συγχρηματοδοτείται από το ευρωπαϊκό ταμείο περιφερειακής ανάπτυξης (ετπα) στο πλαίσιο του επιχειρησιακού προγράμματος κρήτης και νήσων αιγαίου 2007-2013, καθώς και από ίδιους πόρους της δ.ε.υ.α.χ. ενώ η διάρκεια υλοποίησής του ανέρχεται σε 18 μήνες και αναμένεται να έχει ολοκληρωθεί μέχρι το τέλος του 2015. αντικείμενο του έργου είναι η κατασκευή περίπου 28,5 χιλιομέτρων δικτύων αποχέτευσης ακαθάρτων, τα οποία αποσκοπούν τόσο στην εξυπηρέτηση των κατοίκων της ευρύτερης περιοχής, όσο και στη μείωση της ρύπανσης των θαλασσίων ακτών της περιοχής που είναι βραβευμένες με τη γαλάζια σημαία της ε.ε. και χαρακτηρίζονται από ιδιαίτερο φυσικό κάλλος και έντονο τουριστικό ενδιαφέρον. με την ολοκλήρωση του έργου, το οποίο θα δίδεται σε χρήση τμηματικά, θα έχουν τη δυνατότητα να συνδεθούν με το κεντρικό δίκτυο αποχέτευσης περίπου 2.300 ιδιοκτησίες και έτσι από το 55\% των ιδιοκτησιών που είναι συνδεδεμένες με την αποχέτευση σήμερα, να φτάσουμε στο 95\%, καλύπτοντας έτσι το σύνολο σχεδόν της νέας κυδωνίας. το σημαντικότατο αυτό έργο υποδομής θα συμβάλλει ουσιαστικά όχι μόνο στην προστασία της υγείας και την αναβάθμιση της ποιότητας ζωής των κατοίκων, αλλά και στην προστασία της ευρύτερης θαλάσσιας περιοχής και των ακτών της, ενισχύοντας με αυτό τον τρόπο τις περαιτέρω προσπάθειες όλων μας για την επίτευξη του στόχου της αειφόρου ανάπτυξης στον τόπο μας. & 396 & very low & Low & Socio-Economic & NA & NA & 2014-08-06 & 2014 & 1 & ECO
Frame & v.low & Regional & <500 & 1.8924259 & 1.3590316 & -1.2391092 & 0.9814602 & -0.7922579 & 0.0 & -0.9049211 & -1.0736569 & Recipient & Domestic & European & Mixed & Domestic|ECO & Positive\\
\addlinespace
Greece & http://voria.gr/article/charitsis-extra-primodotisi-espa-gia-nees-thesis-ergasias & 315 & Ernst\&Young: Φρένο στις δημόσιες εγγραφές παγκοσμίως & Private/Non-Public & Online only & Regional/Local & medium = CP is important part of story & Jobs & Positive & EU + National & No myth & Economic development & Positive & National & No myth & NA & NA & NA & NA & Greece & χαρίτσης: έξτρα πριμοδότηση εσπα για νέες θέσεις εργασίας & 2016-02-21 & ευρωπαϊκό κοινωνικό ταμείο & το νέο εσπα προβλέπει πριμοδότηση με επιπλέον 10\% επιχορήγηση στα επιχειρηματικά σχέδια που δημιουργούν νέες θέσεις εργασίας. πριμοδότηση με επιπλέον 10\% επιχορήγηση εκείνων των επιχειρηματικών σχεδίων που θα δημιουργήσουν θέσεις εργασίας, προβλέπεται στο νέο εσπα, όπως ανέφερε ο υφυπουργός οικονομίας, αρμόδιος για θέματα εσπα, αλέξης χαρίτσης, σε συνέντευξή του στην εφημερίδα ελεύθερος τύπος. δίνεται, επίσης, η δυνατότητα σε κοινωνικές και συνεταιριστικές επιχειρήσεις να υποβάλλουν προτάσεις πλέον ισότιμα με όλες τις άλλες μορφές επιχειρηματικότητας, συνέχισε ο υφυπουργός, ενώ καλλιεργείται η φορολογική και ασφαλιστική συνείδηση των επιχειρήσεων, με εισαγωγή κριτηρίων ώστε να καταπολεμηθεί η φοροδιαφυγή και η μαύρη εργασία. ένα ακόμα βασικό καινοτόμο στοιχείο στη φιλοσοφία των νέων προσκλήσεων, όπως είπε ο κ. χαρίτσης, είναι η "ρήτρα ευελιξίας των ταμείων", ώστε να χρηματοδοτούνται παρεμβάσεις αλληλοσυμπληρωματικού χαρακτήρα: απόκτηση/αναβάθμιση εξοπλισμού από το ευρωπαϊκό ταμείο περιφερειακής ανάπτυξης και κάλυψη μισθολογικού/ασφαλιστικού κόστους από το ευρωπαϊκό κοινωνικό ταμείο. νέα προγράμματα ο υφυπουργός υπογράμμισε ότι εκτός από τις τέσσερις προσκλήσεις νέων δράσεων του προγράμματος "ανταγωνιστικότητα, επιχειρηματικότητα και καινοτομία" που δημοσιεύτηκαν, σύντομα θα ανακοινωθεί δράση για την ενίσχυση και τον εμπλουτισμό των δομών και μηχανισμών στήριξης της νεοφυούς επιχειρηματικότητας (clusters, meta-clusters) για να προωθηθεί η δημιουργία εγχώριων αλυσίδων αξίας. "σύντομα θα ανακοινώσουμε δράση για την αναβάθμιση των επαγγελματικών προσόντων αυτοαπασχολουμένων και εργαζομένων, καθώς επίσης δράσεις για τη δημιουργία νέων τουριστικών επιχειρήσεων και την ενίσχυση μικρομεσαίων επιχειρήσεων που απασχολούν πάνω από 50 άτομα. συνολικά οι παραπάνω δράσεις θα έχουν προϋπολογισμό σχεδόν 150 εκατ. ευρώ. προγραμματίζονται, ακόμα, από το επιχειρησιακό πρόγραμμα περιβάλλοντος και μεταφορών, δράσεις ύψους περίπου 500 εκατ. ευρώ. επίσης, σε συνεργασία με το υπουργείο εργασίας θα προκηρύξουμε δράσεις για την κοινωνική και συνεταιριστική οικονομία, συνολικού προϋπολογισμού 136 εκατ. ευρώ" τόνισε ο υφυπουργός. νέα χρηματοδοτικά εργαλεία σχετικά με τα νέα χρηματοδοτικά εργαλεία, ο υφυπουργός ανέφερε ότι το 2016 συνεχίζεται η διάθεση υφιστάμενων προϊόντων στην αγορά, όπως το sme guarantee fund, το jeremie ict/vc, εργαλεία ετεαν (τεπιχ), και το προς διάθεση ποσό προσεγγίζει συνολικά τα 600 εκατ. ευρώ σε όλα τα προϊόντα. "μέχρι να μπει στην αγορά η νέα γενιά χρηματοδοτικών εργαλείων, οι επενδυτές μπορούν να κάνουν χρήση των ήδη υφιστάμενων" είπε ο κ. χαρίτσης και πρόσθεσε: "αφού μελετήσαμε προσεκτικά τα προβλήματα της προηγούμενης περιόδου, ανασχεδιάζουμε τα χρηματοδοτικά εργαλεία της νέας προγραμματικής περιόδου, στοχεύοντας στην καλύτερη δυνατή αξιοποίησή τους. ιδιαίτερη έμφαση θα δοθεί στις κεφαλαιουχικές συμμετοχές και τις μικροπιστώσεις, ενώ βεβαίως έχει προβλεφθεί η δημιουργία εργαλείων για δάνεια και εγγυήσεις. οι μεγάλες ανταποδοτικές υποδομές του jessica plus, το νέο "εξοικονομώ" (για κατοικίες, δημόσια κτίρια και επιχειρήσεις) είναι μερικά ακόμα σημαντικά εργαλεία. η αλλαγή του σχεδιασμού τους εστιάζει στη διεύρυνση των ομάδων των τελικών δικαιούχων, στην επέκταση στις καινοτόμες μικρομεσαίες επιχειρήσεις, στην αύξηση της χρήσης της συνεπένδυσης και της επιχειρηματικής συμμετοχής, στην ενίσχυση της τομεακής και περιφερειακής τους διάστασης, στο νέο καθεστώς των μικροπιστώσεων και στην ουσιαστική ενίσχυση της κοινωνικής οικονομίας". & 463 & medium & Medium & Socio-Economic & Socio-Economic & NA & 2016-02-21 & 2016 & 2 & ECO
Frame & low-medium & Regional & <500 & 1.8924259 & 1.3590316 & -1.2391092 & 0.9814602 & -0.7922579 & 0.0 & -0.9049211 & -1.0736569 & Recipient & Domestic & European & Mixed & Domestic|ECO & Positive\\
Greece & https://www.zougla.gr/page.ashx?pid=2\&cid=0\&aid=1674672 & 366 & zougla.gr & Private/Non-Public & Online only & National & very high = CP is most important issue + CP is mentioned in title/headline & Social justice & Positive & EU & No myth & Environment/green/low-carbon & Positive & EU + Subnational & No myth & Solidarity to poor countries/regions & Positive & EU + Subnational & No myth & Greece & καμίνης από τις βρυξέλλες: "η αθήνα στα δύσκολα χρόνια της κρίσης ένιωθε την ευρώπη δίπλα της" & 2019-02-07 & πολιτική συνοχής & κεντρικός ομιλητής σε εκδήλωση των ευρωσοσιαλιστών στις βρυξέλλες ήταν ο δήμαρχος αθηναίων γιώργος καμίνης ο οποίος απηύθυνε έκκληση για χάραξη μίας πολιτικής συνοχής που θα ενώνει τις νέες μορφωμένες γενιές, τους ανθρώπους που έχουν ανάγκη και τους οραματιστές που θα οικοδομήσουν το μέλλον, στον αντίποδα του ευρωσκεπτικισμού και της απογοήτευσης. στην εκδήλωση για το μέλλον της πολιτικής συνοχής της εε, που διοργάνωσαν οι σοσιαλιστές του ευρωπαϊκού κοινοβουλίου και της επιτροπής περιφερειών, ήταν παρών ο πρόεδρος της ομάδας των σοσιαλιστών στο ευρωπαϊκό κοινοβούλιο, ούντο μπούλμαν, ο οποίος εξήρε τη αντίδραση της αθήνας στις καταστροφικές φωτιές του καλοκαιριού στο μάτι και στο επιτυχημένο σχέδιο εκκένωσης των παιδικών κατασκηνώσεων, σημειώνοντας τον ρόλο των ηγεσιών στην τοπική διακυβέρνηση. "οι πόλεις έχουν τη δυνατότητα να είναι αποτελεσματικές, με την κατάλληλη ηγεσία η οποία οφείλει να δίνει το καλό παράδειγμα" είπε ο κ. μπούλμαν. στην εκδήλωση μετείχαν επίσης ο αμερικανός καθηγητής και οικονομολόγος τζέφρυ σάκς, η πρόεδρος των σοσιαλιστών στην επιτροπή περιφερειών κατιούσα μαρίνι, ο επίτροπος περιβάλλοντος καρμένου βέλλα, ευρωβουλευτές και εκπρόσωποι της τοπικής αυτοδιοίκησης, ενώ παρέστη και ο υποψήφιος των ευρωπαίων σοσιαλιστών για την προεδρία της ευρωπαϊκής επιτροπής, φρανς τίμερμανς. στην ομιλία του ο κ. καμίνης υπογράμμισε ότι "οικονομική ανάπτυξη, κοινωνική δικαιοσύνη και ισχυρή προστασία του περιβάλλοντος πρέπει να συμβαδίζουν, υποστηρίζοντας ότι "η οικογένεια των ευρωπαίων σοσιαλιστών και δημοκρατών σε αυτά τα ζητήματα πρέπει να βρίσκεται στην πρώτη γραμμή. ως δήμαρχος σε μια πολύ δύσκολη περίοδο για την αθήνα, πάντα είχα την ευρώπη, σύμμαχο. ένιωθα την εε στο πλευρό μου. εμβληματικά έργα όπως το πολιτιστικό και αθλητικό συγκρότημα του σεράφειου ή η δημοτική αγορά κυψέλης έγιναν με ευρωπαϊκούς πόρους" σημείωσε ο δήμαρχος αθηναίων, επισημαίνοντας ότι "πρέπει να κατευθύνουμε την πολιτική συνοχής σε ζητήματα που επηρεάζουν τους πολίτες, όπως η αειφόρος ανάπτυξη, η κοινωνική καινοτομία, η μετανάστευση, η ασφάλεια στις θέσεις εργασίας, η ανάπτυξη και η ευημερία, ζητήματα που βρίσκονται στον πυρήνα των ευρωπαϊκών αξιών", επισήμανε ο δήμαρχος αθηναίων. "πρέπει να ενισχύσουμε τη σχέση μας με τους εργαζομένους και τις συνδικαλιστικές οργανώσεις, τους εργοδότες και τις επιχειρήσεις, την κοινωνία των πολιτών, όπως μάλιστα αναφέρεται στην έκθεση για την αειφόρο ισότητα της ομάδας των ευρωπαίων σοσιαλιστών και δημοκρατών" σημείωσε στην ομιλία του ο κ. καμίνης. αναφερόμενος στο παράδειγμα της αθήνας, που προσφάτως τιμήθηκε από την ε.ε. με το βραβείο "ευρωπαϊκή πρωτεύουσα καινοτομίας για το 2018", ο δήμαρχος αθηναίων τόνισε ότι "από την αρχή της θητείας μας εμπιστευτήκαμε τους πολίτες, αναζητώντας, παράλληλα, νέους και πρωτότυπους συμμάχους. καταφέραμε να γίνουμε ο πρώτος δήμος στην ελλάδα με στρατηγική ενσωμάτωσης των μεταναστών όταν επενδύσαμε στη δημιουργία του κέντρου συντονισμού για θέματα μεταναστών και προσφύγων, προσφέροντας χώρο σε 93 διαφορετικούς οργανισμούς που ασχολούνται με την παροχή βοήθειας στους πρόσφυγες στην αθήνα, ενώ, μέσα από μια εκτενή δημόσια διαβούλευση με μια ολόκληρη γειτονιά της αθήνας, πετύχαμε τη δημιουργία ενός πρότυπου κέντρου κοινωνικής επιχειρηματικότητας στην δημοτική αγορά της κυψέλης". ειδικότερα, όσο αφορά στον ρόλο των πόλεων στην πολιτική αστικής ατζέντας της εε, προσέθεσε: "η νέα πολιτική συνοχής οφείλει να αναγνωρίζει και να αντανακλά την ανάδειξη της πόλης ως πολιτικού παράγοντα. οι πόλεις είναι έτοιμες και ώριμες να διαχειριστούν άμεσα τα κονδύλια της πολιτικής συνοχής, έτσι ώστε να δημιουργήσουν τα δικά τους σχέδια και να τα εφαρμόσουν σύμφωνα με τις δικές τους ανάγκες". & 529 & very high & High & Socio-Economic & Socio-Economic & Values & 2019-02-07 & 2019 & 3 & ECO
Frame & high-very high & National & 500-1000 & 1.8924259 & 1.3590316 & -1.2391092 & 0.9814602 & -0.7922579 & 0.0 & -0.9049211 & -1.0736569 & Recipient & European & European & European & European|ECO & Positive\\
Greece & http://voria.gr/article/spiraki-techniki-voithia-gia-aporrofisi-poron-apo-periferies & 297 & Ernst\&Young: Φρένο στις δημόσιες εγγραφές παγκοσμίως & Private/Non-Public & Online only & Regional/Local & very high = CP is most important issue + CP is mentioned in title/headline & Bureaucracy and/or delays & Positive & EU & No myth & Mismanagement & Negative & National & 8.Mismanaged & NA & NA & NA & NA & Greece & σπυράκη: τεχνική βοήθεια για απορρόφηση πόρων από περιφέρειες & 2017-04-25 & διαρθρωτικά ταμεία & την έκθεση με θέμα "οι μελλοντικές προοπτικές της τεχνικής βοήθειας στην πολιτική συνοχής" εισηγήθηκε η ευρωβουλευτής της νδ. οι τοπικές και περιφερειακές αρχές χρειάζονται τεχνική βοήθεια για να αυξήσουν τη χρήση των κοινοτικών πόρων και παρά το γεγονός ότι υπάρχουν διαθέσιμα κονδύλια για αυτό το σκοπό, το αποτέλεσμα είναι πενιχρό. το ευρωκοινοβούλιο με έκθεση με θέμα: "οι μελλοντικές προοπτικές της τεχνικής βοήθειας στην πολιτική συνοχής" προωθεί τεχνική βοήθεια που θα μπορούν να χρησιμοποιήσουν οι τοπικές και περιφερειακές αρχές με τη μέγιστη διαφάνεια. εισηγήτρια της έκθεσης που υπερψηφίστηκε από την επιτροπή περιφερειακής ανάπτυξης του ευρωπαϊκού κοινοβουλίου, από την πλευρά του ευρωπαϊκού λαϊκού κόμματος ήταν η ευρωβουλευτής της νδ και του ελκ μαρία σπυράκη. το ποσό για την τεχνική βοήθεια την περίοδο 2014-2020 ανέρχεται συνολικά στην ευρωπαϊκή ένωση σε 13,4 δισεκατομμύρια ευρώ, ωστόσο το ευρωπαϊκό κοινοβούλιο διαπιστώνει ότι υπάρχουν σοβαρά προβλήματα στον συντονισμό και τη διαφάνεια και πως συχνά η τεχνική βοήθεια δεν φτάνει στις τοπικές αρχές που την έχουν ιδιαίτερη ανάγκη για να αυξήσουν την αποτελεσματικότητα τους στην χρήση των κοινοτικών κονδυλίων. η έκθεση του ευρωπαϊκού κοινοβουλίου υπογραμμίζει τη σημασία της τεχνικής βοήθειας στον τομέα των χρηματοδοτικών εργαλείων λόγω της σύνθετης φύσης τους, στοιχείο που έχει εξαιρετική σημασία για την ελλάδα εξαιτίας της κρίσης ρευστότητας που αντιμετωπίζει. με τις τροπολογίες της κ. σπυράκη που ενσωματώθηκαν στην έκθεση, το ευρωπαϊκό κοινοβούλιο -ζητά από τα κράτη μέλη να δίνουν αποτελεσματικότερη πρόσβαση στην τεχνική βοήθεια στις τοπικές και περιφερειακές αρχές που συνήθως έχουν χαμηλή διοικητική δυνατότητα, -χαιρετίζει τις τεχνικές βελτιώσεις του ευρωπαϊκού κέντρου συμβούλων επενδύσεων (efsi advisory hub) για τον συνδυασμό των πόρων από τα διαρθρωτικά ταμεία με το σχέδιο γιούνκερ, ώστε οι κοινοτικοί πόροι να λειτουργούν συμπληρωματικά με περισσότερη αποτελεσματικότητα -επισημαίνει την ανάγκη, με βάση την έκθεση του ευρωπαϊκού ελεγκτικού συνεδρίου, για μεγαλύτερη προσοχή στα αποτελέσματα που είναι απαραίτητα ώστε η τεχνική βοήθεια να συμβάλει στην αποδοτικότητα των διαρθρωτικών αλλαγών στην ελλάδα. & 311 & very high & High & Governance & Governance & NA & 2017-04-25 & 2017 & 2 & POL
Frame & high-very high & Regional & <500 & 1.8924259 & 1.3590316 & -1.2391092 & 0.9814602 & -0.7922579 & 0.0 & -0.9049211 & -1.0736569 & Recipient & European & European & European & European|POL & Positive\\
Greece & http://voria.gr/article/komision-ependisis-ano-ton-13-dis-evro-se-10-erga-stin-ellada & 318 & Ernst\&Young: Φρένο στις δημόσιες εγγραφές παγκοσμίως & Private/Non-Public & Online only & National & very high = CP is most important issue + CP is mentioned in title/headline & Solidarity to poor countries/regions & Positive & EU & No myth & Economic development & Positive & EU + National & No myth & Economic development & Positive & EU & No myth & Greece & κομισιόν: επενδύσεις άνω των 1,3 δισ. ευρώ σε 10 έργα στην ελλάδα & 2017-03-24 & ευρωπαϊκά διαρθρωτικά και επενδυτικά ταμεία & αυτά τα έργα υποδομής, που συγχρηματοδοτούνται από την εε, προορίζονται να υποστηρίξουν μια στρατηγική ανάπτυξης, ανέφερε η αρμόδια επίτροπος. την ολοκλήρωση, εντός της τρέχουσας δημοσιονομικής περιόδου 2014-2020, των μεγάλων έργων που ξεκίνησαν κατά την περίοδο 2007-2013, με άμεσα οφέλη για τον ελληνικό λαό και για την πραγματική οικονομία της χώρας προωθεί η ευρωπαϊκή επιτροπή με μια επενδυτική προσπάθεια ύψους 1,3 δισ. ευρώ. "αυτά τα έργα υποδομής, που συγχρηματοδοτούνται από την εε, προορίζονται να υποστηρίξουν μια στρατηγική ανάπτυξης, υπό ελληνική ηγεσία, και αποτελούν απτή έκφραση της αλληλεγγύης της εε. πάνω από 35 δισ. ευρώ από τα κονδύλια της εε επενδύονται στη χώρα κατά τη δημοσιονομική περίοδο 2014-2020 και συνιστούν μια ισχυρή βάση για την επιστροφή στη διαρκή ευημερία. σε συνδυασμό με το ευρωπαϊκό ταμείο στρατηγικών επενδύσεων, τον πυρήνα του επενδυτικού σχεδίου του προέδρου juncker, μπορούν να αποτελέσουν επίσης πόλο έλξης περισσότερων ιδιωτικών επενδύσεων", δήλωσε η αρμόδια για την περιφερειακή πολιτική επίτροπος corina creţu. σχεδόν 51 εκατ. ευρώ για την ανάπτυξη υποδομών ευρυζωνικότητας στη χώρα αυτό το επενδυτικό πακέτο αποσκοπεί στην ευρυζωνική κάλυψη αγροτικών και απομακρυσμένων περιοχών της ελλάδας. το έργο αυτό συμβαδίζει με τους στόχους της ψηφιακής ενιαίας αγοράς και θα βοηθήσει να επιτευχθεί ο στόχος της συνδεσιμότητας για ολόκληρη τη χώρα μέχρι το 2020. με τον τρόπο αυτόν θα ενισχυθεί η καινοτομία και η επιχειρηματικότητα, θα αναπτυχθεί η ηλεκτρονική διοίκηση και οι ηλεκτρονικές υπηρεσίες και θα δοθεί στις αγροτικές κοινότητες η δυνατότητα να αναλάβουν κεντρικό ρόλο στη μετάβαση της ελληνικής οικονομίας προς την έξυπνη και βιώσιμη ανάπτυξη. 377 εκατ. ευρώ για μέσα μαζικής μεταφοράς στην αττική το αθηναϊκό μετρό έχει λάβει εντατική χρηματοδοτική υποστήριξη από την εε. πλέον θα έχουν επενδυθεί πάνω από 261 εκατ. ευρώ στην επέκταση της γραμμής 3 διαμέσου των δυτικών προαστίων της αθήνας και μέχρι τον πειραιά, την τρίτη μεγαλύτερη πόλη και τον πιο σημαντικό επιβατικό και εμπορευματικό λιμένα της ελλάδας. το έργο περιλαμβάνει την κατασκευή έξι νέων σταθμών του μετρό και την προμήθεια 17 αμαξοστοιχιών και αναμένεται να έχει τεθεί σε λειτουργία μέχρι το 2020 για να εξυπηρετεί άλλους 174.000 ακόμα κατοίκους της αθήνας. πάνω από 58 εκατ. ευρώ επενδύονται στην επέκταση της γραμμής του τραμ από το κέντρο της πόλης μέχρι τον λιμένα. το έργο αυτό περιλαμβάνει επίσης την κατασκευή 13 σταθμών του τραμ και την προμήθεια 25 βαγονιών και θα δώσει σε άλλους 11.000 ανθρώπους τη δυνατότητα να χρησιμοποιούν το δίκτυο του τραμ ώστε να μειωθεί η κυκλοφοριακή συμφόρηση στην πόλη. τέλος, πάνω από 18 εκατ. ευρώ επενδύονται στη σχεδίαση, την εγκατάσταση και τη λειτουργία ενός ολοκληρωμένου συστήματος αυτόματης είσπραξης εισιτηρίων στην αθήνα, που θα καλύπτει όλα τα δημόσια μέσα μεταφοράς στην ευρύτερη μητροπολιτική περιοχή. τα χάρτινα εισιτήρια θα αντικατασταθούν σταδιακά από ένα σύστημα ηλεκτρονικής επικύρωσης που θα διευκολύνει τους χρήστες στις διαδρομές τους και θα περιορίσει την εισιτηριοδιαφυγή. πάνω από 730 εκατ. ευρώ στην επέκταση του μετρό της θεσσαλονίκης η εε παρέχει χρηματοδοτική υποστήριξη στην κατασκευή ενός σύγχρονου αυτοοδηγούμενου συστήματος μετρό στην πόλη της θεσσαλονίκης για την τόνωση της τοπικής οικονομίας. πάνω από 407 εκατ. ευρώ θα έχουν επενδυθεί στην ολοκλήρωση της κύριας γραμμής μετρό της πόλης, συμπεριλαμβανομένων των εργασιών ανακαίνισης σε σταθμούς του μετρό, της κατασκευής νέων σηράγγων και της προμήθειας 24 αμαξοστοιχιών. αφού τεθεί σε λειτουργία, το 2020, η γραμμή του μετρό αναμένεται να εξυπηρετεί 247.000 επιβάτες ημερησίως και να συμβάλει στη μείωση της οδικής κυκλοφορίας και στη βελτίωση της ποιότητας του αέρα στην περιοχή της θεσσαλονίκης. επιπλέον, 323 εκατ. ευρώ περίπου θα δοθούν για την επέκταση του μετρό μέχρι την πόλη της καλαμαριάς, νοτιοανατολικά της θεσσαλονίκης. με τον τρόπο αυτό θα βελτιωθεί η καθημερινή ζωή σε αυτή την αστική περιοχή, ιδίως για τους εργαζόμενους που μετακινούνται καθημερινά μεταξύ καλαμαριάς και θεσσαλονίκης. σχεδόν 50 εκατ. ευρώ για τη βιώσιμη κινητικότητα στην πελοπόνησσο η σιδηροδρομική γραμμή αθηνών-πατρών συνδέει το λιμάνι της πάτρας με το λιμάνι του πειραιά και τον διεθνή αερολιμένα των αθηνών. αποτελεί έτσι έναν βασικό άξονα μεταφορών πάνω στον διάδρομο ανατολής/ανατολικής μεσογείου του βασικού διευρωπαϊκού δικτύου μεταφορών (δεδ-μ). αυτή η επένδυση ύψους 50 εκατ. ευρώ θα χρηματοδοτήσει εργασίες εκσυγχρονισμού στο τμήμα μεταξύ διακοφτού και ροδοδάφνης. το έργο αυτό θα ενισχύσει την οικονομική ανάπτυξη της περιοχής και θα αυξήσει την ποικιλία των διαθέσιμων μέσων μεταφοράς εμπορευμάτων και επιβατών μεταξύ των μεγάλων πόλεων και λιμένων της χώρας. 38 εκατ. για βελτίωση της συλλογής-επεξεργασίας λυμάτων στην περιφέρεια αττικής με το ποσό αυτό χρηματοδοτείται η κατασκευή ενός σύγχρονου αποχετευτικού δικτύου μήκους σχεδόν 100 χλμ. και μιας μονάδας επεξεργασίας λυμάτων στις πόλεις του κορωπίου και της παιανίας, νοτιοανατολικά της αθήνας, τα οποία καλύπτουν τα περιβαλλοντικά πρότυπα και τις απαιτήσεις που ισχύουν στην εε. αυτό το αναβαθμισμένο σύστημα διαχείρισης λυμάτων θα εξυπηρετεί σχεδόν 95.000 ανθρώπους. 92 εκατ. ευρώ για τη βελτίωση των οδικών συνδέσεων στη βόρεια ελλάδα πάνω από 64 εκατομμύρια ευρώ επενδύονται στην αναβάθμιση της δυτικής περιφερειακής οδού της θεσσαλονίκης και των παράπλευρων οδών, για καλύτερη και ταχύτερη οδική σύνδεση της πόλης με τα μεγάλα οδικά δίκτυα - την εγνατία οδό, τον αυτοκινητόδρομο πειραιά-αθηνών-θεσσαλονίκης-ευζώνων (παθε) και τον διάδρομο ανατολής/ανατολικής μεσογείου του δεδ-μ. τα έργα αυτά θα μειώσουν την κυκλοφοριακή συμφόρηση και έτσι οι κάτοικοι, ιδίως το ένα εκατομμύριο άνθρωποι που ζουν στις δυτικές συνοικίες της πόλης (αμπελόκηποι-μενεμένη, κορδελιό-εύοσμος και παύλου μελά), θα ωφεληθούν από τη μείωση της ατμοσφαιρικής ρύπανσης και της ηχορρύπανσης. επιπλέον, σχεδόν 28 εκατομμύρια ευρώ επενδύονται στην ολοκλήρωση και την αναβάθμιση του αυτοκινητόδρομου α29 (που αποτελεί επίσης μέρος του διαδρόμου ανατολής/ανατολικής μεσογείου), στο τμήμα μεταξύ κορομηλιάς και κρυσταλλοπηγής, στην περιφέρεια δυτικής μακεδονίας. τα έργα αυτά θα μειώσουν στο μισό τον χρόνο ταξιδιού μεταξύ των δύο πόλεων και θα βελτιώσουν σημαντικά την οδική ασφάλεια. ιστορικό τον ιούλιο του 2015 η ευρωπαϊκή επιτροπή παρουσίασε το πρόγραμμα για την απασχόληση και την ανάπτυξη στην ελλάδα, που αποσκοπεί στην πλήρη αξιοποίηση των ευρωπαϊκών διαρθρωτικών και επενδυτικών ταμείων από τη χώρα. κατά τη δημοσιονομική περίοδο 2014-2020 η ελλάδα θα λάβει από την εε κεφάλαια ύψους 35 δισ. ευρώ, στα οποία συμπεριλαμβάνονται 20 δισ. ευρώ από τα ευρωπαϊκά διαρθρωτικά και επενδυτικά ταμεία και 15 δισ. ευρώ από τα γεωργικά ταμεία. η ελλάδα έχει ωφεληθεί από μια εξαιρετικά εμπροσθοβαρή κατανομή των πόρων από τα προγράμματα της πολιτικής συνοχής, προκειμένου να δοθεί ώθηση στην επιλογή και την υλοποίηση έργων σε τοπικό επίπεδο, συμπεριλαμβανομένης και της δεύτερης φάσης πολλών μεγάλων έργων. & 1040 & very high & High & Values & Socio-Economic & Socio-Economic & 2017-03-24 & 2017 & 2 & ECO
Frame & high-very high & National & +1000 & 1.8924259 & 1.3590316 & -1.2391092 & 0.9814602 & -0.7922579 & 0.0 & -0.9049211 & -1.0736569 & Recipient & European & European & European & European|ECO & Positive\\
Greece & http://www.tovima.gr/politics/article/?aid=854478 & 348 & TO BHMA International & Private/Non-Public & Online and Offline & National & very low = CP mentioned once & Political capital/interests & Positive & National & No myth & NA & NA & NA & NA & NA & NA & NA & NA & Greece & tovima.gr - τσίπρας: το 2017 θα είναι έτος υψηλών ρυθμών ανάπτυξης & 2016-12-31 & διαρθρωτικά ταμεία & το 2017 θα είναι έτος υψηλών ρυθμών ανάπτυξης, τόνισε ο πρωθυπουργός αλέξης τσίπρας στο μήνυμά του για τη νέα χρονιά, σημειώνοντας ότι φέτος, η λέξη ανάπτυξη δεν είναι απλά ευχή. ο κ. τσίπρας έκανε αναφορά στους πολίτες που "δεν μπορούν να γιορτάσουν όπως οι υπόλοιποι" "σε αυτούς είναι που πηγαίνει το μυαλό μας. στους ανθρώπους που η οικονομική κρίση οδήγησε στο περιθώριο. στους άνεργους, τους χαμηλόμισθους, τους χαμηλοσυνταξιούχους. τους ανθρώπους που κάθε μέρα παλεύουν για την επιβίωση" σημείωσε, τονίζοντας ότι "πρωτίστως γι' αυτούς δίνουμε τη μάχη για να βγει η χώρα μας από τον ασφυκτικό κλοιό των μνημονίων και της επιτροπείας". όπως επεσήμανε ο κ. τσίπρας, "τη χρονιά που φεύγει καταφέραμε πολλά. όχι όσα θα θέλαμε, αλλά πολλά. σταθεροποιήσαμε την ελληνική οικονομία. περάσαμε επιτέλους σε θετικούς ρυθμούς ανάπτυξης. μειώσαμε την ανεργία κατά τρεις επιπλέον μονάδες. 2,5 εκατομμύρια συμπολίτες μας απόκτησαν πρόσβαση στο δημόσιο σύστημα υγείας. το πρόγραμμα για την ανθρωπιστική κρίση διευρύνθηκε. η ελλάδα από τελευταία, έγινε πρώτη χώρα στις 28 της εε στην απορροφητικότητα πόρων από τα διαρθρωτικά ταμεία". ο πρωθυπουργός έκανε λόγο για υπέρβαση των στόχων του προγράμματος "με σκληρή δουλειά" για πρώτη φορά εδώ και έξι χρόνια ενώ αναφέρθηκε στο επίδομα στους συνταξιούχους και στην αναστολή αύξησης του φπα στα νησιά που δέχονται το βάρος των προσφυγικών ροών, για το 2017. "φιλοδοξούμε το 2017 να είναι η χρονιά που η θα επιβεβαιώσει τον αναντικατάστατο ρόλο της ελλάδας, στο διεθνές και ευρωπαϊκό στερέωμα. αλλά και η χρονιά όπου θα εμπεδωθεί η εμπιστοσύνη και η σταθερότητα στην ελληνική οικονομία. και αυτό για πρώτη φορά μετά από 7 χρόνια κρίσης, δεν είναι απλά μια κοινότυπη πρωτοχρονιάτικη ευχή. αλλά μια ρεαλιστική προσδοκία. γιατί το 2017 θα είναι έτος υψηλών ρυθμών ανάπτυξης". ο κ. τσίπρας εκτίμησε ότι με το κλείσιμο της β΄ αξιολόγησης, ανοίγει ο δρόμος για την ποσοτική χαλάρωση και την πρόσβαση στις αγορές, ενώ χαρακτήρισε το 2017 "έτος των επενδύσεων", αλλά και "χρονιά ολοκλήρωσης των μεγάλων έργων υποδομής που για χρόνια λίμναζαν, συσσωρεύοντας βάρη στα κρατικά ταμεία, προς όφελος των κατεστημένων συμφερόντων". "για να πιάσει τόπο και να επουλώσει τις πληγές της κρίσης, χρειάζεται η ανάπτυξη που έρχεται να είναι ταυτόχρονα και δίκαιη. και είμαστε όλοι εμείς, που μαζί θα χαράξουμε το μονοπάτι της ανάπτυξης, με όρους αξιοπρέπειας, δικαιοσύνης και κοινωνικής προστασίας" διευκρίνισε. και πρόσθεσε: "το 2017, όμως, πέρα από χρονιά ανάκαμψης, θα είναι και το έτος των μεγάλων θεσμικών τομών και αλλαγών που έχει ανάγκη ο τόπος. μέσα στο 2017 θα ολοκληρώσουμε το διάλογο και θα φέρουμε προς ψήφιση στη βουλή τις προτάσεις για τη μεγάλη αναθεώρηση του συντάγματος. με στόχο την διεύρυνση και εμβάθυνση της δημοκρατίας. αλλά και την ενίσχυση της λαϊκής συμμετοχής". τόνισε, ακόμη ότι το 2017 θα είναι και το έτος των μεγάλων θεσμικών τομών και αλλαγών που έχει ανάγκη ο τόπος, ενώ προανήγγειλε μια "μεγάλη τομή" στο εκπαιδευτικό σύστημα εντός του έτους, "για να ξεφύγουμε επιτέλους από ένα σχολείο - εξεταστικό κέντρο", αλλά και αλλαγές στην υγεία και την τοπική αυτοδιοίκηση. "το 2017, τέλος, παλεύουμε, να είναι και έτος ορόσημο για τους εργαζόμενους στη χώρα μας, με την επαναφορά των συλλογικών συμβάσεων εργασίας. ώστε η ελλάδα να σταματήσει να αποτελεί θλιβερή εξαίρεση στην ευρώπη", πρόσθεσε και υπογράμμισε ότι, παρά τις προκλήσεις "για πρώτη φορά ο δρόμος έχει ορίζοντα. βλέπουμε πια το ξέφωτο. ξέφωτο στην οικονομία, στην κοινωνία αλλά και στα μεγάλα εθνικά μας θέματα". ". & 549 & very low & Low & Power & NA & NA & 2016-12-31 & 2016 & 2 & POL
Frame & v.low & National & 500-1000 & 1.8924259 & 1.3590316 & -1.2391092 & 0.9814602 & -0.7922579 & 0.0 & -0.9049211 & -1.0736569 & Recipient & Domestic & Domestic & Domestic & Domestic|POL & Positive\\
\addlinespace
Greece & http://avgi.gr/article/10842/9693924/g-dragasakes-kleidi-gia-ena-neo-ypodeigma-dikaies-kai-biosimes-anaptyxes-e-polyepipede-diakybernese & 305 & avgi.gr & Private/Non-Public & Online and Offline & National & medium = CP is important part of story & Social justice & Positive & National & No myth & Economic development & Positive & EU + National + Subnational & No myth & NA & NA & NA & NA & Greece & γ. δραγασάκης: κλειδί για ένα νέο υπόδειγμα δίκαιης και βιώσιμης ανάπτυξης η πολυεπίπεδη διακυβέρνηση & 2019-03-19 & περιφερειακή πολιτική & "από την πρώτη στιγμή, η κυβέρνηση αυτή έδωσε έμφαση στην άμεση επούλωση των πληγών και την καταπολέμηση των ανισοτήτων" "στην ελλάδα, η κρίση και οι οριζόντιες πολιτικές λιτότητας έπληξαν τόσο βαριά το σύνολο της χώρας, και σε όλα τα επίπεδα, που το πρόβλημα των ανισοτήτων γιγαντώθηκε. από την πρώτη στιγμή, η κυβέρνηση αυτή έδωσε έμφαση στην άμεση επούλωση των πληγών και την καταπολέμηση των ανισοτήτων". αυτό σημείωσε ο αντιπρόεδρος της κυβέρνησης και υπουργός οικονομίας και ανάπτυξης γιάννης δραγασάκης, σήμερα το πρωί, στην έναρξη της 4ης διυπουργικής συνόδου της επιτροπής για την περιφερειακή ανάπτυξη (rdpc) του οοσα, στην αθήνα. η σύνοδος, συνέχισε ο ίδιος, επικεντρώνεται σε μια από τις μεγαλύτερες προκλήσεις της εποχής μας, την καταπολέμηση των ανισοτήτων και την επίτευξη της ανάπτυξης χωρίς αποκλεισμούς. για ν' αντιμετωπίσουμε αποτελεσματικά τις ανισότητες, όπως είπε ο κ. δραγασάκης, δημιουργήσαμε και διαμορφώνουμε: - εθνική αναπτυξιακή στρατηγική - θεσμικό πλαίσιο και εργαλεία σχεδιασμού - πλέγμα στοχευμένων χρηματοδοτικών εργαλείων και ταμείων - θεσμούς και διαδικασίες διαβούλευσης, λογοδοσίας και αξιολόγησης. "κλειδί για ένα νέο υπόδειγμα δίκαιης και βιώσιμης ανάπτυξης, που θ' ανταποκρίνεται στις ανάγκες του μέλλοντος και τις εξελισσόμενες μεγατάσεις, είναι η πολυεπίπεδη διακυβέρνηση, με συνεργασία όλων των επιπέδων διοίκησης. γι' αυτό ενισχύουμε θεσμικά και χρηματοδοτικά την αυτοδιοίκηση" είπε ο υπουργός. υπενθυμίζεται ότι η σύνοδος πραγματοποιείται ανά πέντε χρόνια σε διαφορετικό κράτος - μέλος του οοσα και στη φετινή συμμετέχουν 20 επικεφαλής διαφόρων παραγωγικών υπουργείων από όλον τον κόσμο, ενώ θα εκπροσωπηθούν ακόμα 30 χώρες και οργανισμοί με δικές τους αντιπροσωπείες. όπως σημειώνεται σε ενημερωτικό υλικό του υπουργείου οικονομίας σχετικά με τη σύνοδο, η ελληνική πλευρά βρίσκεται σύμφωνη με την εκτίμηση ότι η περιφερειακή πολιτική είναι η "πολιτική των πολιτικών", στο μέτρο που σε περιφερειακό και τοπικό επίπεδο εφαρμόζονται οι τομεακές πολιτικές με ολοκληρωμένο τρόπο σε άμεση διασύνδεση με τον πολίτη. "στη διάρκεια των είκοσι ετών λειτουργίας της rdpc τέθηκαν βασικά ερωτήματα που απασχολούν την επίτευξη μιας ισόρροπης και δίκαιης ανάπτυξης κι έχουν, πλέον, ωριμάσει οι πολιτικές προβληματισμού (think tank) για τη μετάβαση σε πολιτικές εφαρμογής (do tank). η ελλάδα κατεξοχήν χώρα ιδιαίτερης γεωμορφολογικής πολυπλοκότητας και σύνθετης γεω-στρατηγικής και γεωπολιτικής θέσης έχει ανάγκη πολιτικών απολύτως προσαρμοσμένων σε περιφερειακό και τοπικό επίπεδο" όπως υπογραμμίζεται. "μεγατάσεις: χτίζοντας ένα καλύτερο μέλλον" το ευρύτερο θέμα της συνόδου θα είναι "μεγατάσεις: χτίζοντας ένα καλύτερο μέλλον". οι μεγατάσεις είναι μακροπρόθεσμης διάρκειας κινούσες δυνάμεις, οι οποίες είναι παρούσες σήμερα σε παγκόσμιο επίπεδο και θα είναι, κατά πάσα πιθανότητα, παρούσες και στο μέλλον. σημειώνεται ότι το megatrends hub της ευρωπαϊκής επιτροπής καταγράφει 14 μεγατάσεις μεταξύ των οποίων είναι η επιταχυνόμενη τεχνολογική αλλαγή και η υπερσυνδεσιμότητα, οι αυξανόμενες δημογραφικές ανισορροπίες, η αυξανόμενη σημασία της μετανάστευσης, η συνεχιζόμενη αστικοποίηση, η κλιματική αλλαγή, η υποβάθμιση του περιβάλλοντος και η εξάντληση των πόρων. σε αυτές θα πρέπει να προστεθεί βεβαίως και η παγκοσμιοποίηση. οι μεγατάσεις δεν είναι μεταξύ τους ανεξάρτητες. για παράδειγμα, η μετανάστευση, η οποία είναι δυνατό να προέλθει και από τις επιπτώσεις της κλιματικής αλλαγής και την εξάντληση των φυσικών πόρων, συμβάλλει στην αστικοποίηση ενώ επηρεάζει τα δημογραφικά χαρακτηριστικά των περιοχών υποδοχής (εγκατάστασης). σε ό,τι αφορά στην ελλάδα, όπως σημειώνεται από το υπουργείο οικονομίας, οι προκλήσεις είναι πολλές και για το λόγο αυτό προωθούνται ολοκληρωμένες πολιτικές για την αντιμετώπισή τους όπως το εθνικό σχέδιο δράσης για την ενέργεια και το κλίμα, το εθνικό επιχειρησιακό σχέδιο για την κυκλική οικονομία και η ολιστική αναπτυξιακή στρατηγική. στο ίδιο πλαίσιο εντάσσεται και η ανάθεση στον οοσα της μελέτης με τίτλο "περιφερειακή πολιτική για την ελλάδα μετά το 2020", η οποία εκπονείται στο πλαίσιο της προετοιμασίας της χώρας εν όψει της προγραμματικής περιόδου 2021-2027. πηγή: απε - μπε & 584 & medium & Medium & Socio-Economic & Socio-Economic & NA & 2019-03-19 & 2019 & 3 & ECO
Frame & low-medium & National & 500-1000 & 1.8924259 & 1.3590316 & -1.2391092 & 0.9814602 & -0.7922579 & 0.0 & -0.9049211 & -1.0736569 & Recipient & Domestic & Domestic & Domestic & Domestic|ECO & Positive\\
Greece & http://voria.gr/article/magkriotissti-thessaloniki-evreche-anaptixi-i-poli-omos-den-iche-nero & 301 & Ernst\&Young: Φρένο στις δημόσιες εγγραφές παγκοσμίως & Private/Non-Public & Online only & National & low = CP mentioned more times but NOT important part of story (mainly about others issues) & Bureaucracy and/or delays & Negative & National & No myth & NA & NA & NA & NA & NA & NA & NA & NA & Greece & μαγκριώτης:στη θεσσαλονίκη έβρεχε ανάπτυξη, η πόλη όμως δεν είχε νερό & 2018-03-30 & ταμείο συνοχής & ο αγωγός της αραβησσού ξεπέρασε τα όρια ζωής του και θα "σκάνε" κάθε λίγο παρόμοια προβλήματα, τονίζει ο πρώην υφυπουργός υποδομών γιάννης μαγκριώτης στο συνέδριο για την ανάπτυξη της κεντρικής μακεδονίας, που οργάνωσε η κυβέρνηση με την περιφέρεια, έβρεχε δισεκατομμύρια ευρώ από τους υπουργούς, τον περιφερειάρχη και τον πρωθυπουργό. τις ίδιες ημέρες η περιοχή του πολεοδομικού συγκροτήματος θεσσαλονίκης δεν είχε νερό∙ το πρόβλημα συνεχίζεται, όμως κανείς από τους αρμοδίους δεν ένιωσε την ανάγκη να πει μια κουβέντα, γιατί; αποσιώπησαν το πρόβλημα γιατί οι σύμβουλοι επικοινωνίας τούς είπαν πως όποιος μιλήσει θα το αναδείξει και πρέπει κάποιοι να το χρεωθούν πολιτικά. άθελά του, ο αναπληρωτής υπουργός περιβάλλοντος κ. φάμελλος, χωρίς να είναι ο αρμόδιος, αφού οι αρμόδιοι κρύφτηκαν, κατά την επίσκεψή του στο σημείο του κατεστραμμένου αγωγού, αποκάλυψε το πρόβλημα και είπε: "έχει εντοπιστεί η βλάβη στον υφιστάμενο, περίπου 40 χρόνων, αγωγό. αυτό που επείγει και συζητήσαμε με τον πρόεδρο της ευαθ είναι η επέκταση εγκατάστασης επεξεργασίας νερού, έτσι ώστε το νερό του αλιάκμονα να καλύπτει 100\% τις ανάγκες και να χρησιμοποιούμε το νερό της αραβησσού μόνο σε έκτακτες ανάγκες. το έργο είναι πλήρως μελετημένο και έχει υποβληθεί προς χρηματοδότηση στην περιφέρεια κεντρικής μακεδονίας από τους κοινοτικούς πόρους που της έχουν διατεθεί". άκουσε κανείς να γίνεται καμία αναφορά από υπουργό, πρωθυπουργό και περιφερειάρχη, στο συνέδριο, για ένα έργο τόσο ζωτικής σημασίας, την ώρα μάλιστα που οι κάτοικοι, τα νοσοκομεία, τα σχολεία, οι επαγγελματίες και οι επιχειρήσεις δοκιμάζονταν; οικειοποιήθηκαν έργα άλλων και εξήγγειλαν έργα δισεκατομμυρίων, για τα οποία ούτε μελέτες έχουν ούτε πόρους, και φυσικά χωρίς ορίζοντα έναρξης της κατασκευής τους τα επόμενα δέκα χρόνια, και για ένα έργο τόσο αναγκαίο, που είναι μελετημένο και έτοιμο για δημοπράτηση, δεν είπαν λέξη. γιατί όμως το έργο αυτό δεν έχει γίνει μέχρι σήμερα, αφού τα προβλήματα με την άντληση και παροχέτευση νερού από την αραβησσό είναι μεγάλα και έχουν αναδειχτεί τουλάχιστον εδώ και δεκαπέντε χρόνια; ποιος δεν θυμάται τα προβλήματα που δημιουργούσε στο περιβάλλον, στις καλλιέργειες και στη σταθερότητα του εδάφους στην περιοχή της αραβησσού, που είχε ξεσηκώσει τις διαμαρτυρίες των κατοίκων της και των οικολογικών οργανώσεων; η κυβέρνηση της νδ το 2008, όπως θυμούνται όλοι, προκήρυξε τον διαγωνισμό για την ιδιωτικοποίηση της ευαθ και δεν ασχολήθηκε με το πρόβλημα που ήταν ήδη έντονο. ο διαγωνισμός αυτός ακυρώθηκε από την κυβέρνηση του πασοκ το 2009, γιατί ήμασταν ενάντια στην ιδιωτικοποίηση του νερού και προχωρήσαμε στη μελέτη του εκσυγχρονισμού και της επέκτασης του υφιστάμενου διυλιστηρίου που επεξεργάζεται το νερό από τον αλιάκμονα. στο αρχές του 2011 δημοπρατήθηκε το έργο με προϋπολογισμό 45 εκατ. ευρώ από το ταμείο συνοχής της εε και το τομεακό πρόγραμμα του υπεκα∙ στο τέλος του 2011 υπήρχε μειοδότης ανάδοχος. η νέα κυβέρνηση, που ανέλαβε μετά τις εκλογές του 2012, ακύρωσε τον διαγωνισμό για λόγους που ποτέ δεν δημοσιοποιήθηκαν. από τότε μέχρι σήμερα κανείς δεν ασχολήθηκε, όπως κάνουν και τώρα. το έργο θα είχε τελειώσει το 2014 και η θεσσαλονίκη θα είχε υπερκάλυψη για πενήντα χρόνια με 300.000 κυβικά μέτρα νερό της καλύτερης ποιότητας ημερησίως και θα μπορούσε να εντάξει και τις περιαστικές περιοχές της θεσσαλονίκης, ιδιαίτερα τις βιομηχανικές και τις τουριστικές. και τώρα θα ξεδιπλωθεί το γαϊτανάκι της αλληλομετάθεσης και της απόκρυψης των ευθυνών ενώ η πόλη, οι άνθρωποί της και η κάθε οικονομική και κοινωνική δραστηριότητα θα πλήττεται από τις αβελτηρίες αρμοδίων κάθε απόχρωσης τα τελευταία οκτώ χρόνια. ο αγωγός της αραβησσού ξεπέρασε τα όρια ζωής του και θα "σκάνε" κάθε λίγο παρόμοια προβλήματα. ευτυχώς που δεν ακύρωσαν και το δεύτερο έργο περίπου 38 εκατ. ευρώ, τον β΄ κλάδο του κεντρικού αποχετευτικού αγωγού, που θα παίρνει αστικά λύματα από το κέντρο και τη δυτική θεσσαλονίκη για τον βιολογικό της σίνδου, αποτρέποντας μεγάλες ποσότητες να πέφτουν στον θερμαϊκό στην πρώτη νεροποντή από υπερχειλίσεις του υπάρχοντος αγωγού, συμβάλλοντας στη μόλυνσή του και την εμφάνιση του φυτοπλαγκτόν. βέβαια αντί να τελειώσει το 2014 , ακόμη δεν έχει ολοκληρωθεί και οι αρμόδιοι δεν δίνουν χρονοδιάγραμμα ολοκλήρωσής του. υποσχέσεις δισεκατομμυρίων έβρεξε στη θεσσαλονίκη και από πολλούς∙ το νερό όμως μας έλειπε και συχνά θα μας λείπει αν δεν σοβαρευτούν οι αρμόδιοι. *ο γιάννης μαγκριώτης είναι συντονιστής της πρωτοβουλίας για τη νέα σοσιαλδημοκρατία και πρώην υφυπουργός υποδομών του πασοκ & 683 & low & Low & Governance & NA & NA & 2018-03-30 & 2018 & 3 & POL
Frame & low-medium & National & 500-1000 & 1.8924259 & 1.3590316 & -1.2391092 & 0.9814602 & -0.7922579 & 0.0 & -0.9049211 & -1.0736569 & Recipient & Domestic & Domestic & Domestic & Domestic|POL & Negative\\
Greece & http://www.efsyn.gr/arthro/mellon-tis-eyropis-pernaei-apo-tin-ypaithro & 377 & Η Εφημερίδα των Συντακτών & Private/Non-Public & Online and Offline & National & high = CP is most important issue in story (can also cover other issues) & Institutional bargaining over funding & Negative & EU & No myth & Solidarity to poor countries/regions & Negative & EU & No myth & NA & NA & NA & NA & Greece & το μέλλον της ευρώπης περνάει από την ύπαιθρο & 2018-12-27 & πολιτική συνοχής & οι παγκόσμιες πολιτικές εξελίξεις τα τελευταία χρόνια καθορίστηκαν από την ψήφο της υπαίθρου. το 60\% αυτού του πληθυσμού στην αμερική ψήφισε τραμπ, το 55\% στο ηνωμένο βασίλειο στήριξε το brexit, ενώ ήταν οι αγροτικές περιοχές που έστειλαν τη λεπέν στον δεύτερο γύρο των προεδρικών εκλογών. το μεγαλύτερο λάθος όμως που μπορεί το πολιτικό σύστημα να κάνει είναι να αγνοήσει αυτό το μήνυμα, να χαρακτηρίσει συλλήβδην αυτόν τον κόσμο "ακραίο" και να μην εξετάσει τα αίτια αυτής της ψήφου. αυτά δεν είναι βέβαια άλλα από τις χαμηλές προοπτικές ανάπτυξης, την αίσθηση αποδυνάμωσης του κράτους δικαίου, τις αυξανόμενες διαπεριφερειακές ανισότητες και την ολοένα και πιο διαδεδομένη πεποίθηση ότι δεν υπάρχει καμία προοπτική σε αυτές τις περιοχές, τόσο για τους παλιούς όσο και για τους νέους. ολες αυτές οι παράμετροι συνιστούν ένα εκρηκτικό μείγμα, με ολέθριες επιπτώσεις για το μέλλον της ευρώπης και της ίδιας της δημοκρατίας. σε αυτά τα μηνύματα η ευρωπαϊκή επιτροπή, δυστυχώς, αντί να απαντήσει με περισσότερη αλληλεγγύη, συντονισμένες παρεμβάσεις, αλλά και προγραμματισμένο στρατηγικό συντονισμό, προσαρμοσμένο στις τοπικές συνθήκες, προτείνει στο νέο πολυετές δημοσιονομικό πλαίσιο 2021-2027 δραστικές μειώσεις στα κονδύλια, τόσο για την πολιτική συνοχής όσο και για την κοινή αγροτική πολιτική. η πρότασή της αυτή, ωστόσο, δεν δείχνει μόνο την έλλειψη στρατηγικής και πολιτικού θάρρους, αλλά και την έλλειψη κατεύθυνσης και προορισμού που θα όφειλε να ενσταλάξει στους ευρωπαίους πολίτες, ειδικά σε όσους αισθάνονται ξεχασμένοι και θεωρούν την ενωση την πηγή των προβλημάτων τους. πώς όμως μπορούμε να προωθήσουμε μια υπεύθυνη παραγωγή τροφίμων ή να καταπολεμήσουμε την ερήμωση της υπαίθρου, εάν η ευρωπαϊκή επιτροπή όντως μειώσει τα κονδύλια για τη γεωργία κατά 16\%; πώς μπορούμε να εξαλείψουμε τις διαπεριφερειακές ανισότητες, αν μειωθεί το ταμείο συνοχής κατά 12\%; πώς μπορούμε να μειώσουμε την ανεργία, να εξασφαλίσουμε αξιοπρεπή εργασία για όλους και να στρέψουμε τους νέους ξανά στην ύπαιθρο, εάν η ενωση εξακολουθεί να μειώνει τα ευρωπαϊκά κοινωνικά ταμεία; στον αντίποδα όλων αυτών των εξελίξεων, η ομάδα των ευρωσοσιαλιστών, τόσο στο ευρωπαϊκό κοινοβούλιο με εισηγήτρια την isabelle thomas όσο και στην επιτροπή των περιφερειών, αγωνίζεται προκειμένου να μην υπάρξει περικοπή σε αυτές τις πολιτικές, θέτοντας παράλληλα την ανάγκη προσαρμογής τους στις σύγχρονες προκλήσεις της παγκοσμιοποίησης, της κλιματικής αλλαγής, της ψηφιοποίησης και των ανισοτήτων. πιστεύουμε πως το νέο πολυετές δημοσιονομικό πλαίσιο οφείλει να δώσει μια απάντηση σε αυτές τις προκλήσεις που αντιμετωπίζει η ευρώπη και συνδέονται με το ίδιο της το μέλλον. κορωνίδα των προτάσεών μας είναι πως η κοινή αγροτική πολιτική μετά το 2020 θα πρέπει να προσεγγίσει τη γεωργία και τα τρόφιμα με τολμηρό και καινοτόμο τρόπο -μια ολιστική προσέγγιση- ως ζητήματα που έχουν κοινωνική, περιβαλλοντική και οικονομική διάσταση. γι' αυτή τη μετάβαση στην αειφόρο γεωργία και την ενίσχυση της αγροτικής υπαίθρου η εκθεση για την ευημερία και την ισότητα, που παρουσίασε πρόσφατα η ομάδα των σοσιαλιστών και δημοκρατών στο ευρωπαϊκό κοινοβούλιο, καθορίζει τους βασικούς στόχους που θα πρέπει να εκπληρώσει η καπ μετά το 2020, επαναφέροντας την αγροτική πολιτική στο επίκεντρο του ευρωπαϊκού σχεδιασμού. οι προτάσεις της έκθεσης δίνουν βάρος στη διασφάλιση αξιοπρεπούς αγροτικού εισοδήματος, στην ενίσχυση των αγροτών στην εφοδιαστική αλυσίδα και στη ρύθμιση της αγοράς, προστατεύοντας παράλληλα την υγεία των καταναλωτών και το περιβάλλον. οσες συντηρητικές και λαϊκιστικές δυνάμεις επενδύουν στη μεταβίβαση της απόφασης στο επόμενο ευρωπαϊκό κοινοβούλιο, ελπίζοντας πως οι συσχετισμοί θα είναι ακόμη περισσότερο υπέρ τους, ρισκάρουν την περαιτέρω διάλυση της υπαίθρου, με απρόβλεπτες συνέπειες για ολόκληρο το οικοδόμημα της ε.ε. σε αυτό το ενδεχόμενο η απάντηση θα πρέπει να είναι ηχηρή και περνάει μόνο μέσα από την ενίσχυση των προοδευτικών δυνάμεων στις ευρωεκλογές τον ερχόμενο μάιο. & 584 & high & High & Power & Values & NA & 2018-12-27 & 2018 & 3 & POL
Frame & high-very high & National & 500-1000 & 1.8924259 & 1.3590316 & -1.2391092 & 0.9814602 & -0.7922579 & 0.0 & -0.9049211 & -1.0736569 & Recipient & European & European & European & European|POL & Negative\\
Greece & http://www.thetoc.gr/oikonomia/article/sta-53806-dis-eurw-ta-kathara-esoda-tou-proupologismou & 347 & The TOC & Private/Non-Public & Online only & National & very low = CP mentioned once & Improve governance & Positive & EU + National + Subnational & No myth & NA & NA & NA & NA & NA & NA & NA & NA & Greece & στα 53,806 δισ. ευρώ τα καθαρά έσοδα του προϋπολογισμού |thetoc.gr & 2018-11-21 & ευρωπαϊκό κοινωνικό ταμείο & από φόρους επί αγαθών και υπηρεσιών προβλέπεται ότι θα εισπραχθούν έσοδα ύψους 27,559 δισ. ευρώ, αυξημένα κατά 682 εκατ. τα καθαρά έσοδα του κρατικού προϋπολογισμού, σε δημοσιονομική βάση, μετά την αφαίρεση των επιστροφών φόρων, προβλέπεται να διαμορφωθούν στα 53,806 δισ. ευρώ, παρουσιάζοντας αύξηση κατά 391 εκατ. ευρώ ή 0,7\%, έναντι του στόχου του μπδς 2019-2022. η αύξηση αυτή προβλέπεται να πραγματοποιηθεί, παρά τη σημαντική άνοδο των επιστροφών φόρων κατά 633 εκατ. ευρώ. αναλυτικότερα οι προβλέψεις του προϋπολογισμού για την πορεία των εσόδων ανά κατηγορία είναι οι ακόλουθες: -φόροι * φόροι επί αγαθών και υπηρεσιών από φόρους επί αγαθών και υπηρεσιών προβλέπεται ότι θα εισπραχθούν έσοδα ύψους 27,559 δισ. ευρώ, αυξημένα κατά 682 εκατ. ευρώ ή 2,5\% έναντι του στόχου του μπδς 2019-2022. ειδικότερα: - τα έσοδα από φπα αναμένεται να ανέλθουν στα 17,210 δισ. ευρώ, αυξημένα κατά 565 εκατ. ευρώ έναντι του στόχου του μπδς 2019-2022. - οι φόροι κατανάλωσης προβλέπονται στα 7,381 δισ. ευρώ και είναι μειωμένοι κατά 51 εκατ. ευρώ έναντι του στόχου του μπδς 2019-2022. φόροι και δασμοί επί εισαγωγών από φόρους και δασμούς επί εισαγωγών προβλέπονται έσοδα 237 εκατ. ευρώ, αυξημένα κατά 15 εκατ. ευρώ έναντι του στόχου του μπδς 2019-2022. * τακτικοί φόροι ακίνητης περιουσίας από τους τακτικούς φόρους ακίνητης περιουσίας, αναμένεται να εισπραχθούν έσοδα ύψους 2,801 δισ. ευρώ, μειωμένα κατά 215 εκατ. ευρώ έναντι του στόχου του μπδς 2019-2022, λόγω της εφαρμογής της δημοσιονομικής παρέμβασης μείωσης του ενφια έτους 2019 κατά 10\% μεσοσταθμικά, ή 260 εκατ. ευρώ. * λοιποί φόροι παραγωγής από τους λοιπούς φόρους παραγωγής προβλέπεται ότι θα εισπραχθούν 944 εκατ. ευρώ, πλησίον του στόχου του μπδς 2019-2022 ύψους 955 εκατ. ευρώ. * φόρος εισοδήματος από το φόρο εισοδήματος αναμένεται να εισπραχθούν έσοδα ύψους 16,796 δισ. ευρώ, παρουσιάζοντας αύξηση κατά 298 εκατ. ευρώ ή 1,8\% έναντι του στόχου του μπδς 2019-2022. ειδικότερα: - ο φόρος εισοδήματος φυσικών προσώπων προβλέπεται να διαμορφωθεί στα 11,070 δισ. ευρώ, αυξημένος κατά 211 εκατ. ευρώ έναντι του στόχου, ως αποτέλεσμα της δευτερογενούς επίδρασης των συνταξιοδοτικών παρεμβάσεων. - ο φόρος εισοδήματος νομικών προσώπων προβλέπεται να παρουσιάσει αύξηση κατά 162 εκατ. ευρώ σε σχέση με το στόχο και να διαμορφωθεί στα 4,420 δισ. ευρώ. * φόροι κεφαλαίου οι φόροι κεφαλαίου προβλέπεται να ανέλθουν σε 159 εκατ. ευρώ, αυξημένοι κατά 8 εκατ. ευρώ έναντι του στόχου του μπδς 2019-2022. * λοιποί τρέχοντες φόροι τα έσοδα από τους λοιπούς τρέχοντες φόρους προβλέπεται να ανέλθουν στο ποσό των 2,631 δισ. ευρώ, αυξημένα κατά 158 εκατ. ευρώ έναντι του στόχου του μπδς 2019-2022, λόγω αυξημένων εσόδων από τα διάφορα μη ταξινομημένα φορολογικά έσοδα. κοινωνικές εισφορές τα έσοδα από κοινωνικές εισφορές προβλέπεται να διαμορφωθούν στα 58 εκατ. ευρώ, αυξημένα κατά 8 εκατ. ευρώ έναντι του στόχου του μπδς 2019-2022. μεταβιβάσεις τα έσοδα από μεταβιβάσεις αναμένεται να ανέλθουν στα 4,049 δισ. ευρώ, μειωμένα κατά 28 εκατ. ευρώ έναντι του στόχου του μπδς 2019-2022. εξαυτών, ποσό 291 εκατ. ευρώ θα προέλθει από τη μεταφορά αποδόσεων λόγω της διακράτησης ομολόγων του ελληνικού δημοσίου στα χαρτοφυλάκια των κεντρικών τραπεζών του ευρωσυστήματος (anfas), καθώς και από το πρόγραμμα αγοράς ομολόγων (smps) της ευρωπαϊκής κεντρικής τράπεζας (εκτ). πωλήσεις αγαθών και υπηρεσιών από τις πωλήσεις αγαθών και υπηρεσιών προβλέπεται να εισπραχθούν έσοδα ύψους 773 εκατ. ευρώ, έναντι στόχου 833 εκατ. ευρώ. λοιπά τρέχοντα έσοδα τα λοιπά τρέχοντα έσοδα προβλέπεται να διαμορφωθούν στα 1,500 δισ. ευρώ, πλησίον του στόχου του μπδς 2019-2022, ύψους 1,498 δισ. ευρώ. πωλήσεις παγίων περιουσιακών στοιχείων προβλέπονται έσοδα 335 εκατ. ευρώ αυξημένα κατά 212 εκατ. ευρώ έναντι του στόχου, κυρίως λόγω της είσπραξης το 2019 αντί για το 2018 του ντιτίμου από την αξιοποίηση της έκτασης του πρώην διεθνούς αεροδρομίου του ελληνικού. επιστροφές φόρων οι επιστροφές αχρεωστήτως εισπραχθέντων εσόδων προβλέπεται να διαμορφωθούν στα 4,818 δισ. ευρώ, αυξημένες κατά 633 εκατ. ευρώ ή 15,1\%, έναντι των προβλέψεων του μπδς 2019-2022. έσοδα προγράμματος δημοσίων επενδύσεων τα έσοδα του πδε σε ταμειακή βάση προβλέπεται να ανέλθουν στο ποσό των 3,740 δισ. ευρώ, αυξημένα κατά 150 εκατ. ευρώ έναντι του στόχου του μπδς 2019-2022, λόγω μεταφοράς στο πδε απολήψεων από την εε που μέχρι πρότινος εμφανίζονταν στον τακτικό προϋπολογισμό. σημειώνεται ότι βασική πηγή εσόδων για την χρηματοδότηση των έργων του πδε αποτελούν οι εισροές από την ευρωπαϊκή ένωση και ειδικότερα από το ευρωπαϊκό ταμείο περιφερειακής ανάπτυξης (ετπα), το ευρωπαϊκό γεωργικό ταμείο εγγυήσεων (εγτε), το ευρωπαϊκό κοινωνικό ταμείο (εκτ), το ταμείο συνοχής, το ευρωπαϊκό γεωργικό ταμείο αγροτικής ανάπτυξης (εγταα), το ευρωπαϊκό ταμείο θάλασσας και αλιείας (ετθα), καθώς και τα ταμεία για τη μετανάστευση. & 753 & very low & Low & Governance & NA & NA & 2018-11-21 & 2018 & 3 & POL
Frame & v.low & National & 500-1000 & 1.8924259 & 1.3590316 & -1.2391092 & 0.9814602 & -0.7922579 & 0.0 & -0.9049211 & -1.0736569 & Recipient & Domestic & European & Mixed & Domestic|POL & Positive\\
Greece & http://www.real.gr/oikonomia/arthro/kretsou\_se\_papadimouli\_stoxos\_tis\_komision\_h\_auksisi\_tis\_xrimatodotisis\_ton\_mikromesaion\_epixeiriseon-491640/ & 376 & real.gr & Private/Non-Public & Online and Offline & National & very high = CP is most important issue + CP is mentioned in title/headline & Economic development & Positive & EU & No myth & Research \& innovation & Positive & EU & No myth & Improve governance & Positive & EU & No myth & Greece & κρέτσου σε παπαδημούλη: στόχος της κομισιόν η αύξηση της χρηματοδότησης των μικρομεσαίων επιχειρήσεων & 2018-08-30 & ευρωπαϊκό κοινωνικό ταμείο & για να χρησιμοποιούνται πιο συγκεντρωτικά οι περιορισμένοι πόροι, η επιτροπή πρότεινε να παρέχεται υποστήριξη για παραγωγικές επενδύσεις μόνο σε μμε, εκτός από τις περιπτώσεις στις οποίες οι επενδύσεις σε μεγαλύτερες επιχειρήσεις εμπεριέχουν συνεργασία με μμε για δραστηριότητες έρευνας και καινοτομίας. σύμφωνα με την κ. κρέτσου, "το αποτέλεσμα μόχλευσης που θα έχει η χρηματοδότηση από το ετπα αναμένεται να αυξήσει περαιτέρω τη χρήση των χρηματοδοτικών μέσων, βασικού μηχανισμού για την υλοποίηση επενδύσεων που παράγουν έσοδα ή εξοικονομούν κόστος. οι μμε θα μπορούν να επωφεληθούν από την παροχή δανείων, από επενδύσεις μετοχικού ή οιονεί μετοχικού κεφαλαίου ή από εγγυήσεις στο πλαίσιο κάθε στόχου πολιτικής. οι μμε που αναπτύσσουν έρευνα και καινοτομία θα έχουν επίσης ευκαιρίες χρηματοδότησης στο πλαίσιο του επόμενου προγράμματος-πλαισίου έρευνας και καινοτομίας (horizon europe). η επιτροπή πρότεινε επίσης τη δημιουργία ενός νέου ταμείου, του investeu, στο οποίο θα συγκεντρωθούν τα υπό κεντρική διαχείριση, σήμερα, χρηματοδοτικά μέσα". επίσης, οι προτάσεις της επιτροπής για τον κανονισμό περί κοινών διατάξεων, που θα θέσουν τους κανόνες για το ευρωπαϊκό ταμείο περιφερειακής ανάπτυξης (ετπα), το ταμείο συνοχής, το ευρωπαϊκό κοινωνικό ταμείο , το ευρωπαϊκό ταμείο θάλασσας και αλιείας, το ταμείο ασύλου, μετανάστευσης και ένταξης, το ταμείο εσωτερικής ασφάλειας και το μέσο για τη διαχείριση των συνόρων και των θεωρήσεων περιέχουν ογδόντα βασικές απλοποιήσεις που αναμένεται να οδηγήσουν σε μείωση κατά 20-25 \% του διοικητικού κόστους που συνδέεται με την πολιτική συνοχής, ιδίως με το σύστημα διαχείρισης και ελέγχου. οι μικρές και μεσαίες επιχειρήσεις θα επωφεληθούν επίσης από αυτή τη μείωση της διοικητικής επιβάρυνσης. & 248 & very high & High & Socio-Economic & Socio-Economic & Governance & 2018-08-30 & 2018 & 3 & ECO
Frame & high-very high & National & <500 & 1.8924259 & 1.3590316 & -1.2391092 & 0.9814602 & -0.7922579 & 0.0 & -0.9049211 & -1.0736569 & Recipient & European & European & European & European|ECO & Positive\\
\addlinespace
Greece & http://www.naftemporiki.gr/finance/story/858070/etep-nea-xrimatodotika-ergaleia-gia-tin-enisxusi-ton-mme & 334 & naftemporiki.gr & Private/Non-Public & Online and Offline & National & medium = CP is important part of story & Ineffective goal achievement & Balanced & EU + National & No myth & NA & NA & NA & NA & NA & NA & NA & NA & Greece & ετεπ: νέα χρηματοδοτικά εργαλεία για την ενίσχυση των μμε & 2014-09-18 & διαρθρωτικά ταμεία & στην υιοθέτηση νέων χρηματοδοτικών εργαλείων προσανατολίζεται η ευρωπαϊκή τράπεζα επενδύσεων (ετεπ) για τη βελτίωση των οικονομικών δεικτών και με κεντρικό άξονα την περαιτέρω ενίσχυση των μικρομεσαίων επιχειρήσεων όπως αποκάλυψε ο πρόεδρος της ετεπ, β. χόιερ, μετά την απογευματινή συνάντηση που είχε με τον υπουργό ανάπτυξης, νίκο δένδια. ο κ. χόιερ σχολίασε ότι είναι σημαντική η πρόοδος στον τομέα της ανεργίας "αλλά η κρίση δεν τελείωσε και θα θεωρήσουμε ότι αυτό συμβαίνει όταν αρχίσουν να δημιουργούνται θέσεις εργασίας". επίσης, υπογράμμισε ότι ενώ ο στόχος για το 2014 ήταν να διατεθεί στην ελλάδα περίπου 1 δισ. ευρώ, στο εννεάμηνο έχει διατεθεί ήδη 1,4 δισ. ευρώ. το πρόβλημα μεταφοράς της ρευστότητας στην πραγματική οικονομία ήταν από τα βασικά θέματα συζήτησης στην απογευματινή συνάντηση και σύμφωνα με πληροφορίες, το υπουργείο σχεδιάζει να προχωρήσει σε ενδελεχή έρευνα για τα αίτια αυτής της κατάστασης. είναι χαρακτηριστικό ότι από τα 400 εκατ. ευρώ που "κάθονται" στα διαρθρωτικά ταμεία από προγράμματα που σχετίζονται με το εσπα, έχουν απορροφηθεί 168 εκατ. ευρώ. ο υπουργός, στις δηλώσεις που έκανε μετά τη συνάντηση, εκτίμησε ότι το πρόβλημα που εντοπίζεται στη μεταφορά της ρευστότητας προς τις μικρομεσαίες επιχειρήσεις θα βελτιωθεί από το υπό σύσταση δεύτερο υποταμείο του ελληνικού επενδυτικού ταμείου, στο οποίο συμμετέχει η ετεπ. ο πρόεδρος της ετεπ, από την πλευρά του, δήλωσε πεπεισμένος ότι τα πράγματα θα βελτιωθούν περαιτέρω μέσω των δράσεων και αυτού του εργαλείου. & 227 & medium & Medium & Socio-Economic & NA & NA & 2014-09-18 & 2014 & 1 & ECO
Frame & low-medium & National & <500 & 1.8924259 & 1.3590316 & -1.2391092 & 0.9814602 & -0.7922579 & 0.0 & -0.9049211 & -1.0736569 & Recipient & Domestic & European & Mixed & Domestic|ECO & Neutral\\
Greece & http://avgi.gr/article/10813/9128059/-strategike-epiloge-gia-ten-periphereia-attikes-e-sterixe-tes-demosias-ygeias- & 304 & avgi.gr & Private/Non-Public & Online and Offline & National & low = CP mentioned more times but NOT important part of story (mainly about others issues) & Public services & Positive & Subnational & No myth & Public services & Positive & EU + Subnational & No myth & NA & NA & NA & NA & Greece & "στρατηγική επιλογή για την περιφέρεια αττικής η στήριξη της δημόσιας υγείας" & 2018-08-30 & ευρωπαϊκό ταμείο περιφερειακής ανάπτυξης & η ρένα δούρου προωθεί τη συμπλήρωση και βελτίωση του εξοπλισμού σε 21 νοσοκομεία της αττικής με συνολικό προϋπολογισμό 13.356.880 ευρώ "πιστή στον κοινωνικό της ρόλο και στη στρατηγική επιλογή της να στηρίζει έμπρακτα το δικαίωμα όλων των πολιτών στην υγεία", η περιφέρεια αττικής ενέκρινε τη χρηματοδότηση για ακόμα μία παρέμβαση ενίσχυσης των νοσοκομείων της αττικής. συγκεκριμένα, με την υπογραφή της πρόσκλησης "προμήθεια εξοπλισμού μονάδων δευτεροβάθμιας και τριτοβάθμιας φροντίδας υγείας στην περιφέρεια αττικής", η περιφερειάρχης ρένα δούρου προωθεί τη συμπλήρωση και βελτίωση του εξοπλισμού σε 21 νοσοκομεία της αττικής με συνολικό προϋπολογισμό 13.356.880 ευρώ. στο πλαίσιο της δράσης αυτής, καλούνται να υποβάλουν προτάσεις στο επιχειρησιακό πρόγραμμα "αττική 2014-2020" τα: γ.ν.α. σισμανόγλειο, γ.ν.α. κατ, νοσοκομείο θείας πρόνοιας "η παμμακάριστος", γ.ν. παίδων πεντέλης, γ.ν. ν.ιωνίας "κωνσταντοπούλειο" - πατησίων, γενικό ογκολογικό νοσοκομείο κηφισιάς "οι άγιοι ανάργυροι", γ.ν.α. "γ.γεννηματάς", γ.ν.α. "ο ευαγγελισμός", γ.ν.ν.θ.α. "η σωτηρία", γ.ν.α. αλεξάνδρα, α.ο.ν.α. "ο άγιος σάββας", γ.ν.α. ιπποκράτειο, γ.ν.α. κοργιαλένειο - μπενάκειο ε.ε.ς., γ.ν.α. "η ελπίς", ν.α. \& δ.ν.α. "ανδρέας συγγρός", γ.ν.α. "λαϊκό", γ.ν. παίδων "αγλαΐα κυριακού", νοσοκομείο "έλενα βενιζέλου", γ.ν.π. "τζάνειο", π.γ.ν. "αττικόν", ψυχιατρικό νοσοκομείο αττικής. η πρόσκληση, που εντάσσεται στον άξονα προτεραιότητας 10 "ανάπτυξη - αναβάθμιση στοχευμένων κοινωνικών υποδομών και υποδομών υγείας" και συγχρηματοδοτείται από το ευρωπαϊκό ταμείο περιφερειακής ανάπτυξης (ετπα), στοχεύει στη διασφάλιση της εύρυθμης λειτουργίας των νοσοκομείων και της δημόσιας υγείας, την αναβάθμιση των παρεχόμενων υπηρεσιών αλλά και την ισότιμη πρόσβαση σε υπηρεσίες υγείας υψηλού επιπέδου. άλλος ένας κρίκος στην αλυσίδα των στοχευμένων παρεμβάσεων της περιφέρειας για την υγεία όπως αναφέρει η περιφέρεια αττικής, η πρόσκληση αυτή έρχεται να συμπληρώσει στοχευμένες παρεμβάσεις της περιφέρειας για τη στήριξη του τομέα της υγείας, όπως η, σε συνεργασία με το υπουργείο υγείας, χρηματοδότηση της προμήθειας υπερσύγχρονου και απαραίτητου εξοπλισμού για τα νοσοκομεία της αττικής, ύψους 40,6 εκατομμυρίων ευρώ (απόφαση για την οποία μάλιστα χρειάστηκε και νομοθετική ρύθμιση), η χρηματοδότηση της προμήθειας νέων ασθενοφόρων για το εκαβ με 3,6 εκατομμύρια ευρώ, η χρηματοδότηση της προμήθειας εξοπλισμού πρωτοβάθμιας υγείας και κέντρου υγείας κερατσινίου με 9,5 εκατομμύρια ευρώ, αλλά και η υπογραφή προγραμματικής σύμβασης με το υπουργείο εθνικής άμυνας για την κάλυψη επειγουσών ελλείψεων ιατροτεχνολογικού και μηχανολογικού εξοπλισμού των στρατιωτικών νοσοκομείων με προϋπολογισμό 20 εκ. ευρώ. με τις παραπάνω δράσεις ολοκληρώνεται η πρώτη φάση ενός ολοκληρωμένου σχεδίου ενίσχυσης της πρωτοβάθμιας, δευτεροβάθμιας και τριτοβάθμιας φροντίδας υγείας με ένα μίγμα εθνικής και κοινοτικής χρηματοδότησης, ύψους 90 εκατομμυρίων ευρώ περίπου. α α α email εκτυπωση κατηγορία ελλάδα ροη κατηγοριας & 437 & low & Low & Socio-Economic & Socio-Economic & NA & 2018-08-30 & 2018 & 3 & ECO
Frame & low-medium & National & <500 & 1.8924259 & 1.3590316 & -1.2391092 & 0.9814602 & -0.7922579 & 0.0 & -0.9049211 & -1.0736569 & Recipient & Domestic & Domestic & Domestic & Domestic|ECO & Positive\\
Greece & http://voria.gr/article/miosi-dapanon-gia-ti-georgia-protini-i-komision & 320 & Ernst\&Young: Φρένο στις δημόσιες εγγραφές παγκοσμίως & Private/Non-Public & Online only & National & very high = CP is most important issue + CP is mentioned in title/headline & Institutional bargaining over funding & Negative & EU & No myth & NA & NA & NA & NA & NA & NA & NA & NA & Greece & μείωση δαπανών για τη γεωργία προτείνει η κομισιόν & 2018-05-02 & πολιτική συνοχής & σε μειώσεις των δαπανών της κοινής αγροτικής πολιτικής αλλά και της συνοχής, αναμένεται να προχωρήσει σήμερα η ευρωπαϊκή επιτροπή πρόκειται για μια από τις σημαντικότερες διαπραγματεύσεις των τελευταίων ετών, η οποία θα ξεκινήσει με βάση τη σημερινή πρόταση της επιτροπής, ενώ τα κράτη μέλη καλούνται να καταλήξουν σε συμφωνία με ομοφωνία, κάτι που κάνει ακόμη πιο δύσκολο το εγχείρημα. το πρόβλημα δημιουργείται με την αποχώρηση του ηνωμένου βασιλείου από την εε και το κενό που θα προκαλέσει στον κοινοτικό προϋπολογισμό η απουσία της συνεισφοράς του, το οποίο εκτιμάται πως μπορεί να φτάσει μέχρι και τα 100 δισεκατομμύρια ευρώ για την επταετία μετά το 2020. επιπλέον, η εε θα πρέπει να βρει πρόσθετα χρήματα και για τη χρηματοδότηση των νέων πολιτικών, όπως της φύλαξης των εξωτερικών συνόρων, της εσωτερικής ασφάλειας και της ευρωπαϊκής άμυνας. αύξηση συνεισφοράς και μείωση δαπανών συνεπώς το σημερινό πακέτο θα προβλέπει την αύξηση της συνεισφοράς στον κοινοτικό προϋπολογισμό για τα 27 κράτη μέλη και ταυτόχρονα μια μείωση των δαπανών για την κοινή αγροτική πολιτική και την πολιτική συνοχής. ορισμένες χώρες του κοινοτικού βορρά, όπως η ολλανδία, η δανία, η σουηδία και η αυστρία δεν θέλουν να αυξήσουν την συνεισφορά τους στον προϋπολογισμό, αντίθετα η γερμανία και η γαλλία είναι πρόθυμες, αλλά σε καμία περίπτωση δεν θα καλύψουν όλο το κενό. είναι προφανές ότι η μείωση των δαπανών για την γεωργία θα έχει επιπτώσεις και στο εισόδημα των ελλήνων αγροτών, αφού θα επιφέρει μείωση των κοινοτικών ενισχύσεων, ενώ από τη μείωση των δαπανών της συνοχής οι επιπτώσεις θα είναι μικρότερες. η επιτροπή θα ζητήσει από τα κράτη μέλη να επιταχύνουν τη διαπραγμάτευση ώστε να υπάρξει συμφωνία πριν από τη λήξη της θητείας της παρούσας ευρωβουλής, το μάιο του 2019. νίκος μπέλλος, βρυξέλλες & 284 & very high & High & Power & NA & NA & 2018-05-02 & 2018 & 3 & POL
Frame & high-very high & National & <500 & 1.8924259 & 1.3590316 & -1.2391092 & 0.9814602 & -0.7922579 & 0.0 & -0.9049211 & -1.0736569 & Recipient & European & European & European & European|POL & Negative\\
Greece & http://news247.gr/eidiseis/koinonia/ypoyrgeio-ergasias-paraiththhke-h-gg-diaxeirishs-koinotikwn-kai-allwn-porwn.4324300.html & 311 & news247.gr & Private/Non-Public & Online only & National & medium = CP is important part of story & Mismanagement & Balanced & National & No myth & NA & NA & NA & NA & NA & NA & NA & NA & Greece & υπουργείο εργάσιας: παραιτήθηκε η γγ διαχειρίσης κοινοτικών και άλλων πόρων & 2016-10-18 & ευρωπαϊκό κοινωνικό ταμείο & η δήμητρα χαλικιά ήρθε σε ανοιχτή αντιπαράθεση με την πολιτική της προϊστάμενο, ρία αντωνοπούλου και στη συνέχεια υπέβαλε την παραίτησή της, η οποία και έγινε δεκτή την παραίτησή της από τη θέση της γενικής γραμματέως διαχειρίσης κοινοτικών και άλλων πόρων του υπουργείου εργασίας υπέβαλε η δήμητρα χαλικιά, η οποία, σύμφωνα με πηγές του υπουργείου, έγινε αποδεκτή. η γενική γραμματέας συγκρούστηκε ανοικτά με την αναπληρώτρια υπουργό εργασίας ράνια αντωνοπούλου, ενώ είχε αποστείλει και επιστολή της προς τον αλέξη τσίπρα με την οποία κατηγορούσε την πολιτική της προϊσταμένη για συκοφαντία, προσωπική ιδιοτέλεια, κακές επιλογές στη διαχείριση κοινοτικών χρημάτων και πολιτικές που πριμοδοτούν συγκεκριμένα κέντρα επαγγελματικής κατάρτισης. η επιστολή δημοσιεύθηκε σε κυριακάτικη εφημερίδα και η κ. αντωνοπούλου, με ανακοίνωσή της τη δευτέρα και χωρίς να αναφέρεται προσωπικά στην κ. χαλικιά, έκανε λόγο για επιλογή τροφοδότησης με ανυπόστατες, ψευδείς ή ανακριβείς πληροφορίες που δημιουργεί μείζον θέμα πολιτικής δεοντολογίας. το μαξίμου φαίνεται πως είχε λάβει γνώση του περιστατικού και η παραίτηση έμοιαζε μονόδρομος. η γενική γραμματεία διαχείρισης κοινοτικών και άλλων πόρων, του υπουργείου εργασίας, κοινωνικής ασφάλισης και πρόνοιας, είναι αρμόδια για το συντονισμό, τη διαχείριση και την υλοποίηση παρεμβάσεων που αφορούν στην ανάπτυξη του ανθρώπινου δυναμικού της χώρας και συγχρηματοδοτούνται από το ευρωπαϊκό κοινωνικό ταμείο (εκτ), καθώς και από άλλα εθνικά ή/και ευρωπαϊκά χρηματοδοτικά μέσα. στόχοι της γενικής γραμματείας διαχείρισης κοινοτικών \& άλλων πόρων είναι η καταπολέμηση της ανεργίας, η διασφάλιση των θέσεων εργασίας, η κοινωνική και εργασιακή ένταξη των ευπαθών ομάδων και η ανάπτυξη του τομέα της κοινωνικής οικονομίας. & 245 & medium & Medium & Governance & NA & NA & 2016-10-18 & 2016 & 2 & POL
Frame & low-medium & National & <500 & 1.8924259 & 1.3590316 & -1.2391092 & 0.9814602 & -0.7922579 & 0.0 & -0.9049211 & -1.0736569 & Recipient & Domestic & Domestic & Domestic & Domestic|POL & Neutral\\
Greece & http://www.newsbeast.gr/greece/arthro/2190903/prosklisi-se-1-400-anergous-neous-gia-simmetochi-se-epidotoumeno-programma & 338 & Newsbeast.gr & Private/Non-Public & Online only & National & medium = CP is important part of story & Jobs & Factual & EU + National + Subnational & No myth & NA & NA & NA & NA & NA & NA & NA & NA & Greece & πρόσκληση σε 1.400 ανέργους νέους για συμμετοχή σε επιδοτούμενο πρόγραμμα & 2016-04-01 & ευρωπαϊκό κοινωνικό ταμείο & η διαδικασία υποβολής των αιτήσεων ξεκίνησε δύο δράσεις για την κατάρτιση συνολικά 1.400 ανέργων, 18-24 ετών, έχουν αναλάβει να υλοποιήσουν από κοινού η ειδική υπηρεσία εφαρμογής συγχρηματοδοτούμενων ενεργειών από το ευρωπαϊκό κοινωνικό ταμείο (ευε/εκτ) της γενικής γραμματείας διαχείρισης κοινοτικών και άλλων πόρων του υπουργείου εργασίας κοινωνικής ασφάλισης και κοινωνικής αλληλεγγύης και ο σύνδεσμος βιομηχανιών βόρειας ελλάδας (σββε), ως συνδικαιούχος. το πρόγραμμα θα υλοποιηθεί στις περιφέρειες: ηπείρου, δυτικής και κεντρικής μακεδονίας και ανατολικής μακεδονίας και θράκης και όπως επισημαίνεται σε σχετική ανακοίνωση, η διαδικασία υποβολής των αιτήσεων ξεκίνησε μέσω του συνδέσμου στο διαδίκτυο: www.sbbe-edu.gr. στη συγκεκριμένη ιστοσελίδα παρέχονται λεπτομερή στοιχεία για το πρόγραμμα και οδηγίες για τη συμπλήρωση της αίτησης. οι δράσεις είναι οι εξής: τονίζεται ότι οι προαναφερόμενες δράσεις εντάσσονται στο επιχειρησιακό πρόγραμμα "ανάπτυξη ανθρώπινου δυναμικού" και είναι υποψήφιες προς ένταξη στο εσπα 2014-2020 και στο επιχειρησιακό πρόγραμμα "ανάπτυξη ανθρώπινου δυναμικού, εκπαίδευση και διά βίου μάθηση". συγχρηματοδοτούνται, δε, από το ευρωπαϊκό κοινωνικό ταμείο (εκτ).πρόκειται για ένα ολοκληρωμένο πρόγραμμα (πράξη) επαγγελματικής κατάρτισης, συμβουλευτικής υποστήριξης και πιστοποίησης προσόντων που απευθύνεται: σε 500 ανέργους νέους στον πρώτο τομέα και σε 900 ανέργους στον δεύτερο.έχει συνολική διάρκεια 380 ώρες, εκ των οποίων 120 ώρες θεωρητική κατάρτιση και 260 ώρες πρακτική άσκηση, και θα υλοποιηθεί σε τέσσερις διαδοχικές φάσεις: & 213 & medium & Medium & Socio-Economic & NA & NA & 2016-04-01 & 2016 & 2 & ECO
Frame & low-medium & National & <500 & 1.8924259 & 1.3590316 & -1.2391092 & 0.9814602 & -0.7922579 & 0.0 & -0.9049211 & -1.0736569 & Recipient & Domestic & European & Mixed & Domestic|ECO & Neutral\\
\addlinespace
Greece & http://www.dikaiologitika.gr/eidhseis/oikonomia/167038/aisiodoksia-katainen-gia-tin-elliniki-oikonomia & 327 & dikaiologitika.gr & Private/Non-Public & Online only & National & medium = CP is important part of story & Economic development & Positive & EU + National & No myth & NA & NA & NA & NA & NA & NA & NA & NA & Greece & αισιοδοξία κατάινεν για την ελληνική οικονομία & 2017-07-24 & διαρθρωτικά ταμεία & "η ανάπτυξη έχει επιστρέψει, οι μεταρρυθμίσεις έχουν γίνει, το πρωτογενές πλεόνασμα είναι μεγάλο και υπάρχει μεγάλη ανάγκη επενδύσεων στη χώρα" σημειώνει ο αντιπρόεδρος της ευρωπαϊκής επιτροπής γίρκι κατάινεν, σε του συνέντευξη στο απε-μπε. ταυτοχρόνως, την ικανοποίησή του για το βαθμό αξιοποίησης του σχεδίου γιούνκερ από την ελλάδα κυρίως σε ό,τι αφορά τη στήριξη των μικρομεσαίων επιχειρήσεων που δεν έχουν πρόσβαση σε δάνεια από τράπεζες, εκφράζει ο αντιπρόεδρος της ευρωπαϊκής επιτροπής, αρμόδιος για την απασχόληση, την ανάπτυξη, τις επενδύσεις και την ανταγωνιστικότητα, στη συνέντευξη στο απε-μπε. όπως αναφέρει, η ελλάδα είναι όγδοη στην στην κατάταξη των χωρών που κάνουν μεγαλύτερη χρήση του επενδυτικού ταμείου με βάση το αεπ. εξάλλου, ο αντιπρόεδρος της επιτροπής επισημαίνει μιλώντας στο απε πως οι μεταρρυθμίσεις που έχει εφαρμόσει η χώρα έχουν βελτιώσει κατά πολύ το επιχειρηματικό περιβάλλον στην ελλάδα, ενώ δηλώνει αισιόδοξος ότι η εμπιστοσύνη επιστρέφει με αποτέλεσμα την τόνωση του επενδυτικού ενδιαφέροντος, μεταξύ άλλων και από το εξωτερικό. "η ελλάδα τα πάει σχετικά καλά. είναι όγδοη στην κατάταξη των χωρών που κάνουν μεγαλύτερη χρήση του επενδυτικού ταμείου με βάση το αεπ. είμαι πολύ χαρούμενος που η ελλάδα και συγκεριμένα οι ελληνικές επιχειρήσεις έχουν ανακαλύψει το ευρωπαϊκό ταμείο στρατηγικών επενδύσεων (ετσε) γιατί αυτό το ταμείο φτιάχτηκε για να αντιμετωπιστεί η αδυναμία πρόσβασης στις αγορές. η ελλάδα, λοιπόν, πληροί όλα τα κριτήρια αφού οι τράπεζές της δεν μπορούν να χρηματοδοτήσουν τις επιχειρήσεις όσο χρειάζεται. γι' αυτό το ετσε είναι το τέλειο εργαλείο για την ελλάδα και μακάρι να το χρησιμοποιήσει ακόμα περισσότερο, γιατί όχι να φτάσει στην πρώτη θέση." τονίζει ο αντιπρόεδρος της κομισιόν. "σύμφωνα με τη λίστα των χωρών που επωφελούνται περισσότερο του σχεδίου γιούνκερ ανάλογα με το αεπ, πρώτη είναι η εσθονία, μετά η ισπανία, η βουλγαρία και η πορτογαλία. επομένως, το ετσε φαίνεται ότι λειτουργεί σύμφωνα με το σχεδιασμό, δηλαδή αντιμετωπίζοντας τις αδυναμίες της αγοράς. σε ό,τι αφορά τον αριθμό των έργων η γερμανία τα πάει πολύ καλά αλλά το ίδιο και η ιταλία και η ισπανία. αυτός είναι ο μόνος τρόπος για να αξιολογήσει κανείς εάν το ταμείο μπορεί να κάνει τη διαφορά σε ένα συγκεκριμένο κράτος-μέλος. αυτό που θέλουμε να προωθήσουμε στην ελλάδα είναι το συνδυασμό κεφαλαίων από ετσε με τα διαρθρωτικά ταμεία. γιατί με αυτόν τον τρόπο θα μπορέσουμε να πολλαπλασιάσουμε τον αντίκτυπο των διαρθρωτικών ταμείων. επίσης, θέλουμε να ενθαρρύνουμε την ελλάδα να αναπτύξει επενδυτικές πλατφόρμες, οι οποίες θα συγκεντρώνουν μεγάλα ποσά από μικρές επενδύσεις. είχα μια ιδέα που δεν έχει ακόμα εφαρμοστεί αλλά την πρότεινα στις ελληνικές αρχές. να δημιουργηθεί μια πλατφόρμα επενδύσεων στον τουριστικό τομέα. να φτιαχτεί μια εταιρία δημόσιων και ιδιωτικών συμφερόντων στην οποία το ετσε θα μπορούσε να βάλει κάποια κεφάλαια, όπως και τα διαρθρωτικά ταμεία αλλά και ιδιώτες, τράπεζες ή επενδυτικές εταιρίες και βάσει αυτού του κεφαλαιακού υπόβαθρου θα μπορούσαν να παρέχονται μακροπρόθεσμα δάνεια από το ετσε σε επιχειρηματίες. αυτή η πλατφόρμα θα μπορούσε δηλαδή να χορηγεί δάνεια σε επιχειρηματίες που θέλουν να βελτιώσουν την επιχείρησή τους ή να επενδύσουν σε κάτι καινούργιο και δεν έχουν πρόσβαση σε δάνεια από τράπεζες. συγκεντρώνοντας δέκα εκατομμύρια από το ετσε, και δύο ή περισσότερα από ιδιώτες, ο αντίκτυπος θα είναι δεκάδες εκατομμύρια σε δάνεια προς επιχειρηματίες." δηλώνει ο κ. κατάινεν στο απε-μπε & 529 & medium & Medium & Socio-Economic & NA & NA & 2017-07-24 & 2017 & 2 & ECO
Frame & low-medium & National & 500-1000 & 1.8924259 & 1.3590316 & -1.2391092 & 0.9814602 & -0.7922579 & 0.0 & -0.9049211 & -1.0736569 & Recipient & Domestic & European & Mixed & Domestic|ECO & Positive\\
Greece & http://www.naftemporiki.gr/story/1059067 & 335 & naftemporiki.gr & Private/Non-Public & Online and Offline & National & very low = CP mentioned once & Cultural development & Positive & EU + National + Subnational & No myth & NA & NA & NA & NA & NA & NA & NA & NA & Greece & κρήτη: εγκαινιάζεται το μουσείο ολοκαυτώματος βιάννου & 2016-01-29 & ευρωπαϊκό ταμείο περιφερειακής ανάπτυξης & τα εγκαίνια του μουσείου ολοκαυτώματος βιάννου, στο χώρο του πανδημοτικού ηρώου στη θέση σελί αμιρών, θα πραγματοποιηθούν την ερχόμενη κυριακή, παρουσία του υπουργού εσωτερικών και διοικητικής ανασυγκρότησης παναγιώτη κουρουμπλή, του περιφερειάρχη κρήτης σταύρου αρναουτάκη και των τοπικών αρχών. το μουσείο θα λειτουργήσει σε ειδικά διαμορφωμένο ισόγειο χώρο στη θέση σελί αμιρών, ενώ οι επισκέπτες του θα έχουν την ευκαιρία να γνωρίσουν τα αξιόλογα εκθέματά του και να αντλήσουν πολύτιμες ιστορικές πληροφορίες, για την κατοχή και την αντίσταση στη βιάννο. το έργο "βελτίωση - ανάδειξη - προβολή του μουσείου ολοκαυτώματος στα αμιρά δήμου βιάννου", συνολικού προϋπολογισμού 240.000 ευρώ, χρηματοδοτήθηκε από το ευρωπαϊκό ταμείο περιφερειακής ανάπτυξης (ε.τ.π.α.) στο πλαίσιο του περιφερειακού επιχειρησιακού προγράμματος κρήτης και νήσων αιγαίου 2007-2013, και στο πρόγραμμα (ολοκληρωμένα σχέδια ανάπτυξης περιοχών υπαίθρου) ο.σ.α.π.υ. του δήμου βιάννου. επίσης, η διάσωση υλικών αντικειμένων, αρχειακών και άλλων πηγών και τεκμηρίων της συγκεκριμένης ιστορικής περιόδου, ώστε να προάγεται η συνειδητοποίηση της σημασίας των πηγών για τη γνώση της ιστορίας και την έγκυρη και τεκμηριωμένη ανάπλαση του παρελθόντος. σύμφωνα με το πρόγραμμα που ανακοινώθηκε από τον δήμο βιάννου και την ένωση θυμάτων ολοκαυτώματος, το μεσημέρι της κυριακής θα τελεσθεί ο αγιασμός των εγκαινίων και στη συνέχεια θα γίνουν χαιρετισμοί από τον μητροπολίτη αρκαλοχωρίου καστελίου και βιάννου ανδρέα και τον δήμαρχο βιάννου παύλο μπαριτάκη. τέλος, θα γίνει η παρουσίαση του μουσείου από τον δήμαρχο και προβολή ταινίας με μαρτυρίες επιζώντων των εκτελέσεων, ενώ θα ακολουθήσει μουσική εκδήλωση από τη χορωδία βιάννου "στάθης μάστορας". & 246 & very low & Low & Socio-Economic & NA & NA & 2016-01-29 & 2016 & 2 & ECO
Frame & v.low & National & <500 & 1.8924259 & 1.3590316 & -1.2391092 & 0.9814602 & -0.7922579 & 0.0 & -0.9049211 & -1.0736569 & Recipient & Domestic & European & Mixed & Domestic|ECO & Positive\\
Greece & http://voria.gr/article/katapeltis-i-komision-gia-tin-poria-ton-ergon-tis-evath & 317 & Ernst\&Young: Φρένο στις δημόσιες εγγραφές παγκοσμίως & Private/Non-Public & Online only & Regional/Local & high = CP is most important issue in story (can also cover other issues) & Mismanagement & Negative & EU & No myth & NA & NA & NA & NA & NA & NA & NA & NA & Greece & καταπέλτης η κομισιόν για την πορεία των έργων της ευαθ & 2018-04-27 & ευρωπαϊκό ταμείο περιφερειακής ανάπτυξης & η επίτροπος περιφεριακής ανάπτυξης κορίνα κρέτσου επισημαίνει καθυστερήσεις και κίνδυνο να επηρεαστεί η ολοκλήρωση του έργου του τηλεελέγχου. τον κώδωνα του κινδύνου να μην ολοκληρωθεί ως τα τέλη του 2020 το έργο προϋπολογισμού 4.000.000 ευρώ "τηλεέλεγχος και αυτοματισμός του συστήματος ύδρευσης της περιοχής εξυπηρέτησης της ευαθ, α.ε.", κρούει η επίτροπος περιφεριακής ανάπτυξης κορίνα κρέτσου. είχε προηγηθεί κατεπείγουσα ερώτηση από την ευρωβουλευτή της νδ και του ελκ μαρία σπυράκη για τη χρηματοδότηση από κοινοτικούς πόρους έργων υδροδότησης της ευαθ. "όσον αφορά την προγραμματική περίοδο 2014-2020, στις 4 απριλίου 2017, εγκρίθηκε για χρηματοδότηση από τη διαχειριστική αρχή κεντρικής μακεδονίας το έργο "τηλεέλεγχος και αυτοματισμός του συστήματος ύδρευσης της περιοχής εξυπηρέτησης της ευαθ, α.ε." (4 εκατ. ευρώ). ωστόσο, σημειώθηκε καθυστέρηση στη διαδικασία του διαγωνισμού και τα έγγραφα του διαγωνισμού υποβλήθηκαν προς έγκριση στη διαχειριστική αρχή στις 9 μαρτίου 2018. υπάρχει ο κίνδυνος η καθυστέρηση αυτή να επηρεάσει την ολοκλήρωση του έργου, η οποία αρχικά προβλεπόταν για τα τέλη του 2020. επιπλέον, η ευαθ, α.ε. υπέβαλε, στις 9 μαρτίου 2018, χωριστή αίτηση συγχρηματοδότησης για την επέκταση των εγκαταστάσεων επεξεργασίας υδάτων της θεσσαλονίκης. η εν λόγω αίτηση βρίσκεται στο στάδιο της αξιολόγησης", απάντησε η κ. κρέτσου. αναλυτικά η ερώτηση της μαρίας σπυράκη έχει ως εξής: θέμα: προβλήματα υδροδότησης στη θεσσαλονίκη για έβδομη συνεχή ημέρα παρατηρούνται προβλήματα στην υδροδότηση της θεσσαλονίκης. ειδικότερα, σε πολλές περιοχές έχουν εδώ και μέρες παντελή έλλειψη νερού, ενώ περιοδικές διακοπές παρατηρούνται και σε περιοχές όπου έχει αποκατασταθεί το πρόβλημα. ως αποτέλεσμα, ευπαθείς ομάδες όπως ηλικιωμένοι και παιδιά στις εν λόγω περιοχές δεν έχουν πρόσβαση σε καθαρό, πόσιμο νερό, καθώς έχει παρατηρηθεί θολότητα του νερού στις περιοχές όπου επανέρχεται μερικώς η υδροδότηση. επιπρόσθετα, στην ιστοσελίδα της αρμόδιας εταιρίας ύδρευσης \& αποχέτευσης θεσσαλονίκης α.ε. φαίνεται να έχει ενταχθεί έργο με τίτλο "τηλεέλεγχος και αυτοματισμός του συστήματος ύδρευσης της περιοχής εξυπηρέτησης της ευαθ", προϋπολογισμού 4,15 εκ. ευρώ. έχοντας υπόψη ότι το πρόβλημα οφείλεται σε κακή συντήρηση των υποδομών και δικτύων διανομής του νερού, ερωτάται η επιτροπή: 1. τι προβλέπεται για τη χρηματοδότηση από κοινοτικούς πόρους έργων υδροδότησης; 2. έχει χρηματοδοτηθεί κατά το παρελθόν η εταιρία ύδρευσης \& αποχέτευσης θεσσαλονίκης α.ε. από κοινοτικούς πόρους και έχει υποβάλει σχετικά σχέδια για χρηματοδότηση βελτίωσης υποδομών δικτύου την τελευταία πενταετία; 3. σε ποια φάση υλοποίησης βρίσκεται το συγχρηματοδοτούμενο έργο και πότε προβλέπεται η ολοκλήρωσή του; η απάντηση της επιτρόπου κ. κρέτου έχει ως εξής: 1. στο πλαίσιο της συμφωνίας εταιρικής σχέσης για την περίοδο 2014-2020 προβλέπονται συγκεκριμένες δράσεις για τον έλεγχο και τη μείωση των διαρροών, καθώς και για την αποκατάσταση, την επέκταση και τον εκσυγχρονισμό των εγκαταστάσεων παροχής και διανομής ύδατος. 2. κατά την προγραμματική περίοδο 2007-2013 η εταιρεία ύδρευσης και αποχέτευσης θεσσαλονίκης ("ευαθ, α.ε.") δεν έλαβε καμία χρηματοδότηση από το ευρωπαϊκό ταμείο περιφερειακής ανάπτυξης ούτε από το ταμείο συνοχής. η ευαθ, α.ε. έλαβε εκτεταμένη χρηματοδότηση κατά την περίοδο 2000-2006, μεταξύ άλλων και για το έργο της υδροδότησης της θεσσαλονίκης από τη λεκάνη του αλιάκμονα. 3. όσον αφορά την προγραμματική περίοδο 2014-2020, στις 4 απριλίου 2017, εγκρίθηκε για χρηματοδότηση από τη διαχειριστική αρχή κεντρικής μακεδονίας το έργο "τηλεέλεγχος και αυτοματισμός του συστήματος ύδρευσης της περιοχής εξυπηρέτησης της ευαθ, α.ε." (4 εκατ. ευρώ). ωστόσο, σημειώθηκε καθυστέρηση στη διαδικασία του διαγωνισμού και τα έγγραφα του διαγωνισμού υποβλήθηκαν προς έγκριση στη διαχειριστική αρχή στις 9 μαρτίου 2018. υπάρχει ο κίνδυνος η καθυστέρηση αυτή να επηρεάσει την ολοκλήρωση του έργου, η οποία αρχικά προβλεπόταν για τα τέλη του 2020. επιπλέον, η ευαθ, α.ε. υπέβαλε, στις 9 μαρτίου 2018, χωριστή αίτηση συγχρηματοδότησης για την επέκταση των εγκαταστάσεων επεξεργασίας υδάτων της θεσσαλονίκης. η εν λόγω αίτηση βρίσκεται στο στάδιο της αξιολόγησης. & 603 & high & High & Governance & NA & NA & 2018-04-27 & 2018 & 3 & POL
Frame & high-very high & Regional & 500-1000 & 1.8924259 & 1.3590316 & -1.2391092 & 0.9814602 & -0.7922579 & 0.0 & -0.9049211 & -1.0736569 & Recipient & European & European & European & European|POL & Negative\\
Greece & https://www.voria.gr/article/erchete-ton-oktovrio-to-4th-thessaloniki-animation-festival & 365 & Ernst\&Young: Φρένο στις δημόσιες εγγραφές παγκοσμίως & Private/Non-Public & Online only & Regional/Local & low = CP mentioned more times but NOT important part of story (mainly about others issues) & Research \& innovation & Positive & EU + National + Subnational & No myth & NA & NA & NA & NA & NA & NA & NA & NA & Greece & έρχεται τον οκτώβριο το 4th thessaloniki animation festival & 2018-10-09 & ευρωπαϊκό κοινωνικό ταμείο & τo δεύτερο μεγαλύτερο animation festival στην ελλάδα παρουσιάζει η addart σε συνδιοργάνωση με τον δήμο θεσσαλονίκης και το τμήμα τουρισμού. για τέταρτη χρονιά έρχεται στη θεσσαλονίκη το animation festival στο κέντρο της πόλης από τις 18-21 οκτωβρίου. τo δεύτερο μεγαλύτερο animation festival στην ελλάδα παρουσιάζει η addart σε συνδιοργάνωση με τον δήμο θεσσαλονίκης και το τμήμα τουρισμού. προβολές θα γίνονται στον ιστορικό κεντρικό κινηματογράφο "μακεδονικον" και επιπλέον, στη δημοτική κεντρική βιβλιοθήκη θα πραγματοποιούνται εργαστήρια και παιδικές προβολές. μερικοί προσκεκλημένοι: -tim allen, βασικός animator στο isle of dogs του wes anderson -nick \& nancy phelps μουσικοσυνθέτης σε animation ταινίες -federico vallarino από την toon boom. -ίριδα ζιόγκα που θα παρουσιάσει το νέο της επιτυχημένο project man wanted -νίκος δεληγκάρης που θα παρουσιάσει την ψυχολογία και σημασία στην σύνθεση και χρώμα της εικόνας. θα πραγματοποιηθούν ομιλίες και πάνελ με θέμα το animation και τις εφαρμογές του, αλλά και τις ευκαιρίες που υπάρχουν στο χώρο του animation στην ελλάδα. φέτος, για πρώτη φορά στην ιστορία του, ο θεσμός taf thessaloniki animation festival, φιλοξενεί μια ιδιαίτερη ενότητα με θέμα "animearth". η ενότητα αυτή υλοποιείται στο πλαίσιο του επιχειρησιακού προγράμματος "ανάπτυξη ανθρώπινου δυναμικού, εκπαίδευση και δια βίου μάθηση" και συγχρηματοδοτείται από την ευρωπαϊκή ένωση (ευρωπαϊκό κοινωνικό ταμείο) και από εθνικούς πόρους. πρόκειται ουσιαστικά για το επιχειρησιακό πρόγραμμα του εσπα που ασχολείται με την εκπαίδευση και την απασχόληση. μερικοί από τους στόχους της προσπάθειας αφορούν: - στην ανάπτυξη και αξιοποίηση των ικανοτήτων του ανθρώπινου δυναμικού της χώρας, - στη βελτίωση της ποιότητας της εκπαίδευσης, της δια βίου μάθησης, της ευαισθητοποίησης για την περιβαλλοντική εκπαίδευση, - στη σύνδεση της εκπαίδευσης με την αγορά εργασίας και - στην ενδυνάμωση της κοινωνικής συνοχής και της κοινωνικής ένταξης ευάλωτων κοινωνικά ομάδων (εκο). η προβολή των ταινιών του αφιερώματος είναι δωρεάν για το κοινό και εξασφαλίζεται η πρόσβαση σε αμεα. special συλλογές: -το ουκρανικό και ουγγρικό πανόραμα animation -τα αφιερώματα human rigths animated με την στήριξη του ιδρύματος heinrich-böll-stiftung gr όπου μέσα από την τέχνη του animation θα προβάλλονται επιλεγμένες ταινίες μικρού μήκους που θα θίγουν ή θα αναδεικνύουν θέματα ανθρωπίνων δικαιωμάτων. επιστρέφουν οι chloé alliez \& violette delvoye καλεσμένες από πέρυσι όπου θα παρουσιάσουν σε παγκόσμια πρεμιέρα την ολοκληρωμένη σειρά τους world in (stop) motion. ένα επεισόδιο της το "ruined picture" γυρίστηκε στο πλαίσιο του taf πέρυσι μαζί με συμμετέχοντες σε διήμερο εργαστήριο stop motion. τα πρωινά του σαββάτου και της κυριακής θα διεξαχθούν προβολές και workshops για παιδιά. στο taf θα διεξαχθεί διαγωνισμός ταινιών animation μικρού μήκους και θα υπάρξουν βραβεύσεις σε διάφορες κατηγορίες. & 412 & low & Low & Socio-Economic & NA & NA & 2018-10-09 & 2018 & 3 & ECO
Frame & low-medium & Regional & <500 & 1.8924259 & 1.3590316 & -1.2391092 & 0.9814602 & -0.7922579 & 0.0 & -0.9049211 & -1.0736569 & Recipient & Domestic & European & Mixed & Domestic|ECO & Positive\\
Greece & http://www.rodiaki.gr/article/374279/paroysiash-toy-ergoy-co-create-sth-rodo & 344 & Rodiaki.gr & Private/Non-Public & Online and Offline & Regional/Local & medium = CP is important part of story & Research \& innovation & Factual & Subnational & No myth & NA & NA & NA & NA & NA & NA & NA & NA & Greece & παρουσίαση του  έργου co-create στη ρόδο | η ροδιακη & 2017-09-25 & ευρωπαϊκό ταμείο περιφερειακής ανάπτυξης & η περιφέρεια νοτίου αιγαίου και η ενεργειακη αε διοργανώνουν εκδήλωση παρουσίασης του έργου co-create, ενός σημαντικού οχήματος για την ανάπτυξη των δημιουργικών κλάδων. η εκδήλωση θα πραγματοποιηθεί την πέμπτη 28 σεπτεμβρίου στις 11 το πρωί στην αίθουσα του περιφερειακού συμβουλίου. το έργο συγχρηματοδοτείται από το ευρωπαϊκό ταμείο περιφερειακής ανάπτυξης και από εθνικούς πόρους και υλοποιείται στο πλαίσιο του διακρατικού προγράμματος interreg med. & 63 & medium & Medium & Socio-Economic & NA & NA & 2017-09-25 & 2017 & 2 & ECO
Frame & low-medium & Regional & <500 & 1.8924259 & 1.3590316 & -1.2391092 & 0.9814602 & -0.7922579 & 0.0 & -0.9049211 & -1.0736569 & Recipient & Domestic & Domestic & Domestic & Domestic|ECO & Neutral\\
\addlinespace
Greece & http://www.in.gr/entertainment/cinema/festival/article/?aid=1500126670 & 328 & in & Private/Non-Public & Online only & National & very low = CP mentioned once & Cultural development & Positive & EU + Subnational & No myth & NA & NA & NA & NA & NA & NA & NA & NA & Greece & 19ο φνθ: από τον άλφρεντ χίτσκοκ στους rolling stones & 2017-01-24 & ευρωπαϊκό ταμείο περιφερειακής ανάπτυξης & rolling stones, placebo, laibach. τρεις κορυφαίες ροκ μπάντες σε εκρηκτικές εμφανίσεις θα μας δώσουν μια πρώτη γεύση από τα μουσικά ντοκιμαντέρ που θα δούμε στο πολυαγαπημένο καθιερωμένο τμήμα της μουσικής, στο 19ο φεστιβάλ ντοκιμαντέρ θεσσαλονίκης. πλάι στη μουσική, μία άλλη μεγάλη αγάπη, ο κινηματογράφος πρωταγωνιστεί σε μια νέα ενότητα που εγκαινιάζεται φέτος με τίτλο σινεμά και ξεναγεί τους θεατές σε ρεύματα και αξέχαστες στιγμές της έβδομης τέχνης. το φεστιβάλ παρουσιάζει τη μαγεία του άλφρεντ χίτσκοκ και του αμπάς κιαροστάμι και αποκαλύπτει τα μυστικά της πιο διάσημης κινηματογραφικής σχολής της ευρώπης. μουσική σινεμά το 19ο φεστιβάλ ντοκιμαντέρ θεσσαλονίκης χρηματοδοτείται από την ευρωπαϊκή ένωση - ευρωπαϊκό ταμείο περιφερειακής ανάπτυξης, στο πλαίσιο του πεπ κεντρικής μακεδονίας 2014-2020. & 114 & very low & Low & Socio-Economic & NA & NA & 2017-01-24 & 2017 & 2 & ECO
Frame & v.low & National & <500 & 1.8924259 & 1.3590316 & -1.2391092 & 0.9814602 & -0.7922579 & 0.0 & -0.9049211 & -1.0736569 & Recipient & Domestic & European & Mixed & Domestic|ECO & Positive\\
Greece & http://www.news.gr/oikonomia/oikonomika-nea/article/259009/efkairia-gia-10-dis-evro-apo-adiatheta-kondylia-th.html & 294 & News.gr & Private/Non-Public & Online only & National & high = CP is most important issue in story (can also cover other issues) & Solidarity to poor countries/regions & Positive & EU & NA & Social awareness/inclusion & Positive & EU & No myth & Institutional bargaining over funding & Positive & EU + Other country & No myth & Greece & ευκαιρία για 10 δισ. ευρώ από αδιάθετα κονδύλια της ε.ε. για το προσφυγικό - news.gr & 2016-03-18 & διαρθρωτικά ταμεία & έως και 10 δισ. ευρώ θα μπορούσαν να βρεθούν για να βοηθήσουν στην αντιμετώπιση της προσφυγικής κρίσης από τη χρήση των αδιάθετων χρημάτων από τον προϋπολογισμό της ευρωπαϊκής ένωσης σύμφωνα με δημοσίευμα της euractiv. πέντε κορυφαίοι ευρωβουλευτές από την επιτροπή προϋπολογισμού του ευρωπαϊκού κοινοβουλίου και της επιτροπής περιφερειακής ανάπτυξης βρίσκονται πίσω από την πρωτοβουλία που θα αναζητήσει τα αδιάθετα χρήματα από τα 430 δισ. ευρώ των πόρων των διαρθρωτικών ταμείων από την περίοδο 2007-2013, προκειμένου να δώσουν μία άμεση λύση στην προσφυγική κρίση. για τα χρήματα δεν θα απαιτούνται νέα κεφάλαια από τα κράτη μέλη, διότι αφορούν πόρους που είχαν ήδη δεσμευτεί για την υλοποίηση έργων, τα οποία εν τέλει δεν πραγματοποιήθηκαν ποτέ. τα διαρθρωτικά ταμεία χρηματοδοτούν έργα τα οποία συμβάλουν στην περιφερειακή σύγκλιση και τη μείωση των ανισοτήτων στο εισόδημα και τον πλούτο. ωστόσο δεν βρίσκουν όλα τα έργα δημόσιους ή ιδιωτικούς τελικούς δικαιούχους. λόγω του ότι οι πόροι αυτοί προέρχονται από τα αδιάθετα χρήματα από την προγραμματική περίοδο 2007-2013, η χρήση τους δεν θα προκαλέσει κάποια επιπλέον επιβάρυνση στον προϋπολογισμό της ένωσης. υπό κανονικές συνθήκες, τα χρήματα αυτά θα επέστρεφαν (κατ' αναλογία) στις εθνικές κυβερνήσεις που τα είχαν συνεισφέρει. το ακριβές ποσό το οποίο φτάνουν αυτά τα χρήματα δεν έχει ακόμα καθοριστεί καθώς οι λογαριασμοί για την περίοδο του προϋπολογισμού δεν έχουν εκκαθαριστεί ακόμα. η euractiv έθεσε την πρόταση υπόψιν της ευρωπαϊκής επιτροπής, η οποία συμφώνησε πως ως ιδέα έχει αξία. ένας εκπρόσωπος της επιτροπής σημείωσε: "κάθε προσπάθεια πρέπει να καταβληθεί προκειμένου να δούμε κατά πόσο μπορούμε να εξαντλήσουμε την ευελιξία του προϋπολογισμού της εε. θα εξετάσουμε προσεκτικά την πρόταση αυτή". τα πολυπόθητα χρήματα θα είναι ευπρόσδεκτα από τους ηγέτες της εε, οι οποίοι συναντώνται στις βρυξέλλες σήμερα και αύριο προκειμένου να καταλήξουν σε μία συμφωνία με την τουρκία για να περιορίσουν τη ροή των προσφύγων προς την εε. τόσο αυτοί, όσο και η επιτροπή και το κοινοβούλιο, θα πρέπει να υποστηρίξουν την ιδέα αν πρόκειται να εφαρμοστεί. οι επικεφαλής των κυβερνήσεων συζητούν επίσης πρόσθετη χρηματοδότηση για την ελλάδα προκειμένου να διαχειριστεί την άφιξη των προσφύγων στα νησιά από την τουρκία, καθώς και ένα πακέτο συνολικά 6 δις για την άγκυρα, προκειμένου να κρατήσει τους πρόσφυγες στην τουρκία. η γερμανία καγκελάριος άγκελα μέρκελ κάλεσε την ευρωπαϊκή επιτροπή να βρει πρόσθετη χρηματοδότηση για την κρίση από τους υπάρχοντες πόρους της εε. η επιτροπή εκτιμά ότι περίπου 2 με 10 δισ. ευρώ θα μπορούσαν αν είναι τα αδιάθετα υπόλοιπα από την προγραμματική περίοδο 2007-2013, που ονομάζεται και πολυετές δημοσιονομικό πλαίσιο. το ακριβές νούμερο αναμένεται στις αρχές του 2017. σύμφωνα με την πρόταση, τα χρήματα αυτά θα συγκεντρωνόντουσαν σε ένα προσωρινό ταμείο αντιμετώπισης της προσφυγικής κρίσης (πταπκ), υπό την αιγίδα της ευρωπαϊκής επιτροπής. η μαρία σπυράκη, ηγείται αυτής της πρωτοβουλίας της ομάδας των ευρωβουλευτών του ευρωπαϊκού λαϊκού κόμματος. η σπυράκη δήλωσε: "αντί να συζητάμε επ αόριστο για το ποιος θα πρέπει να πληρώσει περισσότερο και το που θα πρέπει να πάνε τα χρήματα, η επιτροπή θα πρέπει να φροντίσει ούτως ώστε όλοι οι διαθέσιμοι πόροι να αξιοποιηθούν προκειμένου να υποστηριχθούν κράτη μέλη που έχουν πληγεί περισσότερο από την κρίση, ακόμα και αν τα χρήματα αυτά θα έπρεπε να επιστρέψουν πίσω στα κράτη μέλη." η ελλάδα έχει υποφέρει τα μέγιστα από την κρίση και θα μπορούσε να επωφεληθεί περισσότερο από όλους από τα ανακυκλούμενα χρήματα. "πρέπει να υποστηρίξουμε επίσης και κράτη μέλη όπως η γερμανία, η αυστρία και η σουηδία τα οποία λαμβάνουν έκτακτα μέτρα προκειμένου να αντιμετωπίζουν το μακροχρόνιο αντίκτυπο της προσφυγικής κρίσης", σημείωσε η σπυράκη. η σπυράκη και οι άλλοι ευρωβουλευτές πιστεύουν ότι θα πρέπει να παρθεί μία απόφαση ως προς το ποιες χώρες θα πάρουν χρήματα και πόσα. αυτό θα πρέπει να γίνει με κριτήριο τους πρόσφυγες που καλείται η κάθε χώρα να διαχειριστεί. η ιδέα ήδη κυκλοφορεί στους κύκλους του ελκ, το οποίο είναι το μεγαλύτερο κόμμα στο ευρωπαϊκό κοινοβούλιο. έχει ήδη την υποστήριξη από μεγάλα ονόματα της επιτροπής προϋπολογισμών όπως τους lambert van nistelrooij, τον jean-marian marinescu, τον reimer böge, και τον josé manuel fernandes. η σπυράκη είναι σίγουρη ότι εν τέλει το κόμμα της θα στηρίξει την πρότασή της. ο πρόεδρος της επιτροπής jean-claude juncker, ο πρόεδρος του συμβουλίου donald tusk - που προεδρεύει των σημερινών συζητήσεων για το προσφυγικό - αλλά και η γερμανίδα καγκελάριος άγκελα μέρκελ είναι όλοι μέλη του ιδίου κεντροδεξιού κόμματος. το σχέδιο αυτό θα μπορούσε να συζητηθεί στο ευρωπαϊκό κοινοβούλιο τον απρίλιο, κατά τη διάρκεια της συζήτησης για την τροποποίηση του προϋπολογισμού της ένωσης που έχει ήδη προτείνει η επιτροπή για την αντιμετώπιση της προσφυγικής κρίσης. το συμβούλιο της ευρώπης θα μπορούσε, εν συνεχεία να αντιδράσει, σε οποιαδήποτε τροπολογία γινόταν από τα μέλη του ευρωπαϊκού κοινοβουλίου. & 768 & high & High & Values & Socio-Economic & Power & 2016-03-18 & 2016 & 2 & ECO
Frame & high-very high & National & 500-1000 & 1.8924259 & 1.3590316 & -1.2391092 & 0.9814602 & -0.7922579 & 0.0 & -0.9049211 & -1.0736569 & Recipient & European & European & European & European|ECO & Positive\\
Greece & http://www.rodiaki.gr/article/387784/egkrithhkan-oi-oroi-diakhryxhs-gia-to-parko-toy-rodinioy & 388 & Rodiaki.gr & Private/Non-Public & Online and Offline & Regional/Local & very low = CP mentioned once & Environment/green/low-carbon & Factual & Subnational & No myth & NA & NA & NA & NA & NA & NA & NA & NA & Greece & εγκρίθηκαν οι όροι διακήρυξης για το πάρκο του ροδινιού | η ροδιακη & 2018-04-05 & ευρωπαϊκό ταμείο περιφερειακής ανάπτυξης & τους όρους διακήρυξης για το έργο "προστασία και ανάδειξη του πάρκου ροδίνι", ενέκρινε στη χθεσινή της συνεδρίαση η οικονομική επιτροπή του δήμου ρόδου και αμέσως μετά το πάσχα αναμένεται να βγει "στον αέρα" με διεθνή διαγωνισμό ενώ θα πρέπει να περάσει και από το ελεγκτικό συνέδριο. το έργο είναι συνολικού προϋπολογισμού 560.000 ευρώ, είναι ενταγμένο στο επιχειρησιακό πρόγραμμα "νότιο αιγαίο 2014-2020" και συγχρηματοδοτείται από το ευρωπαϊκό ταμείο περιφερειακής ανάπτυξης. ο προϋπολογισμός του έργου που θα δημοπρατηθεί αναλύεται ως εξής: δαπάνη εργασιών 328.466,25 ευρώ, γενικά έξοδα-όφελος εργολάβου 59.123,93 ευρώ και απρόβλεπτα 58.138,53 ευρώ. οι εργασίες που προβλέπεται να γίνουν σ' αυτόν τον εμβληματικό χώρο είναι, μεταξύ άλλων: διαμόρφωση μονοπατιών πρόσβασης και περιπάτου, καθαρισμός των μονοπατιών από τα υλικά των κατολισθήσεων και προσχώσεων και διαμόρφωση της επιφάνειας τους όπου απαιτείται. θα γίνουν εργασίες στήριξης των μονοπατιών, αποκατάσταση και ανακατασκευή των τειχών και πρανών, επίσης θα γίνει εκσκαφή δια χειρών για την απομάκρυνση των υλικών και τη διαμόρφωση θεμελίου, θα γίνει κατασκευή τοίχου από σκυρόδεμα, τείχη με λιθοδομή από πωρόλιθο και άλλες πολλές εργασίες προκειμένου, όπως είπε ο πρόεδρος της οικονομικής, αντιδήμαρχος, κ. σάββας διακοσταματίου, το ροδίνι να αποκτήσει πλέον την όψη και τη λειτουργικότητα προκειμένου να δεχθεί εκ νέου επισκέπτες. ακόμα, σ' αυτή την τελευταία συνεδρίαση της οικονομικής, αναδείχθηκε μειοδότης για την παροχή κέτερινγκ στο μουσικό γυμνάσιο ρόδου, ώστε να έχουν τα παιδιά σίτιση για το έτος 2018. επίσης, εγκρίθηκε ο διαγωνισμός για τον καθαρισμό των πάρκων του δήμου ρόδου, προϋπολογισμού 78 χιλιάδων ευρώ. τέλος, η οικονομική, χθες, επικύρωσε το αποτέλεσμα του διαγωνισμού για τον ανάδοχο που θα αναλάβει τη φύλαξη της κοιλάδας των πεταλούδων καθ' όλη τη διάρκεια της νέας τουριστικής περιόδου. & 280 & very low & Low & Socio-Economic & NA & NA & 2018-04-05 & 2018 & 3 & ECO
Frame & v.low & Regional & <500 & 1.8924259 & 1.3590316 & -1.2391092 & 0.9814602 & -0.7922579 & 0.0 & -0.9049211 & -1.0736569 & Recipient & Domestic & Domestic & Domestic & Domestic|ECO & Neutral\\
Greece & http://www.kathimerini.gr/833675/article/oikonomia/epixeirhseis/anw-toy-11-dis-sthn-agora-mesw-espa-ews-telos-oktwvrioy & 332 & kathimerini.gr & Private/Non-Public & Online and Offline & National & very high = CP is most important issue + CP is mentioned in title/headline & Improve governance & Positive & EU & No myth & Institutional bargaining over funding & Negative & EU & No myth & NA & NA & NA & NA & Greece & ανω του 1,1 δισ. στην αγορά μέσω εσπα έως τέλος οκτωβρίου & 2015-10-09 & περιφερειακή πολιτική & με αφορμή την ψήφιση σήμερα από το ευρωκοινοβούλιο της τροποποίησης του κανονισμού, που επιτρέπει την καταβολή πριν από την προγραμματισμένη περίοδο του συνόλου της κοινοτικής συμμετοχής, δηλαδή το 95\% των πόρων του εσπα, η αρμόδια επίτροπος για την περιφερειακή πολιτική, κορίνα κρέτσου, αναγνώρισε τη δυσκολία του εγχειρήματος. ρευστότητα άνω του 1,1 δισ. ευρώ αναμένεται έως τα τέλη οκτωβρίου από την ευρωπαϊκή επιτροπή για την ολοκλήρωση του παλιού εσπα έως τα τέλη του χρόνου. από αυτά, τα 825 εκατ. ευρώ θα δοθούν στο πλαίσιο της απόφασης για τη μείωση της εθνικής συμμετοχής στο παλιό εσπα και της αυξημένης προκαταβολής που δίνεται στο νέο εσπα, ενώ σήμερα η ελληνική πλευρά υποβάλλει αίτημα πληρωμών για 325 εκατ. ευρώ, προκειμένου να πληρωθούν οφειλές προς δικαιούχους. στα κονδύλια που θα εισρεύσουν το προσεχές διάστημα για την αποπληρωμή έργων του εσπα, περιλαμβάνονται άλλα 300 εκατ. ευρώ που δόθηκαν ως προκαταβολή από την ετεπ έναντι ενός νέου δανείου 1 δισ. ευρώ, επιπλέον 50 εκατ. ευρώ ως επέκταση της σύμβασης ενός παλαιότερου δανείου και 270 εκατ. ευρώ επίσης από την ετεπ για τους αυτοκινητόδρομους. με δεδομένη πια την ανοχή της ε.ε. για την ευέλικτη εφαρμογή των κανονισμών των διαρθρωτικών ταμείων και τη βούληση να μη χαθούν κοινοτικοί πόροι, ζητούμενο πλέον αποτελεί η επιτάχυνση από την πλευρά της χώρας, προκειμένου όσα έργα δεν μεταφερθούν στο επόμενο εσπα, να ολοκληρωθούν έως τα τέλη του 2015 και να αποφευχθεί η επιβάρυνση των προγράμματος δημοσίων επενδύσεων. με αφορμή την ψήφιση σήμερα από το ευρωκοινοβούλιο της τροποποίησης του κανονισμού, που επιτρέπει την καταβολή πριν από την προγραμματισμένη περίοδο του συνόλου της κοινοτικής συμμετοχής, δηλαδή το 95\% των πόρων του εσπα, η αρμόδια επίτροπος για την περιφερειακή πολιτική, κορίνα κρέτσου, αναγνώρισε τη δυσκολία του εγχειρήματος. σε δηλώσεις της χθες μετά το τέλος της συνάντησης με τον υπουργό αλλά και τον υφυπουργό οικονομίας, γιώργο σταθάκη και αλέξη χαρίτση, παραδέχθηκε ότι "δεν ήταν εύκολο να υποστηρίξω στο κοινοβούλιο τα ειδικά μέτρα". με τον τρόπο αυτό σχολίασε εμμέσως πλην σαφώς τις αντιρρήσεις της χριστιανικής ενωσης στο ευρωπαϊκό κοινοβούλιο που αντιτίθεται στην αλλαγή των κανονισμών που επιτρέπουν την πρόωρη εκταμίευση έως τα τέλη του 2015 της κοινοτικής συμμετοχής. είναι χαρακτηριστικό ότι ο χέρμπερτ ρόιλ από το χριστιανοδημοκρατικό κόμμα, επικεφαλής της χριστιανικής ενωσης στο ευρωπαϊκό κοινοβούλιο, αποκάλεσε το σχέδιο "εξωφρενικό", ενώ ο επίσης χριστιανοδημοκράτης γιόαχιμ τσέλερ, αντιπρόεδρος της επιτροπής περιφερειακής ανάπτυξης, κάνει λόγο για "αμαρτωλή περίπτωση". μεταξύ αυτών που διαφωνούν είναι και ο γερμανός υπουργός οικονομικών βόλφγκανγκ σόιμπλε. μέσα από τις δύο σχετικές ρυθμίσεις -εκταμίευση του συνόλου της κοινοτικής συμμετοχής για το παλιό εσπα και αύξηση της προκαταβολής από το 7\% στο 14\% για το νέο εσπα- που συνολικά αθροίζουν 2 δισ. ευρώ, η ελληνική πλευρά προσδοκά να ολοκληρώσει τα έργα του παλιού εσπα, έστω και με περικοπές. ηδη από τα τέλη σεπτεμβρίου έχει στείλει, σύμφωνα με τον υφυπουργό οικονομίας αλέξη χαρίτση, τα αναθεωρημένα σχέδια, από τα οποία προκύπτουν ποια έργα θα ολοκληρωθούν, ποια όχι και ποια θα μεταφερθούν στην επόμενη περίοδο. συνολικά το παλιό εσπα θα κλείσει με περίπου 20 δισ. ευρώ κοινοτικής χρηματοδότησης. για την αποφυγή, πάντως, απώλειας κοινοτικών πόρων ακόμη και μετά την αλλαγή του κανονισμού -εξέλιξη που θα κρίνει κατά πόσο η χώρα μας θα αιτηθεί παράταση στην υλοποίηση του εσπα- το υπουργείο θα πρέπει να επιταχύνει τις πληρωμές το τρίμηνο που απομένει έως τα τέλη του 2015, εξοφλώντας τις οφειλές του. αντίστοιχα, οι αρμόδιοι φορείς θα πρέπει να επισπεύσουν τις εργασίες σε έργα που υλοποιούνται ήδη. είναι χαρακτηριστικό ότι μετά μια μακρά περίοδο καθυστερήσεων οι εργοληπτικές επιχειρήσεις καλούνται τώρα να δουλέψουν διπλοβάρδιες και να ολοκληρώσουν όσο μεγαλύτερο μέρος γίνεται από το φυσικό αντικείμενο κάθε έργου, προκειμένου τα μεγάλα κυρίως έργα να μη μεταφερθούν στο νέο εσπα. & 604 & very high & High & Governance & Power & NA & 2015-10-09 & 2015 & 1 & POL
Frame & high-very high & National & 500-1000 & 1.8924259 & 1.3590316 & -1.2391092 & 0.9814602 & -0.7922579 & 0.0 & -0.9049211 & -1.0736569 & Recipient & European & European & European & European|POL & Positive\\
Greece & http://e-thessalia.gr/oaed-xekina-epichorigoumeno-programma-gia-proslipsi-10-000-anergon/ & 308 & e-thessalia.gr & Private/Non-Public & Online only & Regional/Local & very low = CP mentioned once & Jobs & Factual & National & No myth & NA & NA & NA & NA & NA & NA & NA & NA & Greece & οαεδ: ξεκινά επιχορηγούμενο πρόγραμμα για πρόσληψη 10.000 ανέργων - e-thessalia.gr & 2016-12-02 & ευρωπαϊκό κοινωνικό ταμείο & την έναρξη υλοποίησης του προγράμματος επιχορήγησης επιχειρήσεων για την πρόσληψη 10.000 ανέργων ηλικίας 30-49 ετών ανακοίνωσε ο οαεδ. πρόκειται για τους εγγεγραμμένους στα μητρώα του οργανισμού για χρονικό διάστημα τουλάχιστον τριών μηνών και άνω και με έμφαση στους μακροχρόνια ανέργους με στόχο να προσληφθούν σε νέες θέσεις πλήρους απασχόλησης, σε ιδιωτικές επιχειρήσεις, φορείς κοινωνικής και αλληλέγγυας οικονομίας και γενικά εργοδότες του ιδιωτικού τομέα που ασκούν τακτικά οικονομική δραστηριότητα. οι επιχειρήσεις που ενδιαφέρονται να συμμετάσχουν στο πρόγραμμα θα μπορούν να υποβάλουν την αίτηση τους ηλεκτρονικά μόνο μέσω του πληροφοριακού συστήματος κρατικών ενισχύσεων (πσκε) του υπουργείου οικονομίας και ανάπτυξης (στη διεύθυνση http://www.ependyseis.gr/mis). προϋπόθεση για την ένταξη της επιχείρησης στο πρόγραμμα είναι να μην έχει προβεί πρώτον, σε μείωση του προσωπικού της λόγω καταγγελίας σύμβασης (απόλυση χωρίς αντικατάσταση) στο χρονικό διάστημα των τριών μηνών (ημερολογιακά) που προηγούνται της ημερομηνίας υποβολής της αίτησης υπαγωγής στο πρόγραμμα και δεύτερον, σε αλλαγή του καθεστώτος απασχόλησης -το ίδιο χρονικό διάστημα- για το σύνολο του προσωπικού που δηλώνει κατά την αίτηση υπαγωγής της στο πρόγραμμα. κάθε επιχείρηση μπορεί να ενταχθεί στο πρόγραμμα για έναν έως και πέντε ωφελούμενους, ανάλογα με το προσωπικό που απασχολεί κατά την υποβολή της αίτησής της. οι ωφελούμενοι πρέπει να είναι: i) άνεργοι εγγεγραμμένοι στο μητρώο ανέργων του οαεδ για χρονικό διάστημα τουλάχιστον τριών μηνών πριν από την υπόδειξή τους από τα αρμόδια κπα2 και ii) μακροχρόνια άνεργοι, δηλαδή εγγεγραμμένοι στο μητρώο ανέργων του οαεδ για χρονικό διάστημα τουλάχιστον 12 μηνών μέχρι την υπόδειξή τους από τα αρμόδια κπα2. επίσης πρέπει να είναι ελληνες πολίτες ή πολίτες άλλου κράτους μέλους της εε ή να είναι ομογενείς που έχουν δικαίωμα διαμονής και απασχόλησης στη χώρα μας ή πολίτες τρίτων χωρώ,ν που έχουν άδεια διαμονής για εξαρτημένη εργασία τουλάχιστον για όσο χρονικό διάστημα διαρκεί το πρόγραμμα. οι ωφελούμενοι άνεργοι δεν υποβάλλουν οι ίδιοι αίτηση, αλλά υποδεικνύονται με συστατικό σημείωμα μόνο από την αρμόδια υπηρεσία (κπα2), όπου ανήκει η έδρα ή το υποκατάστημα που θα απασχοληθούν σύμφωνα με όσα ορίζει η δημόσια πρόσκληση. ο οαεδ θα καταβάλει στις επιχειρήσεις που θα ενταχθούν στο πρόγραμμα επιχορήγηση, η οποία ανέρχεται στο ποσό των 15 ευρώ την ημέρα, για την πρόσληψη ανέργων εγγεγραμμένων στα μητρώα του για χρονικό διάστημα τριών έως και 12 μηνών και στο ποσό των 18 ευρώ για την πρόσληψη μακροχρόνια ανέργων (εγγεγραμμένων στα μητρώα ανέργων για διάστημα άνω των 12 μηνών). η επιχορήγηση των επιχειρήσεων για κάθε προσλαμβανόμενο άνεργο θα ξεκινά από την ημερομηνία πρόσληψής του και θα καταβάλλεται για χρονικό διάστημα 12 μηνών. to "πρόγραμμα επιχορήγησης επιχειρήσεων για την απασχόληση 10.000 ανέργων ηλικίας 30-49 ετών" εντάσσεται στο επιχειρησιακό πρόγραμμα "ανάπτυξη ανθρώπινου δυναμικού, εκπαίδευση και δια βίου μάθηση 2014 -2020 (επ αναδ εδβμ)" και συγχρηματοδοτείται από το ευρωπαϊκό κοινωνικό ταμείο. πηγή: iefimerida facebook twitter google+ pinterest linkedin email εκτύπωση whatsapp viber & 464 & very low & Low & Socio-Economic & NA & NA & 2016-12-02 & 2016 & 2 & ECO
Frame & v.low & Regional & <500 & 1.8924259 & 1.3590316 & -1.2391092 & 0.9814602 & -0.7922579 & 0.0 & -0.9049211 & -1.0736569 & Recipient & Domestic & Domestic & Domestic & Domestic|ECO & Neutral\\
\addlinespace
Greece & http://www.naftemporiki.gr/finance/story/1016366/oi-touristikes-epixeiriseis-pou-mporoun-na-entaxthoun-sto-espa-2014-2020 & 380 & naftemporiki.gr & Private/Non-Public & Online and Offline & National & low = CP mentioned more times but NOT important part of story (mainly about others issues) & Economic development & Positive & EU + National & No myth & NA & NA & NA & NA & NA & NA & NA & NA & Greece & οι τουριστικές επιχειρήσεις που μπορούν να ενταχθούν στο εσπα 2014 - 2020 & 2015-10-14 & ευρωπαϊκό ταμείο περιφερειακής ανάπτυξης & τις πολύ μικρές, μικρές και μεσαίες επιχειρήσεις πουδραστηριοποιούνται αποκλειστικά στον τουρισμό και απασχολούν κατά μέσο όρο την τελευταία τιρετία τουλάχιστον μία θέση μισθωτής εργασίας θα ενισχύσει το πρόγραμμα "ενίσχυση τουριστικών μμε επιχειρήσεων για τον εκσυγχρονισμό τους και την ποιοτική αναβάθμιση των παρεχόμενων υπηρεσιών" από το επιχειρησιακό πρόγραμμα ανταγωνιστικότητα - επιχειρηματικότητα - καινοτομία (επανεκ) του εσπα 2014 - 2020. οι επιχειρήσεις θα πρέπει να έχουν συσταθεί έως την 31/12/2013. οι επενδυτικές προτάσεις μπορούν να έχουν προϋπολογισμό ύψους επένδυσης (επιχορηγούμενος προϋπολογισμός) από 15.000 ευρώ έως 150.000 ευρώ, ενώ το ποσοστό ενίσχυσης ορίζεται κατά μέγιστο στο 40\%, με δυνατότητα προσαύξησης κατά 10 μονάδες στην περίπτωση πρόσληψης νέου προσωπικού και μόνο μετά την πιστοποίηση επίτευξης αυτού του στόχου. στην ηλεκτρονική σελίδα του υπουργείου οικονομίας για το εσπα (www.espa.gr) σημειώνονται τα εξής: προϋποθέσεις συμμετοχής οι βασικές προϋποθέσεις συμμετοχής των επιχειρήσεων που υποβάλλουν επενδυτική πρόταση, είναι οι ακόλουθες: δεν έχουν δικαίωμα υποβολής πρότασης οι δημόσιες επιχειρήσεις, οι δημόσιοι φορείς ή δημόσιοι οργανισμοί ή/και οι θυγατρικές τους, καθώς και επιχειρήσεις που εξομοιώνονται με αυτές. προϋπολογισμός έργων ανά γεωγραφική κατανομή το πρόγραμμα χρηματοδοτείται με το συνολικό ποσό των 50.000.000 ευρώ (δημόσια δαπάνη) και θα υλοποιηθεί με δύο κύκλους προκήρυξης (1ος κύκλος 2015: 40\% και 2ος κύκλος 2016: 60\%). το παρόν πρόγραμμα του πρώτου κύκλου χρηματοδοτείται με συνολικό ποσό 20 εκατ. ευρώ (δημόσια δαπάνη) και κατανέμεται στις περιφέρειες της χώρας, ως εξής: η δημόσια δαπάνη του προγράμματος συγχρηματοδοτείται από το ελληνικό δημόσιο και το ευρωπαϊκό ταμείο περιφερειακής ανάπτυξης (ετπα) με εφαρμογή της ρήτρας ευελιξίας για τη χρηματοδότηση παρεμβάσεων που εμπίπτουν στο πεδίο ενίσχυσης του εκτ. καθεστώς ενίσχυσης στο πλαίσιο του προγράμματος, ενισχύονται, σύμφωνα με τον υπ' αριθμ. εε 1407/2013 κανονισμό (de minimis), έργα συνολικού προϋπολογισμού ύψους επένδυσης (επιχορηγούμενος π/υ) από 15.000 ευρώ έως 150.000 ευρώ. το ποσοστό ενίσχυσης ορίζεται κατά μέγιστο στο 40\% των επιλέξιμων δαπανών του επενδυτικού σχεδίου. στην περίπτωση πρόσληψης νέου προσωπικού, το ποσοστό επιχορήγησης προσαυξάνεται κατά 10 μονάδες μόνο μετά την πιστοποίηση επίτευξης αυτού του στόχου. υπάρχει δυνατότητα καταβολής προκαταβολής, μέχρι και το 40\% της αναλογούσας δημόσιας δαπάνης έναντι ισόποσης εγγυητικής επιστολής προερχόμενης από τράπεζα ή άλλο χρηματοπιστωτικό ίδρυμα εγκατεστημένο σε ένα εκ των κρατών μελών της εε. διάρκεια έργων η διάρκεια ολοκλήρωσης των έργων ορίζεται σε 24 μήνες από την ημερομηνία έκδοσης της απόφασης ένταξης του επενδυτικού σχεδίου. επιλέξιμες δαπάνες αναλυτικά, οι επιλέξιμες κατηγορίες δαπανών, οι οποίες θα τύχουν επιχορήγησης, στο πλαίσιο της παρούσας προδημοσίευσης, θα ορισθούν στην πρόσκληση του προγράμματος. ενδεικτικές κατηγορίες επιλέξιμων δαπανών για τη διετή περίοδο ενίσχυσης είναι: σημειώνεται ότι είναι επιλέξιμο και το μισθολογικό κόστος υφιστάμενου ή νέου προσωπικού που θα απασχοληθεί στο συγκεκριμένο έργο, με εφαρμογή της ρήτρας ευελιξίας στο ετπα για τη χρηματοδότηση παρεμβάσεων εκτ. ως έναρξη επιλεξιμότητας δαπανών ορίζεται η ημερομηνία υποβολής της αίτησης χρηματοδότησης. σημειώνεται ότι η παρούσα ανακοίνωση αποτελεί προδημοσίευση. υποβολή, αξιολόγηση, ένταξη για τη χρηματοδότηση από το παρόν πρόγραμμα απαιτείται η υποβολή προς αξιολόγηση επιχειρηματικού σχεδίου με βάση τη διαδικασία που θα καθοριστεί στην πρόσκληση του προγράμματος. οι προϋποθέσεις συμμετοχής, τα απαιτούμενα δικαιολογητικά, ο τρόπος υποβολής των προτάσεων, οι επιλέξιμες κατηγορίες δαπανών οι οποίες θα τύχουν επιχορήγησης, η διαδικασία εξέτασης και αξιολόγησης των προτάσεων, η ένταξή τους για χρηματοδότηση, οι υποχρεώσεις των δικαιούχων στην περίπτωση έγκρισης της αίτησης ενίσχυσης και οι λοιποί όροι του προγράμματος, θα περιγραφούν αναλυτικά στην πρόσκληση του προγράμματος, (καθώς και στο σύνολο των εγγράφων που την συνοδεύουν). σημειώνεται ότι κατά τον έλεγχο και την αξιολόγηση των προτάσεων θα γίνουν αυστηρές διασταυρώσεις δεδομένων, ώστε να επαληθευθεί η ακρίβεια των στοιχείων που περιέχονται στις επενδυτικές προτάσεις και να αποφευχθούν φαινόμενα απάτης σε βάρος του ενωσιακού προϋπολογισμού. η αξιολόγηση των αιτήσεων χρηματοδότησης θα είναι συγκριτική, βάσει των κριτηρίων που θα ορισθούν στην πρόσκληση του προγράμματος. η προδημοσίευση ουδεμία έννομη δέσμευση του δημοσίου γεννά ως προς την τελική πρόσκληση του προγράμματος και η διαχειριστική αρχή του επανεκ διατηρεί αναλλοίωτο το δικαίωμα να τροποποιήσει του όρους που αναφέρονται στην παρούσα. e-mail επικοινωνίας: infoepan@mou.gr γραμμή επικοινωνίας: 801 1136 300. & 655 & low & Low & Socio-Economic & NA & NA & 2015-10-14 & 2015 & 1 & ECO
Frame & low-medium & National & 500-1000 & 1.8924259 & 1.3590316 & -1.2391092 & 0.9814602 & -0.7922579 & 0.0 & -0.9049211 & -1.0736569 & Recipient & Domestic & European & Mixed & Domestic|ECO & Positive\\
Greece & https://www.eleftherostypos.gr/politiki/394333-kostas-mpakogiannis-o-dimos-na-einai-mprostaris/ & 357 & Ελεύθερος Τύπος & Private/Non-Public & Online and Offline & National & very low = CP mentioned once & Social awareness/inclusion & Positive & EU + Subnational & No myth & Poor communication of funding/rules & Positive & National & No myth & NA & NA & NA & NA & Greece & κώστας μπακογιάννης: ο δήμος να είναι μπροστάρης & 2019-04-09 & ευρωπαϊκό κοινωνικό ταμείο & για τα ναρκωτικά ο κ. μπακογιάννης είπε πως είναι ντροπή ότι το ελληνικό κράτος δεν έχει στρατηγική και το σχέδιο που υπάρχει από το 2012-13 είναι ξεχασμένο στα συρτάρια. "θέλουμε ένα δήμο που θα βγει μπροστά να αναλάβει τις ευθύνες του. δεν θέλουμε ένα δήμο που απλώς θα διαμαρτύρεται, θα καταγγέλλει και θα σχολιάζει". αυτό το μήνυμα έστειλε ο κώστας μπακογιάννης στο ραδιοφωνικό debate του real fm με το δημοσιογράφο νίκο χατζηνικολάου που πραγματοποιήθηκε στο αμφιθέατρο του ιεκ ακμη. οι υποψήφιοι δήμαρχοι αθηναίων απάντησαν σε ερωτήσεις σπουδαστών του ιεκ ακμη και του μητροπολιτικού κολλεγίου για τα θέματα της ασφάλειας, της καθαριότητας, των ναρκωτικών και του προσφυγικού. ειδικά για τα ναρκωτικά ο κώστας μπακογιάννης είπε πως είναι ντροπή το γεγονός ότι το ελληνικό κράτος δεν έχει στρατηγική για τα ναρκωτικά και το σχέδιο που υπάρχει από το 2012-13 είναι ξεχασμένο στα συρτάρια. "ο δήμος να είναι μπροστάρης. να πάρει πρωτοβουλία και να καλύψει το κενό που αφήνει το κεντρικό κράτος", ξεκαθάρισε σημειώνοντας ότι το ευρωπαϊκό κοινωνικό ταμείο δίνει χρηματοδοτήσεις από το 2015 αλλά δεν έχουμε αξιοποιήσει ούτε ένα ευρώ. "απαιτείται στρατηγική και συνεργασία όλων των φορέων. έχουμε το πλεονέκτημα να μπορούμε να υιοθετήσουμε καλές πρακτικές από το εξωτερικό", εξήγησε και πρότεινε τη λειτουργία ενός κέντρου πρόληψης σε κάθε δημοτική κοινότητα. ο δήμος να είναι - όπως χαρακτηριστικά είπε - γενναιόδωρος καθώς δεν μπορεί να δίνει λιγότερα από 150.000 ευρώ το χρόνο για πολιτικές πρόληψης, να λειτουργήσουν χώροι εποπτευόμενης χρήσης αλλά σε σημεία που θα αποφασιστούν μετά από διαβούλευση και συζήτηση στις γειτονιές με τους κατοίκους, να δοθεί έμφαση στο street work και να αποκτήσουμε δομές όπως υπνωτήρια και κέντρα ημέρας. "μιλάμε για ένα τεράστιο ζήτημα το οποίο θα πρέπει να το προσεγγίσουμε με πάρα πολύ προσοχή και πάρα πολύ μεγάλο σεβασμό γιατί μιλάμε για ανθρώπινες ζωές και για ανθρώπινα δικαιώματα χρηστών και κατοίκων", υπογράμμισε ο υποψήφιος δήμαρχος αθηναίων επισημαίνοντας ότι "όταν το συζητάμε αυτό ανοικτά δεν κάνουμε εμπόριο φόβου" και πως "δεν μπορούμε να κόβουμε και να ράβουμε μία αλήθεια στα μέτρα των όποιων σκοπιμοτήτων μας". "όπου περπατώ στις γειτονιές της αθήνας, εισπράττω μια κραυγή αγωνίας", είπε για το θέμα της ασφάλειας και επανέλαβε την πρότασή του για τη λειτουργία ενός ενιαίου συντονιστικού κέντρου στο οποίο θα συμμετέχουν όλοι οι συναρμόδιοι φορείς. "είναι σημαντικό διότι είδατε τι έγινε προχθές στα εξάρχεια, που δεν μπορούσαν να συντονιστούν οι δυνάμεις ασφαλείας και μετά έβγαινε το ένα υπουργείο και κατηγορούσε το άλλο. το αποτέλεσμα είναι, ότι έχουνε γίνει τα εξάρχεια εμπόλεμη ζώνη. έχουμε αυτή τη στιγμή έρθει αντιμέτωποι με παραστρατιωτικές ομάδες στην πραγματικότητα και όσο εμείς μιλάμε και δίνουμε ιδεολογικές και ιδεοληπτικές μάχες μεταξύ μας, έχουμε επιτρέψει στη βία και φόβο να κυριαρχήσουν", ανέφερε. επ'αυτού μίλησε για συνεργασία της ελληνικής και τη δημοτικής αστυνομίας καθώς και για συνεργασία με ιδιωτικές εταιρείες φύλαξης για συγκεκριμένα σημεία, όπως γίνεται στο πεδίον του άρεως. παράλληλα φέρνοντας ως παράδειγμα ότι δίπλα στη νομική υπάρχει πιάτσα ναρκωτικών και μπροστά από την ασοεε ανθεί το παραεμπόριο, ξεκαθάρισε: "το άσυλο είναι άσυλο ιδεών, είναι άσυλο διαλόγου. δεν είναι άσυλο εγκληματικότητας και παραβατικότητας". απαντώντας σε ερώτηση για τη χρήση καμερών είπε πως "θα πρέπει να αξιοποιήσουμε κάθε μέσο το οποίο έχουμε στη διάθεσή μας" και πως "υπάρχει ένα θεσμικό πλαίσιο για τα προσωπικά δεδομένα το οποίο οφείλουμε όλοι να το σεβόμαστε" αλλά υπάρχουν πλέον και νέες τεχνολογίες. "υπάρχουν κάμερες οι οποίες δεν καταγράφουν πρόσωπα αλλά λειτουργούν με infrared. αυτές, ειδικά τις νυχτερινές ώρες, μπορεί να είναι εξαιρετικά χρήσιμες. τι θέλω να πω; ας απελευθερωθούμε από τους δογματισμούς του χθες και ας προχωρήσουμε μπροστά", σημείωσε. μάλιστα κατέθεσε εκ νέου την πρότασή του για ενοποίηση του εθνικού αρχαιολογικού μουσείου με το πολυτεχνείο και το ακροπόλ για "να στεγάσουμε τη σπουδαιότερη αρχαιολογική συλλογή του πλανήτη" με τα 200.000 ευρήματα που βρίσκονται σήμερα στις αποθήκες του μουσείου. "να μετατρέψουμε την περιοχή σε πόλο έλξης για εκατομμύρια επισκέπτες. έτσι θα αλλάξει η αθήνα και έτσι μπορεί να αλλάξει και η ελλάδα διότι η πρωτεύουσα είναι η βιτρίνα της χώρας", τόνισε. "ο δήμος θα πρέπει να εστιάσει στην πρόληψη. μία πόλη φωτεινή, ταχτοποιημένη, καθαρή είναι μία πόλη φιλική στον μόνιμο κάτοικο, στον επισκέπτη, στον εργαζόμενο, στον τουρίστα και εχθρική για τον παραβάτη και τον εγκληματία", συμπλήρωσε και έφερε ως παράδειγμα το έργο των σδιτ για τον φωτισμό. "είναι ένα έργο το οποίο δίνει μια πραγματική λύση σε ένα πραγματικό πρόβλημα με ορίζοντα τριετίας. όλοι όσοι είμαστε σε αυτό το τραπέζι έχει νόημα να προχωρήσουμε στην υλοποίηση αυτού του έργου άμεσα", πρόσθεσε ενώ αντίστοιχα για την καθαριότητα είπε ότι "θα πρέπει να πάψουμε να δαιμονοποιούμε όλα τα μέσα που έχουμε στη διάθεσή μας". "εγώ δεν μιλάω για ιδιωτικοποιήσεις. δεν μιλάω για απολύσεις. λέω όμως το εξής πάρα πολύ απλό. να κάνουμε αυτό που κάνουν 9 στις 10 ευρωπαϊκές πρωτεύουσες. να έχουμε μικτά συστήματα", υπογράμμισε και τόνισε πως για παράδειγμα το σδιτ για το φωτισμό το έχει εγκρίνει διϋπουργική επιτροπή με τη μισή κυβέρνηση μέσα. "άρα λοιπόν, ας μην βουτάμε στην αριστεροσύνη του 2014 αλά καρτ και μετά να την ανταλλάσσουμε με την κυβερνώσα αριστερά του 2019 όποτε μας βολεύει", ανέφερε ενώ σε άλλη του παρέμβαση επ'αυτού πρόσθεσε: "δεν μπορεί όταν μιλάει ο τσακαλώτος να μιλάει για "αξιοποίηση", και όταν μιλάνε οι άλλοι για "ιδιωτικοποίηση" και μάλιστα αυτό να λέγεται από μία κυβέρνηση η οποία έχει δώσει τη δημόσια περιουσία για 99 χρόνια". ο υποψήφιος δήμαρχος αθηναίων σχετικά με το προσφυγικό είπε ότι από το 2015 μέχρι σήμερα δεν είχαμε σχέδιο, ότι έχουμε χειριστεί ένα βουνό χρήματα και τα αποτελέσματα είναι πενιχρά. ειδικότερα τόνισε ότι "ο δήμος οφείλει να διεκδικήσει να είναι φορέας διαχείρισης των ευρωπαϊκών κονδυλίων" και να υιοθετήσει καλές πρακτικές από το εξωτερικό όπως πχ από το άμστερνταμ όπου "έχουν δημιουργήσει μια πλατφόρμα στην οποία μετέχουν όλες οι μκο υπό το δήμο ο οποίος λειτουργεί ως συντονιστής και καταλύτης". τέλος, ο κώστας μπακογιάννης απαντώντας σε σχετική ερώτηση ξεκαθάρισε ότι δεν υπάρχει κανένα περιθώριο συζήτησης με τη χρυσή αυγή. "όλους μας ενώνει η πίστη μας στη δημοκρατία. η χρυσή αυγή είναι απέναντι από τη δημοκρατία. δεν είναι απλώς εκτός της δημοκρατίας", δήλωσε. & 997 & very low & Low & Socio-Economic & Governance & NA & 2019-04-09 & 2019 & 3 & ECO
Frame & v.low & National & 500-1000 & 1.8924259 & 1.3590316 & -1.2391092 & 0.9814602 & -0.7922579 & 0.0 & -0.9049211 & -1.0736569 & Recipient & Domestic & European & Mixed & Domestic|ECO & Positive\\
Greece & http://www.kathimerini.gr/869068/article/epikairothta/kosmos/handelsblatt-o-soimple-empodise-tis-kyrwseis-kata-ispanias-kai-portogalias & 333 & kathimerini.gr & Private/Non-Public & Online and Offline & National & low = CP mentioned more times but NOT important part of story (mainly about others issues) & Political leverage & Positive & EU + Other country & No myth & NA & NA & NA & NA & NA & NA & NA & NA & Greece & handelsblatt: ο σόιμπλε εμπόδισε τις κυρώσεις κατά ισπανίας και πορτογαλίας, απε-μπε | kathimerini & 2016-07-27 & διαρθρωτικά ταμεία & ο γερμανός υπουργός οικονομικών προφανώς επηρέασε την απόφαση της κομισιόν να μην επιβάλλει κυρώσεις λόγω των ελλειμμάτων τους στις δύο χώρες της ιβηρικής, αναφέρει δημοσίευμα της γερμανικής οικονομικής εφημερίδας handelsblatt. όπως δήλωσε την τετάρτη ο αρμόδιος για το ευρώ και αντιπρόεδρος της κομισιόν βάλντις ντομπρόφσκις, η εε δεν θα επιβάλει τελικά κυρώσεις στην ισπανία και την πορτογαλία λόγω της δύσκολης οικονομικής κατάστασης των δύο χωρών. σύμφωνα με τη handelsblatt, επενέβη ο γερμανός υπουργός οικονομικών βόλφγκανγκ σόϊμπλε για να εμποδίσει την επιβολή τους. διάφοροι επίτροποι, μεταξύ των οποίων ο γερμανός γκίντερ έτινγκερ, έλαβαν ένα τηλεφώνημα από τον σόιμπλε, όπως πληροφορήθηκε η εν λόγω εφημερίδα από υψηλόβαθμο διπλωμάτη της εε. μερικά τηλεφωνήματα, σύμφωνα με τη handelsblatt, τα έκανε από κοινού με τον ισπανό ομόλογό του, ντε γκουίντος, από την κίνα όπου γινόταν η σύνοδος κορυφής των g-20. προφανώς τον σόιμπλε τον ενδιαφέρει να στηρίξει την παλαιά και νυν χριστιανοδημοκρατική κυβέρνηση της μαδρίτης, αναφέρει το δημοσίευμα της γερμανικής εφημερίδας και προσθέτει πως αποτέλεσμα της παρέμβασης σόιμπλε ήταν να ταχθούν τελικά την τετάρτη μόνο τέσσερις επίτροποι υπέρ των κυρώσεων. εντούτοις, ο ζαν κλοντ γιουνκέρ προειδοποίησε την πορτογαλία και την ισπανία ότι η ακύρωση των κυρώσεων δεν σημαίνει ότι οι δύο αυτές χώρες θα παραμείνουν στο απυρόβλητο για πάντα: "σήμερα πήραμε δύο αποφάσεις: να διαγράψουμε τις κυρώσεις και να αναστείλουμε τις ενισχύσεις από τα διαρθρωτικά ταμεία", είπε στη γερμανική εφημερίδα. η ισπανία και η πορτογαλία πρέπει να πάρουν τα ενδεδειγμένα μέτρα στα σχέδια προϋπολογισμού τους για το 2017. μόνον έτσι θα μπορέσουν να αποτρέψουν την αναστολή των ενισχύσεων από τα διαρθρωτικά ταμεία, η οποία "θα έχει δυσμενέστερες οικονομικές συνέπειες για την ισπανία και την πορτογαλία από την επιβολή κυρώσεων", είπε χαρακτηριστικά στη handelsblatt ο γιουνκέρ. & 281 & low & Low & Power & NA & NA & 2016-07-27 & 2016 & 2 & POL
Frame & low-medium & National & <500 & 1.8924259 & 1.3590316 & -1.2391092 & 0.9814602 & -0.7922579 & 0.0 & -0.9049211 & -1.0736569 & Recipient & European & European & European & European|POL & Positive\\
Greece & http://www.avgi.gr/article/10927/8827802/ten-politike-synoches-meta-to-2020-syzeta-to-symboulio-genikon-ypotheseon-stis-bryxelles & 322 & avgi.gr & Private/Non-Public & Online and Offline & National & very high = CP is most important issue + CP is mentioned in title/headline & Institutional bargaining over funding & Factual & EU + National & No myth & NA & NA & NA & NA & NA & NA & NA & NA & Greece & την πολιτική συνοχής μετά το 2020, συζητά το συμβούλιο γενικών υποθέσεων στις βρυξέλλες & 2018-04-12 & πολιτική συνοχής & η συζήτηση αυτή πρόκειται να τροφοδοτήσει με προτάσεις την προετοιμασία, από την ευρωπαϊκή επιτροπή, της δέσμης νομοθετικών μέτρων σχετικά με την πολιτική συνοχής, μέτρα τα οποία θα ισχύσουν μετά το 2020 το στρατηγικό πλαίσιο και οι προτεραιότητες της πολιτικής συνοχής μετά το 2020, είναι από τα βασικά θέματα στην "ατζέντα" του συμβουλίου γενικών υποθέσεων που συνεδριάζει σήμερα στις βρυξέλλες, σε επίπεδο υπουργών εξωτερικών ή ευρωπαϊκών υποθέσεων. πιο συγκεκριμένα, οι υπουργοί θα ανταλλάξουν απόψεις και θα επικεντρωθούν: - στις επενδυτικές προτεραιότητες της πολιτικής συνοχής αναφορικά με τις περιοχές που θα ενταχθούν και θα καλύπτονται από τη μελλοντική πολιτική καθώς και τα κριτήρια για την κατανομή των πόρων, θέμα που αφορά ιδιαίτερα την ελλάδα. - στον τρόπο με τον οποίο θα επιταχυνθεί η εφαρμογή της πολιτικής και θα βελτιωθεί η αποτελεσματικότητά της. η συζήτηση αυτή πρόκειται να τροφοδοτήσει με προτάσεις την προετοιμασία, από την ευρωπαϊκή επιτροπή, της δέσμης νομοθετικών μέτρων σχετικά με την πολιτική συνοχής, μέτρα τα οποία θα ισχύσουν μετά το 2020. α α α email εκτυπωση κατηγορία ευρώπη ροη κατηγοριας & 168 & very high & High & Power & NA & NA & 2018-04-12 & 2018 & 3 & POL
Frame & high-very high & National & <500 & 1.8924259 & 1.3590316 & -1.2391092 & 0.9814602 & -0.7922579 & 0.0 & -0.9049211 & -1.0736569 & Recipient & Domestic & European & Mixed & Domestic|POL & Neutral\\
Greece & http://www.inewsgr.com/23/Ta-orata-kai-aorata-synora-tis-evropis-stin-afylachti-diavasi-tou-protou-programmatos.htm & 329 & iNewsgr.com & Private/Non-Public & Online only & National & high = CP is most important issue in story (can also cover other issues) & Social justice & Positive & EU & No myth & NA & NA & NA & NA & NA & NA & NA & NA & Greece & "tα ορατά και αόρατα σύνορα της ευρώπης" στην "αφύλαχτη διάβαση" του πρώτου προγράμματος & 2017-10-13 & πολιτική συνοχής & παρακολουθείστε, στην εκπομπή "αφύλαχτη διάβαση" του πρώτου προγράμματος, "tα ορατά και αόρατα σύνορα της ευρώπης", το σάββατο 14 οκτωβρίου 2017 και ώρα 12:00-13:00. η εκπομπή "αφύλαχτη διάβαση" ταξίδεψε στις βρυξέλλες και παρακολούθησε τις εργασίες της επιτροπής περιφερειών της ευρώπης στη σκιά του δημοψηφίσματος της καταλονίας. περιγράφοντας πέρα, όμως, από τα ορατά υπάρχουν διάχυτα και τα αόρατα σύνορα. εκείνα που δημιουργούν πολλαπλές ανισότητες, που διαχωρίζουν πληθυσμούς και γεννούν αποκλεισμούς και ανυπέρβλητες δυσκολίες στη διαβίωση των πολιτών. στη γνωμοδότηση που συνέταξε ο ιταλός mauro d' atis και υπερψηφίστηκε από την επιτροπή περιφερειών της ευρώπης, γίνεται σαφώς λόγος για κοινωνική δικαιοσύνη. περισσότερα για το θέμα μπορείτε να διαβάσετε στις ανταποκρίσεις του θωμά σίδερη από τις βρυξέλλες, στο site της ερτ www.ert.gr: - οι περιφέρειες της ευρώπης συζητούν για την πολιτική συνοχής - τα κοινωνικά δικαιώματα στο επίκεντρο της ευρωπαϊκής επιτροπής περιφερειών καλεσμένος στο στούντιο του πρώτου προγράμματος είναι ο αντιπρόεδρος της επιτροπής coter (επιτροπή για την κοινωνική συνοχή) σπύρος σπυρίδων. στην εκπομπή μιλούν, ακόμη: ο mauro d' atis για τα κοινωνικά δικαιώματα και οι anders knape, christian buchmann, deirdre forde, mark weinmeister, mieczysław struk και józsef ribanyi για το μέλλον της πολιτικής, oικονομικής και κοινωνικής συνοχής των ευρωπαϊκών περιφερειών. έρευνα-παρουσίαση: θωμάς σίδερης συντονιστείτε στο πρώτο πρόγραμμα 91,6 και 105,8! εκπομπή: "αφύλαχτη διάβαση" "τα ορατά και αόρατα σύνορα της ευρώπης" τις εκπομπές της "αφύλαχτης διάβασης" μπορείτε να τις ακούτε -μετά την πρώτη ραδιοφωνική τους μετάδοση- και στο web radio της ερτ: & 243 & high & High & Socio-Economic & NA & NA & 2017-10-13 & 2017 & 2 & ECO
Frame & high-very high & National & <500 & 1.8924259 & 1.3590316 & -1.2391092 & 0.9814602 & -0.7922579 & 0.0 & -0.9049211 & -1.0736569 & Recipient & European & European & European & European|ECO & Positive\\
\addlinespace
Greece & http://www.newsit.gr/politikh/Tin-Kalymno-kai-tin-Ko-episkeptetai-o-eyrovoyleytis-Notis-Marias/742521 & 341 & newsit.gr & Private/Non-Public & Online only & National & medium = CP is important part of story & Social awareness/inclusion & Balanced & EU + National & No myth & NA & NA & NA & NA & NA & NA & NA & NA & Greece & την κάλυμνο και την κω επισκέπτεται ο ευρωβουλευτής νότης μαριάς & 2017-07-26 & ευρωπαϊκό ταμείο περιφερειακής ανάπτυξης & την παρασκευή 28 ιουλίου, πρόεδρος του κόμματος "ελλαδα - ο αλλος δρομος" ευρωβουλευτής, νότης μαριάς θα επισκεφθεί την κω προκειμένου να συναντηθεί με τοπικούς φορείς και πολίτες για να συζητήσει μαζί τους τα προβλήματα που προέκυψαν από το πρόσφατο χτύπημα του εγκέλαδου. ο κ. μαριάς, στις 24 ιουλίου έστειλε επιστολή στην ευρωπαία επίτροπο κορίνα κρέτσου ζητώντας πλήρη και ταχεία αποζημίωση των σεισμοπαθών της κω. στην επιστολή του ζήτησε από την ευρωπαϊκή επιτροπή την πλήρη και ταχεία αποζημίωση των πληγέντων καθώς και την αποκατάσταση των εκεί ζημιών με κονδύλια από το ευρωπαϊκό ταμείο περιφερειακής ανάπτυξης και από το ευρωπαϊκό ταμείο αλληλεγγύης και πρότεινε οι εθνικές δαπάνες για αποκατάσταση ζημιών λόγω του σεισμού να μην μετρούν στο δημοσιονομικό έλλειμμα το οποίο ως γνωστόν ελέγχεται από το δρακόντειο σύμφωνο σταθερότητας. ο ανεξάρτητος ευρωβουλευτής νότης μαριάς, σήμερα τετάρτη 26 ιουλίου βρίσκεται στην κάλυμνο για προγραμματισμένη επίσκεψη προκειμένου να συναντηθεί με φορείς, πολίτες και φίλους του κόμματος. & 151 & medium & Medium & Socio-Economic & NA & NA & 2017-07-26 & 2017 & 2 & ECO
Frame & low-medium & National & <500 & 1.8924259 & 1.3590316 & -1.2391092 & 0.9814602 & -0.7922579 & 0.0 & -0.9049211 & -1.0736569 & Recipient & Domestic & European & Mixed & Domestic|ECO & Neutral\\
Greece & https://www.zougla.gr/greece/article/anakino8ikan-apo-to-iki-ta-apotelesmata-gia-ti-xorigisi-700-didaktorikon-ipotrofion & 370 & zougla.gr & Private/Non-Public & Online only & National & low = CP mentioned more times but NOT important part of story (mainly about others issues) & Research \& innovation & Positive & EU + National & No myth & NA & NA & NA & NA & NA & NA & NA & NA & Greece & ανακοινώθηκαν από το ικυ τα αποτελέσματα για τη χορήγηση 700 διδακτορικών υποτροφιών & 2018-09-07 & ευρωπαϊκό κοινωνικό ταμείο & τα αποτελέσματα 700 διδακτορικών υποτροφιών ανακοίνωσε σήμερα, παρασκευή, το ίδρυμα κρατικών υποτροφιών. η κάθε υποτροφία ανέρχεται στα 816,90 ευρώ μηνιαίως και διαρκεί μέχρι 36 μήνες. σύμφωνα με το ικυ, πρόκειται για το μεγαλύτερο αριθμό υποτροφιών που χορηγείται για εκπόνηση διδακτορικής έρευνας στην ελλάδα, από οποιοδήποτε φορέα σε όλη την ιστορία της χώρας. "εκτιμάται ότι ένα τέτοιο γεγονός θα συντελέσει στην παραμονή, αν όχι και επιστροφή, στη χώρα, σημαντικού κεφαλαίου από το καλύτερο ανθρώπινο δυναμικό", αναφέρει το ικυ σε σχετική ανακοίνωση. οι εν λόγω υποτροφίες εντάσσονται στο πλαίσιο του δεύτερου κύκλου της πράξης "ενίσχυση του ανθρώπινου ερευνητικού δυναμικού μέσω της υλοποίησης διδακτορικής έρευνας", που συγχρηματοδοτείται από την ελλάδα και την ευρωπαϊκή ένωση (ευρωπαϊκό κοινωνικό ταμείο και υλοποιούνται κατόπιν κοινής απόφασης των υπουργείων παιδείας και οικονομίας \& ανάπτυξης. συνολικά, όπως ενημέρωσε το ικυ, υποβλήθηκαν 1.740 ηλεκτρονικές αιτήσεις, η αξιολόγηση των οποίων υλοποιήθηκε εξατομικευμένα από αξιολογητές μέλη δεπ των ελληνικών αει και ερευνητές των ερευνητικών ιδρυμάτων της χώρας. επιπλέον, η αξιολόγηση έγινε αυτοματοποιημένα μέσω ολοκληρωμένης πλατφόρμας αξιολόγησης των ερευνητικών προτάσεων, με στόχο την αύξηση της αντικειμενικότητας των αξιολογήσεων. & 176 & low & Low & Socio-Economic & NA & NA & 2018-09-07 & 2018 & 3 & ECO
Frame & low-medium & National & <500 & 1.8924259 & 1.3590316 & -1.2391092 & 0.9814602 & -0.7922579 & 0.0 & -0.9049211 & -1.0736569 & Recipient & Domestic & European & Mixed & Domestic|ECO & Positive\\
Greece & https://www.902.gr/eidisi/eyrovoyli/168028/poliorkitikos-krios-gia-tin-entasi-tis-antilaikis-epithesis-ta-eyroenosiaka & 299 & 902.gr & Private/Non-Public & Online and Offline & National & high = CP is most important issue in story (can also cover other issues) & Economic development & Negative & EU + National & No myth & Ineffective goal achievement & Negative & EU & No myth & Jobs & Negative & EU & No myth & Greece & πολιορκητικός κριός για την ένταση της αντιλαϊκής επίθεσης τα ευρωενωσιακά κονδύλια (video) & 2018-09-12 & ευρωπαϊκά διαρθρωτικά και επενδυτικά ταμεία & video of ο κ. παπαδακης για τα ταμεια χρηματοδοτησης της ελλαδας έκθεση σχετικά με "την εφαρμογή ειδικών μέτρων για την ελλάδα βάσει του κανονισμού 2015/1839" που προέβλεπαν μεταξύ άλλων εμπροσθοβαρή προχρηματοδότηση για τα χρηματοδοτικά προγράμματα της περιόδου 2014-2020, όσον αφορά τα ευρωπαϊκά διαρθρωτικά και επενδυτικά ταμεία, συζητήθηκε στην ολομέλεια του ευρωκοινοβουλίου στο στρασβούργο. στη συζήτηση παρενέβη ο ευρωβουλευτής του κκε κώστας παπαδάκης, τονίζοντας τα παρακάτω: "τα ευρωπαϊκά διαρθρωτικά και επενδυτικά ταμεία (εδετ) αποτελούν από τα βασικά εργαλεία της εε για την άμεση και έμμεση ενίσχυση της κερδοφορίας των μεγάλων επιχειρηματικών ομίλων. αυτό ομολογεί η ίδια η έκθεση που αναφέρει ότι "το 63 \% του συνόλου των πληρωμών (...) αφορούσε ενισχύσεις για επιχειρήσεις και επιχειρηματικά σχέδια, συμβάλλοντας άμεσα στην κερδοφορία των επιχειρήσεων και τη μείωση του επιχειρηματικού κινδύνου", ενώ το υπόλοιπο 37\% "αφορούσε δράσεις κρατικών ενισχύσεων για έργα υποδομών" που έχουν ανάγκη και πάλι οι μεγάλοι όμιλοι για τη διευκόλυνση της δράσης και της κερδοφορίας τους. άλλωστε στα χρόνια των ρεκόρ απορροφητικότητας, για τα οποία επαίρεται η κυβέρνηση συριζα - ανελ, απογειώθηκε η αντεργατική επίθεση και οι μεγάλες απώλειες σε μισθούς, συντάξεις, εργατικά δικαιώματα. το κκε στηρίζει τις διεκδικήσεις του εργατικού και λαϊκού κινήματος που απαιτεί όλα αυτά τα κονδύλια που έτσι κι αλλιώς προέρχονται από τη σκληρή φορολογία των λαϊκών εισοδημάτων. να χρηματοδοτηθούν έργα ζωτικής σημασίας για την ανακούφιση και την ικανοποίηση των λαϊκών αναγκών". & 224 & high & High & Socio-Economic & Socio-Economic & Socio-Economic & 2018-09-12 & 2018 & 3 & ECO
Frame & high-very high & National & <500 & 1.8924259 & 1.3590316 & -1.2391092 & 0.9814602 & -0.7922579 & 0.0 & -0.9049211 & -1.0736569 & Recipient & Domestic & European & Mixed & Domestic|ECO & Negative\\
Greece & http://e-thessalia.gr/egkeniastike-klisto-gymnastirio-tis-skiathou/ & 306 & e-thessalia.gr & Private/Non-Public & Online only & Regional/Local & very low = CP mentioned once & Infrastructure & Positive & EU + Subnational & No myth & NA & NA & NA & NA & NA & NA & NA & NA & Greece & εγκαινιάστηκε το κλειστό γυμναστήριο της σκιάθου - e-thessalia.gr & 2016-06-13 & ευρωπαϊκό ταμείο περιφερειακής ανάπτυξης & σε μια λιτή τελετή εγκαινιάστηκε από τον περιφερειάρχη θεσσαλίας κ. κώστα αγοραστό το κλειστό γυμναστήριο στη σκιάθο. το έργο προϋπολογισμού 459.915,64 ευρώ χρηματοδοτήθηκε από το εσπα θεσσαλίας με φορέα υλοποίησης τη διεύθυνση τεχνικών έργων της περιφερειακής ενότητας μαγνησίας και σποράδων. μετά το πέρας των εγκαινίων πραγματοποιήθηκε αγώνας μπάσκετ των ακαδημιών του νησιού. σημειώνεται ότι με την ολοκλήρωσή του έργου, τα παιδιά που εγγράφηκαν στις ακαδημίες μπάσκετ και βόλεϊ στο νησί διπλασιάστηκαν. όπως ανέφερε ο κ. κ. αγοραστός ανέφερε μεταξύ άλλων ότι είναι ένα έργο πρωτοποριακό για την χώρα το οποίο ολοκληρώθηκε σε πολύ σύντομο χρονικό διάστημα. από την πλευρά του ο δήμαρχος σκιάθου κ. δημήτρης πρεβεζάνος δήλωσε ότι "είναι ένα πανέμορφο γήπεδο, το θέλουν τα παιδιά, κάθε μέρα γίνονται προπονήσεις, και αγώνες" στοιχεία του έργου το έργο αφορούσε στη στέγαση του γηπέδου με κατασκευή μικτή δικτυωμάτων στερέωσης και ειδικής μεμβράνης με κινητά ανοίγματα, στην κατασκευή πατώματος σύμφωνα με τις ισχύουσες προδιαγραφές, κερκίδων, βοηθητικών χώρων αποδυτηρίων, κλπ., κατάλληλου φωτισμού, απαιτούμενου εξοπλισμού και διαμόρφωσης περιβάλλοντος χώρου. με την ολοκλήρωση του έργου, δημιουργήθηκε ένας χώρος αναψυχής και άθλησης σύμφωνα με τις απαραίτητες προϋποθέσεις ασφαλείας και λειτουργικότητας, για διάθεση στους χρήστες καθ' όλη τη διάρκεια του έτους. η αναβάθμιση και στέγαση του γηπέδου αποτελούσε πάγιο αίτημα των μαθητών και των φορέων της σκιάθου. με το έργο αυτό η σκιάθος απέκτησε έναν σύγχρονο χώρο άθλησης και αναψυχής. το έργο συγχρηματοδοτήθηκε από το ευρωπαϊκό ταμείο περιφερειακής ανάπτυξης και αποτελούσε τμήμα του ολοκληρωμένου σχεδίου ανάπτυξης της νήσου σκιάθου (οσαν) που υλοποιήθηκε από την περιφέρεια θεσσαλίας. & 250 & very low & Low & Socio-Economic & NA & NA & 2016-06-13 & 2016 & 2 & ECO
Frame & v.low & Regional & <500 & 1.8924259 & 1.3590316 & -1.2391092 & 0.9814602 & -0.7922579 & 0.0 & -0.9049211 & -1.0736569 & Recipient & Domestic & European & Mixed & Domestic|ECO & Positive\\
Greece & https://e-thessalia.gr/dyo-proslipsis-sto-kentro-kinotitas-tou-dimou-notiou-piliou/ & 383 & e-thessalia.gr & Private/Non-Public & Online only & Regional/Local & low = CP mentioned more times but NOT important part of story (mainly about others issues) & Social awareness/inclusion & Positive & EU + Subnational & No myth & NA & NA & NA & NA & NA & NA & NA & NA & Greece & δύο προσλήψεις στο κέντρο κοινότητας του δήμου νοτίου πηλίου - e-thessalia.gr & 2017-11-15 & ευρωπαϊκό κοινωνικό ταμείο & με δύο άτομα θα στελεχωθεί το κέντρο κοινότητας του δήμου νοτίου πηλίου ο οποίος αξιοποιεί τη νέα αυτή δομή προς όφελος του κοινωνικού συνόλου. το κέντρο κοινότητας του δήμου noτίου πηλίου θα στελεχωθεί με δύο άτομα, με ειδικότητες κοινωνικού λειτουργού και ψυχολόγου, με διάρκεια σύμβασης ενός έτους από την υπογραφή της σύμβασης, με δυνατότητα ανανέωσης ή παράτασης έως τη λήξη του προγράμματος. η ένταξη του δήμου στο επιχειρησιακό πρόγραμμα "περιφερειακό επιχειρησιακό πρόγραμμα θεσσαλίας 2014-2020", έγινε με απόφαση του περιφερειάρχη θεσσαλίας μετά από θετική αξιολόγηση της πρότασης που υπέβαλε ο δήμος. η πράξη συγχρηματοδοτείται από το ευρωπαϊκό κοινωνικό ταμείο (εκτ). το κέντρο κοινότητας έχει τη δυνατότητα να παρέχει υπηρεσίες που θα αποσκοπούν στη βελτίωση του βιοτικού επιπέδου και θα διασφαλίζει την κοινωνική ένταξη των ωφελουμένων με παροχή συμβουλευτικής υποστήριξης και με δράσεις που θα χαρακτηρίζονται από εκπαιδευτικό, κοινωνικό και επικοινωνιακό χαρακτήρα. η διάρκεια λειτουργίας της δομής προβλέπεται για τρία έτη. ο αντιδήμαρχος νοτίου πηλίου κ. μιλτ. παπαδημητρίου ανέφερε: "με τη δημιουργία του κέντρου κοινότητας αξιοποιούμε τις δυνατότητες των ευρωπαϊκών προγραμμάτων και ενισχύουμε τη διασύνδεση των πολιτών με κοινωνικά προγράμματα, καταπολεμώντας τον κοινωνικό αποκλεισμό και κάθε είδους διάκριση. σημαντικό είναι ότι με το κέντρο κοινότητας δημιουργούνται θέσεις εργασίας. ήδη ο δήμος μας προχωράει στην πρόσληψη δύο ατόμων για τη στελέχωσή του". & 210 & low & Low & Socio-Economic & NA & NA & 2017-11-15 & 2017 & 2 & ECO
Frame & low-medium & Regional & <500 & 1.8924259 & 1.3590316 & -1.2391092 & 0.9814602 & -0.7922579 & 0.0 & -0.9049211 & -1.0736569 & Recipient & Domestic & European & Mixed & Domestic|ECO & Positive\\
\addlinespace
Greece & https://www.newsbeast.gr/greece/arthro/4184651/anoixe-meta-apo-exi-chronia-to-archaiologiko-moyseio-kerkyras & 360 & Newsbeast.gr & Private/Non-Public & Online only & National & very low = CP mentioned once & Cultural heritage & Positive & National & No myth & NA & NA & NA & NA & NA & NA & NA & NA & Greece & άνοιξε, μετά από έξι χρόνια, το αρχαιολογικό μουσείο κέρκυρας & 2018-11-09 & ευρωπαϊκό ταμείο περιφερειακής ανάπτυξης & τα εγκαίνιά του θα γίνουν από την υπουργό πολιτισμού πέρασαν έξι ολόκληρα χρόνια για να ανοίξει και πάλι σήμερα τις πύλες του το αρχαιολογικό μουσείο κέρκυρας, μετά και την ανακαίνισή του (που χρηματοδοτήθηκε από το ευρωπαϊκό ταμείο περιφερειακής ανάπτυξης) προϋπολογισμού 4,5 εκατ. περίπου. στο διώροφο, ανακαινισμένο πλέον, αρχαιολογικό μουσείο κέρκυρας, που κτίστηκε το 1967, στεγάζονται αρχαιολογικά ευρήματα από την αρχαία πόλη της κέρκυρας, αλλά και ολόκληρο το "νησί των φαιάκων", που αφορούν την πρώτη κατοίκησή του στην παλαιολιθική εποχή μέχρι και τους ρωμαϊκούς χρόνους. το σημαντικότερο έκθεμα της έκθεσης είναι το λίθινο αέτωμα της γοργούς-αρτέμιδος, που προέρχεται από το ναό της θεάς αρτέμιδος. το αέτωμα έχει μήκος 17,02 μ. και ύψος 3,18 μ. και έχει κατασκευασθεί από ντόπιο πωρόλιθο. θεωρείται έργο κορίνθιου γλύπτη και χρονολογείται στο πρώτο τέταρτο του 6ου αι. π.χ. η περίοδος κατασκευής του αετώματος συμπίπτει με σημαντικές πολιτικές μεταβολές στην κέρκυρα, όπου το 585 π.χ. απαλλάχθηκε από την ηγεμονία των βακχιαδών της κορίνθου. άγνωστο παραμένει ωστόσο αν η κατασκευή του αετώματος πραγματοποιήθηκε κατά την περίοδο της κορινθιακής κηδεμονίας ή μετά, όταν το νησί απέκτησε την αυτονομία του. η έκθεση του αρχαιολογικού μουσείου κερκύρας είναι οργανωμένη κατά χώρους και κατά θεματικές ενότητες. ο αρχικός πυρήνας της έκθεσης είχε στόχο να παρουσιάσει το εν λόγω αέτωμα της γοργούς μαζί με τα άλλα ευρήματα από το ναό της αρτέμιδος, αλλά και να κάνει γνωστά τα αποτελέσματα της ανασκαφικής έρευνας στο χώρο του νεκροταφείου της αρχαίας πόλης της κέρκυρας, όπου βρέθηκε το θαυμάσιο λιοντάρι του μενεκράτη. στη συνέχεια προστέθηκε και η συλλογή προϊστορικών εργαλείων και άλλων αντικειμένων από διάφορα σημεία του νησιού, με σκοπό να "φωτιστούν" οι απαρχές της ανθρώπινης παρουσίας στο νησί. σειρά μεγαλύτερων και μικρότερων γλυπτών καταδεικνύει την άνθιση της ελληνικής τέχνης στο νησί, αλλά φανερώνει παράλληλα και τις εμπορικές σχέσεις και τις δοξασίες των αντίστοιχων εποχών. οι ανασκαφές στο χώρο των ιερών στο κτήμα του mon-repo, αλλά και σε άλλα τμήματα της αρχαίας πόλης τις δεκαετίες του 1960-1970, ήρθαν να προσθέσουν, σύμφωνα με το αθηναϊκό πρακτορείο ειδήσεων, νέα αξιόλογα ευρήματα τόσο στην έκθεση όσο και στις αποθήκες του αρχαιολογικού μουσείου. παράλληλα εκτίθενται ευρήματα ανασκαφών των δύο τελευταίων δεκαετιών, αλλά και της νομισματικής συλλογής του νησιού. η ανακαίνιση του αρχαιολογικού μουσείου κέρκυρας είχε ολοκληρωθεί το 2017, όμως για την άρτια λειτουργία του, στελεχώθηκε από το αναγκαίο προσωπικό, το τελευταίο διάστημα. το μουσείο υπάγεται στην η' εφορεία προϊστορικών και κλασικών αρχαιοτήτων που εδρεύει στην κέρκυρα. τα επίσημα εγκαίνιά του αναμένεται να γίνουν από την υπουργό πολιτισμού, μυρσίνη ζορμπά. & 416 & very low & Low & Socio-Economic & NA & NA & 2018-11-09 & 2018 & 3 & ECO
Frame & v.low & National & <500 & 1.8924259 & 1.3590316 & -1.2391092 & 0.9814602 & -0.7922579 & 0.0 & -0.9049211 & -1.0736569 & Recipient & Domestic & Domestic & Domestic & Domestic|ECO & Positive\\
Greece & http://www.topontiki.gr/article/320126/espa-enishyseis-60-ekat-gia-emporio-estiasi-kai-ekpaideysi & 382 & topontiki.gr & Private/Non-Public & Online and Offline & National & high = CP is most important issue in story (can also cover other issues) & Economic development & Positive & EU + National & No myth & NA & NA & NA & NA & NA & NA & NA & NA & Greece & εσπα: ενισχύσεις 60 εκατ. για εμπόριο, εστίαση και εκπαίδευση & 2019-04-22 & ευρωπαϊκό ταμείο περιφερειακής ανάπτυξης & από το υπουργείο οικονομίας και ανάπτυξης έχει ανακοινωθεί πρόσκληση για υποβολή προτάσεων στη δράση "εργαλειοθήκη επιχειρηματικότητας: εμπόριο - εστίαση - εκπαίδευση", η οποία υλοποιείται στο πλαίσιο του ε.π. ανταγωνιστικότητα, επιχειρηματικότητα και καινοτομία 2014-2020, με εφαρμογή στο σύνολο των περιφερειών της χώρας. λόγω του μεγάλου ενδιαφέροντος που παρουσιάζει το πρόγραμμα και της επικείμενης λήξης προθεσμίας υποβολής προτάσεων, σήμερα, θα υπενθυμίσουμε τα βασικά του στοιχεία. η δράση συγχρηματοδοτείται από το ευρωπαϊκό ταμείο περιφερειακής ανάπτυξης (ετπα) της ευρωπαϊκής ενωσης και από εθνικούς πόρους με το συνολικό ποσό των 60.000.000 ευρώ και κατανέμεται εξίσου στις επιχειρήσεις των τριών κλάδων, δηλ. 20.000.000 ευρώ σε κάθε έναv από τους: λιανικό εμπόριο (για τους επιλέξιμους καδ που ανήκουν στις κατηγορίες 45 και 47), εστίαση (για τους επιλέξιμους καδ που ανήκουν στην κατηγορία 56) και εκπαίδευση - κοινωνική μέριμνα, χωρίς παροχή καταλύματος (για τους επιλέξιμους καδ που ανήκουν στις κατηγορίες 85 και 88). στόχος της δράσης οπως αναφέρεται στη δημοσιευμένη πρόσκληση, η δράση στοχεύει στην ενίσχυση υφιστάμενων μικρών και πολύ μικρών επιχειρήσεων που δραστηριοποιούνται στο λιανικό εμπόριο, στην παροχή υπηρεσιών ιδιωτικής εκπαίδευσης, στην παροχή υπηρεσιών παιδικού και βρεφονηπιακού σταθμού και λοιπών υπηρεσιών ημερήσιας φροντίδας, προκειμένου να αναβαθμίσουν το επίπεδο επιχειρησιακής οργάνωσης και λειτουργίας τους στους τομείς: α) κατανάλωση ενέργειας, β) χρήση τπε, γ) υγιεινή και ασφάλεια, δ) εφοδιαστική αλυσίδα, ε) πιστοποίηση συστημάτων. επιλέξιμοι δικαιούχοι η δράση απευθύνεται σε επιχειρήσεις οι οποίες θα πρέπει αθροιστικά να ικανοποιούν τις παρακάτω προϋποθέσεις: * να έχουν την ιδιότητα των μικρών ή πολύ μικρών επιχειρήσεων. * να έχουν κλείσει τουλάχιστον τρεις διαχειριστικές χρήσεις δωδεκάμηνης διάρκειας. * να διαθέτουν έναν τουλάχιστον επιλέξιμο καδ όπως αυτοί αναλυτικά παρατίθενται στο παράρτημα ιii της δημοσιευμένης πρόσκλησης, ως ενεργή κύρια ή δευτερεύουσα δραστηριότητα, καθ' όλη τη διάρκεια των τριών αυτών διαχειριστικών χρήσεων. * να έχουν το έτος που προηγείται της ηλεκτρονικής υποβολής της πρότασης χρηματοδότησης, κατ' ελάχιστον: 1 ετήσια μονάδα εργασίας (εμε) για τον κλάδο του λιανικού εμπορίου, 2 εμε για τον κλάδο της εστίασης, 5 εμε για τον κλάδο της εκπαίδευσης - κοινωνικής μέριμνας. επιλέξιμες δαπάνες. σύμφωνα με τη δημοσιευμένη πρόσκληση της δράσης, κατά τη διαμόρφωση από την επιχείρηση του προτεινόμενου επιχορηγούμενου προϋπολογισμού επενδυτικών σχεδίων, θα επιλέγονται δαπάνες από τις ακόλουθες κατηγορίες δαπανών: * εξοικονόμηση ενέργειας - χρήση απε. * αναβάθμιση υγιεινής και ασφάλειας στις διαδικασίες και στους χώρους λειτουργίας της επιχείρησης. * διευκόλυνση πρόσβασης των αμεα. * ενσωμάτωση εξειδικευμένων τεχνολογιών πληροφορικής και επικοινωνιών στις λειτουργίες της επιχείρησης. * βελτίωση των διεργασιών εφοδιαστικής αλυσίδας. * πιστοποίηση υπηρεσιών ή/και διαδικασιών. συγκεκριμένα, οι επιλέξιμες δαπάνες μπορούν να αφορούν στα ακόλουθα: κτίρια, εγκαταστάσεις και περιβάλλων χώρος (έως 100\% του επενδυτικού σχεδίου). παρεμβάσεις για εξοικονόμηση ενέργειας, αναβάθμιση της υγιεινής και ασφάλειας και διευκόλυνση προσβασιμότητας αμεα. μηχανήματα - εξοπλισμός (έως 100\% του επενδυτικού σχεδίου). προμήθεια και εγκατάσταση εξοπλισμού για εξοικονόμηση ενέργειας, αναβάθμιση της υγιεινής και ασφάλειας στους χώρους της επιχείρησης, προμήθεια και εγκατάσταση εξοπλισμού τπε. μεταφορικά μέσα: * μέχρι 25.000 ευρώ και έως 30\% του επενδυτικού σχεδίου για επιχειρήσεις λιανεμπορίου - εστίασης. * εως 100\% για επιχειρήσεις εκπαίδευσης - κοινωνικής μέριμνας. * εως 12.000 ευρώ για μετατροπή κινητήρα πετρελαιοκίνητου/βενζινοκίνητου οχήματος σε κινητήρα διπλού καυσίμου πετρελαίου/βενζίνης - φυσικού αερίου (cng). * ψηφιακή προβολή (έως 8.000 ευρώ). * πιστοποίηση υπηρεσιών ή/και διαδικασιών (έως 14.000 ευρώ, 7.000 ευρώ ανά πιστοποιητικό και μέχρι δύο πιστοποιητικά). * μισθολογικό κόστος εργαζομένων (νέο προσωπικό) - μέχρι 30.000 ευρώ και έως 40\% του π/υ, 15.000 ευρώ ανά εμε και μέχρι δύο εμε. * δαπάνες μελετών/κατάρτισης/παρακολούθησης του επενδυτικού σχεδίου. ως ημερομηνία έναρξης επιλεξιμότητας δαπανών ορίζεται η ημερομηνία δημοσίευσης της πρόσκλησης (19/12/2018). προϋπολογισμός - ποσοστό ενίσχυσης. στο πλαίσιο της δράσης επιλέξιμα είναι επενδυτικά σχέδια προϋπολογισμού από 10.000 ευρώ έως 150.000 ευρώ. η προθεσμία υλοποίησης των επενδυτικών σχεδίων δεν θα πρέπει να ξεπερνά τους 24 μήνες από την ημερομηνία έκδοσης της απόφασης ένταξης. το ποσοστό ενίσχυσης των επενδυτικών σχεδίων ανέρχεται στο 50\% του επιλέξιμου προϋπολογισμού. οι αιτήσεις υποβάλλονται ηλεκτρονικά μέσω του πληροφοριακού συστήματος κρατικών ενισχύσεων (πσκε) www.ependyseis.gr/mis έως την 9η μαΐου 2019 & 638 & high & High & Socio-Economic & NA & NA & 2019-04-22 & 2019 & 3 & ECO
Frame & high-very high & National & 500-1000 & 1.8924259 & 1.3590316 & -1.2391092 & 0.9814602 & -0.7922579 & 0.0 & -0.9049211 & -1.0736569 & Recipient & Domestic & European & Mixed & Domestic|ECO & Positive\\
Greece & https://www.newsbomb.gr/ellada/news/story/892457/paidikoi-stathmoi-espa-2018-ayrio-xekinoyn-oi-aitiseis & 369 & Newsbomb.gr & Private/Non-Public & Online only & National & very low = CP mentioned once & Public services & Positive & EU + National & No myth & NA & NA & NA & NA & NA & NA & NA & NA & Greece & παιδικοί σταθμοί εσπα 2018: αύριο ξεκινούν οι αιτήσεις & 2018-06-12 & ευρωπαϊκό κοινωνικό ταμείο & αύριο το μεσημέρι αναμένεται από την εεταα να εκδοθεί η σχετική προκήρυξη - πρόσκληση για τους βρεφονηπιακούς παιδικούς σταθμούς ώστε οι γονείς να ξεκινήσουν να υποβάλουν τις αιτήσεις για την δωρεάν φιλοξενία των παιδιών τους. αύριο στις 12:30 το μεσημέρι θα γίνει το προγραμματισμενο δσ της εεταα για την έγκριση της πρόσκλησης ώστε το απόγευμα της ίδιας μέρας να βγει στην δημοσιότητα. οι πληροφορίες αναφέρουν ότι η προθεσμία που θα δώσει στους γονείς για να καταθέσουν τις αιτήσεις τους θα είναι 20 μέρες. οι αιτήσεις θα κατατιθενται μόνο ηλεκτρονικά. φέτος θα υπάρχουν αλλαγές σε ότι αφορά τα οικονομικά κριτήρια καθώς θα πρέπει να δηλωνονται και τα επιδόματα. επίσης σε 185 δήμους που υπάρχουν οι υποδομές δεν θα παρέχεται η δωρεάν φιλοξενία στα παιδία ηλικίας 4\_5 ετών καθώς αυτά θα μπορούν να γραφτούν στο προνηπιο. ωστόσο αν κάποιος γονεας επιθυμεί να γράψει το παιδί του σε παιδικό σταθμό θα μπορεί να το κάνει αλλά θα πληρώσει τα έξοδα. στόχος είναι ο αριθμός των παιδιών που θα φιλοξενηθούν την επόμενη σχολική χρονιά (2018-2019) να υπερβεί τις 110.000, από 106.000 που ήταν πέρυσι. σύμφωνα με την κυα η αρχική χρηματοδότηση του προγράμματος ανέρχεται στα 205 εκ. ευρώ, όσο ήταν και τη σχολική χρονιά 2017-2018, με τη διαφορά ότι φέτος είναι αυξημένη η εθνική συμμετοχή. συγκεκριμένα από το πδε θα διατεθούν 137.707.642 εκ. ευρώ, από 132 εκ. πέρυσι και τα υπόλοιπα 67.292.158 εκ. ευρώ καλύπτονται με με χρηματοδότηση από το ευρωπαϊκό κοινωνικό ταμείο στο πλαίσιο του εσπα 2014-2010. πέρυσι το αντίστοιχο ποσό ήταν 73 εκ. ευρώ. ωστόσο η χρηματοδότηση αναμένεται να φτάσει τα 220 εκ., από 216 εκ. ευρώ που διατέθηκαν τελικά για τη σχολική χρονιά 2017-2018. σύμφωνα με την κυα: ωφελούμενες/οι της δράσης είναι: - μητέρες βρεφών, νηπίων και παιδιών ή/και - μητέρες νηπίων, παιδιών, εφήβων και ατόμων με αναπηρία. - άτομα (γυναίκες και άνδρες), στα οποία έχει παραχωρηθεί με δικαστική απόφαση η επιμέλεια παιδιών. - άτομα που βρίσκονται σε χηρεία. οι ωφελούμενες/οι της δράσης θα πρέπει να πληρούν τις κάτωθι προϋποθέσεις: - να είναι εργαζόμενες/οι (μισθωτές/οί, αυτοαπασχολούμενες/οι), ή - να είναι άνεργες/οι οι οποίες/οι να διαθέτουν δελτίο ανεργίας σε ισχύ ή άλλο ισοδύναμο έγγραφο. από όλες τις ως άνω κατηγορίες εξαιρούνται οι τακτικοί υπάλληλοι του δημοσίου, των ν.π.δ.δ. καθώς και υπάλληλοι μόνιμοι και αορίστου χρόνου των ο.τ.α. (α΄ και β΄ βαθμού). τα ετήσια ανώτατα όρια δαπάνης σε ευρώ, που θα καταβάλλονται από την εεταα, ανά κατηγορία δομής και ανά κατηγορία θέσης αυτών , παραμένουν στα ίδια επίπεδα και είναι τα εξής : α. βρεφικοί-βρεφονηπιακοί-παιδικοί σταθμοί: α1.1. βρέφη από 2 μηνών έως 2,5 ετών: 2.375 € χωρίς σίτιση και 2.945 € με σίτιση α1.2. βρέφη από 8 μηνών έως 2,5 ετών: 2.375 € χωρίς σίτιση και 2.945 € με σίτιση α1.3. βρέφη από 18 μηνών έως 2,5 ετών: 2.375 € χωρίς σίτιση και 2.945 € με σίτιση α2. προ-νήπια από 2,5 ετών έως την ηλικία εγγραφής τους στην υποχρεωτική εκπαίδευση: 1805 € χωρίς σίτιση και 2.375 € με σίτιση. τα ανωτέρω ποσά αφορούν σε δομές που, σύμφωνα με το θεσμικό πλαίσιο αδειοδότησής τους, το ωράριο λειτουργίας τους είναι οκτώ (8) ώρες ημερησίως. β. βρεφονηπιακοί σταθμοί ολοκληρωμένης φροντίδας: β1. βρέφη από 8 μηνών έως 2,5 ετών: 2.375 € χωρίς σίτιση και 2.945 € με σίτιση β2. προ-νήπια από 2,5 ετών έως την ηλικία εγγραφής τους στην υποχρεωτική εκπαίδευση: 1.805 € χωρίς σίτιση και 2.375 € με σίτιση β3. προ-νήπια με αναπηρία από 2,5 ετών έως 6,5 ετών: 5.000 €. τα ανωτέρω ποσά αφορούν σε δομές που, σύμφωνα με το θεσμικό πλαίσιο αδειοδότησής τους, το ωράριο λειτουργίας τους είναι οκτώ (8) ώρες ημερησίως. γ. κέντρα δημιουργικής απασχόλησης παιδιών (κ.δ.α.π.) γ1. παιδιά από την ηλικία των 5 ετών (ηλικία εγγραφής τους στην υποχρεωτική εκπαίδευση) έως 12 ετών και παιδιά με ελαφράς μορφής κινητικά ή αισθητηριακά προβλήματα: 1.330 € για την κάθε βάρδια τεσσάρων (4) ωρών κατ' ελάχιστον. δ. κέντρα δημιουργικής απασχόλησης παιδιών με αναπηρίες (κ.δ.α.π.-μ.ε.α.). δ1. παιδιά με αναπηρία ή/και έφηβοι ή/και άτομα με νοητική υστέρηση ή/και κινητική αναπηρία: 5.000 €. το ανωτέρω ποσό αφορά σε δομές που, σύμφωνα με το θεσμικό πλαίσιο αδειοδότησής τους, το ωράριο λειτουργίας τους είναι οκτώ (8) ώρες ημερησίως. τα ανωτέρω ποσά καλύπτουν το συνολικό κόστος των παρεχόμενων υπηρεσιών και ως εκ τούτου οι δομές δεν μπορούν να προβούν σε είσπραξη τροφείων ή άλλης μορφής διδάκτρων ή οποιουδήποτε αντιτίμου από τους ωφελούμενους για την παροχή των υπηρεσιών που καλύπτει η "αξία τοποθέτησης" (voucher). σε περίπτωση που η ωφελούμενη λάβει voucher, το οποίο δεν ενεργοποίησε, λόγω έλλειψης θέσης στη συγκεκριμένη κατηγορία δομής στην οποία ανήκει ηλικιακά το παιδί, μπορεί να το ενεργοποιήσει για πρώτη φορά σε άλλη κατηγορία καθ' όλη τη διάρκεια της σχολικής χρονιάς, αντίστοιχη προς την εν τω μεταξύ σημειωθείσα μεταβολή της ηλικίας του παιδιού. στην περίπτωση αυτή αναπροσαρμόζεται αντιστοίχως η αξία του voucher. κριτήρια επιλογής και σύστημα μοριοδότησης τα κριτήρια επιλογής ωφελουμένων αφορούν το οικογενειακό εισόδημα ( το ύψος του θα περιγραφεί στην πρόσκληση της εεταα), στην κατάσταση απασχόλησης και την εργασιακή σχέση και την οικογενειακή κατάσταση. εφόσον πληρούνται οι προϋποθέσεις συμμετοχής, ισχύουν τα κάτωθι κριτήρια μοριοδότησης: α) ετήσιο δηλωθέν οικογενειακό εισόδημα 90 μόρια β) κατάσταση απασχόλησης, εργασιακή σχέση και είδος απασχόλησης: 90 μόρια λαμβάνουν: - οι εργαζόμενοι-ες με σύμβαση εργασίας μερικής απασχόλησης, - οι εργαζόμενοι-ες με σύμβαση εργασίας ορισμένου χρόνου, - οι εργαζόμενοι-ες με εργόσημο, - οι άνεργοι-ες, οι οποίοι/-ες διαθέτουν δελτίο ανεργίας σε ισχύ ή άλλο ισοδύναμο έγγραφο, 85 μόρια λαμβάνουν: - οι μισθωτοί-ες πλήρους απασχόλησης αορίστου χρόνου, - οι αυτοαπασχολούμενοι-ες. γ) οικογενειακή κατάσταση: 50 μόρια λαμβάνουν: - οι μητέρες που ανήκουν στην ομάδα των αμεα με ποσοστό αναπηρίας 35\% και άνω, - οι μητέρες που έχουν τέκνα που ανήκουν στην ομάδα των αμεα με ποσοστό αναπηρίας 35\% και άνω. 40 μόρια λαμβάνουν: - άτομα που βρίσκονται σε χηρεία, - οι μονογονεϊκές οικογένειες (άγαμες μητέρες), - οι μητέρες που είναι διαζευγμένες ή σε διάσταση, - οι μητέρες που είναι τρίτεκνες-πολύτεκνες, 30 μόρια λαμβάνουν: - οι μητέρες που έχουν σύζυγο που ανήκει στην ομάδα των αμεα με ποσοστό αναπηρίας 67\% και άνω, - οι μητέρες που έχουν άνεργο σύζυγο με δελτίο ανεργίας σε ισχύ ή άλλο ισοδύναμο έγγραφο. δ) αντί "συζύγων", ανάλογη εφαρμογή γίνεται και για άτομα που έχουν συνάψει σύμφωνο συμβίωσης και υπάγονται στις ρυθμίσεις του ν. 3719/2008 ή 4356/2015. & 1060 & very low & Low & Socio-Economic & NA & NA & 2018-06-12 & 2018 & 3 & ECO
Frame & v.low & National & +1000 & 1.8924259 & 1.3590316 & -1.2391092 & 0.9814602 & -0.7922579 & 0.0 & -0.9049211 & -1.0736569 & Recipient & Domestic & European & Mixed & Domestic|ECO & Positive\\
Greece & http://www.rodiaki.gr/article/371365/2-1-ekat-eyrw-apo-to-e-p-notio-aigaio-2014-2020-gia-erga-antiplhmmyrikhs-prostasias-sta-nhsia & 343 & Rodiaki.gr & Private/Non-Public & Online and Offline & Regional/Local & medium = CP is important part of story & Infrastructure & Positive & EU + Subnational & No myth & NA & NA & NA & NA & NA & NA & NA & NA & Greece & 2,1 εκατ. ευρώ  από το ε.π. νότιο αιγαίο 2014 - 2020,  για έργα αντιπλημμυρικής προστασίας στα νησιά | η ροδιακη & 2017-08-14 & ευρωπαϊκό ταμείο περιφερειακής ανάπτυξης & με σκοπό τον σχεδιασμό δράσεων οι οποίες θα συμβάλουν τόσο στην πρόληψη όσο και στην αντιμετώπιση των καταστροφών που προκαλούνται από τις πλημμυρικά φαινόμενα στα νησιά του νοτίου αιγαίου, ο περιφερειάρχης, γιώργος χατζημάρκος, απευθύνει πρόσκληση για την υποβολή προτάσεων για έργα πρόληψης και προστασίας, συνολικής επιλέξιμης δημόσιας δαπάνης 2.124.000 ευρώ, προκειμένου να ενταχθούν στο επιχειρησιακό πρόγραμμα "νότιο αιγαίο 2014 - 2020" στο πλαίσιο του εσπα 2014 - 2020. η πρόσκληση με τίτλο "αντιμετώπιση κινδύνων και καταστροφών - αντιπλημμυρικά έργα" απευθύνεται στους δυνητικούς δικαιούχους: περιφέρεια νοτίου αιγαίου δήμους περιφέρειας νοτίου αιγαίου δημοτικές επιχειρήσεις ύδρευσης - αποχέτευσης (δευα) περιφέρειας νοτίου αιγαίου επιλέξιμες δράσεις στην εν λόγω πρόταση είναι: έργα αντιπλημμυρικής προστασίας όπως έργα αύξησης της αντοχής των κοιτών στη διάβρωση, έργα διαμόρφωσης κατάλληλων κοιτών, έργα ανάσχεσης πλημμυρών, μικρά έργα οδοποιίας κατά μήκος των ρεμάτων κ.α. προμήθεια εξοπλισμού αντιπλημμυρικής προστασίας δεν είναι επιλέξιμες οι δράσεις: έργα συλλογής και αποχέτευσης ομβρίων έργα αποστράγγισης οδικών αξόνων και άλλων τεχνικών έργων παρεμβάσεις που περιλαμβάνουν κάλυψη ρέματος αντιπλημμυρικά έργα, στο πλαίσιο παρεμβάσεων ορεινής υδρονομίας σε δάση σημειακές παρεμβάσεις οι οποίες δεν τεκμηριώνεται ότι συμβάλλουν στην αντιπλημμυρική προστασία της περιοχής σημειώνεται ότι οι επιλέξιμες δράσεις σχεδιάζονται σε στενή συνεργασία με τα αρμόδια υπουργεία και είναι απολύτως ενταγμένες στον εθνικό σχεδιασμό για την αντιμετώπιση καταστροφών από πλημμύρες και πυρκαγιές. σε σχέση με τις πλημμύρες και σε εφαρμογή της οδηγίας 2007/60/εκ, έχει ολοκληρωθεί η προκαταρκτική αξιολόγηση των κινδύνων πλημμύρας από την ειδική γραμματεία υδάτων του υπεν, που περιλαμβάνει και τον προσδιορισμό των ζωνών δυνητικά υψηλού κινδύνου πλημμύρας στο υδατικό διαμέρισμα των νήσων αιγαίου, που αφορά και στην γεωγραφική περιοχή της περιφέρειας νοτίου αιγαίου. οι δράσεις αντιπλημμυρικής προστασίας που θα υλοποιηθούν, θα επικεντρωθούν στις περιοχές των παραπάνω ζωνών. για την υλοποίηση ανάλογων δράσεων εκτός των ζωνών αυτών, απαιτείται τεκμηρίωση της σκοπιμότητας και αιτιολόγηση σχετικά με την ανάσχεση πλημμυρικών φαινομένων εντός της ζώνης. η συγχρηματοδοτούμενη επιλέξιμη δημόσια δαπάνη, από το ευρωπαϊκό ταμείο περιφερειακής ανάπτυξης,,ανέρχεται σε 2.124.000 ευρώ. ως ελάχιστος προϋπολογισμός των υποβαλλόμενων πράξεων, ορίζεται το ποσό των 150.000 ευρώ. προθεσμία υποβολής προτάσεων από τις 14/08/2017 έως τις 30/11/2017. πληροφορίες στην ιστοσελίδα www.pepna.gr. & 351 & medium & Medium & Socio-Economic & NA & NA & 2017-08-14 & 2017 & 2 & ECO
Frame & low-medium & Regional & <500 & 1.8924259 & 1.3590316 & -1.2391092 & 0.9814602 & -0.7922579 & 0.0 & -0.9049211 & -1.0736569 & Recipient & Domestic & European & Mixed & Domestic|ECO & Positive\\
Greece & http://newpost.gr/ellada/693979/pente-erga-40-ekat-eyrw-ylopoiei-h-kentrikh-enwsh-epimelhthriwn-ellados & 296 & Newpost.gr & Private/Non-Public & Online only & National & medium = CP is important part of story & Economic development & Factual & EU + National & No myth & Jobs & Factual & EU + National & No myth & Research \& innovation & Factual & EU + National & No myth & Greece & πέντε έργα 40 εκατ. ευρώ υλοποιεί η κεντρική ένωση επιμελητηρίων ελλάδος & 2018-09-16 & ευρωπαϊκό κοινωνικό ταμείο & πρόκειται για έργα ενταγμένα στο εσπα 2014-2020 και συγχρηματοδοτούνται από το ευρωπαϊκό κοινωνικό ταμείο με συνολικό προϋπολογισμό που υπερβαίνει τα 40 εκατ. ευρώ η κεντρική ένωση επιμελητηρίων ελλάδος (κεεε) έχει αναλάβει ως τελικός δικαιούχος την υλοποίηση 5 έργων σε κρίσιμους τομείς για τις ανάγκες των επιχειρήσεων, τα οποία είναι ενταγμένα στο εσπα 2014-2020 και συγχρηματοδοτούνται από το ευρωπαϊκό κοινωνικό ταμείο (εκτ). σύμφωνα με τα στοιχεία της κεντρικής ένωσης επιμελητηρίων, τα έργα αυτά είναι: 1. "αναβάθμιση λειτουργιών της κεντρικής ένωσης επιμελητηρίων ελλάδος και των επιμελητηρίων της χώρας για την υποστήριξη του επιχειρηματικού περιβάλλοντος". το έργο προβλέπει την υλοποίηση μίας σειράς πολυεπίπεδων υποστηρικτικών δράσεων για τη βέλτιστη διαδικασία αδειοδότησης επιχειρήσεων, από την υποβολή της αίτησης μιας επιχείρησης μέχρι και την έκδοση της τελικής πράξης, με γνώμονα τη συνεχή μείωση του διοικητικού φόρτου για τις επιχειρήσεις. το έργο είναι σε εναρμόνιση με τον ψηφισθέντα νόμο 4497/13-11-2017 "'ασκηση υπαίθριων εμπορικών δραστηριοτήτων, εκσυγχρονισμός της επιμελητηριακής νομοθεσίας και άλλες διατάξεις", μέσω του οποίου προβλέπεται η σύσταση κ.υ.επιχ. στα επιμελητήρια. δημιουργείται ηλεκτρονική πλατφόρμα, μέσω της οποίας θα επιτρέπεται η υποστήριξη των στελεχών των κ.υ.επιχ. και των επιχειρηματιών με στόχο τη μεσοπρόθεσμα "paperless" διευθέτηση και διαχείριση των αιτημάτων/ζητημάτων των επιχειρήσεων σε πάσης φύσης διεκπεραίωση διοικητικών ζητημάτων σε συνεργασία με τις αρμόδιες δημόσιες υπηρεσίες. 2. "αναβάθμιση των ψηφιακών δεξιοτήτων 15.000 εργαζομένων στον ιδιωτικό τομέα". το έργο στοχεύει στην ενίσχυση 15.000 εργαζόμενων σε επιχειρήσεις του ιδιωτικού τομέα και περιλαμβάνει την υλοποίηση ενεργειών διάγνωσης των εκπαιδευτικών αναγκών των ωφελουμένων, την υλοποίηση στοχευμένων προγραμμάτων επαγγελματικής κατάρτισης στη χρήση τεχνολογιών πληροφορικής και επικοινωνιών καθώς και την πιστοποίηση των καταρτισθέντων. 3. "αναβάθμιση και διεύρυνση του θεσμού της μαθητείας". to έργο αφορά στην ενημέρωση και ευαισθητοποίηση των επιχειρήσεων για τον θεσμό της μαθητείας και για τα οφέλη που προκύπτουν από την υιοθέτηση και εφαρμογή του καθώς και στην υποστήριξη των επιχειρήσεων με στόχο την προσφορά 4.000 θέσεων μαθητείας καθώς και την ομαλή ένταξη των μαθητευομένων στο εργασιακό περιβάλλον. 4. "συμβουλευτική, κατάρτιση και πιστοποίηση 1.270 ανέργων". το έργο αφορά στην υλοποίηση μιας δέσμης συνεκτικών ενεργειών συμβουλευτικής, κατάρτισης και πιστοποίησης, οι οποίες αφορούν σε 1.270 ωφελούμενους, μακροχρόνια άνεργους και άτομα που ανήκουν στις ευπαθείς κοινωνικά ομάδες. 5. έργο e-γεμη. αναλυτικότερα, αφορά στο έργο "ψηφιακός μετασχηματισμός του γενικού εμπορικού μητρώου (γεμη) για την κατάθεση ισολογισμών με προηγμένες ψηφιακές υπογραφές και παροχή απομακρυσμένων ψηφιακών υπογραφών προς τις επιχειρήσεις", προϋπολογισμού 11,3 εκατ. ευρώ, το οποίο σχεδιάστηκε μαζί με το υπουργείο οικονομίας και ανάπτυξης και χρηματοδοτείται από το υπουργείο ψηφιακής πολιτικής, τηλεπικοινωνιών και ενημέρωσης , υπενθυμίζεται ότι στόχος του e-γεμη είναι ο εκσυγχρονισμός της ελληνικής επιχειρηματικότητας μέσω της αυτοματοποίησης των διαδικασιών στις οποίες προβαίνουν οι επιχειρήσεις της χώρας, όπως η κατάθεση των ισολογισμών. με τη λειτουργία του έργου θα παρασχεθούν δωρεάν 130.000 απομακρυσμένες ψηφιακές υπογραφές προς τις επιχειρήσεις μέσα από την κεντρική ένωση επιμελητηρίων ελλάδος και συγχρόνως η τριμερής συνεργασία, μέσω του μνημονίου που ήδη υπογράφτηκε, θα προχωρήσει στους γενικότερους στόχους για απλοποίηση και ψηφιοποίηση των διαδικασιών που διέπουν την επιχειρηματικότητα και τελικά την περαιτέρω ανάπτυξη διαλειτουργικότητας μεταξύ του γενικού εμπορικού μητρώου και άλλων δημοσίων πληροφοριακών συστημάτων και βάσεων. "σημείωνεται ότι είναι η πρώτη φορά που τα επιμελητήρια της χώρας συμμετέχουν τόσο ενεργά και δυναμικά στο σχεδιασμό και στην υλοποίηση συγχρηματοδοτούμενων έργων από το ευρωπαϊκό κοινωνικό ταμείο, γεγονός που αναμφίβολα πιστώνεται στο δυναμισμό και την εξωστρέφεια που επιδεικνύει τα τελευταία ιδίως χρόνια ο επιμελητηριακός χώρος" όπως τόνισε στο απε-μπε ο πρόεδρος της κεε και του εβεα κωνσταντίνος μίχαλος. ο ίδιος πρόσθεσε ότι στην περίοδο της κρίσης, οι ελληνικές επιχειρήσεις δοκιμάστηκαν και δοκιμάζονται ακόμη και σήμερα έντονα, ταυτόχρονα όμως οι προσπάθειες τους για την προσαρμογή στα νέα δεδομένα, δεν σταμάτησαν ποτέ. ιδιαίτερα, όταν διαφάνηκαν συνθήκες σταθεροποίησης και ανάκαμψης ο επιχειρηματικός κόσμος, αναζήτησε και επεδίωξε να διαμορφώσει ένα νέο, συλλογικό σχέδιο δράσεων και ενεργειών στήριξης της επιχειρηματικότητας κι όλες αυτές οι εξελίξεις και οι αλλαγές καταγράφονται και στη πορεία των επιμελητηρίων της χώρας και της κεντρικής ένωσης. η κεντρική ένωση επιμελητηρίων ελλάδος, συνέχισε ο κ. μίχαλος, εργάστηκε και εργάζεται συστηματικά ώστε οι κοινοτικοί πόροι, κυρίως αυτοί που προέρχονται από τα προγράμματα του εσπα, να υποστηρίξουν κρίσιμους τομείς για τις ανάγκες των επιχειρήσεων. "αξιόλογες προτάσεις της κεεε, συγχρηματοδοτημένες από το εσπα 2007-2013, υλοποιήθηκαν με μεγάλη επιτυχία. σήμερα, που το εσπα 2014-2020 είναι σε πλήρη εξέλιξη, διαθέτοντας σημαντικούς πόρους για το επιχειρείν, η κεεε βρίσκεται στη ευχάριστη θέση να έχει δρομολογήσει την υλοποίηση μιας σειράς έργων, τα οποία αφορούν σε θέματα κατάρτισης και πιστοποίησης του προσωπικού των επιχειρήσεων σε σύγχρονες ειδικότητες, αντιμετώπισης της ανεργίας, περαιτέρω θεσμικού εκσυγχρονισμού του πλαισίου για το επιχειρείν, ανάπτυξης και ενίσχυσης του θεσμού της μαθητείας, κ.α.", είπε ο κ. μίχαλος. & 771 & medium & Medium & Socio-Economic & Socio-Economic & Socio-Economic & 2018-09-16 & 2018 & 3 & ECO
Frame & low-medium & National & 500-1000 & 1.8924259 & 1.3590316 & -1.2391092 & 0.9814602 & -0.7922579 & 0.0 & -0.9049211 & -1.0736569 & Recipient & Domestic & European & Mixed & Domestic|ECO & Neutral\\
\addlinespace
Greece & http://www.tovima.gr/society/article/?aid=906383 & 350 & TO BHMA International & Private/Non-Public & Online and Offline & National & medium = CP is important part of story & Public services & Positive & National & No myth & Social awareness/inclusion & Positive & National & No myth & NA & NA & NA & NA & Greece & tovima.gr - ξανθός: σχέδια για μόνιμη επιτροπή αντιμετώπισης της εξάρτησης από το αλκοόλ & 2017-10-09 & διαρθρωτικά ταμεία & τα σχέδια του υπουργείου υγείας για τη συγκρότηση μόνιμης επιτροπής για την αντιμετώπιση της εξάρτησης από το αλκοόλ, αποκάλυψε ο υπουργός ανδρέας ξανθός, μιλώντας στο 16ο διεθνές συνέδριο της ευρωπαϊκής εταιρείας για την βιοϊατρική έρευνα για τον αλκοολισμό (esbra) στο ηράκλειο κρήτης. κατά τη διάρκεια του χαιρετισμού που απηύθυνε ο υπουργός υγείας τόνισε ότι "το αλκοόλ αποτελεί ένα σοβαρό πρόβλημα δημόσιας υγείας που επιβαρύνει πολλαπλά τον εξαρτημένο, το οικογενειακό του περιβάλλον και το σύστημα υγείας" και συμπλήρωσε ότι "κατά τη διάρκεια της οικονομικής κρίσης στην ελλάδα η εξάρτηση από το αλκοόλ φαίνεται να κερδίζει έδαφος, επιβαρύνει περισσότερο τα χαμηλότερα κοινωνικά στρώματα και συμβάλλει περαιτέρω στις κοινωνικές ανισότητες, στην περιθωριοποίηση και στον κοινωνικό αποκλεισμό". ο κ. ξανθός υποστήριξε ότι, ενώ στην χώρα μας η κατά κεφαλήν κατανάλωση αλκοόλ είναι συγκριτικά χαμηλότερη από το μέσο ευρωπαϊκό όρο, αλλά "μας προβληματίζει όμως το γεγονός της αύξησης της υπερβολικής κατανάλωσης αλκοόλ στους εφήβους και της "βαριάς" κατά περίπτωση (επεισοδιακής) χρήσης ειδικά στην επαρχία". για τους παραπάνω λόγους το υπουργείο υγείας, σύμφωνα με τον κ. ξανθό σχεδιάζει τη συγκρότηση μόνιμης επιτροπής για την αντιμετώπιση της εξάρτησης από το αλκοόλ, στο πλαίσιο της αναδιοργάνωσης του θεσμικού πλαισίου που αφορά συνολικά τις εξαρτήσεις και την εκπόνηση ενιαίου εθνικού στρατηγικού σχεδίου πρόληψης και αντιμετώπισης τους. "σχεδιάζονται, με χρηματοδότηση από διαρθρωτικά ταμεία της εε, πολυδύναμα κέντρα αντιμετώπισης όλου του φάσματος των εξαρτήσεων ειδικά στην περιφέρεια, κινητές μονάδες για τη διευκόλυνση της πρόσβασης σε μη αστικές περιοχές, νέες μονάδες σωματικής αποτοξίνωσης (από οπιούχα και αλκοόλ), μονάδες απεξάρτησης από το αλκοόλ και τις αναδυόμενες εξαρτήσεις (διαδίκτυο, τζόγος) και προγράμματα κοινωνικής επανένταξης εστιασμένα στις ανάγκες κάθε ομάδας του πληθυσμού-στόχου. κρίσιμο ρόλο, επίσης, μπορεί να διαδραματίσει η ανάπτυξη της πφυ και η προτεραιότητα στην πρόληψη και στην παρέμβαση σε επίπεδο κοινότητας, η εκπαίδευση των επαγγελματιών υγείας στη διαχείριση προβλημάτων που σχετίζονται με το αλκοόλ και η δικτύωση των δομών της πφυ με τις αντίστοιχες της ψυχικής υγείας και της αντιμετώπισης των εξαρτήσεων", εξήγησε ο υπουργός υγείας. τέλος, υπενθυμισε ότι υπουργείο είναι σε συνεργασία με τα συναρμόδια υπουργεία, αλλά και με την εε και τον που, για τον έλεγχο της διαθεσιμότητας των αλκοολούχων ποτών σε ευαίσθητες ομάδες του πληθυσμού (ανήλικοι, έγκυες), της σήμανσης τους με τις απαραίτητες πληροφορίες, τον αυστηρό έλεγχο της οδήγησης υπό την επήρεια αλκοόλ και την αναβαθμισμένη επιτήρηση του προβλήματος σε εθνικό και ευρωπαϊκό επίπεδο. & 385 & medium & Medium & Socio-Economic & Socio-Economic & NA & 2017-10-09 & 2017 & 2 & ECO
Frame & low-medium & National & <500 & 1.8924259 & 1.3590316 & -1.2391092 & 0.9814602 & -0.7922579 & 0.0 & -0.9049211 & -1.0736569 & Recipient & Domestic & Domestic & Domestic & Domestic|ECO & Positive\\
Greece & https://www.dikaiologitika.gr/eidhseis/aftodioikisi/229285/agorastos-stis-vrykselles-ergaleio-gia-perissoteres-ependyseis-stis-koinonikes-ypodomes-to-programma-investeu & 387 & dikaiologitika.gr & Private/Non-Public & Online only & National & high = CP is most important issue in story (can also cover other issues) & Social justice & Positive & EU & No myth & Empowerment of institutions & Balanced & EU + National & No myth & NA & NA & NA & NA & Greece & αγοραστός στις βρυξέλλες: "εργαλείο για περισσότερες επενδύσεις στις κοινωνικές υποδομές το πρόγραμμα investeu" - dikaiologitika news & 2018-10-18 & πολιτική συνοχής & αυτό επισήμανε ο πρόεδρος της ένωσης περιφερειών ελλάδας (ενπε), περιφερειάρχης θεσσαλίας και εισηγητής (rapporteur) της επιτροπής των περιφερειών της εε για το investeu κ. κώστας αγοραστός μιλώντας σήμερα στις βρυξέλλες σε ευρωπαϊκό συνέδριο με θέμα: "επενδύσεις στις κοινωνικές υποδομές: από τις καινοτόμες ιδέες σε τοπικό επίπεδο στις χρηματοδοτήσεις μέσω του προγράμματος investeu". "είναι γεγονός αδιαμφισβήτητο ότι η χρηματοδότηση για τις κοινωνικές υποδομές έχει μειωθεί δραστικά λόγω των μέτρων λιτότητας που έχουν ληφθεί σε πολλές χώρες της εε μετά την κρίση. αυτό πρέπει να αλλάξει", προέτρεψε και ανέδειξε το γεγονός ότι στο νέο επενδυτικό πρόγραμμα της εε, που ξεκινά το 2021, προβλέπεται ξεχωριστό σκέλος πολιτικής για τις κοινωνικές επενδύσεις και τις δεξιότητες. "είναι η πρώτη φορά που προβλέπεται ένας τέτοιος πυλώνας χρηματοδότησης, ανοίγει μεγάλο παράθυρο ευκαιρίας ιδίως για χώρες που έχουν πληγεί από την κρίση και οφείλουμε να το αξιοποιήσουμε αρκεί να κινηθούμε με σχέδιο, συνεργασία, ταχύτητα, ευρηματικότητα και ευελιξία", σημείωσε ο κ. κ. αγοραστός προβαίνοντας στην εκτίμηση ότι το συνολικό ύψος του προγράμματος μπορεί να ανέλθει στο 1 τρις ευρώ. συγκολλητική ουσία του ευρωπαϊκού οικοδομήματος η κοινωνική πολιτική μαζί με την πολιτική συνοχής και την καπ. ο πρόεδρος της ένωσης περιφερειών ελλάδας χαρακτήρισε την κοινωνική πολιτική μαζί με την πολιτική συνοχής και την κοινή αγροτική πολιτική ως "τη συγκολλητική ουσία του ευρωπαϊκού οικοδομήματος. το ευρωπαϊκό κοινωνικό κράτος δεν πρέπει να το προσεγγίζουμε ως απολίθωμα του παρελθόντος αλλά ως κεντρική πολιτική επιλογή για το μέλλον, ως βασικό εργαλείο για να έχουν όλοι οι ευρωπαίοι πολίτες ένα καλύτερο μέλλον με ευημερία και κοινωνική δικαιοσύνη", είπε με έμφαση. "ναι" στη συνεργασία αυτοδιοίκησης με κοινωνικούς φορείς: "είτε θα βρούμε τον δρόμο είτε θα ανοίξουμε καινούργιο" απαντώντας σε ερωτήσεις συνέδρων, ο κ. κ. αγοραστός δήλωσε ένθερμος υποστηρικτής της συνεργασίας αυτοδιοίκησης με τους κοινωνικούς φορείς για τη δρομολόγηση και υλοποίηση επενδύσεων στον τομέα των κοινωνικών υποδομών. χρησιμοποίησε μάλιστα χαρακτηριστικά τη φράση του στρατηγού των καρχηδονίων αννίβα, ο οποίος όταν ερωτήθηκε πως σχεδιάζει να διαβούν τα στρατεύματά του τις άλπεις κινούμενα προς τη ρώμη, απάντησε: "είτε θα βρούμε τον δρόμο είτε θα ανοίξουμε καινούργιο". όσο μεγαλύτερο βαθμό ελευθερίας έχει η αυτοδιοίκηση από το κεντρικό κράτος, τόσο βαθύτερο το κοινωνικό και αναπτυξιακό αποτύπωμά της ωστόσο, κατέστησε σαφές ότι "η αυτοδιοίκηση χρειάζεται να αποκτήσει μεγαλύτερο βαθμό ελευθερίας στις χώρες που βρίσκεται υπό τον ασφυκτικό περιορισμό του κεντρικού κράτους. η χώρα μου, η ελλάδα, είναι ένα χαρακτηριστικό τέτοιο παράδειγμα, αλλά όχι το μοναδικό", σημείωσε και πρόσθεσε: στην ελλάδα τα περιφερειακά συμβούλια δεν έχουν τη δυνατότητα να νομοθετούν για θέματα της αρμοδιότητάς τυ, όπως συμβαίνει σε άλλες χώρες της εε. οι αρμοδιότητες της αυτοδιοίκησης είναι περιορισμένες, ενώ όταν το κεντρικό κράτος εκχωρεί αρμοδιότητες, δεν τις συνοδεύει με τους ανάλογους πόρους. υπολογίστε ότι επί του συνόλου των δημοσίων επενδύσεων στην ελλάδα, μόνο το 18\% προέρχεται από την αυτοδιοίκηση, ενώ, για παράδειγμα, στο βέλγιο το αντίστοιχο ποσοστό είναι 92\%. παρ' όλα αυτά στις περιφέρειες στηρίξαμε την κοινωνία στα χρόνια της κρίσης, προχωρήσαμε παραγωγικές επενδύσεις και ολοκληρώσαμε χιλιάδες έργα συντελώντας στο να αποκτήσει η χώρα μας σύγχρονες υποδομές και επιλύοντας προβλήματα δεκαετιών. ταυτόχρονα, μαζί με την κοινωνία προγραμματίζουμε και ωριμάζουμε έργα και πρωτοβουλίες για την επόμενη προγραμματική περίοδο ώστε η ελλάδα να σταθεί δυνατά στα πόδια της. συνεπώς, είναι αποδεδειγμένο ότι όσο μεγαλύτερο βαθμό ελευθερίας έχει η αυτοδιοίκηση από το κεντρικό κράτος, τόσο βαθύτερο θα είναι το κοινωνικό και αναπτυξιακό αποτύπωμά της", επισήμανε ο κ. κ. αγοραστός. ποιες δράσεις και έργα χρηματοδοτεί το investeu στην ομιλία του ο κ. κ. αγοραστός εξήγησε ότι το σκέλος πολιτικής κοινωνικών επενδύσεων και δεξιοτήτων του investeu περιλαμβάνει τη μικροχρηματοδότηση και τη χρηματοδότηση κοινωνικών επιχειρήσεων και της κοινωνικής οικονομίας· δεξιότητες, εκπαίδευση, κατάρτιση και συναφείς υπηρεσίες· κοινωνικές υποδομές (συμπεριλαμβανομένης της κοινωνικής και φοιτητικής στέγασης)· κοινωνική καινοτομία· την υγεία και τη μακροχρόνια περίθαλψη, ένταξη και προσβασιμότητα· πολιτιστικές δραστηριότητες με κοινωνικό σκοπό· ενσωμάτωση ευάλωτων ατόμων, συμπεριλαμβανομένων υπηκόων τρίτων χωρών. *το συνέδριο διοργάνωσαν η ευρωπαϊκή ένωση παροχής υπηρεσιών σε άτομα με αναπηρία (easpd), η ευρωπαϊκή συνομοσπονδία δημόσιας, συνεταιριστικής και κοινωνικής κατοικίας (housing europe), η ευρωπαϊκή σύμπραξη για τη βελτίωση της υγείας, της ισότητας και της ποιότητας ζωής (eurohealthnet), η ευρωπαϊκή συνομοσπονδία των εθνικών οργανισμών που ασχολούνται με τους άστεγους (feantsa), η πλατφόρμα οργανώσεων ηλικιωμένων και μκο που ασχολούνται με ηλικιωμένους age platform-europe και η πλατφόρμα για τη δία βίου εκπαίδευση (lifelong learning platform), στην οποία μετέχουν 43 ευρωπαϊκοί οργανισμοί που δραστηριοποιούνται στην εκπαίδευση και τη νεολαία. πηγή: απε & 716 & high & High & Socio-Economic & Power & NA & 2018-10-18 & 2018 & 3 & ECO
Frame & high-very high & National & 500-1000 & 1.8924259 & 1.3590316 & -1.2391092 & 0.9814602 & -0.7922579 & 0.0 & -0.9049211 & -1.0736569 & Recipient & European & European & European & European|ECO & Positive\\
Greece & http://voria.gr/article/ke-gia-to-metro-thessalonikis-sizitisan-mitsotakis-kretsou & 300 & Ernst\&Young: Φρένο στις δημόσιες εγγραφές παγκοσμίως & Private/Non-Public & Online only & National & high = CP is most important issue in story (can also cover other issues) & Bureaucracy and/or delays & Negative & EU & No myth & Economic development & Positive & National & No myth & NA & NA & NA & NA & Greece & νδ: καμπανάκι κρέτσου σε μητσοτάκη για το μετρό θεσσαλονίκης & 2016-12-22 & περιφερειακή πολιτική & τι συζήτησαν ο πρόεδρος της νέας δημοκρατίας κυριάκος μητσοτάκης με την επίτροπο περιφερειακής πολιτικής της ευρωπαϊκής ένωσης κορίνα κρέτσου. την πορεία των εργασιών για την κατασκευή του μετρό στην θεσσαλονίκη συζήτησαν, μεταξύ άλλων, ο πρόεδρος της νέας δημοκρατίας, κυριάκος μητσοτάκης με την επίτροπο περιφερειακής πολιτικής της ευρωπαϊκής ένωσης, κορίνα κρέτσου, στο πλαίσιο της συνάντησής τους. σύμφωνα με ανακοίνωση της νδ, η κ. κρέτσου ανέφερε τις ιδιαίτερες δυσκολίες που παρουσιάζονται στο πρόγραμμα ολοκλήρωσης του μετρό θεσσαλονίκης, καθώς και της χρηματοδότησης των αρχαιολογικών ανασκαφών, ενώ ζήτησε από τον πρόεδρο της νέας δημοκρατίας τη συμβολή του στην επίλυση αυτών των προβλημάτων. παράλληλα, η ευρωπαία επίτροπος ενημέρωσε τον αρχηγό της αξιωματικής αντιπολίτευσης για την πορεία απορρόφησης του τρέχοντος προγράμματος περιφερειακής ανάπτυξης, αλλά και της απορρόφησης των κοινοτικών κονδυλίων που σχετίζονται με την αντιμετώπιση του μεταναστευτικού - προσφυγικού προβλήματος, όπου παρουσιάζονται καθυστερήσεις, όπως επισημαίνεται στην ανακοίνωση της ν.δ. επίσης, ενημέρωσε τον κυριάκο μητσοτάκη και για τη συνεργασία της ευρωπαϊκής επιτροπής με τις περιφέρειες της χώρας, αλλά και τη συμβολή της επιτροπής σε θέματα διοικητικής μεταρρύθμισης. ο κ. μητσοτάκης επισήμανε, ότι η περιφερειακή πολιτική της ευρωπαϊκής ένωσης για τη νέα δημοκρατία δεν μπορεί να έχει κομματικές παρωπίδες, καθώς η πολιτική αυτή στοχεύει στην αναβάθμιση της ποιότητας της ζωής όλων των πολιτών. τέλος, ο πρόεδρος της νέας δημοκρατίας ζήτησε να ενημερωθεί για τα δυο πιο επιτυχημένα κοινοτικά προγράμματα, της κατάρτισης και των μικρομεσαίων επιχειρήσεων. & 227 & high & High & Governance & Socio-Economic & NA & 2016-12-22 & 2016 & 2 & POL
Frame & high-very high & National & <500 & 1.8924259 & 1.3590316 & -1.2391092 & 0.9814602 & -0.7922579 & 0.0 & -0.9049211 & -1.0736569 & Recipient & European & European & European & European|POL & Negative\\
Greece & http://www.insider.gr/eidiseis/ellada/95175/perifereia-attikis-tria-nea-sholeia-se-nea-peramo-ki-agia-paraskeyi & 330 & insider.gr & Private/Non-Public & Online only & National & very low = CP mentioned once & Infrastructure & Positive & Subnational & No myth & Public services & Positive & Subnational & No myth & NA & NA & NA & NA & Greece & περιφέρεια αττικής: τρία νέα σχολεία σε νέα πέραμο κι αγία παρασκευή & 2018-09-19 & ευρωπαϊκό ταμείο περιφερειακής ανάπτυξης & δύο ακόμα έργα με στόχο τη βελτίωση της καθημερινότητας των πολιτών και την ενίσχυση των υποδομών εκπαίδευσης στην αττική χρηματοδοτεί μέσω του πεπ αττικής η περιφέρεια αττικής, με τη χρηματοδότηση της κατασκευής νέου δημοτικού και νηπιαγωγείου στη νέα πέραμο και νέου νηπιαγωγείου στην αγία παρασκευή. συγκεκριμένα, η περιφερειάρχης αττικής υπέγραψε την έγκριση χρηματοδότησης του έργου "2ο δημοτικό σχολείο και 2ο νηπιαγωγείο νέας περάμου - οικοδομικές εργασίες και η/μ εγκαταστάσεις" από το επιχειρησιακό πρόγραμμα "αττική 2014-2020". το έργο έχει προϋπολογισμό 4.920.000 ευρώ και αφορά στην κατασκευή νέου δημοτικού και νηπιαγωγείου συνολικής επιφάνειας 3.562,94 τ.μ. σε οικόπεδο 11.207,27 τ.μ. επί της οδού νερακίου στην περιοχή "λάκκα καλογήρου" της δημοτικής κοινότητας νέας περάμου του δήμου μεγαρέων. τα νέα σχολεία θα έχουν δυναμικότητα 300 μαθητών και θα αποτελούνται: νέο νηπιαγωγείο στην αγία παρασκευή ακολούθως, η περιφερειάρχης αττικής υπέγραψε την έγκριση χρηματοδότησης του έργου "14ο νηπιαγωγείο αγίας παρασκευής - οικοδομικές εργασίες και η/μ εγκαταστάσεις" στο επιχειρησιακό πρόγραμμα "αττική 2014-2020". το νέο νηπιαγωγείο, προϋπολογισμού 1.000.000 ευρώ, θα έχει συνολική επιφάνεια 336,36 τ.μ. και θα κατασκευαστεί σε οικόπεδο 1.187,61 τ.μ. στο ο.τ. 328 α στην π.ε. πευκάκια του δήμου αγίας παρασκευής. θα έχει δυναμικότητα 50 μαθητών (νηπίων) και θα αποτελείται από είσοδο με πολυδύναμο χώρο, 2 αίθουσες διδασκαλίας, αίθουσα ύπνου, τραπεζαρία, κουζίνα και γραφείο νηπιαγωγών. και οι δύο ανωτέρω δράσεις εντάσσονται στον άξονα προτεραιότητας 11 ("ανάπτυξη - αναβάθμιση στοχευμένων υποδομών εκπαίδευσης"), ο οποίος συγχρηματοδοτείται από το ευρωπαϊκό ταμείο περιφερειακής ανάπτυξης (ετπα) και ο οποίος πλέον έχει ενεργοποιηθεί στο 100\% σε σχέση με τον εγκεκριμένο προϋπολογισμό, εμφανίζοντας προσκλήσεις που ξεπερνούν τα 77.000.000 ευρώ. & 275 & very low & Low & Socio-Economic & Socio-Economic & NA & 2018-09-19 & 2018 & 3 & ECO
Frame & v.low & National & <500 & 1.8924259 & 1.3590316 & -1.2391092 & 0.9814602 & -0.7922579 & 0.0 & -0.9049211 & -1.0736569 & Recipient & Domestic & Domestic & Domestic & Domestic|ECO & Positive\\
Greece & http://avgi.gr/article/10809/9659171/sten-9e-these-e-ellada-se-pososto-kainotomon-epicheireseon-sten-europe & 303 & avgi.gr & Private/Non-Public & Online and Offline & National & very low = CP mentioned once & Research \& innovation & Positive & National & No myth & NA & NA & NA & NA & NA & NA & NA & NA & Greece & στην 9η θέση η ελλάδα σε ποσοστό καινοτόμων επιχειρήσεων στην ευρώπη & 2019-03-06 & ευρωπαϊκό ταμείο περιφερειακής ανάπτυξης & σύμφωνα με τα αποτελέσματα eurostat της πανευρωπαϊκής έρευνας για την καινοτομία των επιχειρήσεων στην εε ανέβηκε τρεις θέσεις η ελλάδα στην ευρωπαϊκή κατάταξη σε ποσοστό καινοτόμων επιχειρήσεων την τριετία 2014-2016, σε σχέση με την προηγούμενη τριετία 2012-2014, κατακτώντας πλέον την 9η θέση. το ποσοστό των καινοτόμων ελληνικών επιχειρήσεων ανήλθε στο 57,7\%, παρουσιάζοντας άνοδο κατά 6,7 ποσοστιαίες μονάδες σε σχέση με την τριετία 2012-2014, σύμφωνα με τα στοιχεία της επίσημης στατιστικής πανευρωπαϊκής έρευνας cis (community innovation survey). για την ελλάδα την έρευνα πραγματοποίησε το εθνικό κέντρο τεκμηρίωσης (εκτ), σε συνεργασία με την ελληνική στατιστική αρχή (ελστατ). η έρευνα αφορά 11.000 περίπου ελληνικές επιχειρήσεις με 10 εργαζόμενους και άνω, σε διάφορους κλάδους οικονομικής δραστηριότητας. όπως προκύπτει από τα αποτελέσματα, που ανακοίνωσε η eurostat, η ελλάδα μοιράζεται με τη γαλλία την 9η θέση, με ποσοστό καινοτόμων επιχειρήσεων 57,7\%. πρώτη χώρα στη ευρωπαϊκή κατάταξη είναι το βέλγιο με ποσοστό καινοτόμων επιχειρήσεων 68,1\%, ενώ ακολουθούν η πορτογαλία με ποσοστό 66,9\% και η φινλανδία με 64,8\%. τη δεκάδα κλείνει η ιρλανδία με ποσοστό καινοτόμων επιχειρήσεων 57,3\%. αξίζει να σημειωθεί ότι το ποσοστό των καινοτόμων επιχειρήσεων στην ελλάδα είναι υψηλότερο κατά 7,1 ποσοστιαίες μονάδες σε σχέση με τον μέσο όρο των 28 ευρωπαϊκών κρατών (50,6\%). όπως προκύπτει από τα αποτελέσματα για την τριετία 2014-2016, οι ελληνικές επιχειρήσεις που καινοτομούν εφάρμοσαν έναν τουλάχιστον από τους ακόλουθους τύπους καινοτομίας: νέα ή σημαντικά βελτιωμένα προϊόντα (καινοτομία προϊόντος), νέες ή σημαντικά βελτιωμένες διαδικασίες (καινοτομία διαδικασίας), νέες οργανωσιακές μεθόδους (οργανωσιακή καινοτομία), νέες μεθόδους μάρκετινγκ (καινοτομία μάρκετινγκ). ιδιαίτερα σημαντική είναι η θέση της ελλάδας στην καινοτομία προϊόντος ή/και διαδικασίας, με ποσοστό 47,1\%, το οποίο και την κατατάσσει στην 7η θέση μεταξύ των ευρωπαϊκών κρατών και πολύ πάνω από τον ευρωπαϊκό μέσο όρο (39,5\%). ηγέτιδα ευρωπαϊκή χώρα στην καινοτομία προϊόντος ή/και διαδικασίας είναι το βέλγιο με ποσοστό 62,1\% και ακολουθούν η πορτογαλία με ποσοστό 58,5\% και η φινλανδία με 58,2\%. τη δεκάδα κλείνει η εσθονία με ποσοστό καινοτομίας προϊόντος ή/και διαδικασίας 44,4\%. παράλληλα, με αύξηση 8,4 ποσοστιαίων μονάδων για την περίοδο 2014-2016 σε σχέση με την προηγούμενη περίοδο 2012-2014 και ποσοστό καινοτομίας οργάνωσης ή/και μάρκετινγκ 46,7\%, η ελλάδα βρίσκεται πλέον στην 6η θέση μεταξύ των ευρωπαϊκών κρατών και 10,1 ποσοστιαίες μονάδες υψηλότερα από τον ευρωπαϊκό μέσο όρο καινοτομίας οργάνωσης ή/και μάρκετινγκ (36,6\%). αξίζει να σημειωθεί ότι στην καινοτομία οργάνωσης ή/και μάρκετινγκ, ηγέτιδα ευρωπαϊκή χώρα είναι το λουξεμβούργο με ποσοστό 52,8\% και ακολουθούν η αυστρία με ποσοστό 51,8\% και η ιρλανδία με 49,5\%. τη δεκάδα κλείνει το ηνωμένο βασίλειο, με ποσοστό καινοτομίας οργάνωσης ή/και μάρκετινγκ 40,5\%. με βάση τα στοιχεία για τις ελληνικές επιχειρήσεις την περίοδο 2014-2016, στον τομέα της βιομηχανίας καινοτομεί το 59,5\% των επιχειρήσεων του τομέα, και στον τομέα των υπηρεσιών καινοτομεί το 56,5\% των επιχειρήσεων του τομέα. από τον τομέα της βιομηχανίας αξίζει να αναλυθεί περισσότερο ο κλάδος της μεταποίησης, καθώς αποτελεί και κορμό του συγκεκριμένου τομέα με πληθυσμό άνω των 4.000 επιχειρήσεων. το ποσοστό των καινοτόμων επιχειρήσεων στον κλάδο της μεταποίησης αγγίζει περίπου το 60\%, με αύξηση 4,4 ποσοστιαίων μονάδων (59,5\% την τριετία 2014-2016 και 55,1\% την τριετία 2012-2014). στους επιμέρους κλάδους της μεταποίησης διακρίνονται αυτοί της "παραγωγής χημικών" με ποσοστό καινοτόμων επιχειρήσεων 80,6\%, της "παραγωγής ποτών" με ποσοστό 72,5\%, της "παραγωγής βασικών φαρμακευτικών προϊόντων και φαρμακευτικών σκευασμάτων" με ποσοστό καινοτόμων επιχειρήσεων 69,2\% και της "κατασκευής πλαστικών" με ποσοστό 69,1\%. αντίστοιχα, στον τομέα των υπηρεσιών το μεγαλύτερο ποσοστό (62,7\%) καινοτόμων επιχειρήσεων καταγράφεται στον κλάδο της "ενημέρωσης και επικοινωνίας" στον οποίο διακρίνεται και αυτή την τριετία ο επιμέρους κλάδος των "δραστηριοτήτων προγραμματισμού η/υ, παροχής συμβουλών και συναφών δραστηριοτήτων" με ποσοστό 82,1\% καινοτόμων επιχειρήσεων. h έκδοση "βασικοί δείκτες για την καινοτομία στις ελληνικές επιχειρήσεις 2014-2016" διατίθεται στη διεύθυνσηhttp://metrics.ekt.gr/el/node/364, ενώ η πλήρης σειρά των δεικτών θα είναι σύντομα διαθέσιμη σε ειδική αναλυτική έκδοση του εκτ. σημειώνεται ότι τα στατιστικά στοιχεία και οι δείκτες για την έρευνα, ανάπτυξη και καινοτομία στην ελλάδα, τα οποία παράγονται και εκδίδονται από το εκτ, αποστέλλονται σε τακτική βάση στη eurostat και τον οοσα. η ανάλυση των στοιχείων και οι σχετικοί δείκτες δημοσιεύονται σε έντυπες και ηλεκτρονικές εκδόσεις του εκτ που διατίθενται στον δικτυακό τόπο http://metrics.ekt.gr. η έρευνα cis για την τριετία 2014-2016 χρηματοδοτήθηκε από το υποέργο 5 "παραγωγή δεικτών ris3 για τα έτη 2016-2023" που υλοποιείται από το εθνικό κέντρο τεκμηρίωσης και εντάσσεται στην πράξη "εγκατάσταση μηχανισμού παρακολούθησης (monitoring mechanism) της υλοποίησης της εθνικής στρατηγικής ris3-συλλογή και επεξεργασία δεικτών". h πράξη συγχρηματοδοτείται από την ελλάδα και την ευρωπαϊκή ένωση (ευρωπαϊκό ταμείο περιφερειακής ανάπτυξης) μέσω του επιχειρησιακού προγράμματος "ανταγωνιστικότητα, επιχειρηματικότητα, καινοτομία 2014-2020". & 799 & very low & Low & Socio-Economic & NA & NA & 2019-03-06 & 2019 & 3 & ECO
Frame & v.low & National & 500-1000 & 1.8924259 & 1.3590316 & -1.2391092 & 0.9814602 & -0.7922579 & 0.0 & -0.9049211 & -1.0736569 & Recipient & Domestic & Domestic & Domestic & Domestic|ECO & Positive\\
\addlinespace
Greece & http://www.avgi.gr/article/10951/8728355/alexes-charitses-oi-10-basikes-theseis-gia-to-mellon-tes-politikes-synoches & 323 & avgi.gr & Private/Non-Public & Online and Offline & National & very high = CP is most important issue + CP is mentioned in title/headline & Solidarity to poor countries/regions & Positive & EU + National & NA & Political leverage & Factual & EU + National & NA & Bureaucracy & Factual & EU + National & NA & Greece & αλέξης χαρίτσης: οι 10 βασικές θέσεις για το μέλλον της  πολιτικής συνοχής & 2018-02-23 & πολιτική συνοχής & "η πολιτική συνοχής αποτελεί καθοριστικής σημασίας παράγοντα στην εξέλιξη του ευρωπαϊκού οικοδομήματος και την εμβάθυνση στις ευρωπαϊκές αρχές για την ευημερία των πολιτών" να αναζητηθούν πρόσθετες ίδιες πηγές χρηματοδότησης του προϋπολογισμού της εε (π.χ. φόρος στις χρηματοπιστωτικές συναλλαγές) για να διατηρηθούν οι πόροι του προϋπολογισμού της εε στα ίδια επίπεδα, προτείνει η ελληνική κυβέρνηση, στο πλαίσιο της συζήτησης που έχει ξεκινήσει για το μέλλον της πολιτικής συνοχής μετά το 2020. η ελληνική κυβέρνηση, μέσω του αναπληρωτή υπουργού οικονομίας αλέξη χαρίτση, παρεμβαίνει δυναμικά στη συζήτηση και προτείνει 10 βασικές θέσεις. η πολιτική συνοχής αποτελεί το βασικό εργαλείο που διαθέτει η ευρωπαϊκή ένωση για τη χρηματοδότηση έργων υποδομών, την καταπολέμηση της ανεργίας και του κοινωνικού αποκλεισμού, τη στήριξη των επιχειρήσεων και γενικότερα, την προώθηση της σύγκλισης μεταξύ των κρατών μελών της. στην ελλάδα η πολιτική συνοχής υλοποιείται μέσα από τα προγράμματα του εσπα. στην εισήγησή του ο κ. χαρίτσης επισημαίνει ότι η ευρωπαϊκή ένωση βρίσκεται σε ένα κρίσιμο σταυροδρόμι. η νεοφιλελεύθερη διαχείριση της κρίσης σε συνδυασμό με τις αδυναμίες στην αρχιτεκτονική του κοινού νομίσματος έχουν παροξύνει τις ανισότητες μεταξύ των κρατών - μελών και έχουν πολλαπλασιάσει τις φυγόκεντρες τάσεις στο εσωτερικό της ένωσης. το brexit ήταν το πρώτο καμπανάκι κινδύνου, η ενίσχυση των ρατσιστικών και ξενοφοβικών δυνάμεων σε όλη την ευρώπη, το δεύτερο. για να απαντήσει στις αυξημένες προκλήσεις η ευρωπαϊκή ένωση, χρειάζεται να αναβαθμίσει την πολιτική συνοχής και να την πλαισιώσει με επαρκείς πόρους. τυχόν αποδυνάμωσή της, όπως διεκδικούν κάποιες φωνές με αφορμή το brexit, θα στερούσε από την ένωση τα μέσα με τα οποία διασφαλίζει τη συνοχή της και θα καθιστούσε το μέλλον της αβέβαιο. η πολιτική συνοχής, τονίζει ο κ. χαρίτσης, δεν πρέπει να τοποθετείται απέναντι στις πολιτικές για τη βελτίωση της ανταγωνιστικότητας. η συνοχή και η ανταγωνιστικότητα είναι στόχοι αλληλένδετοι. η ανταγωνιστικότητα των ευρωπαϊκών οικονομιών είναι όρος για την επίτευξη βιώσιμης συνοχής, ενώ η συνοχή αποτελεί προϋπόθεση για τη βελτίωση της ανταγωνιστικότητας καθώς επιτρέπει την καλύτερη και ουσιαστικότερη αξιοποίηση του πιο σημαντικότερου πλεονεκτήματος που διαθέτει η ευρώπη - των ανθρώπων της. για το υπουργείο οικονομίας, η πολιτική συνοχής αποτελεί καθοριστικής σημασίας παράγοντα στην εξέλιξη του ευρωπαϊκού οικοδομήματος και την εμβάθυνση στις ευρωπαϊκές αρχές για την ευημερία των πολιτών. παράλληλα, συμβάλλει στην προώθηση της ανάπτυξης με ολοκληρωμένο τρόπο, συνεκτιμώντας τις προκλήσεις από την παγκοσμιοποίηση, την κλιματική αλλαγή και τις δημογραφικές εξελίξεις. απαιτείται, λοιπόν, συνεκτικός σχεδιασμός με ξεκάθαρους στόχους και περιεχόμενο. ειδικότερα: 1 διατήρηση (αν όχι αύξηση) των πόρων του προϋπολογισμού της εε οι πολλαπλές προκλήσεις που αντιμετωπίζουν τα κράτη μέλη αλλά και η ευρωπαϊκή ένωση στο σύνολό της, καθιστούν απαραίτητη τη διατήρηση του προϋπολογισμού της ένωσης -αν όχι την αύξησή του. ειδικότερα, η πολιτική συνοχής αποτελεί στοιχείο ταυτότητας για την ευρωπαϊκή ένωση καθώς είναι το βασικό εργαλείο που υπηρετεί τη σύγκλιση μεταξύ των κρατών μελών. είναι λοιπόν απαραίτητη τουλάχιστον η διατήρηση του ύψους των κονδυλίων που διατίθενται από τον προϋπολογισμό της εε για την πολιτική συνοχής μετά το 2020 και η αποφυγή της αποδυνάμωσής της, παρά τους αυξανόμενους χρηματοδοτικούς περιορισμούς και τις συνέπειες του brexit. για να διατηρηθούν οι πόροι στα ίδια επίπεδα πρέπει να αναζητηθούν πρόσθετες ίδιες πηγές χρηματοδότησης του προϋπολογισμού της εε (πχ φόρος στις χρηματοπιστωτικές συναλλαγές). 2 "όχι" στη μείωση του προϋπολογισμού της συνοχής και της καπ είμαστε αντίθετοι σε ενδεχόμενη μείωση των κονδυλίων της πολιτικής συνοχής και της κοινής αγροτικής πολιτικής (καπ), υπέρ χρηματοδοτικών προγραμμάτων που διαχειρίζεται κεντρικά η ευρωπαϊκή επιτροπή, όπως αυτά του cef (connecting europe facility) και efsi (european fund for strategic investments - σχέδιο γιούνκερ). τα προγράμματα αυτά είναι προφανώς χρήσιμα και η χώρα μας κάνει ήδη τη βέλτιστη χρήση τους (πρώτοι, μεταξύ των χωρών μελών, στην αξιοποίηση και των δύο χρηματοδοτικών προγραμμάτων). ωστόσο αυτά πρέπει να λειτουργούν συμπληρωματικά στους πόρους της συνοχής (εσπα) ώστε να μεγιστοποιούνται οι δυνατότητες χρηματοδότησης. επιπλέον, για τις χώρες που παρουσιάζουν υστέρηση χρειάζεται να υπάρχει συγκεκριμένη και ποσοτικοποιημένη κατανομή των κονδυλίων του efsi. 3 επιμερισμένη διαχείριση των κονδυλίων η αρχή της επιμερισμένης διαχείρισης - η από κοινού δηλαδή και ισότιμη διαχείριση των ευρωπαϊκών κονδυλίων από τα κράτη μέλη, την τοπική αυτοδιοίκηση και την ευρωπαϊκή επιτροπή- είναι συστατικό στοιχείο της κοινής ευθύνης και της συμμετοχής στη διαδικασία ανάπτυξης, βάσει της οποίας η ευρωπαϊκή επιτροπή με τους εθνικούς και τοπικούς φορείς διαχειρίζονται τους κοινοτικούς πόρους. η διατήρηση της επιμερισμένης διαχείρισης υπογραμμίζει τη βούληση των κρατών μελών και των οργανισμών τοπικής αυτοδιοίκησης για ανάπτυξη και πρόοδο των κοινωνιών και μαζί με την αρχή της την πολυεπίπεδης διακυβέρνησης αποτελούν δομικά χαρακτηριστικά της πολιτικής συνοχής και πρέπει να παραμείνουν ακέραια και στο μέλλον. 4 αναλογική στήριξη όλων των περιφερειών η νέα αρχιτεκτονική της πολιτικής συνοχής μετά το 2020 πρέπει να στηρίζει αναλογικά όλες τις περιφέρειες της εε και να μην απευθύνεται μόνο στις πιο αδύναμες, καθώς ακόμη και στις πιο ανεπτυγμένες περιφέρειες εκδηλώνονται προβλήματα φτώχειας και κοινωνικού αποκλεισμού. προβλήματα που οξύνονται από τις πολιτικές λιτότητας. 5 πυξίδα οι κοινωνικοί δείκτες η κατανομή των πόρων της συνοχής για την περίοδο μετά το 2020, πρέπει να καθοριστεί, εκτός από το περιφερειακό αεπ, και με τη χρήση άλλων κοινωνικών ή χωρικών δεικτών, όπως το επίπεδο φτώχειας, υλικής στέρησης, ανεργίας και απασχόλησης (με μεγαλύτερη βαρύτητα από ό,τι μέχρι σήμερα), προσβασιμότητας σε υπηρεσίες, νησιωτικότητας, ορεινότητας κλπ, ώστε να αποτυπώνεται ακριβέστερα η πραγματική κατάσταση των περιφερειών (πχ. ενδοπεριφερειακές ανισότητες) και να αντιμετωπίζονται πιο αποτελεσματικά οι ανάγκες. 6 εργασία για τους νέους σε μια ευρώπη με ιδιαίτερα υψηλή ανεργία και με σημαντικό τμήμα του πληθυσμού να απειλείται με κοινωνικό αποκλεισμό, απαιτείται να δοθεί εκ νέου έμφαση στις πολιτικές για τη δημιουργία ποιοτικών και καλά αμειβόμενων θέσεων εργασίας και την προώθηση της κοινωνικής ένταξης. ειδικότερα για την καταπολέμηση της ανεργίας των νέων, πρέπει να ενισχυθεί η πρωτοβουλία για την απασχόληση των νέων και άλλες αντίστοιχες πρωτοβουλίες. 7 'αρση των ανισοτήτων h κατανομή των πόρων της πολιτικής συνοχής 2014-2020 σε συγκεκριμένους θεματικούς άξονες φαίνεται, κατ' αρχάς, να έχει θετικό αντίκτυπο στο σύνολο της εε. ωστόσο, απαιτείται ο επαναπροσδιορισμός αυτών των αξόνων βάσει μίας "προσαρμοσμένης στον τόπο" (place based) προσέγγισης, ώστε να υπηρετούνται πιο αποτελεσματικά οι στόχοι της κάλυψης υπαρκτών βασικών αναγκών που αντιμετωπίζουν οι κάτοικοι στις περιφέρειες της ε.ε και η άρση των ανισοτήτων. στόχοι οι οποίοι αποτελούν θεμελιώδεις αρχές της εε, σύμφωνα με την καταστατική συνθήκη της. 8 ο κανόνας ν+3 υποστηρίζουμε τη διατήρηση του κανόνα ν+3 - της αυτόματης δηλαδή, αποδέσμευσης των πόρων για επιπλέον τρία χρόνια μετά την τυπική λήξη της προγραμματικής περιόδου-καθώς σε διαφορετική περίπτωση, η ταχεία απορρόφηση των πόρων θα υπερισχύει δυστυχώς της φιλόδοξης αποτελεσματικότητας και θα παραμένει βασικό κριτήριο επιτυχίας των προγραμμάτων, σε βάρος της σκοπιμότητας των έργων. 9 ενιαίες αρχές και δίκαιη μεταχείριση η πολιτική συνοχής πρέπει να διέπεται από ενιαίες αρχές, κανόνες και διαδικασίες για όλα τα κράτη μέλη και όλες τις περιφέρειες. είμαστε αντίθετοι σε κάθε απόπειρα διαφοροποίησης στις διαδικασίες διαχείρισης και εφαρμογής των προγραμμάτων της πολιτικής συνοχής μετά το 2020, μεταξύ των κρατών - μελών, καθώς δεν είναι καθόλου προφανές ποια θα μπορούσαν να είναι τα διαφανή και αντικειμενικά κριτήρια για τη διασφάλιση της αναλογικής και δίκαιης μεταχείρισης. 10 απλοποίηση των κανονισμών είναι απολύτως απαραίτητη η απλοποίηση των διαδικασιών διαχείρισης των πόρων σε όλα τα επίπεδα. ειδικότερα για τα χρηματοδοτικά εργαλεία, χρειάζεται η απλοποίηση των κανονισμών και οι συνέργειες μεταξύ των κρατών μελών ώστε να κινητοποιούνται περισσότερες ιδιωτικές επενδύσεις και να μεγιστοποιείται η μόχλευση των πόρων τους. στο πλαίσιο αυτό θα πρέπει να δίνεται η ίδια ευελιξία για τη διαχείριση των κονδυλίων τόσο στα χρηματοδοτικά εργαλεία που διαχειρίζεται κεντρικά η ευρωπαϊκή επιτροπή, όσο και σε εκείνα που υλοποιούνται με την ευθύνη του κράτους μέλους. α α α email εκτυπωση κατηγορία οικονομικά νέα ροη κατηγοριας & 1244 & very high & High & Values & Power & Governance & 2018-02-23 & 2018 & 3 & ECO
Frame & high-very high & National & +1000 & 1.8924259 & 1.3590316 & -1.2391092 & 0.9814602 & -0.7922579 & 0.0 & -0.9049211 & -1.0736569 & Recipient & Domestic & European & Mixed & Domestic|ECO & Positive\\
Greece & http://www.rodiaki.gr/article/333251/protash-ethnikhs-diapragmateyshs-gia-ta-nhsia-toy-aigaioy-katethese-o-manos-konsolas-binteo & 391 & Rodiaki.gr & Private/Non-Public & Online and Offline & Regional/Local & medium = CP is important part of story & Solidarity to poor countries/regions & Negative & EU + National & No myth & NA & NA & NA & NA & NA & NA & NA & NA & Greece & πρόταση εθνικής διαπραγμάτευσης για τα νησιά του αιγαίου κατέθεσε ο μάνος κόνσολας (βίντεο) | η ροδιακη & 2016-03-05 & περιφερειακή πολιτική & στην εκδήλωση της βουλής των ελλήνων και της επιτροπής περιφερειών της βουλής για την επέτειο της ενσωμάτωσης της δωδεκανήσου, μίλησε ο βουλευτής δωδεκανήσου, κ. μάνος κόνσολας, καταθέτοντας μια ολοκληρωμένη πρόταση για τις ευρωπαϊκές και εθνικές πολιτικές για τη νησιωτικότητα. ο μάνος κόνσολας χαρακτήρισε την πρότασή του, ως ένα νέο πλαίσιο διεκδίκησης, όχι μόνο για τα δωδεκάνησα αλλά και για όλο το νησιωτικό χώρο. μίλησε για μια εθνική διαπραγμάτευση για την καθιέρωση και αναγνώριση ενός ειδικού καθεστώτος για τα νησιά του αιγαίου. δεν ζητάμε προνόμια, αλλά ένα πλαίσιο που θα εξισορροπεί τις ανισότητες που αντιμετωπίζουν οι νησιωτικές περιοχές. πολύ απλά θέλουμε ένα πλαίσιο που θα μας δίνει τις ίδιες ευκαιρίες αλλά και τις ίδιες δυνατότητες ανάπτυξης με αυτές που υπάρχουν στην ηπειρωτική χώρα και σε άλλες ευρωπαϊκές περιφέρειες. έκανε λόγο για ένα τεράστιο έλλειμμα οριζόντιων πολιτικών για τα νησιά μας σε εθνικό αλλά και ευρωπαϊκό επίπεδο και τόνισε ότι περιφερειακή πολιτική της ευρωπαϊκής ένωσης, δεν συνιστά πολιτική για τις νησιωτικές περιφέρειες, αφού δεν αντιμετωπίζει προβλήματα που συνδέονται με τα εντελώς ιδιαίτερα χαρακτηριστικά αυτών των περιφερειών. ο κ. κόνσολας επισήμανε ότι: "η αδυναμία σχεδιασμού και υλοποίησης εθνικών και ευρωπαϊκών πολιτικών για τα νησιά μας, επέτεινε τις ανισότητες. ποιες είναι αυτές οι ανισότητες: - είναι το υψηλό κόστος μεταφορών που δημιούργησε προβλήματα ανταγωνιστικότητας, με συνέπειες στην πραγματική οικονομία . - είναι η έντονη περιφερειακότητα, που συνίσταται σε διαφορές στους δείκτες ανάπτυξης μικρών και μεγάλων νησιών. - είναι οι κλειστές, μικρής κλίμακας και εύθραυστες, σχετικά, τοπικές οικονομίες. σε όλα αυτά ήρθαν να προστεθούν νέα προβλήματα. αναφέρομαι σε κάποια από αυτά: - είναι η κατάργηση των μειωμένων συντελεστών φπα, που κάποιοι θεωρούν προνόμιο αλλά δεν είναι. - είναι η ένταξη της περιφέρειας νοτίου αιγαίου στο στόχο 2, που την κατατάσσει στις ανεπτυγμένες περιφέρειες με βάση στοιχεία προ δεκαετίας. έκτοτε όμως ο δείκτης μείωσης του αεπ στην περιφέρεια ν. αιγαίου είναι ο μεγαλύτερος συγκριτικά. - είναι η τεράστια επιβάρυνση που δέχονται τα νησιά μας από τις ανεξέλεγκτες μεταναστευτικές ροές με αρνητικές συνέπειες στην τοπική οικονομία και στον τουρισμό". ο βουλευτής δωδεκανήσου ανέφερε ότι υπάρχουν 21 νησιωτικές περιφέρειες στην ευρώπη εκ των οποίων οι 13 διέπονται από ένα ειδικό καθεστώς. οι 7 νησιωτικές περιφέρειες έχουν ένα ειδικό καθεστώς αυτονομίας με ειδικές φορολογικές και διοικητικές αρμοδιότητες, παράδειγμα τα νησιά φερόες που ανήκουν στη δανία. οι 6 νησιωτικές περιφέρειες έχουν κάποιο ιδιαίτερο καθεστώς διαχείρισης και φορολογικής αντιμετώπισης. ο μάνος κόνσολας περιέγραψε, τους τρεις άξονες μιας νέας ευρωπαϊκής πολιτικής για τα νησιά μας ως εξής: 1. η διαμόρφωση μιας αμιγούς και αυτόνομης νησιωτικής πολιτικής από την πλευρά της ευρωπαϊκής ένωσης. σήμερα οι δράσεις της ευρωπαϊκής ένωσης που αφορούν στα νησιά μας, υλοποιούνται μέσα από την περιφερειακή πολιτική. χρειαζόμαστε, όμως, πολιτικές και δράσεις που θα απευθύνονται αποκλειστικά στις νησιωτικές περιφέρειες. 2. η δημιουργία ταμείου νησιωτικότητας, που θα αποτελεί εργαλείο άσκησης και ανάπτυξης αυτών των πολιτικών. 3. ο επαναπροσδιορισμός των κριτηρίων για την ένταξη των νησιωτικών περιοχών στο στόχο 1 ή στο στόχο 2. ασφαλώς και έχουν ισχύ τα επιχειρήματα που θέτει η περιφέρεια νοτίου αιγαίου, χρειαζόμαστε όμως μια θεσμική διεύρυνση των κριτηρίων που κατατάσσουν μια περιφέρεια στις λιγότερο ή περισσότερο ανεπτυγμένες. δεν μπορεί να είναι μόνο το αεπ, πρέπει να συνυπολογίζονται οι ποσοτικοί και ποιοτικοί δείκτες που δίνουν την πραγματική εικόνα. το νότιο αιγαίο είναι η κατ' εξοχήν νησιωτική περιφέρεια της ευρώπης, αποτυπώνει την πολυδιάσπαση και τη γεωγραφική ασυνέχεια με 79 νησιά, εκ των οποίων τα 48 είναι κατοικημένα και με ορατές και έντονες ενδοπεριφερειακές ανισότητες που δημιουργούν, δυστυχώς, το φαινόμενο της διπλής νησιωτικότητας. η πολιτική, όμως, για την νησιωτικότητα εντάσσεται και στο πλαίσιο ενός εθνικού σχεδιασμού για τη βελτίωση υποδομών, την οικονομική ανάπτυξη και την αύξηση της απασχόλησης. είναι καιρός να αποκτήσουμε μια εθνική νησιωτική πολιτική με ουσιαστικό περιεχόμενο σε όλους τους τομείς της ανθρώπινης δραστηριότητας". παράλληλα, ο βουλευτής δωδεκανήσου έθεσε και τις βασικές αρχές μιας εθνικής πολιτικής για τη νησιωτικότητα ως εξής: "- η ρήτρα νησιωτικότητας πρέπει να έχει οριζόντια και απόλυτη εφαρμογή σε κάθε νομοσχέδιο, σε κάθε διοικητική πράξη. παρά το γεγονός ότι η ρήτρα νησιωτικότητας αποτελεί νόμο του κράτους, δεν υπάρχει εφαρμογή της. αυτό πρέπει να αλλάξει. ο κ. πρόεδρος της βουλής είναι εδώ. θα του προτείνω να εντάξει τη ρήτρα νησιωτικότητας στον κανονισμό που διέπει τη νομοθετική διαδικασία και τη λειτουργία του κοινοβουλίου. σε κάθε νομοσχέδιο, να υπάρχει η έκθεση επιπτώσεων στις νησιωτικές περιοχές που προκύπτει από τη ρήτρα νησιωτικότητας, όπως υπάρχει και η αντίστοιχη έκθεση του γενικού λογιστηρίου. - δημιουργία αυτόνομου υπουργείου νησιωτικότητας με δυνατότητα άσκησης οριζόντιων πολιτικών και δράσεων. - αναπτυξιακός νόμος για τα νησιά με ποσοτικά και ποιοτικά κριτήρια. προσανατολισμένος στην καινοτομία, στη στήριξη της νεανικής επιχειρηματικότητας και στη διεύρυνση του παραγωγικού ιστού. - αναβάθμιση της διοικητικής εξυπηρέτησης των πολιτών στα νησιά και μείωση του διοικητικού κόστους. πως; με την ανάπτυξη τεχνολογιών πληροφορικής και επικοινωνίας, με τη δημιουργία one stop shop σε κάθε νησί, στα οποία ο πολίτης θα κάνει όλες τις συναλλαγές του με το κράτος, διοικητικού και οικονομικού χαρακτήρα. - διαμόρφωση ολοκληρωμένου ενεργειακού σχεδιασμού για τα νησιά μας. η διασύνδεση της κρήτης και ορισμένων νησιών των κυκλάδων με το ηπειρωτικό δίκτυο προχωρά. εθνικοί λόγοι επιτάσσουν τη διασύνδεση και της δωδεκανήσου παράλληλα με την ανάπτυξη ανανεώσιμων πηγών ενέργειας και έξυπνων ενεργειακών συστημάτων, προκειμένου να διασφαλιστεί ενεργειακή επάρκεια. - εφαρμογή σχεδίου διαχείρισης των υδατικών πόρων και γενικότερα, εφαρμογή μέτρων πολιτικής που να εξασφαλίζουν τη βιώσιμη ανάπτυξη και την πληθυσμιακή αύξηση στα νησιά σε συνθήκες κοινωνικής και ατομικής ευημερίας. το 2016 ακολουθούμε, ακόμα, το παρωχημένο και κοστοβόρο μοντέλο μεταφοράς νερού στα νησιά αντί να αξιοποιήσουμε τους φυσικούς πόρους αλλά και τις δυνατότητες της τεχνολογίας". ο μάνος κόνσολας έκλεισε την ομιλία του, αναφερόμενος στην πίεση που δέχονται τα νησιά μας από την αύξηση των μεταναστευτικών ροών, επισημαίνοντας: "κατέθεσα πρόταση στο υπουργείο οικονομικών για να ανασταλεί η αύξηση των συντελεστών φπα στη δεύτερη ομάδα των νησιών που είναι προγραμματισμένη να ισχύσει από την 1η ιουνίου. εκτιμώ ότι η πρόταση αυτή εκφράζει την κοινή λογική, είναι μια αυτονόητη κίνηση που οφείλει να κάνει η κυβέρνηση, τη στιγμή, μάλιστα, που το δημοσιονομικό όφελος από την κατάργηση των μειωμένων συντελεστών φπα αμφισβητείται. το γραφείο προϋπολογισμού της βουλής αλλά και το ινστιτούτο φορολογικών μελετών το αποδεικνύουν και το επιβεβαιώνουν. αν θέλουμε να στείλουμε ένα μήνυμα ενότητας, ένα μήνυμα σε όλους τους πολίτες της δωδεκανήσου, ας ενώσουμε τις δυνάμεις μας σε αυτή την προσπάθεια". & 1020 & medium & Medium & Values & NA & NA & 2016-03-05 & 2016 & 2 & ECO
Frame & low-medium & Regional & +1000 & 1.8924259 & 1.3590316 & -1.2391092 & 0.9814602 & -0.7922579 & 0.0 & -0.9049211 & -1.0736569 & Recipient & Domestic & European & Mixed & Domestic|ECO & Negative\\
Greece & http://news247.gr/eidiseis/politiki/ependytiko-safari-pappa-synanthseis-me-funds-sto-londino.3892187.html & 314 & news247.gr & Private/Non-Public & Online only & National & low = CP mentioned more times but NOT important part of story (mainly about others issues) & Improve governance & Positive & National & No myth & Bureaucracy and/or delays & Positive & EU + National & No myth & NA & NA & NA & NA & Greece & επενδυτικό σαφάρι παππά. συναντήσεις με funds στο λονδίνο & 2016-02-04 & ευρωπαϊκό κοινωνικό ταμείο & σειρά συναντήσεων του ν. παππά με επενδυτικά σχήματα στο λονδίνο. τους εξέθεσε τα δεδομένα ανάκαμψης της οικονομίας και τις συνθήκες που διαμορφώνουν ευνοϊκό επενδυτικό περιβάλλον στην ελλάδα σειρά συναντήσεων με επενδυτικά σχήματα είχε ο υπουργός επικρατείας νίκος παππάς, στο περιθώριο της συμμετοχής του πρωθυπουργού στη διεθνή διάσκεψη δωρητών για τη συρία στο λονδίνο. κατά τις συναντήσεις ο υπουργός επικρατείας εξέθεσε τα δεδομένα ανάκαμψης της ελληνικής οικονομίας, καθώς και τις συνθήκες που διαμορφώνουν ένα ευνοϊκό επενδυτικό περιβάλλον στην ελλάδα. ειδικότερα, σύμφωνα με κυβερνητικούς κύκλους, ο κ. παππάς συνάντησε εκπροσώπους της blackrock, του credit suisse group, της wellington management company και της oceanwood capital και τους ενημέρωσε για τα θετικά μακροοικονομικά στοιχεία που παρουσιάζει η ελληνική οικονομία: * την ανθεκτικότητα της ελληνικής οικονομίας στις αναταράξεις του 2015 με το αεπ να κλείνει το 2015 με ύφεση σημαντικά μικρότερη (-0,7\%) από αυτή που υπολογιζόταν * την υπερκάλυψη του στόχου των δημοσίων εσόδων κατά 2 δισ. ευρώ περίπου και αντίστοιχα του στόχου πρωτογενούς πλεονάσματος το οποίο διαμορφώνεται στο 0,4\% του αεπ, υψηλότερο από το στόχο του 0,25\% του αεπ * τη μείωση της ανεργίας από το 26\% στο 24,4\% με τη δημιουργία 87.000 θέσεων εργασίας * την μεγαλύτερη στην ευρωζώνη άνοδο του δείκτη βιομηχανικής παραγωγής κατά 3,3\%, την αύξηση της αξίας των εξαγωγών (χωρίς πετρελαϊκά προϊόντα) κατά 16,6 δισ. ευρώ, στο ανώτατο επίπεδο από της είσοδο της ελλάδας στην ονε με την οποία το έλλειμμα τρεχουσών συναλλαγών περιορίζεται δραστικά στο 0,47\% του αεπ το 2015 (2,12\% το 2014, 12,37\% το 2009) επιπλέον, ο υπουργός επικρατείας ενημέρωσε σχετικά με το δυναμικό επενδυτικό περιβάλλον όπως διαμορφώνεται στην ελλάδα με: * με την αύξηση του πδε κατά 5,5\% στα 6,75 δισ. ευρώ και το άλμα της χώρας μας από την 6η στην 1η θέση ως προς την απορρόφηση ευρωπαϊκών διαρθρωτικών πόρων, με απορρόφηση του 97\% των διαθέσιμων πόρων του 207-2013 και 5 δισ. ευρώ μόνο τους τελευταίους 4 μήνες. επιπλέον, είναι έτοιμο το σύστημα διαχείρισης και ελέγχου για το ευρωπαϊκό κοινωνικό ταμείο 2014-2020 και η ροή πληρωμών έχει ήδη ξεκινήσει στην ελλάδα, το πρώτο κράτος-μέλος που το επιτυγχάνει. * την αναβάθμιση της πιστοληπτικής ικανότητας της χώρας από την standard \& poor's και τη δρομολόγηση της επιστροφής σε κανονικό δανεισμό από την εκτ. * την ανάσχεση του αποπληθωρισμού μετά από 34 μήνες και την αναζωογόνηση της εσωτερικής ζήτησης. * την άνοδο του δείκτη pmi στον μεταποιητικό τομέα πάνω από το όριο των 50 μονάδων, δηλ. την είσοδο του τομέα σε φάση επέκτασης. * τη διευκόλυνση σδιτ, νέα σχήματα χρηματοδότησης που θα ενισχύσουν τη ρευστότητα κατά 1,25 δισ. ευρώ, και το νέο αναπτυξιακό νόμο που θα ψηφιστεί σύντομα από τη βουλή και θα ενισχύει τη νέα επιχειρηματικότητα, τομείς υψηλής προστιθέμενης αξίας και έντασης γνώσης, θα δημιουργεί ελκυστικό πλαίσιο για ξένες άμεσες επενδύσεις, σταθερό φορολογικό πλαίσιο για 7 έτη, και θα παρέχει επενδυτικά κίνητρα. * την αλλαγή του θεσμικού πλαισίου για τη διαχείριση του ιδιωτικού χρέους * τα μέτρα βελτίωσης της ενεργειακής αποδοτικότητας και διαχείρισης ενέργειας * την επιτάχυνση της διαδικασίας αδειοδότησης επενδύσεων και την απελευθέρωση επαγγελμάτων και αγορών με ενίσχυση του ανταγωνισμού σχετικά με την ανάδειξη της ελλάδας σε ενεργειακό κόμβο μεταξύ μεσογείου και βαλκανίων, αναφέρθηκαν η υλοποίηση του αγωγού tap, της μεγαλύτερης άμεσης ξένης επένδυσης στα βαλκάνια, η ιδιωτικοποίηση της δεσφα ως κλειδί για την είσοδο κεφαλαίων και ανάπτυξη υποδομών, οι τριμερείς ενεργειακές συνεργασίες με κύπρο-ισραήλ και με κύπρο-αίγυπτο για αξιοποίηση των κοιτασμάτων της αν. μεσογείου, η μεταφορά και παροχή υγροποιημένου φυσικού αερίου στη να ευρώπη, καθώς και τα σχέδια για ενίσχυση της εγκατάστασης στη ρεβυθούσσα και για μια πλωτή εγκατάσταση στη β. ελλάδα. επιπλέον, παρασχέθηκε ενημέρωση σχετικά με την αξιοποίηση δεδομένων για την καταπολέμηση διαφθοράς και φοροδιαφυγής, για τις ταχείες διερευνήσεις των λιστών καταθετών του εξωτερικού, για τη σημαντική βελτίωση της θέσης της χώρας ως προς την καταπολέμηση της διαφθοράς. σύμφωνα με τους κυβερνητικούς κύκλους, σε σχέση με την ανακεφαλαιοποίηση των τραπεζών επισημάνθηκε η επιτυχία με ποσό πολύ χαμηλότερο από αυτό που είχε αρχικά εκτιμηθεί και η άντληση κεφαλαίων από ιδιώτες επενδυτές, που σηματοδοτεί την εμπιστοσύνη στο ελληνικό τραπεζικό σύστημα, το οποίο είναι έτοιμο να αναλάβει το ρόλο του για την χρηματοδότηση της ανάκαμψης της οικονομίας. αναφορικά με την ασφαλιστική μεταρρύθμιση, εξηγήθηκαν οι αιτίες που δημιούργησαν την αναγκαιότητα παρέμβασης, όπως η υψηλή ανεργία και η κακοδιαχείριση του παρελθόντος, και αναλύθηκαν οι σχεδιαζόμενες παρεμβάσεις: ενιαίο ταμείο, κοινοί κανόνες υπολογισμού εισφορών και παροχών, η παροχή εθνικής σύνταξης, ενώ έγινε σαφής η αναγκαιότητα αποφυγής νέας περικοπής των συντάξεων. τέλος, εκτέθηκαν τα δεδομένα σχετικά με την πορεία του ελληνικού προγράμματος και την ολοκλήρωση της πρώτης αξιολόγησης το συντομότερο δυνατό ως κρίσιμου ορόσημου για την πορεία της ελληνικής οικονομίας και την ευρύτερη σταθερότητα. οι ίδιες πηγές υπογραμμίζουν ότι στόχος της κυβέρνησης παραμένει η αποτελεσματική αντιμετώπιση των παθογενειών του παρελθόντος, η ανάκαμψη της οικονομίας σε περιβάλλον πολιτικής σταθερότητας με ευρεία στήριξη του προγράμματος προσαρμογής και αναγκαίων μεταρρυθμίσεων μακράς πνοής που θα απελευθερώσουν τις παραγωγικές δυνάμεις της χώρας. & 810 & low & Low & Governance & Governance & NA & 2016-02-04 & 2016 & 2 & POL
Frame & low-medium & National & 500-1000 & 1.8924259 & 1.3590316 & -1.2391092 & 0.9814602 & -0.7922579 & 0.0 & -0.9049211 & -1.0736569 & Recipient & Domestic & Domestic & Domestic & Domestic|POL & Positive\\
Greece & http://www.dikaiologitika.gr/eidhseis/oikonomia/111051/esee-gsevee-to-etean-na-anavathmistei-se-anaptyksiako-tameio-gia-tous-mikromesaious & 326 & dikaiologitika.gr & Private/Non-Public & Online only & National & high = CP is most important issue in story (can also cover other issues) & Economic development & Positive & EU + Subnational & No myth & Poor communication of funding/rules & Negative & No actor & No myth & NA & NA & NA & NA & Greece & εσεε- γσεβεε: το ετεαν να αναβαθμιστεί σε αναπτυξιακό ταμείο για τους μικρομεσαίους & 2016-06-13 & διαρθρωτικά ταμεία & η αναβάθμιση του ετεαν αε σε αναπτυξιακό ταμείο ήταν το θέμα της κοινής συνέντευξης τύπου που πραγματοποιήθηκε σήμερα στα γραφεία της εσεε από τον πρόεδρο της εσεε βασίλη κορκίδη, τον πρόεδρο της γσεβεε γιώργο καββαθά και τον πρόεδρο της ετεan αε κώστα γαλιάτσο, στο πλαίσιο της εκστρατείας ενημέρωσης του επιχειρηματικού κόσμου για τα υφιστάμενα χρηματοδοτικά εργαλεία του ετεαν αλλά και για τις δυνατότητες πρόσβασής τους σε εναλλακτικές πηγές ρευστότητας. επιπλέον, παρουσιάστηκε η δυνατότητα αναβάθμισης σε αναπτυξιακό ταμείο μικρομεσαίας επιχειρηματικότητας (ατμε) σύμφωνα με την πάγια κοινή πρόταση και παλαιότερη πρωτοβουλία της εσεε και της γσεβεε. η εσεε πριν ένα περίπου μήνα ενημέρωσε πανελλαδικά για τον υποστηρικτικό ρόλο της ετεαν αε στη μικρομεσαία ελληνική επιχειρηματικότητα με 3 ενεργά προγράμματα χρηματοδότησης μετά από συνάντηση που πραγματοποιήθηκε μεταξύ του προέδρου της εσεε, κ. βασίλη κορκίδη και του προέδρου της ετεαν αε κ. κώστα γαλιάτσου, όπου συζητήθηκαν τρόποι ευρύτερης αξιοποίησης του ταμείου και ανάδειξής του ως κύριου μέσου ενίσχυσης της μικρομεσαίας επιχειρηματικότητας. οι δύο πρόεδροι συμφώνησαν για την ανάγκη μετασχηματισμού του ετεαν σε ένα εργαλείο που θα έχει ως αποκλειστικό στόχο την χορήγηση πιστώσεων αλλά και τη μεταφορά τεχνογνωσίας στις επιχειρήσεις μικρού και μεσαίου μεγέθους, που αποτελούν τη συντριπτική πλειοψηφία της εγχώριας οικονομίας. ιδιαίτερη έμφαση δόθηκε στην παράταση έως και τον προσεχή σεπτέμβριο που πήραν τα χρηματοδοτικά προγράμματα της ετεαν αε, που προέρχονταν από τα διαρθρωτικά ταμεία της ευρωπαϊκής ένωσης και αφορούσαν την προγραμματική περίοδο 2007-2013. έγινε, μάλιστα, γνωστό πως από αρχή του 2016 μέχρι σήμερα έχουν διατεθεί 30 εκ. ευρώ δάνεια, αλλά αυτά που μπορεί να διαθέσει το ταμείο ανέρχονται σε περίπου 60 εκ. ευρώ, ενώ συμπεριλαμβανομένης της ισόποσης συμμετοχής των τραπεζών, τα συνολικά προς διάθεση κεφάλαια εκτιμώνται σε 120 εκ. ευρώ. επισημαίνεται πως επειδή οι συγκεκριμένοι διαθέσιμοι πόροι έχουν πεδίο αναφοράς την προηγούμενη προγραμματική περίοδο, ελλοχεύει ο κίνδυνος απώλειάς τους, εάν δεν απορροφηθούν μέχρι τις 15/7/16. ένα σημαντικό κίνητρο προσέλκυσης των ενδιαφερομένων αποτυπώνεται στο ιδιαίτερα ευνοϊκό επιτόκιο, το οποίο σε μεσοσταθμική βάση κυμαίνεται περίπου στο 4,0\%. η εσεε και η γσεβεε απευθυνόμενες στις πολύ μικρές, μικρές και μεσαίες εμπορικές επιχειρήσεις που έχουν υποστεί την τελευταία εξαετία δραματική μείωση του διαθέσιμου εισοδήματός τους, παραθέτουν εκ νέου, αναλυτικά τα ενεργά προγράμματα και τις διαθέσιμες δράσεις του τεπιχ της ετεαν αε καθώς και τα απαιτούμενα 6 βήματα στη διαδικασία χρηματοδότησής τους. το ετεαν, με την βοήθεια των δύο οργανώσεων θα επιδιώξει την αναβάθμιση των υπηρεσιών που προσφέρει προς τις μικρομεσαίες επιχειρήσεις, όπως: - διεύρυνση του δανεισμού με την χορήγηση μικροπιστώσεων μέχρι 10.000 €. - πρόβλεψη όσο το δυνατόν περισσότερων και ευέλικτων εγγυοδοτικών προϊόντων. - διεύρυνση της συμμετοχής στο κεφάλαιο των επιχειρήσεων, όχι μόνο για τα "start ups", τα οποία εν πολλοίς δεν υπάρχουν στην ελληνική αγορά αλλά και για τα "follow ups" και τα "restarts". οι πηγές του νέου αναπτυξιακού φορέα θα προέρχονται από τον συνδυασμό των διαθέσιμων πόρων του εσπα, του προγράμματος δημοσίων επενδύσεων και των διαρθρωτικών ταμείων. πηγές για το ταμείο μεταξύ άλλων, μπορούν να είναι κεφάλαια από ευρωπαϊκούς και διεθνείς χρηματοπιστωτικούς θεσμούς και ιδρύματα, όπως είναι το ευρωπαϊκό ταμείο επενδύσεων της ετεπ, το ebrd κλπ., γεγονός που θα συντελέσει στην όσο το δυνατόν μεγαλύτερη μόχλευση των αρχικά διαθέσιμων πόρων. τα προβλήματα που αντιμετωπίζουν την δεδομένη χρονική στιγμή οι επιχειρήσεις ως προς την πρόσβασή τους στις πηγές ρευστότητας είναι: - η ελλιπής πληροφόρηση - η ασυμμετρία προσφοράς και ζήτησης κεφαλαίων - ο κατακερματισμός των επιχειρηματικών κεφαλαίων - οι καταγεγραμμένες αποκλίσεις των επιτοκίων χορηγήσεων - οι αυστηροί όροι χορήγησης πιστώσεων, ιδίως σε χώρες που αντιμετωπίζουν δημοσιονομικά προβλήματα - οι ζητούμενες υπέρμετρες προσωπικές εγγυήσεις ως προϋπόθεση χρηματοδότησης. σε ότι αφορά τη διαθεσιμότητα δανείων στην ευρωπαϊκή ένωση, αυτή είναι μεγαλύτερη των αναγκών των μμε επιχειρήσεων, με εξαίρεση την ελλάδα. σύμφωνα με την τελευταία διαθέσιμη έκθεση του ευρωπαϊκού κοινοβουλίου που συνέκρινε τα εμπόδια για τη χρηματοδότηση μεταξύ των λεγόμενων "παλαιών χωρών" (ε.ε.-15) και των "νέων" (ε.ε.-13), αποκαλύπτει ότι η ευκολία με την οποία οι μμε μπορούν να αποκτήσουν εξωτερικούς πόρους, εξαρτάται σε μεγάλο βαθμό από το μακροοικονομικό πλαίσιο, την κατάσταση και τη δομή του τραπεζικού τομέα. η πρόσβαση σε εξωτερικά κεφάλαια τείνει να είναι πιο εύκολη για τις μμε που βρίσκονται σε κράτη μέλη με τα υψηλότερα επίπεδα ανάπτυξης των ενδιάμεσων χρηματοπιστωτικών οργανισμών, πιο προηγμένες χρηματιστηριακές αγορές, πιο αποτελεσματικά νομικά συστήματα ή υψηλότερο κατά κεφαλήν αεπ. επιπλέον, το επίπεδο της χρηματοπιστωτικής διαμεσολάβησης είναι χαμηλότερο στις παλαιές χώρες, σε σχέση με τις νέες: ο δανεισμός του ιδιωτικού τομέα ως ποσοστό του αεπ διαμορφώθηκε (εκτός από κύπρο και μάλτα) σε 64\% στις νέες, έναντι 148\% στις παλαιές. τα μη εξυπηρετούμενα δάνεια ήταν στο 11\% στις νέες, σε σύγκριση με το 7\% στις παλαιές και αντιμετωπίζουν μεγάλα εμπόδια, εάν βρίσκονται σε ταχύτερα αναπτυσσόμενες οικονομίες ή σε χώρες με πιο υγιή τραπεζικό τομέα. στις νέες χώρες, περίπου, το 18\% των μμε λαμβάνουν τραπεζικά δάνεια και το 14\% εμπορικές πιστώσεις, ενώ στις παλαιές, περίπου 23\% των μμε έχουν τραπεζικό δανεισμό και 20\% εμπορικές πιστώσεις. τέλος, οι πρόεδροι παρουσίασαν την πρότασή τους για τον διευρυμένο ρόλο του ετεαν και των οργανώσεων, για τον χαρακτήρα της συνδιαχείρισης που πρέπει να έχουν οι πηγές του ταμείου και την αισιοδοξία τους ότι, μετά από τις αναγκαίες επαφές με το υπουργείο οικονομικών, η πρόταση θα νομοθετηθεί σύντομα. εξέφρασαν δε την ελπίδα ότι το φετινό καλοκαίρι δεν θα είναι σαν το περυσινό και κατά την διάρκειά του, η κρίση ρευστότητας θα απομακρυνθεί. ο πρόεδρος της ετεαν αε κωνσταντίνος γαλιάτσος δήλωσε: "στο πλαίσιο, μάλιστα, της εντατικοποίησης της συνεργασίας του ετεαν με φορείς εκπροσώπησης του επιχειρηματικού κόσμου, όπως η εσεε και η γσεβεε, ο κ. γαλιάτσος απηύθυνε κάλεσμα στους επιχειρηματίες προκειμένου να προσέλθουν στα πιστωτικά ιδρύματα και να εκδηλώσουν έμπρακτο ενδιαφέρον για διαθέσιμα δάνεια, είτε για κεφάλαια κίνησης είτε για επενδυτικούς σκοπούς". & 930 & high & High & Socio-Economic & Governance & NA & 2016-06-13 & 2016 & 2 & ECO
Frame & high-very high & National & 500-1000 & 1.8924259 & 1.3590316 & -1.2391092 & 0.9814602 & -0.7922579 & 0.0 & -0.9049211 & -1.0736569 & Recipient & Domestic & European & Mixed & Domestic|ECO & Positive\\
Greece & http://www.newsbeast.gr/politiki/arthro/2823310/borisof-ellada-ke-voulgaria-tha-diadramatisoun-simantiko-rolo-sti-valkaniki & 340 & Newsbeast.gr & Private/Non-Public & Online only & National & very low = CP mentioned once & Territorial cooperation & Positive & EU + National + Subnational & No myth & Infrastructure & Positive & National + Other country & No myth & NA & NA & NA & NA & Greece & μπορίσοφ: ελλάδα και βουλγαρία θα διαδραματίσουν σημαντικό ρόλο στη βαλκανική & 2017-09-06 & διαρθρωτικά ταμεία & "με τη σεμνότητα που μας διακρίνει ξεκινάμε με πιο μικρά πράγματα" την πεποίθησή του ότι η ελλάδα και η βουλγαρία μπορούν να διαδραματίσουν σημαντικό ρόλο στη βαλκανική, σηματοδοτώντας την ανάπτυξη και τη σταθερότητα της περιοχής, εξέφρασε από την καβάλα ο πρωθυπουργός της γειτονικής χώρας μπόικο μπορίσοφ. ο κ. μπορίσοφ στις δηλώσεις του μετά την υπογραφή του μνημονίου συνεργασίας μεταξύ των δυο χωρών για την κατασκευή της νέας σιδηροδρομικής σύνδεσης που θα συνδέει συνολικά έξι μεγάλα εμπορικά λιμάνια της ελλάδας και της βουλγαρίας, υπογράμμισε ότι η υλοποίηση του έργου για το οποίο υπογράφηκε η συμφωνία, αποτελεί κοινό έργο υποδομής και των δυο χωρών ενισχύοντας σημαντικά της επιμέρους υποδομές τους. ο βούλγαρος πρωθυπουργός ευχαρίστησε όλους όσοι συνέβαλαν ώστε να προχωρήσει αυτό το σημαντικό έργο, ενώ ευχαρίστησε και προσωπικά τον έλληνα πρωθυπουργό γιατί, όπως ανέφερε, συνέβαλε ώστε να βγει η χώρα από την οικονομική κρίση. "η υπογραφή του μνημονίου", τόνισε ο κ. μπορίσοφ, "αποτελεί μια συνέχεια της ανάπτυξης των ενεργειακών και μεταφορικών υποδομών, μια συνέχεια της διασύνδεσης των περιοχών με τη διακίνηση των εμπορευμάτων, των ανθρώπων και της ενίσχυσης του τουριστικού τομέα". αναφερόμενος στο νέο υπό κατασκευή έργο επισήμανε ότι αφορά μεν τις δυο χώρες ωστόσο σημείωσε ότι "στην επικείμενη τετραμερή συνάντηση που θα γίνει στη βάρνα θα ενσωματώσουμε και τους ρουμάνους και τους σέρβους ομολόγους μας επειδή το έργο αφορά και τις μεταφορές στον δούναβη και τη στρατηγική του δουνάβεως [...] οπότε είναι πολύ σημαντικό να εντάξουμε και τις χώρες αυτές. μας παρουσιάζεται έτσι και μια ευκαιρία να δείξουμε στους ευρωπαίους εταίρους μας πόσο σημαντικά περιφερειακά έργα αναπτύσσονται στην περιοχή μας. αυτό είναι σημαντικό και όσον αφορά τα διαρθρωτικά ταμεία και τη στάση μας απέναντι στην ευρωπαϊκή τράπεζα όσον αφορά την εξεύρεση χρηματοδότησης, έτσι ώστε να πείσουμε τους ευρωπαίους εταίρους μας ότι παρουσιάζοντας σήμερα ένα έγγραφο ουσιαστικά στοχεύουμε στο να γίνουν έργα υποδομής και σημαντικές εγκαταστάσεις". ο κ. μπορίσοφ ανακοίνωσε επίσης ότι έπειτα από συνεννόηση με τον αλέξη τσίπρα οι δυο πρωθυπουργοί θα ζητήσουν τον οκτώβριο μια συνάντηση με τον ζαν κλοντ γιούνκερ για να του παρουσιάσουν τις κοινές ιδέες και προτάσεις τους για την περαιτέρω ανάπτυξη της περιοχής των βαλκανίων. η υπογραφή της συμφωνίας για τη νέα σιδηροδρομική σύνδεση υπογράφηκε στο ιστορικό ξενοδοχείο "ιμαρέτ", στην παλιά πόλη της καβάλας, στη χερσόνησο της παναγίας, και την εκδήλωση παρακολουθήσαν πολλοί εκπρόσωποι των τοπικών κοινωνιών της περιφέρειας ανατολικής μακεδονίας και θράκης με επικεφαλής τον περιφερειάρχη χρήστο μέτιο, δημάρχους της περιοχής, αντιπεριφερειάρχες και πολλούς εκπροσώπους των παραγωγικών και επιστημονικών φορέων. απαντώντας σε ερώτηση δημοσιογράφου από τη βουλγαρία, ο πρωθυπουργός μπόικο μπορίσοφ τόνισε ότι "είχαμε τη δυνατότητα να μιλήσουμε με τον ομόλογο μου και χθες στο δείπνο για το πρόβλημα των βαλκανίων. ένα από τα μεγαλύτερα προβλήματα είναι ότι εάν κινηθούμε στο επίπεδο, ποια χώρα είναι η μεγαλύτερη (...) τότε το αποτέλεσμα είναι πολύ επικίνδυνο (...) για αυτό με τη σεμνότητα που μας διακρίνει (σ.σ. μιλώντας για τον έλληνα πρωθυπουργό και τον εαυτό του) ξεκινάμε με πιο μικρά πράγματα", είπε ο κ. μπορισοφ κάνοντας αναφορά στα έργα υποδομής σε επίπεδο αυτοκινητοδρόμων και σιδηροδρομικής σύνδεσης που βρίσκονται σε εξέλιξη. & 500 & very low & Low & Socio-Economic & Socio-Economic & NA & 2017-09-06 & 2017 & 2 & ECO
Frame & v.low & National & <500 & 1.8924259 & 1.3590316 & -1.2391092 & 0.9814602 & -0.7922579 & 0.0 & -0.9049211 & -1.0736569 & Recipient & Domestic & European & Mixed & Domestic|ECO & Positive\\
\addlinespace
Greece & https://e-thessalia.gr/sto-espa-thessalias-chrimatodotisi-ton-kentron-kinotitas-skopelou-kileler-ke-sofadon/ & 354 & e-thessalia.gr & Private/Non-Public & Online only & Regional/Local & high = CP is most important issue in story (can also cover other issues) & Social awareness/inclusion & Positive & Subnational & No myth & Social justice & Positive & Subnational & No myth & NA & NA & NA & NA & Greece & στο εσπα θεσσαλίας η χρηματοδότηση των κέντρων κοινότητας σκοπέλου, κιλελέρ και σοφάδων - e-thessalia.gr & 2017-02-22 & ευρωπαϊκό κοινωνικό ταμείο & στο εσπα θεσσαλίας 2014-2020 εντάσσονται, μετά από απόφαση του περιφερειάρχη θεσσαλίας κ. κώστα αγοραστού, για χρηματοδότηση για τα επόμενα 3 χρόνια, τα παρακάτω έργα που αφορούν σε δομές "κέντρο κοινότητας" σε δήμους της θεσσαλίας : α) κέντρο κοινότητας δήμου κιλελέρ, με προϋπολογισμό 112.320 € και δικαιούχο το δήμο κιλελέρ β) κέντρο κοινότητας δήμου σκοπέλου, με προϋπολογισμό 112.320 € και δικαιούχο το δήμο σκοπέλου γ) κέντρο κοινότητας δήμου σοφάδων με παράρτημα ρομά με προϋπολογισμό 353.616 € και δικαιούχο τη δημοτική κοινωφελή επιχείρηση δήμου σοφάδων (δη.κοι.ε.δ.ς.). όπως ανέφερε σε δηλώσεις του ο κ. κ. αγοραστός: "στη δύσκολη οικονομική συγκυρία, που βιώνει η χώρα η στήριξη των συνανθρώπων μας αποτελεί πράξη ευθύνης για την περιφέρεια θεσσαλίας. στεκόμαστε με σεβασμό και ενδιαφέρον προς τον συνάνθρωπο και ενισχύουμε στην πράξη δίκτυα κοινωνικής στήριξης. στόχος μας είναι η βελτίωση της ποιότητας ζωής των ανθρώπων αυτών, η ενίσχυση της κοινωνικής συνοχής και η πρόληψη φαινομένων περιθωριοποίησης και κοινωνικού αποκλεισμού. στηρίζουμε τις ευπαθείς κοινωνικές ομάδες δημιουργούμε θέσεις εργασίας". οι αποφάσεις ένταξης των 3 έργων στο εσπα / π.ε.π. θεσσαλίας 2014-2020/ άξονας προτεραιότητας 2.α "ανάπτυξη και αξιοποίηση ικανοτήτων ανθρώπινου δυναμικού - ενεργός κοινωνική ενσωμάτωση" που συγχρηματοδοτείται από το ευρωπαϊκό κοινωνικό ταμείο (εκτ), εκδίδονται μετά την ολοκλήρωση και θετική αξιολόγηση από την ειδική υπηρεσία διαχείρισης ε.π. περιφέρειας θεσσαλίας. η δράση "κέντρο κοινότητας", εντάσσεται αφενός στην ευρύτερη στρατηγική της περιφέρειας θεσσαλίας για την περίοδο 2014-2020, για την κοινωνική ένταξη, την καταπολέμηση της φτώχειας και την εμβάθυνση της κοινωνικής συνοχής, και αφετέρου στις ανάγκες της τρέχουσας περιόδου της χώρας για τον σχεδιασμό και την υλοποίηση πολιτικών εθνικής εμβέλειας που συμβάλλουν στην αντιμετώπιση της φτώχειας και στην κοινωνική ένταξη. τα κέντρα κοινότητας έχουν ως στόχο να υποστηρίξουν τους οτα α' βαθμού στην εφαρμογή πολιτικών κοινωνικής προστασίας και στην ανάπτυξη ενός τοπικού σημείου αναφοράς για την υποδοχή, εξυπηρέτηση και διασύνδεση των πολιτών με όλα τα κοινωνικά προγράμματα και υπηρεσίες που υλοποιούνται στην περιοχή παρέμβασης του "κέντρου κοινότητας". δραστηριοποιούνται στα πεδία: - υποδοχή - ενημέρωση - υποστήριξη των πολιτών - συνεργασία με υπηρεσίες και δομές - παροχή υπηρεσιών που αποσκοπούν στη βελτίωση του βιοτικού επιπέδου και διασφαλίζουν την κοινωνική ένταξη των ωφελουμένων ωφελούμενοι των κέντρων κοινότητας είναι οι πολίτες που διαβιούν στους αντίστοιχους δήμους, με προτεραιότητα στα μέλη των ευάλωτων κοινωνικά ομάδων, τους ωφελούμενους κοινωνικών προγραμμάτων, τους ωφελούμενους του προγράμματος "κοινωνικό εισόδημα αλληλεγγύης", άτομα που διαβιούν σε συνθήκες φτώχειας, άτομα και οικογένειες που αντιμετωπίζουν δυσκολία κάλυψης βασικών αναγκών διαβίωσης, πρόσβασης σε κοινωνικές υπηρεσίες και κοινωνικά αγαθά, ανέργους, αμεα, και γενικότερα άτομα που έχουν πληγεί από την οικονομική κρίση. η ένταξη των έργων αυτών εκδίδεται στο πλαίσιο της σχετικής πρόσκλησης του πεπ θεσσαλίας 2014-2020 που είχε εκδοθεί και απευθύνονταν στο σύνολο των δήμων της θεσσαλίας. & 443 & high & High & Socio-Economic & Socio-Economic & NA & 2017-02-22 & 2017 & 2 & ECO
Frame & high-very high & Regional & <500 & 1.8924259 & 1.3590316 & -1.2391092 & 0.9814602 & -0.7922579 & 0.0 & -0.9049211 & -1.0736569 & Recipient & Domestic & Domestic & Domestic & Domestic|ECO & Positive\\
Greece & https://www.newsbeast.gr/financial/greek-financial/arthro/4494974/nees-imerominies-gia-ypovoli-aitiseon-sto-espa & 372 & Newsbeast.gr & Private/Non-Public & Online only & National & low = CP mentioned more times but NOT important part of story (mainly about others issues) & Economic development & Factual & EU + National & No myth & NA & NA & NA & NA & NA & NA & NA & NA & Greece & νέες ημερομηνίες για υποβολή αιτήσεων στο εσπα & 2019-02-06 & ευρωπαϊκό ταμείο περιφερειακής ανάπτυξης & την τροποποίηση των ημερομηνιών έναρξης και λήξης ηλεκτρονικής υποβολής αιτήσεων χρηματοδότησης για τις προσκλήσεις των δράσεων του εσπα: "εργαλειοθήκη ανταγωνιστικότητας για μικρές και πολύ μικρές επιχειρήσεις" και "εργαλειοθήκη επιχειρηματικότητας: εμπόριο - εστίαση - εκπαίδευση", ανακοίνωσε το υπουργείο οικονομίας και ανάπτυξης. ειδικότερα, όπως αναφέρει το αθηναϊκό πρακτορείο ειδήσεων, για τη δράση "εργαλειοθήκη ανταγωνιστικότητας για μικρές και πολύ μικρές επιχειρήσεις" η έναρξη ηλεκτρονικών υποβολών αιτήσεων τροποποιείται από τις 06/02/2019 στις 20/02/2019 (ώρα 10:00:00). η πρόσκληση θα παραμείνει ανοιχτή μέχρι εξαντλήσεως του προϋπολογισμού και το αργότερο μέχρι τη συμπλήρωση 18 μηνών από την αρχική δημοσίευσή της. για τη δράση "εργαλειοθήκη επιχειρηματικότητας: εμπόριο-εστίαση-εκπαίδευση" η ημερομηνία έναρξης ηλεκτρονικών υποβολών αιτήσεων τροποποιείται από τις 06/02/2019 στις 27/02/2019 καθώς και η ημερομηνία λήξης ηλεκτρονικών υποβολών αιτήσεων από τις 19/04/2019 στις 09/05/2019 και ώρα 23:59:59. οι δράσεις συγχρηματοδοτούνται από το ευρωπαϊκό ταμείο περιφερειακής ανάπτυξης (ετπα) της ευρωπαϊκής ένωσης και από εθνικούς πόρους. για κάθε πρόσθετη πληροφορία σχετικά με τη δράση οι ενδιαφερόμενοι μπορούν να απευθύνονται: & 173 & low & Low & Socio-Economic & NA & NA & 2019-02-06 & 2019 & 3 & ECO
Frame & low-medium & National & <500 & 1.8924259 & 1.3590316 & -1.2391092 & 0.9814602 & -0.7922579 & 0.0 & -0.9049211 & -1.0736569 & Recipient & Domestic & European & Mixed & Domestic|ECO & Neutral\\
Greece & http://voria.gr/article/i-ekti-mpienale-thessalonikis-theli-na-tarakounisi-oli-tin-poli & 316 & Ernst\&Young: Φρένο στις δημόσιες εγγραφές παγκοσμίως & Private/Non-Public & Online only & Regional/Local & very low = CP mentioned once & Cultural development & Positive & EU & No myth & NA & NA & NA & NA & NA & NA & NA & NA & Greece & η φετινή μπιενάλε θέλει να ταρακουνήσει όλη τη θεσσαλονίκη & 2017-09-21 & ευρωπαϊκό ταμείο περιφερειακής ανάπτυξης & ο καλλιτεχνικός θεσμός υπόσχεται ένα δυνατό παράλληλο πρόγραμμα εκθέσεων και performances που θα διαχέεται σε κάθε άκρη της πόλης. με διάθεση να ταρακουνήσει κάθε γωνιά της πόλης έρχεται η 6η μπιενάλε σύγχρονης τέχνης της θεσσαλονίκης, η οποία υπόσχεται ένα δυνατό παράλληλο πρόγραμμα εκθέσεων και performances που θα διαχέεται από τη δυτική είσοδο της πόλης μέχρι την καλαμαριά και από το επταπύργιο μέχρι το λιμάνι. η μπιενάλε, που κλείνει φέτος τα δέκα της χρόνια, ξεκινά στις 30 σεπτεμβρίου και ολοκληρώνεται στις 4 ιανουαρίου 2018, φιλοξενώντας τα έργα 95 καλλιτεχνών από όλον τον κόσμο με θέμα τις "φαντασιακές εστίες / ιmagined homes". βέβαια, τα πράγματα αρχικά δεν ήταν καθόλου ρόδινα για τη φετινή διοργάνωση, η οποία είχε να αντιμετωπίσει εμπόδια, με κυριότερο τη σημαντικότατη μείωση του προϋπολογισμού σε σχέση με τις προηγούμενες χρονιές. συγκεκριμένα, η 6η μπιενάλε είχε στη διάθεσή της μόλις το 20\% του συνολικού budget των προηγούμενων διοργανώσεων, όπου οι πόροι έφταναν το ένα εκατομμύριο ανά διετία. σημειώνεται ότι το εσπα ήταν που χρηματοδότησε τις προηγούμενες τρεις μπιενάλε, ενώ η φετινή διοργάνωση υποστηρίχθηκε οικονομικά από το ευρωπαϊκό ταμείο περιφερειακής ανάπτυξης, με συγχρηματοδότηση από την ελλάδα και την ευρωπαϊκή ένωση. "η ομάδα πείσμωσε και έκανε ό,τι καλύτερο μπορούσε με τα μέσα που διέθετε" τόνισε μέλος της έκτης μπιενάλε, μιλώντας στη voria.gr. πράγματι, αυτή η διοργάνωση στηρίχθηκε σε μεγάλο βαθμό στις δικές της δυνάμεις αλλά και τις συνεργασίες που έχει καλλιεργήσει με φορείς και ιδρύματα τόσο της πόλης όσο και του εξωτερικού. η 6η μπιενάλε σε αντίθεση με τις προηγούμενες χρονιές δεν κάλεσε έναν κεντρικό επιμελητή από το εξωτερικό αλλά βασίστηκε σε επιμελητές του κρατικού μουσείου σύγχρονης τέχνης, διαμορφώνοντας ένα ομαδικό επιμελητικό σχήμα. ακόμα, για πρώτη φορά φέτος απηύθυνε ανοιχτή πρόσκληση σε καλλιτέχνες, στην οποία ανταποκρίθηκαν 1.452 εικαστικοί από διάφορες χώρες του εξωτερικού. μια ακόμα πρωτοτυπία της έκτης κατά σειράς διοργάνωσης είναι πως παράλληλα με το πρόγραμμα της, έχει ξεκινήσει και ένα πρόγραμμα διαμονής οκτώ καλλιτεχνών οι οποίοι φιλοξενούνται από τις 25 μαΐου στην πόλη με στόχο να δημιουργήσουν έργα που θα εκτεθούν στη μπιενάλε. ανάμεσα στις δράσεις της μπιενάλε ξεχωρίζει το φεστιβάλ περφόρμανς θεσσαλονίκης, το οποίο θα διεξαχθεί φέτος από τις 13 έως τις 21 οκτωβρίου και το οποίο οργανώνεται από το κρατικό μουσείο σύγχρονης τέχνης με την υποστήριξη των 52ων δημητρίων. περιλαμβάνει μια σειρά από ζωντανές δράσεις ελλήνων και ξένων καλλιτεχνών όπως και ένα αφιέρωμα στην ana mendieta (κούβα, 1948 - νέα υόρκη, 1985), μια εικαστικό που διεκδικεί εμβληματική θέση στη φεμινιστική ιστορία της τέχνης. το αφιέρωμα θα παρουσιάζεται καθ' όλη τη διάρκεια της μπιενάλε στο κμστ, στη μονή λαζαριστών (30.09.2017 - 14.01.2018). η ομάδα της διοργάνωσης υποστηρίζει ότι η μπιενάλε έχει καθιερωθεί πλέον στον διεθνή χάρτη των εικαστικών, γεγονός που, όπως λέει, αποδεικνύεται και από τη μεγάλη συμμετοχή καλλιτεχνών στο open call της μπιενάλε αλλά και από το γεγονός ότι στη μπιενάλε θα έρθουν και ξένοι καλλιτέχνες - που δεν έγιναν δεκτοί- με δικά τους έξοδα. όπως λένε τα μέλη της ομάδας, ένας από τους βασικούς στόχους της διοργάνωσης που ήταν ο καλλιτεχνικός θεσμός να αγκαλιαστεί από την πόλη έχει επιτευχθεί καθώς όπως αναφέρουν, τα τηλέφωνα του κρατικού μουσείου σύγχρονης τέχνης χτυπούν εδώ και μέρες από πολίτες που ρωτούν πότε θα αρχίσει φέτος η μπιενάλε. η συνεργασία με τις δημιουργικές δυνάμεις της θεσσαλονίκης, η οποία θα βοηθήσει στην διοργάνωση πολλών παράλληλων δράσεων σε όλη την πόλη, αλλά και η εξωστρέφεια της μπιενάλε με τη συνεργασία της με πλήθος ξένων καλλιτεχνών αποτελεί το δίπολο, που η ομάδα της μπιενάλε πιστεύει ότι θα την κάνει να ξεχωρίσει. ελλάδα και τουρκία ενώνουν τις καλλιτεχνικές τους δυνάμεις σημειώνεται πως φέτος η μπιενάλε της θεσσαλονίκης θα φιλοξενήσει μέρος της μπιενάλε του τσανάκ καλέ που επρόκειτο να πραγματοποιηθεί πέρσι αλλά ακυρώθηκε λόγω των πολιτικών εξελίξεων. το θέμα των δύο μπιενάλε ήταν παρόμοιο και έτσι αποφασίστηκε να υπάρξει αυτή η συνεργασία. λίγα λόγια για το θέμα της μπιενάλε η εστία ως φαντασιακή σύλληψη και κατασκευή αποτελεί το θεματικό πυρήνα της 6ης μπιενάλε θεσσαλονίκης. με τους όρους "εστία" ή "σπίτι" αναφερόμαστε όχι μόνο στην κατοικία, αλλά και στην κοινότητα/πατρίδα, τον τόπο όπου νιώθει κανείς ασφαλής και αποδεκτός, έχει τις ρίζες του και αναπτύσσει τον πυρήνα των κοινωνικών και οικογενειακών του σχέσεων. στη σημερινή εποχή ωστόσο τα παραπάνω έπαψαν να είναι βεβαιότητες και αποτελούν ζητούμενα για μεγάλες πληθυσμιακές ομάδες που υποχρεώνονται να εκτοπιστούν, να εγκαταλείψουν τη μόνιμη κατοικία τους και να αναζητήσουν έναν νέο, περισσότερο ασφαλή τόπο διαμονής. οι τόποι αυτοί υφίστανται στο φαντασιακό ως κατασκευή, ως προβολή και προσδοκία. εξάλλου, η μαζικότητα και τα ιδιαίτερα χαρακτηριστικά της βίαιης μετακίνησης τροποποιούν ριζικά και το σπίτι - προορισμό. όχι μόνο το διασπορικό υποκείμενο, αλλά και το περιβάλλον υποδοχής αλλάζει, δεχόμενο τον κλυδωνισμό της διαχείρισης- ενσωμάτωσης και τις επιπτώσεις της παρουσίας των νέων κατοίκων. ως φαντασιακή σύλληψη, η εστία που φέρουμε μέσα μας, όταν μετακινούμαστε από ανάγκη ή από επιλογή, αποτελεί μια φαντασιακή κατασκευή υπό συνεχή αναδιαμόρφωση που επιδιώκει να ανταποκριθεί σε επείγουσες ανάγκες επιβίωσης και βελτίωσης των συνθηκών διαβίωσης. & 813 & very low & Low & Socio-Economic & NA & NA & 2017-09-21 & 2017 & 2 & ECO
Frame & v.low & Regional & 500-1000 & 1.8924259 & 1.3590316 & -1.2391092 & 0.9814602 & -0.7922579 & 0.0 & -0.9049211 & -1.0736569 & Recipient & European & European & European & European|ECO & Positive\\
Greece & http://www.avgi.gr/article/10813/8725619/diabouleuse-draseon-gia-ten-koinonike-entaxe-ton-roma-apo-to-ypourgeio-ergasias & 379 & avgi.gr & Private/Non-Public & Online and Offline & National & medium = CP is important part of story & Social awareness/inclusion & Positive & EU + National & No myth & NA & NA & NA & NA & NA & NA & NA & NA & Greece & διαβούλευση δράσεων για την κοινωνική ένταξη των ρομά από το υπουργείο εργασίας & 2018-02-22 & ευρωπαϊκό κοινωνικό ταμείο & προτεραιότητα η μετεγκατάσταση 70 οικισμών σε εξαθλίωση, τόνισε η θ. φωτίου η σχεδιαζόμενη πολιτική κοινωνικής ένταξης των ρομά, σύμφωνα με τις προτάσεις της ειδικής γραμματείας, καθώς και η ενεργοποίηση συγκεκριμένων συγχρηματοδοτούμενων δράσεων για την κοινωνική ένταξη των ρομά από το ευρωπαϊκό κοινωνικό ταμείο, ήταν το θέμα της συνάντησης που πραγματοποιήθηκε ανάμεσα στην ειδική γραμματεία για την κοινωνική ένταξη των ρομά και σε εκπροσώπους των περιφερειών και των διαχειριστικών αρχών τους, στο υπουργείο εργασίας, κοινωνικής ασφάλισης και κοινωνικής αλληλεγγύης. κατά τη διάρκεια της συνάντησης, η αναπληρώτρια υπουργός κοινωνικής αλληλεγγύης, θεανώ φωτίου, απηύθηνε έναν σύντομο χαιρετισμό, επισημαίνοντας, μεταξύ άλλων ότι, πριν από τη δημιουργία της ειδικής γραμματείας ρομά, το κράτος δεν είχε εικόνα για την κατάσταση των καταυλισμών και των οικισμών. "με τη σύσταση της ειδικής γραμματείας, χαρτογραφήσαμε 371 οικισμούς ρομά σε όλη τη χώρα. ξέρουμε πλέον σε κάθε περιφέρεια πόσοι καταυλισμοί υπάρχουν και σε ποια κατάσταση βρίσκονται, με κάθε λεπτομέρεια" διευκρίνισε η ίδια. "τους κατατάξαμε σε τρεις κατηγορίες. σε αυτούς που βρίσκονται σε κατάσταση εξαθλίωσης επικεντρώνουμε την προσοχή μας και δίνουμε προτεραιότητα. γνωρίζουμε πλέον ότι σε κατάσταση εξαθλίωσης βρίσκονται 70 οικισμοί σε όλη την επικράτεια, οι οποίοι θα μετεγκατασταθούν με συνέργεια της ειδικής γραμματείας ρομά και των δήμων" τόνισε η αναπληρώτρια υπουργός κοινωνικής αλληλεγγύης. παράλληλα, σημείωσε ότι οι κατευθυντήριες γραμμές δράσης της ειδικής γραμματείας συνοψίζονται στην εξάλειψη των ακραία υποβαθμισμένων καταυλισμών και θυλάκων, στη θεμελίωση των βάσεων για την επίτευξη της σταδιακής, αλλά πλήρους κοινωνικής ένταξης των κατοίκων, με έμφαση στην εκπαίδευση, την απασχόληση και την υγεία και στην ανάδειξη καινοτόμων πιλοτικών παρεμβάσεων που να λειτουργούν ως πρότυπα. "συγκροτήσαμε ένα θεσμικό πλαίσιο για την προσωρινή μετεγκατάσταση, ώστε να ακολουθηθούν ορισμένοι κανόνες. οι οικισμοί δεν θα είναι γκέτο, θα βρίσκονται κοντά στον αστικό ιστό και θα διαθέτουν τις προβλεπόμενες υπηρεσίες, σε συνεργασία με τους δήμους. οι δήμοι προσφέρουν την έκταση. εμείς στο υπουργείο, στην εδική γραμματεία ρομά, με ένα σύνολο τεχνικών συμβούλων, κάνουμε τα σχέδια της μετεγκατάστασης. στη συνέχεια, δίνουμε στον δήμο όλα τα σχέδια έτοιμα- πάντα σε συνεργασία με τις τοπικές αρχές -για δημοπράτηση. είμαστε ήδη έτοιμοι για την άμφισσα. θα μετεγκαταστήσουμε τους ανθρώπους που ζουν σε εξαθλίωση πολύ γρήγορα. με ταχύτατους ρυθμούς" συμπλήρωσε η κ. φωτίου, προσθέτοντας ότι, για κάθε συνοικισμό που μετεγκαθίσταται, θα εκδίδεται σχετική κοινή υπουργική απόφαση, που θα συνοδεύεται από το αντίστοιχο σχέδιο γενικής διάταξης οικισμού. σύμφωνα με την κ. φωτίου, η υλοποίηση των σχεδίων, αν χρειαστεί, θα χρηματοδοτηθεί από τον κρατικό προϋπολογισμό. "δεν θα περιμένουμε χρόνια, για να μετεγκαταστήσουμε τους ρομά και να τους βγάλουμε από την εξαθλίωση" ανέφερε χαρακτηριστικά. "ακόμη μία καινοτόμα δράση στην οποία προχωράμε, είναι τα πολύκεντρα. πρόκειται για προκατασκευασμένους οικίσκους, όπου τα παιδιά μπορούν να πλένονται και να ντύνονται, να υποστηρίζονται εκπαιδευτικά και υγειονομικά και να παρέχονται δράσεις ολιστικές για τη νεολαία. τα πολύκεντρα τοποθετούνται σε οικόπεδο του δήμου ή κάποιου φορέα που το παραχωρεί γι' αυτήν τη χρήση. όλο το θεσμικό πλαίσιο είναι σε ισχύ" δήλωσε η κ. φωτίου. τέλος, είπε ότι οι δράσεις πρέπει να γίνουν γρήγορα. "η ελπίδα μας είναι τα παιδιά. η νέα γενιά των ρομά. γι' αυτό, με ενδιαφέρουν πολύ οι δράσεις εκπαίδευσης. δεν είναι τυχαίο ότι προϋπόθεση, για να λάβει ο δικαιούχος το κοινωνικό εισόδημα αλληλεγγύης (κεα), είναι να παρακολουθεί το παιδί του την υποχρεωτική εκπαίδευση. αλλιώς, χάνει το κεα. η ηδικα πρόκειται να συνδεθεί με το "my school", ώστε να γνωρίζουμε ανά πάσα στιγμή ποιο παιδί σταματά το σχολειό. από τα στοιχεία που έχουμε στη διάθεσή μας, γνωρίζουμε ότι ήδη καταγράφεται τεράστια αύξηση της παρακολούθησης των παιδιών ρομά στο σχολείο. όλα αυτά είναι μέτρα που θεωρούμε ότι θα φέρουν τους ρομά σε μία νέα σύγχρονη εποχή και γνωρίζουμε ότι ήδη τα έχει αγκαλιάσει η κοινότητα των ρομά" σχολίασε η κ. φωτίου. α α α email εκτυπωση κατηγορία ελλάδα ροη κατηγοριας & 615 & medium & Medium & Socio-Economic & NA & NA & 2018-02-22 & 2018 & 3 & ECO
Frame & low-medium & National & 500-1000 & 1.8924259 & 1.3590316 & -1.2391092 & 0.9814602 & -0.7922579 & 0.0 & -0.9049211 & -1.0736569 & Recipient & Domestic & European & Mixed & Domestic|ECO & Positive\\
Greece & https://www.dikaiologitika.gr/eidhseis/aftodioikisi/246509/kostas-agorastos-exoume-stoxo-tin-paroxi-ypsiloy-epipedou-frontidas-pros-tin-koinonia & 355 & dikaiologitika.gr & Private/Non-Public & Online only & National & high = CP is most important issue in story (can also cover other issues) & Infrastructure & Positive & EU + Subnational & No myth & NA & NA & NA & NA & NA & NA & NA & NA & Greece & κώστας αγοραστός: έχουμε στόχο την παροχή υψηλού επιπέδου φροντίδας προς την κοινωνία - dikaiologitika news | ειδήσεις τώρα - σήμερα & 2019-03-01 & ευρωπαϊκό ταμείο περιφερειακής ανάπτυξης & νέο εξοπλισμό αποκτά σύντομα το γενικό νοσοκομείο καρδίτσας μετά την έγκριση για υπογραφή της σύμβασης προμήθειας που δόθηκε από τον περιφερειάρχης θεσσαλίας κ. κώστα αγοραστό. συγκεκριμένα πρόκειται για ενταγμένο στο ε.σ.π.α./ π.ε.π. θεσσαλίας 2014-2020 έργο "προμηθεια και εγκατασταση ιατροτεχνολογικου εξοπλισμου στις κλινικες του γ.ν. καρδιτσας", προϋπολογισμού 325.350,00 ευρώ. το έργο συγχρηματοδοτείται από το ευρωπαϊκό ταμείο περιφερειακής ανάπτυξης και εθνικούς πόρους, μέσω του προγράμματος δημοσίων επενδύσεων. όπως ανέφερε σε δηλώσεις του ο κ. κώστας αγοραστός: "σε μια δύσκολη περίοδο για τη χώρα και κατ' επέκταση και για το εθνικό σύστημα υγείας, επιθυμούμε μέσω του εσπα να βελτιώσουμε τις υποδομές των νοσοκομείων της θεσσαλίας καθιστώντας τα αποδοτικότερα και πιο λειτουργικά. η αιρετή περιφέρεια κινήθηκε βάσει ενός ολοκληρωμένου σχεδίου πάνω στον τομέα της υγείας και της κοινωνικής πρόνοιας, στηρίζοντας τον συνάνθρωπό σε μια δύσκολη οικονομική συγκυρία. εξοπλίζουμε και αναβαθμίζουμε τη δημόσια υγεία, με στόχο την παροχή υψηλού επιπέδου φροντίδας προς το κοινωνικό σύνολο". ειδικότερα, μετά τη σύνταξη των τευχών δημοπράτησης από τον φορέα υλοποίησης και κύριο του έργου που είναι το γενικό νοσοκομείο καρδίτσας και την υποβολή τους στην ειδική υπηρεσία διαχείρισης περιφέρειας θεσσαλίας, διενεργήθηκε ο προβλεπόμενος προέλεγχος που οδήγησε σε θετικό αποτέλεσμα και προέγκριση δημοπράτησης του έργου. ακολουθεί πλέον η έγκριση και δημοσιοποίηση του διαγωνισμού από το νοσοκομείο. ο διαγωνισμός αφορά σε προμήθεια νέων ειδών καθώς και αντικατάσταση εξοπλισμού παρωχημένης τεχνολογίας για την κάλυψη των αναγκών διαφόρων κλινικών και τμημάτων του γενικού νοσοκομείου καρδίτσας. συγκεκριμένα για κάθε κλινική / τμήμα θα αποκτηθούν: οφθαλμολογικη: ένα χειρουργικό μικροσκόπιο με συμπαρατήρηση, ένα σύστημα βιομετρίας παχυμετρίας. αναισθησιολογικο: ένα αναισθησιολογικό συγκρότημα. ωρλ: ένας τυμπανογράφος, ένας διαγνωστικός ακοομετρητής, ένα φορητό μετωπιαίο κάτοπτρο. ουρολογικη: μία μονάδα για διουρηθρικές επεμβάσεις. καρδιολογικη: ένας αναπνευστήρας εντατικής θεραπείας, ένας φορητός αναλυτής αερίων αίματος - ηλεκτρολυτών. παθολογοανατομικο: ένας μικροτόμος αυτόματος με σύστημα ψύξης, μία επιδαπέδια ιστοκινέτα κλειστού τύπου. παθολογικη: τρία φορητά οξύμετρα δακτύλου. ορθοπεδικη: ένα σύστημα πριονίου - τρυπανίου μπαταρίας. φαρμακειο: ένα ψυγείο - συντήρηση φαρμάκων. η προμήθεια του νέου εξοπλισμού και η αντικατάσταση παλαιού με νέο και σύγχρονο τεχνολογικά εξοπλισμό θα επιφέρει τον εκσυγχρονισμό των παρεχόμενων υπηρεσιών του νοσοκομείου και την ανάπτυξη και εφαρμογή νέων και σύγχρονων ιατρικών τεχνικών, προς όφελος των ασθενών και ωφελούμενου πληθυσμού από το νοσοκομείο καρδίτσας. διαβάστε περισσότερες ειδήσεις για τις αυτοδιοικητικές εκλογές στα eklogikanews.gr & 372 & high & High & Socio-Economic & NA & NA & 2019-03-01 & 2019 & 3 & ECO
Frame & high-very high & National & <500 & 1.8924259 & 1.3590316 & -1.2391092 & 0.9814602 & -0.7922579 & 0.0 & -0.9049211 & -1.0736569 & Recipient & Domestic & European & Mixed & Domestic|ECO & Positive\\
\addlinespace
Greece & https://www.altsantiri.gr/kosmos/i-ellada-allazei-selida-mpainei-se-mia-nea-fasi-tonise-o-schoinas/ & 385 & ALTSANTIRI.gr & Private/Non-Public & Online only & National & medium = CP is important part of story & Solidarity to poor countries/regions & Positive & EU & No myth & NA & NA & NA & NA & NA & NA & NA & NA & Greece & "η ελλάδα αλλάζει σελίδα, μπαίνει σε μια νέα φάση", τόνισε ο σχοινάς | altsantiri.gr & 2018-04-27 & διαρθρωτικά ταμεία & ο εκπρόσωπος τύπου της κομισιόν, μαργαρίτης σχοινάς, σε δηλώσεις του στην ανοιχτή συνεδρίαση της ετήσιας συνέλευσης των μελών του συνδέσμου ελληνικών βιομηχανιών τροφίμων, αναφέρθηκε στην ελλάδα, τονίζοντας ότι "αλλάζει σελίδα. είναι μια περίοδος πολύ κρίσιμη όπου μετά από σχεδόν δέκα χρόνια πρωτοφανούς δημοσιονομικής προσαρμογής η ελλάδα μπαίνει σε μια νέα φάση". όπως τόνισε στην ομιλία του, "όλα τα ελληνικά προγράμματα έτσι όπως εφαρμόστηκαν στη χώρα μας, από τη σκοπιά της ευρώπης, ποτέ δεν είχαν ιδωθεί ως προγράμματα αμιγώς οικονομικοκεντρικά. υπήρχε ένα σημαντικό δημοσιονομικό πρόσημο αλλά η ευρώπη πάντοτε έβλεπε τα προγράμματα προσαρμογής της ελληνικής οικονομίας μετά την κρίση ως μια μεγάλη ευκαιρία, ίσως την τελευταία, να αποκτήσει η ελλάδα ένα νέο, άξιο, σύγχρονο κράτος με μεταρρυθμίσεις σε όλο το εύρος των παραγωγικών και διοικητικών θεσμών της χώρας, με τράπεζες που θα λειτουργούν υπέρ της ανάπτυξης και της πραγματικής οικονομίας, με παιδεία ανοιχτή σε όλους, με μια οικονομία που δεν θα αφήνει χρέη στους νέους έλληνες αλλά θα δημιουργεί ευκαιρίες. τα ελληνικά προγράμματα σχεδιάστηκαν και για αυτό τον σκοπό και πολλές από αυτές τις μεταρρυθμίσεις, ίσως οι περισσότερες έχουν ήδη εισαχθεί στην ελληνική έννομη τάξη". ωστόσο, επεσήμανε ότι "η ψήφιση των μεταρρυθμίσεων δεν είναι αρκετή. είναι αναγκαία αλλά όχι ικανή συνθήκη για το πέρασμα στη νέα εποχή. πρέπει όλοι μαζί, ο πολιτικός κόσμος, οι παραγωγικές τάξεις της χώρας και οι κοινωνικοί εταίροι να έχουν μια αυξημένη θέληση οι μεταρρυθμίσεις να εφαρμοστούν στην πράξη σωστά. θα είναι ένα μεγάλο λάθος, ίσως ένα δεύτερο λάθος που δεν πρέπει να ξανακάνουμε, να περιμένουμε να μας επιβάλλουν το αυτονόητο". "η επιστροφή της ελλάδας στην κανονικότητα γεννάει ένα ευρύτερο κλίμα αισιοδοξίας", σύμφωνα με τον κ. σχοινά, ο οποίος στη συνέχεια ανέλυσε τους τρεις λόγους για τους οποίους βλέπει το μέλλον της ελλάδας και της ελληνικής οικονομίας με καλύτερες προοπτικές. συγκεκριμένα, ανέφερε ότι "οι ιστορικές θυσίες των ελλήνων, χωρίς προηγούμενο, άνοιξαν το κλειδί της μετάβασης στην κανονικότητα. δεν υπάρχει προηγούμενο χώρας σε καμία περιοχή του κόσμου που να δέχθηκε την απώλεια σχεδόν του 25\% του αεπ της χωρίς να γίνει επανάσταση. εκείνο το καλοκαίρι του ιουλίου όταν η ελλάδα κινδύνεψε να χωριστεί σε 40\% ευρωπαίους και 60\% μη ευρωπαίους, εκείνες τις μαύρες μέρες που η χώρα βρέθηκε στον αέρα γεωπολιτικά, είχαμε την ευτυχή κατάληξη αυτό το ρήγμα να μην βαθύνει αλλά να επουλωθεί". παράλληλα, "το χρέος της χώρας βρίσκεται πλέον στα χέρια της ευρωπαϊκής οικογένειας. το 80\% ίσως και παραπάνω του χρέους βρίσκεται στον esm δηλαδή στους εταίρους". τέλος, "η ευρώπη στη νέα περίοδο θα βρίσκεται δίπλα και όχι απέναντι στην ελλάδα. μόνο το τελευταία τρία χρόνια αυτή η χώρα δέχθηκε χρηματοδοτήσεις πέρα από τα 45,9 δισ. ευρώ χαμηλότοκων δανείων από τον esm, 15,9 δισ. ευρώ από τα διαρθρωτικά ταμεία. δεν υπάρχει προηγούμενο τέτοιας κοινοτικής συνδρομής από τα κοινοτικά ταμεία και σε αυτά πρέπει να προστεθούν τα 9,3 δισ. ευρώ από το σχέδιο γιούνκερ". & 470 & medium & Medium & Values & NA & NA & 2018-04-27 & 2018 & 3 & ECO
Frame & low-medium & National & <500 & 1.8924259 & 1.3590316 & -1.2391092 & 0.9814602 & -0.7922579 & 0.0 & -0.9049211 & -1.0736569 & Recipient & European & European & European & European|ECO & Positive\\
Greece & http://e-thessalia.gr/domes-kdif-ke-kifi-synechizoun-ti-litourgia-tous-meso-tou-espa-thessalias/ & 295 & e-thessalia.gr & Private/Non-Public & Online only & Regional/Local & very high = CP is most important issue + CP is mentioned in title/headline & Social justice & Positive & EU + Subnational & No myth & Social awareness/inclusion & Positive & EU + Subnational & No myth & NA & NA & NA & NA & Greece & δομές & 2016-06-11 & ευρωπαϊκό κοινωνικό ταμείο & στο πλαίσιο του εσπα / περιφερειακό επιχειρησιακό πρόγραμμα θεσσαλίας 2014-2020, και ειδικότερα του άξονα προτεραιότητας 2.α "ανάπτυξη και αξιοποίηση ικανοτήτων ανθρώπινου δυναμικού - ενεργός κοινωνική ενσωμάτωση", εκδόθηκαν με απόφαση του περιφερειάρχη θεσσαλίας κ. κώστα αγοραστού 2 προσκλήσεις για δράσεις που συγχρηματοδοτούνται από το ευρωπαϊκό κοινωνικό ταμείο (ε.κ.τ.), συνολικού προϋπολογισμού 1.418.000 € , και ειδικότερα : 1. συνέχιση λειτουργίας κέντρων διημέρευσης - ημερήσιας φροντίδας για άτομα με αναπηρίες (κδηφ) για διάρκεια 3(τρία) έτη , προϋπολογισμού 1.008.000 € με δικαιούχους και φορείς λειτουργίας α) την κοινωνική επιχείρηση κοινωνικής προστασίας και αλληλεγγύης - δημοτικό ινστιτούτο επαγγελματικής κατάρτισης (κεκπα - διεκ) του δήμου βόλου, για τη δομή στο βόλο και β) την πανελλήνια οργάνωση φροντίδας ευαίσθητων κοινωνικών ομάδων (ποφεκο) για τη δομή στα τρίκαλα. 2. συνέχιση λειτουργίας κέντρων ημερήσιας φροντίδας ηλικιωμένων (κηφη) στη θεσσαλία για διάρκεια 1(ένα) έτος , προϋπολογισμού 410.000 € με δικαιούχους και φορείς λειτουργίας α) την κοινωνική επιχείρηση κοινωνικής προστασίας και αλληλεγγύης - δημοτικό ινστιτούτο επαγγελματικής κατάρτισης (κεκπα-διεκ) του δήμου βόλου για τις 3 δομές στο δήμο και β) τον δήμο τρικκαίων για τη δομή στα τρίκαλα. οι προσκλήσεις αυτές αφορούν στις δομές "κδηφ" και "κηφη" που χωροθετούνται στη θεσσαλία και η λειτουργία τους χρηματοδοτήθηκε κατά την προηγούμενη προγραμματική περίοδο 2007 - 2013 από το ευρωπαϊκό κοινωνικό ταμείο (εκτ) μέσω του επιχειρησιακού προγράμματος "ανάπτυξη ανθρώπινου δυναμικού". όπως ανέφερε ο κ. κ. αγοραστός σκοπός των δράσεων είναι : 1. μέσω της δράσης των κδηφ παρέχονται υπηρεσίες ημερήσιας φροντίδας και παραμονής σε άτομα με αναπηρίες. επίσης, υλοποιούνται δράσεις δικτύωσης και συνεργασίας με κοινωνικούς φορείς / φορείς παροχής κοινωνικών υπηρεσιών καθώς και με άλλες δομές παροχής παρεμφερών υπηρεσιών και την τοπική κοινότητα γενικότερα. οι ωφελούμενοι των 2 δομών προβλέπεται να είναι 35 άτομα σε μηνιαία βάση και αντίστοιχα 35 άτομα θα αποδεσμευτούν από την φροντίδα αυτών για το διάστημα υποστήριξής τους στις δομές. η επιλογή των ωφελουμένων προβλέπεται να γίνει από τις δομές μετά τη θετική αξιολόγηση και ένταξη των πράξεων στο π.ε.π. θεσσαλίας 2014-2020, μέσα από ευρεία δημοσιότητα και συγκεκριμένα κριτήρια. 2. μέσω της δράσης των κηφη παρέχονται υπηρεσίες ημερήσιας φροντίδας σε ηλικιωμένα άτομα μη δυνάμενα να αυτοεξυπηρετηθούν απόλυτα (κινητικές δυσκολίες, άνοια κλπ), των οποίων το οικογενειακό περιβάλλον που τα φροντίζει, εργάζεται ή αντιμετωπίζει σοβαρά κοινωνικά και οικονομικά προβλήματα ή προβλήματα υγείας και αδυνατεί να ανταποκριθεί στη φροντίδα που έχει αναλάβει. οι ωφελούμενοι προβλέπεται να είναι 82 άτομα, η δε επιλογή τους θα γίνει με συγκεκριμένα κριτήρια μετά τη θετική αξιολόγηση και ένταξη των πράξεων στο π.ε.π. θεσσαλίας 2014-2020. οι προσκλήσεις με τα συνημμένα έγγραφά τους είναι αναρτημένες στον δικτυακό τόπο της ειδικής υπηρεσίας διαχείρισης ε.π. περιφέρειας θεσσαλίας www.thessalia-espa.gr. & 432 & very high & High & Socio-Economic & Socio-Economic & NA & 2016-06-11 & 2016 & 2 & ECO
Frame & high-very high & Regional & <500 & 1.8924259 & 1.3590316 & -1.2391092 & 0.9814602 & -0.7922579 & 0.0 & -0.9049211 & -1.0736569 & Recipient & Domestic & European & Mixed & Domestic|ECO & Positive\\
Greece & http://www.avgi.gr/article/10951/9475685/me-360-ekat-euro-tha-chrematodotethoun-dyo-programmata-tou-anaptyxiakou-nomou-to-2019 & 324 & avgi.gr & Private/Non-Public & Online and Offline & National & very low = CP mentioned once & Economic development & Positive & EU + National & No myth & NA & NA & NA & NA & NA & NA & NA & NA & Greece & με 360 εκατ. ευρώ θα χρηματοδοτηθούν δύο προγράμματα του αναπτυξιακού νόμου το 2019 & 2019-01-07 & ευρωπαϊκά διαρθρωτικά και επενδυτικά ταμεία & τα ποσά - κατά είδος ενισχύσεων - για τα επενδυτικά σχέδια που υπάγονται στo καθεστώς ενισχύσεων του αναπτυξιακού νόμου 4399/2016 "γενική επιχειρηματικότητα" και "ενισχύσεις μηχανολογικού εξοπλισμού", του έτους 2018, καθορίζονται με κοινή απόφαση του υπουργού οικονομίας και ανάπτυξης γιάννη δραγασάκη και του αναπληρωτή υπουργού οικονομικών γιώργου χουλιαράκη. πιο συγκεκριμένα, σύμφωνα με την απόφαση που δημοσιεύτηκε στη διαύγεια: α. για το καθεστώς "γενική επιχειρηματικότητα", το ποσό φτάνει τα 210.000.000 ευρώ, β. για το καθεστώς "ενισχύσεις μηχανολογικού εξοπλισμού", στα 150.000.000 ευρώ. το συνολικό ποσό της επιχορήγησης, της επιδότησης χρηματοδοτικής μίσθωσης και της επιδότησης του κόστους της δημιουργούμενης απασχόλησης των καθεστώτων του νόμου 4399/2016 που προκηρύσσονται το έτος 2018, καθορίζεται στο ποσό των 140.000.000 ευρώ για το καθεστώς "γενική επιχειρηματικότητα". όπως αναφέρεται στην απόφαση, μέρος των παραπάνω ποσών αφορά στις ενισχύσεις της φορολογικής απαλλαγής, της επιχορήγησης, της επιδότησης χρηματοδοτικής μίσθωσης "και της επιδότησης του κόστους της δημιουργούμενης απασχόλησης για τα επενδυτικά σχέδια που θα υπαχθούν, σύμφωνα με την κοινή απόφαση των υπουργών οικονομίας και ανάπτυξης και αγροτικής ανάπτυξης και τροφίμων για τον καθορισμό των ειδών επενδυτικών σχεδίων του τομέα πρωτογενούς γεωργικής παραγωγής των πολύ μικρών, μικρών και μεσαίων επιχειρήσεων που έχει εκδοθεί βάσει του κανονισμού 702/2014 της επιτροπής ‪της 25ης ιουνίου‬ "για την κήρυξη ορισμένων κατηγοριών ενισχύσεων στους τομείς της γεωργίας και δασοκομίας και σε αγροτικές περιοχές συμβιβάσιμων με την εσωτερική αγορά ...." και των ρυθμίσεων των άρθρων 7 παρ. 6 β (ββ) και 3 παρ. 3 του νόμου 4399/2016". πηγές χρηματοδότησης - επιβάρυνση κρατικού προϋπολογισμού όπως διευκρινίζεται, τα ποσά των επιχορηγήσεων, της επιδότησης χρηματοδοτικής μίσθωσης και της επιδότησης του κόστους της δημιουργούμενης απασχόλησης των επενδυτικών σχεδίων της απόφασης καλύπτονται από τον προϋπολογισμό δημοσίων επενδύσεων, στον οποίο εγγράφεται η δαπάνη των 140.000.000 ευρώ, και δύναται να προέλθουν από εθνικούς πόρους ή τα ευρωπαϊκά διαρθρωτικά και επενδυτικά ταμεία. επιπλέον, όπως αναφέρεται: "από τις διατάξεις της απόφασης εκτιμάται ότι: - για το 2018 δεν θα προκύψει δαπάνη σε βάρος του προϋπολογισμού δημοσίων επενδύσεων καθώς και απώλεια φορολογικών εσόδων. - για τo 2019 θα προκύψει δαπάνη 40.000.000 ευρώ από τον προϋπολογισμό δημοσίων επενδύσεων και απώλεια φορολογικών εσόδων ύψους 80.000.000 ευρώ - για τα επόμενα έτη η δαπάνη που θα προκύψει σε βάρος του προϋπολογισμού δημοσίων επενδύσεων συναρτάται από τον βαθμό υλοποίησης των επενδυτικών σχεδίων που θα υπαχθούν στο καθεστώς ενίσχυσης "γενική επιχειρηματικότητα" του αναπτυξιακού ν.4399/2016 και πάντως δεν θα υπερβαίνει το ποσό των 100.000.000 ευρώ, ενώ η απώλεια φορολογικών εσόδων δεν θα υπερβαίνει το ποσό των 280.000.000 ευρώ". & 417 & very low & Low & Socio-Economic & NA & NA & 2019-01-07 & 2019 & 3 & ECO
Frame & v.low & National & <500 & 1.8924259 & 1.3590316 & -1.2391092 & 0.9814602 & -0.7922579 & 0.0 & -0.9049211 & -1.0736569 & Recipient & Domestic & European & Mixed & Domestic|ECO & Positive\\
Greece & https://www.dikaiologitika.gr/eidhseis/paideia/230486/ilektroniki-ypovoli-aitiseon-gia-proslipseis-ekpaidefton-sta-kentra-dia-viou-mathisis & 356 & dikaiologitika.gr & Private/Non-Public & Online only & National & medium = CP is important part of story & Research \& innovation & Factual & EU + National & No myth & NA & NA & NA & NA & NA & NA & NA & NA & Greece & ηλεκτρονική υποβολή αιτήσεων για προσλήψεις εκπαιδευτών στα κέντρα διά βίου μάθησης - dikaiologitika news - ειδησεις & 2018-10-26 & ευρωπαϊκό κοινωνικό ταμείο & αναρτήθηκε στην ιστοσελίδα του ιδρύματος νεολαίας και διά βίου μάθησης (ι.νε.δι.βι.μ.) η πρόσκληση εκδήλωσης ενδιαφέροντος για σύναψη σύμβασης έργου για θέσεις εκπαιδευτών ενήλικων στα "κέντρα διά βίου μάθησης -νέα φάση" στο παρακάτω link: https://www.inedivim.gr/ανακοινώσεις/θέσεις-εκπαιδευτών-ενηλίκων-στα-κέντρα-διά-βίου-μάθησης-νεα-φαση-πρόσκληση-εκδήλωσης η προθεσμία υποβολής ηλεκτρονικών αιτήσεων ορίζεται: από 23/10/2018 (ώρα 11:00) έως και 06/11/2018 (ώρα 12:00). οι υποψήφιοι θα αποστείλουν στο ι.νε.δι.βι.μ. ταχυδρομικά ή με ταχυμεταφορές τα δικαιολογητικά τους εντός σφραγισμένου φακέλου διάστασης α4 έως και τις 09/11/2018, στη δ/νση: αχαρνών 417 και κοκκινάκη τ.κ. 11143, στο τμήμα κεντρικής γραμματείας 2ο όροφος. φάκελοι δικαιολογητικών που θα κατατεθούν ιδιοχείρως δεν θα γίνονται δεκτοί. το εν λόγω έργο εντάσσεται στο επιχειρησιακό πρόγραμμα "ανάπτυξη ανθρώπινου δυναμικού, εκπαίδευση και διά βίου μάθησης 2014-2020)" με τίτλο πράξεων "κέντρα διά βίου μάθησης (κ.δ.β.μ. )- νέα φάση", που συγχρηματοδοτούνται από την ελλάδα και την ευρωπαϊκή ένωση (ευρωπαϊκό κοινωνικό ταμείο -εκτ). & 173 & medium & Medium & Socio-Economic & NA & NA & 2018-10-26 & 2018 & 3 & ECO
Frame & low-medium & National & <500 & 1.8924259 & 1.3590316 & -1.2391092 & 0.9814602 & -0.7922579 & 0.0 & -0.9049211 & -1.0736569 & Recipient & Domestic & European & Mixed & Domestic|ECO & Neutral\\
Greece & https://www.newsbeast.gr/greece/arthro/4737633/i-anisychitiki-ereyna-gia-tin-athina-kai-o-kindynos-gia-tin-ygeia-ton-katoikon-tis-polis & 361 & Newsbeast.gr & Private/Non-Public & Online only & National & very low = CP mentioned once & Environment/green/low-carbon & Positive & National & No myth & NA & NA & NA & NA & NA & NA & NA & NA & Greece & η ανησυχητική έρευνα για την αθήνα και ο κίνδυνος για την υγεία των κατοίκων της πόλης & 2019-04-09 & ευρωπαϊκό ταμείο περιφερειακής ανάπτυξης & τι αποκαλύπτει το εθνικό αστεροσκοπείο αθηνών όσο περνάνε τα χρόνια, οι κάτοικοι της αθήνας αντιμετωπίζουν ολοένα μεγαλύτερο θερμικό κίνδυνο για το σώμα τους και παράλληλα νιώθουν περισσότερη θερμική δυσφορία, ακόμη και τις νυχτερινές ώρες, όπως δείχνει μια νέα έρευνα επιστημόνων του ινστιτούτου ερευνών περιβάλλοντος και βιώσιμης ανάπτυξης (ιεπβα) του εθνικού αστεροσκοπείου αθηνών (εαα). επιπλέον, σταδιακά εμφανίζονται ολοένα πιο πρόωρα και ταυτόχρονα αυξάνονται σε διάρκεια μέσα στο έτος οι μέρες και οι νύχτες που υπάρχει θερμικό στρες, ενώ η κατάσταση προβλέπεται να γίνει ακόμη χειρότερη τις επόμενες δεκαετίες. οι ερευνητές, που έκαναν σήμερα σχετική ανακοίνωση στο διεθνές συνέδριο της ευρωπαϊκής ένωσης γεωεπιστημών (egu) στη βιέννη, μελέτησαν την μεταβολή των επιπέδων θερμικού στρες στην αθήνα κατά τα τελευταία 50 χρόνια, με βάση μετρήσεις του ιστορικού κλιματικού σταθμού του εαα στο θησείο. την ερευνητική ομάδα αποτελούσαν οι γιώργος καταβούτας, δήμητρα φουντά, κώστας β.βαρώτσος και χρήστος γιαννακόπουλος. χρησιμοποιήθηκαν δύο βιοκλιματικοί δείκτες, ο heat index (hi) και ο humidex (hd), οι οποίοι λαμβάνουν υπόψη τη συνδυασμένη επίδραση της θερμοκρασίας και της υγρασίας στα επίπεδα θερμικής άνεσης ή δυσφορίας στον ανθρώπινο οργανισμό. οι δείκτες αυτοί είναι ευρέως διαδεδομένοι και χρησιμοποιούνται από πολλές υπηρεσίες και οργανισμούς παγκοσμίως. σύμφωνα με την έρευνα, η συχνότητα εμφάνισης έντονου θερμικού στρες ("hot-extreme caution" σύμφωνα με την κλίμακα του hi ή "great discomfort- avoid exertion" σύμφωνα με την κλίμακα του hd) είναι στην αθήνα δύο έως τρεις φορές μεγαλύτερη από τα τέλη της δεκαετίας του 1990 και μετά, σε σχέση με τις προηγούμενες δεκαετίες. η έρευνα αποκαλύπτει επίσης τη σταδιακή αύξηση της διάρκειας της περιόδου (εποχής) της θερμικής δυσφορίας, που παρατηρείται κατά τα τελευταία 50 χρόνια. οι ημέρες έντονης δυσφορίας (great discomfort-hd index) ξεκινούν σταδιακά νωρίτερα και επεκτείνονται αργότερα μέσα στο χρόνο, με αποτέλεσμα η χρονική περίοδος της θερμικής δυσφορίας κατά τις δύο τελευταίες δεκαετίες να είναι πλέον διπλάσια σε σχέση με την αντίστοιχη στη δεκαετία του 1970. όπως δήλωσε η δ. φουντά στο αθηναϊκό και μακεδονικό πρακτορείο ειδήσεων, "είναι εντυπωσιακό ότι ακόμα και κατά τις νυχτερινές ώρες η συχνότητα εμφάνισης έντονης ή μέτριας θερμικής δυσφορίας έχει τριπλασιαστεί κατά τις τελευταίες δύο δεκαετίες. αυτό αυξάνει ακόμα περισσότερο τον θερμικό κίνδυνο, αφού κατά τις νυχτερινές ώρες ο ανθρώπινος οργανισμός δεν μπορεί εύκολα να "ανακάμψει" από το θερμικό στρες της ημέρας". εκτός από τις παρατηρούμενες τάσεις κατά τις τελευταίες δεκαετίες, η έρευνα πραγματοποίησε και μια προσομοίωση του μελλοντικού κλίματος στην αθήνα, με τη χρήση κατάλληλων κλιματικών μοντέλων προσαρμοσμένων στην περιοχή. οι προσομοιώσεις έδειξαν ένα υπερδιπλασιασμό της συχνότητας εμφάνισης έντονης δυσφορίας κατά το τέλος του 21ου αιώνα (2071-2100), σε σχέση με την περίοδο αναφοράς (1971-2000). όλα τα αποτελέσματα επιβεβαιώθηκαν και από τους δύο βιοκλιματικούς δείκτες που χρησιμοποιήθηκαν. η κ. φουντά τόνισε ότι "η έκθεση του πληθυσμού σε αυξημένο θερμικό κίνδυνο αποτελεί αυτή τη στιγμή μια από τις σημαντικότερες απειλές για την ανθρώπινη υγεία παγκοσμίως. οι κάτοικοι των πόλεων είναι πιο ευπαθείς, λόγω της αθροιστικής επίδρασης του φαινομένου της αστικής θερμικής νησίδας. η παγκόσμια θέρμανση, σε συνδυασμό με την αυξανόμενη αστικοποίηση, αλλά και τη "γήρανση" του πληθυσμού -ιδιαίτερα στις ευρωπαϊκές χώρες- εκτοξεύουν τον θερμικό κίνδυνο. είναι σημαντικό να εκτιμήσουμε, να κατανοήσουμε και να επισημάνουμε τον κίνδυνο αυτό, προκειμένου να ληφθούν κατάλληλα μέτρα από την πολιτεία για τον μετριασμό και την έγκαιρη και αποτελεσματική αντιμετώπιση του κινδύνου". η έρευνα πραγματοποιήθηκε στο πλαίσιο του προγράμματος "θεσπια 2" (θεμελίωση συνεργιστικών και ολοκληρωμένων μεθοδολογιών και εργαλείων παρακολούθησης, διαχείρισης και πρόγνωσης περιβαλλοντικών παραμέτρων και πιέσεων"), που εντάσσεται στη "δράση στρατηγικής ανάπτυξης ερευνητικών και τεχνολογικών φορέων" και χρηματοδοτείται από το επιχειρησιακό πρόγραμμα "ανταγωνιστικότητα, επιχειρηματικότητα και καινοτομία" στο πλαίσιο του εσπα 2014-2020, με τη συγχρηματοδότηση της ελλάδας και της ευρωπαϊκής ένωσης (ευρωπαϊκό ταμείο περιφερειακής ανάπτυξης). & 598 & very low & Low & Socio-Economic & NA & NA & 2019-04-09 & 2019 & 3 & ECO
Frame & v.low & National & 500-1000 & 1.8924259 & 1.3590316 & -1.2391092 & 0.9814602 & -0.7922579 & 0.0 & -0.9049211 & -1.0736569 & Recipient & Domestic & Domestic & Domestic & Domestic|ECO & Positive\\
\addlinespace
Greece & http://www.kathimerini.gr/830524/article/epikairothta/politikh/synergates-venizeloy-agnoia-kai-akatasxeth-dhmagwgikh-roph-apo-tsipra---kammeno & 331 & kathimerini.gr & Private/Non-Public & Online and Offline & National & very low = CP mentioned once & Political leverage & Negative & EU + National & No myth & NA & NA & NA & NA & NA & NA & NA & NA & Greece & συνεργάτες βενιζέλου: αγνοια και ακατάσχετη δημαγωγική ροπή από τσίπρα - καμμένο & 2015-09-11 & διαρθρωτικά ταμεία & "το αγγλικό δίκαιο είναι το διεθνώς αποδεκτό και σταθερό δίκαιο των χρηματοοικονομικών συναλλαγών, δηλαδή ένα ουδέτερο δίκαιο που δεν μπορεί να επηρεάσει ούτε ο δανειστής (ευρωζώνη) ούτε ο οφειλέτης (ελλάδα), ενώ το ευρωπαϊκό δίκαιο είναι το δίκαιο που διαμορφώνει μονομερώς ο δανειστής", αναφέρουν συνεργάτες του ευάγγελου βενιζέλου με εμπειρία στις νομικές και χρηματοοικονομικές πτυχές διεθνών συμβάσεων. παράλληλα, κάνουν λόγο για "άγνοια και ακατάσχετη δημαγωγική ροπή των αυτοκόλλητων κ.κ. τσίπρα και καμμένου". αναλυτικά η ανακοίνωση: "πράγματι το δεύτερο πρόγραμμα και το αντίστοιχο δάνειο είχε συναφθεί με τον efsf που έχει τη μορφή ανώνυμης εταιρείας και διέπεται απο το αγγλικό δίκαιο. σε περίπτωση όμως αναγκαστικής εκτέλεσης ίσχυε το δίκαιο του τόπου της εκτέλεσης, δηλαδή το ελληνικό δίκαιο. "το μνημόνιο τσίπρα και το αντίστοιχο δάνειο έχει συναφθεί με τον esm που είναι διακυβερνητικός οργανισμός των κρατών μελών της ευρωζώνης και διέπεται από το δίκαιο της ευρωπαϊκής ένωσης, η σχετική όμως δανειακή σύμβαση που υπέγραψε ο κ. τσακαλώτος δεν περιέχει πρόβλεψη για το δίκαιο που διέπει τη σύμβαση. υποθέτουν ότι επειδή ο esm διέπεται από το δίκαιο της εε, αυτό ισχύει και για τη δανειακή σύμβαση. η ελλάδα αποδέχθηκε το δίκαιο του δανειστή; μπορούν να μας πουν με την ευκαιρία ποιες συγκεκριμένες ρυθμίσεις διέπουν τη νέα δανειακή σύμβαση και πού έγκειται η βελτίωση; ποια βλαπτική ρύθμιση περιέχει το αγγλικό δίκαιο που βελτιώνει η νέα σύμβαση; "αρκεί να θυμηθούν οι ενδιαφερόμενοι ότι το αγγλικό δίκαιο προβλέπει ρήτρες ομαδικής δράσης (cacs) των κομιστών των κρατικών ομολόγων. τέτοιες ρήτρες δεν περιείχε το ελληνικό δίκαιο μέχρι το 2012 και έπρεπε να εισαχθούν αναδρομικά - κατά τα πρότυπα του αγγλικού δικαίου - για να καταστεί δυνατό (με απόφαση της πλειοψηφίας των ομολογιούχων) το κούρεμα και η μεγάλη αναδιάρθρωση του χρέους που έγινε το 2012 και πρέπει να συμπληρωθεί τώρα στο ίδιο πλαίσιο και όχι με τους ανέφικτους και επικίνδυνους τρόπους που έλεγε ο συριζα. "η άγνοια και η ακατάσχετη δημαγωγική ροπή των αυτοκόλλητων κκ. τσίπρα και καμμένου τους εμποδίζουν να αντιληφθούν ότι το αγγλικό δίκαιο είναι το διεθνώς αποδεκτό και σταθερό δίκαιο των χρηματοοικονομικών συναλλαγών, δηλαδή ένα ουδέτερο δίκαιο που δεν μπορεί να επηρεάσει ούτε ο δανειστής ( ευρωζώνη ) ούτε ο οφειλέτης ( ελλάδα ), ενώ το ευρωπαϊκό δίκαιο είναι το δίκαιο που διαμορφώνει μονομερώς ο δανειστής ! "ήδη στο πλαίσιο του μνημονίου τσίπρα, για την ενδιάμεση χρηματοδότηση που έγινε με κονδύλια που ανήκουν στις χώρες μέλη της εε ( efsm ) που δεν μετέχουν στην ζώνη του ευρώ, χορηγήθηκε εγγύηση της ευρωπαϊκής επιτροπής που με τη σειρά της έλαβε ως εγγύηση δεσμεύσεις επί κονδυλίων που δικαιούται η ελλάδα από τα διαρθρωτικά ταμεία της εε. τέτοια επιτυχία!" & 421 & very low & Low & Power & NA & NA & 2015-09-11 & 2015 & 1 & POL
Frame & v.low & National & <500 & 1.8924259 & 1.3590316 & -1.2391092 & 0.9814602 & -0.7922579 & 0.0 & -0.9049211 & -1.0736569 & Recipient & Domestic & European & Mixed & Domestic|POL & Negative\\
Greece & https://www.altsantiri.gr/ergasia/proslipseis-gia-36-mines-sto-perifereiako-tameio-anaptyxis-attikis/ & 384 & altsantiri & Private/Non-Public & Online only & National & low = CP mentioned more times but NOT important part of story (mainly about others issues) & Jobs & Factual & Subnational & No myth & NA & NA & NA & NA & NA & NA & NA & NA & Greece & προσλήψεις για 36 μήνες στο περιφερειακό ταμείο ανάπτυξης αττικής | altsantiri & 2019-03-20 & ευρωπαϊκό ταμείο περιφερειακής ανάπτυξης & το περιφερειακό ταμείο ανάπτυξης αττικής απευθύνει πρόσκληση εκδήλωσης ενδιαφέροντος για την σύναψη σύμβασης μίσθωσης έργου με το περιφερειακό ταμείο ανάπτυξης αττικής, συνολικά πέντε (5) ατόμων, στο πλαίσιο της πράξης "κέντρο καινοτομίας περιφέρειας αττικής (κε.κ.π.α.)" με κωδικό οπς (mis) 5030984, του άξονα προτεραιότητας 01 "ενίσχυση των μηχανισμών και των επενδύσεων των μμε της περιφέρειας αττικής στην έρευνα και την καινοτομία" του επιχειρησιακού προγράμματος "αττική" 2014 - 2020 που συγχρηματοδοτείται από το ευρωπαϊκό ταμείο περιφερειακής ανάπτυξης (ετπα) συνολικής διάρκειας τριάντα έξη (36) μηνών ως εξής: δύο (2) συμβούλους για την έρευνα και την καινοτομία (με αρμοδιότητες στην δημιουργική οικονομία του κλάδου ris) ένα (1) σύμβουλο για την έρευνα και την καινοτομία (με αρμοδιότητες στην γαλάζια οικονομία του κλάδου ris) ένα (1) σύμβουλο για την έρευνα και την καινοτομία (με αρμοδιότητες στην βιώσιμη οικονομία των αναγκών του κλάδου ris) ένα (1) στέλεχος διοικητικής υποστήριξης οι ενδιαφερόμενοι καλούνται να συμπληρώσουν την αίτηση όπως αυτή περιλαμβάνεται στο παραρτημα v της παρούσας και να την υποβάλουν, είτε αυτοπροσώπως, είτε με άλλο εξουσιοδοτημένο από αυτούς πρόσωπο, εφόσον η εξουσιοδότηση φέρει την υπογραφή τους θεωρημένη από δημόσια αρχή, είτε ταχυδρομικά με συστημένη επιστολή, στα γραφεία της υπηρεσίας μας στην ακόλουθη διεύθυνση: περιφερειακο ταμειο αναπτυξης αττικης, συγγρού 15-17, τ.κ. 11743, αθήνα, υπόψη κ. θεόδωρου μαραιδώνη (τηλ. επικοινωνίας: 213-2063707), με την σήμανση: για την υπ. αριθμ. 1321/18-03-2019 πρόσκληση εκδήλωσης ενδιαφέροντος για σύναψη σύμβασης μίσθωσης έργου, για την θέση με κωδικό ........., στο πλαίσιο της πράξης κέντρο καινοτομίας περιφέρειας αττικής (κε.κ.π.α.)" με κωδικό οπς (mis) 5030984 του ε.π. "αττική 2014-2020". & 261 & low & Low & Socio-Economic & NA & NA & 2019-03-20 & 2019 & 3 & ECO
Frame & low-medium & National & <500 & 1.8924259 & 1.3590316 & -1.2391092 & 0.9814602 & -0.7922579 & 0.0 & -0.9049211 & -1.0736569 & Recipient & Domestic & Domestic & Domestic & Domestic|ECO & Neutral\\
Greece & https://e-thessalia.gr/stis-820-000-evro-synolo-tou-neou-exoplismou-sto-nosokomio-tou-volou-agorazete-aktinologikos-yperichotomografos/ & 353 & e-thessalia.gr & Private/Non-Public & Online only & Regional/Local & medium = CP is important part of story & Infrastructure & Positive & EU + National + Subnational & No myth & NA & NA & NA & NA & NA & NA & NA & NA & Greece & στις 820.000 ευρώ το σύνολο του νέου εξοπλισμού στο νοσοκομείο του βόλου- αγοράζεται ακτινολογικός υπερηχοτομογράφος - e-thessalia.gr & 2017-07-06 & ευρωπαϊκό ταμείο περιφερειακής ανάπτυξης & νέο ακτινολογικό εξοπλισμό αποκτά το γενικό νοσοκομείο βόλου προϋπολογισμού 350.000 €, μετά από έγκριση του περιφερειάρχη θεσσαλίας κ. κώστα αγοραστού. συγκεκριμένα εντάσσεται στο εσπα / π.ε.π. θεσσαλίας 2014-2020, το έργο "προμήθεια ιατροτεχνολογικού και ακτινολογικού εξοπλισμού" προϋπολογισμού 350.000 €, με δικαιούχο το νοσοκομείο βόλου. σημειώνεται ότι πριν λίγες ημέρες είχε ενταχθεί και το έργο "προμήθεια και εγκατάσταση πλήρους ψηφιακής χειρουργικής αίθουσας του γενικού νοσοκομείου βόλου" προϋπολογισμού 470.000 €. όπως ανέφερε ο κ. κ. αγοραστός μέσω του εσπα η αιρετή περιφέρεια σε συνεργασία με την 5η υπε και τις διοικήσεις των νοσοκομείων της θεσσαλίας, επιθυμεί να βελτιώσει τις υποδομές τους, να τα καταστήσει αποδοτικότερα και λειτουργικότερα. εξοπλίζουμε και αναβαθμίζουμε τη δημόσια υγεία, με στόχο την παροχή υψηλού επιπέδου φροντίδας προς το κοινωνικό σύνολο. στεκόμαστε με σεβασμό και ενδιαφέρον στον ασθενή και τις ανάγκες του". μετά την ολοκλήρωση της αξιολόγησης από την ειδική υπηρεσία διαχείρισης ε.π. περιφέρειας θεσσαλίας, υπογράφηκαν από τον περιφερειάρχη, οι αποφάσεις ένταξης των έργων στο π.ε.π. θεσσαλίας 2014 - 2020 με συγχρηματοδότηση από το ευρωπαϊκό ταμείο περιφερειακής ανάπτυξης (ετπα). το έργο "προμήθεια ιατροτεχνολογικού και ακτινολογικού εξοπλισμού" προϋπολογισμού 350.000 €, για το νοσοκομείο βόλου αφορά σε προμήθεια και εγκατάσταση ακτινολογικού εξοπλισμού σε αντικατάσταση του πεπαλαιωμένου και πέραν του ορίου ζωής υπάρχοντος βασικού εξοπλισμού. περιλαμβάνει την προμήθεια ακτινολογικού υπερηχοτομογράφοu με τεχνική ελαστογραφίας, συνοδευόμενο από τέσσερις ηχοβόλες κεφαλές, τροχήλατο, έγχρωμο εκτυπωτή α4 τεχνολογίας laser/phaser. επίσης περιλαμβάνει την αντικατάσταση του παλαιού ακτινογραφικού συστήματος με νέο πλήρως ψηφιακό, με δύο ψηφιακούς ανιχνευτές και επιπλέον σταθμό διάγνωσης εξετάσεων. η εγκατάσταση θα γίνει στην ήδη υπάρχουσα και λειτουργική ακτινολογική αίθουσα, η οποία διαθέτει όλες τις απαραίτητες ηλεκτρομηχανολογικές υποδομές και πληροί τις προδιαγραφές ακτινοπροστασίας. μέσω αυτής θα επιτευχθεί εκσυγχρονισμός των παρεχόμενων διαγνωστικών και απεικονιστικών υπηρεσιών του νοσοκομείου και ανάπτυξη και εφαρμογή νέων και σύγχρονων ιατρικών τεχνικών, προς όφελος των ασθενών και του συστήματος υγείας. το έργο "προμήθεια και εγκατάσταση πλήρους ψηφιακής χειρουργικής αίθουσας του γενικού νοσοκομείου βόλου" προϋπολογισμού 470.000 € αφορά στην προμήθεια και εγκατάσταση μίας ολοκληρωμένης ψηφιακής χειρουργικής αίθουσας, high definition με κεντρικό έλεγχο του ενδοσκοπικού και του περιφερειακού εξοπλισμού που χρησιμοποιείται κατά την διάρκεια μίας επέμβασης. ο χειρισμός του εξοπλισμού αυτού αλλά και ολοκλήρου του συστήματος γίνεται ψηφιακά εντός της αίθουσας, μέσω δυο οθονών (μια εντός του χειρουργικού πεδίου και μια εκτός) και οι παράμετροι των συσκευών έχουν ρεαλιστική ή αριθμητική απεικόνιση στις οθόνες αφής. τόσο ο εν λόγω εξοπλισμός όσο και οι δύο οθόνες απεικόνισης θα είναι τοποθετημένες σε ειδικούς βραχίονες αναρτημένους από την οροφή της χειρουργικής αίθουσας. επιπρόσθετα το σύστημα θα είναι εξοπλισμένο με μία εντοιχισμένη οθόνη απεικόνισης τουλάχιστον 42 ιντσών. θα παρέχονται δυνατότητες αμφίδρομης επικοινωνίας, εικόνας - ήχου, με οποιοδήποτε απομακρυσμένο σημείο επιλεχθεί από τον χρήστη για εφαρμογές τηλεδιάσκεψης και live surgery. με την εκπαίδευση του ιατρικού, νοσηλευτικού και τεχνικού προσωπικού για την υποστήριξη της αδιάλειπτης λειτουργίας της αίθουσας, η άμεση έναρξη χρήσης της, θα δημιουργήσει τις απαραίτητες συνθήκες για την αντιμετώπιση κλασικών αλλά και εξειδικευμένων περιστατικών που πριν δρομολογούνταν σε ειδικά και κεντρικά νοσηλευτικά ιδρύματα, μειώνοντας αισθητά τους χρόνους των επεμβάσεων προς όφελος των ασθενών, και αναβαθμίζοντας το επίπεδο των παρεχόμενων υπηρεσιών του γενικού νοσοκομείου βόλου. & 508 & medium & Medium & Socio-Economic & NA & NA & 2017-07-06 & 2017 & 2 & ECO
Frame & low-medium & Regional & 500-1000 & 1.8924259 & 1.3590316 & -1.2391092 & 0.9814602 & -0.7922579 & 0.0 & -0.9049211 & -1.0736569 & Recipient & Domestic & European & Mixed & Domestic|ECO & Positive\\
Greece & http://newpost.gr/ellada/702670/epistrefei-to-treno-sth-dytikh-ellada & 310 & Newpost.gr & Private/Non-Public & Online only & National & medium = CP is important part of story & Infrastructure & Positive & EU + Subnational & No myth & NA & NA & NA & NA & NA & NA & NA & NA & Greece & επιστρέφει το τρένο στη δυτική ελλάδα & 2018-11-03 & ευρωπαϊκό ταμείο περιφερειακής ανάπτυξης & δημοσίευση πριν από 27' / ανανεώθηκε πριν από 26' επτά χρόνια μετά το τελευταίο δρομολόγιο του τρένου από την πάτρα προς τον πύργο έχουν δημιουργηθεί πλέον οι προϋποθέσεις για την επαναλειτουργία της γραμμής σε πρώτο χρόνο μέχρι την κάτω αχαΐα και στη συνέχεια μέχρι την πρωτεύουσα της ηλείας. ειδικότερα, όσον αφορά στην επαναλειτουργία της σιδηροδρομικής σύνδεσης της πάτρας με τον πύργο, η εργοσε ετοιμάζει τις μελέτες, οι οποίες αναμένεται να έχουν ολοκληρωθεί μέχρι το τέλος του χρόνου, ώστε να ενταχθεί στην τρέχουσα προγραμματική περίοδο. μάλιστα σε αυτές τις μελέτες περιλαμβάνεται και η αναβάθμιση των σταθμών στους οποίους θα σταματά το τρένο. όπως ανέφερε στο αθηναϊκό - μακεδονικό πρακτορείο ειδήσεων ο περιφερειάρχης δυτικής ελλάδας απόστολος κατσιφάρας, "η επαναλειτουργία της σιδηροδρομικής σύνδεσης πάτρα - πύργος είναι πολιτική και αναπτυξιακή προτεραιότητα", ενώ όσον αφορά στην χρηματοδότηση του έργου, είπε ότι "η περιφέρεια θα συμμετέχει με 50\% από τους πόρους του εσπα, του περιφερειακού προγράμματος, γιατί ουσιαστικά θέλουμε να έχουμε έναν προαστιακό σιδηρόδρομο από την πάτρα μέχρι τον πύργο." μάλιστα, όπως τόνισε σε αυτό το σημείο, "η περιφέρεια δεν ζητά μόνο, αλλά συμμετέχει με πόρους." επίσης, ο περιφερειάρχης είπε στο απε - μπε ότι "θέλουμε να αναβαθμίσουμε όλους τους ιστορικούς σταθμούς του τρένου, ή κατά το δυνατόν τους περισσότερους, έτσι ώστε να δώσουμε και μία τουριστική προοπτική." σχετικά με τον προαστιακό σιδηρόδρομο από την πάτρα μέχρι την κάτω αχαΐα, το έργο, σύμφωνα με την περιφέρεια, αναμένεται να παραδοθεί το πρώτο τετράμηνο του 2019, ενώ έχει ήδη ολοκληρωθεί η εργολαβία που αφορά τα τεχνικά μέρη της αναβάθμισης της σιδηροδρομικής γραμμής, αποβάθρων και τεχνικών, στη μετρική γραμμή του προαστιακού σιδηροδρόμου. επίσης, προβλέπεται η ανακατασκευή ορισμένων ισόπεδων διαβάσεων, όπως και η κατασκευή και διαμόρφωση οκτώ στάσεων, αφού θα εξυπηρετούνται και οι κάτοικοι των νότιων συνοικιών της πάτρας. μάλιστα, σε κάθε στάση προβλέπεται η διαμόρφωση αποβάθρας, συμπεριλαμβανομένης της κατασκευής κεκλιμένων διαδρόμων πρόσβασης των ατόμων με ειδικές ανάγκες και κινητικά προβλήματα, της τοποθέτησης στεγάστρων, εξοπλισμού καθώς και ηλεκτροφωτισμού. ακόμη, θα εγκατασταθούν νέα αυτόματα συστήματα ισόπεδων διαβάσεων και θα αποκατασταθεί η λειτουργία των σημερινών. παράλληλα, αναμένεται να κατατεθεί και πρόταση, εντός του επόμενου διμήνου για την αναβάθμιση τεσσάρων σταθμών, δηλαδή του αγίου ανδρέα της πάτρας, του μιντιλογλίου, των καμινίων και της κάτω αχαΐας. όσον αφορά τη χρονική διάρκεια της διαδρομής από την πάτρα μέχρι την κάτω αχαΐα, υπολογίζεται ότι θα διαρκεί περίπου μισή ώρα. ακόμη ένα σιδηροδρομικό έργο στην δυτική ελλάδα είναι η αναβάθμιση της γραμμής κατάκολο - πύργος - αρχαία ολυμπία. η περιφέρεια δυτικής ελλάδας έχει ήδη εκδώσει πρόσκληση για ένταξη του έργου στο περιφερειακό επιχειρησιακό πρόγραμμα εσπα 2014 - 2020, προϋπολογισμού 5.000.000 ευρώ, το οποίο εντάσσεται στην ολοκληρωμένη χωρική επένδυση κατάκολο - πύργος - αρχαία ολυμπία. μάλιστα για το συγκεκριμένο έργο πρόκειται να κατατεθούν τρεις προτάσεις. η πρώτη, θα υποβληθεί από τον οσε μέχρι 15 νοεμβρίου και θα αφορά την ανακαίνιση - εκσυγχρονισμό των γραμμών και την ασφάλεια του δικτύου. η δεύτερη θα υποβληθεί από τη γαιαοσε και θα αφορά την ανακατασκευή σταθμών. η τρίτη θα αφορά το οπτικοακουστικό, πληροφοριακό υλικό και θα υποβληθεί επίσης από τη γαιαοσε, άμεσα. ακόμη, στο πλαίσιο υλοποίησης του έργου προβλέπονται ανακατασκευές στη σιδηροδρομική υποδομή δηλαδή γραμμή και αφύλακτες διαβάσεις, ανακατασκευή κτιρίων του σταθμού πύργου και στον περιβάλλοντα χώρο και στον σταθμό της ολυμπίας. τέλος, να σημειωθεί ότι το έργο συγχρηματοδοτείται από το ευρωπαϊκό ταμείο περιφερειακής ανάπτυξης. & 536 & medium & Medium & Socio-Economic & NA & NA & 2018-11-03 & 2018 & 3 & ECO
Frame & low-medium & National & 500-1000 & 1.8924259 & 1.3590316 & -1.2391092 & 0.9814602 & -0.7922579 & 0.0 & -0.9049211 & -1.0736569 & Recipient & Domestic & European & Mixed & Domestic|ECO & Positive\\
Greece & https://www.zougla.gr/page.ashx?pid=2\&cid=0\&aid=1696893 & 367 & zougla.gr & Private/Non-Public & Online only & National & low = CP mentioned more times but NOT important part of story (mainly about others issues) & Social awareness/inclusion & Positive & EU + Subnational & No myth & NA & NA & NA & NA & NA & NA & NA & NA & Greece & παρουσιάστηκε το πρόγραμμα του δήμου αθηναίων για τη συνύπαρξη κατοίκων και προσφύγων στις γειτονιές του & 2019-04-15 & ευρωπαϊκό ταμείο περιφερειακής ανάπτυξης & το πιλοτικό ευρωπαϊκό πρόγραμμα ένταξης του δήμου αθηναίων, για τους πρόσφυγες, το curing the limbo, παρουσιάστηκε σήμερα σε ειδική εκδήλωση στο "σεράφειο".το curing the limbο είναι ένα ευρωπαϊκό πιλοτικό πρόγραμμα του δήμου αθηναίων, που συγχρηματοδοτείται από το ευρωπαϊκό ταμείο περιφερειακής ανάπτυξης μέσω της πρωτοβουλίας αστικές καινοτόμες δράσεις (uia) και απευθύνεται σε αναγνωρισμένους ενήλικες πρόσφυγες που έχουν αποκτήσει άσυλο στην ελλάδα μετά το 2015 και μιλούν ελληνικά, αγγλικά, αραβικά, φαρσί ή γαλλικά. στο πρόγραμμα, οι πρόσφυγες παρακολουθούν μαθήματα ελληνικών, αγγλικών, ηλεκτρονικών υπολογιστών και σεμινάρια οπτικοακουστικής έκφρασης στο σεράφειο και στις γειτονιές της πόλης, συμμετέχουν σε συνεδρίες εξατομικευμένης επαγγελματικής συμβουλευτικής, ενισχύουν την πρόσβασή τους στην αγορά εργασίας, ενημερώνονται για τη διαδικασία ενοικίασης ακινήτων στην αθήνα και αποκτούν τη δυνατότητα αίτησης για οικονομικά προσιτή κατοικία. σε ανακοίνωση του δήμου, τονίζεται ότι "με το πρόγραμμα οι πρόσφυγες γίνονται μέρος της πόλης. διασυνδέονται με τις ομάδες ενεργών πολιτών του συναθηνά και συμμετέχουν σε αθλητικές, εκπαιδευτικές, ψυχαγωγικές, περιβαλλοντικές και καλλιτεχνικές δράσεις. μαζί λοιπόν, η πόλη, οι παλιοί και οι νέοι κάτοικοι ενεργοποιούνται, σχεδιάζουν, συνεργάζονται και συνυπάρχουν". επίσης σημειώνεται ότι "η δημιουργία ενός δυναμικού, καινοτόμου και βιώσιμου μοντέλου ένταξης, καθιστά την αθήνα πρωτοπόρο, σε ευρωπαϊκό επίπεδο, στον πυλώνα της κοινωνικής καινοτομίας και του νέου τρόπου αντιμετώπισης σύγχρονων προβλημάτων στις πόλεις". ο δήμαρχος αθηναίων, γιώργος καμίνης, επισήμανε ότι ο δήμος διαχειρίστηκε το προσφυγικό, "όχι ως ένα συγκυριακό ζήτημα, αλλά ως ένα φαινόμενο που θα απασχολήσει για χρόνια την ελλάδα και την ευρώπη. γι' αυτό και με απόλυτο σεβασμό στα ανθρώπινα δικαιώματα και τις διεθνείς υποχρεώσεις, διαφυλάξαμε την κανονικότητα της πόλης και την αρμονική συνύπαρξη κατοίκων και προσφύγων. είμαστε ο πρώτος φορέας τοπικής αυτοδιοίκησης με σχέδιο διαχείρισης του προσφυγικού", δήλωσε. στην εκδήλωση τοποθετήθηκαν εκ μέρους των εταίρων του προγράμματος, η αμαλία ζέπου, αντιδήμαρχος κοινωνίας των πολιτών και καινοτομίας, ο λευτέρης παπαγιαννάκης, αντιδήμαρχος μεταναστών και προσφύγων, η θάλεια δραγώνα, επικεφαλής της ομάδας του πανεπιστημίου αθηνών, ο joshua kyller country manager της catholic relief services (crs), η jana frey country director της international rescue committee (irc) και η μαρία λογοθέτη, διευθυντήρια του γραφείου δημάρχου αθηναίων και πρόεδρος της εταιρείας ανάπτυξης και τουριστικής προβολής αθηνών. περισσότερες πληροφορίες σχετικά με το πρόγραμμα, δίνονται στην ηλεκτρονική διεύθυνση curingthelimbo.gr & 356 & low & Low & Socio-Economic & NA & NA & 2019-04-15 & 2019 & 3 & ECO
Frame & low-medium & National & <500 & 1.8924259 & 1.3590316 & -1.2391092 & 0.9814602 & -0.7922579 & 0.0 & -0.9049211 & -1.0736569 & Recipient & Domestic & European & Mixed & Domestic|ECO & Positive\\
\addlinespace
Greece & http://www.news.gr/kosmos/evroph/article/271384/ntomprovskis-isos-pagosoyn-ta-diarthrotika-kefalai.html & 336 & News.gr & Private/Non-Public & Online only & National & high = CP is most important issue in story (can also cover other issues) & Political leverage & Factual & EU & No myth & NA & NA & NA & NA & NA & NA & NA & NA & Greece & ντομπρόβσκις: ίσως παγώσουν τα διαρθρωτικά κεφάλαια για την ισπανία και την πορτογαλία - news.gr & 2016-07-03 & διαρθρωτικά ταμεία & ο αντιπρόεδρος της ευρωπαϊκής επιτροπής αρμόδιος για το ευρώ, ο βάλντις ντομπρόβσκις, διεμήνυσε ότι η πρόσβαση στα διαρθρωτικά ταμεία της εε μπορεί να παγώσει για την ισπανία και την πορτογαλία λόγω της απόκλισης από τους δημοσιονομικούς στόχους τους το 2015, σε συνέντευξή του που δημοσιεύθηκε χθες. "η ισπανία και η πορτογαλία δεν εκπλήρωσαν τους συμφωνημένους δημοσιονομικούς στόχους. αυτό είναι αναμφισβήτητο", είπε ο ντομπρόβσκις σε συνέντευξή του που δημοσιεύθηκε χθες στο εβδομαδιαίο γερμανικό ειδησεογραφικό περιοδικό der spiegel. το ζήτημα αυτό θα βρεθεί "πολύ σύντομα" στην ημερήσια διάταξη στις βρυξέλλες, υπογράμμισε ο ντομπρόβσκις. "εάν η επιτροπή και το συμβούλιο διαπιστώσουν ότι η ισπανία και πορτογαλία απέτυχαν να εκπληρώσουν τους στόχους τους, η επιτροπή θα προτείνει μεταξύ άλλων σε ποιο βαθμό θα παγώσουν τα διαρθρωτικά κεφάλαια" για τις δύο χώρες. "οι δύο χώρες δεν διόρθωσαν έγκαιρα τα ελλείμματά τους, θα λάβουμε κατά συνέπεια τις αναγκαίες αποφάσεις. αυτή η απόφαση επαφίεται πάντως στο σύνολο του κολεγίου των επιτρόπων. δεν θέλω να προδικάσω" το αποτέλεσμα των συζητήσεων, πρόσθεσε ο ντομπρόβσκις. οι επίτροποι της εε συναντώνται την τρίτη για να συζητήσουν σχετικά και αναμένεται να λάβουν μια απόφαση με συναίνεση. οι κυρώσεις σε περίπτωση αθέτησης των δημοσιονομικών στόχων συμπεριλαμβάνουν την επιβολή προστίμου ίσου με το 0,2\% του αεπ. εάν ληφθεί μια τέτοια απόφαση, η πορτογαλία και η ισπανία θα είναι οι πρώτες χώρες μέλη της ευρωζώνης στις οποίες θα επιβληθεί η καταβολή προστίμου. το 2015, το ισπανικό δημόσιο έλλειμμα ανήλθε στο 5\% του αεπ, πολύ πάνω από το όριο του συμφώνου σταθερότητας (3\%) και τους στόχους που είχε ορίσει η κομισιόν (4,2\%). η πορτογαλία κατέγραψε έλλειμμα 4,4\% του αεπ την περασμένη χρονιά ενώ ο καθορισμένος στόχος ήταν να το μειώσει κάτω από το 3\%. ενώ οι βρυξέλλες αποφάσισαν τον μάιο να αναβάλουν για την 5η ιουλίου την απόφασή τους για την επιβολή κυρώσεων ή μη στη λισαβόνα και τη μαδρίτη, η ευρωπαϊκή επιτροπή παραμένει διχασμένη για το ζήτημα, σύμφωνα με ενημερωμένες πηγές. η απόφαση της κομισιόν αναμένεται κατόπιν να εγκριθεί από τους υπουργούς οικονομικών στο συμβούλιο τους στις βρυξέλλες που αναμένεται να συγκληθεί την 12η ιουλίου. όπως και ο ευρωπαίος επίτροπος οικονομικών και δημοσιονομικών υποθέσεων πιερ μοσκοβισί στις αρχές του ιουνίου, ο ντομπρόβσκις δήλωσε εξάλλου ότι η γαλλία πρέπει να εκπληρώσει απόλυτα τους στόχους της το 2017. "σε κάθε περίπτωση, η γαλλία πρέπει να επιτύχει τους δημοσιονομικούς της στόχους το επόμενο έτος", είπε. παρά τις υποσχέσεις της, η γαλλία δεν έχει ακόμη καταφέρει να μειώσει κάτω από 3\% του αεπ το δημοσιονομικό έλλειμμά της και έχει εξασφαλίσει πολλές περιόδους χάριτος, με το 2017 να θεωρείται η καταληκτική προθεσμία. & 425 & high & High & Power & NA & NA & 2016-07-03 & 2016 & 2 & POL
Frame & high-very high & National & <500 & 1.8924259 & 1.3590316 & -1.2391092 & 0.9814602 & -0.7922579 & 0.0 & -0.9049211 & -1.0736569 & Recipient & European & European & European & European|POL & Neutral\\
Greece & http://www.newsbeast.gr/financial/arthro/2809601/makron-to-ikonomiko-ntampingk-bori-na-odigisi-se-dialisi-tis-e-e & 337 & Newsbeast.gr & Private/Non-Public & Online only & National & very low = CP mentioned once & Territorial cooperation & Negative & EU + Other country & 3.Firms/jobs relocate to poor countries & NA & NA & NA & NA & NA & NA & NA & NA & Greece & μακρόν: το οικονομικό ντάμπινγκ μπορεί να οδηγήσει σε διάλυση της ε.ε. & 2017-08-24 & διαρθρωτικά ταμεία & "καμιά κοινή γνώμη στις αναπτυγμένες χώρες δεν θα δεχτεί την απόσπαση αλλοδαπών εργαζομένων" αν δεν υπάρξει κάποια τροποποίηση της ευρωπαϊκής οδηγίας που αφορά την απόσπαση αλλοδαπών εργαζομένων, το "οικονομικό ντάμπινγκ" που εφαρμόζουν ορισμένες χώρες-μέλη μπορεί να οδηγήσει στην "καταστροφή της εε", εκτίμησε σήμερα ο πρόεδρος της γαλλίας εμανουέλ μακρόν, ο οποίος πραγματοποιεί επίσημη επίσκεψη στη ρουμανία. "ορισμένοι πολιτικοί ή επιχειρηματικοί κύκλοι" στην ευρώπη θέλουν "να πλήξουν τα διαρθρωτικά ταμεία και να αναπτύξουν ένα μοντέλο φορολογικού και κοινωνικού ντάμπινγκ" είπε ο γάλλος πρόεδρος σε συνέντευξη τύπου που παραχώρησε στο βουκουρέστι. αν δεν γίνει κάτι, αυτό θα οδηγήσει "στη διάλυση της ευρωπαϊκής ένωσης" καθώς "καμιά κοινή γνώμη στις αναπτυγμένες χώρες δεν θα δεχτεί το σύστημα αυτό", πρόσθεσε. ο μακρόν, σύμφωνα με δημοσίευμα του reuters που αναμεταδίδει το αθηναϊκό πρακτορείο ειδήσεων, συναντήθηκε σήμερα στο βουκουρέστι με τον πρόεδρο της ρουμανίας κλάους ιοάνις. στο επίκεντρο της συνάντησης, αλλά και όλης της περιοδείας του γάλλου προέδρου σε χώρες της ανατολικής ευρώπης, βρέθηκε και πάλι το θέμα της απόσπασης εργαζομένων. η ευρωπαϊκή ένωση προσπαθεί το τελευταίο διάστημα να γεφυρώσει τις διαφορές μεταξύ των χωρών της ανατολικής και της δυτικής ευρώπης σχετικά με την πρακτική αυτή που επιτρέπει σε εταιρείες οι οποίες εδρεύουν σε χώρες όπου οι μισθοί είναι χαμηλοί να μεταφέρουν εργαζομένους τους σε άλλα κράτη. όπως υποσχέθηκε και στην προεκλογική εκστρατεία του, ο μακρόν είπε ότι ελπίζει μέχρι τα τέλη οκτωβρίου να βρεθεί μια συμβιβαστική λύση στο θέμα αυτό, ώστε να περιοριστεί το διάστημα της απόσπασης των αλλοδαπών εργαζομένων και να ενισχυθούν οι έλεγχοι για να αποτραπούν οι απάτες. χθες, ο μακρόν συναντήθηκε με την πολιτική ηγεσία της αυστρίας, της τσεχίας και της σλοβακίας. η βιένη στηρίζει τη γαλλική θέση, όμως η τσεχία και η σλοβακία δέχτηκαν μεν να αναζητήσουν μια συμβιβαστική λύση, χωρίς ωστόσο να δεσμευτούν επί της ουσίας. επιφυλακτικός εμφανίστηκε και ο πρόεδρος της ρουμανίας, ο οποίος παραδέχτηκε τη "δυσαρέσκεια" της δυτικής ευρώπης για το θέμα αυτό και εκτίμησε ότι θα πρέπει "να βελτιωθεί η οδηγία" αλλά "χωρίς να εξαλειφθεί ο ανταγωνισμός ή η ελεύθερη αγορά". & 334 & very low & Low & Socio-Economic & NA & NA & 2017-08-24 & 2017 & 2 & ECO
Frame & v.low & National & <500 & 1.8924259 & 1.3590316 & -1.2391092 & 0.9814602 & -0.7922579 & 0.0 & -0.9049211 & -1.0736569 & Recipient & European & European & European & European|ECO & Negative\\
Greece & https://www.in.gr/2019/04/01/economy/oikonomikes-eidiseis/eurobank-symfonia-etean-gia-xrimatodotisi-mikromesaion-epixeiriseon/ & 373 & in.gr & Private/Non-Public & Online only & National & very low = CP mentioned once & Economic development & Factual & EU + National & No myth & NA & NA & NA & NA & NA & NA & NA & NA & Greece & eurobank: συμφωνία με ετεαν για χρηματοδότηση μικρομεσαίων επιχειρήσεων | in.gr & 2019-04-01 & ευρωπαϊκό ταμείο περιφερειακής ανάπτυξης & η eurobank, στο πλαίσιο του ολοκληρωμένου προγράμματός της για τη στήριξη της ελληνικής επιχειρηματικότητας, υπέγραψε συμφωνία με το εθνικό ταμείο επιχειρηματικότητας \& ανάπτυξης (ετεαν αε) για τη χορήγηση χρηματοδοτήσεων με ευνοϊκούς όρους σε πολύ μικρές, μικρές και μεσαίες επιχειρήσεις μέσω της δράσης "επιχειρηματική χρηματοδότηση -τεπιχ ii". σε συνέχεια της συμφωνίας eurobank - ετεαν και της ενεργοποίησης του ταμείου επιχειρηματικότητας ιι (τεπιχ ιι), θα διατεθούν περίπου 170 εκατ. ευρώ σε μικρομεσαίες επιχειρήσεις με ευνοϊκό επιτόκιο, καθώς η συμμετοχή του τεπιχ ii ανέρχεται στο 40\% της κάθε χρηματοδότησης και είναι άτοκη. σύμφωνα με τη σχετική ανακοίνωση, το νέο εργαλείο χρηματοοικονομικής τεχνικής διευκολύνει σημαντικά την πρόσβαση στη χρηματοδότηση των μικρομεσαίων επιχειρήσεων και συμβάλλει ουσιαστικά στην ενίσχυση της επιχειρηματικότητας της χώρας. μέσω της νέας συμφωνίας με το ετεαν και της συμμετοχής της στο τεπιχ ιι, η eurobank παρέχει δύο μορφές χρηματοδότησης με ευνοϊκούς όρους για την χρηματοδότηση επενδυτικών σχεδίων και την χρηματοδότηση κεφαλαίου κινήσεως καλύπτοντας μεγάλο εύρος επιχειρηματικών αναγκών για την ανάπτυξη αλλά και την διευκόλυνση του εμπορικού/συναλλακτικού κυκλώματος της κάθε επιλέξιμης επιχείρησης. πρόκειται για: * δάνεια επενδυτικού σκοπού στα οποία παρέχεται ποσό χρηματοδότησης έως και 1.500.000 ευρώ, * δάνεια επιχειρηματικής ανάπτυξης (κεφάλαια κίνησης) όπου παρέχεται ποσό έως και 500.000 ευρώ. επιπλέον, δίνεται η δυνατότητα χρηματοδότησης και σε επενδυτικά σχέδια που έχουν υποβληθεί/εγκριθεί στο πλαίσιο των δράσεων του εσπα 2014-2020 για την κάλυψη του μέρους της ιδιωτικής συμμετοχής μέσω μακροπρόθεσμο δανείου . τα παρεχόμενα δάνεια μπορούν να χορηγηθούν έως την εξάντληση του διαθέσιμου προϋπολογισμού και όχι πέραν της 31.10.2023. οι πόροι του ταμείου επιχειρηματικότητας ιι συγχρηματοδοτούνται από το ευρωπαϊκό ταμείο περιφερειακής ανάπτυξης (ετπα) και το ελληνικό δημόσιο και προέρχονται από την συνεισφορά πόρων από το επιχειρησιακό πρόγραμμα ανταγωνιστικότητα, επιχειρηματικότητα και καινοτομία (επανεκ) και τα περιφερειακά επιχειρησιακά προγράμματα (πεπ). η υποβολή αιτημάτων δανειοδότησης από τις ενδιαφερόμενες επιχειρήσεις γίνεται σε ηλεκτρονικό σύνδεσμοκαι ακολούθως υποχρεούνται να υποβάλλουν το φυσικό φάκελο σε υποκατάστημα της τράπεζας. & 310 & very low & Low & Socio-Economic & NA & NA & 2019-04-01 & 2019 & 3 & ECO
Frame & v.low & National & <500 & 1.8924259 & 1.3590316 & -1.2391092 & 0.9814602 & -0.7922579 & 0.0 & -0.9049211 & -1.0736569 & Recipient & Domestic & European & Mixed & Domestic|ECO & Neutral\\
Greece & https://www.news.gr/oikonomia/article/1403368/to-treno-epistrefi-to-2019-sti-ditiki-ellada-ti-provlepete.html & 371 & News.gr & Private/Non-Public & Online only & National & very low = CP mentioned once & Infrastructure & Positive & EU + National + Subnational & No myth & NA & NA & NA & NA & NA & NA & NA & NA & Greece & το τρένο επιστρέφει το 2019 στη δυτική ελλάδα - τι περιλαμβάνει το νέο έργο & 2018-11-03 & ευρωπαϊκό ταμείο περιφερειακής ανάπτυξης & θέλουμε να αναβαθμίσουμε όλους τους ιστορικούς σταθμούς ώστε να δώσουμε και μία τουριστική προοπτική, τονίζει ο περιφερειάρχης δυτ. ελλάδας, απ. κατσιφάρας επτά χρόνια μετά το τελευταίο δρομολόγιο του τρένου από την πάτρα προς τον πύργο έχουν δημιουργηθεί πλέον οι προϋποθέσεις για την επαναλειτουργία της γραμμής σε πρώτο χρόνο μέχρι την κάτω αχαΐα και στη συνέχεια μέχρι την πρωτεύουσα της ηλείας. ειδικότερα, όσον αφορά στην επαναλειτουργία της σιδηροδρομικής σύνδεσης της πάτρας με τον πύργο, η εργοσε ετοιμάζει τις μελέτες, οι οποίες αναμένεται να έχουν ολοκληρωθεί μέχρι το τέλος του χρόνου, ώστε να ενταχθεί στην τρέχουσα προγραμματική περίοδο. μάλιστα σε αυτές τις μελέτες περιλαμβάνεται και η αναβάθμιση των σταθμών στους οποίους θα σταματά το τρένο. όπως ανέφερε στο αθηναϊκό - μακεδονικό πρακτορείο ειδήσεων ο περιφερειάρχης δυτικής ελλάδας απόστολος κατσιφάρας, η επαναλειτουργία της σιδηροδρομικής σύνδεσης πάτρα - πύργος είναι πολιτική και αναπτυξιακή προτεραιότητα, ενώ όσον αφορά στην χρηματοδότηση του έργου, είπε ότι "η περιφέρεια θα συμμετέχει με 50\% από τους πόρους του εσπα, του περιφερειακού προγράμματος, γιατί ουσιαστικά θέλουμε να έχουμε έναν προαστιακό σιδηρόδρομο από την πάτρα μέχρι τον πύργο". μάλιστα, όπως τόνισε σε αυτό το σημείο, "η περιφέρεια δεν ζητά μόνο, αλλά συμμετέχει με πόρους". "θέλουμε να αναβαθμίσουμε όλους τους ιστορικούς σταθμούς του τρένου, ή κατά το δυνατόν τους περισσότερους, έτσι ώστε να δώσουμε και μία τουριστική προοπτική". σχετικά με τον προαστιακό σιδηρόδρομο από την πάτρα μέχρι την κάτω αχαΐα, το έργο, σύμφωνα με την περιφέρεια, αναμένεται να παραδοθεί το πρώτο τετράμηνο του 2019, ενώ έχει ήδη ολοκληρωθεί η εργολαβία που αφορά τα τεχνικά μέρη της αναβάθμισης της σιδηροδρομικής γραμμής, αποβάθρων και τεχνικών, στη μετρική γραμμή του προαστιακού σιδηροδρόμου. επίσης, προβλέπεται η ανακατασκευή ορισμένων ισόπεδων διαβάσεων, όπως και η κατασκευή και διαμόρφωση οκτώ στάσεων, αφού θα εξυπηρετούνται και οι κάτοικοι των νότιων συνοικιών της πάτρας. μάλιστα, σε κάθε στάση προβλέπεται η διαμόρφωση αποβάθρας, συμπεριλαμβανομένης της κατασκευής κεκλιμένων διαδρόμων πρόσβασης των ατόμων με ειδικές ανάγκες και κινητικά προβλήματα, της τοποθέτησης στεγάστρων, εξοπλισμού καθώς και ηλεκτροφωτισμού. ακόμη, θα εγκατασταθούν νέα αυτόματα συστήματα ισόπεδων διαβάσεων και θα αποκατασταθεί η λειτουργία των σημερινών. παράλληλα, αναμένεται να κατατεθεί και πρόταση, εντός του επόμενου διμήνου για την αναβάθμιση τεσσάρων σταθμών, δηλαδή του αγίου ανδρέα της πάτρας, του μιντιλογλίου, των καμινίων και της κάτω αχαΐας. όσον αφορά τη χρονική διάρκεια της διαδρομής από την πάτρα μέχρι την κάτω αχαΐα, υπολογίζεται ότι θα διαρκεί περίπου μισή ώρα. αναβάθμιση της γραμμής κατάκολο - πύργος - αρχαία ολυμπία ακόμη ένα σιδηροδρομικό έργο στην δυτική ελλάδα είναι η αναβάθμιση της γραμμής κατάκολο - πύργος - αρχαία ολυμπία. η περιφέρεια δυτικής ελλάδας έχει ήδη εκδώσει πρόσκληση για ένταξη του έργου στο περιφερειακό επιχειρησιακό πρόγραμμα εσπα 2014 - 2020, προϋπολογισμού 5.000.000 ευρώ, το οποίο εντάσσεται στην ολοκληρωμένη χωρική επένδυση κατάκολο - πύργος - αρχαία ολυμπία. μάλιστα για το συγκεκριμένο έργο πρόκειται να κατατεθούν τρεις προτάσεις. η πρώτη, θα υποβληθεί από τον οσε μέχρι 15 νοεμβρίου και θα αφορά την ανακαίνιση - εκσυγχρονισμό των γραμμών και την ασφάλεια του δικτύου. η δεύτερη θα υποβληθεί από τη γαιαοσε και θα αφορά την ανακατασκευή σταθμών. η τρίτη θα αφορά το οπτικοακουστικό, πληροφοριακό υλικό και θα υποβληθεί επίσης από τη γαιαοσε, άμεσα. ακόμη, στο πλαίσιο υλοποίησης του έργου προβλέπονται ανακατασκευές στη σιδηροδρομική υποδομή δηλαδή γραμμή και αφύλακτες διαβάσεις, ανακατασκευή κτιρίων του σταθμού πύργου και στον περιβάλλοντα χώρο και στον σταθμό της ολυμπίας. τέλος, να σημειωθεί ότι το έργο συγχρηματοδοτείται από το ευρωπαϊκό ταμείο περιφερειακής ανάπτυξης. σχετικά tags δυτική ελλάδα σιδηρόδρομος & 553 & very low & Low & Socio-Economic & NA & NA & 2018-11-03 & 2018 & 3 & ECO
Frame & v.low & National & 500-1000 & 1.8924259 & 1.3590316 & -1.2391092 & 0.9814602 & -0.7922579 & 0.0 & -0.9049211 & -1.0736569 & Recipient & Domestic & European & Mixed & Domestic|ECO & Positive\\
Greece & https://www.in.gr/2019/02/14/greece/stin-athina-korina-kretsou-gia-tin-proothisi-tis-perifereiakis-politikis/ & 358 & in.gr & Private/Non-Public & Online only & National & very high = CP is most important issue + CP is mentioned in title/headline & Economic development & Positive & EU + National & No myth & NA & NA & NA & NA & NA & NA & NA & NA & Greece & στην αθήνα η κορίνα κρέτσου για την προώθηση της περιφερειακής πολιτικής | in.gr & 2019-02-14 & πολιτική συνοχής & ingr > ελλάδα > στην αθήνα η κορίνα κρέτσου για την προώθηση της περιφερειακής πολιτικής στην αθήνα η κορίνα κρέτσου για την προώθηση της περιφερειακής πολιτικής η επίτροπος κορίνα κρέτσου θα συναντηθεί με τον υπουργό υποδομών, χρήστο σπίρτζη και αύριο με τον υπουργό οικονομίας γιάννη δραγασάκη, καθώς και με τον υφυπουργό στάθη γιαννακίδη. η επίτροπος περιφερειακής πολιτικής κορίνα κρέτσου επισκέπτεται σήμερα, πέμπτη, στην αθήνα "για να συζητήσει τρόπους με τους οποίους η πολιτική συνοχής μπορεί να στηρίξει περαιτέρω την ελληνική οικονομία και την περιφερειακή ανάπτυξη στη χώρα". στο πλαίσιο αυτών των επιδιώξεων, όπως αναφέρει η ανακοίνωση της ευρωπαϊκής επιτροπής, συναντάται σήμερα με τον υπουργό υποδομών, μεταφορών και δικτύων χρήστο σπίρτζη και αύριο με τον υπουργό οικονομίας και ανάπτυξης γιάννη δραγασάκη, καθώς και με τον υφυπουργό οικονομίας και ανάπτυξης στάθη γιαννακίδη. την κυριακή, η κ. κρέτσου θα συμμετάσχει σε διάλογο με τους πολίτες για το μέλλον της ευρώπης στην ολυμπία, μαζί με τον επίτροπο μετανάστευσης, εσωτερικών υποθέσεων και ιθαγένειας, δημήτρη αβραμόπουλο. & 158 & very high & High & Socio-Economic & NA & NA & 2019-02-14 & 2019 & 3 & ECO
Frame & high-very high & National & <500 & 1.8924259 & 1.3590316 & -1.2391092 & 0.9814602 & -0.7922579 & 0.0 & -0.9049211 & -1.0736569 & Recipient & Domestic & European & Mixed & Domestic|ECO & Positive\\
\addlinespace
Greece & http://www.avgi.gr/article/10951/9737106/ee-263-ekat-euro-gia-erga-ypodomes-sten-ellada & 325 & avgi.gr & Private/Non-Public & Online and Offline & National & very high = CP is most important issue + CP is mentioned in title/headline & Infrastructure & Positive & EU & No myth & Territorial cooperation & Positive & EU & No myth & Public services & Positive & EU & No myth & Greece & εε: 263 εκατ. ευρώ για έργα υποδομής στην ελλάδα & 2019-04-02 & περιφερειακή πολιτική & κατασκευή του αγωγού διασύνδεσης φυσικού αερίου ανάμεσα στην κομοτηνή και τη στάρα ζαγόρα (βουλγαρία) - παροχή ενέργειας στην κρήτη - διασύνδεση ηλεκτρικής ενέργειας ανάμεσα στην κρήτη και την πελοπόννησο την επένδυση ευρωπαϊκών κονδυλίων ύψους 4 δισ. ευρώ σε 25 μεγάλα έργα υποδομής σε 10 κράτη μέλη, μεταξύ των οποίων και στην ελλάδα, ανακοίνωσε σήμερα η επίτροπος για την περιφερειακή πολιτική, κορίνα κρέτσου. "τα 25 αυτά έργα αποτελούν ισάριθμα παραδείγματα τρόπων με τους οποίους η εε εργάζεται για να βελτιώσει την καθημερινή ζωή των πολιτών μας, από τη βελτίωση του πόσιμου νερού έως την παροχή ταχύτερων σιδηροδρομικών μεταφορών και τη λειτουργία σύγχρονων νοσοκομείων", σημείωσε η κ. κρέτσου, προσθέτοντας ότι κατά την τρέχουσα δημοσιονομική περίοδο, έχουν εγκριθεί 258 μεγάλα έργα υποδομής αξίας 32 δισ. ευρώ από κονδύλια της εε. "τα έργα αυτά αποτελούν, κατά κάποιον τρόπο, τους πρέσβεις της πολιτικής συνοχής και είμαι υπερήφανη για κάθε ένα από αυτά", κατέληξε η ρουμάνα επίτροπος. συγκεκριμένα για την ελλάδα, η επίτροπος αναφέρθηκε αρχικά σ' ένα διασυνοριακό έργο, την ευρωπαϊκή χρηματοδότηση της κατασκευής του αγωγού διασύνδεσης φυσικού αερίου μήκους 182 χιλιομέτρων ανάμεσα στην κομοτηνή (ελλάδα) και τη στάρα ζαγόρα (βουλγαρία) ύψους 33 εκατ. ευρώ. σύμφωνα με την ανακοίνωση της επιτροπής, ο αγωγός είναι ένα ευρωπαϊκό έργο κοινού ενδιαφέροντος, το οποίο συμβάλλει στην επίτευξη των στόχων της ενεργειακής ένωσης, καθώς τα συστήματα φυσικού αερίου των δύο χωρών θα συνδεθούν για πρώτη φορά, διαφοροποιώντας τις πηγές ενέργειας στην περιοχή και αυξάνοντας την ενεργειακή ασφάλεια. εξάλλου, σημειώνεται ότι χάρη στην αύξηση του ανταγωνισμού στην αγορά φυσικού αερίου, οι καταναλωτές θα απολαμβάνουν χαμηλότερες τιμές. επιπλέον, σε ό,τι αφορά την ελλάδα, η επιτροπή ανακοίνωσε επενδύσεις σε ακόμα δύο μεγάλα έργα υποδομής, που στοχεύουν στην ενίσχυση της αποτελεσματικότητας των δημόσιων υπηρεσιών καθώς και στην βελτίωση της παροχής ενέργειας στην κρήτη. ειδικότερα, σχεδόν 135 εκατ. ευρώ επενδύονται στο τηλεπικοινωνιακό σύστημα "σύζευξις ii", το οποίο, σύμφωνα με την επιτροπή, στο τελικό του στάδιο θα ενώσει ολόκληρο το ελληνικό δίκτυο δημόσιων υπηρεσιών. στο σύστημα θα συνδεθούν επιπλέον 600.000 δημόσιοι υπάλληλοι και 34.000 εγκαταστάσεις, με αποτέλεσμα σημαντικές οικονομίες κλίμακας και παροχή καλύτερων υπηρεσιών στο κοινό. τέλος, με σχεδόν 95 εκατ. ευρώ θα χρηματοδοτηθεί η κατασκευή διασύνδεσης ηλεκτρικής ενέργειας ανάμεσα στην κρήτη και την πελοπόννησο, η οποία περιλαμβάνει δύο υποβρύχια καλώδια μήκους 135 χλμ. το έργο αυτό θα μειώσει το κόστος του ηλεκτρικού ρεύματος στην κρήτη, αντικαθιστώντας τις δαπανηρές πετρελαϊκές μονάδες ηλεκτροπαραγωγής με ηλεκτρική ενέργεια από την ηπειρωτική ελλάδα. εκτός από την ελλάδα, στο επενδυτικό πακέτο περιλαμβάνονται επίσης η βουλγαρία, η τσεχία, η γερμανία, η ουγγαρία, η ιταλία, η μάλτα, η πολωνία, η πορτογαλία και η ρουμανία. συνολικά, τα έργα καλύπτουν ευρύ φάσμα τομέων: την υγεία, τις μεταφορές, την έρευνα, το περιβάλλον και την ενέργεια. μαζί με την εθνική συγχρηματοδότηση, οι επενδύσεις σε αυτά τα έργα ανέρχονται συνολικά σε 8 δισ. ευρώ. & 457 & very high & High & Socio-Economic & Socio-Economic & Socio-Economic & 2019-04-02 & 2019 & 3 & ECO
Frame & high-very high & National & <500 & 1.8924259 & 1.3590316 & -1.2391092 & 0.9814602 & -0.7922579 & 0.0 & -0.9049211 & -1.0736569 & Recipient & European & European & European & European|ECO & Positive\\
Greece & https://www.naftemporiki.gr/finance/story/1423022 & 359 & naftemporiki.gr & Private/Non-Public & Online and Offline & National & very low = CP mentioned once & Infrastructure & Positive & EU + National & No myth & Research \& innovation & Positive & EU + National & No myth & NA & NA & NA & NA & Greece & 31 εκατ. για την πρώτη στην ε.ε. εθνική υποδομή ευφυούς γεωργίας & 2018-12-11 & διαρθρωτικά ταμεία & στην εποχή της ευφυούς γεωργίας εισέρχεται ο εγχώριος πρωτογενής τομέας, καθώς εντός των προσεχών ημερών αναμένεται η προκήρυξη του έργου "ψηφιακής γεωργίας" από το υπουργείο ψηφιακής πολιτικής, τηλεπικοινωνιών και ενημέρωσης και από το υπουργείο αγροτικής ανάπτυξης και τροφίμων, όπως γνωστοποίησαν χθες οι δύο υπουργοί, νίκος παππάς και σταύρος αραχωβίτης, μετά το πέρας της συνάντησης που πραγματοποίησαν με τον ευρωπαίο επίτροπο γεωργίας και αγροτικής ανάπτυξης φιλ χόγκαν. ειδικότερα, η πρώτη φάση του έργου έχει ολοκληρωθεί και μέσω του προγράμματος αναμένεται να τοποθετηθούν 6.500 επίγειοι σταθμοί οι οποίοι θα "παρακολουθούν" τις καλλιέργειες, με τον αρχικό στόχο να αφορά την κάλυψη περισσότερων από 15 εκατ. στρέμματα, της μισής δηλαδή καλλιεργητικής έκτασης της χώρας. ο συνολικός προϋπολογισμός του προγράμματος ανέρχεται στα 31 εκατ. ευρώ και θα ενταχθεί στο πρόγραμμα δημοσίων επενδύσεων. αξίζει να σημειωθεί πως πρόκειται για την πρώτη εθνική υποδομή ευφυούς γεωργίας στην ευρώπη. "η ελλάδα έχει σημειώσει τεράστια πρόοδο στον τομέα της ψηφιοποίησης και θεωρούμε ότι θα αποτελεί ένα μοντέλο βέλτιστης πρακτικής που θα μπορέσουν να ακολουθήσουν και άλλες χώρες" τόνισε ο φιλ χόγκαν. από την πλευρά του ο κ. παππάς επεσήμανε ότι μέσω της εφαρμογής της ευφυούς γεωργίας επιτυγχάνεται μείωση του κόστους από 30\% έως 40\%, αναλόγως με την καλλιέργεια, υπογραμμίζοντας "αυτό από μόνο του είναι ένα ξεχωριστό κίνητρο, συν βεβαίως ότι διαμορφώνονται αλυσίδες αξίας. διότι ένα προϊόν ψηφιακής γεωργίας είναι πάρα πολύ καλό σήμα αξίας για τον καταναλωτή". σε αυτό το σημείο ο επίτροπος σημείωσε ότι "οι υπουργοί μού είπαν πως σύντομα θα υπάρξουν και κάποια καλά νέα όσον αφορά τα χρηματοδοτικά εργαλεία τα οποία πλέον θα είναι στη διάθεση των παραγωγών αλλά και των διαφόρων αγροδιατροφικών επιχειρήσεων. παράλληλα, σε ευρωπαϊκό επίπεδο εξετάζεται η δυνατότητα να δοθούν οικονομικά κίνητρα από το 2020 και μετά. θα υπάρξει λοιπόν ένας συνδυασμός πολλών κονδυλίων: έχουμε τα διαρθρωτικά ταμεία, το ευρωπαϊκό ταμείο επενδύσεων, όλα αυτά λοιπόν θα χρησιμοποιηθούν και στο πρόγραμμα αγροτικής ανάπτυξης για να υπάρχει πρόσβαση σε πιστώσεις". η δεύτερη φάση του προγράμματος, σύμφωνα με όσα ανέφερε ο κ. παππάς θα αφορά "την ανάπτυξη δεξιοτήτων του αγροτικού πληθυσμού και την αξιοποίηση ελληνικών και ευρωπαϊκών κονδυλίων ούτως ώστε νέοι έλληνες γεωπόνοι να μπορέσουν να συμβάλλουν τα μέγιστα, για να γίνουν αυτές οι τεχνολογίες, τεχνολογίες των ανθρώπων". στο μεταξύ, για στήριξη των ελληνικών θέσεων σχετικά με την κοινή αγροτική πολιτική (καπ) τόσο στον προϋπολογισμό όσο και στην εξωτερική σύγκλιση από την ευρωπαϊκή επιτροπή μίλησε ο υπουργός αγροτικής ανάπτυξης, ο οποίος μεταξύ άλλων ανέφερε ότι "νομίζω ότι είμαστε στην ίδια πλευρά, έχουμε κατανόηση και η τελική διαμόρφωση θα είναι προς όφελος των ελλήνων παραγωγών". παράλληλα, ερωτηθείς για το ζήτημα των δικαιωμάτων και τη χορήγηση μιας ενιαίας ενίσχυσης ανά στρέμμα, ο ευρωπαίος επίτροπος ανέφερε ότι "η κατάργηση των δικαιωμάτων είναι ένα ζήτημα που το προτείνω σε όλα τα μέλη. τα δικαιώματα είναι εμπόδιο. η κατάργησή τους θα ενισχύσει ουσιαστικά την προσπάθεια στήριξης των νέων γεωργών". σύμφωνα με τον ίδιο οι έλληνες αγρότες πέρα από τα "εργαλεία" που προβλέπονται στην καπ θα πρέπει, προκειμένου να αυξήσουν τα επίπεδα της παραγωγής και της κερδοφορίας τους, να επικεντρωθούν στο θέμα της "ποιότητας". αναφερόμενος στα προγράμματα προώθησης των αγροτικών προϊόντων ο κ. χόγκαν στάθηκε στη στήριξη της ελληνικής φέτας, σημειώνοντας ότι "τα τελευταία πέντε χρόνια έχουν αυξηθεί 8\% κατ' έτος οι εξαγωγές φέτας από την ελλάδα", προσθέτοντας ότι στις διεθνείς συμφωνίες που έχουμε συνάψει (σ.σ.: ε.ε με τρίτες χώρες) έχουμε προσπαθήσει να εξασφαλίσουμε την πλήρη προστασία της φέτας και αναφέρομαι σε συμφωνίες με την ιαπωνία, το βιετνάμ, το μεξικό. αυτό που απομένει είναι να μπορέσουμε να αδράξουμε και τα οφέλη από την προστασία αυτή". & 586 & very low & Low & Socio-Economic & Socio-Economic & NA & 2018-12-11 & 2018 & 3 & ECO
Frame & v.low & National & 500-1000 & 1.8924259 & 1.3590316 & -1.2391092 & 0.9814602 & -0.7922579 & 0.0 & -0.9049211 & -1.0736569 & Recipient & Domestic & European & Mixed & Domestic|ECO & Positive\\
Greece & https://www.newsit.gr/topikes-eidhseis/kataskeyi-neou-agogou-metaforas-nerou-stous-fournous/2479551/ & 363 & newsit.gr & Private/Non-Public & Online only & National & medium = CP is important part of story & Infrastructure & Factual & EU + Subnational & No myth & NA & NA & NA & NA & NA & NA & NA & NA & Greece & κατασκευή νέου αγωγού μεταφοράς νερού στους φούρνους & 2018-04-11 & ευρωπαϊκό ταμείο περιφερειακής ανάπτυξης & τη σύμβαση υλοποίησης του έργου κατασκευής αγωγού μεταφοράς αφαλατωμένου νερού και λοιπά έργα υποδομών για την ύδρευση του οικισμού φούρνων του δήμου φούρνων κορσεών ικαρίας, υπέγραψε σήμερα η περιφερειάρχης βορείου αιγαίου, χριστιάνα καλογήρου. το έργο προϋπολογισμού περίπου 230.000 ευρώ θα ολοκληρωθεί σε 18 μήνες από σήμερα. χρηματοδοτείται, δε, από το ευρωπαϊκό ταμείο περιφερειακής ανάπτυξης, είναι ενταγμένο στο επιχειρησιακό πρόγραμμα "υποδομές μεταφορών, περιβάλλον και αειφόρος ανάπτυξη 2014-2020" και υλοποιείται με την επίβλεψη της διεύθυνσης τεχνικών έργων της περιφερειακής ενότητας σάμου της περιφέρειας βορείου αιγαίου. πηγή: απε - μπε & 88 & medium & Medium & Socio-Economic & NA & NA & 2018-04-11 & 2018 & 3 & ECO
Frame & low-medium & National & <500 & 1.8924259 & 1.3590316 & -1.2391092 & 0.9814602 & -0.7922579 & 0.0 & -0.9049211 & -1.0736569 & Recipient & Domestic & European & Mixed & Domestic|ECO & Neutral\\
Greece & https://www.newsit.gr/kosmos/elliniko-endiaferon-synedriasi-tou-symvouliou-genikon-ypotheseon-stis-vrykselles/2480524/ & 368 & newsit.gr & Private/Non-Public & Online only & National & very high = CP is most important issue + CP is mentioned in title/headline & Institutional bargaining over funding & Factual & EU & No myth & Ineffective goal achievement & Factual & EU & No myth & NA & NA & NA & NA & Greece & με ελληνικό ενδιαφέρον η συνεδρίαση του συμβουλίου γενικών υποθέσεων στις βρυξέλλες & 2018-04-12 & πολιτική συνοχής & πιο συγκεκριμένα, οι υπουργοί θα ανταλλάξουν απόψεις και θα επικεντρωθούν: - στις επενδυτικές προτεραιότητες της πολιτικής συνοχής αναφορικά με τις περιοχές που θα ενταχθούν και θα καλύπτονται από τη μελλοντική πολιτική, καθώς και τα κριτήρια για την κατανομή των πόρων, θέμα που αφορά ιδιαίτερα την ελλάδα. - στον τρόπο με τον οποίο θα επιταχυνθεί η εφαρμογή της πολιτικής και θα βελτιωθεί η αποτελεσματικότητά της. η συζήτηση αυτή πρόκειται να τροφοδοτήσει με προτάσεις την προετοιμασία, από την ευρωπαϊκή επιτροπή, της δέσμης νομοθετικών μέτρων σχετικά με την πολιτική συνοχής, μέτρα τα οποία θα ισχύσουν μετά το 2020. πηγη: απε-μπε & 96 & very high & High & Power & Socio-Economic & NA & 2018-04-12 & 2018 & 3 & POL
Frame & high-very high & National & <500 & 1.8924259 & 1.3590316 & -1.2391092 & 0.9814602 & -0.7922579 & 0.0 & -0.9049211 & -1.0736569 & Recipient & European & European & European & European|POL & Neutral\\
Greece & https://www.newsit.gr/politikh/stin-athina-i-epitropos-korina-kretsou/2718489/ & 362 & Newsit.gr & Private/Non-Public & Online only & National & very high = CP is most important issue + CP is mentioned in title/headline & Solidarity to poor countries/regions & Positive & EU & No myth & NA & NA & NA & NA & NA & NA & NA & NA & Greece & στην αθήνα η επίτροπος κορίνα κρέτσου - ειδήσεις & 2019-02-14 & πολιτική συνοχής & η επίτροπος περιφερειακής πολιτικής κορίνα κρέτσου επισκέπτεται σήμερα στην αθήνα "για να συζητήσει τρόπους με τους οποίους η πολιτική συνοχής μπορεί να στηρίξει περαιτέρω την ελληνική οικονομία και την περιφερειακή ανάπτυξη στη χώρα". στο πλαίσιο αυτών των επιδιώξεων, όπως αναφέρει η ανακοίνωση της ευρωπαϊκής επιτροπής, συναντάται σήμερα με τον υπουργό υποδομών, μεταφορών και δικτύων χρήστο σπίρτζη και αύριο με τον υπουργό οικονομίας και ανάπτυξης γιάννη δραγασάκη, καθώς και με τον υφυπουργό οικονομίας και ανάπτυξης στάθη γιαννακίδη. την κυριακή, η κ. κρέτσου θα συμμετάσχει σε διάλογο με τους πολίτες για το μέλλον της ευρώπης στην ολυμπία, μαζί με τον επίτροπο μετανάστευσης, εσωτερικών υποθέσεων και ιθαγένειας, δημήτρη αβραμόπουλο. & 106 & very high & High & Values & NA & NA & 2019-02-14 & 2019 & 3 & ECO
Frame & high-very high & National & <500 & 1.8924259 & 1.3590316 & -1.2391092 & 0.9814602 & -0.7922579 & 0.0 & -0.9049211 & -1.0736569 & Recipient & European & European & European & European|ECO & Positive\\
\addlinespace
Italy & http://www.agi.it/cagliari/notizie/fondi\_ue\_con\_approvazione\_po\_fesr\_2014\_2020\_in\_arrivo\_930\_mln-201507150950-cro-rt10024 & 443 & AGI & Private/Non-Public & Online only & National & very high = CP is most important issue + CP is mentioned in title/headline & Economic development & Positive & Subnational & No myth & Improve governance & Positive & Subnational & No myth & NA & NA & NA & NA & Italy & fondi ue: con approvazione po fesr 2014-2020 in arrivo 930 mln & 2015-07-15 & fondi strutturali & (agi) - cagliari, 15 lug. - con l'approvazione del po fesr 2014-2020 da parte della commissione europea, in sardegna arrivano 930 milioni di euro, 465 da finanziamenti europei e il resto da cofinanziamento regionale. fondi - spiega la presidenza delal giunta - che potranno essere utilizzati per garantire supporto alla ricerca e allo sviluppo di almeno 1.576 imprese sarde, agevolare il credito e creare infrastrutture informatiche, sostenere il settore dell'energia e lo sviluppo turistico. "e' un grande risultato per la sardegna, una cifra importante che ci consentira' di intervenire in settori nevralgici della nostra economia regionale, un risultato ottenuto grazie a un serrato lavoro collettivo per cui ringrazio tutti coloro che hanno partecipato", afferma l'assessore della programmazione e del bilancio raffaele paci. il fesr 14/20 conta 35 linee di intervento, decisamente ridotte rispetto alle 120 inizialmente previste per concentrare le risorse sui settori piu' strategici, come richiesto dall'unione europea, dopo un confronto costante con sindacati, associazioni, parti economiche, enti locali. dei 930 milioni, 213 sono destinati alla competitivita' del sistema economico, 164 alla valorizzazione turistica e culturale, 150 all'efficienza energetica, 130 all'agenda digitale, 128 a ricerca e innovazione, 55 ad ambiente e prevenzione del rischio idrogeologico, 51 a inclusione sociale e poverta'. per il via libera da parte della commissione europea ha giocato un ruolo importante anche il pra, piano di rafforzamento amministrativo, ossia uno strumento necessario per l'approvazione del programma che ha l'obiettivo di migliorare la gestione dei fondi strutturali anche attraverso il rafforzamento delle competenze interne all'amministrazione regionale, puntando sulla indispensabile qualita' istituzionale necessaria alla gestione delle risorse. "ora dobbiamo soprattutto spendere strategicamente i fondi europei", spiega l'assessore, "invertendo nettamente e definitivamente la rotta del passato che per troppe volte ha portato a una scarsa qualita' della spesa con conseguente taglio dei fondi. stiamo gia' lavorando all'interno della cabina di regia e dell'unita' di progetto della programmazione unitaria, per definire gli interventi con la visione d'insieme che e' alla base della programmazione unitaria dei fondi sulla quale e' impostata la manovra finanziaria". & 346 & very high & High & Socio-Economic & Governance & NA & 2015-07-15 & 2015 & 1 & ECO
Frame & high-very high & National & <500 & -0.5155635 & -0.1667445 & -0.9728619 & 0.5015415 & 1.3050799 & 3.2 & 0.6840199 & -1.2389660 & Payer & Domestic & Domestic & Domestic & Domestic|ECO & Positive\\
Italy & http://www.ansa.it/europa/notizie/rubriche/altrenews/2018/05/04/studio-ue-grazie-a-fondi-coesione-aumenta-identita-europea\_43f89c61-06c9-40e8-a0ec-10a973c3c864.html & 484 & ANSA.it & Private/Non-Public & Online only & National & high = CP is most important issue in story (can also cover other issues) & Civic participation/collaboration & Positive & EU & No myth & NA & NA & NA & NA & NA & NA & NA & NA & Italy & studio ue, grazie a fondi coesione aumenta identità europea - altre news - ansa europa & 2018-05-04 & politica di coesione & (ansa) - bruxelles, 4 mag - investire nella politica di coesione "migliora la percezione dei vantaggi" derivanti dall'integrazione europea da parte dei cittadini, ma troppo spesso i suoi risultati non sono comunicati in maniera efficace. questa la conclusione principale del progetto cohesify, finanziato dal programma ue horizon2020, che per due anni ha visto la collaborazione di 8 università europee, fra cui il politecnico di milano. l'intento era comprendere l'impatto che ha la politica di coesione sulla formazione dell'identità europea dei cittadini, e in che modo questa viene comunicata. per farlo, sono state effettuate centinaia di interviste e analizzati 17 casi regionali, fra cui la lombardia. la ricerca plaude al "cambiamento significativo" della strategia comunicativa regionale fatto nell'ultimo anno, ma ritiene che vada ancora migliorato l'uso dei social media. solo 18.100 dei partecipanti al sondaggio di cohesify, infatti, hanno detto di aver trovato informazioni sulla politica di coesione attraverso i social, terzo peggior risultato fra i 17 'casi studio'. per i lombardi, la tv e la stampa nazionale restano le principali fonti d'informazione, ma, sottolinea la ricerca, nel 20\% dei casi questi parlano di fondi ue in maniera negativa collegandoli a scandali o ritardi nei progetti. un quadro che, però, non sembra intaccare la visione positiva dell'unione. "il numero di cittadini a favore dell'integrazione ue è più alto della media dei case studies analizzati", scrivono i ricercatori.(ansa). & 235 & high & High & Socio-Economic & NA & NA & 2018-05-04 & 2018 & 3 & ECO
Frame & high-very high & National & <500 & -0.5155635 & -0.1667445 & -0.9728619 & 0.5015415 & 1.3050799 & 3.2 & 0.6840199 & -1.2389660 & Payer & European & European & European & European|ECO & Positive\\
Italy & http://www.ansa.it/europa/notizie/rubriche/economia/2016/07/26/verso-prime-multe-ue-a-spagna-e-portogalloma-saranno-minime\_172a0fce-6ee9-4e6d-aa9f-c20428534a9c.html & 411 & ANSA.it & Private/Non-Public & Online only & National & low = CP mentioned more times but NOT important part of story (mainly about others issues) & Political leverage & Balanced & EU + Other country & No myth & NA & NA & NA & NA & NA & NA & NA & NA & Italy & verso prime multe ue a spagna e portogallo,ma saranno minime - economia e lavoro - ansa europa & 2016-07-26 & fondi strutturali & (ansa) - bruxelles, 26 lug - la commissione europea è orientata a proporre per la prima volta delle multe 'minime' da applicare a spagna e portogallo per non aver rispettato gli impegni presi sul fronte della riduzione dei deficit eccessivi. a quanto si è appreso, l'importo delle sanzioni dovrebbe oscillare tra lo 0,01 e lo 0,03\% dei rispettivi pil. l'ultima parola sull'ammontare delle multe spetta comunque al collegio dei commissari che domani avrà il dossier sul tavolo. in ogni caso, quella che sarà varata domani dalla commissione sarà una proposta (tecnicamente una 'raccomandazione') che dovrà essere poi approvata - o eventualmente modificata a maggioranza qualificata - dagli stati membri in sede di consiglio. non sono invece previste per ora decisioni sul possibile congelamento dei fondi strutturali destinati ai due paesi. teoricamente questa sanzione potrebbe arrivare a bloccare il 50\% degli importi messi a disposizione dal bilancio ue, ma pare che a bruxelles nessuno - per motivi sia economici che politici - voglia prendere in considerazione questa eventualità. su cui, tra l'altro, sarebbe chiamato a pronunciarsi anche il parlamento europeo. lo scorso 7 luglio la commissione ha sancito che nè madid nè lisbona hanno adottato nel 2014 e nel 2015 misure "efficaci" per rispettare gli obiettivi di bilancio fissati per l'anno passato e per quest'anno. facendo così scattare per la prima volta, nell'ambito della procedura per deficit eccessivo, l'iter che può portare a multe e sanzioni. nei giorni scorsi i governi di spagna e portogallo hanno presentato a bruxelles le loro richieste "motivate" per annullare eventuali multe invocando circostanze eccezionali. richieste che, sebbene non accolte in pieno, hanno sicuramente contribuito a rendere poco più che simboliche eventuali sanzioni che altrimenti avrebbero potuto arrivare fino a un massimo dello 0,2\% del pil (senza contare il blocco dei fondi strutturali). inoltre, sempre in base alle informazioni raccolte alla vigilia della riunione del collegio, sia per madrid che per lisbona bruxelles sarebbe orientata a dare rispettivamente due e un anno di tempo in più per riportare sotto al 3\% il rapporto deficit-pil.(ansa). & 345 & low & Low & Power & NA & NA & 2016-07-26 & 2016 & 2 & POL
Frame & low-medium & National & <500 & -0.5155635 & -0.1667445 & -0.9728619 & 0.5015415 & 1.3050799 & 3.2 & 0.6840199 & -1.2389660 & Payer & European & European & European & European|POL & Neutral\\
Italy & http://lanuovaferrara.gelocal.it/ferrara/cronaca/2017/08/10/news/la-legalita-e-di-tutti-per-noi-i-controlli-sono-al-primo-posto-1.15720842 & 424 & La Nuova Ferrara & Private/Non-Public & Online and Offline & Regional/Local & very low = CP mentioned once & Improve governance & Positive & Subnational & No myth & NA & NA & NA & NA & NA & NA & NA & NA & Italy & "la legalità è di tutti per noi i controlli sono al primo posto"  - cronaca - la nuova ferrara & 2017-08-10 & fondi strutturali & l'assessore gazzolo alla festa pd fa il punto del post sisma il comitato: segnalate decine e decine di casi anomali bondeno. tra il pubblico presente martedì sera alla festa de l'unità non mancano cittadini smaniosi di raccontare la loro storia, spesso di sofferenza dopo il sisma. compresi i membri del comitato "verifica della ricostruzione", che ha già presentato circa una quarantina di esposti. i controlli, evidenzia però l'assessore paola gazzolo, "sono stati messi al primo posto. la legalità deve appartenere a tutti - dice l'assessore regionale al comitato -. è un bene che si facciano insieme le segnalazioni". i controlli, spiega enrico cocchi (regione): "riguardano il 100\% delle pratiche. dal momento che l'autorizzazione abitativa viene rilasciata dal comune. abbiamo utilizzato lo stesso meccanismo dei fondi strutturali, invece, per le imprese. si è anche "incappati" nei controlli dell'unione europea, perché le risorse statali possono distorcere il principio della libera concorrenza". dopo quelli preventivi, vi sono controlli a campione di regione e comuni. anche per la cosiddetta "permanenza dei requisiti", "rispetto all'impegno - dice cocchi - che il privato si è preso nei confronti del pubblico. abbiamo segnalato un'ottantina di casi - dice cocchi - alla procura, che sta indagando per verificare se vi sono stati episodi di dolo". daniele biancardi, ex amministratore nei comuni di bondeno e cento che fa parte del comitato verifica ricostruzione: "abbiamo segnalato decine e decine di casi, in cui un edificio inagibile prima del sisma, che non poteva essere finanziato, lo è stato. pensiamo che i relativi controlli potevano essere fatti preventivamente. persino nella mostra fotografica che avete allestito (dice alla festa de l'unità "il terremoto 5 anni dopo" a cura di bracciano lodi e foto di maurizio guerzoni, inaugurata proprio martedì sera con l'assessore regionale paola gazzolo, ndr), ci sono due casi di questo tipo". ma le lamentale sono varie: da una donna che si dice "costretta a lavarsi con l'acqua fredda", perché le centraline e gli impianti sono nella parte danneggiata. fino al cittadino che attende il cas (contributo autonoma sistemazione) e vede separata la propria famiglia. decine e decine di storie, che fanno capire una cosa: l'emergenza terremoto anche da queste parti è tutt'altro che finita. a distanza di cinque anni dalle devastanti scosse del 20 e 29 maggio tanto è stato fatto ma tanto resta ancora da fare con la burocrazia che spesso e volentieri finisce con l'intralciare le operazioni. la parola fine è ancora assai lontana dall'essere scritta e chissà mai quando ciò avverrà. si va avanti a piccoli passi, fra tante difficoltà quotidiane. mirco peccenini ©riproduzione riservata. & 436 & very low & Low & Governance & NA & NA & 2017-08-10 & 2017 & 2 & POL
Frame & v.low & Regional & <500 & -0.5155635 & -0.1667445 & -0.9728619 & 0.5015415 & 1.3050799 & 3.2 & 0.6840199 & -1.2389660 & Payer & Domestic & Domestic & Domestic & Domestic|POL & Positive\\
Italy & https://www.ecodibergamo.it/stories/ansa/commissione-ue-al-lavoro-su-caso-bekaert\_1284372\_11/ & 438 & L'Eco di Bergamo & Private/Non-Public & Online and Offline & Regional/Local & medium = CP is important part of story & Mismanagement & Balanced & EU + National & 3.Firms/jobs relocate to poor countries & NA & NA & NA & NA & NA & NA & NA & NA & Italy & commissione ue al lavoro su caso bekaert & 2018-07-12 & fondi strutturali & bruxelles - gli eurodeputati pd nicola danti e simona bonafè fanno sapere che la commissaria europea margrethe vestager ha risposto oggi alla lettera con cui lo scorso 27 giugno avevano chiesto chiarimenti su eventuali violazioni delle regole europee sulla concorrenza nel caso bekaert. vestager ha confermato che gli uffici della commissione sono già al lavoro per ricostruire i fatti legati a questa vicenda, in particolare attraverso la richiesta di informazioni alle autorità rumene nell'obiettivo di verificare se le regole ue siano state rispettate o meno. "accogliamo la risposta della commissaria alla concorrenza vestager come una notizia molto positiva. nella sua lettera, la commissaria ha infatti ribadito che l'ue "non ammette l'uso di fondi strutturali o di aiuti di stato in modo tale da incoraggiare la delocalizzazione di servizi e produzioni in un altro stato membro". su questo, la commissione è attualmente in attesa di ricevere dalla romania le informazioni necessarie a valutare i passi successivi da compiere. a distanza di due settimane dalla nostra richiesta possiamo dire che qualcosa si sta muovendo, anche se siamo solo al primo passo. & 181 & medium & Medium & Governance & NA & NA & 2018-07-12 & 2018 & 3 & POL
Frame & low-medium & Regional & <500 & -0.5155635 & -0.1667445 & -0.9728619 & 0.5015415 & 1.3050799 & 3.2 & 0.6840199 & -1.2389660 & Payer & Domestic & European & Mixed & Domestic|POL & Neutral\\
\addlinespace
Italy & https://www.huffingtonpost.it/paola-caporossi/basilicata-al-voto-cosa-conviene-sapere\_a\_23698510/ & 465 & L'Huffington Post & Private/Non-Public & Online only & National & low = CP mentioned more times but NOT important part of story (mainly about others issues) & Bureaucracy and/or delays & Negative & Subnational & No myth & NA & NA & NA & NA & NA & NA & NA & NA & Italy & basilicata al voto: cosa conviene sapere & 2019-03-24 & fondi strutturali & la regione basilicata al voto, ma cosa sappiamo della sua macchina amministrativa? come ha funzionato in questi anni, con quali numeri? non solo numeri di bilancio, ma anche di raggiungimento degli obiettivi, di trasparenza nella rendicontazione, di servizi ai cittadini, di pagamento alle imprese fornitrici. non basta valutare le politiche, rientranti nelle responsabilità della giunta: sull'azione di qualunque governo regionale incide anche la capacità amministrativa, rientrante nelle responsabilità soprattutto dei dirigenti regionali e senza la quale anche le miglior politiche pubbliche sono destinate a restare lettera morta. non trascuriamola, dunque. misurare la capacità amministrativa di una regione non significa voler dare pagelle, ma dare attuazione alle disposizioni normative vigenti sugli obblighi di trasparenza e accountability che investono tutte le pa. è dovere di queste ultime rendicontare il proprio operato a cittadini e imprese, che sono utenti e finanziatori delle regioni, prima ancora che elettori. per capire come ha funzionato la macchina regionale della basilicata non basta guardare al bilancio: serve conoscerne la performance qualitativa rispetto alle altre regioni italiane. nella misurazione qualitativa comparata che il rating pubblico (cfr. nota 1) effettua annualmente, la basilicata risulta terz'ultima tra le regioni, nell'ambito del rapporto dedicato che è in uscita nel prossimo mese di aprile (cfr. nota 2). la basilicata ottiene, infatti, uno score complessivo pari a 33 su 100: fanno peggio solo la calabria, con 31, e il molise, con 19. si tratta di un punteggio decisamente insufficiente per la basilicata, decisamente al di sotto di quello medio delle regioni a statuto ordinario (di seguito rso), che, invece, è sopra la sufficienza, con 52 su 100. a incidere su quel risultato, oltre al quasi 93\% di territorio regionale svantaggiato, è anche il basso pil, che per la basilicata supera di poco i 20.000 euro pro capite, a fronte degli oltre 26.000 di media delle rso. va, tuttavia, considerato che campania e puglia hanno un pil persino inferiore (rispettivamente 18.235 e 17.428 euro pro capite) e, ciò nonostante, ottengono uno score superiore a quello della basilicata. score che per la campania si avvicina quasi alla sufficienza, con 46 su 100. per comprendere meglio il rating qualitativo complessivo della basilicata, serve un'analisi disaggregata per singole aree. paradossalmente, l'area bilancio è quella in cui la regione ottiene non solo il suo punteggio migliore, ma anche l'unico sopra la sufficienza, con 52 su 100. l'area, invece, in cui la basilicata ottiene il punteggio peggiore è la gestione degli appalti, dove non solo risulta ultima tra le rso, insieme al molise, ma si colloca nella pericolosa fascia di rating fallible, con score pari a 10 su 100. la basilicata risulta ultima anche nell'area impatto ambientale, con score pari a 30 su 100. riguardo, infine, alle altre tre aree oggetto di analisi del rating pubblico, la basilicata risulta sempre agli ultimi posti: quint'ultima nella gestione del personale, quart'ultima nella governance, terz'ultima nei servizi e rapporti con i cittadini. come si spiega un quadro complessivo così disastrato della macchina amministrativa lucana? molto dipende dallo scarso livello di trasparenza: a differenza anche di altre regioni del sud, la basilicata risulta non preoccuparsi molto di rendere conto ai propri cittadini di come vengono spesi i soldi pubblici. naturalmente, la basilicata non va male in tutti gli indicatori del rating pubblico: tra quelli di bilancio, si distingue, anzi, per essere una delle regioni con maggiore capacità di spesa e di riscossione, in percentuali superiori persino a quelle di regioni performanti come lombardia e toscana. ciò nonostante, la basilicata è la regione a so con la maggiore incidenza di nuovi residui passivi, quasi l'80\%, a fronte di una media rso di 57. presumibilmente, rispetto al passato la basilicata comincia ad avere un disallineamento tra competenza e cassa, indice di una crescente inadeguatezza nel far fronte con tempestività alle proprie obbligazioni. nonostante il suo basso pil pro capite, la basilicata presenta la maggiore pressione finanziaria tra le rso: quasi 2.500 euro a persona a fronte di un reddito imponibile che non arriva a 10.500 euro pro capite. inaspettato è il risultato della basilicata per la spesa sanitaria: presenta, infatti, l'avanzo maggiore tra le regioni a statuto ordinario, con 17 euro. il suo sistema sanitario è, almeno a livello gestionale, una sorta di best practice del meridione. non a caso, è l'unica regione del sud che non ha dovuto sottostare a un piano di rientro. quel dato è ancora più rilevante se si considera che sette regioni su quindici a so risultano, invece, in disavanzo, in alcun casi anche molto marcato. naturalmente, questo non comporta, di per sé, un giudizio positivo sulla qualità della sanità lucana. inatteso, ma per motivi opposti, è anche il risultato della basilicata nell'utilizzo dei fondi europei: regione pioniera nell'uso proficuo dei fondi strutturali negli anni '70, poi cresciuta come pil fino ad entrare in phasing out dall'area "convergenza" nella stagione di programmazione 2007-2013, è purtroppo ricaduta nel novero delle aree di ritardo anche a causa della crisi economica scoppiata nel 2011. nell'ambito dell'area governance, la basilicata sembra gestire oculatamente il patrimonio immobiliare pubblico, con un quasi pareggio tra saldi fitti attivi e passivi: -0,2 euro pro capite, a fronte della maggioranza delle rso che presentano un saldo pro capite negativo. sulla performance complessiva della basilicata incide negativamente anche un personale dell'amministrazione con età superiore alla media rso: i dipendenti lucani, infatti, hanno un'età media di oltre 55 anni a fronte dei circa 50 dei dipendenti di regioni performanti come lombardia, emilia e toscana. sono anche i dipendenti che, insieme a quelli di calabria e abruzzo, comportano un'incidenza del loro costo sulla spesa corrente non sanitaria superiore alla media del 9,7\%: per la basilicata, infatti, è il 16,5\%. un segnale negativo è anche l'assenza di trasparenza nell'erogazione dei premi ai dirigenti, in termini sia di premi erogati rispetto a quelli stanziati, sia del loro grado di differenziazione. è, invece, positivo il dato sui giorni di assenza dei dipendenti lucani: con circa 32 giorni è il più basso tra le rso, insieme a quello del lazio. in merito ai servizi, è da apprezzare lo sforzo che la basilicata ha compiuto per modernizzarsi, aprendo un portale dedicato ai servizi online. anche se, a ben guardare, i servizi effettivamente fruibili restano pochi. tra i servizi in ambito sanitario, la basilicata risulta carente rispetto alle altre rso nell'assistenza domiciliare alle persone over 65 anni: sotto la media del 3\% risultano oltra alla campania e alla calabria, anche la basilicata, insieme a tre regioni del centro italia, quali lazio, marche e umbria. in tema di appalti e rapporto con le imprese fornitrici, il punteggio ottenuto dalla basilicata sconta la mancanza di trasparenza: non sono pubblicati, infatti, i dati necessari per verificare la ricorrenza delle imprese aggiudicatarie degli appalti, né la percentuale di affidamenti diretti rispetto alle gare pubbliche. la basilicata è, poi, decisamente in difetto nel pagamento ai fornitori: salda, infatti, le fatture con oltre tre mesi di ritardo. tra le regioni italiane fa peggio solo il molise, a fronte di giorni medi di ritardo delle regioni a so corrispondenti a poco più di un mese. infine, l'area impatto ambientale: nonostante il basso score complessivo riportato nell'area, la basilicata raggiunge un ottimo risultato nell'indicatore relativo alla produzione di energia elettrica attraverso impianti geotermoelettrici, eolici e fotovoltaici, infatti, non solo è la regione benchmark, ma supera il 90\% di energia rinnovabile sul totale di energia prodotta: quasi il doppio del valore medio rso (48,2\%). in conclusione, dalla fotografia complessiva della macchina amministrativa lucana gli scuri sembrano prevalere sui chiari, soprattutto in termini di volontà e capacità di rendicontazione ai cittadini. conviene tenerne conto per chiedere alla nuova giunta - di qualunque colore politico uscirà dal voto di domenica prossima - uno sforzo maggiore al riguardo. 1. indice qualitativo delle pa creato da fondazione etica sulla base della metodologia esg, utilizzata sui mercati finanziari, e sulla base delle indicazioni normative vigenti in italia. 2. p.caporossi, a cura di, "w le regioni?", ed. rubbettino, aprile 2019. & 1355 & low & Low & Governance & NA & NA & 2019-03-24 & 2019 & 3 & POL
Frame & low-medium & National & +1000 & -0.5155635 & -0.1667445 & -0.9728619 & 0.5015415 & 1.3050799 & 3.2 & 0.6840199 & -1.2389660 & Payer & Domestic & Domestic & Domestic & Domestic|POL & Negative\\
Italy & http://www.ilsole24ore.com/art/notizie/2015-02-01/fondi-ue-cofinanziamenti-ridotti-74-miliardi-081306.shtml?uuid=ABbMoXnC & 425 & Il Sole 24 ORE & Private/Non-Public & Online and Offline & National & high = CP is most important issue in story (can also cover other issues) & Bureaucracy and/or delays & Balanced & National & 9.Inaccessible & Institutional bargaining over funding & Balanced & National + Subnational & No myth & Institutional bargaining over funding & NA & NA & NA & Italy & fondi ue, cofinanziamenti ridotti di 7,4 miliardi & 2015-02-01 & fondi strutturali & il governo tiene fede all'impegno del sottosegretario a palazzo chigi di non cancellare risorse ma destinarle a piani di investimento vincolati al territorio sarà di 7,4 miliardi il taglio del cofinanziamento nazionale ai fondi strutturali europei per il 2014-2020. la riduzione del cofinanziamento dal 45-50\% al 25\% della programmazione totale inciderà sulle dotazioni di campania, calabria e sicilia, con un taglio da 4.448 milioni ai tre piani operativi regionali (por) e una riduzione da 2.978 milioni alla quota dei piani operativi nazionali destinata al sud. dal taglio si salveranno solo i due piani nazionali per la scuola e per l'occupazione dove il cofinanziamento per il sud raggiungerà il livello massimo del 45\%. il governo ha però confermato in pieno la promessa fatta dal sottosegretario alla presidenza del consiglio, graziano delrio, di non cancellare queste risorse facendole invece confluire in "programmi complementari" paralleli che avranno scadenze e procedure meno rigide (in termini di obiettivi di spesa) rispetto alla programmazione dei fondi ue ma rispetteranno l'ancoraggio territoriale e le priorità programmatiche. a quantificare nei dettagli l'operazione della riduzione del cofinanziamento è il cipe, comitato interministeriale per la programmazione economica, che nell'ultima riunione di giovedì scorso ha approvato il quadro finanziario nazionale a sostegno della programmazione comunitaria 2014-2020: a fronte di fondi strutturali ue per lo sviluppo regionale (fesr) e sociale (fse) pari a 32.686 milioni, il cofinanziamento nazionale in senso stretto si attesterà a 20.085 milioni per un totale di programmazione di 51.771 milioni (61,2\% a carico dei fondi ue, 38,8\% dei fondi nazionali). la quota delle cinque regioni meno sviluppate (oltre alle tre penalizzate, ci sono puglia e basilicata che mantengono cofinanziamenti vicini al 50\%) sarà ovviamente più bassa della media nazionale, il 33,4\%: 11,3 miliardi su 22,4 di fondi ue. le regioni di transizione (abruzzo, molise e sardegna) avranno cofinanziamenti per 1.337,9 milioni a fronte di fondi ue per 1.387,7 milioni, mentre le regioni più sviluppate avranno cofinanziamenti per 7.468,3 milioni a fronte di fondi comunitari per 7.867 milioni. come si vede, nel centro-nord, il cofinanziamento è praticamente pari ai fondi ue, quindi al 50\% della programmazione totale. nei numeri del cipe si chiarisce anche che il cofinanziamento nazionale allargato (se si comprendono cioè i 7,4 miliardi di "programmi complementari" e altre assegnazioni minori) arriva a 24 miliardi di risorse statali (previste nel fondo di rotazione ad hoc) e 4,4 miliardi di quota regionale. nello spostamento di risorse la penalizzazione maggiore riguarda i piani operativi regionali: oltre 800 milioni per la calabria, circa 1,8 miliardi per la campania, circa 2 miliardi per la sicilia. tra i piani nazionali lo sconto maggiore lo subiscono il pon reti con circa 650 milioni e il pon imprese e competitività con circa 800 milioni. sui "programmi complementari" - che ricordano da vicino il piano azione coesione (pac) con cui l'italia ha abbassato il cofinanziamento per il ciclo 2007-2013 per raggiungere target di spesa altrimenti lontanissimi - la delibera cipe non aggiunge altro, per ora. è probabile che parta ora una programmazione parallela che si incroci con i "piani strategici" di azione e coesione finanziati anche con l'altra grande cassaforte per le infrastrutture nel sud, il fondo sviluppo coesione (fsc). anche per la programmazione di questo fondo sarà il cipe a decidere entro aprile, come previsto dalla legge di stabilità 2015. importo dei cofinanziamenti statali e regionali & 583 & high & High & Governance & Power & Power & 2015-02-01 & 2015 & 1 & POL
Frame & high-very high & National & 500-1000 & -0.5155635 & -0.1667445 & -0.9728619 & 0.5015415 & 1.3050799 & 3.2 & 0.6840199 & -1.2389660 & Payer & Domestic & Domestic & Domestic & Domestic|POL & Neutral\\
Italy & http://www.ansa.it/sito/notizie/economia/2015/05/02/fondi-ue-litalia-deve-ancora-utilizzare-12-milioni-di-euro\_ff4b553f-5c9a-4bb8-a7ff-3e7f668ed871.html & 458 & ANSA.it & Private/Non-Public & Online only & National & high = CP is most important issue in story (can also cover other issues) & Bureaucracy and/or delays & Negative & National & 10.Slow spend & NA & NA & NA & NA & NA & NA & NA & NA & Italy & fondi ue, l'italia deve ancora utilizzare 12 miliardi di euro - economia & 2015-05-02 & fondi strutturali & sono i paesi del nord europa a farsi carico degli sforzi economici maggiori per sostenere l'ue. in termini pro-capite, l'italia è all'undicesimo posto. nel saldo dare/avere con l'europa, tra il 2007 e il 2013 ogni italiano ha 'versato' 623 euro. lo sostiene la cgia di mestre, secondo cui l'italia ha utilizzato 35,4 miliardi di euro dei 47,3 messi a disposizione dai fondi strutturali. pertanto, dobbiamo ancora utilizzare 12 miliardi di euro. la maggior parte dei 47,3 miliardi di euro arriva dall'europa e fa parte della programmazione 2007-2013. l'incidenza dei finanziamenti utilizzati fino ad ora sul totale dei contributi assegnati, che include anche il cofinanziamento nazionale, ha raggiunto il 74,8\%. "per non perdere 12 miliardi di fondi europei e nazionali - segnala il segretario della cgia giuseppe bortolussi - dovremo spenderli e rendicontarli entro la fine del 2015, scadenza che difficilmente l'ue prorogherà. alla luce del fatto che nel 2013 abbiamo rendicontato 5,7 miliardi e nel 2014 attorno ai 7,5, appare difficile che nei pochi mesi che rimangono alla fine di quest'anno riusciremo a spendere e a contabilizzare tutta questa dozzina di miliardi". l'elaborazione degli artigiani di mestre, comunque, è proseguita analizzando il contributo finanziario netto allo sviluppo di tutti i paesi dell'ue. nel periodo 2007-2013, l'italia, ad esempio, ha versato a bruxelles 109,7 miliardi di euro e ne ha ricevuti, attraverso i programmi comunitari, 71,8. "nel rapporto dare/avere con l'ue - conclude - in questo settennato abbiamo registrato un saldo negativo di 37,8 miliardi di euro. dopo la germania, il regno unito e la francia, siamo il quarto contribuente netto a garantire l'azione dell'unione. se, invece, prendiamo come parametro di riferimento il dato pro-capite, sono i paesi nordici a guidare la graduatoria, mentre l'italia scivola all'undicesimo posto, con uno sforzo economico per residente pari a soli 623 euro". analizzando la differenza assoluta tra le risorse versate all'unione e quelle accreditate a ciascun stato dell'ue tra il 2007 e il 2013, il maggior contributore è la germania, con 83,5 miliardi di euro. seguono il regno unito, con 48,8 miliardi, la francia, con 46,5 miliardi e l'italia con 37,8. se, invece, prendiamo come termine di raffronto il dato pro-capite, il maggior sostenitore dell'ue è il belgio, con 1.714 euro. immediatamente dopo scorgiamo i paesi bassi (1.569 euro), la danimarca (1.346 euro), la svezia (1.195 euro), la germania (1.034 euro), il lussemburgo (997 euro), il regno unito (759 euro), la francia (707 euro), la finlandia (689 euro), l'austria (674 euro), l'italia (623 euro) e cipro (197 euro). tutti gli altri 17 paesi, invece, sono percettori netti, ovvero hanno ottenuto più di quanto hanno versato a bruxelles. uno spagnolo, ad esempio, ha ricevuto 355 euro, un polacco 1.522 euro, un portoghese 2.100 euro e un greco 2.960 euro. & 501 & high & High & Governance & NA & NA & 2015-05-02 & 2015 & 1 & POL
Frame & high-very high & National & 500-1000 & -0.5155635 & -0.1667445 & -0.9728619 & 0.5015415 & 1.3050799 & 3.2 & 0.6840199 & -1.2389660 & Payer & Domestic & Domestic & Domestic & Domestic|POL & Negative\\
Italy & http://ilcentro.gelocal.it/laquila/cronaca/2016/02/16/news/fatture-false-commercialisti-e-titolari-di-aziende-nei-guai-nell-aquilano-1.12968005 & 462 & il Centro & Private/Non-Public & Online and Offline & Regional/Local & very low = CP mentioned once & Fraud/Corruption & Negative & Subnational & No myth & NA & NA & NA & NA & NA & NA & NA & NA & Italy & fatture false, commercialisti e titolari di aziende nei guai nell'aquilano - cronaca - il centro & 2016-02-16 & fondi strutturali & due consulenti e tre imprese che commerciano giocattoli accusati di evasione fiscale e truffa. sequestrati beni immobili e soldi per 200mila euro l'aquila. evasione fiscale e truffa ai danni dello stato e della ue sono i reati ipotizzati per i responsabili di tre imprese aquilane nel commercio di giocattoli e per due commercialisti ai quali il nucleo di polizia tributaria della guardia di finanza ha sequestrato, su disposizione dell'autorità giudiziaria, disponibilità finanziarie e beni immobili per circa 200.000 euro. la misura cautelare appena eseguita arriva al termine di complesse indagini di polizia giudiziaria svolte dai finanzieri del capoluogo abruzzese dopo una segnalazione alla procura della repubblica di l'aquila della locale agenzia delle entrate. in pratica, l'ufficio finanziario segnalava che da controlli di natura fiscale era stato accertato che le aziende in questione erano ricorse all'uso e all'emissione di fatture fittizie, per un importo appunto di 200.000 euro, per evadere l'iva e le altre imposte sui redditi. da approfondimenti dei finanzieri è emerso che gli indagati attraverso gli stessi documenti fiscali fittizi avevano ottenuto un contributo comunitario nell'ambito dei fondi strutturali por fesr 2007/2013 stanziati per attrarre nuove imprese nell'area del cratere sismico. & 204 & very low & Low & Governance & NA & NA & 2016-02-16 & 2016 & 2 & POL
Frame & v.low & Regional & <500 & -0.5155635 & -0.1667445 & -0.9728619 & 0.5015415 & 1.3050799 & 3.2 & 0.6840199 & -1.2389660 & Payer & Domestic & Domestic & Domestic & Domestic|POL & Negative\\
Italy & http://www.ilfattoquotidiano.it/2017/02/06/terremoto-via-libera-meta-alla-procedura-semplificata-per-la-proposta-ue-di-finanziare-tutta-la-ricostruzione/3372365/ & 481 & Il Fatto Quotidiano & Private/Non-Public & Online and Offline & National & very low = CP mentioned once & Infrastructure & Balanced & EU + National & No myth & Institutional bargaining over funding & Balanced & EU + National & No myth & NA & NA & NA & NA & Italy & terremoto, procedura semplificata per proposta ue di finanziare la ricostruzione. m5s: "compromesso al ribasso" - il fatto quotidiano & 2017-02-06 & fondi strutturali & il via libera è arrivato. ma con un compromesso che complica la situazione. e per ora si tratta solo un orientamento espresso dai capigruppo: ora occorre attendere il voto della commissione e poi della plenaria, probabilmente a marzo. i coordinatori dei gruppi della commissione sviluppo regionale del parlamento europeo si sono espressi a favore della procedura di voto semplificata per modificare il regolamento sui fondi strutturali con l'obiettivo di consentire il finanziamento del 100\% dei costi della ricostruzione post terremoto da parte della ue. senza bisogno, quindi, del contributo finanziario dell'italia (e di tutti gli altri paesi che ne avranno bisogno). un esito che da un certo punto di vista potrebbe mettere in difficoltà roma, perché se il conto fosse saldato interamente da bruxelles il governo non potrebbe più utilizzare le "spese eccezionali" per il sisma come arma di trattativa con la commissione nel negoziato sulla manovra correttiva. la procedura semplificata in teoria dovrebbe tradursi in un'accelerazione mesi rispetto all'iter normale. ma rosa d'amato, capo-delegazione del movimento 5 stelle in europa, sostiene che non sarà così e attacca: "i gruppi politici a cui appartengono forza italia, ncd e partito democratico non hanno sostenuto la nostra posizione ma hanno ceduto ad un compromesso che allungherà i tempi e le procedure: la procedura sarà sì semplificata ma con apertura di emendamenti e votazione finale che slitterà". il movimento 5 stelle "ha votato a favore e abbiamo chiesto a tutti gli eurodeputati italiani di impegnarsi allo stesso modo nei confronti dei coordinatori dei loro gruppi. sono incredibili le dichiarazioni di alcuni eurodeputati del partito democratico che parlano di una vittoria e di un loro sostegno alla procedura accelerata, quando il coordinatore del loro gruppo è stato il primo a cedere al compromesso. noi non ci arrendiamo, ora terremo alta l'attenzione su questo accordo perché il compromesso riapre la lotteria degli emendamenti che rischiano di far abbassare la quota di ricostruzione finanziata che oggi è 100\%". prima della riunione dei coordinatori, l'eurodeputato del pd e vice presidente della commissione regi andrea cozzolino aveva ribadito quanto detto la settimana scorsa: "come socialisti e democratici europei, tutti uniti, abbiamo deciso di difendere la procedura di voto semplificata e di sostenere con forza la proposta della commissione di aumentare al 100\% il cofinanziamento per la ricostruzione delle zone colpite da disastri naturali". stessa posizione per i conservatori e riformisti europei: "insieme al collega remo sernagiotto, come deputati del gruppo ecr al parlamento europeo sosteniamo, in accordo con il nostro coordinatore in commissione, il ricorso alla procedura semplificata", ha detto raffaele fitto. "la nostra priorità è, e lo sarà sempre, far sì che i finanziamenti per la ricostruzione siano approvati il più velocemente possibile". "la delegazione italiana del ppe - hanno rivendicato dal canto loro elisabetta gardini e lorenzo cesa - ha lavorato intensamente affinché la posizione di tutto il gruppo ppe potesse convergere verso una rapida adozione del testo in linea con il carattere di urgenza della proposta. in questo è stato fondamentale l'aiuto del capogruppo, manfred weber". "accordare una procedura semplificata a tale proposta permetterà all'ue di mostrare la sua azione concreta e la sua solidarietà alle popolazioni colpite. non stiamo chiedendo un'eccezione per l'italia ma uno strumento che sia a disposizione di tutti gli stati membri colpiti da catastrofi naturali". & 555 & very low & Low & Socio-Economic & Power & NA & 2017-02-06 & 2017 & 2 & ECO
Frame & v.low & National & 500-1000 & -0.5155635 & -0.1667445 & -0.9728619 & 0.5015415 & 1.3050799 & 3.2 & 0.6840199 & -1.2389660 & Payer & Domestic & European & Mixed & Domestic|ECO & Neutral\\
\addlinespace
Italy & http://www.ilsole24ore.com/art/notizie/2015-01-22/mobilita-obbligatoria-pa-ecco-prossime-mosse-governo-100023.shtml?uuid=ABG5v4hC & 466 & Il Sole 24 ORE & Private/Non-Public & Online and Offline & National & very low = CP mentioned once & Jobs & Positive & National & No myth & NA & NA & NA & NA & NA & NA & NA & NA & Italy & mobilità obbligatoria nella pa, ecco le prossime mosse del governo & 2015-01-22 & fondo sociale europeo & dopo l'annuncio via tweet del ministro marianna madia dello "slocco" di un migliaio di posti presso gli uffici giudiziari da ricoprire con la mobilità volontaria, il governo prepara le prossime mosse per la gestione degli esuberi delle province e delle città metropolitane. martedì prossimo è stato infatti convocato l'osservatorio nazionale per l'attuazione della legge delrio, il tavolo cui partecipano regioni, anci e upi oltreché i ministeri della pa, degli affari regionali, dell'economia e dell'interno. all'incontro sono stati convocati anche i sindacati. oggetto del confronto l'attuazione del cronoprogramma previsto in legge di stabilità con il taglio delle dotazioni organiche di province e città metropolitane rispettivamente del 50 e del 30\%. il governo prevede un percorso molto graduale per gestire la mobilità obbligatoria che riguarda poco meno di 20mila dipendenti sui circa 44mila che attualmente hanno un contratto a tempo indeterminato in questi enti in fase di riordino. una procedura soft e senza impatti sulle buste paga prima dell'aprile del 2017, quando i dipendenti in mobilità che ancora non fossero stati trasferiti ad altri enti o amministrazioni potrebbero essere "collocati in disponibilità", come prevede l'articolo 33 del dlgs 165/2001, e subire un taglio del 20\% sull'indennità base. mentre per arrivare all'ipotesi estrema, quella del licenziamento perché proprio non è stato trovato un ricollocamento adeguato, bisognerà arrivare all'aprile del 2019. ma l'esecutivo esclude che anche un solo dipendente possa essere licenziato. i numeri reali in gioco sarebbero in realtà molto minori di 20mila: escludendo gli 8mila dei centri per l'impiego, si arriva a 12mila e di questi circa 3.500 hanno un'età media attorno ai sessant'anni. risultato: 8.500 dipendenti (esclusi i dirigenti) da ricollocare. la mossa della giustizia con la pubblicazione del bando per un migliaio di posti vacanti, il ministero guidato da andrea orlando ha aperto le danze: non potranno fare domanda di trasferimento negli uffici giudiziari solo i dipendenti delle province ma funzione pubblica si aspetta da lì il flusso maggiore. anche perchè l'80\% dei posti coperti con il precedente bando del luglio 2013 è stato coperto proprio da loro. per finanziare l'operazione è stato firmato (è di imminente pubblicazione anche questo) il dpcm madia-padoan che attiva il fondo da 30 milioni di euro per le previste compensazioni che accompagneranno la mobilità: sia a livello retributivo sia per quanto riguarda la capacità di spesa delle amministrazioni coinvolte la dote dovrà garantire la sosteniblità finanziara dell'operazione (ai sensi del dl 90/2014). il vertice di martedì l'osservatorio nazionale dovrà definire i tagli degli organici e individuare i contingenti del personale degli enti in riordino. poi il cronoprogramma prevede una scadenza entro aprile: regioni e dipartimento funzione pubblica individueranno i posti disponibili per assumere il personale in soprannumero delle ex province. ma attenzione: quei posti saranno disponibili solo in parte, restano infatti i vincoli del blocco del turn over e la precedenza ai vincitori di concorso. la procedura speciale prevista dalla legge delrio si conclude a fine 2016. a quel punto si aprirà il confronto con i sindacati per l'eventuale utilizzo dei contratti a tempo parziale degli addetti ancora in mobilità (dirigenti e più anziani esclusi) e dal febbraio 2017, fatte le ulteriori verifiche, si entra nella procedura ordinaria prevista dal testo unico del pubblico impiego per gli eventuali soprannumeri residui. i quali ultimi, solo da aprile di quell'anno, potranno davvero rischiare il taglio del 20\% della busta paga base ed entrare nella prospettiva del licenziamento per mancato ricollocamento che scatterebbe 24 mesi dopo. centri per l'impiego e uffici giudiziari fin qui il piano per la mega-mobilità nelle province, un percorso che, come detto, non riguarderà i circa 8mila dipendenti dei centri per l'impiego, compresi i 1.200 contrattisti a termine per i quali in stabilità è prevista una dote di 60 milioni (via fondo sociale europeo) da usare per i rinnovi che dovranno assicurare le regioni per non far mancare personale coinvolto nella gestione del garanzia giovani. restano fuori dal processo i contrattisti a termine delle province impegnati su altri fronti per i quali è scattata la proroga dei termini disposta con il decreto di fine anno. numeri alla mano, con la copertura del migliaio di posti previsti dal bando giustizia il fabbisogno di personale per gli uffici giudiziari scenderebbe a circa 7mila posizioni, poco meno degli esuberi effettivi che si potrebbero determinare con l'attuazione della legge delrio sulle province. ovviamente lo scambio non sarà così semplice e immediato (non tutti i dipendenti sono fungibili) ma a colpo d'ochhio l'equivalenza dimostra che la grande operazione di mobilità in corso avrebbe tutti i margini per essere condotta in porto senza sacrifici. & 789 & very low & Low & Socio-Economic & NA & NA & 2015-01-22 & 2015 & 1 & ECO
Frame & v.low & National & 500-1000 & -0.5155635 & -0.1667445 & -0.9728619 & 0.5015415 & 1.3050799 & 3.2 & 0.6840199 & -1.2389660 & Payer & Domestic & Domestic & Domestic & Domestic|ECO & Positive\\
Italy & http://www.ansa.it/valledaosta/notizie/2018/02/06/famigliavoucher-asilo-nido-a-770-nuclei\_c617b9ef-6896-4e70-8fe0-2b0436ed1522.html & 475 & ANSA.it & Private/Non-Public & Online only & National & very low = CP mentioned once & Public services & Positive & Subnational & No myth & NA & NA & NA & NA & NA & NA & NA & NA & Italy & famiglia,voucher asilo nido a 770 nuclei - valle d'aosta & 2018-02-06 & fondo sociale europeo & (ansa) - aosta, 6 feb - sono 770 le famiglie che beneficeranno dei voucher a sostegno della spesa per gli asili nido, finanziati dal fondo sociale europeo. lo ha riferito oggi l'assessore regionale alla sanità e politiche sociali, luigi bertschy, rispondendo a una interpellanza presentata da andrea padovani (mouv-coalition citoyenne). "e' questa - secondo bertschy - una prima risposta alle necessità delle famiglie" nell'ambito di un più ampio progetto sul 'fattore famiglia'. secondo padovani "il fattore famiglia, così come si intende normalmente, è fallito". il consigliere di minoranza ha inoltre osservato che "il voucher per gli asili è un provvedimento spot che non risolve la grave crisi economica che le famiglie stanno affrontando". & 112 & very low & Low & Socio-Economic & NA & NA & 2018-02-06 & 2018 & 3 & ECO
Frame & v.low & National & <500 & -0.5155635 & -0.1667445 & -0.9728619 & 0.5015415 & 1.3050799 & 3.2 & 0.6840199 & -1.2389660 & Payer & Domestic & Domestic & Domestic & Domestic|ECO & Positive\\
Italy & http://www.ilsole24ore.com/art/notizie/2018-05-16/lega-cinque-stelle-reddito-cittadinanza-pagano-i-burocrati-bruxelles-164421.shtml?uuid=AE5iiVpE & 428 & Il Sole 24 ORE & Private/Non-Public & Online and Offline & National & high = CP is most important issue in story (can also cover other issues) & Social justice & Balanced & EU + National & No myth & Improve governance & Factual & EU + National & No myth & NA & NA & NA & NA & Italy & m5s-lega, perché gli "eurocrati" di bruxelles non possono finanziare il reddito di cittadinanza & 2018-05-17 & fondo sociale europeo & una tra le principali avversarie del governo lega-cinque stelle? l'europa "dei burocrati". la fonte di finanziamento del reddito di cittadinanza, tra i piatti forti del "contratto per il governo del cambiamento" ? neppure a dirlo, sempre l'europa. a quanto si apprende dalla bozza dell'intesa, ancora in via di elaborazione tra salvini e di maio, i due partiti sarebbero intenzionati a finanziare il reddito di cittadinanza attingendo al fondo sociale europeo, uno strumento per sostenere politiche di contrasto alla disoccupazione e all'esclusione sociale. il problema è che, a quanto è emerso da bruxelles, la commissione europea è intenzionata a trasformare il fondo in uno strumento di stimolo alle "riforme strutturali" chieste dalla ue. tradotto: il denaro sarà elargito solo in cambio di riforme allineate al progetto di una maggiore coesione fra gli stati membri. non proprio quanto si annuncia nell'accordo fra salvini e di maio, dove spuntano - fra le altre cose - la proposta di "ridiscutere il contributo italiano alla ue", una "modifica radicale del fiscal compact" e un intero capitolo sul "business dell'immigrazione". con toni lontani dalle spinte solidali che si intravedono per il prossimo bilancio comunitario. cos'è il fondo sociale e come vorrebbero usarlo lega e cinque stelle il fondo sociale europeo è il più antico fra i fondi strutturali, le risorse impiegate dall'unione per affrontare alcuni temi chiavi sull'agenda dell'unione. per il periodo che va dal 2014 al 2020 si parla di un budget di circa 87 miliardi di euro in arrivo dal bilancio comunitario e altri 37 miliardi messi sul piatto dagli stati membri, per un totale di circa 120 miliardi di euro. le risorse del fse sono destinate però alle cosiddette politiche attive, come la formazione di lavoratori disoccupati, o a misure di inclusione sociale per le fasce più deboli della popolazione. qual è il nesso con il reddito di cittadinanza, di fatto un sussidio (passivo) da 780 euro per le persone sotto a una certa soglia di reddito? la risposta è nella bozza di programma di cinque stelle e lega, quando gli autori spiegano che "andrà avviato un dialogo nelle sedi comunitarie al fine di applicare il provvedimento a8 0292/2017" approvato dall'europarlamento nell'ottobre dell'anno scorso. il testo in questione invita la commissione ad esaminare le "possibilità di finanziamento per aiutare ciascuno stato membro a istituire un regime di reddito minimo, ove inesistente", oltre a "a monitorare specificamente l'utilizzo del 20\% della dotazione complessiva dell'fse destinato alla lotta contro la povertà e l'esclusione sociale". insomma: il neogoverno lega-cinque stelle si candiderebbe tra i paesi beneficiari dei nuovi finanziamenti, assicurandosi parte delle risorse necessarie alle misura. solo una parte, appunto. il costo complessivo del reddito di cittadinanza è stimato da lega e cinque stelle a 17 miliardi di euro l'anno. il bacino del fondo sociale potrebbe garantire, come ha scritto oggi il sole 24 ore, non più di 330 milioni di euro. una percentuale minima. il paradosso politico. e i rischi effettivi l'attenzione ai soldi della ue, e al ruolo della commissione, suona abbastanza inusuale rispetto agli appelli alla "rinegoziazione dei trattati europei" che salvini ha posto sul tavolo fin dal primo giorno di dialogo con i cinque stelle. ma il cortocircuito rischia di essere più pratico che politico. l'agenzia ansa ha preso visione in anteprima della proposta di un nuovo fse elaborato dalla commissione per il periodo 2021-2017. il nuovo strumento, forte di un budget ue di circa 90 miliardi di euro, affiderà la gestione delle risorse ai governi (e non alla regioni, come succede ora) e si concentrerà sul conseguimento di una maggiore "coesione sociale". da attuarsi con riforme che vadano nella stessa direzione della ue, il vero vincolo per ottenere i finanziamenti ambìti (anche) dall'eventuale maggioranza tra grillini e leghisti. le premesse potrebbero essere migliori. tra gli obiettivi ue c'è l'integrazione di stranieri provenienti da paesi terzi, cioè extraeuropei. non il contrario, ma quasi, di quello che si legge al capitolo "immigrazione" della bozza di intesa tra salvini e di maio: dal "superamento della convenzione di dublino" alla "riduzione della pressione dei flussi sulle frontiere esterne". & 696 & high & High & Socio-Economic & Governance & NA & 2018-05-17 & 2018 & 3 & ECO
Frame & high-very high & National & 500-1000 & -0.5155635 & -0.1667445 & -0.9728619 & 0.5015415 & 1.3050799 & 3.2 & 0.6840199 & -1.2389660 & Payer & Domestic & European & Mixed & Domestic|ECO & Neutral\\
Italy & http://www.corriere.it/esteri/17\_settembre\_08/tajani-migranti-sentenza-apre-porta-riforma-dublino-3cae1da2-940a-11e7-8bb4-7facc48f24a3.shtml & 448 & Corriere della Sera & Private/Non-Public & Online and Offline & National & very low = CP mentioned once & Solidarity to poor countries/regions & Balanced & National + Other country & No myth & Political leverage & Balanced & EU + Other country & No myth & NA & NA & NA & NA & Italy & tajani: "migranti, la sentenza apre  la porta alla riforma di dublino" & 2017-09-07 & fondi strutturali & il presidente del parlamento europeo è soddisfatto: "il dispositivo riconosce l'emergenza profughi e dice che non si può scaricarla interamente sulle spalle di pochi paesi". "la sentenza della corte di giustizia contro il ricorso dei paesi di visegrad sul ricollocamento dei rifugiati -- dice il presidente del parlamento europeo, antonio tajani -- è molto positiva, perché conferma sia la validità della posizione del parlamento europeo che quella della decisione per sé. inoltre contiene un altro elemento importante, spingendo verso il cambiamento del regolamento di dublino. il dispositivo infatti riconosce l'emergenza profughi e dice che non si può scaricarla interamente sulle spalle di pochi paesi. aggiunta alla presa di posizione della cancelliera angela merkel e alla vigilia della risoluzione che il parlamento si appresta a prendere in ottobre, la sentenza della corte agevola la riforma di dublino, che dovrà eliminare la regola di tenere i rifugiati nello stato di primo ingresso". ma l'accordo sui ricollocamenti scadrà comunque il 26 settembre. cosa significa questo? "nulla ai fini della procedura d'infrazione, che rimane in quanto riguarda la fase precedente e, in assenza di fatti nuovi, porterà alla sanzione contro i morosi. è un punto politico importante. questi paesi hanno fatto venire meno la solidarietà verso paesi come l'italia, che si erano prodigati quando volevano uscire dal giogo sovietico e dalle dittature comuniste. anche adesso continuiamo a sostenerli, con i fondi strutturali europei, visto che sono beneficiari netti. ma la solidarietà non può essere a senso unico, altrimenti non ha senso stare nell'unione. detto questo sono convinto che occorra aiutare anche finanziariamente i paesi come l'ungheria a proteggere i confini esterni, ma sono due cose completamente diverse. ricordo poi che in questo caso non si tratta di riallocare masse sterminate, parliamo di poche migliaia di rifugiati legalmente riconosciuti, non di migranti economici. e in condizioni di flessibilità, visto che possono fare richieste specifiche, per esempio più gente che parli la loro lingua o l'inglese, o più donne con bambini". tornando alla scadenza del 26 settembre, che come dice lei non mette in discussione la procedura d'infrazione: l'accordo verrà prorogato? "intanto ricordo che ci sono circa 4 mila profughi in italia e 2 mila in grecia che potrebbero essere riallocati domani mattina. finora ne sono stati riallocati 28 mila su 160 mila. altri 6 mila non sarebbero la panacea, ma significherebbero un altro piccolo passo in avanti. certo sarebbe auspicabile che l'accordo venga rinnovato e il commissario avramopoulos spinge in questo senso". l'europarlamento sta lavorando sulla proposta della commissione per riformare dublino. a che punto siamo? "facciamo la nostra parte. stiamo modificando la proposta, anche in direzione dell'interesse dell'italia e dei paesi con frontiere esterne". in che senso? "la commissione suggerisce che il ricollocamento scatti una volta raggiunto il 150\% della quota di rifugiati spettante a un determinato paese. noi proponiamo invece che scatti prima, al 100\%, cioè appena superato il limite assegnato a un dato paese. nello stesso pacchetto dovremo anche stabilire regole certe: le liste di paesi di provenienza la cui situazione giustifica la richiesta d'asilo devono essere omogenee, le stesse per tutti. per fare un esempio, se uno non può chiedere lo status di rifugiato in germania non deve poterlo fare neanche in italia e in nessun altro paese della ue. non dovranno cioè ripetersi casi come quello di gorizia, dove un gruppo di rifugiati respinto da austria e germania rimase parcheggiato nella speranza di essere accolto in italia. voteremo in commissione il 12 ottobre e mi auguro che entro l'autunno ci possa essere il voto definitivo dell'assemblea. sarà una spinta forte verso il consiglio dei ministri, che fin qui è rimasto fermo. certo la posizione di angela merkel, in favore della riforma, è un aiuto. il 22 settembre ne parlerò anche col presidente macron a parigi". mercoledì prossimo, il presidente della commissione juncker fa il discorso sullo stato dell'unione davanti al parlamento. l'ultima sua visita a strasburgo in luglio è stata burrascosa; trovandosi in un'aula semivuota disse: "siete ridicoli". incidente chiuso? cosa vi aspettate da lui? "proprio oggi juncker è venuto alla conferenza dei presidenti dei gruppi parlamentari insieme al vicepresidente timmermans per anticipare le linee del discorso. mi sembra che l'atteggiamento nei confronti del parlamento sia di grande rispetto. l'incidente è chiuso. io martedì parlerò in aula, per dire che il parlamento deve diventare il luogo dove si discute il futuro dell'ue. da juncker aspettiamo messaggi positivi soprattutto sull'economia reale, l'ambiente e il miglioramento delle istituzioni per renderle più conformi alle istanze dei cittadini. aspettiamo anche indicazioni sulla brexit, che comunque non è la priorità assoluta. l'europa va avanti anche se il regno unito decide di uscire". sulla brexit lei propone di posporre fino a dicembre l'esame del consiglio europeo. perché? "in assenza di proposte precise di londra sui punti chiave, in primis il destino dei 3,5 milioni di cittadini europei residenti nel regno unito, non si può che rinviare l'esame". & 838 & very low & Low & Values & Power & NA & 2017-09-07 & 2017 & 2 & ECO
Frame & v.low & National & 500-1000 & -0.5155635 & -0.1667445 & -0.9728619 & 0.5015415 & 1.3050799 & 3.2 & 0.6840199 & -1.2389660 & Payer & Domestic & European & Mixed & Domestic|ECO & Neutral\\
Italy & https://www.ilsole24ore.com/art/impresa-e-territori/2019-04-05/nei-piccoli-centri-ue-finanzia-banda-larga-573-milioni--210904.shtml?uuid=AB0Dx3kB & 413 & Il Sole 24 ORE & Private/Non-Public & Online and Offline & National & high = CP is most important issue in story (can also cover other issues) & Infrastructure & Positive & EU + National & No myth & NA & NA & NA & NA & NA & NA & NA & NA & Italy & nei piccoli centri la  ue finanzia  la banda larga con 573 milioni & 2019-04-05 & politica di coesione & internet veloce in oltre 7mila comuni italiani e per più di 12 milioni di cittadini. l'unione europea finanzierà con oltre 573 milioni di euro la diffusione della banda larga veloce in italia. lo ha annunciato ieri la commissione ue, che ha adottato un nuovo programma nell'ambito della politica di coesione: i finanziamenti copriranno il 60\% dei costi ammissibili dei progetti. l'obiettivo dell'iniziativa è ridurre il cosiddetto "digital divide" nel nostro paese, portando l'accesso veloce a internet in aree in cui non è al momento disponibile. secondo le stime della commissione europea, rese note ieri, il programma interesserà oltre 7mila comuni italiani, per un totale di 12,5 milioni di abitanti, il 20\% della popolazione italiana, e quasi un milione di imprese coinvolte. il progetto fa parte del "piano digitale italiano - banda ultralarga", la strategia nazionale per la rete d'accesso di nuova generazione, e riguarderà, in particolare, le cosiddette "aree bianche". si tratta di quelle zone, presenti in tutte le regioni italiane, nelle quali le infrastrutture di banda larga sono inesistenti ed è improbabile che le sole forze di mercato riescano a realizzare il necessario potenziamento dei collegamenti internet. l'azione dovrà essere completata al massimo entro la fine del 2020. la strategia nazionale per la banda ultralarga, di cui fa parte il programma, punta ad assicurare una velocità di connessione ad internet pari ad almeno a 100 megabit al secondo (mbps) nell'85\% delle case italiane e in tutti gli edifici pubblici, soprattutto scuole e ospedali. e di 30 mbps in tutti gli altri luoghi. l'italia - va ricordato - è il secondo percettore di fondi di investimento strutturali, inclusi i fondi della politica di coesione, con 44,7 miliardi di risorse destinate per il periodo 2014-2020. di questi, 1,9 miliardi sono dedicati agli investimenti nei servizi digitali e nella banda larga. il nostro paese è anche il secondo maggiore beneficiario del piano juncker, il piano europeo per gli investimenti, con 63,3 miliardi già mobilitati e oltre 286mila piccole e medie imprese che hanno potuto trarre vantaggio da un migliore accesso ai finanziamenti. per il prossimo bilancio europeo la commissione propone 43,5 miliardi per fondi di coesione assegnati all'italia, per supportare la ripresa economica del paese. una dotazione finanziaria che registra un aumento di 8,5 miliardi di euro, nonostante un contesto di generale riduzione dei fondi destinati alla politica di coesione. © riproduzione riservata progetto realizzato con il contributo finanziario della commissione europea. dei contenuti editoriali sono responsabili esclusivamente gli autori e il gruppo 24 ore & 426 & high & High & Socio-Economic & NA & NA & 2019-04-05 & 2019 & 3 & ECO
Frame & high-very high & National & <500 & -0.5155635 & -0.1667445 & -0.9728619 & 0.5015415 & 1.3050799 & 3.2 & 0.6840199 & -1.2389660 & Payer & Domestic & European & Mixed & Domestic|ECO & Positive\\
\addlinespace
Italy & http://lanuovaferrara.gelocal.it/ferrara/cronaca/2016/12/02/news/prove-di-apertura-all-ex-teatro-verdi-1.14508318 & 404 & La Nuova Ferrara & Private/Non-Public & Online and Offline & Regional/Local & very low = CP mentioned once & Cultural development & Positive & Subnational & No myth & NA & NA & NA & NA & NA & NA & NA & NA & Italy & prove di apertura all'ex teatro verdi  - cronaca - la nuova ferrara & 2016-12-03 & fondo europeo di sviluppo regionale & si schiudono le porte per una visita. il progetto di rinascita sarà presentato all'ariostea l'ex teatro verdi attende da più di 30 anni: ancora non si parla di riapertura, ma domani schiuderà le porte in segno benaugurante. inizia infatti il percorso partecipato che porterà alla sua riapertura il progetto è finanziato dall'asse 6 "città attrattive e partecipate" del por fesr emilia romagna 2014-2020: (un programma alimentato dal fondo europeo di sviluppo regionale); l'obiettivo è dare vita a un "laboratorio aperto" sui temi del turismo e della mobilità sostenibili, con particolare attenzione alla bicicletta. la mattinata di domani sarà organizzata in due momenti distinti. alle 10 sarà possibile per i cittadini visitare l'ex teatro verdi (situato nell'omonima piazza verdi) con un inquadramento storico a cura di francesco scafuri, responsabile dell'ufficio ricerche storiche del servizio beni monumentali centro storico del comune di ferrara, e la presenza dei referenti di città della cultura - cultura della città, che illustreranno il progetto di ristrutturazione. alle 11 l'iniziativa proseguirà nell sala agnelli della biblioteca ariostea, dove verrà presentato il percorso di rigenerazione. dopo l'introduzione del sindaco di ferrara, tiziano tagliani, ci sarà un intervento di daniela ferrara della struttura dell'autorità di gestione del por fesr emilia romagna. i lavori proseguiranno con interventi tematici: ilda curti porterà l'esperienza delle case di quartiere di torino, a cui seguirà la presentazione di casi di successo: le "serre dei giardini margherita" di bologna a cura di kilowatt e "upcycle" di milano da parte di avanzi. & 257 & very low & Low & Socio-Economic & NA & NA & 2016-12-03 & 2016 & 2 & ECO
Frame & v.low & Regional & <500 & -0.5155635 & -0.1667445 & -0.9728619 & 0.5015415 & 1.3050799 & 3.2 & 0.6840199 & -1.2389660 & Payer & Domestic & Domestic & Domestic & Domestic|ECO & Positive\\
Italy & http://www.ansa.it/toscana/notizie/2017/05/22/fondi-ue-58-mln-per-la-formazione\_2ce953bb-4d2c-4b26-8822-5168e99990f9.html & 399 & ANSA.it & Private/Non-Public & Online only & National & very high = CP is most important issue + CP is mentioned in title/headline & Jobs & Positive & Subnational & No myth & NA & NA & NA & NA & NA & NA & NA & NA & Italy & fondi ue: 5,8 mln per la formazione - toscana & 2017-05-22 & fondo sociale europeo & (ansa) - firenze, 22 mag - un nuovo stanziamento di 5,8 milioni di euro è stato recentemente deliberato dalla giunta toscana su proposta dell'assessore alla formazione e al lavoro, cristina grieco. serviranno a finanziare ulteriori progetti di formazione territoriale, facendo così scorrere la graduatoria approvata lo scorso febbraio che aveva consentito, grazie ad uno stanziamento di 3,5 milioni di euro, di approvare 53 progetti su 156. ciò consentirà di ampliare le figure e le aree professionali coinvolte. la regione conta in questo modo di arrivare a finanziare pressoché tutti i circa 155 progetti risultati finanziabili in quanto pienamente rispondenti alle richieste del bando. la metà dello stanziamento deriva dai fondi del fse (il fondo sociale europeo), il 34\% dalla quota statale e il 15\% da risorse regionali. & 128 & very high & High & Socio-Economic & NA & NA & 2017-05-22 & 2017 & 2 & ECO
Frame & high-very high & National & <500 & -0.5155635 & -0.1667445 & -0.9728619 & 0.5015415 & 1.3050799 & 3.2 & 0.6840199 & -1.2389660 & Payer & Domestic & Domestic & Domestic & Domestic|ECO & Positive\\
Italy & http://www.ilsole24ore.com/art/commenti-e-idee/2015-11-23/futuro-rischio-le-politiche-regionali-070816.shtml?uuid=ACdxOSfB & 397 & Il Sole 24 ORE & Private/Non-Public & Online and Offline & National & very high = CP is most important issue + CP is mentioned in title/headline & Institutional bargaining over funding & Balanced & EU & No myth & NA & NA & NA & NA & NA & NA & NA & NA & Italy & futuro a rischio per le politiche regionali & 2015-11-23 & fondi strutturali & distratta dalle risorse della vecchia programmazione ancora da spendere e a rischio disimpegno e dai ritardi accumulati in avvio del periodo 2014-2020, l'italia si sta perdendo il dibattito, già iniziato, per modificare "in modo sostanziale" il bilancio europeo. l'ultima a chiederlo in ordine di tempo è stata la corte dei conti ue, sollecitando chiarezza soprattutto sui risultati raggiunti. le politiche regionali e le risorse a esse destinate nel bilancio potrebbero essere le vittime principali di questa riforma che ha visto il suo momento iniziale di confronto a bruxelles nella prima conferenza annuale sul bilancio (nessun italiano tra i relatori) intitolata non a caso "budget ue focalizzato sui risultati". sotto la pressione di nuove e in qualche caso imprevedibili esigenze, le principali voci attuali di spesa rischiano di essere nettamente ridimensionate. messaggio, questo, che è arrivato dai vertici della commissione, ma anche da alcuni ministri non del tutto marginali negli equilibri europei, come il tedesco wolfgang schäuble, esplicito nel chiedere che "i soldi europei siano spesi per raggiungere obiettivi europei", al contrario di quanto è avvenuto finora con il budget dell'unione modellato dai compromessi politici frutto di ragioni storiche e in difesa di interessi nazionali. dopo decenni in cui più di due terzi del bilancio (quasi mille miliardi nel periodo 2014-2020) sono andati all'agricoltura e alla coesione regionale e nell'impossibilità oggettiva di aumentare le entrate, diventa inevitabile agire sul mix delle spese. le nuove esigenze riguardano prima di tutto la politica estera dell'unione, dal controllo delle frontiere esterne all'accoglienza delle centinaia di migliaia di migranti che arrivano in europa per sfuggire alle guerre e alla povertà. altra voce di spesa crescente è legata al cambiamento climatico, causa di emergenze imprevedibili a cui bisognerebbe rispondere con l'immediatezza e la flessibilità che le regole di oggi non consentono. l'elenco è destinato ad allungarsi. in molti, dunque, stanno pensando che 360 miliardi destinati alle politiche regionali siano troppi e mal distribuiti, tanto più che nessuno è mai riuscito a misurarne in modo oggettivo la reale efficacia in termini di sviluppo, crescita economica e nuovi posti di lavoro. da queste basi si andrà dipanando il dibattito nei prossimi mesi, con alcuni paletti già fissati: non si può chiedere ai contribuenti europei di pagare di più, perciò bisogna spendere meglio possibile le risorse limitate disponibili per coprire vecchie e nuove esigenze. le politiche (leggi "spese") di cui non si riuscirà a dimostrare l'utilità anche in termini economici, saranno inevitabilmente ridimensionate. politiche regionali e fondi strutturali, a cui l'italia attinge per oltre 40 miliardi come secondo paese beneficiario, sono sotto tiro. perciò, care regioni e cari ministeri, datevi da fare per utilizzare presto e bene le risorse della programmazione 2014-2020, su progetti veri e sostenibili: presto i soldi per i "por" e i "pon" - se ci saranno ancora - saranno comunque molti, molti di meno. & 481 & very high & High & Power & NA & NA & 2015-11-23 & 2015 & 1 & POL
Frame & high-very high & National & <500 & -0.5155635 & -0.1667445 & -0.9728619 & 0.5015415 & 1.3050799 & 3.2 & 0.6840199 & -1.2389660 & Payer & European & European & European & European|POL & Neutral\\
Italy & http://livesicilia.it/2017/07/16/in-sicilia-la-formazione-e-ferma-ecco-come-funziona-in-lombardia\_872586/ & 431 & Live Sicilia & Private/Non-Public & Online only & Regional/Local & low = CP mentioned more times but NOT important part of story (mainly about others issues) & Public services & Positive & EU + Subnational & No myth & NA & NA & NA & NA & NA & NA & NA & NA & Italy & in sicilia la formazione è ferma "ecco come funziona in lombardia" & 2017-07-16 & fondo sociale europeo & gianni bocchieri (in foto) guida il settore a milano, ed è di origini ragusane: "rovesciare la prospettiva: prima la persona". vota 1/5 2/5 3/5 4/5 5/5 0/5 0 voti stampa palermo - in sicilia i corsi non partono ancora. ed è così ormai da più di due anni. da quando, cioè, i vari bandi per la formazione professionale sono stati travolti dai ricorsi ai tribunali amministrativi. ricorsi che in qualche caso hanno costretto l'assessorato alla "marcia indietro", a rivedere i bandi per gli enti. anzi, a revocarli, per far ripartire tutto con bandi nuovi. a loro volta impugnati. e oggi, nel caso dell'avviso 8 da 136 milioni, in attesa di due pronunce: il cga a fine luglio, il tar a settembre. "ma i corsi partiranno comunque" ha fatto sapere l'assessore marziano. una ripartenza attesa da circa 4 mila operatori. gli unici a poter essere "ripescati", soldi alla mano. e che da tempo lamentano i "guai" legati a questo stallo: senza reddito, costretti a prestiti e a sacrifici enormi. una formazione che soffre, quella siciliana. non solo per i ritardi attuali, ma anche per la zavorra ereditata dal passato: fino a pochi anni fa erano oltre 7 mila gli addetti (la metà di tutta italia), enti travolti dagli scandali, finiti nelle inchieste giudiziarie. c'è invece, un'altra formazione. e il confronto con la lombardia, oggi, va al di là dello stereotipo del nord "efficiente" a fronte di un sud sempre boccheggiante. anche perché, a dirigere il dipartimento formazione e istruzione della regione lombardia è un burocrate di origini siciliane. ragusane, per l'esattezza. gianni bocchieri guida gli uffici della formazione lombarda. e il suo racconto è un insieme di buone pratiche da seguire. un modo di guardare al settore da un'altra prospettiva. dove la "domanda" viene prima dell'offerta. e dove nessuno pensa di rivolgersi ai tribunali per impugnare i vari avvisi. per un motivo molto semplice: "in lombardia - racconta bocchieri - la situazione è molto diversa dall'isola. se dal punto di vista burocratico-amministrativo possiamo chiamarli avvisi, dal punto di vista pratico funzionano in modo opposto rispetto a quelli delle altre regioni. in lombardia non finanziano gli operatori che erogano la formazione od i servizi al lavoro, ma le persone che ne hanno bisogno, che fanno richiesta alla regione la quale paga all'ente scelto dalla persona stessa la quota di risorse necessaria a coprire i servizi erogati. non sono quindi le commissioni di valutazione regionali a fare le graduatorie di valutazione. sono le singole persone che fanno la loro 'graduatoria sul campo', scegliendo tra i circa 700 enti accreditati per la formazione e per i servizi al lavoro". da lì, insomma, oltre a un più stretto legame tra le aspirazioni degli allievi e i progetti previsti, anche una specie di "selezione naturale" dei soggetti che "fanno formazione". "oltre a questo pilastro - spiega bocchieri - il sistema lombardo si regge su quello dell'erogazione dei finanziamenti 'a risultato raggiunto', a costi standard ossia a costi predefiniti e non a piè di lista per coprire la retribuzione dei formatori e le spese gestionali degli enti. con questi accorgimenti, assieme ad una programmazione con tempi certi per alcune tipologie di formazione - penso ai percorsi di istruzione e formazione professionale per l'assolvimento dell'obbligo d'istruzione dei giovani fino a 16 anni - garantiamo il regolare avvio di tutte le annualità formative ed un sistema di servizi di formazione ed accompagnamento al lavoro sempre aperto ed orientato ai fabbisogni dei destinatari, più che alle esigenze degli operatori". un sistema capovolto, insomma. prima il soggetto da formare. dopo, solo dopo, gli enti. "secondo me in sicilia, come è stato fatto in lombardia, - ribadisce bocchieri - occorre rovesciare la prospettiva: mettere la persona al centro delle politiche e dei processi amministrativi, per orientare il sistema verso una formazione che si concluda con l'attivazione delle persone nella ricerca attiva di un lavoro e con l' inserimento lavorativo. del resto è questa la direzione indicata dagli organismi europei che vigilano sulla spesa del fondo sociale europeo. credo che non sia un caso se quest'anno tra le quattro buone pratiche selezionate a livello europeo per la categoria istruzione e formazione - aggiunge il dirigente - c'è proprio il nostro sistema della dote unica lavoro di regione lombardia. un modello di sostegno per l'inserimento lavorativo e per la riqualificazione professionale avviato a fine 2013, che la commissione ha esaminato con attenzione e che ha ritenuto di dover diffondere a livello europeo come buona pratica da replicare". un altro punto di forza del sistema lombardo sarebbe la possibilità di "far fare esperienze lavorative all'interno dei percorsi, nella duplice forma dell'alternanza scuola lavoro o dell'apprendistato formativo. una spinta concreta in tal senso - spiega bocchieri - potrebbe consistere in una previsione che vincoli gli enti a realizzare almeno un'esperienza lavorativa significativa all'interno di ogni percorso formativo". chi deve portare avanti queste attività? quante persone servono? nell'isola, come detto, specie sotto elezioni si assiste all'impennata delle assunzioni negli enti. il numero totale è giunto persino oltre quota settemila, pochi anni fa. adesso, tra revoche di accreditamento e fallimenti, i lavoratori che possono ancora trovare spazio nel settore sono circa quattromila. tanti? pochi? "in lombardia - dice bocchieri - abbiamo circa 700 operatori accreditati per la formazione e per i servizi al lavoro, che sono liberi di assumere chi vogliono per realizzare le migliori attività formative, fermo restando i requisiti professionali minimi fissati dalla disciplina regionale per essere accreditati, ossia per essere abilitati ad erogare la formazione sul territorio. non abbiamo un albo dei formatori perché sono gli enti che devono liberamente sceglierli per fornire un servizio non al di sotto degli standard anche qualitativi fissati dalla regione". anche in questo caso, quindi, il numero degli operatori si "cuce" sulle necessità, sulla domanda. un meccanismo che, spiega sempre bocchieri, ha reso la formazione lombarda immune, negli ultimi anni, da complicazioni di natura giudiziaria: "anche in lombardia, prima di liberarci della 'tirannia del bando', delle valutazioni, delle graduatorie, degli accessi agli atti e dei ricorsi, - spiega - ci sono stati casi di cui si è occupata la magistratura. il rimedio è stato proprio l'introduzione del sistema dotale, per cui l'ente è scelto dalla singola persona e non da una commissione di valutazione. certo, con questo sistema, la burocrazia regionale, in primis il direttore generale, cede una parte della sua presunta sovranità anche solo per definire i criteri di valutazione. ma è proprio questo arretramento della burocrazia - conclude bocchieri - che può rendere più trasparente il sistema, che si libera così anche dell'alea dei criteri che accompagna l'uscita di ogni avviso". share domenica 16 luglio 2017 - 06:00 & 1118 & low & Low & Socio-Economic & NA & NA & 2017-07-16 & 2017 & 2 & ECO
Frame & low-medium & Regional & +1000 & -0.5155635 & -0.1667445 & -0.9728619 & 0.5015415 & 1.3050799 & 3.2 & 0.6840199 & -1.2389660 & Payer & Domestic & European & Mixed & Domestic|ECO & Positive\\
Italy & http://ilpiccolo.gelocal.it/trieste/cronaca/2017/02/08/news/riqualificazione-energetica-quarnero-al-top-1.14848253 & 477 & Il Piccolo & Private/Non-Public & Online and Offline & Regional/Local & low = CP mentioned more times but NOT important part of story (mainly about others issues) & Infrastructure & Factual & Other country & No myth & Environment/green/low-carbon & Positive & Other country & No myth & NA & NA & NA & NA & Italy & riqualificazione energetica, quarnero al top  - cronaca - il piccolo & 2017-02-09 & fondo europeo di sviluppo regionale & interventi su 141 edifici nella regione con il sostegno di fondi ue. investiti oltre 12 milioni fiume. il boom delle facciate. è stata definita in questo modo la corsa all'efficientamento energetico che negli ultimi due anni ha riguardato fiume e la regione del quarnero e gorski kotar. gli interventi di riqualificazione energetica, che hanno il sostegno dell'ue, sono stati compiuti in 141 edifici, di cui 91 a fiume. per i lavori di ristrutturazione delle facciate, che hanno permesso di ridurre il consumo di energia migliorando condizioni abitative e benessere termico, sono stati spesi 91,3 milioni di kune (12,3 milioni di euro). il fondo croato per l'efficienza energetica ha assicurato ecobonus per un valore di 36,5 milioni di kune (4,9 milioni di euro), coprendo oltre un terzo delle spese sostenute dagli inquilini. fiume (con i suoi 130mila abitanti) e la regione sono al primo posto nella graduatoria nazionale relativa ai lavori di coibentazione, per i quali nel biennio 2015-2016 è stato manifestato un forte interesse. ad esempio a zagabria (che conta 790 mila abitanti) la riqualificazione ha riguardato 35 immobili, mentre in tutta la regione zagabrese la ristrutturazione delle facciate è stata compiuta in 55 edifici. a spalato l'efficientamento energetico ha riguardato 45 edifici (in tutta la contea 64 immobili), mentre a osijek è stato ritoccato un solo involucro edilizio. oltre a questi lavori di miglioria nei condomini, sono stati rimessi a nuovo centinaia di tetti. a stimolare gli inquilini a richiedere mutui per questi interventi sono stati non solo le garanzie di risparmio energetico, che in alcuni casi ha superato il 50\%, ma anche la concessioni di prestiti a condizioni assolutamente agevolate. al bando per l'efficientamento energetico in croazia hanno aderito gli abitanti di 624 edifici di tutto il paese, dei quali 140 dislocati nella regione di fiume. non solo edifici privati: proprio in questi giorni - presente anche il ministro croato dell'edilizia, lovro kuscevic - si è tenuta nella scuola elementare fiumana nikola tesla (ex manin) la cerimonia che ha segnato la chiusura dei lavori di rinnovamento energetico: un progetto che ha richiesto una spesa di 850mila euro. il 30\% dell'investimento è stato coperto dal fondo europeo di sviluppo regionale, mentre il 40\% è stato coperto dal fondo croato per l'efficienza energetica. (a.m.) & 386 & low & Low & Socio-Economic & Socio-Economic & NA & 2017-02-09 & 2017 & 2 & ECO
Frame & low-medium & Regional & <500 & -0.5155635 & -0.1667445 & -0.9728619 & 0.5015415 & 1.3050799 & 3.2 & 0.6840199 & -1.2389660 & Payer & European & European & European & European|ECO & Neutral\\
\addlinespace
Italy & https://www.tpi.it/2018/10/19/kuciak-accusato-suicidio-carcere/ & 459 & TPI & Private/Non-Public & Online only & National & low = CP mentioned more times but NOT important part of story (mainly about others issues) & Fraud/Corruption & Negative & Other country & 7.Fraud & NA & NA & NA & NA & NA & NA & NA & NA & Italy & caso kuciak, uno degli accusati muore in carcere: probabile suicidio & 2018-10-19 & fondi strutturali & uno dei principali accusati per l'omicidio del giornalista slovacco ján kuciak e della sua fidanzata martina kušnírová, l'individuo che la polizia ha sino ad ora identificato come ronald r., avrebbe commesso suicidio nel corso della notte. l'uomo era stato arrestato insieme ad altre sette persone nel corso del raid di settembre ed a lui si era arrivati per via della sua somiglianza all'identikit di una persona notata nei giorni precedenti all'agguato mentre si aggirava nei dintorni dell'abitazione della coppia uccisa. ronald è il nipote di miroslav marček, a sua volta accusato di essere stato l'autista del commando omicida, ma a differenza dello zio era a piede libero perché stranamente la polizia lo aveva rilasciato dopo un breve interrogatorio. la portavoce della polizia di nitra, božena bruchterová, ha immediatamente respinto ogni ipotesi divergente da quella del suicidio, sostenendo che i primi rilievi autoptici escludono la possibilità che altre persone fossero presenti quando l'indiziato si è tolto la vita e che una autopsia chiarirà presto ogni possibile dubbio. ján kuciak è stato ucciso a soli 27 anni insieme alla fidanzata nella sua casa a velka maca, una località dell'ovest della slovacchia, il 22 febbraio 2018. kuciak ha condotto molte inchieste su scandali che hanno coinvolto la politica slovacca in casi di corruzione, frode e evasione fiscale. in particolare, kuciak stava scavando su fondi europei che giungono in slovacchia, e che ammontano a 15 miliardi di euro nel periodo 2014-2020, sui quali le organizzazioni criminali internazionali da tempo avevano posto la propria attenzione. dietro l'assassinio del giovane giornalista ci sarebbe proprio il suo ultimo articolo, che si concentra sugli affari di alcuni uomini italiani in slovacchia legati ad ambienti vicini alla 'ndrangheta calabrese. secondo le indagini svolte da kuciak, questi "uomini d'affari", che da anni investono in aziende slovacche, avrebbero avuto accesso ai fondi strutturali provenienti da bruxelles. decine di milioni di euro ricevuti per finanziare progetti fittizi e che presumibilmente sono stati portati in calabria, principale centro di interessi di una delle mafie internazionali più potenti al mondo. & 348 & low & Low & Governance & NA & NA & 2018-10-19 & 2018 & 3 & POL
Frame & low-medium & National & <500 & -0.5155635 & -0.1667445 & -0.9728619 & 0.5015415 & 1.3050799 & 3.2 & 0.6840199 & -1.2389660 & Payer & European & European & European & European|POL & Negative\\
Italy & https://napoli.repubblica.it/cronaca/2019/04/14/news/fondi\_europei\_ed\_equo\_compenso\_a\_napoli\_liberi\_professionisti\_a\_confronto-224026633/ & 402 & Repubblica.it & Private/Non-Public & Online and Offline & National & medium = CP is important part of story & Jobs & Balanced & Subnational & No myth & NA & NA & NA & NA & NA & NA & NA & NA & Italy & fondi europei ed equo compenso, a napoli liberi professionisti a confronto & 2019-04-14 & politica di coesione & dalla politica di coesione europea all'autonomia differenziata: anche il governatore de luca al convegno organizzato martedì 16 aprile da confprofessioni campania le occasioni rappresentate dalle programmazione dei fondi europei, dallo sportello del lavoro autonomo presso i centri per l'impiego e dall'equo compenso. ma anche i rischi connessi ai progetti di autonomia differenziata. i liberi professionisti campani incontrano il presidente della regione, vincenzo de luca, in un convegno organizzato da confprofessioni campania martedì 16 aprile, alle 14.30, al renaissance naples hotel mediterraneo (via ponte di tappia 25; ingresso libero). tra i temi che verranno trattati nel corso dell'incontro la destinazione di risorse economiche a sostegno degli studi professionali, le modalità attraverso cui favorire l'incontro tra domanda e offerta, gli strumenti per far conoscere le chance professionali e favorirne la ridistribuzione, non ultimo, l'opportunità di tutelare il valore delle prestazioni professionali attraverso l'equo compenso. il convegno, moderato dal dott. raffaele ianuario, segretario di giunta di confprofessioni campania, sarà aperto dai saluti del vice presidente dott. vincenzo schiavo ed ospiterà gli interventi di gaetano stella, presidente nazionale di confprofessioni, dell'avvocato francesco mazzella, presidente confprofessioni campania, del notaio ludovico maria capuano e di enrico tezza, che illustrerà il progetto "social dialogue for sustainability of european professional firm". al governatore on. vincenzo de luca saranno affidate le conclusioni. nel corso dell'incontro, l'avvocato paola fiorillo, componente della giunta nazionale di confprofessioni, illustrerà la petizione online \#iononlavorogratis, attraverso cui si chiede al governo di dare immediata attuazione alla norma sull'equo compenso, introdotta dalla legge di bilancio 2018 e, sistematicamente, disattesa dalle pubbliche amministrazioni. "si tratta di un appuntamento importante -spiega il presidente di confprofessioni campania, l'avvocato francesco mazzella - per approfondire opportunità e rischi legati al mondo delle libere professioni, anche attraverso il confronto con la regione campania, che ha patrocinato la nostra iniziativa e sarà presente con de luca, al quale evidenzieremo il ruolo attivo che l'ente può avere per la crescita economica e la tutela di un importante comparto economico quale quello dei liberi professionisti. confprofessioni è la principale organizzazione di rappresentanza dei liberi professionisti in italia. fondata nel 1966 rappresenta e tutela gli interessi generali della categoria nel rapporto con le controparti negoziali e con le istituzioni politiche comunitarie nazionali e territoriali a tutti i livelli. riconosciuta parte sociale nel 2001, l'azione della confederazione mira alla qualificazione e alla promozione delle attività intellettuali nel contesto economico e sociale, proponendosi come fattore strategico per lo sviluppo e il benessere del paese e contribuendo, assieme alle istituzioni politiche e alle altre forze sociali, alla crescita culturale ed economica della società. & 438 & medium & Medium & Socio-Economic & NA & NA & 2019-04-14 & 2019 & 3 & ECO
Frame & low-medium & National & <500 & -0.5155635 & -0.1667445 & -0.9728619 & 0.5015415 & 1.3050799 & 3.2 & 0.6840199 & -1.2389660 & Payer & Domestic & Domestic & Domestic & Domestic|ECO & Neutral\\
Italy & http://www.ansa.it/europa/notizie/rubriche/altrenews/2018/04/30/bilancio-ue-solidarieta-e-riformebruxelles-fissa-paletti\_6ea279cc-b5d9-44b2-b7dd-1a4f1235287f.html & 436 & ANSA.it & Private/Non-Public & Online only & National & high = CP is most important issue in story (can also cover other issues) & Institutional bargaining over funding & Balanced & EU + National & No myth & Political leverage & Balanced & EU + Other country & No myth & NA & NA & NA & NA & Italy & bilancio ue: solidarietà e riforme, bruxelles fissa paletti - altre news - ansa europa & 2018-04-30 & fondi strutturali & bruxelles - tagli ai fondi per le politiche agricole e di coesione; maggiori sforzi finanziari da parte degli stati; nuove condizionalità che regolano i rubinetti degli aiuti: bruxelles fissa i suoi paletti, per far quadrare il bilancio ue 2021-2027, ed i paesi, che nei mesi scorsi avevano già tracciato le loro linee rosse, affilano le armi in attesa del giorno della verità. il budget-day è fissato per mercoledì, quando il commissario guenther oettinger illustrerà la sua ricetta: un esercizio di equilibrismi, per coprire il buco da dieci miliardi l'anno conseguenza della brexit, e raggranellare altri quindici miliardi annui per le nuove priorità, come difesa comune, sicurezza, migrazione e gestione delle frontiere esterne. finanze che potrebbero essere raccolte anche con l'introduzione di una tassa sugli imballaggi di plastica non riciclabile. cresce la spesa per i paesi - il bilancio 2014-2020 vale circa mille miliardi di euro, l'1\% del reddito nazionale lordo ue, ma oettinger ha già annunciato che dovrà salire ad "almeno l'1,12\%: in parte dipenderà anche se verrà incluso il fondo per lo sviluppo dell'africa. un boccone difficile da far digerire ai 27 governi, in molti ostaggio di movimenti euroscettici. i tagli maggiori a pac e coesione - l'italia, che è un 'contributore netto' e ogni anno versa circa 14 miliardi ricevendone 11,6, è preoccupata da possibili tagli alla coesione. nel settennato in corso, attraverso i 5 fondi strutturali il paese è stato il secondo beneficiario in europa, dopo la polonia, con un pacchetto complessivo di 44,6 miliardi, oltre ai 27 miliardi in aiuti diretti della pac. la sforbiciata dovrebbe aggirarsi attorno al 5-6\%. ma il think thank farm europe mette in guardia: "un taglio del 5\% a prezzi correnti, si traduce in realtà in un taglio del 20\% in termini reali". condizionalita' - se italia, francia e germania sperano davvero che bruxelles vada fino infondo sulla condizionalità che lega lo stanziamento dei fondi strutturali al rispetto dello stato di diritto e della solidarietà, le 'ribelli' polonia e ungheria sono già sul piede di guerra. refrattarie ad accogliere i richiedenti asilo e in rotta di collisione con l'ue sui principi democratici per le riforme varate di recente, i governi di varsavia e budapest sono tra i principali beneficiari di fondi strutturali in ue. secondo le indiscrezioni circolate, a beneficiare dei fondi distolti, sotto forma di incentivi, saranno paesi in prima linea nella gestione dei migranti, come grecia e italia. altra condizionalità prevista, sotto forma di incentivi che arriveranno da un tesoretto accantonato, è quella legata alle riforme strutturali contenute nelle raccomandazioni per paese. & 433 & high & High & Power & Power & NA & 2018-04-30 & 2018 & 3 & POL
Frame & high-very high & National & <500 & -0.5155635 & -0.1667445 & -0.9728619 & 0.5015415 & 1.3050799 & 3.2 & 0.6840199 & -1.2389660 & Payer & Domestic & European & Mixed & Domestic|POL & Neutral\\
Italy & http://www.ansa.it/umbria/notizie/2017/08/30/su-sito-regione-programmi-agenda-urbana\_a9b47035-c8c8-4331-8434-3650b46c00af.html & 409 & ANSA.it & Private/Non-Public & Online only & National & high = CP is most important issue in story (can also cover other issues) & Infrastructure & Factual & Subnational & No myth & NA & NA & NA & NA & NA & NA & NA & NA & Italy & su sito regione programmi agenda urbana - umbria & 2017-08-30 & fondo europeo di sviluppo regionale & quali interventi prevede l'agenda urbana dell'umbria e come incideranno su sviluppo, qualità della vita e servizi nelle cinque città interessate? sul sito istituzionale della regione umbria è pubblicata ora anche la presentazione sintetica ed efficace dei programmi che, tra le misure del programma operativo fesr (fondo europeo di sviluppo regionale) 2014-2020, riguarderanno perugia, terni, foligno, spoleto e città di castello. a realizzare la sintesi sono stati tre studenti del liceo scientifico galilei di perugia che hanno partecipato a uno degli stage formativi proposti dalla regione nel progetto "alternanza scuola lavoro". nelle scorse settimane letizia lare lantone, margherita brunetti e francesco pio cassano (tutti e tre della classe terza n del liceo perugino), hanno ampliato le proprie conoscenze sulla programmazione comunitaria e in particolare sul tema dell'agenda urbana, sotto la guida di funzionari e dirigenti dei servizi regionali programmazione comunitaria e generale. & 145 & high & High & Socio-Economic & NA & NA & 2017-08-30 & 2017 & 2 & ECO
Frame & high-very high & National & <500 & -0.5155635 & -0.1667445 & -0.9728619 & 0.5015415 & 1.3050799 & 3.2 & 0.6840199 & -1.2389660 & Payer & Domestic & Domestic & Domestic & Domestic|ECO & Neutral\\
Italy & http://www.ansa.it/europa/notizie/rubriche/europa\_delle\_regioni/2016/07/08/summit-territori-ue-raddoppiare-livello-investimenti\_18891771-027a-442f-b7fd-dd8a0b5fd146.html & 483 & ANSA.it & Private/Non-Public & Online only & National & low = CP mentioned more times but NOT important part of story (mainly about others issues) & Institutional bargaining over funding & Balanced & EU + National + Subnational & No myth & NA & NA & NA & NA & NA & NA & NA & NA & Italy & summit territori ue, raddoppiare livello investimenti - europa delle regioni & 2016-07-08 & fondi strutturali & bruxelles, - rilanciare gli investimenti, ridurre le disparità tra regioni e ridonare ai cittadini la fede nella ue. e' questo il triplice messaggio lanciato dai 700 rappresentanti di enti locali riuniti oggi e domani a bratislava per il settimo summit europeo delle città e delle regioni. "abbiamo bisogno - ha affermato il presidente del comitato delle regioni, cdr, markku markkula - di raddoppiare il livello dei nostri investimenti con un approccio congiunto pubblico-privato. dobbiamo sviluppare progetti concreti in cui la dimensione territoriale è cruciale". uno studio dell'ocse rileva che dallo scoccare della crisi, nel 2008, gli investimenti sono diminuiti in molte aree europee. la dichiarazione di bratislava, approvata oggi, chiede alla ue di invertire la rotta e la esorta inoltre a semplificare la propria legislazione, e quindi l'accesso ai programmi comunitari di sviluppo, mentre guarda con grande favore al fondo europeo per gli investimenti strategici, feis, creato dalla commissione ue nel 2015. i membri del cdr presenti nella capitale slovacca, propongono però di aumentare il peso dei fondi strutturali nel feis, garantendo che anche le regioni più deboli possano beneficiarne. "la strategia economica - ha rilevato il commissario ue all'energia maros sefcovic - deve sfruttare i mattoni di base della nuova economia emergente in europa, le energie rinnovabili, l'economia della condivisione e i nuovi modelli di trasporto, e per far questo servono ingenti investimenti pubblici" e potenziare "i collegamenti tra regioni a livello transfrontaliero" in modo da "costruire mercati comuni". il vertice di bratislava si consuma ad una settimana dall'inizio della presidenza slovacca e, soprattutto, a due dalla brexit. "la risposta politica è qui oggi a bratislava" ha commentato raffaele cattaneo, presidente dell'assemblea della regione lombardia e della commissione per le politiche di coesione territoriale al cdr. "regioni e città - ha insistito - devono lavorare insieme per ricostruire un'europa che parta dal basso, la nostra voce deve essere sufficientemente forte per connettersi, stare insieme e rispondere in senso contrario alle tendenze disgregatrici". cattaneo ha inoltre indicato un altro obiettivo politico per le regioni: invertire l'impostazione della commissione ue, che punta sugli "investimenti centralizzati, come il piano juncker governato da bruxelles con gli stati", per tornare a potenziare le "politiche di coesione" con interventi "decentrati" a più stretto contatto con i cittadini. & 374 & low & Low & Power & NA & NA & 2016-07-08 & 2016 & 2 & POL
Frame & low-medium & National & <500 & -0.5155635 & -0.1667445 & -0.9728619 & 0.5015415 & 1.3050799 & 3.2 & 0.6840199 & -1.2389660 & Payer & Domestic & European & Mixed & Domestic|POL & Neutral\\
\addlinespace
Italy & http://ilpiccolo.gelocal.it/trieste/cronaca/2018/02/16/news/un-milione-e-mezzo-a-tre-progetti-anti-cancro-1.16488865 & 451 & Il Piccolo & Private/Non-Public & Online and Offline & Regional/Local & medium = CP is important part of story & Research \& innovation & Positive & Subnational & No myth & NA & NA & NA & NA & NA & NA & NA & NA & Italy & un milione e mezzo a tre progetti anti cancro  - cronaca - il piccolo & 2018-02-17 & fondo europeo di sviluppo regionale & fondi europei per creare nuovi test di diagnosi e filtri per radioprotezione. coinvolte imprese triestine puntano a migliorare l'ambiente della ricerca e della medicina i progetti finanziati sul territorio dai fondi por fesr, il programma operativo del fondo europeo di sviluppo regionale che prevede "investimenti a favore della crescita e dell'occupazione" del friuli venezia giulia. si tratta di - i-smart, innovativo saggio molecolare associato a risposta terapeutica; iort, intra operative radiation therapy e sermi4cancer, surface enhanced raman microrna per la diagnosi del cancro. tutti vedono coinvolte diverse imprese triestine in attività di ricerca industriale. le tre nuove idee hanno ricevuto complessivamente quasi un milione e mezzo di euro. il progetto i-smart mira alla messa a punto di un kit innovativo per un test diagnostico molecolare semplificato e rapido, per la diagnosi oncologica personalizzata non invasiva da campioni clinici sia d'archivio che freschi e vede collaborare insieme alphagenics biotech srl, dotcom srl e il laboratorio nazionale del consorzio interuniversitario per le biotecnologie (lncib). iort punta alla creazione di un innovativo filtro schermante per radioprotezione, costruito attraverso la combinazione di materie prime come una resina termoplastica biocompatibile con un materiale schermante metallico e con l'impiego di nuove tecniche, come la stampa 3d. coinvolge nella realizzazione diversi soggetti, logic srl, r3place srl e l'università di trento. il progetto sermi4cancer ha come obiettivo invece la creazione di un prototipo di test rapido per la misurazione di microrna circolanti in pazienti affetti da cancro al fegato e da patologie croniche collegate. vede coinvolti come partner dino paladin, l'università degli studi di trieste-dipartimento di ingegneria ed architettura, la fondazione italiana fegato, alphagenics biotech srl e insiel mercato spa. i fondi por fesr sono stati approvati dalla commissione europea con diverse finalità: rilancio occupazionale, creazione di nuove start up, collaborazioni tra imprese e centri di ricerca, rilancio della propensione agli investimenti del sistema produttivo, riconversione energetica di edifici pubblici e sviluppo urbano. (mi.b.) & 325 & medium & Medium & Socio-Economic & NA & NA & 2018-02-17 & 2018 & 3 & ECO
Frame & low-medium & Regional & <500 & -0.5155635 & -0.1667445 & -0.9728619 & 0.5015415 & 1.3050799 & 3.2 & 0.6840199 & -1.2389660 & Payer & Domestic & Domestic & Domestic & Domestic|ECO & Positive\\
Italy & http://www.ilmessaggero.it/primopiano/politica/migranti\_m5s-3224664.html & 447 & ilmessaggero.it & Private/Non-Public & Online and Offline & National & low = CP mentioned more times but NOT important part of story (mainly about others issues) & Political leverage & Negative & EU + National & No myth & NA & NA & NA & NA & NA & NA & NA & NA & Italy & migranti, m5s: stop ai fondi a chi non li ricolloca & 2017-09-07 & fondi strutturali & "i migranti sbarcati in italia devono essere ricollocati negli altri paesi europei subito! la sentenza della corte di giustizia europea, che respinge i ricorsi presentati nel 2015 da slovacchia e ungheria, deve essere immediatamente rispettata o, ancora una volta, l'unione europea avrà perso la faccia". lo chiedono i parlamentari europei del movimento 5 stelle. "l'unione europea - proseguono - non ha nessuna intenzione di prorogare la decisione e questa della corte di giustizia europea dunque potrebbe essere una vittoria di pirro se non arriva una decisione immediata del consiglio di prolungare i ricollocamenti e renderli obbligatori". "finora hanno lasciato il nostro paese appena 8.402 migranti sui 35 mila promessi (dati aggiornati al 1 settembre 2017), appena il 24\% del totale. bulgaria, repubblica ceca, danimarca, estonia, ungheria, irlanda, polonia, slovacchia e inghilterra non rispettano i patti presi in sede di consiglio. questo è l'assurdo: l'europa oggi è talmente debole che non riesce nemmeno a far rispettare le decisioni che prende. persino la svizzera e la norvegia che non fanno parte dell'unione europea si sono mostrati solidali accettando il ricollocamento di circa 1.500 migranti", rimarcano gli esponenti m5s. "nel 2017 - sottolineano i 5 stelle - sono sbarcati nel nostro paese 99.742 migranti: il 79\% dei migranti sbarcati in europa ha toccato le coste italiane. per essere chiari: tutti gli stati europei devono accogliere, l'immigrazione è un tema europeo e nessun paese può considerarsi escluso. quello delle ricollocazione non è un sistema perfetto ma è un primo passo che l'europa compie verso una redistribuzione obbligatoria e permanente delle quote dei richiedenti asilo, così come chiesto più volte dal gruppo efdd - movimento 5 stelle in europa". "la decisione della corte è importante in quanto sottolinea il principio di solidarietà tra gli stati membri, già sancito dai trattati, ma troppo spesso disatteso. davanti alla chiusura di slovacchia e ungheria, quindi, la corte di giustizia alza il cartellino rosso e, aprendo un precedente, richiama tutti i paesi dell'unione alle proprie responsabilità. perché, come scrivono i giudici, il meccanismo contribuisce effettivamente e in modo proporzionato a far sì che la grecia e l'italia possano far fronte alle conseguenze della crisi migratoria del 2015", aggiungono gli europarlamentari m5s. "ora - concludono - andiamo avanti con la nostra proposta, presentata in sede di riforma del regolamento di dublino, di sospendere l'erogazione dei fondi strutturali a quei paesi che non cooperano nella ricollocazione dei richiedenti asilo". & 404 & low & Low & Power & NA & NA & 2017-09-07 & 2017 & 2 & POL
Frame & low-medium & National & <500 & -0.5155635 & -0.1667445 & -0.9728619 & 0.5015415 & 1.3050799 & 3.2 & 0.6840199 & -1.2389660 & Payer & Domestic & European & Mixed & Domestic|POL & Negative\\
Italy & http://www.ansa.it/europa/notizie/rubriche/voceeurodeputati/2018/04/20/fondi-ue-castaldom5s-rilanciare-risorse-proprie-unione\_d214833a-6485-4509-ada8-522357a724c4.html & 472 & ANSA.it & Private/Non-Public & Online only & National & high = CP is most important issue in story (can also cover other issues) & Institutional bargaining over funding & Negative & EU + National & No myth & Mismanagement & Negative & Subnational & 8.Mismanaged & NA & NA & NA & NA & Italy & fondi ue: castaldo(m5s), rilanciare risorse proprie unione - la voce degli eurodeputati - ansa europa & 2018-04-20 & politica di coesione & (ansa) - roma, 20 apr - nell'ambito della politica di coesione dell'unione europea "bisogna difendere l'ammontare del fondo partendo dal rilancio delle risorse proprie dell'unione e andando a integrare le risorse del bilancio ue con nuove entrate. il movimento 5 stelle da sempre lotta per inserire carbon tax, web tax e tobin tax come possibili soluzioni per ovviare a questo problema". a dirlo è il vice presidente del parlamento europeo, fabio massimo castaldo (m5s), a margine dell'incontro 'la politica di coesione verso il quadro finanziario pluriennale post 2020' organizzato a roma da parlamento europeo e commissione europea. "c'è uno scenario inquietante, quello di un taglio di un terzo dei fondi", continua castaldo spiegando che in questo caso paesi come l'italia sarebbero completamente esclusi dalla politica di coesione dell'ue. per scongiurare questo rischio, secondo castaldo: "come sistema politico, dobbiamo impostare la sfida e, per vincerla, dobbiamo renderci conto che c'è un interesse nazionale ed europeo da difendere". alcune regioni del sud italia "hanno dimostrato in questi anni di non essere sempre all'altezza del sistema, abbiamo perso circa 160 milioni di euro nella scorsa programmazione, un dato che non possiamo permetterci e per questo serve maggiore coordinamento e pressione nazionale sull'operato delle nostre regioni", ha concluso l'europarlamentare 5 stelle. (ansa). & 218 & high & High & Power & Governance & NA & 2018-04-20 & 2018 & 3 & POL
Frame & high-very high & National & <500 & -0.5155635 & -0.1667445 & -0.9728619 & 0.5015415 & 1.3050799 & 3.2 & 0.6840199 & -1.2389660 & Payer & Domestic & European & Mixed & Domestic|POL & Negative\\
Italy & https://tribunatreviso.gelocal.it/treviso/cronaca/2018/11/10/news/pochi-soldi-per-la-cultura-si-uniscono-quattro-comuni-1.17449593 & 396 & Tribuna di Treviso & Private/Non-Public & Online and Offline & Regional/Local & very low = CP mentioned once & Cultural development & Positive & Subnational & No myth & NA & NA & NA & NA & NA & NA & NA & NA & Italy & pochi soldi per la cultura si uniscono quattro comuni & 2018-11-11 & fondo sociale europeo & silea, casale, casier e roncade decidono di mettere insieme le (poche) risorse si comincia con una rassegna teatrale che durerà cinque mesi: "è una sfida" silea un unico programma culturale, una gestione comune, una vera unione dei servizi nel settore cultura per far fronte a casse sempre più asciutte. silea, casale, casier e roncade provano a percorrere una strada che le amministrazioni sono già solite utilizzare, ma per altri servizi, come la polizia locale. il primo test è "teatro si fa in quattro", rassegna lunga 5 mesi che porterà nei quattro comuni rappresentazioni di livello, solitamente destinate a palcoscenici prestigiosi come il goldoni di venezia. ma per dirla con le parole dell'assessore di casier simona guardati, "è un test di laboratorio", in vista dell'obiettivo finale: ovvero un unico settore cultura, in grado di alzare gli standard mettendo insieme soldi, idee e anche dipendenti. con i bilanci ridotti all'osso quello della cultura è un comparto che soffre, ma "teatro si fa in quattro", - fatta salva la prova del "botteghino" - "dimostra che l'unione fa la forza". "da tempo lavoriamo a progetti culturali in rete", spiega l'assessore di silea angela trevisin, "ma, per la prima volta, abbiamo strutturato un progetto articolato sia per quantità di spettacoli sia per il maggiore coinvolgimento della cittadinanza. le reti culturali degli enti locali fanno parte degli obiettivi del programma per il fondo sociale europeo, segno che la direzione da perseguire è unire le forze e promuovere un unico territorio. è nostra intenzione abbattere i confini territoriali, mantenendo le specificità che ci caratterizzano, a favore di un concetto più esteso di comunità per una cultura condivisa e partecipata". lo scopo è poi creare un nuovo centro di riferimento, "in periferia spesso l'offerta culturale è sacrificata rispetto al capoluogo; la nostra è una sfida. non abbiamo proposto una sagra", aggiunge guardati. si parte domenica 18, alle 17, al teatrino delle elementari di casier con il mago di oz, poi si proseguirà sempre di domenica alle 17 con altri nove spettacoli, organizzati con la compagnia stivalaccio. il 25 novembre romeo e giulietta, che a marzo sarà al goldoni, a 25 euro di ingresso. al teatro parrocchiale di silea costerà al massimo 8 euro. ci sono riduzioni per under 30, over65, e iscritti alle biblioteche comunali. -- & 381 & very low & Low & Socio-Economic & NA & NA & 2018-11-11 & 2018 & 3 & ECO
Frame & v.low & Regional & <500 & -0.5155635 & -0.1667445 & -0.9728619 & 0.5015415 & 1.3050799 & 3.2 & 0.6840199 & -1.2389660 & Payer & Domestic & Domestic & Domestic & Domestic|ECO & Positive\\
Italy & https://www.lastampa.it/2019/04/04/cronaca/fondi-europei-piemonte-promosso-bHmLYh3rZZuncGKbpKSmfJ/pagina.html & 449 & LaStampa.it & Private/Non-Public & Online and Offline & National & very high = CP is most important issue + CP is mentioned in title/headline & Bureaucracy and/or delays & Positive & Subnational & No myth & Research \& innovation & Positive & Subnational & No myth & Jobs & Positive & Subnational & No myth & Italy & fondi europei: piemonte promosso & 2019-04-04 & fondo europeo di sviluppo regionale & il piemonte è una delle regioni europee in cui la politica di coesione funziona meglio grazie alla capacità amministrativa di gestire i fondi comunitari e di mettere in atto progetti di grande valore, ed è addirittura la prima in italia per l'utilizzo nella ricerca. non lo ha dichiarato sergio chiamparino ma corina cretu, la commissaria europea per la politica regionale, intervenendo al convegno "piemonte: dove la coesione funziona. successi, sfide e opportunità della politica dell'unione europea che rimodella le regioni", tenutosi nella sala conferenze del museo egizio di torino. una freccia all'arco del governatore uscente, attaccato da alberto cirio, lo sfidante del centrodestra, proprio sulla capacità di spendere i fondi europei. fronti di sviluppo la commissaria, accompagnata dalla parlamentare europea del piemonte, mercedes bresso, ha poi sottolineato che, grazie a queste risorse, ricerca e innovazione hanno ricevuto uno stimolo importante: si sviluppano veicoli elettrici ed ibridi, si sperimentano la digitalizzazione dei processi di business e le nuove tecnologie sanitarie. tutto ciò è reso possibile dall'inclusione nei processi di ricerca e innovazione delle piccole e medie imprese, mirando a renderle sempre più innovative e produttive. prossimamente la collaborazione tra regione e unione europea dovrà orientarsi sulla programmazione del periodo 2021-2027, che assegnerà all'italia il 7\% di fondi in più rispetto ad oggi (la ripartizione avverrà entro fine anno) e vedrà come settori chiave la trasformazione produttiva, la ricerca e l'innovazione, la cultura, il turismo, la rigenerazione delle aree urbane, la mobilità sostenibile, la lotta all'esclusione sociale. l'obiettivo sarà aumentare la platea delle persone che potranno usufruirne e nello stesso tempo i livelli di competitività del sistema. i traguardi il convegno è stato aperto dalle assessore regionali al lavoro, gianna pentenero, e alle attività produttive, giuseppina de santis, che hanno illustrato i risultati finora raggiunti dal piemonte grazie ad assegnazioni superiori al miliardo di euro con la programmazione 2014-2020. con il fondo europeo di sviluppo regionale sono stati impegnati 500 milioni per la realizzazione di infrastrutture e investimenti produttivi nei campi della ricerca, dello sviluppo dell'automotive e dell'aerospazio, con attenzione particolare ai parchi della salute di torino e novara. grazie al fondo sociale europeo la regione ha impegnato in questi anni 567 milioni per favorire, tra le altre cose, l'inserimento dei disoccupati e delle categorie sociali meno favorite, finanziato azioni di formazione, l'avviamento di 429 contratti di apprendistato in 175 aziende, la creazione di 400 nuove imprese con lo sportello mip-mettersi in proprio. in totale 300 mila le persone coinvolte. inoltre è stata messa in evidenza la capacità di certificazione del piemonte che, rispetto al fondo sociale, si colloca al primo posto tra le regioni italiane per obiettivi di spesa. & 453 & very high & High & Governance & Socio-Economic & Socio-Economic & 2019-04-04 & 2019 & 3 & POL
Frame & high-very high & National & <500 & -0.5155635 & -0.1667445 & -0.9728619 & 0.5015415 & 1.3050799 & 3.2 & 0.6840199 & -1.2389660 & Payer & Domestic & Domestic & Domestic & Domestic|POL & Positive\\
\addlinespace
Italy & http://roma.corriere.it/notizie/cronaca/16\_giugno\_18/inchiesta-procura-velletri-finanza-anderlucci-droga-evasione-fiscale-riciclaggio-appalto-confisca-43-milioni-castelli-d8bba1a6-3533-11e6-8ef0-3c2327086418.shtml & 418 & Corriere della Sera & Private/Non-Public & Online and Offline & National & low = CP mentioned more times but NOT important part of story (mainly about others issues) & Fraud/Corruption & Negative & Subnational & 7.Fraud & NA & NA & NA & NA & NA & NA & NA & NA & Italy & dalla droga all'appalto per l'asilo confiscati 43 milioni ai castelli & 2016-06-18 & fondo sociale europeo & l'organizzazione guidata da sergio anderlucci è accusata di traffico di stupefacenti, evasione fiscale e riciclaggio. una delle aziende del gruppo si era aggiudicata i lavori per la costruzione di una materna e di un nido ad albano con i contributi del fondo sociale europeo. tra i prestanome la moglie, i figli, la nuora e la consuocera a due anni dal sequestro è scattata la confisca dei beni acquisiti dall'organizzazione guidata da sergio anderlucci, che grazie al traffico di droga, all'evasione fiscale e al riciclaggio aveva accumulato risorse tali da potersi aggiudicare anche gli appalti pubblici. a questo gruppo la finanza ha ora sottratto il controllo di un patrimonio del valore di 43 milioni (per la precisione 43.566.156 mila euro): 18 aziende con sedi a roma e a latina, nove immobili in provincia di roma, otto tra auto e moto e 27 rapporti finanziari. anderlucci, già fallito nel 2005 e condannato nel 2003, nell'ultimo decennio aveva dichiarato al fisco poco più di 32 mila euro. nel mirino 122 soggetti nel corso dell'inchiesta, iniziata nel 2013, il gico ha compiuto accertamenti su 122 soggetti tra persone fisiche e giuridiche e ha ricostruito sia la fitta rete degli interessi commerciali dell'organizzazione, sia l'entità degli investimenti, localizzati tra albano e genzano e realizzati anche con l'utilizzo di prestanome, pregiudicati e familiari compresi (in particolare la moglie, i figli, la nuora e la consuocera di anderlucci). "se non fosse intervenuto il sequestro (tra luglio e dicembre 2014, ndr) - scrive il giudice nel provvedimento con cui ha accolto le accuse del procuratore di velletri francesco prete e del pm giovanni taglialatela - verosimilmente l'intero "marchingegno" sarebbe andato a buon fine senza che ne rimanesse traccia". l'appalto per la scuola il sistema architettato dal gruppo prevedeva, secondo l'accusa, la creazione di società e cooperative operanti in vari settori che nel giro di qualche mese venivano liquidate "in modo che i contratti venissero trasferiti a un nuovo soggetto giuridico": lo scopo, riuscito, era di evadere le imposte. proprio in questo contesto il gruppo era riuscito a introdursi nel settore degli appalti: una delle aziende sequestrate si era aggiudicata i lavori per realizzare una scuola materna e un asilo nido con i contributi del fondo sociale europeo. una volta scattato il sequestro, il progetto "cecchina 2", ad albano, è stato portato a termine dall'amministrazione giudiziaria nominata dal tribunale e inaugurato ad aprile scorso. edilizia e pulizie le aziende confiscate operavano una nell'"attività di edilizia in genere", tre nei "servizi di logistica e organizzazione aziendale a imprese", una nella "consulenza e assistenza economico finanziaria ai soci", tre nella "costruzione di edifici residenziali e non residenziali", una nell'"attività di pulizia", una nelle "altre attività di pulizia specializzata di edifici e di impianti e di macchinari industriali", una nei "lavori generali di costruzione edifici", una nella "gestione di uffici temporanei, uffici, residence", una nel "commercio all'ingrosso di articoli antincendio/antinfortunistica", una nelle "altre attività creditizie", una nella "gestione di immobili propri", una nel "noleggio autoveicoli", una nell'"attività di organizzazioni associative" e una nelle "altre attività di consulenza imprenditoriale e pianificazione aziendale. il tribunale ha disposto anche la sorveglianza speciale nei confronti di 5 indagati, con l'obbligo di soggiorno per tre di loro. & 548 & low & Low & Governance & NA & NA & 2016-06-18 & 2016 & 2 & POL
Frame & low-medium & National & 500-1000 & -0.5155635 & -0.1667445 & -0.9728619 & 0.5015415 & 1.3050799 & 3.2 & 0.6840199 & -1.2389660 & Payer & Domestic & Domestic & Domestic & Domestic|POL & Negative\\
Italy & https://www.agi.it/perugia/notizie/regioni\_umbria\_primi\_finalisti\_european\_social\_sound-201503101254-cro-rpg1002 & 395 & AGI & Private/Non-Public & Online and Offline & Regional/Local & low = CP mentioned more times but NOT important part of story (mainly about others issues) & Cultural development & Positive & Subnational & No myth & NA & NA & NA & NA & NA & NA & NA & NA & Italy & regioni: umbria, primi finalisti 'european social sound' & 2015-03-11 & fondo sociale europeo & (agi) - perugia, 10 mar. - gli adius con il brano 'coscienza sporca', i deep blue trio con la canzone 'when i was child', gli esperimenti di selenia con il brano 'previsione totale' e the staplers con 'like a rainfall': sono queste le prime quattro band che si sono aggiudicate un posto alla finalissima del concorso 'european social sound', organizzato dalla regione umbria con l'obiettivo di far conoscere e diffondere in particolare tra i giovani le informazioni sulle opportunita' offerte dal fondo sociale europeo. le quattro band sono state selezionate fra le dodici che si sono esibite nella prima tappa di qualificazione, al serendipity di foligno. due delle canzoni sono state scelte dal pubblico, con votatori elettronici, e le altre due dalla giuria di qualita' composta dalla dj tamara taylor, da flavio manieri, organizzatore di eventi musicali ed esperto del suono, e sid, reporter per webzine musicali. venerdi 13 marzo la sfida si sposta a terni, al 'queency' per la seconda tappa del concorso durante la quale altri dodici gruppi si sfideranno per conquistare altri quattro posti per la finalissima che si disputera' il 27 marzo all'afterlife di perugia.(agi) pg2/sep & 192 & low & Low & Socio-Economic & NA & NA & 2015-03-11 & 2015 & 1 & ECO
Frame & low-medium & Regional & <500 & -0.5155635 & -0.1667445 & -0.9728619 & 0.5015415 & 1.3050799 & 3.2 & 0.6840199 & -1.2389660 & Payer & Domestic & Domestic & Domestic & Domestic|ECO & Positive\\
Italy & http://www.ilfoglio.it/economia/2015/09/24/perch-la-polonia-ha-detto-un-sorprendente-s-alle-quote-per-i-rifugiati\_\_\_1-v-133102-rubriche\_c258.htm & 416 & Il Foglio Magazine & Private/Non-Public & Online only & National & low = CP mentioned more times but NOT important part of story (mainly about others issues) & Political leverage & Balanced & EU + Other country & No myth & Solidarity to poor countries/regions & Positive & EU + Other country & No myth & NA & NA & NA & NA & Italy & perché la polonia ha detto un sorprendente "sì" alle quote per i rifugiati & 2015-09-24 & fondi strutturali & le conseguenze della frattura esterna con gli alleati di visegrad e di quella interna tra le due "comunità morali" di varsavia la guerra in ucraina ha provocato uno scisma nel gruppo visegrad, il foro informale, nato all'indomani del crollo dei regimi comunisti, che riunisce i governi di varsavia, praga, budapest e bratislava. i polacchi biasimano mosca, ungheresi e slovacchi mal tollerano le sanzioni alla russia. i cechi galleggiano tra le linee. la crisi sui rifugiati ha inizialmente ripristinato la coesione nel quartetto. "no" unanime alle quote obbligatorie proposte dall'europa sull'accoglienza degli uomini e delle donne che stanno marciando sulla rotta balcanica o sbarcando sulle coste greche e italiane: questa la posizione comune espressa dai quattro governi. eppure martedì, quando la questione è stata messa ai voti nella riunione dei ministri degli interni dell'unione europea, la polonia, contrariamente agli altri, ha accettato il piano comunitario. articoli correlati il non piano europeo per l'immigrazione il punto di orban un esodo in disordine sparso piotr stachanczyk, sottosegretario agli interni con delega all'immigrazione, ha spiegato che se la polonia si fosse opposta non avrebbe potuto negoziare, si sarebbe isolata. da qui la scelta di appoggiare le quote e lavorare affinché la ridistribuzione dei rifugiati si basasse sui soli numeri concordati (la polonia dovrà per ora ospitarne dai 4.600 ai 4.800), senza particolari meccanismi vincolanti. decade in questo modo la possibilità di creare precedenti. così ha spiegato stachanczyk, citato dal portale eubserver. a ogni modo non tutto può ridursi ai passaggi tecnici. a varsavia, c'è da credere, avranno soppesato bene le inconvenienze che si sarebbero prodotte nel caso in cui ci si fosse impuntati. la polonia vive oggi un momento eccellente, in termini di stabilità economica. ci sono inevitabilmente sacche di "vinti", ma il paese inizia a funzionare. molto dipende dal rapporto fruttuoso con la germania, che assorbe più di un quinto dell'export e più di un quarto dell'import della polonia, risultandone di gran lunga il primo partner commerciale. a questo s'aggiungono i numerosi investimenti industriali operati dai colossi tedeschi. oltre ai fondi strutturali europei. il loro corretto utilizzo ha aiutato la polonia a fare crescita. insomma, il "no" allo smistamento dei rifugiati, preteso da bruxelles su impulso di berlino, avrebbe potuto creare attriti con i due principali generatori di sviluppo del paese. la scelta del governo, guidato dalla centrista ewa kopacz, che ha preso il posto di donald tusk dopo la nomina di quest'ultimo a presidente del consiglio europeo e che in questi ultimi giorni ha aperto sui rifugiati, è stata contestata duramente da destra. beata szydlo, la candidata a primo ministro di diritto e giustizia, il partito populista destinato a vincere le legislative del 25 ottobre, ha affermato che il "sì" alle quote è una presa in giro nei confronti di slovacchi, cechi e ungheresi, denunciando inoltre la lesione degli interessi nazionali e la scarsa considerazione nei confronti della volontà dei polacchi. in effetti i sondaggi indicano che i cittadini sono prevalentemente contrari all'arrivo dei rifugiati. certo è che il dibattito è molto umorale, se non irrazionale. al punto che gazeta wyborcza, il principale giornale del paese, ha promosso nei giorni scorsi un inserto, uscito in contemporanea su quaranta testate, dedicato proprio ai rifugiati. una sorta di faq per informare correttamente e riportare la discussione politica dentro i confini del reale. non è servito a molto. & 567 & low & Low & Power & Values & NA & 2015-09-24 & 2015 & 1 & POL
Frame & low-medium & National & 500-1000 & -0.5155635 & -0.1667445 & -0.9728619 & 0.5015415 & 1.3050799 & 3.2 & 0.6840199 & -1.2389660 & Payer & European & European & European & European|POL & Neutral\\
Italy & http://www.ansa.it/europa/notizie/rubriche/altrenews/2017/02/08/migranti-ueanche-fondi-sviluppo-regionale-per-integrazione\_0315875c-9e86-4ced-9416-755ced73bf06.html & 432 & ANSA.it & Private/Non-Public & Online only & National & medium = CP is important part of story & Social awareness/inclusion & Balanced & EU + National & No myth & Infrastructure & Positive & EU + National & No myth & NA & NA & NA & NA & Italy & migranti: ue,anche fondi sviluppo regionale per integrazione - altre news - ansa europa & 2017-02-08 & fondo europeo di sviluppo regionale & bruxelles - "la mia idea è accelerare insieme agli stati membri per esplorare le possibilità di uso dei fondi fesr (fondo europeo di sviluppo regionale, ndr) per i migranti", perché i ricollocamenti "non stanno andando molto bene", "grecia e italia hanno ragione a chiedere un aiuto aggiuntivo". così la commissaria europea alla politica regionale, corina cretu, durante un incontro ristretto con la stampa al quale ansa ha preso parte, alla vigilia del viaggio in italia che la porterà a pompei, napoli norcia e perugia. alcuni stati stanno combinando "molto bene" fondi europei diversi, "come la germania", ma altri "devono fare dei cambiamenti ai loro programmi operativi". al termine del periodo di programmazione 2007-2013, le autorità italiane hanno ottenuto una modifica nella destinazione dei fondi del programma "sicurezza", che ha permesso il cofinanziamento ue dell'acquisto di due pattugliatori per i soccorsi in mare. "siamo pronti a discutere se sia necessario aiutare in altro modo", spiega cretu, "per il 2014-2020 abbiamo già aggiunto alcuni soldi per i rifugiati", che per l'italia valgono 1,6 miliardi "extra", e da qualche mese è stato lanciato il secondo bando dell'iniziativa 'urban innovative actions', che si rivolge direttamente alle città. "i migranti sono una delle priorità in italia e speriamo di ricevere progetti di alta qualità che porteranno alcuni finanziamenti aggiuntivi", afferma la commissaria. & 222 & medium & Medium & Socio-Economic & Socio-Economic & NA & 2017-02-08 & 2017 & 2 & ECO
Frame & low-medium & National & <500 & -0.5155635 & -0.1667445 & -0.9728619 & 0.5015415 & 1.3050799 & 3.2 & 0.6840199 & -1.2389660 & Payer & Domestic & European & Mixed & Domestic|ECO & Neutral\\
Italy & http://notizie.tiscali.it/regioni/toscana/articoli/fondi-ue-via-assegni-ricollocazione/ & 471 & Tiscali & Private/Non-Public & Online only & National & very low = CP mentioned once & Jobs & Positive & Subnational & No myth & NA & NA & NA & NA & NA & NA & NA & NA & Italy & fondi ue: via a assegni ricollocazione & 2017-07-22 & fondo sociale europeo & (ansa) - firenze, 22 lug - via libera in toscana all'avviso per gli assegni di ricollocazione, una sperimentazione regionale della norma nazionale, che prevede interventi mirati e percorsi individuali per facilitare l'incontro tra domanda e offerta di lavoro. con il nuovo avviso, finanziato dal programma operativo regionale del fondo sociale europeo 2014-2020, i beneficiari dei voucher formativi individuali potranno richiedere al centro per l'impiego il rilascio dell'assegno per l'assistenza alla ricollocazione, una volta realizzato almeno il 70\% delle ore previste dal percorso di formazione finanziato con il voucher e comunque non oltre 30 giorni dal termine dell'attività. l'assegno di ricollocazione è spendibile presso il proprio centro per l'impiego o presso un operatore accreditato tra quelli che risponderanno all'avviso. chi ha terminato l'attività formativa prima dell'uscita dell'avviso, potrà presentare richiesta di assegno entro 30 giorni dalla data di pubblicazione sul bollettino ufficiale della regione toscana. & 155 & very low & Low & Socio-Economic & NA & NA & 2017-07-22 & 2017 & 2 & ECO
Frame & v.low & National & <500 & -0.5155635 & -0.1667445 & -0.9728619 & 0.5015415 & 1.3050799 & 3.2 & 0.6840199 & -1.2389660 & Payer & Domestic & Domestic & Domestic & Domestic|ECO & Positive\\
\addlinespace
Italy & http://www.ansa.it/toscana/notizie/2017/12/15/fondi-ue-a-toscana-732-mln-da-fondo-fse\_63d564ba-7696-408b-b6ac-235c6b969d10.html & 476 & ANSA.it & Private/Non-Public & Online only & National & low = CP mentioned more times but NOT important part of story (mainly about others issues) & Social justice & Positive & Subnational & No myth & NA & NA & NA & NA & NA & NA & NA & NA & Italy & fondi ue: a toscana 732 mln da fondo fse - toscana & 2017-12-15 & fondo sociale europeo & (ansa) - firenze, 15 dic - ammontano a 732 milioni di euro le risorse a disposizione della toscana dal fondo sociale europeo (fse) per il settennato 2014-2020, ed il 42\% di questo importo (pari a 308 milioni) è già stato movimentato e in larga misura impegnato. una grande quantità di bandi e progetti sono partiti permettendo di realizzare circa 7 mila interventi che hanno raggiunto oltre 430 mila destinatari. si tratta di interventi diretti in molti casi verso le categorie più deboli e vulnerabili: i giovani, le donne, i disoccupati di lunga durata, i soggetti svantaggiati (disabili, ex detenuti, ex tossicodipendenti etc). e' quanto emerso oggi a firenze, in occasione dell'incontro annuale organizzato dalla regione toscana per fare il punto sull'avanzamento del programma, intitolato 'da 60 anni le persone al centro'. tra i presenti il presidente della toscana enrico rossi, il vicepresidente della regione monica barni, e l'assessore regionale al lavoro e istruzione monica barni. & 157 & low & Low & Socio-Economic & NA & NA & 2017-12-15 & 2017 & 2 & ECO
Frame & low-medium & National & <500 & -0.5155635 & -0.1667445 & -0.9728619 & 0.5015415 & 1.3050799 & 3.2 & 0.6840199 & -1.2389660 & Payer & Domestic & Domestic & Domestic & Domestic|ECO & Positive\\
Italy & http://www.lastampa.it/2016/01/29/economia/dagli-sgravi-fiscali-alla-maternit-ecco-le-nuove-tutele-per-le-partite-iva-lvMLsM63dwC8Gv4lL0ELDO/pagina.html & 470 & LaStampa.it & Private/Non-Public & Online and Offline & National & very low = CP mentioned once & Jobs & Positive & National & No myth & NA & NA & NA & NA & NA & NA & NA & NA & Italy & dagli sgravi fiscali alla maternità. ecco le nuove tutele per le partite iva & 2016-01-29 & fondi strutturali & decolla il piano contro la povertà. in arrivo aiuti a 280 mila famiglie il governo dà lo stop ai contratti capestro che per troppo tempo hanno penalizzato le partite iva. vara un nuovo pacchetto di misure e di tutele, di fatto il primo pezzo del "jobs act dei lavoratori autonomi", compreso un importante pacchetto di detrazioni fiscali, e scrive finalmente regole precise per il lavoro agile, lo "smart working" che oggi sta prendendo sempre più piede anche in italia. "colpiamo clausole e condotte abusive", ha spiegato ieri il ministro del lavoro giuliano poletti al termine del consiglio dei ministri che ha dato il via libera al nuovo disegno di legge. "cerchiamo di aumentare le tutele per questo lavoro nelle transazioni commerciali e fare in modo che i soggetti non vengano colpiti da contratti capestro perché essendo lavoratori autonomi hanno poche alternative". obiettivo della legge: "costruire per prestatori d'opera materiali e intellettuali non imprenditori un sistema di diritti e di welfare moderno capace di sostenere il loro presente e di tutelare il loro futuro". agevolazioni fiscali questa è una delle misure più attese. prevede la possibilità di dedurre dalle tasse il 100\% delle spese sostenute per i servizi personalizzati di certificazione delle competenze, orientamento, ricerca e sostegno all'auto-imprenditorialità finalizzate all'inserimento o reinserimento nel mercato del lavoro ed il 100\% delle spese relative alla partecipazione a convegni, congressi e corsi di aggiornamento professionale, come pure il 100\% delle spese assicurative destinate a garantire questi lavoratori contro il mancato pagamento delle prestazioni di lavoro autonomo. si ipotizza un tetto massimo di 10 mila euro. appalti e orientamento sportelli dedicati sono previsti nelle amministrazioni pubbliche per dare pubblicità e favorire la partecipazione agli appalti, come pure nei centri per l'impiego e per tutti i soggetti accreditati che offrono servizi per il lavoro, per raccogliere domande ed offerte di lavoro, fornire informazioni su avvio delle attività, agevolazioni pubbliche e accesso al credito. fondi strutturali i lavoratori autonomi vengono equiparati ai piccoli imprenditori e quindi potranno accedere ai fondi strutturali europei. maternità le lavoratrici autonome non saranno più obbligate a sospendere del tutto l'attività lavorativa durante i 5 mesi di maternità previsti dalla legge e percepiranno l'indennità di maternità indipendentemente dalla effettiva astensione dall'attività lavorativa. malattia grave i trattamenti terapeutici di malattie oncologiche sono equiparati alla degenza ospedaliera. e comunque in caso di malattia superiore a due mesi si potrà sospendere il pagamento dei contributi sociali fino a due anni. li si potrà saldare a rate al termine della malattia per un periodo pari al triplo della fase di sospensione del pagamento. lavoro agile il "lavoro agile", lo smart working che oggi va tanto di moda, precisa come prima cosa la nuova legge, "non consiste in una nuova tipologia contrattuale ma in una modalità flessibile di svolgimento del lavoro subordinato finalizzata ad incrementare la produttività agevolando al contempo la conciliazione dei tempi di lavoro e di vita". può essere eseguito in parte all'interno dei locali aziendali e in parte all'esterno, "entro i soli limiti di durata massima dell'orario di lavoro giornaliero e settimanale, derivanti dalla legge e dalla contrattazione collettiva". anche "al fine di evitare equivoci interpretativi" è poi previsto che il lavoratore che presta attività di lavoro subordinato in modalità agile abbia diritto di ricevere un trattamento economico e normativo "non inferiore a quello complessivamente applicato ai lavoratori che svolgono le medesime mansioni esclusivamente all'interno dell'azienda". quindi gli incentivi di carattere fiscale e contributivo, eventualmente riconosciuti grazie agli incrementi di produttività ed efficienza, dovranno essere corrisposti anche quando l'attività lavorativa è prestata in modalità di lavoro agile. ed infine il datore di lavoro è tenuto a garantire al lavoratore la salute e la sicurezza, consegnandogli, a tal fine, un'informativa scritta nella quale sono individuati i rischi generali e i rischi specifici connessi allo smart working. twitter @paoloxbaroni alcuni diritti riservati. & 650 & very low & Low & Socio-Economic & NA & NA & 2016-01-29 & 2016 & 2 & ECO
Frame & v.low & National & 500-1000 & -0.5155635 & -0.1667445 & -0.9728619 & 0.5015415 & 1.3050799 & 3.2 & 0.6840199 & -1.2389660 & Payer & Domestic & Domestic & Domestic & Domestic|ECO & Positive\\
Italy & https://livesicilia.it/2018/08/13/val-simeto-progetto-da-31-milioni-ce-lok-del-governo-regionale\_988081/ & 423 & Live Sicilia & Private/Non-Public & Online only & Regional/Local & medium = CP is important part of story & Cultural development & Positive & Subnational & No myth & Public services & Positive & Subnational & No myth & NA & NA & NA & NA & Italy & val simeto, progetto da 31 milioni c'è l'ok del governo regionale & 2018-08-13 & fondo europeo di sviluppo regionale & musumeci: "intervento a favore delle aree interne e montane" palermo - via libera del governo musumeci alla strategia per l'area interna sperimentale val simeto. l'iniziativa coinvolge i comuni di adrano (capofila) e biancavilla nel catanese e centuripe in provincia di enna. con l'approvazione della delibera, da parte della giunta, si avvia il percorso tecnico per l'elaborazione dell'accordo di programma-quadro da parte dei ministeri competenti, dell'agenzia per la coesione territoriale e della regione siciliana, sbloccando così circa 31 milioni di euro. il progetto "liberare radici per generare cultura" rientra nel ciclo di programmazione del fondo europeo di sviluppo regionale 2014/2020, che ha assegnato un ruolo centrale allo sviluppo locale e alle politiche territoriali. cinque le aree interne della sicilia interessate (madonie, simeto etna, nebrodi, terre sicane e calatino), caratterizzate da un più elevato e differenziato grado di marginalità e svantaggio. le aree sono state individuate in funzione della loro elevata distanza dai centri di offerta di servizi-base relativi ai settori della salute, dell'istruzione e dell'accessibilità. "le aree interne e montane - sottolinea il presidente della regione, nello musumeci - sono quelle più vulnerabili e che rischiano quotidianamente di spopolarsi sempre di più. mettendo in campo queste risorse finanziarie proviamo a evitarlo. sono, comunque, fortemente preoccupato del tempo perduto, che in sicilia risulta molto più pericoloso che in altre parti d'italia. servono crono-programmi di lavoro precisi che permettano di recuperare i ritardi e passare all'attività pratica. in questo momento di grandissima difficoltà economica, attivare tempestivamente queste risorse dedicate a territori marginali assume un'importanza strategica. le aree interne sono ricche di risorse naturali e culturali esclusive che, se opportunamente valorizzate, potrebbero innescare nuovi percorsi di crescita e di sviluppo". gli interventi previsti nel piano sono mirati a innalzare il livello quantitativo e qualitativo dei servizi essenziali rivolti alla popolazione. le risorse, la cui quasi totalità graverà sui fondi europei (fesr, fse e feasr), verranno destinate a vari ambiti: istruzione, salute, digitalizzazione, viabilità, tutela del territorio, artigianato, energia, agroalimentare. in particolare, oltre tredici milioni di euro verranno utilizzati per la manutenzione straordinaria di diverse strade provinciali e statali, sei milioni per interventi che riducano i consumi energetici in edifici pubblici, più di quattro milioni per il potenziamento dei servizi sanitari. (ansa). share lunedì 13 agosto 2018 - 11:22 & 386 & medium & Medium & Socio-Economic & Socio-Economic & NA & 2018-08-13 & 2018 & 3 & ECO
Frame & low-medium & Regional & <500 & -0.5155635 & -0.1667445 & -0.9728619 & 0.5015415 & 1.3050799 & 3.2 & 0.6840199 & -1.2389660 & Payer & Domestic & Domestic & Domestic & Domestic|ECO & Positive\\
Italy & http://www.adnkronos.com/soldi/lavoro/2015/02/26/emilia-romagna-fse-mln-euro-per-disoccupati\_AIOqnsHN3uAOywFJHpTdEM.html & 463 & Adnkronos & Private/Non-Public & Online only & National & high = CP is most important issue in story (can also cover other issues) & Jobs & Positive & Subnational & No myth & NA & NA & NA & NA & NA & NA & NA & NA & Italy & emilia romagna: da fse 40 mln di euro per disoccupati & 2015-02-26 & fondo sociale europeo & ammontano a 40 milioni di euro le risorse del fondo sociale europeo (fse) destinate dalla giunta regionale dell'emilia romagna a due avvisi pubblici di finanziamento per interventi di politica attiva del lavoro rivolti a inoccupati, disoccupati e persone svantaggiate. "entrambe le azioni approvate dalla giunta sono previste dal programma dei primi 100 giorni di mandato", spiega l'assessore regionale alla formazione e al lavoro, patrizio bianchi, rimarcando che "è stato rispettato l'impegno preso, anche nella consapevolezza che fosse importante avviare la nuova programmazione del fondo sociale europeo 2014-2020 partendo dalle necessità delle persone in cerca di lavoro e delle persone a maggiore rischio di esclusione sociale". l'obiettivo, infatti, "è stimolare il dinamismo della società in tutte le sue componenti e attraverso il lavoro - conclude bianchi - avviare una nuova fase di sviluppo in cui ritrovare una nuova coesione sociale". il primo avviso pubblico al quale possono rispondere gli enti di formazione professionale in partenariato con le imprese, prevede la realizzazione di politiche attive rivolte a persone inoccupate e disoccupate: percorsi formativi per il conseguimento di una qualifica professionale di operatori o di tecnici spendibili nei diversi settori e nelle differenti funzioni del sistema produttivo regionale. la richiesta al sistema formativo è di saper cogliere la domanda di professionalità delle imprese e insieme a queste progettare e realizzare percorsi formativi mirati. il secondo bando al quale possono rispondere gli enti di formazione professionale intende finanziare, invece, piani di intervento"territoriali definiti e realizzati in collaborazione con soggetti, pubblici e privati per favorire l'inserimento lavorativo delle persone in carico ai servizi sociali a rischio di esclusione, marginalità e discriminazione. entrambi i bandi hanno una dotazione finanziaria di 20 milioni di euro ciascuno. l'invito a presentare operazioni per l'accesso all'occupazione di persone inattive scade il 23 aprile 2015, mentre per il bando sull'inclusione lavorativa delle persone svantaggiate scade il 2 aprile 2015. & 317 & high & High & Socio-Economic & NA & NA & 2015-02-26 & 2015 & 1 & ECO
Frame & high-very high & National & <500 & -0.5155635 & -0.1667445 & -0.9728619 & 0.5015415 & 1.3050799 & 3.2 & 0.6840199 & -1.2389660 & Payer & Domestic & Domestic & Domestic & Domestic|ECO & Positive\\
Italy & http://www.adnkronos.com/lavoro/2015/08/05/rassegna-stampa-lavoro-nei-quotidiani-oggi\_4hhaQm92lwXxbsSfAl0XfK.html & 403 & Adnkronos & Private/Non-Public & Online only & National & medium = CP is important part of story & Ineffective goal achievement & Negative & EU + National + Subnational & 4.No added value & Bureaucracy and/or delays & Negative & EU + National + Subnational & No myth & NA & NA & NA & NA & Italy & rassegna stampa: il lavoro nei quotidiani di oggi & 2015-08-05 & fondi strutturali & ampio spazio, nei giornali in edicola, alla riforma della pubblica amministrazione e al nuovo cda della rai e ancora commenti sul sud. in un'intervista al 'messaggero', il costituzionalista sabino cassese, ex ministro della funzione pubblica del gabinetto ciampi, afferma che "l'amministrazione pubblica è un grande malato che va curato, mi sembra che, in questo senso, la riforma contenga una buona diagnosi alla quale segue una buona cura". "questa è una legge che abbraccia molti temi: il procedimento amministrativo, la digitalizzazione, le camere di commercio e molto altro. ma è fuori discussione che il cuore della riforma è la riorganizzazione della dirigenza", spiega sottolineando che "è importante che sia stato introdotto il principio della non inamovibilità: il posto a vita non c'è più e deve essere meritato". molte le interviste ai neo-consiglieri della rai. fra quelle rilasciate da carlo freccero, al 'corriere della sera' dice: "la situazione è dura, invecchia tutto rapidamente, il futuro diventa presente, la tv si fa rete, la tv generalista oggi è un antico concetto rispetto al video on demand. la rai dovrà uscire dal ghetto". a 'repubblica', guelfo guelfi ricorda: "ho l'esperienza di quello che ho sempre fatto, un pubblicitario prestato alla comunicazione pubblica". sempre 'repubblica' interpella giancarlo mazzuca: "sono sempre stato un uomo di mediazione. di certo non andrò a scontri particolari ma cercherò sempre una sintesi". al 'tempo' arturo diaconale dichiara: "credo che l'azienda abbia dei grandissimi meriti, ma anche che si sia eccessivamente burocratizzata nel corso del tempo. ecco, da liberale sono convinto che uno degli obiettivi sia una sua sburocratizzazione". all'unità' rita borioni confessa: "la prima cosa che mi viene in mente è che l'italia non si racconta da un sacco di tempo, viene raccontata dagli altri". sul fronte politico, al 'corriere della sera' pierluigi bersani, ex segretario del pd, ricorda che "l'ispirazione dell'ulivo era l'idea di un partito riformista che tenesse rapporti con una radicalità di sinistra e con una radicalità civica" e che "sul piano del programma i punti sono europa, investimenti per l'occupazione e il sud, liberalizzazioni, politica industriale e fedeltà fiscale". sempre il 'corriere della sera' intervista renato brunetta, capogruppo di forza italia alla camera: "credo che una grosse koalition, nata da un'approfondita trattativa e da un dettagliato programma, consentirebbe di varare serenamente le riforme necessarie al paese tanto in campo costituzionale che in campo economico. e sarebbe la vera pacificazione". torna sulle polemiche scaturite dal ddl concorrenza, in un'intervista al 'corriere della sera', andrea martella, vicepresidente del gruppo pd alla camera e relatore del provvedimento: "non c'è alcuna marcia indietro o frenata sulle liberalizzazioni. ma la concorrenza deve aumentare: questo è solo il primo passo. e ne dovranno seguire molti altri per cambiare la società e l'economia italiana. questa è una sfida cruciale anche dal punto di vista dello sviluppo". e aggiunge: "abbiamo cercato di ascoltare i cittadini e abbiamo lavorato per combattere i poteri forti, favorire la crescita e creare nuove opportunità per i consumatori. il provvedimento, però, è complesso e articolato: per questo abbiamo fatto quasi 80 audizioni. e comunque questa è la prima legge sulla concorrenza dal 2009: già questo mi sembra un segnale chiaro". ancora ampio spazio alla 'questione sud'. in un'intervista a 'repubblica', sandro gozi, sottosegretario alla presidenza del consiglio con la delega per le politiche europee, afferma: "il governo renzi ha avviato una fase nuova nelle politiche per il mezzogiorno a cominciare da un significativo recupero della capacità di spesa dei fondi strutturali europei". e aggiunge: "ora siamo impegnati nella programmazione 2014-2020 e puntiamo a concludere entro settembre la fase di approvazione dei progetti. e di questo il mezzogiorno si gioverà in maniera particolare". l''unità' intervista domenico arcuri, ad invitalia: "e' arrivato il momento di girare pagina, per davvero di girarla. siamo all'alba della nuova stagione dei fondi europei. abbiamo cinque anni di tempo: decidiamo, una volta per tutte, che undici programmi operativi nazionali e 22 operativi regionali, che migliaia di linee di azione, che fontane, rotonde, fiere e sponsorizzazioni di eventi musicali non sono servite e non servono a produrre sviluppo". sul tema sud, 'repubblica' intervista guntram wolff, direttore del bruegel institute, think-tank di bruxelles: "parliamoci chiaro: l'esperienza in quasi tutta europa, dal sud d'italia fino alla vecchia germania est, ha dimostrato che l'efficacia dei fondi strutturali è molto dubbia. la produttività non cresce e i trasferimenti netti, anche quando durano per decenni, non hanno dato un contributo efficace". ad 'avvenire', gianfranco viesti, docente di economia applicata all'università di bari, illustra una delle misure possibili, quella del reddito minimo: "mi riferisco al modello reis avanzato dall'alleanza contro le povertà o al sia creato dall'ex ministro giovannini durante il governo letta. l'importante, comunque, è trovare il modo di sostenere quelle famiglie che sono finite in profonda sofferenza e fanno fatica anche a soddisfare i bisogni primari. il rilancio in tempi brevi può passare anche da interventi di rilancio, come il rafforzamento dei collegamenti aerei". il 'mattino' intervista adriano giannola, curatore del rapporto svimez: "noi abbiamo fatto il nostro rapporto annuale, tocca alla politica intervenire e non ridurre questo dibattito a un fuoco estivo,per ritrovarci tra un anno con i problemi aggravati". parla di sicurezza, in un'intervista al 'tempo', cosimo ferri, sottosegretario alla giustizia: "io punterei sul perfezionamento del testo unico sull'immigrazione, agendo sull'istituto dell'espulsione dell'immigrato colpevole di reato come sanzione sostitutiva della pena, ampliandone l'ambito di applicazione nei presupposti normativi ed agevolandone le modalità attuative". sul fronte energetico, 'avvenire' intervista il presidente dell'unione petrolifera, alessandro gilotti: "la strada maestra per il futuro della raffinazione è quella del cosiddetto 'rightsizing': nuovi servizi e nuovi prodotti. un taglio che sia anche un restyling. quello che oggi ci chiede il mercato è emissioni più basse e prodotti diversi, tutte cose che richiedono grandi investimenti. bisogna unire i motori tradizionali ai nuovi combustibili fossili e ridurre i consumi aumentando l'efficienza. realisticamente il futuro sarà sempre più dei motori ibridi. poi sul fronte carburante c'è da lavorare su quelli con minore contenuto di carbonio. come i combustibili gassosi e i nuovi biocarburanti". a livello locale, dopo lo sblocco dei fondi per il giubileo, la 'stampa' intervista ignazio marino, sindaco di roma: "davanti al piano di rientro che il comune ha rispettato, il mef ci ha informato che sono stati sbloccati 50 milioni dal piano di gestione del debito storico di roma, fondi che vanno ad aggiungersi agli altri 150 di cui avevamo già disponibilità e questo significa che da questa mattina il campidoglio è pronto ad investire per le opere del giubileo secondo un dettagliato piano di opere infrastrutturali, di manutenzione urbana". torna sui fatti dell'aeroporto di roma, in un'intervista a 'qn', pietro giordano, presidente di adiconsum: "ben vengano le sanzioni di fronte ai fatti gravissimi che sono accaduti a fiumicino. del resto l'enac, come l'autorità di controllo sui trasporti, ha il potere di comminare multe anche molto pesanti e persino revocare temporaneamente la licenza di volo". & 1183 & medium & Medium & Socio-Economic & Governance & NA & 2015-08-05 & 2015 & 1 & ECO
Frame & low-medium & National & +1000 & -0.5155635 & -0.1667445 & -0.9728619 & 0.5015415 & 1.3050799 & 3.2 & 0.6840199 & -1.2389660 & Payer & Domestic & European & Mixed & Domestic|ECO & Negative\\
\addlinespace
Italy & https://notizie.tiscali.it/regioni/toscana/articoli/innovazione-microcredito-tasso-zero/ & 454 & Tiscali & Private/Non-Public & Online only & National & high = CP is most important issue in story (can also cover other issues) & Economic development & Positive & Subnational & No myth & Research \& innovation & Positive & Subnational & No myth & NA & NA & NA & NA & Italy & innovazione, microcredito a tasso zero & 2018-09-21 & fondo europeo di sviluppo regionale & (ansa) - firenze, 21 set - microcredito a tasso zero per investimenti in nuove tecnologie, rivolto a micro e piccole imprese e a liberi professionisti della toscana. e' il bando, approvato dalla regione e cofinanziato dal programma operativo regionale (por) del fondo europeo di sviluppo regionale (fesr) 2014-2020, aperto dal 20 settembre 2018 fino ad esaurimento risorse, circa 0,7 mln di euro. l'obiettivo è sostenere e incrementare gli investimenti in macchinari, impianti e beni intangibili di accompagnamento nei processi di riorganizzazione e ristrutturazione in linea con la strategia di ricerca e innovazione per la specializzazione intelligente in toscana (ris 3) che ha tre priorità tecnologiche: ict information communication technologies e fotonica, fabbrica intelligente, chimica e nanotecnologia. può essere presentata una sola domanda di aiuto - da inviare online sul sito www.toscanamuove.it - e il valore totale del progetto di investimento ammesso non può essere inferiore a 10.000 euro e superiore a 40.000. & 155 & high & High & Socio-Economic & Socio-Economic & NA & 2018-09-21 & 2018 & 3 & ECO
Frame & high-very high & National & <500 & -0.5155635 & -0.1667445 & -0.9728619 & 0.5015415 & 1.3050799 & 3.2 & 0.6840199 & -1.2389660 & Payer & Domestic & Domestic & Domestic & Domestic|ECO & Positive\\
Italy & http://www.ansa.it/europa/notizie/rubriche/europa\_delle\_regioni/2017/07/12/il-belga-lambertz-e-nuovo-presidente-comitato-regioni-ue\_5fa9137e-8c15-4825-a010-f6acbea27154.html & 474 & ANSA.it & Private/Non-Public & Online only & National & very low = CP mentioned once & Empowerment of institutions & Positive & EU & No myth & NA & NA & NA & NA & NA & NA & NA & NA & Italy & il belga lambertz è nuovo presidente comitato regioni ue - europa delle regioni & 2017-07-12 & politica di coesione & bruxelles - è il belga karl-heinz lambertz il nuovo presidente del comitato europeo delle regioni (cdr), che, dopo due anni e mezzo di vicepresidenza, è stato eletto per acclamazione per succedere al finlandese markku markkula (ppe). senatore e rappresentante della piccola comunità germanofona del belgio (circa 76mila persone), lambertz è membro del cdr dal 1999 e nel giugno 2011 è diventato presidente del gruppo del pse, carica oggi ricoperta dalla governatrice dell'umbria catiuscia marini. dal 2000 lambertz fa parte anche del congresso del poteri locali e regionali del consiglio d'europa, di cui è vicepresidente. "dopo il voto sulla brexit, la disintegrazione e la disunione non sono più dei rischi immaginari", ma "noi continueremo ad avere un rapporto privilegiato con i membri britannici" del cdr, ha detto lambertz durante il suo discorso d'insediamento. il neopresidente ha poi annunciato che dal prossimo autunno il comitato ospiterà un discorso sullo stato dell'unione dal punto di vista degli enti locali, così come già fa la commissione ue. "senza la dimensione territoriale, l'europa rischia di esser un albero senza terreno, incapace di radicarsi e condannata a scomparire", ha chiosato lambertz, che in questo senso ha difeso l'importanza della politica di coesione. "un'unione senza politica di coesione non è l'europa che vogliamo", ha detto. & 216 & very low & Low & Power & NA & NA & 2017-07-12 & 2017 & 2 & POL
Frame & v.low & National & <500 & -0.5155635 & -0.1667445 & -0.9728619 & 0.5015415 & 1.3050799 & 3.2 & 0.6840199 & -1.2389660 & Payer & European & European & European & European|POL & Positive\\
Italy & http://www.ansa.it/lombardia/notizie/2017/10/25/lombardia-con-fse-investiti-440-mln-per-lavoro-e-formazione\_cd166536-133a-4ca1-9fbc-1aadc3b2946a.html & 405 & ANSA.it & Private/Non-Public & Online only & National & very high = CP is most important issue + CP is mentioned in title/headline & Jobs & Positive & Subnational & No myth & NA & NA & NA & NA & NA & NA & NA & NA & Italy & lombardia: con fse investiti 440 mln per lavoro e formazione - lombardia & 2017-10-25 & fondo sociale europeo & (ansa) - milano, 25 ott - con il fondo sociale europeo la lombardia ha investito nel triennio 2015-2017 440 milioni di euro, destinati allo sviluppo di politiche attive per il lavoro e delle filiere professionalizzanti. lo ha reso noto l'assessore regionale all'istruzione valentina aprea, sottolineando che sono oltre 96 mila i cittadini lombardi accompagnati al lavoro grazie allo strumento dello dote unica lavoro. aprea ha parlato dei progetti avviati grazie al fondo sociale europeo in lombardia durante l'evento annuale por-fse 2014-2020, che ha aperto l'edizione 2017 di expotraining, fiera dedicata al lavoro e formazione. "con il fondo sociale abbiamo attivato risorse finanziare sul territorio pari a circa 440 milioni di euro nel triennio 2015-2017 per favorire la crescita attraverso un nuovo modello lombardo di politiche attive del lavoro (dote unica lavoro), lo sviluppo delle filiere professionalizzanti lombarde 4.0 che abbiamo definito 'a scuola di mestieri del futuro' e la modernizzazione delle politiche sociali per la promozione dell'autonomia e inclusione delle persone a rischio di esclusione" ha spiegato aprea. "grazie agli investimenti di questi anni - ha sottolineato l'assessore - abbiamo innovato il sistema di istruzione regionale: nell'anno formativo 2017/2018 sarà infatti possibile il passaggio da 'operatore qualificato' a 'tecnico del futuro' senza frequentare il percorso scolastico statale e sostenere l'esame di stato". "stiamo investendo con dote unica lavoro circa 140 milioni di euro. grazie a queste risorse stiamo accompagnando al lavoro più di 96.000 cittadini e cittadine, di cui più di 43.600 donne. ad oggi, più di 63.000 partecipanti hanno raggiunto un risultato occupazionale positivo" ha aggiunto. (ansa). & 271 & very high & High & Socio-Economic & NA & NA & 2017-10-25 & 2017 & 2 & ECO
Frame & high-very high & National & <500 & -0.5155635 & -0.1667445 & -0.9728619 & 0.5015415 & 1.3050799 & 3.2 & 0.6840199 & -1.2389660 & Payer & Domestic & Domestic & Domestic & Domestic|ECO & Positive\\
Italy & http://www.ilsole24ore.com/art/notizie/2017-02-23/scuola-fedeli-partira-percorso-conoscenza-leggi-razziali-144121.shtml?uuid=AEGsr1b & 446 & Il Sole 24 ORE & Private/Non-Public & Online and Offline & National & very low = CP mentioned once & Public services & Positive & EU + National & No myth & NA & NA & NA & NA & NA & NA & NA & NA & Italy & scuola, fedeli: partirà percorso di conoscenza sulle leggi razzialiscuola, fedeli: partirà percorso di conoscenza sulle leggi razziali & 2017-02-23 & fondo sociale europeo & sarà avviato in tutte le scuole italiane un percorso di conoscenza sui ciò che è avvenuto con le leggi razziali. lo ha annunciato oggi la ministra dell'istruzione, valeria fedeli, in occasione della sua visita alla scuola ebraica di roma. accolta dalla presidente della comunità ebraica di roma, ruth dureghello, e dai direttori delle scuole ebraiche, rabbino benedetto carucci e milena pavoncello, la ministra ha visitato alcune classi dell'istituto che tra elementari, medie e liceo ospita circa 900 studenti, mentre altri 100 bimbi frequentano l'asilo in una sede distaccata. rispondendo alle domande dei cronisti a margine della visita, il ministro ha poi assicurato che "sono state già state accantonate" le risorse del pon (programma operativo nazionale) da destinare alle scuole paritarie. e ha dichiarato che quella di erasmus è un'esperienza che "va estesa". "impegno di tutti perchè scuola ebraica sia sostenuta" "mi sono seduta tra i banchi di una quinta elementare e ho assistito a parte di una lezione - ha raccontato fedeli - era sui 10 comandamenti e da laica ho ritrovato in quella lezione lo straordinario impegno sancito con l'articolo 3 della nostra costituzione e cioè il contrasto a ogni discriminazione". "queste scuole, fulcro della vita della nostra comunità, devono sopravvivere e trovare, attraverso l'impegno concreto del governo, motivo per andare avanti" ha osservato ruth dureghello. e la ministra le ha assicurato che "c'è l'impegno di tutti a far sì che la scuola ebraica sia sostenuta" perché "è una straordinaria ricchezza". "senza di voi saremmo tutti più poveri" ha aggiunto dopo aver ricevuto in dono dai bimbi della primaria un album, da loro illustrato, sulle principali ricorrenze ebraiche. "in questa scuola si pratica il nuovo dizionario della crusca perché c'è scritto 'alla ministra'" si è rallegrata, sfogliandolo, valeria fedeli, che alla declinazione al femminile della sua carica tiene molto. l'inno di mameli cantato da tutti ha chiuso la visita istituzionale. "già accantonate risorse pon per paritarie" "le risorse del pon da destinare alle scuole paritarie verranno sbloccate nei prossimi giorni, dopo un necessario passaggio burocratico a bruxelles" ha assicurato fedeli. che ha poi spiegato: "c'è una novità nell'ultima legge di bilancio: si prevede che anche le scuole paritarie, quelle davvero verificate e accreditate, partecipino ai bandi per le risorse del pon scuola, al pari di quelle statali. una scelta importante perché bisogna rispettare le leggi italiane". tuttavia - ha ricordato il ministro - poiché l'accordo di partenariato siglato dal governo italiano con la commissione europea vieta la partecipazione delle scuole paritarie, il miur ha avviato un procedimento di modifica dell'accordo assieme alla bruxelles. nelle more, per consentire comunque l'avvio dell'avviso quadro del pon (la parte relativa al fondo sociale europeo) e dei relativi bandi, sono state comunque accantonate le necessarie risorse finanziarie, che consentiranno di partire con l'indizione di bandi specifici per le scuole paritarie, non appena sarà completata la modifica dell'accordo. "erasmus è esperienza che va estesa" l'esperienza erasmus "va estesa, dando la possibilità a tutti di poter entrare nel progetto" ha dichiarato poi la ministra dell'istruzione, rispondendo a chi le chiedeva se il programma europeo di mobilità, a 30 anni dal suo varo, avesse bisogno di fare il tagliando. "la generazione erasmus è la generazione che ha già costruito l'europa positiva. ci si incrocia, si costruiscono culture comuni e studi comuni. ci si innamora, quindi si creano vite e nativi europei" ha aggiunto fedeli. & 574 & very low & Low & Socio-Economic & NA & NA & 2017-02-23 & 2017 & 2 & ECO
Frame & v.low & National & 500-1000 & -0.5155635 & -0.1667445 & -0.9728619 & 0.5015415 & 1.3050799 & 3.2 & 0.6840199 & -1.2389660 & Payer & Domestic & European & Mixed & Domestic|ECO & Positive\\
Italy & http://roma.repubblica.it/cronaca/2016/06/18/news/roma\_confiscati\_beni\_per\_43\_milioni\_a\_gruppo\_di\_trafficanti\_di\_droga-142280312/ & 412 & Repubblica.it & Private/Non-Public & Online and Offline & National & very low = CP mentioned once & Fraud/Corruption & Negative & Subnational & 7.Fraud & NA & NA & NA & NA & NA & NA & NA & NA & Italy & albano, confiscati beni per 43 milioni a gruppo di trafficanti di droga & 2016-06-18 & fondo sociale europeo & finanzieri del comando provinciale di roma hanno sottoposto a confisca 18 aziende tra società di persone e di capitali, immobili, auto/motoveicoli e numerosi rapporti finanziari, per un valore complessivo di oltre 43 milioni di euro, riconducibili a componenti del gruppo facente capo a sergio anderlucci e coinvolto in traffici di droga. secondo le indagini patrimoniali coordinate dalla procura della repubblica di velletri, l'organizzazione attingendo a risorse finanziarie rinvenienti principalmente dallo spaccio di stupefacenti ha costituito una serie di società, in prevalenza con sede nel comune di albano laziale (roma) e operanti in svariati settori economici. alcune di queste società sarebbero state poi utilizzate per commettere ulteriori illeciti di natura penale tributaria. contestualmente alla confisca di beni, sono state notificate, di concerto con i commissariati di albano laziale e genzano, cinque misure di prevenzione della sorveglianza speciale di pubblica sicurezza nei confronti di altrettante persone, tre delle quali sono state sottoposte anche all'obbligo di soggiorno. in particolare, scrive il giudice della prevenzione, "il 'sistema' architettato dal gruppo prevedeva di mettere in liquidazione, di volta in volta, una delle cooperative, in modo che i contratti attribuiti a tale soggetto giuridico, in fase di estinzione, venissero trasferiti ad un nuovo soggetto giuridico, tanto che, se non fosse intervenuto, provvidenzialmente, il sequestro di prevenzione disposto nell'ambito del presente procedimento, verosimilmente l'intero 'marchingegno' sarebbe andato a buon fine senza che ne rimanesse traccia? l'intero meccanismo permetteva al gruppo di non versare (come in effetti non ha mai versato) imposte allo stato". gli accertamenti patrimoniali sviluppati dagli specialisti del gico del nucleo polizia tributaria di roma, che hanno riguardato 122 entità tra persone fisiche e giuridiche, hanno permesso di ricostruire una fitta rete degli interessi commerciali di investimenti, localizzati sempre nell'area dei castelli romani, anche con l'impiego di familiari o di terzi prestanome. a fronte di un ingente patrimonio, venivano presentati al fisco profili reddituali inconsistenti. le costruzioni edili erano una delle principali attività del gruppo anderlucci che era riuscito ad introdursi negli appalti degli enti locali: in particolare, una delle aziende oggetto di sequestro, nel luglio 2014, aveva realizzato un complesso destinato ad ospitare una scuola materna ed un asilo nido e aveva avuto accesso, tra l'altro, a contributi del fondo sociale europeo. dopo il sequestro disposto dal tribunale di roma, è stata l'amministrazione giudiziaria nominata dai giudici a portare a termine i lavori appaltati nell'aprile 2016, quando è stato inaugurato il progetto "cecchina 2", nell'area plus del comune di albano laziale. la confisca ha riguardato il patrimonio aziendale di 18 società di persone e capitali, attive tra l'altro nell'edilizia, nei servizi di logistica, nella consulenza e assistenza economico finanziaria, nelle pulizie, nel commercio all'ingrosso di articoli antincendio e antinfortunistica, in attività creditizie, nel noleggio autoveicoli; 9 unità immobiliari in provincia di roma; 8 tra auto e moto; 27 rapporti finanziari. valore complessivo di stima: 43.566.156,00 euro. & 490 & very low & Low & Governance & NA & NA & 2016-06-18 & 2016 & 2 & POL
Frame & v.low & National & <500 & -0.5155635 & -0.1667445 & -0.9728619 & 0.5015415 & 1.3050799 & 3.2 & 0.6840199 & -1.2389660 & Payer & Domestic & Domestic & Domestic & Domestic|POL & Negative\\
\addlinespace
Italy & http://www.ansa.it/sito/notizie/topnews/2017/10/07/italia-perde-160-mln-di-fondi-ue\_e926991a-8646-4a65-9421-892294ef4c9e.html & 467 & ANSA.it & Private/Non-Public & Online only & National & very high = CP is most important issue + CP is mentioned in title/headline & Mismanagement & Negative & EU + National & 10.Slow spend & NA & NA & NA & NA & NA & NA & NA & NA & Italy & italia perde 160 mln di fondi ue - ultima ora & 2017-10-07 & fondo europeo di sviluppo regionale & (ansa) - bruxelles, 7 ott - sono quasi 160 i milioni di euro provenienti dalle casse europee che l'italia ha perso in modo definitivo per non essere stata capace di spenderli entro tempi e modalità dettati dalle regole ue. il dato emerge dalla chiusura dei conti relativi al periodo 2007-2013 e riguarda programmi finanziati nell'ambito del fondo europeo di sviluppo regionale. una situazione che a bruxelles definiscono negativa, ma non drammatica, visto che l'italia è comunque riuscita a spendere 34,4 miliardi di fondi, grazie all'accelerazione impressa negli ultimi due anni da un'apposita task force voluta dalla commissaria europea alle politiche regionali corina cretu. dei 159.535.986 euro "disimpegnati", più del 70\% è la fetta persa dalla regione sicilia. il primo vero esame si avrà a fine 2018, quando in base alle nuove regole ue si dovranno chiudere i conti per il 2015. il rischio, avvertono a bruxelles, è quello di perdere già allora alcuni dei fondi europei messi a bilancio se non saranno stati debitamente impiegati. & 172 & very high & High & Governance & NA & NA & 2017-10-07 & 2017 & 2 & POL
Frame & high-very high & National & <500 & -0.5155635 & -0.1667445 & -0.9728619 & 0.5015415 & 1.3050799 & 3.2 & 0.6840199 & -1.2389660 & Payer & Domestic & European & Mixed & Domestic|POL & Negative\\
Italy & http://notizie.tiscali.it/esteri/articoli/bilancio-ue-2021-20-cifre-novita-proposta-commissione/ & 469 & Tiscali & Private/Non-Public & Online only & National & low = CP mentioned more times but NOT important part of story (mainly about others issues) & Institutional bargaining over funding & Factual & EU & No myth & Improve governance & Positive & EU + National & No myth & NA & NA & NA & NA & Italy & bilancio ue 2021-20, cifre e novità di proposta commissione & 2018-05-02 & politica di coesione & bruxelles, 2 mag. (askanews) - l'attesa proposta per il nuovo quadro finanziario pluriennale (mff) dell'ue dopo il 2020, presentata oggi a bruxelles dal presidente della commissione europea jean-claude juncker e dal commissario al bilancio guenther oettinger alla plenaria dell'europarlamento e poi alla stampa, non contiene grandi sorprese in termini quantitativi, ma introduce comunque alcune innovazioni importanti. l'impressione è che siano state affrontate in modo adeguato le due sfide più importanti: l'adattamento del bilancio alla nuova situazione senza il regno unito dopo la brexit, che comporta una perdita di risorse di circa 15 miliardi di euro, e le nuove priorità sollecitate dagli stati membri (immigrazione e frontiere, ricerca e innovazione, sicurezza e difesa, programmi per i giovani e nuovi strumenti per l'eurozona). il negoziato che comincia ora con i governi e con il parlamento europeo potrebbe comunque ritoccare le cifre, e ridimensionare le novità.complessivamente, per il periodo 2021-2027, la commissione propone un bilancio da 1.279 miliardi di euro, in impegni, espressi in prezzi correnti, ovvero tenendo conto dell'inflazione (equivalenti a 1.135 miliardi di euro espressi in prezzi del 2018); una cifra pari all'1,114\% del reddito nazionale lordo dell'ue a 27. un po di più dei 1.087,1 miliardi di euro che erano stati stanziati per il quadro finanziario di bilancio in corso, 2014-2020, che rappresentava l'1,03\% del reddito nazionale lordo dei ventotto.da dove prendere i soldi per finanziare l'aumento delle risorse per le nuove priorità? la riduzione dei finanziamenti colpirà essenzialmente due settori: la pac, la politica agricola comune (37\% del bilancio attuale), e la politica di coesione (35\% del bilancio). l'orientamento della commissione è di proporre una riduzione al 60\% del peso totale di queste due voci nel bilancio complessivo, con un 7\% in meno nella coesione e il 5\% in meno nella pac.proprio la redistribuzione dei finanziamenti nella politica di coesione sarà una delle cose a cui guardare con più attenzione: dopo la crisi, i paesi del sud ad alta disoccupazione (compresa l'italia) vorrebbero riprendersi una parte della cospicua fetta di torta che l'ultima volta era andata ai nuovi paesi membri dell'europa centro orientale, le cui economie vanno oggi meglio proprio grazie ai fondi ue. la politica di coesione, inoltre, verrà parzialmente riorientata, con un ruolo sempre più importante a sostegno delle riforme strutturali e dell'integrazione a lungo termine dei migranti.la commissione intende poi semplificare la struttura del bilancio, riducendo di oltre un terzo il numero dei programmi che passeranno dagli attuali 58 a 37, ad esempio riunendo in nuovi programmi integrati le fonti di finanziamento attualmente frammentate e razionalizzando profondamente l'uso degli strumenti finanziari, anche tramite il nuovo fondo investeu.inoltre, la proposta presentata oggi punta a rendere il bilancio ue più flessibile, per poter spostare, se necessario, una parte delle risorse sia fra i diversi programmi che al loro interno. e' previsto anche il rafforzamento degli strumenti di gestione delle crisi e la creazione di una nuova 'riserva dell'unione' che permetta di affrontare eventi imprevisti e rispondere a situazioni di emergenza in settori quali la sicurezza e le migrazioni.le innovazioni più importanti, comunque, riguardano il nuovo meccanismo che penalizzerebbe finanziariamente i paesi membri in cui si registrino delle 'carenze' nello stato di diritto, e i due nuovi strumenti di bilancio per sostenere rispettivamente le riforme strutturali e la stabilizzazione degli investimenti (e dell'occupazione) in caso di 'shock asimmetrici'.il 'nuovo meccanismo volto a proteggere il bilancio dell'ue dai rischi finanziari connessi a carenze generalizzate per quanto riguarda lo stato di diritto negli stati membri', secondo la definizione quasi imbarazzante della nota della commissione, è in realtà quello che è rimasto del tentativo di aumentare, con la minaccia di chiudere il rubinetto dei finanziamenti comunitari, la pressione sui regimi illiberali e autoritari già al potere in alcuni paesi dell'est (soprattutto polonia e ungheria) e che potrebbero affermarsi anche in altri stati membri con l'onda lunga populista e nazionalista.la proposta prevede che, in particolare quando vi siano segnalazioni da parte della corte dei conti ue, dell'ufficio anti frode (olaf), di organizzazioni internazionali (per esempio il consiglio d'europa) o sentenze della corte europea di giustizia, la commissione possa proporre di sospendere, ridurre o restringere l'accesso ai finanziamenti comunitari in modo proporzionale alla natura, alla gravità e alla portata delle carenze relative allo stato di diritto. la proposta dalla commissione verrebbe poi sottoposta all'approvazione dal consiglio, ma con votazione 'a maggioranza qualificata inversa'. il che significa che potrebbe essere bocciata solo se vi si opponesse la maggioranza qualificata degli stati.e' chiaro che, per ottenere l'approvazione unanime necessaria in consiglio ue per il nuovo quadro finanziario pluriennale, la commissione non poteva proporre un meccanismo sanzionatorio diretto per chi viola lo stato di diritto, che sembrerebbe fatto su misura contro l'ungheria e la polonia. ecco allora quest'idea di guardare non al principio in sé (rispetto delle regole democratiche), ma piuttosto alle 'conseguenze finanziarie' che le 'carenze generalizzate' in questo campo potrebbero avere. resta da dimostrare che, così 'dissimulato', il meccanismo sarà comunque approvato e avrà l'efficacia sperata.quanto ai nuovi strumenti di bilancio, il primo prevede un programma di sostegno finanziario e tecnico alle riforme strutturali (quelle previste dal 'semestre europeo' dell'eurozona), con una dotazione complessiva di 25 miliardi di euro. questo dispositivo potrà essere usato anche come 'meccanismo di convergenza' per gli stati membri non ancora appartenenti alla zona euro che si preparano ad entrarvi.il secondo strumento sarebbe finalizzato a una funzione di stabilizzazione degli investimenti, per contribuire a mantenerne il livello in caso di gravi shock economici 'asimmetrici' in uno o più stati membri. inizialmente opererà attraverso prestiti 'back-to-back' garantiti dal bilancio dell'ue con un massimale di 30 miliardi di euro, cui si abbinerà un'assistenza finanziaria ai paesi interessati a copertura dell'onere degli interessi. i prestiti forniranno un sostegno finanziario aggiuntivo nei momenti in cui le finanze pubbliche sono sotto pressione.altra novità importante è quella che riguarda il sistema di finanziamento, con la proposta di diversificare le fonti di entrate del bilancio ue e introdurre nuove 'risorse proprie'. la commissione propone innanzitutto di semplificare l'attuale risorsa propria basata sull'imposta sul valore aggiunto (iva), e di ridurre dal 20\% al 10\% gli importi che gli stati membri trattengono all'atto della riscossione dei tributi doganali e versano al bilancio ue.inoltre, la proposta prevede tre nuove risorse proprie, che dovrebbero finanziare il 12\% circa del bilancio totale dell'ue e potrebbero apportare fino a 22 miliardi di euro all'anno: il 20\% delle entrate provenienti dal sistema di scambio delle quote di emissioni; un'aliquota di prelievo del 3\% applicata alla nuova base imponibile consolidata comune per l'imposta sulle società (che verrà introdotta gradualmente, una volta adottata la legislazione necessaria); un contributo nazionale calcolato in base alla quantità di rifiuti non riciclati di imballaggi in plastica di ciascuno stato membro (0,80 ç al chilogrammo).infine, la brexit non pone solo il problema della perdita della contribuzione del regno unito al bilancio ue. con londra fuori dall'ue finisce il 'rebate' britannico (la 'correzione', o restituzione al regno unito di una parte della sua contribuzione annuale), e bisognerà cancellare anche i meccanismi compensativi ('correzioni sulle correzioni') che col tempo erano stati introdotti a favore di alcuni paesi 'contributori netti' (germania, olanda, danimarca, austria e svezia). si tratta di una semplificazione, ma che sarà piuttosto complicata da attuare, perché comporterà, in concreto, che quei paesi dovranno pagare di più alle case comunitarie.e se la francia e la germania si sono già dichiarate disponibili, in principio, ad aumentare leggermente la propria contribuzione, ci sono paesi, come l'austria, l'olanda, la danimarca che non intendono dare un euro in più. neanche per cancellare il loro 'rebate' ora che, senza il regno unito, non ha più ragion d'essere.per convincere i paesi recalcitranti, la commissione propone di eliminare tutte le 'correzioni' non di colpo, come sarebbe logico con l'uscita di londra dall'ue, ma progressivamente nell'arco di cinque anni.sulla base delle proposte di oggi, la commissione presenterà nelle prossime settimane ulteriori dettagli sui futuri programmi di spesa settoriali. la decisione sul futuro bilancio pluriennale spetterà poi al consiglio, che delibererà all'unanimità, dopo l'approvazione del parlamento europeo. perché non si ripeta quanto è successo l'ultima volta, nel 2013, con il ritardo nell'avvio dei principali programmi di spesa e il rinvio di progetti in grado di stimolare la ripresa economica; la commissione spinge perché ai negoziati sul nuovo bilancio sia accordata la massima priorità, in modo da giungere a un accordo prima delle elezioni europee di maggio 2019. & 1465 & low & Low & Power & Governance & NA & 2018-05-02 & 2018 & 3 & POL
Frame & low-medium & National & +1000 & -0.5155635 & -0.1667445 & -0.9728619 & 0.5015415 & 1.3050799 & 3.2 & 0.6840199 & -1.2389660 & Payer & European & European & European & European|POL & Neutral\\
Italy & https://www.ecodibergamo.it/stories/ansa/al-via-a-concorso-per-aspiranti-reporter-su-uso-fondi-ue\_1275620\_11/ & 400 & L'Eco di Bergamo & Private/Non-Public & Online and Offline & Regional/Local & very high = CP is most important issue + CP is mentioned in title/headline & Cultural development & Positive & EU & No myth & NA & NA & NA & NA & NA & NA & NA & NA & Italy & al via a concorso per aspiranti reporter su uso fondi ue & 2018-04-12 & fondo di coesione & bruxelles - al via l'edizione 2018 del concorso ue per aspiranti giornalisti sui temi delle politiche regionali e dei fondi europei. sono infatti ora ufficialmente aperte le candidature per il programma 'youth4regions', dove in palio per i vincitori c'è un viaggio a bruxelles e la possibilità di cimentarsi in prima persona con i temi dell'unione europea, raccontando ai lettori la 'settimana europea delle regioni e delle città 2018' prevista dall'8 all'11 ottobre. per partecipare, gli studenti di giornalismo possono inviare alla giuria del concorso i loro migliori articoli (fra 400 e 1000 parole) o video (massimo 3 minuti) che riguardano un progetto finanziato dai fondi strutturali ue (fesr o fondo di coesione). lo si può fare in tutte le 24 lingue ufficiali dell'ue, entro la mezzanotte del 29 giugno. i 28 vincitori (uno per stato membro) saranno invitati dalla commissione europea a entrare a far parte del 'team comunicazione' che coprirà la settimana delle regioni, il più grande evento politico dell'unione dedicato alla politica di coesione che ogni anno riunisce a bruxelles centinaia di personalità politiche provenienti da tutta europa. & 186 & very high & High & Socio-Economic & NA & NA & 2018-04-12 & 2018 & 3 & ECO
Frame & high-very high & Regional & <500 & -0.5155635 & -0.1667445 & -0.9728619 & 0.5015415 & 1.3050799 & 3.2 & 0.6840199 & -1.2389660 & Payer & European & European & European & European|ECO & Positive\\
Italy & http://www.ansa.it/europa/notizie/rubriche/altrenews/2018/09/06/pa-guida-europea-allappalto-perfetto-usando-fondi-ue\_a7f49ce6-e6a3-437f-9412-073b2a075340.html & 430 & ANSA.it & Private/Non-Public & Online only & National & very high = CP is most important issue + CP is mentioned in title/headline & Improve governance & Positive & EU & No myth & NA & NA & NA & NA & NA & NA & NA & NA & Italy & pa: guida europea all'appalto 'perfetto' usando fondi ue - altre news - ansa europa & 2018-09-06 & fondi strutturali e di investimento europei & (ansa) - bruxelles, 6 set - una guida pratica in italiano per aiutare i funzionari pubblici a orientarsi nel labirinto di burocrazia che spesso circonda le procedure di aggiudicazione dei fondi europei. è il nuovo strumento pubblicato dalla commissione ue per venire incontro alle esigenze di chi si occupa quotidianamente di finanziamenti europei e deve assicurare procedure di aggiudicazione degli appalti pubblici efficienti e trasparenti. gli orientamenti coprono il processo dalla a alla z, dalla preparazione e pubblicazione degli inviti alla selezione e valutazione delle offerte, fino all'attuazione dei contratti. "aiutare gli stati membri a organizzare gare d'appalto affidabili per gli investimenti dell'ue è fondamentale per salvaguardare il bilancio comunitario da errori e garantire il massimo impatto di ogni euro speso dall'unione, a diretto vantaggio dei cittadini", ha dichiarato la commissaria ue alla politica regionale, corina cretu. i fondi strutturali e d'investimento europei convogliano nell'economia reale dell'ue oltre 450 miliardi di euro nel periodo 2014-2020, metà dei quali vengono investiti attraverso appalti pubblici. (ansa). & 170 & very high & High & Governance & NA & NA & 2018-09-06 & 2018 & 3 & POL
Frame & high-very high & National & <500 & -0.5155635 & -0.1667445 & -0.9728619 & 0.5015415 & 1.3050799 & 3.2 & 0.6840199 & -1.2389660 & Payer & European & European & European & European|POL & Positive\\
Italy & http://www.ansa.it/valledaosta/notizie/2016/03/02/rollandinnon-a-europa-serie-a-e-serie-b\_b16caff8-66ba-4ae9-9cfa-5c90260ef50a.html & 426 & ANSA.it & Private/Non-Public & Online only & National & medium = CP is important part of story & Solidarity to poor countries/regions & Negative & EU + Subnational & No myth & NA & NA & NA & NA & NA & NA & NA & NA & Italy & rollandin, no a europa serie a e serie b - valle d'aosta & 2016-03-02 & politica di coesione & (ansa) - aosta, 2 mar - "nella definizione delle politiche, nell'assegnazione di finanziamenti a favore dell'ammodernamento delle infrastrutture di trasporto, non si deve commettere l'errore di discriminare tra un'europa urbana di serie a e una rurale, isolata e periferica di serie b". lo ha detto il presidente della regione valle d'aosta, augusto rollandin, intervenendo a bruxelles alla riunione della commissione politica di coesione territoriale del comitato delle regioni. "non possiamo concentrare i nostri sforzi in via esclusiva per la creazione di snodi centrali futuristici che rischierebbero di diventare delle oasi nel deserto", ha aggiunto. rollandin ha poi concluso: "ai cittadini delle aree rurali e svantaggiate - ha detto - devono essere garantiti dei servizi di trasporto adeguati e sostenibili, che favoriscano la lotta allo spopolamento, sostengano l'economia locale e creino un'europa realmente più inclusiva. è questa una via maestra per la costruzione e il rafforzamento di un vero mercato unico europeo". & 154 & medium & Medium & Values & NA & NA & 2016-03-02 & 2016 & 2 & ECO
Frame & low-medium & National & <500 & -0.5155635 & -0.1667445 & -0.9728619 & 0.5015415 & 1.3050799 & 3.2 & 0.6840199 & -1.2389660 & Payer & Domestic & European & Mixed & Domestic|ECO & Negative\\
\addlinespace
Italy & http://www.ansa.it/europa/notizie/rubriche/europa\_delle\_regioni/2016/09/29/trasporti-cattaneo-ue-investa-anche-su-reti-regionali\_fea40762-7a10-4e23-811d-531eb41811e4.html & 485 & ANSA.it & Private/Non-Public & Online only & National & low = CP mentioned more times but NOT important part of story (mainly about others issues) & Infrastructure & Balanced & EU & No myth & NA & NA & NA & NA & NA & NA & NA & NA & Italy & trasporti: cattaneo, ue investa anche su reti regionali - europa delle regioni & 2016-09-29 & politica di coesione & (ansa) - bruxelles, 29 set - potenziare gli investimenti nei collegamenti mancanti tra le regioni frontaliere dell'ue, ma evitare di concentrarsi solamente sulle grandi reti di trasporto ten-t. è quanto emerso da una riunione congiunta fra la commissione politica di coesione territoriale e bilancio dell'ue (coter) del comitato europeo delle regioni (cdr) e la commissione per i trasporti (tran) del parlamento europeo. "c'è un grande bisogno di investimenti nel settore dei trasporti per la realizzazione di progetti regionali e locali", ha dichiarato il presidente del consiglio regionale della lombardia e della coter, raffaele cattaneo, sottolineando però che l'impegno non deve essere focalizzato anche sulle reti locali. "gli enti locali e regionali - ha detto - sono nella maggior parte dei casi responsabili diretti delle infrastrutture che vengono realizzate in un determinato territorio e per questo motivo queste istituzioni devono essere coinvolte in tutte le fasi di ideazione e realizzazione di progetti di trasporto". i leader locali e i deputati europei hanno esaminato anche un elenco di 15 principali progetti transfrontalieri individuati da uno studio del gruppo dei verdi al parlamento europeo tra oltre 250 collegamenti transfrontalieri. tra questi collegamenti vi sono, ad esempio, il corridoio praga-norimberga e 15 km mancanti di binari elettrificati al confine tedesco-polacco. "il fatto che manchino queste infrastrutture alle frontiere non è soltanto fonte di imbarazzo, ma è anche economicamente dannoso, in quanto rappresentano collegamenti necessari per i pendolari e anche per i trasporti a lunga distanza", ha osservato cattaneo.(ansa). & 248 & low & Low & Socio-Economic & NA & NA & 2016-09-29 & 2016 & 2 & ECO
Frame & low-medium & National & <500 & -0.5155635 & -0.1667445 & -0.9728619 & 0.5015415 & 1.3050799 & 3.2 & 0.6840199 & -1.2389660 & Payer & European & European & European & European|ECO & Neutral\\
Italy & https://www.ilmessaggero.it/mondo/juncker\_ministri\_italiani\_bugiardi\_conte-4403472.html & 473 & Il Messaggero & Private/Non-Public & Online and Offline & National & very low = CP mentioned once & Political capital/interests & Negative & EU & No myth & NA & NA & NA & NA & NA & NA & NA & NA & Italy & juncker: "alcuni ministri italiani sono dei bugiardi. gli italiani lo sanno?" & 2019-04-02 & fondi strutturali & "alcuni ministri italiani sono bugiardi quando non rivelare gli importi stanziati per l'italia da parte dell'unione europea". lo ha detto il presidente della commissione europea, jean-claude juncker, parlando in un'intervista a euranetplus, riportata sul profilo twitter della radio. "all'italia abbiamo dato 130 miliari. il piano-juncker ha generato investimenti dell'ordine di 63,3 miliardi. i fondi strutturali, sostegno europeo per rinvigorire l'economia, sono più di 44 miliardi", ha aggiunto juncker durante l'intervista. e ancora: "c'è un solo italiano che lo sa? visto che alcuni ministri italiani dicono il contrario, sono dei bugiardi". & 101 & very low & Low & Power & NA & NA & 2019-04-02 & 2019 & 3 & POL
Frame & v.low & National & <500 & -0.5155635 & -0.1667445 & -0.9728619 & 0.5015415 & 1.3050799 & 3.2 & 0.6840199 & -1.2389660 & Payer & European & European & European & European|POL & Negative\\
Italy & http://www.ansa.it/europa/notizie/rubriche/altrenews/2018/09/21/paese-puo-essere-isolato-ma-non-cacciato-da-schengen\_27e0aa81-e63b-4596-af56-8eb488e4be91.html & 487 & ANSA.it & Private/Non-Public & Online only & National & very low = CP mentioned once & Political leverage & Balanced & EU + Other country & No myth & NA & NA & NA & NA & NA & NA & NA & NA & Italy & paese può essere 'isolato' ma non cacciato da schengen - altre news - ansa europa & 2018-09-21 & fondi strutturali & (ansa) - bruxelles, 21 set - "nessuno stato viene cacciato da schengen. in caso di una seria carenza alle frontiere esterne, sulla base dell'articolo 29 del codice schengen", esiste la possibilità, per gli altri paesi confinanti di introdurre i controlli ai loro confini. così natasha bertaud, portavoce della commissione europea per la migrazione, a chi chiede se uno stato possa essere cacciato da schengen. la recente proposta sul rafforzamento di frontex, nel caso che il paese non rispetti le raccomandazioni dell'agenzia ue, o si rifiuti di cooperare, prevede anche la possibilità di attivare l'art 29. ieri il presidente francese emmanuel macron aveva detto: "i paesi che non vogliono più frontex e più solidarietà usciranno da schengen, quelli che non vogliono più europa non avranno i fondi strutturali". (ansa). & 129 & very low & Low & Power & NA & NA & 2018-09-21 & 2018 & 3 & POL
Frame & v.low & National & <500 & -0.5155635 & -0.1667445 & -0.9728619 & 0.5015415 & 1.3050799 & 3.2 & 0.6840199 & -1.2389660 & Payer & European & European & European & European|POL & Neutral\\
Italy & http://www.corriere.it/notizie-ultima-ora/Economia/Legge-stabilita-mld-banda-larga-trasporti-CorSera/02-10-2015/1-A\_020154478.shtml & 491 & Corriere della Sera & Private/Non-Public & Online and Offline & National & very low = CP mentioned once & Infrastructure & Positive & National & No myth & NA & NA & NA & NA & NA & NA & NA & NA & Italy & legge stabilita': 5 mld per banda larga e trasporti (corsera) & 2015-10-02 & fondi strutturali & 09:45 roma (mf-dj)--investimenti da 5 miliardi nelle infrastrutture e nella banda larga. e' questo, secondo il corriere della sera, il piano straordinario di investimenti che il governo punta a finanziare in deficit ricorrendo alle clausole europee di flessibilita'. a questi 5 miliardi di investimenti aggiuntivi che verrebbero scomputati dalla spesa e dal deficit pubblico, prosegue il giornale, se ne aggiungerebbero altri 5 provenienti dal bilancio ue. il piano per lo sviluppo della rete telematica in banda larga, quello per contrastare il dissesto idrogeologico, l'edilizia scolastica, le strade, le ferrovie: sono alcuni dei progetti destinati a rientrare nel piano. per essere sterilizzata dal deficit, la spesa deve riguardare opere previste dai progetti europee delle reti transnazionali, dal nuovo piano juncker, o che possono beneficiare del finanziamento dei fondi strutturali ue, e siano cantierabili nello stesso 2016. red/alu (fine) mf-dj news 0209:45 ott 2015 & 149 & very low & Low & Socio-Economic & NA & NA & 2015-10-02 & 2015 & 1 & ECO
Frame & v.low & National & <500 & -0.5155635 & -0.1667445 & -0.9728619 & 0.5015415 & 1.3050799 & 3.2 & 0.6840199 & -1.2389660 & Payer & Domestic & Domestic & Domestic & Domestic|ECO & Positive\\
Italy & https://www.agi.it/cronaca/stato\_emergenza\_regioni\_maltempo-4595424/news/2018-11-08/ & 490 & AGI & Private/Non-Public & Online only & National & very low = CP mentioned once & Public services & Positive & National + Subnational & No myth & Infrastructure & Positive & National + Subnational & No myth & NA & NA & NA & NA & Italy & deciso lo stato d'emergenza per le regioni colpite dal maltempo & 2018-11-08 & fondi strutturali & il consiglio dei ministri ha deliberato lo stato di emergenza per le regioni colpite dal maltempo e ha stanziato "le primissime risorse" per far fronte ai danni pari a 53,5 milioni di euro. ulteriori 200 milioni arriveranno con un successivo decreto della presidenza del consiglio. lo ha detto barbara lezzi, ministro per il sud, lasciando palazzo chigi al termine del consiglio dei ministri. a questi fondi, ha spiegato il ministro, si aggiungeranno le risorse "dei fondi strutturali ue che sono in capo alle regioni", per un importo compreso fra i 3 e i 4 miliardi. nello specifico, all'ordine del giorno del cdm c'era la dichiarazione dello stato di emergenza nei territori delle regioni calabria, emilia-romagna, friuli-venezia giulia, lazio, liguria, lombardia, sardegna, sicilia, toscana, veneto e delle province autonome di trento e bolzano "colpiti dagli eccezionali eventi meteorologici che si sono verificati a partire dal 2 ottobre 2018". & 152 & very low & Low & Socio-Economic & Socio-Economic & NA & 2018-11-08 & 2018 & 3 & ECO
Frame & v.low & National & <500 & -0.5155635 & -0.1667445 & -0.9728619 & 0.5015415 & 1.3050799 & 3.2 & 0.6840199 & -1.2389660 & Payer & Domestic & Domestic & Domestic & Domestic|ECO & Positive\\
\addlinespace
Italy & http://corrierealpi.gelocal.it/belluno/cronaca/2017/01/22/news/obiettivo-efficienza-per-tre-scuole-1.14758893 & 482 & Corriere delle Alpi & Private/Non-Public & Online and Offline & Regional/Local & low = CP mentioned more times but NOT important part of story (mainly about others issues) & Infrastructure & Balanced & Subnational & No myth & NA & NA & NA & NA & NA & NA & NA & NA & Italy & obiettivo efficienza per tre scuole  - cronaca - corriere delle alpi & 2017-01-23 & fondi strutturali & gli ispettori della certificazione emas promuovono la città. e bonan invoca più punti nei bandi per i fondi europei feltre. le scuole di pasquer, boscariz e vignui balzano in cima alla lista delle priorità di miglioramento energetico del comune, che ha realizzato le diagnosi energetiche dei tre istituti durante la ricognizione ambientale per il rinnovo della certificazione emas. ora si tratta di trovare i soldi per isolarle, cambiare gli infissi, rinnovare gli impianti, insomma trasformarle in tre scuole efficienti dal punto di vista dell'energia. e feltre proprio per la sua attenzione all'efficienza energetica - è l'appello dell'assessore valter bonan - meriterebbe un premio sotto forma di punteggi in più per chiedere fondi all'europa. quello dell'emas è un modello che tiene conto dei parametri sulla riduzione dei consumi in funzione degli investimenti legati al miglioramento dell'efficienza. per rinnovare la certificazione sono serviti due giorni di approfondita verifica e la città è stata promossa a pieni voti e senza riserve dagli ispettori, nella valutazione sul miglioramento previsto nell'ambito della gestione ambientale sotto ogni aspetto. raggiunto il 90,8 per cento nell'utilizzo di fonti rinnovabili nel consumo di energia, avendo posto questo criterio nei contratti di fornitura dell'ente. quasi azzerato l'utilizzo delle caldaie a gasolio, sostituite con quelle a metano. c'è stata poi una riduzione di 164 tonnellate emesse di anidride carbonica grazie agli interventi di qualificazione energetica nelle scuole e in un altro paio di edifici di servizio. proseguendo su questa strada, sono stati analizzati i consumi consolidati e lo stato degli impianti elettrici e termici delle scuole del pasquer, boscariz e vignui, da cui sono emerse le scelte di efficientamento, che presentano ampi margini soprattutto in termini di coibentazione degli edifici. a snocciolare i dati e fare il punto della situazione è stato ieri in municipio l'assessore all'ambiente valter bonan, che parla di "investimento intelligente non solo sul piano dei risultati ambientali, ma anche su quello dei tempi di ammortamento dei costi per il risparmio economico gestionale. intendiamo individuare delle risorse a sostegno degli interventi". se per il pasquer e boscariz si tratta di strutture con un ampio bacino d'utenza, nel caso di vignui c'è l'impegno con la frazione di difendere un istituto scolastico più decentrato. continua insomma l'impegno dell'amministrazione sul fronte delle scuole, basti pensare ai lavori in corso sull'edificio che ospita le elementari vittorino da feltre e che non è ancora concluso, con la speranza di ottenere il contributo di 440 mila euro sul bando della regione per il prossimo stralcio. altro versante di impegno dell'amministrazione, quello dell'illuminazione pubblica: "c'è molto da fare sia nel contesto urbano che frazionale", prosegue l'assessore valter bonan. a breve prenderà il via il rifacimento dell'impianto di illuminazione del centro storico da 325 mila euro con l'evoluzione costituita dall'impiego di lampade a led, ma "prima o poi dovrebbe uscire anche un bando sui fondi strutturali europei coordinato dalla regione sull'illuminazione pubblica e ci candideremo avendo già individuato le priorità degli interventi nel picil (piano comunale di illuminazione pubblica)". e proprio a proposito di bandi, l'assessore feltrino invoca un riconoscimento concreto della certificazione ambientale da parte della regione nell'assegnazione di punteggi in più in termini di valorizzazione per l'accesso ai fondi strutturali europei. raffaele scottini ©riproduzione riservata & 561 & low & Low & Socio-Economic & NA & NA & 2017-01-23 & 2017 & 2 & ECO
Frame & low-medium & Regional & 500-1000 & -0.5155635 & -0.1667445 & -0.9728619 & 0.5015415 & 1.3050799 & 3.2 & 0.6840199 & -1.2389660 & Payer & Domestic & Domestic & Domestic & Domestic|ECO & Neutral\\
Italy & http://www.pmi.it/economia/lavoro/news/93880/formazione-occupazione-bandi-in-emilia-romagna.html?utm\_source=feedburner\&utm\_medium=feed\&utm\_campaign=Feed\%253A\%2BPMI.it\%2B-\%2BNews & 398 & PMI.it & Private/Non-Public & Online only & National & high = CP is most important issue in story (can also cover other issues) & Jobs & Positive & Subnational & No myth & NA & NA & NA & NA & NA & NA & NA & NA & Italy & formazione e occupazione: bandi in emilia romagna & 2015-03-02 & fondo sociale europeo & dal fondo sociale europeo (fse) arrivano risorse economiche fino a 40 milioni di euro per l'avvio di iniziative e bandi che hanno come fine quello di contrastare la disoccupazione sul territorio della regione emilia romagna, fondi destinati a incentivare l'occupazione e interventi a favore delle persone svantaggiate. attraverso le risorse del fse, infatti, sono stati approvati due avvisi pubblici destinati agli enti di formazione professionale in partnership con le imprese locali: si parla, rispettivamente, dell'avvio di politiche attive rivolte a inoccupati e disoccupati (nello specifico percorsi formativi per conseguire una qualifica professionale di operatori o di tecnici), e di piani di intervento territoriali da realizzare con soggetti, pubblici e privati per incentivare l'inserimento lavorativo dei soggetti in carico ai servizi sociali appartenenti a categorie a rischio di esclusione e discriminazione. per quanto riguarda il bando per l'occupazione di persone inattive, la scadenza è fissata per il 23 aprile 2015. l'avviso per l'inclusione lavorativa delle persone svantaggiate scade il 2 aprile 2015. & 168 & high & High & Socio-Economic & NA & NA & 2015-03-02 & 2015 & 1 & ECO
Frame & high-very high & National & <500 & -0.5155635 & -0.1667445 & -0.9728619 & 0.5015415 & 1.3050799 & 3.2 & 0.6840199 & -1.2389660 & Payer & Domestic & Domestic & Domestic & Domestic|ECO & Positive\\
Italy & https://www.lastampa.it/2019/02/22/cronaca/parco-della-salute-milioni-per-lo-sviluppo-e-la-ricerca-inKNYClObLUeGJGlTHXSDO/pagina.html & 417 & LaStampa.it & Private/Non-Public & Online and Offline & National & very low = CP mentioned once & Infrastructure & Positive & Subnational & No myth & NA & NA & NA & NA & NA & NA & NA & NA & Italy & parco della salute, 138 milioni per lo sviluppo e la ricerca & 2019-02-22 & fondo europeo di sviluppo regionale & parco della salute, un altro passo avanti: questa volta sul fronte dello sviluppo e della ricerca, uno degli "asset" del nuovo polo ospedaliero. la giunta regionale ha approvato il programma attuativo di ricerca, sviluppo e innovazione: mobilita risorse pubbliche e private per 138 milioni, di cui 90 dal fondo sviluppo e coesione 2014-2020, 28,475 a carico dei privati partecipanti ai progetti di ricerca, 20 dal fondo europeo di sviluppo regionale 2014-2020. leggi anche: parco salute, pubblicato il bando: il 26 aprile la scadenza delle domande il finanziamento il finanziamento è suddiviso in due filoni. il primo è l'ampliamento del centro di biotecnologie molecolari-incubatore di ricerca dell'università di torino, da realizzarsi nell'area ex scalo vallino: svolgerà un ruolo di catalizzatore della ricerca nel campo della genetica, delle tecnologie applicate alla medicina, dell'ingegneria bio-medica e della bio-ingegneria (il costo totale è stimato in 30 milioni, aggiudicazione a novembre 2019 e completamento ad aprile 2022). il secondo filone - 106,7 milioni, aggiudicazione nel 2019/2020 e completamento 2025 - rimanda a progetti di ricerca e sviluppo e infrastrutture di ricerca. "sono già attive diverse iniziative regionali sui temi strategici che verranno sviluppati nel futuro parco - spiega l'assessore giuseppina de santis -: una fra tutte è la piattaforma tecnologica salute e benessere, con il finanziamento di progetti che già attuano sinergie concrete tra università, ricerca pubblica e impresa. il nostro ruolo è di continuare nell'azione che abbiamo intrapreso in questi anni per attrarre investitori ed essere competitivi nel contesto internazionale. una sfida ambiziosa che vale 4 mila posti di lavoro". soddisfatto sergio chiamparino: "questo ultimo provvedimento conferma l'impegno dell'amministrazione nella realizzazione di una struttura che rappresenterà l'investimento più innovativo in città e in piemonte per molti anni a venire". & 298 & very low & Low & Socio-Economic & NA & NA & 2019-02-22 & 2019 & 3 & ECO
Frame & v.low & National & <500 & -0.5155635 & -0.1667445 & -0.9728619 & 0.5015415 & 1.3050799 & 3.2 & 0.6840199 & -1.2389660 & Payer & Domestic & Domestic & Domestic & Domestic|ECO & Positive\\
Italy & http://www.ansa.it/europa/notizie/rubriche/politica/2018/03/21/migrantitajani-chi-riceve-fondi-ue-mostri-solidarieta\_accfce28-4961-4956-8b6b-4d26cea5e354.html & 460 & ANSA.it & Private/Non-Public & Online only & National & high = CP is most important issue in story (can also cover other issues) & Political leverage & Negative & EU + National & 6.Does not defend EU values (eg.gender/law/democracy) & Solidarity to poor countries/regions & Balanced & EU + National & No myth & NA & NA & NA & NA & Italy & migranti:tajani, chi riceve fondi ue mostri solidarietà - politica estera, sicurezza e difesa - ansa europa & 2018-03-21 & politica di coesione & (ansa) - bruxelles, 21 mar - "la politica di coesione è stata fondamentale per l'europa in passato, lo è oggi e lo sarà in futuro", "è il simbolo della solidarietà" all'interno dell'unione. tuttavia, questa solidarietà "non deve solo essere usata, deve manifestarsi anche nell'aiutare altri paesi quando ne hanno bisogno, come italia, germania e grecia sulla questione dei migranti". così il presidente del parlamento europeo, antonio tajani, intervenendo durante una trasmissione di euronews dedicata ai 30 anni della politica di coesione. "noi italiani paghiamo molti soldi per i fondi di coesione in altri paesi, e ora abbiamo bisogno di solidarietà sulla questione dei rifugiati. è inaccettabile usare soldi provenienti da italia, grecia, germania, ed essere contro la solidarietà sui rifugiati", ha insistito tajani. "la solidarietà dovrebbe essere mostrata nei due sensi", "è molto importante ricordare che se alcuni paesi vogliono più soldi per la politica di coesione per costruire infrastrutture, questi devono anche mostrare solidarietà aiutando gli altri paesi" ha fatto eco la commissaria ue alla politica regionale, corina cretu. (ansa). & 173 & high & High & Power & Values & NA & 2018-03-21 & 2018 & 3 & POL
Frame & high-very high & National & <500 & -0.5155635 & -0.1667445 & -0.9728619 & 0.5015415 & 1.3050799 & 3.2 & 0.6840199 & -1.2389660 & Payer & Domestic & European & Mixed & Domestic|POL & Negative\\
Italy & http://www.ecodibergamo.it/stories/ansa/fondi-ue-italia-ultima-per-spesa-e-sotto-media-28-su-impegni\_1269367\_11/ & 492 & L'Eco di Bergamo & Private/Non-Public & Online and Offline & Regional/Local & very high = CP is most important issue + CP is mentioned in title/headline & Bureaucracy and/or delays & Negative & National + Other country & No myth & NA & NA & NA & NA & NA & NA & NA & NA & Italy & fondi ue: italia ultima per spesa e sotto media 28 su impegni & 2018-02-08 & fondo europeo di sviluppo regionale & bruxelles - maglia nera per la spesa dei fondi ue, la più bassa dei 28, e decisamente sotto la media europea per la loro assegnazione, anche se questa è leggermente migliorata. e' l'ultima fotografia della performance dell'italia, scattata dagli ultimi dati resi disponibili online dalla commissione ue. il nostro paese chiude infatti il 2017 con il 42\% dei fondi strutturali impegnati in progetti già selezionati salendo così, rispetto al 37\% di fine ottobre, dal 23esimo al 22esimo posto. la media dei 28 stati membri è, a confronto, più alta di ben 10 punti (52,53\%). l'italia ha quindi assegnato 31,876 miliardi di euro su un totale di 75,065 miliardi disponibili di fondi (cofinanziamento nazionale compreso), ma di questi solo 6,22 miliardi sono stati realmente spesi sul territorio. il tasso di spesa italiano è infatti fermo all'8\%, il più basso dell'unione insieme a quello di malta, croazia e spagna, a fronte di una media ue di oltre il doppio, con il 16,32\%. prima della classe nell'ue per impegno dei fondi strutturali è l'ungheria del premier euroscettico viktor orban (94\%, e con il 13\% di spesa), che negli ultimi due mesi del 2017 è riuscita a surclassare l'irlanda, rimasta ferma al (79\%, ma con il 25\% di spesa). rispetto a fine ottobre, la polonia, maggiore beneficiario dei fondi ue (105 miliardi in 7 anni), è salita dal 40\% al 55\% nell'assegnazione delle risorse, con un tasso di spesa a fine 2017 del 13\%. per quanto riguarda l'italia, nonostante la performance relativamente positiva con il fondo europeo di sviluppo regionale (51\% assegnazioni e 5\% di spesa su 34 miliardi), percentuali ben peggiori si registrano con il fondo agricolo per lo sviluppo rurale (feasr) e il fondo sociale ue (fse). la penisola è primo beneficiario in ue per entrambi i fondi, ma risulta 22esima su 28 nelle assegnazioni del feasr (34\% su 21 miliardi) e 25esima (33\% su 17,7 miliardi) per il fse. & 333 & very high & High & Governance & NA & NA & 2018-02-08 & 2018 & 3 & POL
Frame & high-very high & Regional & <500 & -0.5155635 & -0.1667445 & -0.9728619 & 0.5015415 & 1.3050799 & 3.2 & 0.6840199 & -1.2389660 & Payer & Domestic & European & Mixed & Domestic|POL & Negative\\
\addlinespace
Italy & http://www.ansa.it/europa/notizie/rubriche/altrenews/2017/07/18/industria-ue-aiuta-regioni-nella-transizione-tecnologica\_4930b617-c89b-402d-bd4f-432fa57d29f8.html & 442 & ANSA.it & Private/Non-Public & Online only & National & high = CP is most important issue in story (can also cover other issues) & Research \& innovation & Positive & EU + National + Subnational & No myth & Infrastructure & Negative & EU & No myth & NA & NA & NA & NA & Italy & industria: ue aiuta regioni nella transizione tecnologica - altre news - ansa europa & 2017-07-18 & politica di coesione & bruxelles - dare un sostegno alle regioni nell'accompagnare la transizione tecnologica delle loro industrie e puntare su nuovi settori some i big data e la bioeconomia per potenziare i partenariati fra regioni, anche di stati diversi. sono i due progetti pilota lanciati oggi dalla commissione ue per aiutare i territori europei a investire nei settori in cui sono competitivamente forti, la cosiddetta "specializzazione intelligente", cercando così di accompagnare la globalizzazione. in parallelo, bruxelles ha svelato una serie di iniziative per snellire la burocrazia e aiutare i territori nel creare ambienti favorevoli al business. "stati e regioni si aspettano di avere fondi ue per le loro infrastrutture", ma è un "modo di pensare tradizionale" che "non prende in considerazione altri elementi fondamentali per la crescita", ha dichiarato il vicepresidente con delega alla crescita, jyrki katainen. "la specializzazione intelligente cambia il modo in cui la commissione, e speriamo anche le regioni, vedono la politica di coesione", anche in vista dei probabili tagli al bilancio ue per il post 2020, ha aggiunto. secondo la commissaria alla politica regionale, corina cretu, un esempio positivo di utilizzo di fondi ue per investimenti mirati sono le marche, che hanno deciso di puntare sull'innovazione nel calzaturiero. "apprezziamo molto lo sforzo della commissione di concentrarsi sugli investimenti che dovrebbero fare le regioni per avere vantaggi competitivi", ha dichiarato arnaldo abruzzini, presidente di eurochambres. & 226 & high & High & Socio-Economic & Socio-Economic & NA & 2017-07-18 & 2017 & 2 & ECO
Frame & high-very high & National & <500 & -0.5155635 & -0.1667445 & -0.9728619 & 0.5015415 & 1.3050799 & 3.2 & 0.6840199 & -1.2389660 & Payer & Domestic & European & Mixed & Domestic|ECO & Positive\\
Italy & http://ilpiccolo.gelocal.it/trieste/cronaca/2018/07/29/news/dalle-casse-di-muggia-esce-un-chip-riservato-alle-famiglie-bisognose-1.17106414 & 440 & Il Piccolo & Private/Non-Public & Online and Offline & Regional/Local & low = CP mentioned more times but NOT important part of story (mainly about others issues) & Public services & Positive & Subnational & No myth & NA & NA & NA & NA & NA & NA & NA & NA & Italy & dalle casse di muggia esce un chip riservato alle famiglie bisognose  - cronaca - il piccolo & 2018-07-29 & fondo sociale europeo & la giunta marzi destina a interventi di sostegno al welfare i 50 mila euro dell'ultimo "avanzo" d'amministrazione di riccardo tosques muggia l'ultimo avanzo d'amministrazione del comune di muggia, che vale precisamente 50.030,22 euro, sarà destinato alle famiglie in difficoltà. questa la decisione della giunta retta dal sindaco laura marzi. "purtroppo negli ultimi anni la forbice che riguarda le persone in difficoltà è sempre più ampia, e lo è anche a muggia. è una problematica presente in varie forme, e per questo l'amministrazione comunale, a conferma delle scelte politiche portate avanti in questi ultimi anni a livello locale, ritiene prioritari l'impegno e l'attenzione nei confronti del welfare e del sostegno alle fasce più deboli della popolazione", commenta l'assessore alle politiche sociali luca gandini. la cifra totale sarà ripartita in due quote. quella decisamente più rilevante, pari a 42.923,72 euro, andrà agli utenti in carico al servizio sociale, utenti che hanno maturato il diritto a una serie di precise prestazioni sociali. esattamente 39.493,72 euro verranno messi in campo per la presa in carico di soggetti in situazione di disagio economico-sociale attraverso il fondo per l'autonomia possibile. le risorse assegnate nel 2017 dalla direzione centrale salute e politiche per la famiglia della regione per il fondo per l'autonomia possibile e non completamente esaurite verranno utilizzate, in particolare, a parziale copertura del fabbisogno del 2018 dei contributi previsti dal fondo stesso. gli altri 3.430 euro verranno invece destinati come contributo di integrazione al reddito attraverso la cosiddetta misura d'inclusione attiva (mia) che, normata dalla regione, viene erogata nell'ambito di un percorso concordato finalizzato a superare le condizioni di difficoltà del nucleo familiare. l'erogazione è subordinata alla sottoscrizione di un patto fra il cittadino e l'ente locale, patto che prevede una serie di attività rivolte ai componenti del nucleo familiare tra le quali "azioni di ricerca attiva di lavoro, adesione a progetti di formazione o inclusione lavorativa, frequenza e impegno scolastico, comportamenti di prevenzione e cura volti alla tutela della salute ed espletamento di attività utili alla collettività". la seconda quota, pari a 7.106,50 euro, sarà infine destinata all'abbattimento delle rette attese dagli enti gestori dei servizi per la prima infanzia e a carico delle famiglie presenti nelle relative liste d'attesa, attraverso l'erogazione dei contributi del fondo sociale europeo per quanto riguarda l'anno scolastico 2017-18. "questi sono interventi che spesso passano inosservati ai più perché non sono facilmente percepibili come l'asfaltatura di una strada - conclude gandini - ma in questi anni siamo riusciti a far fronte a molte situazioni critiche in questo specifico campo, e molto stiamo ancora facendo". -- & 453 & low & Low & Socio-Economic & NA & NA & 2018-07-29 & 2018 & 3 & ECO
Frame & low-medium & Regional & <500 & -0.5155635 & -0.1667445 & -0.9728619 & 0.5015415 & 1.3050799 & 3.2 & 0.6840199 & -1.2389660 & Payer & Domestic & Domestic & Domestic & Domestic|ECO & Positive\\
Italy & http://www.secoloditalia.it/2015/11/merkel-in-difficolta-sui-profughi-vuol-punire-leuropa-dellest/ & 455 & secoloditalia.it & Private/Non-Public & Online and Offline & National & medium = CP is important part of story & Political leverage & Negative & EU + Other country & No myth & NA & NA & NA & NA & NA & NA & NA & NA & Italy & merkel è in difficoltà sui profughi: adesso vuol punire l’europa dell’est & 2015-11-29 & fondi strutturali & condividi nel discorso al bundestag, giovedì scorso, era sembrata una minaccia. angela merkel aveva detto che la sopravvivenza di schengen dipende dalla corretta applicazione delle quote ue per i rifugiati. "una distribuzione solidale dei profughi", aveva scandito in parlamento, "non è un dettaglio secondario: riguarda la questione se lo spazio schengen potrà sopravvivere". nei giorni successivi, varie fonti vicine alla cancelliera hanno tentato di sminare la pericolosa correlazione, smentendo che si trattasse di una minaccia. ma alla vigilia del vertice di oggi con la turchia, alcuni media tedeschi hanno parlato di un'irritazione talmente alle stelle che merkel starebbe cercando meccanismi finanziari per "punire" i paesi dell'est europa, notoriamente riottosi ad accogliere profughi. merkel vuole togliere il budget ai paesi dell'est per dare tre miliardi alla turchia l'arma per colpire l'ungheria, la polonia e gli altri partner che si sono mostrati particolarmente ingenerosi potrebbero essere i tre miliardi di euro promessi dalla ue alla turchia proprio per governare l'enorme flusso di profughi. secondo indiscrezioni, i tedeschi starebbero insistendo per attingere al budget europeo; la commissione europea insiste per prendere solo 500 milioni da lì e ricavare il resto in base al solito meccanismo delle quote (alla germania spetterebbe di sborsare 540 milioni, al regno unito 420 milioni eccetera). berlino insiste, invece, per prendere tutta la somma dal budget: significherebbe ridurre molti fondi strutturali destinati in buona parte ai paesi dell'est. cdu mette in difficoltà la merkel: tanti paletti per concedere permesso di soggiorno intanto la cancelliera continua a fare i conti con le polemiche interne sui rifugiati, anche se i flussi provenienti dai balcani sono rallentati molto a causa dell'inverno. fa molto discutere una notizia anticipata da "spiegel" secondo cui la cdu potrebbe decidere al congresso di metà dicembre di assumere nel proprio programma un "obbligo di integrazione" per tutti i migranti. il partito di angela merkel proporrebbe l'obbligo per chi vuole venire a vivere in germania di accettare la parità tra i sessi e la superiorità delle leggi tedesche rispetto alla sharia, di non accettare la discriminazione di donne, omosessuali e appartenenti ad altre religioni e di riconoscere l'esistenza di israele. nel caso i migranti violassero uno di questi principi, la cdu suggerirebbe di tagliare gli aiuti sociali o di togliere il permesso di soggiorno. correlati & 386 & medium & Medium & Power & NA & NA & 2015-11-29 & 2015 & 1 & POL
Frame & low-medium & National & <500 & -0.5155635 & -0.1667445 & -0.9728619 & 0.5015415 & 1.3050799 & 3.2 & 0.6840199 & -1.2389660 & Payer & European & European & European & European|POL & Negative\\
Italy & http://economia.ilmessaggero.it/flashnews/ue\_investe\_per\_ferrovia\_sud\_italia\_74\_7\_miliardi\_di\_euro\_per\_migliori\_collegamenti\_calabria\_puglia\_verso\_nord\_italia-3261230.html & 419 & economia.ilmessaggero.it & Private/Non-Public & Online and Offline & National & very high = CP is most important issue + CP is mentioned in title/headline & Infrastructure & Positive & EU + Subnational & No myth & NA & NA & NA & NA & NA & NA & NA & NA & Italy & ue investe per ferrovia sud italia: 74,7 miliardi di euro per migliori collegamenti calabria-puglia verso nord italia & 2017-09-25 & fondo europeo di sviluppo regionale & (teleborsa) - stanziati 74,7 milioni di euro dal fondo europeo di sviluppo regionale (fesr) per i lavori di ammodernamento di 70 km di linea ferroviaria che collega la calabria, dalla provincia di cosenza, alla puglia. la commissaria per la politica regionale corina cretu ha precisato come "l'ammodernamento di questo collegamento ferroviario regionale contribuirà alla crescita dell'economia locale e favorirà il turismo e il commercio". questa linea fa parte della rete transeuropea di trasporto globale (rte-t). il progetto mira in particolare a sviluppare il trasporto intermodale delle merci dal porto di gioia tauro verso il nord italia, passando per taranto e bari. i lavori dovrebbero terminare nell'estate del 2019. ulteriori informazioni sui fondi europei in italia sono disponibili nella piattaforma open data. & 125 & very high & High & Socio-Economic & NA & NA & 2017-09-25 & 2017 & 2 & ECO
Frame & high-very high & National & <500 & -0.5155635 & -0.1667445 & -0.9728619 & 0.5015415 & 1.3050799 & 3.2 & 0.6840199 & -1.2389660 & Payer & Domestic & European & Mixed & Domestic|ECO & Positive\\
Italy & http://www.ilsole24ore.com/art/commenti-e-idee/2018-06-01/come-spendere-meglio-fondi-ue-170336.shtml?uuid=AEcEO2xE & 439 & Il Sole 24 ORE & Private/Non-Public & Online and Offline & National & very high = CP is most important issue + CP is mentioned in title/headline & Improve governance & Balanced & EU + National & 8.Mismanaged & Tackle brain drain & Negative & National & No myth & Bureaucracy & Negative & EU + Subnational & 10.Slow spend & Italy & come spendere meglio i fondi ue & 2018-06-01 & politica di coesione & la strategia italiana sulle politiche di coesione europee post2020 comunicata alla commissione ue dal precedente governo e l'eccellente rapporto sui conti territoriali 2017 evidenziano problemi di grande rilevanza per il paese. il governo gentiloni ha chiesto di mantenere le risorse della coesione, darle a tutte le regioni, non condizionarle alle riforme strutturali, garantire la "addizionalità" con i fondi nazionali, semplificare la spesa e limitare i controlli, spostandoli sulla verifica dei risultati da parte delle autorità nazionali. in sintesi, l'italia ha chiesto all'europa di lasciare le cose come stanno e limitare la sua presenza nella co-gestione. il rapporto sui conti economici territoriali 2017, di cui il sole 24 ore ha scritto nelle scorse settimane, evidenzia la continua diminuzione dei fondi nazionali e la difficoltà a spenderli nelle regioni del mezzogiorno. in un contesto europeo con divari di reddito ed occupazione crescenti tra nord e sud europa e crescenti diseguaglianze tra gruppi sociali, la commissione europea ha proposto di redistribuire i fondi, riducendoli per i paesi del nord e dell'est e aumentandoli per quelli del sud, compresa l'italia dove i fondi nazionali sono diminuiti dal 50\% all'11\% nella scorsa programmazione. quali cambiamenti sono necessari perché la coesione, con maggiori risorse europee e nazionali, abbia un impatto significativo sulla crescita dell'economia italiana? innanzitutto, la politica di coesione manca di una "visione del futuro" e di una strategia sulla quale fare convergere risorse ed energie. e mentre i giovani vanno a cercarsela emigrando e spopolando le aree interne, i fondi nazionali sono spesi senza programmi operativi che stabiliscano impegni cogenti, tempi certi, controlli e valutazioni della spesa. al contrario, i programmi cofinanziati dalla ue sono spesi con certezza e più celermente. i programmi cofinanziati dalla ue nel mezzogiorno, supplendo alla spesa ordinaria, si disperdono in centinaia d'interventi di ordinaria amministrazione o di emergenza, privi sia di impatto "strutturale" che di effetti "congiunturali", per i tempi amministrativi lunghissimi. in questo modo si mantiene l'esistente anche se non competitivo. non si aiuta la trasformazione delle imprese e la loro internazionalizzazione. gli aiuti e qualche sporadica opera pubblica non incidono sulle decisioni d'investimento. le regioni forti si rivolgono alle reti globali per produrre ed esportare e vedono il mezzogiorno come un peso che si riflette sulla imposizione fiscale e contributiva piuttosto che una opportunità. dunque, una riprogrammazione dei programmi 2014-2020 sarebbe da considerare. occorrerebbe concentrare la spesa su progetti "strutturali" per costo e per impatto, ridurne drasticamente il numero ed il peso sulla amministrazione e lasciare ai fondi nazionali l'intervento di gestione ordinaria e la pioggia di piccoli aiuti. è necessario, inoltre, semplificare i processi amministrativi, troppo frammentati tra istituzioni e oggetto di mediazioni, accordi, patti, poteri di veto e financo leggi di riforma dello stato disattese, in un processo in cui sono implicati decine di uffici e che alla fine produce solo un'accozzaglia di interventi, affannosamente messi insieme piuttosto che una strategia. che senso ha disporre di più fondi quando solo nella passata programmazione sono stati sottratti 11 miliardi al co-finanziamento nazionale (pac) e il fondo sviluppo e coesione ha una spesa al 21\%, a causa della incapacità di spesa delle amministrazioni del sud e centrali? gli operatori nazionali e la ue invocano riforme strutturali. da uno studio ismeri risulta che dal 2012 al 2015 su 42 riforme richieste del consiglio, 10 riguardano pubblica amministrazione, quattro l'ambiente per le imprese; sette l'accesso all'impiego. per capire la rilevanza della p.a., il ciclo delle opere pubbliche superiori a 20 milioni di euro dura da 10 a 14 anni! i tempi amministrativi "morti", "di attraversamento" sono in media il 75\% dei tempi necessari a realizzare un'opera. quindi, solo per le pratiche amministrative si sprecano tra i 7 e i 10 anni. con i piani di rafforzamento amministrativo (pra) 28 tre regioni e ministeri che gestiscono fondi europei si erano impegnati su tempi e semplificazione. non sappiamo se i risultati promessi alla fine della fase 1 nel 2017 siano stati raggiunti. comunque, un richiamo per le amministrazioni che non hanno operato per realizzare riforme di esclusivo interesse dei cittadini a sud come a nord è necessaria perché vi sia un impatto non solo sull'attuale programma ma anche sulla credibilità del paese sulla trattativa per il bilancio post2020. il nuovo governo non potrà evitare la questione. & 726 & very high & High & Governance & Socio-Economic & Governance & 2018-06-01 & 2018 & 3 & POL
Frame & high-very high & National & 500-1000 & -0.5155635 & -0.1667445 & -0.9728619 & 0.5015415 & 1.3050799 & 3.2 & 0.6840199 & -1.2389660 & Payer & Domestic & European & Mixed & Domestic|POL & Neutral\\
\addlinespace
Italy & http://www.ansa.it/europa/notizie/rubriche/altrenews/2018/10/10/regioni-ue-contro-tagli-a-coesione-ripartire-dai-territori\_755928a3-ae3f-43f4-8af4-87ba495af179.html & 410 & ANSA.it & Private/Non-Public & Online only & National & very high = CP is most important issue + CP is mentioned in title/headline & Institutional bargaining over funding & Balanced & EU + National + Subnational & No myth & NA & NA & NA & NA & NA & NA & NA & NA & Italy & regioni ue contro tagli a coesione, ripartire dai territori - altre news - ansa europa & 2018-10-10 & politica di coesione & (ansa) - bruxelles, 10 ott - evitare tagli alla politica di coesione post-2020, potenziare il contributo al bilancio pluriennale e rilanciare l'europa partendo dai territori: questi i punti principali emersi dalla plenaria del comitato europeo delle regioni (cdr) a bruxelles, da cui sono usciti una serie di pareri tra cui una presa di posizione molto forte sul bilancio pluriennale dell'ue (mff). secondo le regioni europee, il budget post-2020 dovrebbe contenere risorse adeguate e nuovi strumenti per i territori per stimolare investimenti su lavoro, ambiente e integrazione. il testo adottato si oppone ai tagli alla politica di coesione e all'agricoltura, e chiede di portare il contributo al bilancio comunitario all'1,3 \% del pil degli stati membri dopo il 2020. posizione che riflette il discorso sullo stato dell'unione del presidente del cdr, karl-heinz lambertz, che ieri ha ribadito la centralità delle regioni per il progetto ue, e che ha respinto i "tagli sproporzionati" alla politica di coesione post-2020 proposti dall'esecutivo comunitario. adottato anche il testo del consigliere comunale di vittorio veneto marco dus, che chiede più fondi per il programma europeo life per la salvaguardia dell'ambiente. la commissione ha proposto di portare il budget del programma a 5,45 miliardi dopo il 2020, ma l'aumento non è giudicato sufficiente dal cdr considerato che in futuro i programmi sull'energia pulita che rientrano in 'horizon 2020' saranno parte di life. "chiediamo quindi ai co-legislatori - spiega dus - di aumentare il budget post-2020 dai 5,45 ai 6,78 miliardi" di euro. il testo del sindaco di valdengo (biella) e vicepresidente dell'anci, roberto pella, sottolinea invece l'importanza delle politiche di promozione dello sport nelle città per migliorare la qualità della vita e creare occupazione per i giovani. "lo sport rappresenta il 2,12\% del pil dell'ue e 5,67 milioni di occupati", ha sottolineato pella. "sono cifre importanti", che vanno ad abbracciare "quel mondo che soffre di più rispetto agli altri": quello della "disoccupazione giovanile". adottato inoltre il testo del presidente della provincia di avellino, domenico gambacorta, che chiede di migliorare le competenze digitali dei giovani europei agendo sulle infrastrutture, con l'ampliamento della copertura della banda larga nei paesi ue, ma anche la formazione.(ansa). & 376 & very high & High & Power & NA & NA & 2018-10-10 & 2018 & 3 & POL
Frame & high-very high & National & <500 & -0.5155635 & -0.1667445 & -0.9728619 & 0.5015415 & 1.3050799 & 3.2 & 0.6840199 & -1.2389660 & Payer & Domestic & European & Mixed & Domestic|POL & Neutral\\
Italy & http://www.ansa.it/piemonte/notizie/2018/06/01/regione-515-mln-contro-disoccupazione\_a60b7fa8-4586-46a9-a0b8-fa3f5b0117e1.html & 468 & ANSA.it & Private/Non-Public & Online only & National & medium = CP is important part of story & Jobs & Positive & Subnational & No myth & NA & NA & NA & NA & NA & NA & NA & NA & Italy & regione, 51,5 mln contro disoccupazione - piemonte & 2018-06-01 & fondo sociale europeo & (ansa) - torino, 1 giu - la regione piemonte stanzia 51,5 milioni, in gran parte dal fondo sociale europeo, per la lotta contro la disoccupazione. lo prevede una delibera approvata oggi dalla giunta chiamparino su proposta sell'assessora al lavoro e formazione professionale, gianna pentenero. il provvedimento prevede percorsi per l'occupabilità e l'aggiornamento delle competenze, rivolti a persone disoccupate in possesso di qualifica, diploma o laurea, che vogliano conseguire una specializzazione. o anche rivolti occupati e disoccupati per il conseguimento di qualifica o abilitazione professionale. previsti anche percorsi per l'inclusione socio-lavorativa di soggetti vulnerabili come detenuti, disabili, e stranieri. una novità sono i percorsi brevi per la riqualificazione e ricollocazione professionale delle persone che hanno perso il lavoro e sono inserite nelle misure di politica attiva oppure provengono da aziende in crisi. & 135 & medium & Medium & Socio-Economic & NA & NA & 2018-06-01 & 2018 & 3 & ECO
Frame & low-medium & National & <500 & -0.5155635 & -0.1667445 & -0.9728619 & 0.5015415 & 1.3050799 & 3.2 & 0.6840199 & -1.2389660 & Payer & Domestic & Domestic & Domestic & Domestic|ECO & Positive\\
Italy & http://www.pmi.it/economia/finanziamenti/news/102005/fondi-ue-per-regioni-pmi.html?utm\_source=feedburner\&utm\_medium=feed\&utm\_campaign=Feed\%253A\%2BPMI.it\%2B-\%2BNews & 464 & PMI.it & Private/Non-Public & Online only & National & very high = CP is most important issue + CP is mentioned in title/headline & Economic development & Positive & National + Subnational & No myth & Environment/green/low-carbon & Positive & National + Subnational & No myth & NA & NA & NA & NA & Italy & fondi ue per regioni e pmi italiane & 2015-09-04 & fondo europeo di sviluppo regionale & nuovi finanziamenti europei per investimenti innovativi in sicilia, basilicata e veneto: focus su pmi, efficienza energetica e banda barga. la commissione ue ha approvato i primi finanziamenti regionali - si comincia con veneto (300 milioni), basilicata (413 milioni) e sicilia (3,41 mld), stanziati nell'ambito dei programmi 2014-2020 per creare posti di lavoro, potenziare l'innovazione. erogati dal fondo europeo di sviluppo regionale (fesr), i fondi ue per l'italia saranno integrati da un cofinanziamento nazionale. le pmi potranno farne uso per internazionalizzarsi, diventare più competitive, lanciare nuovi prodotti sul mercato. investimenti anche per estendere la banda larga e creare nuovi servizi online. inoltre, i programmi comprendono misure per favorire efficientamento energetico, uso delle rinnovabili e riconversione energetica degli edifici pubblici. saranno infine favoriti raccolta differenziata, trattamento acque reflue e approvvigionamento idrico. & 133 & very high & High & Socio-Economic & Socio-Economic & NA & 2015-09-04 & 2015 & 1 & ECO
Frame & high-very high & National & <500 & -0.5155635 & -0.1667445 & -0.9728619 & 0.5015415 & 1.3050799 & 3.2 & 0.6840199 & -1.2389660 & Payer & Domestic & Domestic & Domestic & Domestic|ECO & Positive\\
Italy & http://www.adnkronos.com/soldi/lavoro/2015/03/30/mila-presenze-oltre-mila-colloqui-iolavoro\_DsqR9GNgH0fdE4myawYl8N.html & 421 & Adnkronos & Private/Non-Public & Online only & National & low = CP mentioned more times but NOT important part of story (mainly about others issues) & Jobs & Positive & Subnational & No myth & NA & NA & NA & NA & NA & NA & NA & NA & Italy & 9mila presenze e oltre 15mila colloqui a iolavoro & 2015-03-30 & fondo sociale europeo & ben 9mila partecipanti hanno sostenuto oltre 15mila colloqui di lavoro, hanno assistito a un centinaio tra workshop, seminari, conferenze su tematiche del lavoro e della formazione, hanno incontrato e dialogato con 120 tra aziende, agenzie per il lavoro, franchisor, agenzie formative e istituti tecnici e professionali e hanno avuto la possibilità di approfondire la conoscenza delle misure e dei percorsi di autoimprenditorialità per la creazione della propria impresa. questi i numeri di iolavoro. si riconferma la grande partecipazione alle iniziative per l'orientamento formativo e professionale ai mestieri worldskills. oltre 2.000 ragazzi delle scuole medie hanno partecipato ai tour dei mestieri (cuoco, cameriere, pasticcere, grafico, meccanico, sarto, acconciatore, estetista e muratore). cento docenti hanno affiancato 300 studenti che sono stati coinvolti attivamente nell'organizzazione dei laboratori. l'appuntamento con i campionati dei mestieri (validi per l'accesso ai nazionali e agli europei di goteborg 2016) sarà dal 21 al 23 ottobre durante iolavoro: in gara i ragazzi del piemonte e, per la prima volta, quelli della liguria. sono, inoltre, in corso accordi con le altre regioni per aprire le competizioni ai giovani provenienti da tutta italia. "esperienze come worldskills - dichiara l'assessore all'istruzione, lavoro e formazione professionale, giovanna pentenero - contribuiscono a modificare la cultura del mondo del lavoro. spesso si tende a pensare che gli istituti tecnici, professionali e le agenzie di formazione non favoriscano un progetto lavorativo dignitoso. e' un errore: esse permettono ai ragazzi di apprendere competenze di carattere tecnico professionali e, inoltre, nessun percorso di studi preclude l'accesso al sistema dell'alta formazione e quindi all'università. eventi come wolrdskills sono efficaci e utili perché mostrano le diverse competenze professionali in una dinamica competitiva, di assoluta innovazione, più attraente per i ragazzi". "in questo senso -avverte- si inseriscono anche le attività di orientamento e riorientamento finanziate con le risorse del fondo sociale europeo e il programma garanzia giovani, dedicato all'individuazione e al recupero dei cosiddetti neet, i giovani tra i 15 e i 29 anni che non sono iscritti né a scuola né all'università, che non lavorano e non cercano un'occupazione. i numeri del progetto regionale sono positivi, e altrettanti saranno quelli del programma nazionale ma io credo che, se davvero si vuole raggiungere la maggior parte dei neet, ci sia bisogno di intercettarli con canali più informali. penso quindi a una garanzia giovani 2, ad azioni mirate e comuni definite in sinergia tra enti locali, ministero dell'istruzione e quindi uffici scolastici regionali, aziende, agenzie per il lavoro e centri per la formazione professionale". iolavoro compie 18 edizioni e si conferma una straordinaria piattaforma al servizio di chi è alla ricerca di uno sbocco professionale. un progetto d'eccellenza unico in italia che ora prosegue il tour con appuntamenti in altre città del piemonte: 14 e 15 maggio a vercelli, 10 e 11 giugno ad alessandria. la 19\textasciicircum{} edizione è prevista il 21, 22, 23 ottobre a torino. "iolavoro si riconferma un appuntamento dai grandi numeri - dichiara l'assessore all'istruzione, lavoro e formazione professionale - in un contesto di cauto ottimismo sulla ripresa economica del nostro paese e del piemonte, iniziative come queste rappresentano un'esperienza felice di politica pubblica che mira a mettere in contatto chi è alla ricerca di un'occupazione con le imprese che richiedono manodopera specializzata. giunto alla diciottesima edizione, è diventato negli anni anche una vetrina del panorama delle agenzie formative della regione, che colgono l'opportunità per mostrare le proprie attività, i corsi e le competenze specifiche. obiettivo è, infatti, avvicinare scuola e mondo del lavoro per valorizzare quel sistema duale che vede nell'alternanza scuola-lavoro e nell'apprendistato le sue forme più consuete e che può considerarsi un mezzo per la sfida alla disoccupazione giovanile". "e' il momento di guardare - ha dichiarato franco chiaramonte, direttore agenzia piemonte lavoro - con ottimismo alle opportunità che vengono dal mercato del lavoro che, finalmente, dà segni di ripresa. chi viene a iolavoro trova concretamente queste opportunità". nel corso della job fair è stato siglato il protocollo d'intesa tra iolavoro e il salone internazionale del libro, nell'ambito del programma garanzia giovani, per l'occupazione giovanile nel settore dell'editoria. presentata, poi, la garanzia giovani per persone con disabilità e consegnato a tiger italia il premio iolavoro-h dedicato alle aziende che si distinguono nell'inserimento lavorativo di persone con disabilità. tra le iniziative volte a favorire l'auto-imprenditorialità, hanno suscitato grande interesse le esperienze degli startupper del programma steps piemonte e i workshop della franchising school che ha proposto anche approfondimenti su crowdfunding e microcredito, le innovative forme di finanziamento alternative al sistema tradizionale. la manifestazione iolavoro è finanziata dal fondo sociale europeo, promossa dalla regione piemonte, organizzata dall'assessorato istruzione, lavoro e formazione professionale della regione piemonte, realizzata dall'agenzia piemonte lavoro in collaborazione con camera di commercio di torino, città metropolitana di torino, città di torino e con la partecipazione del ministero del lavoro, inps piemonte, italia lavoro, centri per l'impiego, servizi per l'impiego francesi pôle-emploi rhône-alpes e rete eures. & 842 & low & Low & Socio-Economic & NA & NA & 2015-03-30 & 2015 & 1 & ECO
Frame & low-medium & National & 500-1000 & -0.5155635 & -0.1667445 & -0.9728619 & 0.5015415 & 1.3050799 & 3.2 & 0.6840199 & -1.2389660 & Payer & Domestic & Domestic & Domestic & Domestic|ECO & Positive\\
Italy & https://www.ilsole24ore.com/art/notizie/2018-11-20/manovra-oggi-bocciatura-bruxelles-ecco-prossime-tappe-214741.shtml?uuid=AEYL6VkG & 414 & Il Sole 24 ORE & Private/Non-Public & Online and Offline & National & very low = CP mentioned once & Political leverage & Balanced & EU + National & No myth & NA & NA & NA & NA & NA & NA & NA & NA & Italy & manovra, oggi la bocciatura di bruxelles. ecco le prossime tappe & 2018-11-21 & fondi strutturali & tra poche ore si compirà il primo passo formale verso l'apertura di una procedura per debito eccessivo nei contronti del governo italiano. la commissione pubblicherà infatti un'altra opinione negativa sul documento programmatico di bilancio. stavolta su quello aggiornato, ma non nei saldi, il 13 novembre. siccome non contiene le modifiche "sostanziali e considerevoli" chieste dalla ue, l'opinione ribadirà quanto scritto il 23 ottobre: la manovra contiene una deviazione dagli impegni "particolarmente grave", si basa su "ipotesi ottimistiche di crescita", mette a rischio "una riduzione adeguata del debito", che resta una "grande vulnerabilità". motivazioni che hanno portato bruxelles a preparare anche l'ormai noto "rapporto sul debito", chiamato 126.3 dall'articolo del trattato che lo descrive. il rapporto sul debito salvo sorprese dell'ultim'ora, o decisioni last minute di juncker - che vedrà il premier conte solo sabato sera - il collegio dei commissari è pronto a pubblicare domani anche il rapporto 126.3. è il documento in cui la commissione chiarisce perché non è convinta dalle ragioni ("fattori rilevanti") che l'italia ha indicato per spiegare l'andamento dei conti. certifica anche che l'italia viola la regola del debito e avverte che la procedura non è più rinviabile. per questo è quindi considerato il primo passo formale che potrebbe condurre all'apertura della procedura. sanzioni solo come extrema ratio ma, appunto, il condizionale è d'obbligo. non solo perché ogni tappa deve essere validata anche dall'ecofin, ma anche perché non è un percorso lineare quello che porta alle sanzioni. anzi, multe e quant'altro (ad esempio il blocco dei fondi strutturali) sono l'ultimo passo in assoluto e potrebbero non verificarsi mai, come accaduto con spagna e portogallo: quando non rispettarono il rientro dal deficit, la commissione impiegò mesi per raccomandare la multa, ma nel frattempo i due governi trovarono un accordo con la ue e la procedura decadde. anche l'italia potrebbe quindi negoziare per mesi e non arrivare mai alle sanzioni. procedura vera e propria da gennaio in ogni caso, l'eventuale lancio vero e proprio della procedura ue è improbabile che avvenga prima di gennaio, ovvero prima che la manovra venga approvata dal parlamento. ma dopo le feste, se la commissione aprisse l'iter e l'ecofin del 22 gennaio lo confermasse, il rischio più immediato previsto dalle regole sarebbe un altro: la richiesta di una manovra correttiva da fare in 3-6 mesi. e solo dopo scatterebbero le sanzioni pecuniarie che possono andare dallo 0,2\% allo 0,5\% del pil. sempre che nel frattempo lo spread non raggiunga livelli tali da rendere necessari interventi pesanti e immediati. & 437 & very low & Low & Power & NA & NA & 2018-11-21 & 2018 & 3 & POL
Frame & v.low & National & <500 & -0.5155635 & -0.1667445 & -0.9728619 & 0.5015415 & 1.3050799 & 3.2 & 0.6840199 & -1.2389660 & Payer & Domestic & European & Mixed & Domestic|POL & Neutral\\
\addlinespace
Italy & http://www.agi.it/perugia/notizie/regioni\_chiacchieroni\_rafforzare\_le\_funzioni\_svolte\_da\_seu-201507220957-cro-rt10047 & 480 & AGI & Private/Non-Public & Online only & National & low = CP mentioned more times but NOT important part of story (mainly about others issues) & Improve governance & Balanced & Subnational & No myth & NA & NA & NA & NA & NA & NA & NA & NA & Italy & regioni: chiacchieroni, rafforzare le funzioni svolte da seu & 2015-07-22 & fondi strutturali & (agi) - perugia, 22, lug. - il consigliere regionale gianfranco chiacchieroni (pd), con una interrogazione, chiede alla giunta regionale dell'umbria se si intenda "rafforzare le funzioni svolte da seu (servizio europa), adeguandone la forma societaria in modo tale da incrementarne i livelli di imprenditorialita' ed efficienza".chiacchieroni spiega che seu (agenzia partecipata dalla regione umbria) si occupa di 'fondi europei relativamente a formazione e progettazione ed ha costruito in questi anni di attivita' una rete consolidata di relazioni nazionali ed europee". "in ragione di cio' - sostiene l'esponente del pd - e considerato che la giunta regionale impegna quasi tutte le energie e risorse del proprio apparato per la gestione dei fondi strutturali europei, sarebbe estremamente utile ed opportuno rafforzare ed incrementare i servizi forniti da seu".(agi) pg2/mav & 128 & low & Low & Governance & NA & NA & 2015-07-22 & 2015 & 1 & POL
Frame & low-medium & National & <500 & -0.5155635 & -0.1667445 & -0.9728619 & 0.5015415 & 1.3050799 & 3.2 & 0.6840199 & -1.2389660 & Payer & Domestic & Domestic & Domestic & Domestic|POL & Neutral\\
Italy & http://www.repubblica.it/politica/2016/11/02/news/il\_deficit\_le\_virgole\_e\_i\_conti\_in\_ordine-151157660/ & 444 & La Repubblica.it & Private/Non-Public & Online and Offline & National & very low = CP mentioned once & Infrastructure & Balanced & EU + National & No myth & NA & NA & NA & NA & NA & NA & NA & NA & Italy & il deficit, le virgole e i conti in ordine & 2016-11-02 & fondi strutturali & rimettere in piedi questo paese è un'opera titanica. renzi dimostri di esserne all'altezza giocando a viso aperto. in europa occorre chiarezza. se la manovra era scritta sull'acqua prima del sisma, ora lo è forse ancora di più le nostre vite per uno zero virgola. posta in questi termini, la contesa tra roma e bruxelles sui costi della ricostruzione e della messa in sicurezza del nostro patrimonio abitativo e culturale è più che surreale. e' penosa. l'aritmetica rivendica il suo primato sulla politica. la contabilità si pretende superiore alla solidarietà. anche quando in gioco non ci sono solo le cifre del deficit, ma i numeri delle vittime di un tremendo terremoto che tra il 24 agosto e il 30 ottobre ha ucciso quasi 300 persone, devastato 200 comuni e distrutto i tesori d'arte di cui si è nutrita la cultura occidentale. renzi, addolorato, alza la voce: "se dopo quello che è successo qualcuno mi parla ancora di regole europee, significa che ha perso la testa". la commissione europea, "delusa", risponde a tono: la lettera con la quale il governo italiano indica le due emergenze sisma-migranti come "circostanze eccezionali" che giustificano l'aumento del disavanzo strutturale dello 0,4\% (invece della promessa riduzione dello 0,6\%), è "poco costruttiva". se ci soffermassimo agli aspetti formali, questo sì, sembrerebbe uno "scontro di civiltà". e tutti, elettori ed eletti, dovrebbero schierarsi compatti, senza se e senza ma, dalla parte della democrazia e contro la "tecnocrazia". per solide ragioni etiche (il presente e il futuro dei nostri figli) e non per le solite mozioni retoriche (una vacua "concordia nazionale", che significa tutto e niente). ma in questa triste vicenda ci sono questioni sostanziali sulle quali non si può sorvolare, sia pure sull'onda del dolore che suscitano i volti sfigurati dei sopravvissuti di preci o i frontoni sfregiati delle chiese di norcia. sono questioni sulle quali il governo non può e non deve sbagliare, se vuole ottenere il supporto dei partner in europa e il sostegno delle opposizioni in italia. in europa occorre una chiarezza che finora è obiettivamente mancata. se la manovra economica era scritta sull'acqua prima del sisma, ora lo è forse ancora di più. nella lettera di risposta ai rilievi della ue, padoan ha cifrato i maggiori costi per la ricostruzione in due decimi di pil, cioè 3,4 miliardi. a leggere i testi della legge di bilancio si scopre invece che gli stanziamenti previsti dal governo sono molto inferiori: 100 milioni per la "ricostruzione privata" (cioè "per la concessione del credito d'imposta maturato in relazione all'accesso ai finanziamenti agevolati") più altri 200 milioni "per la concessione di contributi finalizzati alla ricostruzione pubblica". in tutto fanno 300 milioni. se a questi si aggiungono gli altri 300 milioni di "cofinanziamento regionale di fondi strutturali", il totale delle risorse per il 2017 fa solo 600 milioni. come si arriva ai 3,4 miliardi di "flessibilità aggiuntiva" chiesti all'europa? per quali incontrollabili rivoli della spesa, diversa da quella necessaria al dopo-sisma, rischia di disperdersi lo stanziamento "eccezionale" preteso dal governo in deroga al patto di stabilità? se è questo il dubbio che serpeggia a bruxelles, la reazione più appropriata da roma non deve essere l'ira funesta, ma la collaborazione istituzionale. la manovra è malpensata, malfatta e malscritta. l'europa, evidentemente, teme che la vera "circostanza eccezionale" (per la quale il premier chiede la possibilità di fare più deficit) non sia il terremoto, ma sia il referendum. e cioè che quei 2,8 miliardi di fondi stanziati per il sisma (di cui non c'è traccia nelle tabelle della legge di stabilità, e che risultano dalla differenza tra i 3,4 miliardi richiesti in disavanzo e i 600 milioni effettivamente iscritti a bilancio), più che a finanziare la messa in sicurezza di case chiese e scuole, servano a coprire le "mancette referendarie": dalla quattordicesima ai pensionati al bonus alle mamme, dai fondi per il trasporto in campania ai ponti sullo stretto in sicilia. può apparire odioso quanto si vuole. ma allo stato attuale, viste le troppe incongruenze della manovra, è un sospetto legittimo. il 4 dicembre l'italia va alle urne per la riforma costituzionale. a primavera si vota per le presidenziali in francia. subito dopo tocca alle legislative in germania. arginare l'uso elettorale dei deficit pubblici è un dovere comune. dunque, renzi ha un modo molto semplice per fugare i sospetti di bruxelles: chiarisca in modo inequivocabile com'è articolata la legge di stabilità. spieghi dove e come saranno usati quei due decimi in più di spesa, con destinazione esclusiva agli investimenti del dopo terremoto. il ragionamento vale anche in italia. l'appello alla famosa e fumosa "coesione nazionale" può avere qualche senso solo se è costruito sulla totale trasparenza delle azioni e delle intenzioni. di fronte a questa tragedia italiana non possono esserci zone d'ombra. dal terremoto del belice del 1968 abbiamo avuto sette eventi sismici, costati 122 miliardi. su 30 milioni di abitazioni, 15 milioni sono state costruite prima del 1974, e sono considerate a rischio sismico. mettere in sicurezza gli immobili delle zone più esposte al pericolo richiede 130 miliardi. rimettere in piedi questo paese è un'opera titanica. renzi dimostri di esserne all'altezza, giocando a viso aperto ma con i conti in regola. solo se fa questo può presentarsi in parlamento e mettere con le spalle al muro una destra berlusconiana che deve ancora farsi perdonare lo scandalo vergognoso delle malinconiche "new town" dell'aquila, e un movimento grillino che deve ancora chiedere scusa per le patetiche fumisterie complottarde di certi suoi stralunati "cittadini". ognuno faccia la sua parte, con rigore ma con responsabilità. questa italia ferita ha bisogno di tutto, fuorché, come cantava de andré, di "regine del tua culpa che affollano i parrucchieri". & 969 & very low & Low & Socio-Economic & NA & NA & 2016-11-02 & 2016 & 2 & ECO
Frame & v.low & National & 500-1000 & -0.5155635 & -0.1667445 & -0.9728619 & 0.5015415 & 1.3050799 & 3.2 & 0.6840199 & -1.2389660 & Payer & Domestic & European & Mixed & Domestic|ECO & Neutral\\
Italy & http://lasentinella.gelocal.it/ivrea/cronaca/2016/09/20/news/meeting-point-di-ivrea-al-via-i-lavori-sara-un-centro-per-le-imprese-1.14122266 & 401 & La Sentinella del Canavese & Private/Non-Public & Online and Offline & Regional/Local & very low = CP mentioned once & Economic development & Positive & EU + Subnational & No myth & Infrastructure & Positive & EU + Subnational & No myth & NA & NA & NA & NA & Italy & meeting point, al via i lavori: sarà un centro per le imprese  - cronaca - la sentinella del canavese & 2016-09-20 & fondo sociale europeo & ottocentomila euro la somma investita, attraverso il progetto canavese business park previsto anche un salone pluriuso da 1.600 metri quadrati e una nuova palestra per la scherma ivrea. alla fine, anche ivrea avrà un salone pluriuso. se ne parla da anni, ma non è mai stato costruito. sarà al meeting point adriano olivetti, una volta che i lavori di riqualificazione dell'edificio saranno completati. sugli 800mila euro i soldi investiti, denaro che arriva per la maggior parte da un bando regionale sui programmi territoriali integrati. l'iter burocratico è cominciato molto tempo fa e ivrea era stata capofila del programma territoriale integrato canavese business park. tra gli interventi proposti, ivrea aveva presentato quello della riqualificazione del meeting point e, attraverso il bando, ha recuperato risorse per il 90\% dell'importo. così, l'edificio di piazza mascagni, che in un primo tempo, alla nascita del nuovo quartiere del parco dora baltea avrebbe dovuto essere abbattuto, non solo tornerà a nuova vita, ma amplierà anche gli spazi a disposizione da dedicare a incubatore di imprese e servizi per le giovani imprese o i lavoratori a partita iva, che possono avere bisogno di spazi, anche condivisi, ma attrezzati, per incontrarsi e fare riunioni. "continuerà - dice enrico capirone, vicesindaco, assessore allo sviluppo economico - ad avere sede l'incubatore di imprese, ma ci saranno più spazi per i servizi. ci sarà anche un centro attrezzato per i co-working, con uffici, sale riunioni, sale per incontri". la facciata del meeting point che dà su piazza mascagni era già stata risistemata un anno fa, attraverso il progetto di pubblica utilità sulle vie di eporedia, legato alle politiche attive del lavoro e finanziato per l'80\% con il fondo sociale europeo. un restyling per l'incubatore di imprese che già da anni è insediato lì (con cinque, sei, realtà per volta che di fatto ruotano su un bando sempre aperto). non è tutto. resterà al meeting point anche la palestra per la scherma, ma smetterà di essere in uno spazio provvisorio (all'ingresso), come accade da quando, quattro anni fa, andò a fuoco la struttura di via san nazario. una delle campate dell meeting point sarà anche rimessa a nuovo e destinata a salone pluriuso da 1.600 metri quadrati. potrà essere utilizzato per eventi, manifestazioni ed esposizioni mentre sarà messo in sicurezza e completamente adeguata alle norme l'ultima campata del grande edificio, oggi adibito ad archivio cartaceo degli uffici giudiziari. e nuova vita avrà anche il magazzino, oggi anche lui bisognoso di una energica risistemata. i lavori, attualmente in corso, sono stati appaltati a un'associazione temporanea di impresa composta dalla ditta cral con sede a verrès, la baratta srl impianti elettrici di aosta e idroservice di champdepraz. nulla cambia, invece, per il parco fotovoltaico realizzato sul tetto dell'edificio dalla photovoltaic systems di chieri, in grado di produrre 750.000 kilowatt/ore. (ri.co.) & 481 & very low & Low & Socio-Economic & Socio-Economic & NA & 2016-09-20 & 2016 & 2 & ECO
Frame & v.low & Regional & <500 & -0.5155635 & -0.1667445 & -0.9728619 & 0.5015415 & 1.3050799 & 3.2 & 0.6840199 & -1.2389660 & Payer & Domestic & European & Mixed & Domestic|ECO & Positive\\
Italy & http://mattinopadova.gelocal.it/padova/cronaca/2016/12/20/news/sociale-picco-di-richieste-di-aiuto-1.14598531 & 394 & il Tirreno & Private/Non-Public & Online and Offline & Regional/Local & very low = CP mentioned once & Social justice & Factual & National & No myth & NA & NA & NA & NA & NA & NA & NA & NA & Italy & sociale, picco di richieste di aiuto  - cronaca - il mattino di padova & 2016-12-20 & fondo sociale europeo & palazzo moroni è costretto a esternalizzare alcuni servizi nel 2017 servizi sociali allo stremo. la crisi economica ha spinto al collasso molte famiglie, e gli uffici comunali non riescono a soddisfare le tante richieste, che continuano a crescere. per questo, la nuova politica è quella di esternalizzare alcuni servizi che non può garantire per mancanza di personale, come la gestione del reinserimento sociale la presa in carico dei percorsi di accoglienza dei minori stranieri. nel primo caso è risultato necessario affidare il servizio, fino al 10 febbraio, a un soggetto esterno per assicurare il miglioramento e il conseguente sviluppo degli interventi. partirà da 48 mila euro di base la contrattazione con le cooperative, che dovranno garantire diversi servizi, dal monitoraggio della gestione diretta e indiretta di anziani, adulti, disabili, minori, immigrati, profughi e famiglie, allo sviluppo di strumenti di informazione e divulgazione di dati nei confronti di altri enti, la rilevazione della qualità percepita nei servizi, fino al supporto nella predisposizione e presentazione di progetti da finanziare con il fondo sociale europeo. nel secondo caso si tratta dei minori stranieri, aumentati dell'8,6\% in italia quest'anno, trend registrato anche in città: via libera alla contrattazione per affidare il servizio di supporto amministrativo fino a dicembre del prossimo anno (38 mila euro). i minorenni che non hanno cittadinanza italiana, ma che si trovano, per qualsiasi causa, nel territorio dello stato privi di assistenza e rappresentanza da parte dei genitori o altri adulti devono obbligatoriamente essere presi in carico dai servizi sociali del comune in cui vengono rintracciati. (l.p.) & 260 & very low & Low & Socio-Economic & NA & NA & 2016-12-20 & 2016 & 2 & ECO
Frame & v.low & Regional & <500 & -0.5155635 & -0.1667445 & -0.9728619 & 0.5015415 & 1.3050799 & 3.2 & 0.6840199 & -1.2389660 & Payer & Domestic & Domestic & Domestic & Domestic|ECO & Neutral\\
Italy & http://www.ansa.it/europa/notizie/rubriche/economia/2014/08/13/uenegoziato-su-41mld-programmazione-2014-2020-a-conclusione\_adac8554-dd77-48f2-a819-2953e7eafbce.html & 486 & ANSA.it & Private/Non-Public & Online only & National & very high = CP is most important issue + CP is mentioned in title/headline & Bureaucracy and/or delays & Balanced & EU + National & No myth & Institutional bargaining over funding & Balanced & EU + National & No myth & NA & NA & NA & NA & Italy & ue, 41mld del 2014-2020 non sono a rischio - ansa.it & 2014-08-16 & fondi strutturali & (ansa) - bruxelles - "i negoziati con roma sull'accordo di partenariato per il 2014-2020 sono a conclusione. per questo non c'è rischio che l'italia possa perdere i 41 miliardi di fondi ue della programmazione", così la commissione ue, che "ringrazia le autorità per l'approccio costruttivo". da bruxelles viene fatto notare come "il negoziato con l'italia non sia un'eccezione. tutti i paesi hanno ricevuto centinaia di osservazioni dettagliate", si stanno negoziando miliardi di euro per i prossimi 7 anni, e quindi è "normale che si discuta a lungo ed in modo dettagliato". a riprova ne è il fatto che la commissione europea ha adottato 13 accordi di partenariato su 28. a seguito delle osservazioni della commissione del 9 luglio - si evidenzia - "il negoziato sull'accordo di partenariato si sta concludendo. rimangono pochissime questioni aperte, anche grazie all'approccio estremamente costruttivo delle autorità italiane". di fatto l'accordo potrebbe essere chiuso entro la fine di settembre e "la commissione ha già cominciato ad esaminare i programmi operativi, alcuni dei quali potrebbero essere adottati entro fine anno". sulla "questione decisiva della capacità amministrativa, la grande novità é l'accordo con l'italia sulla stesura di piani di riforma amministrativa per ogni autorità di gestione dei programmi (siano essi regioni o ministeri), combinati con il ruolo nuovo che giocherà l'agenzia". di conseguenza, "i fondi strutturali della programmazione 2014-2020 non sono a rischio. le risorse saranno a disposizione dell'italia per i prossimi 7 anni". quanto all'assorbimento dei fondi strutturali relativi "al periodo 2007-2013, l'italia è al 59\%, in ritardo rispetto alla media ma in recupero rispetto a qualche anno fa". i programmi più critici restano quelli di sicilia, campania , calabria, e attrattori culturali. (ansa). & 290 & very high & High & Governance & Power & NA & 2014-08-16 & 2014 & 1 & POL
Frame & high-very high & National & <500 & -0.5155635 & -0.1667445 & -0.9728619 & 0.5015415 & 1.3050799 & 3.2 & 0.6840199 & -1.2389660 & Payer & Domestic & European & Mixed & Domestic|POL & Neutral\\
\addlinespace
Italy & http://www.adnkronos.com/lavoro/norme/2015/12/04/mise-via-nuovo-pon-oltre-mln-regioni-del-sud\_5utk2KoS3lJDGgffX7fSfL.html & 456 & Adnkronos & Private/Non-Public & Online only & National & very low = CP mentioned once & Economic development & Positive & National & No myth & NA & NA & NA & NA & NA & NA & NA & NA & Italy & mise: al via nuovo pon, oltre 102 mln a regioni del sud & 2015-12-04 & fondo europeo di sviluppo regionale & più competitività per le piccole e medie imprese meridionali per la competitività delle piccole e medie imprese del mezzogiorno arriva il programma 'iniziativa pmi' 2014-2020 del nuovo programma operativo (pon), con un budget complessivo di 102,5 milioni di euro. ne dà notizia il ministero dello sviluppo economico, precisando che il programma è stato approvato dalla commissione europea nei giorni scorsi. l'ambito territoriale del programma, che mira a migliorare le condizioni di accesso al credito attraverso interventi mirati e basati sul ricorso a strumenti di ingegneria finanziaria, è relativo ad otto regioni: basilicata, calabria, campania, puglia, sicilia abruzzo, molise e sardegna. la dotazione finanziaria del programma deriva da uno specifico conferimento del pon 'imprese e competitività' 2014-2020, già approvato lo scorso 23 giugno e che si è provveduto a riprogrammare.alle risorse stanziate direttamente nell'ambito del programma si aggiungeranno ulteriori risorse - da disciplinare all'interno del previsto accordo di finanziamento tra autorità di gestione e fondo europeo per gli investimenti (fei) - derivanti in parte da fonti di natura nazionale, per un importo analogo a quello previsto come contribuzione fesr (fondo europeo di sviluppo regionale) e in parte dal programma cosme (programma europeo per le pmi). il programma agirà attraverso operazioni di cartolarizzazione di portafogli di prestiti bancari esistenti, in maniera sinergica rispetto alla corrispondente azione svolta dal fondo centrale di garanzia nel pon 'imprese e competitività', che fornisce garanzie alle banche e agli intermediari finanziari riferite sia a singole operazioni finanziarie, sia a portafogli di operazioni. la cartolarizzazione di prestiti esistenti consentirà alle banche che aderiranno all'iniziativa di liberare capitale di vigilanza. il capitale liberato sarà utilizzato dalle stesse banche per erogare nuovi finanziamenti a tasso agevolato alle pmi localizzate nelle regioni del mezzogiorno, per un ammontare complessivo di almeno 1,2 miliardi con un effetto moltiplicatore pari a 6 sulle risorse pubbliche dedicate all'iniziativa. al fine di massimizzare l'efficacia dell'iniziativa, è previsto che una quota consistente dei prestiti da cartolarizzare debba essere assistita dalla garanzia dei confidi. tale previsione consentirà, a parità di risorse stanziate, una liberazione di capitale non solo in capo alle banche ma anche ai confidi, rendendo così disponibili importanti risorse per la concessione di nuove garanzie in favore delle pmi. & 373 & very low & Low & Socio-Economic & NA & NA & 2015-12-04 & 2015 & 1 & ECO
Frame & v.low & National & <500 & -0.5155635 & -0.1667445 & -0.9728619 & 0.5015415 & 1.3050799 & 3.2 & 0.6840199 & -1.2389660 & Payer & Domestic & Domestic & Domestic & Domestic|ECO & Positive\\
Italy & http://www.ottopagine.it/av/attualita/103077/sisma-memoria-delle-vittime-faccia-crescere-la-prevenzione.shtml & 435 & Ottopagine.it & Private/Non-Public & Online only & Regional/Local & very low = CP mentioned once & Infrastructure & Positive & EU + National & No myth & NA & NA & NA & NA & NA & NA & NA & NA & Italy & "sisma, memoria delle vittime faccia crescere la prevenzione" & 2016-11-23 & fondi strutturali & avellino. "il triste ricordo del sisma di 36 anni fa e la memoria delle 2914 persone che il 23 novembre del 1980 persero la vita siano da monito per adeguare i nostri centri storici ai criteri antisismici e soprattutto per dare scuole e asili sicuri ai nostri bambini." così angelo d'agostino, deputato e vicepresidente nazionale di scelta civica. "un paese come il nostro - ha aggiunto il parlamentare - , con tante zone ad alto rischio sismico, ha bisogno di un progetto che serva ad evitare altre tragedie come quella dell'irpinia e di amatrice. non possiamo limitarci solo a ricostruire, il paese ha bisogno di prevenzione. il progetto casa italia, voluto dal governo, va in questa direzione e va sostenuto." "l'irpinia, che a distanza di 36 anni non ha ancora chiuso il capitolo della ricostruzione, merita lo sforzo di una classe dirigente che non deve più dividersi per interessi di bottega, ma deve costruire le condizioni per cogliere con intelligenza ed efficacia le opportunità di crescita offerte dalla ripresa e dai fondi strutturali messi a disposizione dall'unione europea. sarebbe il modo migliore - chiude d'agostino - per onorare la memoria dei nostri tanti morti." redazione & 195 & very low & Low & Socio-Economic & NA & NA & 2016-11-23 & 2016 & 2 & ECO
Frame & v.low & Regional & <500 & -0.5155635 & -0.1667445 & -0.9728619 & 0.5015415 & 1.3050799 & 3.2 & 0.6840199 & -1.2389660 & Payer & Domestic & European & Mixed & Domestic|ECO & Positive\\
Italy & http://www.ansa.it/europa/notizie/rubriche/altrenews/2016/11/03/fondi-uerossidifesa-politiche-coesione-per-europa-piu-equa\_03e8ab38-7cb3-4d5a-a41a-f8e3f9796d74.html & 445 & ANSA.it & Private/Non-Public & Online only & National & very high = CP is most important issue + CP is mentioned in title/headline & Political leverage & Negative & EU + National & No myth & Solidarity to poor countries/regions & Balanced & EU + National & No myth & NA & NA & NA & NA & Italy & fondi ue:rossi,difesa politiche coesione per europa più equa - altre news - ansa europa & 2016-11-03 & politica di coesione & (ansa) - bruxelles, 03 nov - ci sono "strane dimenticanze nei ragionamenti" del presidente della commissione ue jean-claude juncker "rispetto a crescita e sviluppo, dove si dimentica la politica di coesione". "è un fatto grave", perché "gli investimenti di cui l'europa ha bisogno non possono basarsi solo sull'ingegneria finanziaria", e quindi sul cosiddetto 'piano juncker'. lo ha detto all'ansa il presidente della regione toscana enrico rossi, che si trova nelle azzorre per partecipare alla 44esima assemblea generale della conferenza delle regioni marittime periferiche d'europa (crpm). davanti alla crpm, di cui è vicepresidente, rossi ha ribadito la sua linea a sostegno delle politiche di coesione, che vanno "difese" perché sono "nate per abbattere squilibri regionali e per correggere gli errori dell'europa del mercato". il piano juncker è una buona iniziativa, "ma non è abbastanza", ha continuato rossi, perché dipende dalle logiche del mercato ed è quindi "assai probabile che vada verso le realtà più forti" del centro europa, lasciando indietro le regioni periferiche e marittime. secondo rossi serve coordinare fondi del piano juncker e fondi strutturali (con un forte coinvolgimento delle regioni), orientare i fondi strutturali verso gli obiettivi di europa 2020 (incentivando le regioni che lo fanno), e slegare la condizionalità macroeconomica dalle politiche di coesione "perché si rischia di scaricare sulle regioni le difficoltà che arrivano dallo stato centrale". "in gioco ci sono grandi valori, per questo c'è bisogno di grandi politiche", è il monito del governatore toscano. & 244 & very high & High & Power & Values & NA & 2016-11-03 & 2016 & 2 & POL
Frame & high-very high & National & <500 & -0.5155635 & -0.1667445 & -0.9728619 & 0.5015415 & 1.3050799 & 3.2 & 0.6840199 & -1.2389660 & Payer & Domestic & European & Mixed & Domestic|POL & Negative\\
Italy & http://www.ansa.it/emiliaromagna/notizie/2015/11/10/non-fiction-gratis-a-bottega-finzioni\_72860137-c314-4ca1-8580-15489106c71b.html & 407 & ANSA.it & Private/Non-Public & Online only & National & medium = CP is important part of story & Cultural development & Positive & Subnational & No myth & NA & NA & NA & NA & NA & NA & NA & NA & Italy & non fiction gratis a 'bottega finzioni & 2015-11-10 & fondo sociale europeo & (ansa) - bologna, 10 nov - la gratuità delle aree non fiction e produzioni audiovisive e multimediali per bambini e ragazzi realizzate con il co-finanziamento della regione emilia-romagna e del fondo sociale europeo: bottega finzioni, scuola di scrittura fondata da carlo lucarelli (capo bottega), michele cogo (direttore), giampiero rigosi (coordinatore area letteratura) e beatrice renzi (vice direttore) ha presentato i nuovi corsi 2016. per le aree a pagamento, letteratura e fiction, disponibili 12 borse di studio. & 76 & medium & Medium & Socio-Economic & NA & NA & 2015-11-10 & 2015 & 1 & ECO
Frame & low-medium & National & <500 & -0.5155635 & -0.1667445 & -0.9728619 & 0.5015415 & 1.3050799 & 3.2 & 0.6840199 & -1.2389660 & Payer & Domestic & Domestic & Domestic & Domestic|ECO & Positive\\
Italy & http://notizie.tiscali.it/regioni/marche/articoli/fse-marche-assorbite-tutte-risorse-00001/ & 452 & Tiscali & Private/Non-Public & Online only & National & very high = CP is most important issue + CP is mentioned in title/headline & Jobs & Positive & Subnational & No myth & NA & NA & NA & NA & NA & NA & NA & NA & Italy & fse marche, assorbite tutte le risorse & 2017-03-20 & fondo sociale europeo & (ansa) - ancona, 20 mar - il periodo di programmazione 2007-2013 del fondo sociale europeo (fse) si chiude nelle marche con il pieno assorbimento delle risorse disponibili, e una spesa complessiva del 102 per cento. a fronte di 278,7 milioni di euro disponibili, la regione ha pagato 284,8 milioni: un importo superiore alla dotazione assegnata. su 51 mila progetti pervenuti dal territorio, ne sono stati approvati 32 mila, avviati 31 mila e conclusi circa 27 mila. l'analisi è emersa dalla riunione del comitato di sorveglianza del programma operativo regionale (por) che ha approvato il rapporto finale del fse: il principale strumento utilizzato dell'unione europea per sostenere l'occupazione e accompagnare le trasformazioni industriali. i marchigiani che hanno beneficiato degli interventi finanziati con il fse sono stati 100 mila. & 131 & very high & High & Socio-Economic & NA & NA & 2017-03-20 & 2017 & 2 & ECO
Frame & high-very high & National & <500 & -0.5155635 & -0.1667445 & -0.9728619 & 0.5015415 & 1.3050799 & 3.2 & 0.6840199 & -1.2389660 & Payer & Domestic & Domestic & Domestic & Domestic|ECO & Positive\\
\addlinespace
Italy & http://www.affaritaliani.it/puglia/emofilia-pediatrica-a-bari-primo-sistema-integrato-di-diagnostica-a-distanza-524657.html & 457 & Affaritaliani.it & Private/Non-Public & Online only & National & very low = CP mentioned once & Public services & Positive & Subnational & No myth & NA & NA & NA & NA & NA & NA & NA & NA & Italy & emofilia pediatrica, a bari primo sistema integrato di diagnostica a distanza & 2018-02-14 & fondi strutturali & si chiama emo.ti.on la prima soluzione tecnologica integrata per la sicurezza dei bambini affetti da emofilia creata da sei aziende it pugliesi. grazie a questa tecnologia, sarà possibile effettuare diagnostica per immagini a domicilio e videoconsulto medico telematico in favore dei bambini che soffrono di questa malattia rara di origine genetica, dovuta a un difetto della coagulazione del sangue, che colpisce in italia circa 6.000 persone e nel mondo oltre 500.000. la sperimentazione clinica dell'innovativa tecnologia, condotta nell'ultimo anno nella clinica pediatrica 'trambusti' dell'ospedale giovanni xxiii di bari su 15 bambini emofilici tra 3 e 10 anni, si è chiusa positivamente, validando i risultati diagnostici forniti da un prototipo di ecografo digitale portatile e confermando la perfetta trasmissione di dati e immagini attraverso una piattaforma in cloud. è su questo luogo virtuale che i medici possono fornire agli assistenti del malato ('care givers') il proprio teleconsulto contestualmente alla ricezione delle ecografie digitali; indubbio il miglioramento della qualità di vita degli emofilici, sinora costretti a ricorrere con urgenza all'assistenza di centri ambulatoriali e pronto soccorso del territorio o, in assenza di visita, all'infusione di costosi farmaci a scopo preventivo. guarda la gallery i risultati della sperimentazione sono stati presentati nel workshop 'emo.ti.on - tecnologie per la sicurezza dei bambini con emofilia' dall'ats aggiudicataria nel 2016 dell'avviso "cluster tecnologici regionali per l'innovazione" della regione puglia, composta dalla capofila cle di bari, questioncube (start up innovativa di bari), sepi di canosa, system project di andria, tecnolab group di locorotondo e tecnosoft di alberobello. guarda la gallery una volta pubblicati i risultati su una rivista scientifica, il progetto emo.ti.on - che recepisce il piano operativo promosso dal coordinamento regionale malattie rare ed è stato condotto in collaborazione con il dipartimento di scienze biomediche ed oncologia umana dell'università degli studi di bari e con la partecipazione di coremar, aress e abce onlus - sarà pronto per la fase dell'industrializzazione della sonda ecografica. "a due anni dall'inizio del progetto - commenta mariarosaria scherillo, ceo di cle, la società capofila della ats - siamo pronti ad avviare in pochi mesi l'industrializzazione del prodotto che, per il suo carattere innovativo, ha le potenzialità per essere impiegato in ambito nazionale e internazionale. i pazienti colpiti da malattie rare sono 30 milioni in europa, 250 mila in italia e 21.000 in puglia, tra le poche regioni italiane a destinare fondi strutturali europei per queste malattie. emotion si inserisce nell'ambito di una riorganizzazione sanitaria che prevede sempre più un'integrazione organica tra ospedale e territorio in cui l'assistenza domiciliare ricopre un ruolo centrale". cuore del sistema è la piattaforma on line che, oltre alla trasmissione delle ecografie e alle sessioni di teleconsulto, consente ai vari attori coinvolti nel percorso di cura (pazienti, familiari, care givers, medici di famiglia, specialisti e ricercatori) di comunicare tra loro per raggiungere un ambizioso obiettivo: creare, grazie a un motore di ricerca semantico, una modalità di interrogazione in linguaggio naturale dell'intera base di conoscenza medico-scientifica sul tema, dai medicinali disponibili alle posologie consigliate, dalle normative esistenti alle ultime novità sulle malattie rare. primo sistema di questo genere a livello europeo e coperto da segreto industriale, 'emo.ti.on' contribuirebbe a tagliare drasticamente il numero di accessi dei pazienti ai centri ospedalieri, ridurre l'insorgenza di complicazioni grazie alla tempestività degli interventi curativi, utilizzare in modo oculato i farmaci, prevenire danni cronici alle articolazioni dei pazienti, molto complicati da gestire in età adulta e costosi per il sistema sanitario. (gelormini@affaritaliani.it) --------------------------- pubblicato sul tema: emofilia pediatrica, il progetto 'emo.ti.on' & 605 & very low & Low & Socio-Economic & NA & NA & 2018-02-14 & 2018 & 3 & ECO
Frame & v.low & National & 500-1000 & -0.5155635 & -0.1667445 & -0.9728619 & 0.5015415 & 1.3050799 & 3.2 & 0.6840199 & -1.2389660 & Payer & Domestic & Domestic & Domestic & Domestic|ECO & Positive\\
Italy & https://www.repubblica.it/dossier/esteri/fondi-strutturali-europei-progetti-italia/2019/04/08/news/europa\_italia-223343166/ & 420 & La Repubblica.it & Private/Non-Public & Online and Offline & National & very high = CP is most important issue + CP is mentioned in title/headline & Poor communication of funding/rules & Balanced & EU + National & No myth & Mismanagement & Negative & EU + National & No myth & NA & NA & NA & NA & Italy & europa, italia: viaggio tra persone, imprese e territori rinati grazie ai fondi europei & 2019-04-08 & fondi strutturali & al via la sezione di repubblica dedicata a storie attraversate e cambiate dalle politiche di coesione europee. una conoscenza che riteniamo importante, in vista delle prossime elezioni e soprattutto di quel che succederà dopo solo quattro italiani su 10 ne hanno sentito parlare. eppure i fondi europei sono parte della vita economica e sociale delle nostre regioni e città, così come di quelle di tutta l'unione. e il loro sostegno finanziario è spesso ignorato. lo ha detto con toni non diplomatici il presidente della commissione ue jean-claude juncker parlando dei politici italiani "bugiardi" e lo dice il sondaggio di eurobarometro che colloca l'italia tra i paesi che conoscono meno quel che ricevono, pur essendo tra i maggiori beneficiari delle politiche di coesione, di ieri e di oggi: quelle della tornata 2014-2020, per rendere l'idea, hanno stanziato circa 34 miliardi (sui 351,8 del totale dell'unione), e sono appena partiti i negoziati per la programmazione 2021-2027 (per l'italia in ballo oltre 38 miliardi). e' vero, purtroppo siamo anche tra i paesi che hanno registrato più ritardi nell'utilizzo dei fondi, e l'ultimo rapporto della ragioneria ne dà conto, regione per regione. motivo in più per accendere un faro su quel che si è già fatto e su quel che si può fare, vigilare sugli enti locali - che, cofinanziando i progetti, sono responsabili della loro attuazione - , e sviluppare un'attenzione critica basata sulla conoscenza e non sul "sentito dire". ma come vengono usati i fondi strutturali? e da chi? con quali successi, ritardi, difficoltà e realizzazioni? questo progetto nasce proprio per rispondere a queste domande, raccontando storie di persone, imprese e territori attraversati e cambiati dalle politiche di coesione europee. un'iniziativa dedicata ai pezzi di europa che sono in italia. di qui il nome: europa, italia. un approfondimento in vista delle elezioni europee ma soprattutto di quel che succederà dopo il progetto, guidato dalla fondazione brodolini - ente indipendente che fa ricerca, comunicazione e formazione sui temi delle politiche pubbliche, dello sviluppo e dell'innovazione sociale, a livello sia europeo che italiano - è finanziato dalla commissione europea, direzione generale della politica regionale e urbana, e ha repubblica come media partner. il progetto è realizzato con il contributo della commissione europea. il contenuto di questo articolo riflette solo le opinioni dell'autore e la commissione non può essere ritenuta responsabile per qualsivoglia uso fatto delle informazioni qui contenute. & 404 & very high & High & Governance & Governance & NA & 2019-04-08 & 2019 & 3 & POL
Frame & high-very high & National & <500 & -0.5155635 & -0.1667445 & -0.9728619 & 0.5015415 & 1.3050799 & 3.2 & 0.6840199 & -1.2389660 & Payer & Domestic & European & Mixed & Domestic|POL & Neutral\\
Italy & http://www.ansa.it/toscana/notizie/2019/01/23/doc-sul-rock-per-campagna-fse-giovanisi\_01a024b2-0afc-49e9-ad92-8ff7f0ff36a9.html & 488 & ANSA.it & Private/Non-Public & Online only & National & very low = CP mentioned once & Cultural development & Positive & Subnational & No myth & NA & NA & NA & NA & NA & NA & NA & NA & Italy & doc sul rock per campagna fse-giovanisì - toscana & 2019-01-23 & fondo sociale europeo & un docufilm su tre decenni di musica italiana per concludere la campagna di comunicazione 'il futuro addosso' che, su iniziativa di fse (fondo sociale europeo) e giovanisì (il progetto della regione toscana per l'autonomia dei giovani), si è dispiegata per l'intero percorso dell'ultima edizione dello storico concorso fiorentino per band emergenti rock contest. il doc, 'trent'anni di rockcontest', con la regia di giangiacomo de stefano, sarà presentato in prima assoluta venerdì 25 gennaio alle 21 al cinema la compagnia di firenze (ingresso libero fino a esaurimento posti). prodotto da controradio e controradio club, con sonne film, in collaborazione con regione toscana (fse/giovanisì) e società italiana autori editori, il film ripercorre, in poco più di quaranta minuti, tre decenni di musica italiana che si intrecciano con la storia di firenze, attraverso le testimonianze di big come manuel agnelli, dario brunori, piero pelù, lodo guenzi e i subsonica. dalla nascita della radio, nel 1977, "con un budget di centomila lire", come raccontano gli organizzatori, alla prima edizione del concorso nel 1984, gli anni del banana moon, del tenax, della new wave e di una scena musicale fiorentina che vantava band che sarebbero diventate iconiche, come i litfiba o i diaframma. durante la serata un punto informativo ed alcuni video inviteranno i giovani ad informarsi su tirocini, formazione professionale, voucher per l'alta formazione, borse di studio, servizio civile, apprendistato, interventi a favore dei giovani professionisti e su tutte le altre opportunità di giovanisì. i giovani frequentatori del concorso, spiega una nota, "hanno mostrato interesse e apprezzamento per la campagna informativa della regione che ha offerto la possibilità di conoscere progetti e interventi importanti per il loro futuro professionale e personale". le risposte a un questionario distribuito durante il concorso hanno poi "evidenziato la necessità di forme di comunicazione sempre più capaci di aggiornare con tempestività sulle nuove opportunità, sui bandi in uscita e informazioni che, oltre a canali istituzionali e social media, siano in grado di penetrare in luoghi ed eventi in cui si esprime l'aggregazione giovanile, come appunto le manifestazioni musicali" & 346 & very low & Low & Socio-Economic & NA & NA & 2019-01-23 & 2019 & 3 & ECO
Frame & v.low & National & <500 & -0.5155635 & -0.1667445 & -0.9728619 & 0.5015415 & 1.3050799 & 3.2 & 0.6840199 & -1.2389660 & Payer & Domestic & Domestic & Domestic & Domestic|ECO & Positive\\
Italy & http://spettacoli.tiscali.it/news/articoli/emiliano-masterplan-non-a-iniziato/ & 434 & Tiscali Spettacoli - Il canale dedicato al cinema, tv e gossip sempre aggiornato con news, interviste, photogallery, video e curiosità del dorato mondo dello spettacolo & Private/Non-Public & Online only & National & high = CP is most important issue in story (can also cover other issues) & Ineffective goal achievement & Balanced & Subnational & No myth & Mismanagement & Negative & National + Subnational & No myth & NA & NA & NA & NA & Italy & emiliano, il masterplan? non è iniziato & 2017-02-10 & politica di coesione & (ansa) - napoli, 10 feb - "il masterplan non è neanche cominciato, non abbiamo la disponibilità di questi soldi per poterli impegnare". lo ha detto il governatore della puglia, michele emiliano, a margine di un convegno sui fondi ue a napoli. "il fondo nazionale di sviluppo e coesione - ha aggiunto - viene saccheggiato dallo stato italiano quando hanno bisogno di soldi e soprattutto se la cassa non è ancora disponibile. i famosi patti non ci danno ancora i soldi disponibili per cominciare la spesa, per mesi abbiamo fatto propaganda, che deve essere concretizzata". emiliano ha fatto l'esempio della puglia: "siamo una regione - ha detto - che fa il suo dovere fino in fondo, spendiamo tutti i fondi, ma bisogna chiarire che purtroppo le politiche nazionali non funzionano. noi abbiamo i fondi europei e ci mancano quelli nazionali, per questo gli effetti sul pil dei fondi europei sono minimi. rischiamo per questi risultati scarsi che la politica di coesione salti perché molti stati non la vogliono. sarebbe un disastro". & 165 & high & High & Socio-Economic & Governance & NA & 2017-02-10 & 2017 & 2 & ECO
Frame & high-very high & National & <500 & -0.5155635 & -0.1667445 & -0.9728619 & 0.5015415 & 1.3050799 & 3.2 & 0.6840199 & -1.2389660 & Payer & Domestic & Domestic & Domestic & Domestic|ECO & Neutral\\
Italy & http://www.ansa.it/europa/notizie/rubriche/voceeurodeputati/2015/10/29/cozzolino-pd-a-strasburgo-votato-scorporo-investimenti\_c6a8e884-6fae-4d86-93c7-7b4c623eb1fb.html & 450 & ANSA.it & Private/Non-Public & Online only & National & medium = CP is important part of story & Political leverage & Negative & EU + National + Subnational & No myth & NA & NA & NA & NA & NA & NA & NA & NA & Italy & cozzolino (pd), a strasburgo votato scorporo investimenti & 2015-10-29 & fondi strutturali & (ansa) - bruxelles, 29 ott - "a strasburgo abbiamo scelto di votare un emendamento che chiedeva la revisione del patto di stabilità e crescita aprendo alla possibilità dello scorporo degli investimenti fatti nell'ambito dei fondi strutturali, una reale boccata di ossigeno per tanti amministratori locali". lo afferma in una nota andrea cozzolino, parlamentare europeo del pd. "ma l'emendamento purtroppo è stato bocciato, stritolato dall'asse tra i paesi ricchi del nord e quelli poveri dell'est. ma noi italiani abbiamo votato in favore. tutti. anzi no, quasi tutti. il ppe - forza italia, per intenderci - ha votato contro". "troppo facile, in italia - conclude cozzolino - parlare, riempirsi la bocca di proclami, inveire contro l'austerità e contro le rigidità di quel patto di stabilità e crescita locali, e restare in silenzio a bruxelles. la differenza sta nell'avere il coraggio di andare contro, anche contro il proprio gruppo s\&d, come abbiamo fatto noi". & 152 & medium & Medium & Power & NA & NA & 2015-10-29 & 2015 & 1 & POL
Frame & low-medium & National & <500 & -0.5155635 & -0.1667445 & -0.9728619 & 0.5015415 & 1.3050799 & 3.2 & 0.6840199 & -1.2389660 & Payer & Domestic & European & Mixed & Domestic|POL & Negative\\
\addlinespace
Italy & https://www.ilsole24ore.com/art/notizie/2018-09-19/reddito-cittadinanza-m5s-caccia-risorse-vuole-spending-review-europea-133649.shtml?uuid=AEEogd1F & 478 & Il Sole 24 ORE & Private/Non-Public & Online and Offline & National & medium = CP is important part of story & Jobs & Positive & EU + National + Subnational & No myth & Social justice & Balanced & EU + National & No myth & NA & NA & NA & NA & Italy & reddito di cittadinanza, il m5s a caccia di risorse vuole la spending review europea & 2018-09-20 & fondo sociale europeo & una spending review da 830 milioni per l'europa, che consenta di rimpinguare il fondo sociale europeo e recuperare risorse anche per il reddito di cittadinanza italiano. a invocarla sono gli europarlamentari del m5s, in un pacchetto di emendamenti alla proposta di bilancio 2019 depositati in commissione budget che andranno al voto la prossima settimana. certosino il lavoro dei pentastellati europei, in stretto raccordo con roma: per ognuna delle dieci sezioni di cui si compone il bilancio, sfornato dalla commissione lo scorso maggio e atteso al varco del parlamento ue in plenaria il 23 ottobre, sono stati individuati i possibili tagli. la sforbiciata più pesante riguarda l'europarlamento: la convinzione è che si potrebbero risparmiare 333 milioni usando l'accetta sulle "allowances" degli eurodeputati (stipendi, spese di viaggio, indennità transitorie e così via), ricalcolando le pensioni pre e post statuto, annullando i fondi per partiti e fondazioni (vecchio pallino) e riducendo le spese delle missioni per spingere alla sede unica rispetto alle tre attuali (strasburgo, bruxelles e lussemburgo), che da sempre i pentastellati chiedono di unificare. "ogni anno - spiegano - produce un costo che oscilla dai 156 milioni ai 204 milioni, circa il 10\% del bilancio totale del parlamento". ma gli emendamenti vanno oltre: prevedono tagli alle indennità transitorie e agli stipendi dei commissari, nonché a quello del presidente del consiglio europeo. chiedono di rinegoziare gli affitti dei "palazzi" europei. propongono la scure sulle spese per mobilio, parco auto e missioni in generale, incentivando il ricorso ai viaggi in classe economica. la revisione di spesa possibile è stimata appunto in circa 830 milioni. i cinque stelle suggeriscono di impiegarne circa 630 per i programmi che possono contribuire a garantire il reddito di cittadinanza. si tratta di 23 milioni da destinare all'iniziativa a favore dell'occupazione giovanile e di 606,9 milioni per aumentare il fondo sociale europeo (in bilancio previsto a quota 13,7 miliardi): 449,7 milioni per l'obiettivo "investimenti a favore della crescita e dell'occupazione" nelle regioni meno sviluppate, 26,3 milioni per quelle in transizione e 130,9 milioni per le più sviluppate. risorse preziose da cui i cinque stelle di governo sperano di poter attingere per contribuire a realizzare nel 2019, con la manovra d'autunno, l'oneroso cavallo di battaglia del reddito di cittadinanza su cui si sta consumando lo scontro con il ministro dell'economia giovanni tria. era stato il vicepremier luigi di maio, al suo debutto al consiglio ue lo scorso 21 giugno, ad aver indicato il fse come cruciale per migliorare benessere e coesione sociale nell'ue. il tema era stato già posto sul tavolo durante l'incontro con la commissaria marianne thyssen, da cui aveva ottenuto una parziale apertura sul possibile ricorso al fse non per finanziare il reddito di cittadinanza nel suo complesso, ma la riforma dei centri per l'impiego, premessa indispensabile per avviare la misura. "i nostri emendamenti - afferma l'eurodeputato marco valli - sono un piccolo passo verso un'ue meno distante dai cittadini e più equa, che dia l'esempio senza atteggiamenti ipocriti, limando gli sprechi e i privilegi che spesso deplora nelle raccomandazioni ai singoli paesi". la mossa dei pentastellati sul bilancio 2019 punta a coagulare il consenso di altri gruppi, socialisti compresi, intorno a un'idea di europa più attenta a benessere e crescita. riprende la filosofia della risoluzione approvata il 24 ottobre 2017 dal parlamento ue (proposta dalla m5s laura agea) secondo cui la commissione dovrebbe attivarsi affinché il 20\% del fse sia destinato alla lotta contro la povertà e l'esclusione sociale ed esaminare, nella revisione del regolamento dei fondi strutturali, "le possibilità di finanziamento per aiutare ciascuno stato membro a istituire un regime di reddito minimo, ove inesistente, o a migliorare il funzionamento e l'efficacia dei sistemi esistenti". infine, arriva quando ancora risuona l'eco dello scontro con bruxelles sul caso diciotti e della minaccia del governo gialloverde di porre il veto sul bilancio pluriennale 2021-2027. sullo sfondo, le elezioni europee di maggio 2019. con il m5s in cerca di una terza via tra i gruppi tradizionali e le prove di alleanza sovranista capitanata da salvini e orbán. non è un caso che in casa pentastellata sia stato molto apprezzato il piano anti-povertà lanciato da macron in francia, centrato proprio sul reddito universale di attivazione (rua). una misura che riequilibra a sinistra la strategia del presidente francese, cui i cinque stelle - nonostante le frizioni - non fanno mistero di guardare per alleanze future sui banchi di strasburgo. & 749 & medium & Medium & Socio-Economic & Socio-Economic & NA & 2018-09-20 & 2018 & 3 & ECO
Frame & low-medium & National & 500-1000 & -0.5155635 & -0.1667445 & -0.9728619 & 0.5015415 & 1.3050799 & 3.2 & 0.6840199 & -1.2389660 & Payer & Domestic & European & Mixed & Domestic|ECO & Positive\\
Italy & http://www.ansa.it/europa/notizie/rubriche/voceeurodeputati/2019/02/12/fondi-ue-damato-m5s-si-bocci-la-macrocondizionalita\_ce70767d-9ccb-4855-9838-d6dd0bd6b43b.html & 441 & ANSA.it & Private/Non-Public & Online only & National & medium = CP is important part of story & Political leverage & Negative & EU + National & No myth & NA & NA & NA & NA & NA & NA & NA & NA & Italy & fondi ue: d'amato (m5s), si bocci la macrocondizionalità - europa & 2019-02-12 & politica di coesione & (ansa) - strasburgo, 12 feb - "i falchi dell'austerity ci riprovano e vogliono sospendere i fondi europei a quei paesi che non rispettano gli assurdi diktat dell'austerità. questo è un voto contro l'italia e contro le regioni più bisognose dei fondi europei. siamo curiosi di vedere come voteranno tutti gli europarlamentari italiani, soprattutto quelli che fanno parte dei grandi gruppi: staranno dalla parte dei cittadini o eseguiranno gli ordini di scuderia di chi vuole imporre queste assurde regole?". così rosa d'amato, europarlamentare del movimento 5 stelle, interviene sul voto domani a strasburgo sul regolamento disposizioni comuni per il periodo 2021-2027. "noi siamo coerentemente contro e abbiamo presentato un emendamento di rigetto. ci appelliamo a tutti, votate il nostro emendamento", prosegue d'amato. "già nel 2017 avevamo respinto gli assalti dei gruppi dell'establishment con un nostro emendamento che si esprimeva sul futuro della politica di coesione chiaramente contro la macrocondizionalità. stavolta però i cittadini rischiano la sospensione dei fondi - prosegue -. stiamo parlando di miliardi di euro. diciamo no a questo ricatto contro i cittadini". & 177 & medium & Medium & Power & NA & NA & 2019-02-12 & 2019 & 3 & POL
Frame & low-medium & National & <500 & -0.5155635 & -0.1667445 & -0.9728619 & 0.5015415 & 1.3050799 & 3.2 & 0.6840199 & -1.2389660 & Payer & Domestic & European & Mixed & Domestic|POL & Negative\\
Italy & https://www.ilsole24ore.com/art/impresa-e-territori/2019-01-22/fondi-europei-il-parlamento-ue-conferma-conferma-macrocondizionalita-contrari-pd-e-m5s-204509.shtml?uuid=AE8SYvJH & 415 & Il Sole 24 ORE & Private/Non-Public & Online and Offline & National & medium = CP is important part of story & Political leverage & Negative & EU + National & No myth & Institutional bargaining over funding & Balanced & EU + National & No myth & NA & NA & NA & NA & Italy & fondi europei:\&\#8201;il parlamento ue conferma la conferma la \&\#8220;macrocondizionalità\&\#8221;. contrari pd e m5s & 2019-01-22 & fondi strutturali & scontro rimandato a febbraio, quando dovrebbe arrivare in plenaria la riforma dei regolamenti di sette fondi europei, dalla coesione alla migrazione, che contiene la controversa norma sulla "macrocondizionalità economica". la commissione sviluppo regionale del parlamento europeo ha dato il via libera alla bozza di risoluzione (25 sì, 1 no, 9 astensioni) provocando le critiche dei membri italiani, che hanno già annunciato battaglia in plenaria a suon di emendamenti per abolire la macrocondizionalità. in cosa consiste. così come già succede oggi, infatti, anche dal 2021 al 2027 gli stati membri potrebbero subire un blocco dell'erogazione dei fondi strutturali in caso di violazione delle regole di bilancio ue. una possibilità che ha scatenato le proteste dei deputati andrea cozzolino (pd) e rosa d'amato (movimento 5 stelle). cozzolino è stato l'unico a votare contro il testo finale, approvato dopo aver passato al vaglio oltre cento emendamenti. fra questi anche uno di compromesso pd-m5s che prevedeva la possibilità per gli stati, in condizioni particolari, di chiedere un'ulteriore flessibilità da usare per il cofinanziamento nazionale agli investimenti con fondi ue. "i vincoli della condizionalità sono inaccettabili e, su un tale arretramento dai valori e dai propositi dell'unione, il mio voto non può che essere contrario", ha dichiarato cozzolino in una nota in cui se la prende con il centrodestra: "oggi ha rivelato il suo vero volto: l'austerità a qualsiasi prezzo, che va a colpire l'intera popolazione per responsabilità non sue, affossare le aree più bisognose e compromettere gli investimenti strategici sul territorio. una mannaia sotto la quale finisce anche la possibilità di interrompere la procedura sanzionatoria in caso di circostanze economiche eccezionali". estende invece la critica anche ai socialisti europei la pentastellata d'amato, che si è astenuta sul voto finale. "dal ppe e parte dei socialisti arriva l'ennesimo voto contro i cittadini e le regioni più povere. i partiti dell'establishment hanno votato contro la possibilità di scorporare il cofinanziamento nazionale dal calcolo del deficit nell'ambito del patto di stabilità e crescita. per questo motivo, nonostante il positivo aumento dei fondi messi a disposizione, ci siamo astenuti sul voto finale". "l'unione europea si deve basare sul concetto di solidarietà che significa garantire alle regioni più povere e ai cittadini l'accesso ai fondi europei - ha aggiunto - ecco perché in plenaria ripresenteremo due emendamenti su macrocondizionalità e scorporo. la nostra battaglia per il cambiamento continua". & 401 & medium & Medium & Power & Power & NA & 2019-01-22 & 2019 & 3 & POL
Frame & low-medium & National & <500 & -0.5155635 & -0.1667445 & -0.9728619 & 0.5015415 & 1.3050799 & 3.2 & 0.6840199 & -1.2389660 & Payer & Domestic & European & Mixed & Domestic|POL & Negative\\
Italy & http://www.adnkronos.com/soldi/economia/2015/03/16/fondi-chiti-usarli-obiettivi-qualita-per-sviluppo-occupazione\_qxmNnADGAn4mgT3gHVTrVO.html & 437 & Adnkronos & Private/Non-Public & Online only & National & very high = CP is most important issue + CP is mentioned in title/headline & Improve governance & Balanced & EU + National & No myth & Mismanagement & Balanced & EU + National & 8.Mismanaged & Bureaucracy & Negative & EU + National & 10.Slow spend & Italy & fondi ue: chiti, usarli su obiettivi qualità per sviluppo e occupazione & 2015-03-16 & fondi strutturali & ''va dato atto che con le ripetute riprogrammazioni fatte da governo e regioni e grazie alla riduzione ottenuta dall'ue della quota di cofinanziamento nazionale, la programmazione 2007-2013 dei fondi europei sta chiudendosi, rispetto all'utilizzazione delle risorse, con un risultato ad ora positivo. non era scontato alcuni mesi fa''. lo ha detto il senatore vannino chiti, presidente della commissione politiche dell'unione europea, intervenendo alla relazione della corte dei conti europea per l'esercizio 2013. ''la commissione politiche ue del senato - ha aggiunto l'esponente del pd - sta seguendo con attenzione i passaggi relativi alla programmazione e gestione dei fondi strutturali. sono emerse lacune rilevanti, soprattutto nelle regioni ex obiettivo convergenza. la commissione europea ha sollecitato l'italia a rafforzare le capacità di progetto e di gestione. l'obiezione, più che fondata, può essere sinteticamente rappresentata da criticità come la dispersione di fondi in molte iniziative, diffuse territorialmente e non idonee a creare valore aggiunto; dai ritardi nella pubblicazione dei bandi; dalle procedure amministrative lunghe e faticose; da ritardi e slittamenti nei pagamenti". "la relazione della corte dei conti europea sull'esercizio 2013 rileva che le disfunzioni nei pagamenti sono riconducibili a alcuni grandi filoni, tra i quali l'indicazione di costi non ammissibili; progetti, attività o beneficiari non ammissibili; errori gravi negli appalti pubblici. e' necessario - ha concluso chiti - utilizzare meglio gli strumenti di controllo e di correzione esistenti ed è urgente una migliore organizzazione interna delle amministrazioni centrali e regionali, per utilizzare in modo compiuto le risorse su obiettivi di qualità, in grado di contribuire allo sviluppo e all'occupazione''. & 264 & very high & High & Governance & Governance & Governance & 2015-03-16 & 2015 & 1 & POL
Frame & high-very high & National & <500 & -0.5155635 & -0.1667445 & -0.9728619 & 0.5015415 & 1.3050799 & 3.2 & 0.6840199 & -1.2389660 & Payer & Domestic & European & Mixed & Domestic|POL & Neutral\\
Italy & http://notizie.tiscali.it/regioni/toscana/articoli/un-aiuto-chi-torna-casa-ospedale-00001/ & 453 & Tiscali & Private/Non-Public & Online only & National & medium = CP is important part of story & Public services & Positive & Subnational & No myth & NA & NA & NA & NA & NA & NA & NA & NA & Italy & un aiuto per chi torna casa da ospedale & 2018-02-13 & fondo sociale europeo & (ansa) - firenze, 12 feb - un buono servizio per assistere persone over 65 con una limitazione temporanea dell'autonomia o a rischio di non autosufficienza e disabili gravi nelle tre settimane successive al loro ritorno a casa dall'ospedale o da una struttura intermedia di cure. il sostegno è finanziato dal programma operativo regionale (por) del fondo sociale europeo (fse) ed è riservato a chi è residente in toscana. consente di attivare servizi infermieristici, fisioterapici e di assistenza di base a casa oppure di usufruire di un ricovero presso una rsa per massimo 12 giorni. & 94 & medium & Medium & Socio-Economic & NA & NA & 2018-02-13 & 2018 & 3 & ECO
Frame & low-medium & National & <500 & -0.5155635 & -0.1667445 & -0.9728619 & 0.5015415 & 1.3050799 & 3.2 & 0.6840199 & -1.2389660 & Payer & Domestic & Domestic & Domestic & Domestic|ECO & Positive\\
\addlinespace
Italy & http://notizie.tiscali.it/regioni/umbria/articoli/errani-importante-attenzione-ue-00001/ & 422 & Tiscali & Private/Non-Public & Online only & National & low = CP mentioned more times but NOT important part of story (mainly about others issues) & Infrastructure & Balanced & EU + Subnational & No myth & NA & NA & NA & NA & NA & NA & NA & NA & Italy & errani, importante attenzione ue & 2017-05-25 & fondi strutturali & (ansa) - norcia (perugia), 25 mag - "è importante per noi che ci sia un'attenzione dell'unione europea per costruire una strategia sulla ricostruzione e sull'utilizzo dei fondi strutturali che possono consentire a questo territorio di trovare un nuovo sviluppo, perché la nostra grande sfida non è solo quella di ricostruire meglio di prima ma anche quella di dare una nuova prospettiva a questi territori": così vasco errani, commissario straordinario per la ricostruzione, durante la visita della delegazione del comitato europeo delle regioni a norcia. "per costruire il futuro occorrono oltre ai servizi e alle scuole, che stiamo realizzando - ha detto errani -, anche il lavoro e le imprese e da questo punto di vista l'uso mirato dei fondi strutturali sono un'opportunità importante per offrire uno sviluppo di qualità". (ansa). & 131 & low & Low & Socio-Economic & NA & NA & 2017-05-25 & 2017 & 2 & ECO
Frame & low-medium & National & <500 & -0.5155635 & -0.1667445 & -0.9728619 & 0.5015415 & 1.3050799 & 3.2 & 0.6840199 & -1.2389660 & Payer & Domestic & European & Mixed & Domestic|ECO & Neutral\\
Italy & https://livesicilia.it/2019/01/04/fondi-ue-per-scuola-e-formazione-raggiunti-gli-obiettivi-di-spesa\_1024731/ & 461 & Live Sicilia & Private/Non-Public & Online only & Regional/Local & high = CP is most important issue in story (can also cover other issues) & Public services & Positive & National + Subnational & No myth & Jobs & Positive & National + Subnational & No myth & NA & NA & NA & NA & Italy & fondi ue per scuola e formazione "raggiunti gli obiettivi di spesa" & 2019-01-04 & fondo europeo di sviluppo regionale & la nota dell'assessore lagalla che spiega in dettaglio l'utilizzo dei fondi europei. "il 2018 termina in positivo per quanto riguarda gli obiettivi fissati dal programma operativo fse 2014/2020, in particolare relativamente a quanto assegnato all'istruzione e alla formazione professionale, come annunciato dal presidente musumeci". è l'incipit di una nota con cui l'assessore roberto lagalla "esprime la personale soddisfazione per il grande lavoro di squadra che ha permesso di certificare le risorse oltre la misura prevista per il 31 dicembre 2018". "accolgo con soddisfazione il compiacimento espresso dal ministro lezzi e dal presidente musumeci, che ringrazio per la fiducia concessa e la stretta collaborazione, grazie alla quale siamo riusciti a raggiungere gli obiettivi intermedi di spesa assegnati dal programma operativo fse relativi all'istruzione e alla formazione professionale. abbiamo attestato spesa per 118 milioni di euro andando addirittura oltre, di circa 21 milioni, a quanto previsto per la chiusura d'anno. questo significa che ci siamo allineati alle richieste dell'europa e non si perderanno somme rispetto alla dotazione finanziaria stabilita dal programma operativo". stando ai dati dell'assessorato, circa 54 milioni di euro sono stati utilizzati per favorire l'occupabilità, oltre 8 milioni per l'inclusione sociale e la lotta alla povertà, 41 milioni di euro sono andati in favore di azioni a sostegno dell'istruzione e della formazione professionale, in particolare per i percorsi di iefp e per l'alta formazione, mentre per potenziare la capacità istituzionale e l'assistenza tecnica, complessivamente sono stati impegnati circa 13 milioni di euro. e anche sul fesr, il fondo europeo di sviluppo regionale, è stato raggiunto e superato di circa 2 milioni di euro il target intermedio di spesa, certificando 22 milioni su interventi per il miglioramenti dell'edilizia scolastica siciliana. "questo risultato - continua lagalla - è stato raggiunto grazie ad un intenso lavoro di squadra che ha permesso di dare risposte al territorio, favorendo nuove opportunità di crescita. a riguardo, un ringraziamento particolare è rivolto al direttore gianni silvia che, dopo 40 anni di servizio di cui gli ultimi 4 trascorsi alla guida del dipartimento d'istruzione e formazione professionale, ha concluso la sua carriera raggiungendo obiettivi importanti e determinanti per il prosieguo dell'attività di questo assessorato. esprimo gratitudine e stima per il lavoro svolto con spirito di leale collaborazione e rispetto per le istituzioni". share venerdì 04 gennaio 2019 - 11:29 & 397 & high & High & Socio-Economic & Socio-Economic & NA & 2019-01-04 & 2019 & 3 & ECO
Frame & high-very high & Regional & <500 & -0.5155635 & -0.1667445 & -0.9728619 & 0.5015415 & 1.3050799 & 3.2 & 0.6840199 & -1.2389660 & Payer & Domestic & Domestic & Domestic & Domestic|ECO & Positive\\
Italy & http://www.agi.it/genova/notizie/regione\_liguria\_garantisce\_risorse\_per\_servizi\_dei\_centri\_impiego-201507311921-cro-rt10199 & 479 & AGI & Private/Non-Public & Online only & National & low = CP mentioned more times but NOT important part of story (mainly about others issues) & Jobs & Positive & Subnational & No myth & NA & NA & NA & NA & NA & NA & NA & NA & Italy & regione liguria garantisce risorse per servizi dei centri impiego & 2015-07-31 & fondo sociale europeo & (agi) - genova, 31 lug. - la regione liguria, rispettando gli impegni assunti con la citta' metropolitana di genova e le organizzazioni sindacali, dopo un'analisi approfondita del bilancio, garantisce la non interruzione dei servizi assicurati dai centri per l'impiego, tra cui il collocamento ordinario e il collocamento specifico per i disabili e i tirocini per i disabili psichici. gli assessori competenti, ilaria cavo (formazione professionale) e gianni berrino (lavoro e politiche dell'occupazione) hanno reperito risorse per 420mila euro, attinte dal fondo regionale occupazione, che verranno sbloccati con un'imminente delibera di giunta e ulteriori 256mila euro di residui del fondo sociale europeo che si renderanno disponibili con l'assestamento di bilancio. "si tratta di un ennesimo sforzo fatto dalla regione per rimediare ai disastri della legge delrio - affermano cavo e berrino - che, perseguendo un'utopistica spending review, ha dato risultati inutili e fatto lievitare i costi, scaricando sulle regioni e sui territori competenze gravose e personale. una politica che purtroppo il governo continua a perseguire: e' di ieri il progetto di trasferire alle regioni i centri per l'impiego senza dotarci di adeguate risorse. se pur in totale disaccordo con questa impostazione - hanno proseguito - abbiamo fatto questo sforzo per garantire ai cittadini di non essere privati dei servizi, pensando soprattutto ai disabili (che devono continuare a essere assistiti) e ai lavoratori che non devono perdere il posto. siamo consapevoli del fatto che il totale delle risorse messe in campo (676mila euro della regione e 939mila della provincia) coprono circa l'85 per cento del fabbisogno per garantire tutti i servizi dei centri per l'impiego da settembre a dicembre. ribadiamo, pero', che si tratta del massimo sforzo possibile. i servizi non saranno sospesi: al massimo si potra' pensare a una ragionata rimodulazione. il tutto sara' concertato giovedi' prossimo nella riunione con il sindaco metropolitano marco doria". la regione precisa che questi stanziamenti serviranno a "traghettare" i servizi tra settembre e dicembre in attesa che, a gennaio, venga aggiudicata la gara, gia' bandita dalla citta' metropolitana di genova e finanziata totalmente dalla regione liguria, per l'assegnazione della risorse della nuova programmazione 2014-2020 del fondo sociale europeo. "se pur messa in difficolta' da scelte altrui - hanno concluso cavo e berrino - la regione ha fatto la sua parte mettendo il cittadino al centro". (agi) ge2/vic & 386 & low & Low & Socio-Economic & NA & NA & 2015-07-31 & 2015 & 1 & ECO
Frame & low-medium & National & <500 & -0.5155635 & -0.1667445 & -0.9728619 & 0.5015415 & 1.3050799 & 3.2 & 0.6840199 & -1.2389660 & Payer & Domestic & Domestic & Domestic & Domestic|ECO & Positive\\
Italy & http://www.agi.it/politica/notizie/reddito\_cittadinanza\_maroni\_via\_sperimentazione\_lombardia-201505121245-pol-rt10115 & 489 & AGI & Private/Non-Public & Online only & National & medium = CP is important part of story & Social justice & Balanced & Subnational & No myth & NA & NA & NA & NA & NA & NA & NA & NA & Italy & reddito cittadinanza: maroni, via sperimentazione lombardia & 2015-05-12 & fondo europeo di sviluppo regionale & (agi) - milano, 12 mag. - via alla sperimentazione di un 'reddito di cittadinanza' per tutti i lombardi in difficolta' economica. l'annuncio e' stato fatto da roberto maroni, nel corso della presentazione della ripartizione dei fondi strutturali europei per il 2014-20. i due programmi - fondo europeo di sviluppo regionale (fesr) e fondo sociale europeo (fse) - hanno una dotazione complessiva di 1,940 miliardi di euro, e, rispetto al periodo precedente 2007-2013, le risorse destinate alla regione lombardia sono aumentate di circa 640 milioni di euro. "questi interventi - ha spiegato il governatore lombardo - sostengono un modello di crescita che punta sulla ricerca e sull'innovazione, che sono una delle vocazioni della lombardia. sul nostro territorio abbiamo 13 universita', 500 centri di ricerca, 18 irccs, 6 parchi tecnologici, la presenza delle principali societa' del settore. insomma, tutte le condizioni ideali affinche' la lombardia possa diventare la regione d'eccellenza in europa in fatto di innovazione e ricerca". riguardo al fondo sociale europeo maroni ha sottolineato come debba avere "anche la capacita' di ridurre la poverta'". "in un momento di crisi economica come e' quello che stiamo attraversando, ci sono fasce crescenti di popolazione che soffrono e non hanno la possibilita' di raggiungere i requisiti di sussistenza minima. per questo, per noi, il fse avra' anche la finalita' di ridurre la poverta', l'esclusione sociale e promuovere l'innovazione anche in campo sociale", ha continuato, "gli interventi che vanno in questa direzione voglio riunirli nel progetto del 'reddito di cittadinanza'". del 'reddito di cittadinanza', ha osservato il presidente, "si parla da tempo, sui giornali e nel dibattito politico: noi avvieremo in maniera concreta la sperimentazione di misure che consentano a tutti i cittadini di essere davvero tali. i lombardi che vivono in condizioni di poverta' o di esclusione sociale dovranno essere riscattati da questa condizione. voglio utilizzare risorse del fse e del nostro bilancio regionale, che ci diano modo di far partire presto la sperimentazione sul reddito di cittadinanza in maniera concreta". il presidente ha fatto sapere di aver gia' dato mandato agli assessori competenti "di studiare misure che vadano nella direzione del progetto di reddito di cittadinanza" e ha fatto sapere di voler coinvolgere anche "il terzo settore e il mondo del volontariato". "chiunque in lombardia si occupa di questi temi deve essere coinvolto - ha sottolineato -. voglio che partecipino da subito alla definizione di questo progetto speciale". a chi gli faceva notare che la proposta e' una bandiera del m5s, il governatore ha risposto: "per il movimento 5 stelle e' una bandiera, per noi sara' una cosa concreta. loro chiacchierano, hanno anche qualche buona idea, noi passeremo dalle parole ai fatti. la lombardia, prima in italia sperimentera' il reddito di cittadinanza". la proposta del governatore leghista ha subito incontrato il favore dell'opposizione, in particolare del movimento 5 stelle e del partito democratico. "a due giorni dalla nostra mobilitazione nazionale, con la marcia perugia-assisi per il reddito di cittadinanza, maroni si e' svegliato dal letargo e ha dichiarato di essere orientato a definire un progetto di reddito di cittadinanza lombardia. ancora una volta, esattamente come nel caso di piu' autonomia per la regione, le buone idee del movimento 5 stelle dettano l'agenda della politica lombarda e maroni e' costretto a rincorrere", ", ha commentato dario violi, capogruppo del movimento 5 stelle della lombardia. "noi siamo pronti a discuterne da subito", ha affermato, dal canto suo, il segretario lombardo del pd, alessandro alfieri. "l'introduzione del reddito di autonomia era nel nostro programma elettorale per la regione, dunque ben venga il confronto. attenzione, pero' - ha avvertito -: e' un obiettivo importante e ambizioso, che richiede serieta' e concretezza. non bastano gli annunci ne' operazioni boomerang come quella sui ticket". . & 617 & medium & Medium & Socio-Economic & NA & NA & 2015-05-12 & 2015 & 1 & ECO
Frame & low-medium & National & 500-1000 & -0.5155635 & -0.1667445 & -0.9728619 & 0.5015415 & 1.3050799 & 3.2 & 0.6840199 & -1.2389660 & Payer & Domestic & Domestic & Domestic & Domestic|ECO & Neutral\\
Italy & https://www.ecodibergamo.it/stories/ansa/budget-ue-in-2019-piu-fondi-a-ricerca-ten-t-e-erasmus\_1279737\_11/ & 429 & L'Eco di Bergamo & Private/Non-Public & Online and Offline & Regional/Local & medium = CP is important part of story & Institutional bargaining over funding & Factual & EU & No myth & NA & NA & NA & NA & NA & NA & NA & NA & Italy & budget ue: in 2019 più fondi a ricerca, ten-t e erasmus+ & 2018-05-24 & fondo sociale europeo & bruxelles - più risorse per infrastrutture nel quadro dei corridoi ten-t; per ricerca e innovazione; e per il programma erasmus+, mentre si mantengono stabili i fondi destinati alle politiche di coesione e all'agricoltura: emerge dalla proposta di bilancio dell'ue per il 2019, presentata dal commissario europeo al bilancio guenther oettinger, e che ora passerà al vaglio di parlamento e consiglio ue. in tutto la bozza di budget prevede 166 miliardi di euro in impegni (+3\% rispetto al 2018) e 149 in pagamenti (+3\% sul 2018), e si basa sul presupposto che il regno unito, dopo la brexit (mezzanotte del 29 marzo 2019), continui a contribuire. secondo quanto spiegato, i fondi per il sostegno alla crescita economica ammonteranno a quasi 80 miliardi di euro in impegni, con incrementi per una serie di programmi. in particolare, 12,5 miliardi di euro (+8,4\% rispetto al 2018) sono per ricerca e innovazione (horizon 2020), con 194 milioni, per una nuova impresa comune per un calcolatore ad alte prestazioni. 3,8 miliardi sono per le reti infrastrutturali (connecting eeurope facility), (+36,4\% sul 2018). e cresce anche il programma erasmus+, con 2,6 miliardi di euro (+10,4\% sul 2018). altri 233,3 milioni di euro sono per l'iniziativa per il lavoro dei giovani che vivono in regioni caratterizzate da un alto tasso di disoccupazione, cui si aggiungeranno finanziamenti dal fondo sociale europeo. i programmi della politica di coesione 2014-2020 manterranno la loro velocità di crociera nel 2019, con 57 miliardi di euro (+2,8\% rispetto al 2018), ed i finanziamenti per la politica agricola rimarranno stabili a quasi 60 miliardi (+1,2\% rispetto al 2018). ma oltre a consolidare gli sforzi del passato, il progetto di bilancio mira anche a sostenere nuove iniziative. per questo, 103 milioni di euro sono destinati al corpo europeo di solidarietà; 11 milioni all'istituzione dell'autorità europea del lavoro, che contribuirà a garantire un'equa mobilità dei lavoratori nel mercato interno e a semplificare la cooperazione tra le autorità nazionali; 40 milioni, per l'estensione del programma incentrato sull'attuazione delle riforme strutturali negli stati membri. inoltre, 245 milioni sono previsti per la predisposizione del programma europeo di sviluppo del settore industriale della difesa; e 150 milioni di euro per rafforzare la risposta a terremoti, incendi e altre calamità in europa mediante la costituzione di una riserva di mezzi di protezione civile a livello dell'ue ("resceu"), comprese attrezzature e squadre. infine, 5 milioni di euro sono destinati alla creazione della nuova procura europea destinata a perseguire i reati transfrontalieri, compresi frodi, riciclaggio di denaro e corruzione. & 434 & medium & Medium & Power & NA & NA & 2018-05-24 & 2018 & 3 & POL
Frame & low-medium & Regional & <500 & -0.5155635 & -0.1667445 & -0.9728619 & 0.5015415 & 1.3050799 & 3.2 & 0.6840199 & -1.2389660 & Payer & European & European & European & European|POL & Neutral\\
\addlinespace
Italy & http://www.ansa.it/sito/notizie/mondo/europa/2018/05/05/rinascita-forte-bard-da-complesso-militare-a-polo-cultura\_f8af9eb2-935d-4b60-9cac-b1c6ab6f4613.html & 408 & ANSA.it & Private/Non-Public & Online only & National & very low = CP mentioned once & Cultural heritage & Positive & National + Subnational & No myth & Infrastructure & Positive & National + Subnational & No myth & NA & NA & NA & NA & Italy & rinascita forte bard, da complesso militare a polo cultura - europa & 2018-05-05 & fondo europeo di sviluppo regionale & era una delle fortezze militari più conosciute dell'arco alpino, celebre per l'assedio da parte di napoleone nel 1800, e oggi è uno dei poli culturali più importanti del nord-ovest. riedificato dai savoia come complesso di sbarramento alla metà del 19/o secolo (le origini del primo presidio risalgono al 500 d.c.), il forte di bard sorge in una posizione strategica, sopra una gola dove scorre la dora baltea, all'inizio della valle d'aosta. e' stato dismesso nel 1975 dal demanio militare e acquisito dalla regione valle d'aosta nel 1990. il progetto di recupero dell'intero complesso e di rilancio del borgo medievale è stato attuato con il contributo finanziario del fondo europeo di sviluppo regionale (fesr) e del fondo di rotazione statale nell'ambito della riconversione delle aree in declino industriale. "l'italia è un paese che ha un patrimonio culturale immenso che rappresenta lo spirito dell'europa. la rinascita di questo complesso grazie ai fondi europei ridà anima alla storia, facendo rivivere mura antiche e restituendole alla comunità, diventando allo stesso tempo luogo di cultura e motore di sviluppo per il territorio", osserva beatrice covassi, capo della rappresentanza in italia della commissione europea. l'intervento - oltre 500 maestranze coinvolte, 153.737 metri cubi di terreno rimosso e 112.705 metri di cavi elettrici posati - è durato oltre 10 anni e nel 2006 è stata aperta al pubblico l'opera carlo alberto con il museo delle alpi. alla fine il 'rinnovato' forte di bard può contare su una superficie di 14.467 metri, di cui 3.600 metri quadrati di aree espositive e 2.036 metri quadrati di cortili interni, con 283 locali, 385 porte, 296 feritoie e 806 gradini. dal giorno dell'inaugurazione è diventato un polo culturale di richiamo internazionale, passando dalle 98.000 presenze del 2006 alle 312.000 del 2016. il forte si è segnalato per l'organizzazione di eventi e l'allestimento di importanti mostre come "i tesori del principe del liechtenstein" (48.358 visitatori), "marc chagall. la vie" (56.838), "yann arthus bertrand. dalla terra all'uomo" (45.236), "montserrat. opere maggiori dell'abbazia" (48.276), "pablo picasso. il colore inciso" (34.836) e "steve mccurry. mountain men" (49.249). a ciò si aggiungono progetti di produzione artistica, tra cui quelli con il fotografo statunitense steve mccurry, il documentarista e fotografo francese yann arthus-bertrand, magnum photos e il fotografo ceco josef koudelka. la fortezza ha attirato anche l'attenzione di hollywood: nel 2015 sono state girate alcune scene del kolossal della marvel 'avengers: age of ultron'. "in questi dieci anni il forte di bard - spiega il presidente del comitato di indirizzo del complesso, sergio enrico - è diventato davvero 'forte'. il punto debole è che bisogna fare un secondo passo cercando di sfruttare al meglio sul territorio queste potenzialità. abbiamo un luogo conosciuto a livello internazionale, dobbiamo fare in modo che sia una calamita che attiri persone sul territorio". secondo il neo direttore della struttura, maria cristina ronc, "il forte di bard si è già imposto come polo culturale nel nord-italia e ora mi piacerebbe mettere in rete tutti i forti, a partire dalla liguria fino al friuli, con i quali abbiamo molte cose in comune". "ma soprattutto - conclude - punto a fare rete con i beni culturali che sono diffusi sul territorio valdostano". & 556 & very low & Low & Socio-Economic & Socio-Economic & NA & 2018-05-05 & 2018 & 3 & ECO
Frame & v.low & National & 500-1000 & -0.5155635 & -0.1667445 & -0.9728619 & 0.5015415 & 1.3050799 & 3.2 & 0.6840199 & -1.2389660 & Payer & Domestic & Domestic & Domestic & Domestic|ECO & Positive\\
Italy & https://livesicilia.it/2018/09/02/fondi-europei-per-la-sicilia-storia-di-un-grande-spreco\_992381/ & 406 & Live Sicilia & Private/Non-Public & Online only & Regional/Local & very high = CP is most important issue + CP is mentioned in title/headline & Ineffective goal achievement & Negative & EU + National + Subnational & No myth & Mismanagement & Negative & EU + National + Subnational & No myth & Fraud/Corruption & Negative & EU + National + Subnational & No myth & Italy & i fondi europei e la sicilia storia di un grande spreco & 2018-09-03 & fondi strutturali & si rischia il disimpegno delle somme non impiegate, che attualmente ammontano a circa 4,47 miliardi. in una fase di grave contrazione delle risorse pubbliche i fondi strutturali europei costituiscono la quasi totalità degli investimenti in sicilia, destinati ad obiettivi fondamentali come rafforzare la ricerca, lo sviluppo tecnologico e l'innovazione, l'accesso alle tecnologie dell'informazione e della comunicazione, promuovere la competitività delle piccole e medie imprese, il settore agricolo, della pesca e dell'acquacoltura, la prevenzione e la gestione dei rischi ambientali e l'uso efficiente delle risorse naturali, incentivare sistemi di trasporto sostenibili ed eliminare le carenze nelle principali infrastrutture di rete; promuovere l'occupazione e la mobilità dei lavoratori, l'inclusione sociale e combattere la povertà; investire nelle competenze, nell'istruzione e nell'apprendimento permanente, promuovere un'amministrazione pubblica efficiente. paesi come malta e la spagna, attraverso l'efficiente utilizzo dei fondi europei, hanno realizzato infrastrutture didattiche, stradali, marittime e ferroviarie, investimenti tesi a consolidare il legame tra il mondo accademico e l'industria, strutture per il settore del turismo e restauro di siti storici di alto valore turistico, finanziato incentivi alla produzione di energia pulita e pratiche di efficienza energetica volte a ridurre l'impatto di elettricità e i consumi, interventi per ridurre la quantità di rifiuti e deviare i residui verso gli impianti di riciclaggio. in italia ed in sicilia, invece, la corte dei conti ha rilevato che la spesa di queste risorse "è in allarmante ritardo" e si rischia il disimpegno da parte delle istituzioni comunitarie delle somme non impiegate, che attualmente ammontano a circa 4,47 miliardi. al di là delle percentuali di spesa l'utilizzo dei fondi comunitari in questi anni si è caratterizzato per numerose infrazioni, irregolarità, frodi e per il diffuso ricorso a vari espedienti che hanno creato spesa virtuale senza garantirne l'effettività e l'efficienza: dalla candidatura di progetti di importo eccedente la dotazione finanziaria per garantire la sostituzione di quelli eventualmente bocciati, al riutilizzo di progetti originariamente finanziati da altri fondi (cd progetti sponda o retrospettivi). spesso, inoltre, le ingenti risorse comunitarie, anziché essere destinate agli investimenti finalizzati a colmare il gap infrastrutturale e a sostenere lo sviluppo territoriale siciliano creando opportunità di sviluppo e di creazione di reddito consolidato e sostenibile, sono state dirottate verso il conseguimento degli obiettivi di finanza pubblica oppure adoperate come ammortizzatore sociale in grado di produrre reddito congiunturale. tanto che alla fine del quarto ciclo di programmazione comunitaria la sicilia è penultima per pil pro capite, dopo la calabria, e una delle ultime in europa per percentuali di occupati. la corte dei conti, inoltre, ha rilevato che la spesa è stata in gran parte destinata a soddisfare esigenze contingenti con una distribuzione estremamente frammentata (quasi a 'pioggia') delle risorse che, "salvo poche eccezioni di avvisi e bandi destinati ad offrire opportunità di crescita al tessuto imprenditoriale della regione", hanno finanziato investimenti disordinati, nessuna grande opera, pochi interventi sul welfare e sui diritti sociali che hanno creato poco lavoro, povero e precario, e si sono per lo più rivelate particolarmente inefficaci in relazione al contesto economico siciliano caratterizzato da ridotte dimensioni della maggior parte delle aziende, bassa capitalizzazione, difficoltà di accesso al credito, scarsa adattabilità alle mutabili dinamiche del mercato. ciò, peraltro, contribuisce ad incrementare il gap della sicilia rispetto al resto del paese dato che, paradossalmente, le regioni più virtuose nella gestione dei fondi europei sono generalmente quelle più sviluppate che meno avrebbero bisogno delle politiche di coesione". le cause delle difficoltà di impiego dei fondi strutturali sono piuttosto note: la frammentazione dei programmi in una rilevante quantità di obiettivi tematici, linee di intervento, azioni affidati ad una struttura organizzativa complessa composta da nuclei di valutazione, tavoli tecnici, responsabili per obiettivo tematico, dirigenti di servizio afferenti, responsabili di azioni e sotto-azioni, autorità di coordinamento, audit e certificazione distribuiti in diversi enti, assessorati e dipartimenti; la scarsa integrazione tra strutture burocratiche, nonché tra obiettivi e azioni operative; la moltiplicazione delle procedure e degli oneri amministrativi, la scarsa qualità di molti progetti, più legati a logiche opportunistiche di acquisizione di fondi che alla qualità delle azioni, l'inefficacia dei controlli sul rispetto dei tempi e dei requisiti prescritti, norme contabili complesse ed articolate, stringenti vincoli finanziari, una disciplina degli appalti e dei contratti pubblici tra le più complesse in europa, che richiede dai cinque ai sei anni per il completamento di opere infrastrutturali, anche di valore inferiore ai 5 milioni di euro, la difficile congiuntura economica, la fragilità organizzativa e finanziaria degli enti locali, le frequenti modifiche nell'assetto organizzativo delle strutture regionali coinvolte nell'esecuzione dei programmi e l'affidamento dell'attività istruttoria di ammissione al finanziamento agli organismi intermedi, soggetti esterni alla regione, che secondo la corte dei conti "determina alti costi non sempre giustificati dalla qualità delle prestazioni rese". il patto per lo sviluppo del 2016 ed il recente protocollo stipulato con lo stato prevedono la messa a sistema delle risorse ordinarie ed aggiuntive, nazionali ed europee per la realizzazione di infrastrutture ed interventi negli ambiti dell' ambiente, dello sviluppo economico, turismo e cultura, sicurezza, legalità e vivibilità del territorio, nonché la collaborazione del governo centrale nella sorveglianza del rispetto dei tempi e della rispondenza delle opere a quanto previsto. ma per risolvere le criticità che rendono inefficiente l'utilizzo dei fondi europei non si può prescindere da interventi strutturali sull'intera "filiera", dalla programmazione all'organizzazione della burocrazia, dall'attività amministrativa all'assetto e all'efficacia dei controlli: bisogna concentrare la programmazione prevalentemente su grandi interventi strategici articolati in obiettivi e risultati ben identificati e misurabili; potenziare le strutture regionali che si occupano di fondi europei, migliorare la qualità della valutazione d'impatto delle politiche pubbliche e dei programmi operativi; razionalizzare la struttura amministrativa ed apprestare efficaci forme di coordinamento tra gli apparati burocratici, eliminare adempimenti e controlli inefficaci, garantire la corretta applicazione degli strumenti di semplificazione e delle norme sulla valutazione delle performance, integrare meglio le politiche ordinarie con quelle sostenute dai fondi strutturali, prevedere forme di assistenza ai soggetti coinvolti nella elaborazione dei progetti. share domenica 02 settembre 2018 - 15:54 & 1016 & very high & High & Socio-Economic & Governance & Governance & 2018-09-03 & 2018 & 3 & ECO
Frame & high-very high & Regional & +1000 & -0.5155635 & -0.1667445 & -0.9728619 & 0.5015415 & 1.3050799 & 3.2 & 0.6840199 & -1.2389660 & Payer & Domestic & European & Mixed & Domestic|ECO & Negative\\
Italy & http://corrierealpi.gelocal.it/belluno/cronaca/2017/03/19/news/nelle-scuole-arriva-internet-senza-fili-1.15061082 & 427 & Corriere delle Alpi & Private/Non-Public & Online and Offline & Regional/Local & very low = CP mentioned once & Public services & Positive & National + Subnational & No myth & NA & NA & NA & NA & NA & NA & NA & NA & Italy & nelle scuole arriva internet senza fili  - cronaca - corriere delle alpi & 2017-03-20 & fondi strutturali & il comune lancia il piano di cablaggio delle elementari e delle medie per portare il collegamento in tutte le classi feltre. arriva il wi-fi a scuola, sia alle elementari che alle medie. il comune fornisce il cablaggio per la rete internet di tutte le classi, l'istituto comprensivo di feltre ci mette il collegamento con i soldi ottenuti grazie al progetto per il potenziamento degli ambienti digitali, essendo stato selezionato nel bando di fondi pon (il programma operativo nazionale del ministero dell'istruzione, finanziato dai fondi strutturali europei) per le reti wifi nelle scuole. il polo di feltre guidato dalla preside viviana fusaro si è portato a casa 15 mila euro per l'innovazione tecnologica, che permetterà agli insegnanti di aggiornare il registro elettronico sul tablet acquistato con i 500 euro del bonus docenti di renzi, senza aspettare di tornare a casa per collegarsi a internet senza fili. mandato in pensione il registro cartaceo, la rivoluzione della scuola 2.0 è incompleta se mancano i cavi per l'allestimento dei collegamenti. "l'intervento è in corso", dice il sindaco paolo perenzin. "come comune abbiamo preferito fare il cablaggio delle aule in tutti i plessi scolastici, che è l'operazione più onerosa con un investimento sull'ordine dei 24 mila euro, mettendo la scuola nelle condizioni di poter scegliere che tipo di copertura internet adottare". di portare la scuola nell'era super tecnologica si sente parlare da tempo, ma il conto alla rovescia per l'anno zero adesso è partito e si può guardare a un ambiente di apprendimento digitale applicato all'insegnamento per stare al passo con i tempi, quelli appunto delle generazione tecnologica e multimediale tra tablet, lavagne interattive, banchi trapezoidali modulari e didattica innovativa. "ci siamo impegnati con le scuole a portare la connessione, integrando l'iniziativa dell'istituto comprensivo che si occuperà della distribuzione interna", ribadisce l'assessore ai beni comuni valter bonan. "le elementari di nemeggio e di foen hanno già il cablaggio, a mugnai va completato, mentre l'intervento più estensivo riguarda la scuola del boscariz e la media rocca. alle scuole elementari vittorino da feltre si interverrà una volta terminate le lavorazioni attualmente in corso sull'edificio", spiega. "nei prossimi mesi verrà messa a disposizione da telecom la connessione per l'integrazione con il wi-fi. tutte le scuole adesso sono appoggiate alla rete comunale". aspettando i tempi operativi della realizzazione del cablaggio e in attesa della rete wi-fi definitiva per la quale ha ottenuto il finanziamento, la media rocca è stata autorizzata ad installare un apparecchio provvisorio. raffaele scottini & 428 & very low & Low & Socio-Economic & NA & NA & 2017-03-20 & 2017 & 2 & ECO
Frame & v.low & Regional & <500 & -0.5155635 & -0.1667445 & -0.9728619 & 0.5015415 & 1.3050799 & 3.2 & 0.6840199 & -1.2389660 & Payer & Domestic & Domestic & Domestic & Domestic|ECO & Positive\\
Italy & http://www.ansa.it/europa/notizie/rubriche/voceeurodeputati/2018/06/19/agricoltura-comifipiu-risorse-a-settore-meno-a-migranti\_ddfa1ed8-bf15-45ea-ba1c-b35ff439ebf9.html & 433 & ANSA.it & Private/Non-Public & Online only & National & medium = CP is important part of story & Institutional bargaining over funding & Balanced & EU + National & No myth & Social awareness/inclusion & Balanced & EU + National & No myth & NA & NA & NA & NA & Italy & agricoltura: comi(fi),più risorse a settore, meno a migranti - la voce degli eurodeputati - ansa europa & 2018-06-19 & fondo sociale europeo & (ansa) - bruxelles, 19 giu - "rimodulare la distribuzione dei fondi destinati all'italia nel bilancio ue. più risorse per l'agricoltura e meno per l'immigrazione, come richiesto dal ministro degli interni matteo salvini". lo propone l'europarlamentare lara comi (fi), che ha incontrato a bruxelles il commissario ue all'agricoltura phil hogan insieme al governatore della lombardia attilio fontana e dall'assessore all'agricoltura fabio rolfi. "siccome l'italia nel nuovo bilancio avrà maggiori risorse per i fondi di coesione e per l'accoglienza dei migranti, penso possa essere opportuno che queste vengano redistribuite sotto forma di finanziamenti diretti all'agricoltura", afferma l'eurodeputata. e' "una soluzione su cui abbiamo registrato apertura da parte del commissario hogan e che ora toccherà al governo italiano negoziare e ratificare", sottolinea comi. riguardo alla possibilità che la proposta essere in contrasto con quella di finanziare il reddito di cittadinanza attraverso il fondo sociale europeo, così come chiesto dal m5s, comi risponde: "dal programma di governo noi avevamo capito che c'erano le coperture finanziarie. spero che non siano sparite in poche settimane. la questione non é un problema europeo".(ansa). & 187 & medium & Medium & Power & Socio-Economic & NA & 2018-06-19 & 2018 & 3 & POL
Frame & low-medium & National & <500 & -0.5155635 & -0.1667445 & -0.9728619 & 0.5015415 & 1.3050799 & 3.2 & 0.6840199 & -1.2389660 & Payer & Domestic & European & Mixed & Domestic|POL & Neutral\\
Romania & http://www.mediafax.ro/politic/cretu-tarile-ue-au-la-dispozitie-un-buget-consistent-pentru-finantarea-proiectelor-vizand-refugiatii-14743032?utm\_source=feedburner\&utm\_medium=feed\&utm\_campaign=Feed\%253A\%2BMediafaxPolitic\%2B\%2528Mediafax\%2B-\%2BPolitic\%2529 & 558 & Mediafax.ro & Private/Non-Public & Online only & National & very high = CP is most important issue + CP is mentioned in title/headline & Social awareness/inclusion & Positive & EU & No myth & NA & NA & NA & NA & NA & NA & NA & NA & Romania & creţu: ţările ue au la dispoziţie un buget consistent pentru finanţarea proiectelor vizând refugiaţii & 2015-09-23 & fondul european de dezvoltare regională & "prioritatea noastră este să acţionăm în criza refugiaţilor ca o uniune. comisia europeană s-a pronunţat în mod constant şi continuu pentru un răspuns european coordonat", a precizat corina creţu, într-o declaraţie făcută înaintea reuniunii de miercuri a colegiului comisarilor. ea a menţionat că au fost adoptate "măsuri operaţionale concrete pentru a preciza care este calea de urmat", iar preşedintele ce, jean-claude juncker, va prezenta aceste măsuri şefilor de stat şi de guvern miercuri seară, când are loc consiliul european extraordinar. "în calitate de comisar european pentru politica regională, consider că este extrem de important să subliniez faptul că statele membre au deja la dispoziţie, în cadrul noilor programe pentru perioada 2014-2020, un buget consistent cu ajutorul căruia pot să finanţeze o gamă largă de proiecte menite să susţină integrarea eficientă a migranţilor legali şi a refugiaţilor. mă refer în mod concret la investiţii în domenii precum: integrarea socială, sănătatea, educaţia, infrastructura necesară pentru îngrijirea copiilor, precum şi revitalizarea zonelor urbane defavorizate. acolo unde este cazul, astfel de măsuri ar putea fi cuprinse într-un program integrat de dezvoltare urbană", a precizat corina creţu. ea a menţionat că în prezent este în analiză cum se poate contribui cu ajutorul fondurilor structurale la integrarea socială a migranţilor legali şi a refugiaţilor. "cu o alocare de 20 de miliarde de euro pentru toate statele membre în perioada de programare 2014-2020, fondul european de dezvoltare regională (fedr) joacă un rol important în această privinţă. mai mult decât atât, aşa cum am anunţat deja, analizăm cum ar putea contribui şi fondurile europene din cadrul programelor de cooperare transfrontalieră şi a strategiilor macro-regionale la găsirea unor soluţii suplimentare pentru criza migranţilor", a mai spus comisarul european. potrivit corinei creţu, rămâne la latitudinea statelor membre să identifice exact care sunt tipurile de investiţii de care au mai multă nevoie în contextul actual şi pe care le pot realiza prin intermediul fondurilor structurale. "în calitate de comisar european pentru politică regională voi face tot ce îmi stă în putinţă, în limita regulamentului actual, pentru ca împreună cu serviciile mele, să găsim cea mai bună soluţie pentru a ajuta statele membre să se adapteze mai uşor la noul context şi să răspundă cât mai rapid provocărilor generate de fenomenul migraţiei", a menţionat creţu. conținutul website-ului www.mediafax.ro este destinat exclusiv informării și uzului dumneavoastră personal. este interzisă republicarea conținutului acestui site în lipsa unui acord din partea mediafax. pentru a obține acest acord, vă rugăm să ne contactați la adresa vanzari@mediafax.ro. & 422 & very high & High & Socio-Economic & NA & NA & 2015-09-23 & 2015 & 1 & ECO
Frame & high-very high & National & <500 & 1.0255497 & 0.6327204 & -1.4131940 & 0.5015415 & -0.9523600 & 0.0 & -1.0024021 & 0.3314703 & Recipient & European & European & European & European|ECO & Positive\\
\addlinespace
Romania & http://www.romanialibera.ro/economie/fonduri-europene/problema-achizitiilor-publice-poate-lasa-romania-fara-fonduri-ue-393162 & 544 & RomaniaLibera.ro & Private/Non-Public & Online and Offline & National & high = CP is most important issue in story (can also cover other issues) & Fraud/Corruption & Balanced & EU & No myth & NA & NA & NA & NA & NA & NA & NA & NA & Romania & problema achizițiilor publice poate lăsa românia fără fonduri ue & 2015-09-16 & fondurile structurale şi de investiţii europene & curtea de conturi europeană recomandă comisiei europene ca, din 2017, să înceteze plățile din fondurile europene către românia și alte state membre dacă acestea nu-și fac ordine în contractele de achiziţii publice. raportul auditorilor vorbește despre atribuirea arbitrară a contractelor cu bani publici. ultimul raport al curții de conturi europene indică faptul că au fost detectate erori legate de achizițiile publice în aproximativ 40\% din proiectele pentru care achizițiile publice au făcut obiectul unui audit în contextul elaborării rapoartelor anuale ale curții pentru exercițiile 2009-2013. este vorba despre concurență neloială și atribuirea contractelor unor ofertanți care nu prezentaseră neapărat cea mai bună ofertă. printre țările vizate se află și românia. însă, de această dată, la capitolul nereguli în atribuirea contractelor cu bani publici, țara noastră a fost depășită de marea britanie, cehia, spania și italia, state în care curtea a detectat un număr ridicat de erori legate de achiziții publice în perioada 2009-2013. iar sumele care au intrat în alte buzunare decât ar fi trebuit sunt destul de importante, ținând cont că, în acest interval de timp, domeniului politicii regionale i s-au alocat fonduri în valoare de 349 de miliarde de euro prin intermediul fondului european de dezvoltare regională, al fondului de coeziune și al fondului social european. care au fost cele mai grave probleme constatate în urma auditului, în procesul de achiziții publice din românia și care trebuie urgent remediate? până la închiderea ediției nu am primit răspuns la această întrebare adresată curții de conturi europene. în schimb, raportul amintește de cele 4 proceduri de infringement (încălcarea dreptului comunitar) deschise de comisia europeană împotriva româniei în materie de achiziții publice, în acest clasament, grecia ocupând detașat prima poziție cu 18 proceduri. de asemenea, auditorii de la luxemburg au remarcat că, la începutul anului 2015, un număr de state membre nu îndeplineau încă, conform evaluării comisiei europene, condițiile stabilite în ceea ce privește achizițiile publice pentru a putea accesa fondurile structurale și de investiții europene în perioada de programare 2014-2020. printre aceste țări se numără și românia. aceste cerințe sunt considerate a fi condiții prealabile necesare pentru o utilizare eficace și eficientă a sprijinului acordat de uniunea europeană. de aceea, în cazul în care aceste condiții nu sunt îndeplinite până la sfârșitul anului 2016, curtea recomandă comisiei să suspende plățile către statele membre în cauză pe perioada 2014-2020, până la remedierea deficiențelor. în ceea ce privește autoritatea de concurență din românia, aceasta a sancționat cu amenzi usturătoare cele patru firme care au fost găsite vinovate de trucarea licitațiilor organizate de transgaz mediaș, fiind de altfel și singurul caz depistat de consiliul concurenței în care au fost implicate și fonduri -europene. astfel, sc moldocor sa a fost amendată cu 500.000 de euro, iar t.m.u.c.b. s.a cu 2,5 milioane de euro. motivul - cele două companii s-au înțeles în scopul participării cu oferte trucate la procedura de achiziţie publică "conductă de transport gaze 20'' giurgiu ruse" organizată în anul 2011. se modifică legislația legea achizițiilor publice va fi modificată în curând, iar propunerile au fost postate pe site-ul ministerului finanțelor, fiind în dezbatere publică până la sfârșitul lunii august. "actualul proiect de lege al achizițiilor publice lasă portițe prin utilizarea excesivă a unor sintagme - dacă e cazul, când e cazul, de obicei etc. trebuie spus clar - ce, când, cum? astfel de interpretări sunt periculoase pentru toți și trebuie scoase din lege. de asemenea, s-a eliminat ideea atribuirii unui contract pe criteriul prețului minim. acum s-a introdus termenul cost-beneficiu, dar nu există nici un criteriu pentru a se face diferența între oferte", a precizat, pentru "românia liberă", -cristian pârvan, secretarul general al asociației oamenilor de afaceri din românia. de asemenea, am solicitat ministrului fondurilor europene să ne precizeze care au fost propunerile pentru modificarea legislației, făcute în baza experienței din ultimii ani în procesul de absorbție a fondurilor ue, dar nu am primit nici un răspuns până la închiderea ziarului. & 664 & high & High & Governance & NA & NA & 2015-09-16 & 2015 & 1 & POL
Frame & high-very high & National & 500-1000 & 1.0255497 & 0.6327204 & -1.4131940 & 0.5015415 & -0.9523600 & 0.0 & -1.0024021 & 0.3314703 & Recipient & European & European & European & European|POL & Neutral\\
Romania & http://www.newsbucovina.ro/social/126416/modernizarea-serviciului-de-ocupare-a-fortei-de-munca-cu-fonduri-ue-clientii-ajofm-suceava-vor-avea-acces-la-propriul-dosar & 497 & NewsBucovina & Private/Non-Public & Online only & Regional/Local & low = CP mentioned more times but NOT important part of story (mainly about others issues) & Public services & Positive & EU + National + Subnational & No myth & NA & NA & NA & NA & NA & NA & NA & NA & Romania & modernizarea serviciului de ocupare a forţei de muncă cu fonduri ue. clienţii ajofm suceava vor avea acces la propriul dosar & 2015-04-17 & fondul social european & reprezentanţi ai agenţiilor judeţene de ocupare a forţei de muncă din suceava, ilfov şi ialomiţa au participat, vineri, la suceava la lansarea proiectului "fluxuri moderne în serviciul public de ocupare", ce are ca obiectiv modernizarea serviciului de ocupare a forţei de muncă, valoarea proiectului fiind de peste 8,6 milioane de lei. directorul ajofm suceava, mirela adomnicăi, a declarat, vineri, că avându-se în vedere dinamica de pe piaţa muncii din ultima perioadă, s-a constatat că numărul intrărilor şi ieşirilor din evidenţele instituţiei a crescut, ceea ce a determinat o sporire a activităţii pe care o desfăşoară serviciul public de ocupare şi la micşorarea timpului acordat de către funcţionarii ajofm suceava clienţilor, dar şi o aglomerare în spaţiile în care sunt prestate aceste servicii. ea a spus că nevoia de pe piaţa muncii a impus soluţia arhivării electronice, care se va realiza în cadrul acestui proiect lansat, vineri, la suceava. "arhivare electronică va permite, în viitor, ca toţi clienţii să aibă acces direct la propriul dosar, la informaţiile care privesc persoana şi serviciile de care doreşte să beneficieze din partea instituţiei. avem parteneri în acest proiect ajofm ilfov, ajofm ialomiţa, deci implementăm în oglindă acest proiect care va presupune, în final, un sistem profesionist de scanare, astfel încât într-un timp cât mai scurt ne dorim avem scanate toate documentele care se creează în cadrul instituţie şi implementăm un sistem de eliberare a unor bilete cu număr de ordine, astfel încât orice client care doreşte să beneficieze de un serviciu din partea unui funcţionar din cadrul instituţiei să-şi poată obţine un număr de ordine care să-i permită estimarea timpului până la care va putea beneficia de acel serviciu, o jumătate de oră-o oră, astfel încât să nu rămână fizic în cadrul instituţiei, ci să-şi poată rezolva şi alte probleme, astfel încât şi serviciile noastre să poată fi prestate pentru clienţii noştri în mod civilizat, eficient, transparent", a explicat adomnicăi. directorul ajofm suceava a spus că proiectul are un buget de peste 8,6 milioane de lei şi va fi implementat timp de zece luni de zile, iar în cadrul proiectului vor fi achiziţionate sisteme profesionale de scanare, două servere şi un sistem de eliberare a biletelor cu număr de ordine. adomnicăi a precizat că 21 de funcţionari ai serviciului public de ocupare, responsabili de arhivarea şi gestionarea dosarelor, din ajofm suceava, ajofm ialomiţa şi ajofm ilfov, vor urma cursuri de formare profesională, iar transferul de know-how se va face, inclusiv, în cadrul acestui proiect către o parte dintre funcţionarii instituţiei. "bugetul acoperă doar un număr limitat de dosare care pot fi scanate, dar acest număr limitat este determinat şi de perioada scurtă de implementare pe care o avem, de doar zece luni, aceste proiecte fiind implementate, de regulă, în trei ani, dar ceea ce este foarte important este că rămânem cu aceste sisteme profesionale de scanare, astfel încât, de acum încolo, orice document creat în cadrul instituţiei va fi scanat şi arhivat şi sub această formă", a mai spus adomnicăi. proiectul "fluxuri moderne în serviciul public de ocupare", ce are ca obiectiv îmbunătăţirea şi diversificarea serviciilor furnizate cetăţenilor şi instituţiilor, prin simplificarea accesului la serviciile oferite şi fluidizarea schimbului de informaţii şi documente şi accesul la informaţii cuprinse în documente, este cofinanţat din fondul social european, prin programul operaţional sectorial dezvoltarea resurselor umane 2007-2013. & 561 & low & Low & Socio-Economic & NA & NA & 2015-04-17 & 2015 & 1 & ECO
Frame & low-medium & Regional & 500-1000 & 1.0255497 & 0.6327204 & -1.4131940 & 0.5015415 & -0.9523600 & 0.0 & -1.0024021 & 0.3314703 & Recipient & Domestic & European & Mixed & Domestic|ECO & Positive\\
Romania & http://alba24.ro/foto-adr-centru-nou-acord-de-parteneriat-intre-regiunea-centru-si-landul-brandenburg-proiecte-vizate-578442.html & 586 & alba24.ro & Private/Non-Public & Online only & Regional/Local & medium = CP is important part of story & Environment/green/low-carbon & Positive & Subnational & No myth & Social awareness/inclusion & Factual & Subnational & No myth & NA & NA & NA & NA & Romania & foto adr centru: nou acord de parteneriat între regiunea centru și landul brandenburg. proiecte vizate & 2017-06-22 & fondurile structurale & pornind de la interesul comun al ministerului pentru justiție, afaceri europene și protecția consumatorului din landul brandenburg și al agenției pentru dezvoltare regională centru, privind dezvoltarea continuă a colaborării bilaterale și pentru extinderea pe termen lung a schimbului de experiențe, delegația acestui minister din germania s-a aflat timp de trei zile în regiunea centru, pentru a discuta cu oficialitățile române despre stadiul actual și rezultatele parteneriatului dintre cele două regiuni, dar și despre posibilitățile de extindere a cooperării. în cadrul acestei vizite a avut loc evenimentul de semnare a noului acord de cooperare dintre ministerul de justiție, afaceri europene și protecția consumatorului din brandenburg și adr centru. teme cum ar fi valorificarea energiilor regenerabile, formarea profesională continuă, inclusiv a unor muncitori români în germania, schimburile de experiență, cooperarea centrelor europe direct, asigurarea transferului tehnologic, dar și promovarea economică reciprocă, sunt viitoarele direcții de cooperare între cele două regiuni europene, conform parteneriatului recent semnat. ținând cont de experiența și expertiza de care dispune landul brandenburg în aceste domenii, prin organizarea acestei vizite de lucru, adr centru și potențialii beneficiari de fonduri por 2014-2020, vor beneficia de know-how-ul partenerilor germani și își vor putea valorifica bunele lor practici în domeniu. de mare interes pentru regiunea centru sunt finanțările alocate prin por 2014-2020 în cadrul axei prioritare 1, de promovare a transferului tehnologic și respectiv în cadrul axei prioritare 4, care combină prioritățile de investiții aferente obiectivelor tematice privind economia cu emisii scăzute de carbon. accent foarte mare se va pune pe protecţia mediului și promovarea utilizării eficiente a resurselor; promovarea incluziunii sociale și combaterea sărăciei, dar și pe investițiile în educație, competențe și învățare pe tot parcursul vieții. semnatarii acestui acord consideră că astfel, preluând cunoștințe de la partea germană, județele din centrul româniei vor putea atinge obiectivul inclus în strategia europa 2020 privind dezvoltarea unei economii bazate pe cunoaștere și inovare. vizita delegației conduse de anne quart a debutat luni dimineață, printr-o întâlnire de lucru cu președintele cj alba, ion dumitrel, alături de primarul orașului cugir, adrian teban, de directorul parcului industrial cugir, emil muntean, de reprezentanții comunei ciugud, dar și împreună cu echipa tehnică din cadrul cj alba. în cadrul acestei întrevederi s-au discutat teme privind parteneriatul dintre landul brandenburg și județul alba, dar și aspecte privind dezvoltarea colaborării, în special în domeniile culturale și sociale, dar ținând cont de contextul social-economic și de oportunitățile de investiții din zonă. ulterior, la sebeș, delegația germană a avut o întâlnire cu echipa municipalității, condusă de dorin nistor. la sebeș, un oraș cu o îndelungată tradiție germană, s-a discutat despre formare profesională, activități și întâlniri bilaterale ale tinerilor, sprijinirea colaborării în domeniul societății civile și al ong, parteneriate școlare și între orașe. un prim eveniment comun cu brandenburg va avea loc chiar la începutul lunii iulie, când o delegație de tineri din orașul wittstock va vizita zona sebeșului, pentru a dezvolta parteneriatul cu colegii de la liceul german. în alba iulia, delegația germană s-a întâlnit în inima cetății cu gabriel pleșa, viceprimarul municipiului alba iulia și reprezentanții echipei tehnice din primărie. oaspeții au făcut o adevărată incursiune în istorie, în povestea locurilor, dar și a oamenilor care trăiesc aici, atenția fiindu-le captivată de prezentarea rezultatelor proiectelor europene, despre revitalizare socială, despre rolul comunității în dezvoltarea acestui oraș. vizitând investițiile realizate cu fonduri por care au schimbat complet fața cetății vauban, anne quart s-a declarat absolut îndrăgostită de acest oraș, unde vestigiile sunt puse în valoare și integrate în viața comunității. "am fost deosebit de onorați să putem organiza o vizită importantă a unui reprezentant de stat din landul brandenburg în regiunea centru. acest eveniment survine după mai mulți ani, aproape zece, de colaborare continuă și va fi și un moment important pentru definirea colaborării viitoare. de-a lungul timpului am derulat multe proiecte comune, inclusiv și în cadrul unor proiecte transnaționale, dar acum ne punem problema pregătirii de proiecte finanțabile prin por 2014-2020. trebuie să ținem cont de experiența și cunoștințele de care dispune landul brandenburg. vom continua să investim pentru viitorul regiunii dar, de acum, ne preocupă și dezvoltarea relațiilor de cooperare cu germanii prin proiecte pentru societatea civilă, pentru educație și pentru intensificarea schimburilor de experiență în domeniul cercetării. urmează să monitorizăm felul în care rezultatele acestei vizite de lucru vor sprijini potențialii beneficiari de fonduri europene, pentru a-și valorifica bunele practici în domeniu", a declarat, după încheierea vizitei germane în regiunea centru, simion crețu, director general adr centru. pentru ziua de marți, delegația coordonată de anne quart a vizitat două proiecte finanțate cu fonduri europene prin por 2007-2013 în sovata. este vorba de modernizarea zonei lacului ursu și un ștrand combinat cu o bază modernă de tratament, respectiv o investiție publică și una privată. delegația landului brandenburg, a fost ghidată chiar de președintele cj mureș, ferenc peter și primarul din sovata, laszlo fulop, iar germanii au fost impresionați de ce bază de tratament și ce investiții s-au realizat pe "drumul sării". ulterior, a participat în calitate de invitat special la ședința cdr centru, forul deliberativ regional, alături de președintele cdr centru, sándor tamás, care este și președintele cj covasna, la hotel aluniș. în cadrul acestei întrevederi, a dezbătut stadiul dezvoltării regiunii, precum și situația pregătirii de proiecte por 2014-2020, alături de membrii cdr centru. germanii au prezentat câteva sugestii și sfaturi privind impactul și semnificația viitoarelor proiecte finanțabile cu fonduri nerambursabile, prin programele europene, astfel încât cetățenii să sesizeze o schimbare în bine a calității vieții lor, ca urmare a derulării respectivelor investiții. ulterior acestei întâlniri, secretarul de stat anne quart și tamás sándor, ferenc peter și simion crețu au susținut o conferință de presă comună. de la sovata, delegația germană, însoțită de membrii reprezentanței pentru chestiuni de parteneriat bilateral între landul brandenburg și regiunea centru, a participat la vizite de documentare, atât la proiecte finanțate prin por 2007-2013, cât și la obiective de mare interes în sighișoara și sibiu. în acest din urmă oraș a fost organizată o întâlnire cu comunitatea germană, reuniune condusă de martin bottesch și wiegand fleischer, vicepreședintele cj sibiu, după care a avut loc o întrevedere cu consulul germaniei, judith urban. miercuri, 21 iunie, delegația landului brandenburg a părăsit românia, iar în aceeași după-amiază anne quart a avut o întâlnire de lucru la potsdam cu membrii comisiei pentru afaceri europene din parlamentul landului, cărora le-a împărtășit experiența avută în românia. a vorbit în fața colegilor despre proiectele de dezvoltare vizitate, despre temele privind cooperarea bilaterală și despre intensificarea investițiilor germane în regiunea centru. plăcut impresionată și de atracțiile turistice din românia, anne quart a recomandat membrilor acestei comisii din brandenburg să viziteze măcar odată românia și regiunea centru. "trebuie să recunosc faptul că despre această regiune nu știam prea multe lucruri, citind doar rapoartele de lucru și punctele de vedere oficiale asupra derulării unor evenimente și asupra rezultatelor unor proiecte. în schimb, în aceste zile am vizitat locuri și m-am întâlnit cu oameni deosebiți, care mi-au arătat o altă față despre românia, despre regiunea centru. și ceea ce am văzut m-a surprins plăcut, deoarece am văzut că aici se dorește dezvoltare, se dorește progres se lucrează și, mai ales, este multă înțelegere și toleranță. am vorbit cu români, cu maghiari, cu germani și toți mi-au prezentat proiecte și rezultate, dar și oportunitățile de viitor, care înglobează câte puțin din viziunea fiecăruia. mă bucur că am efectuat această vizită și mă bucur că ne putem pune experiențele dobândite de-a lungul timpului la dispoziția partenerilor noștri din județele regiunii centru, pentru propășirea cetățenilor și pentru pregătirea de și mai bune proiecte, absorbind fondurile structurale ale uniunii europene", a declarat miercuri, înainte de plecarea spre germania, anne quart, secretar de stat în cadrul ministerului de justiție, pentru europa și protecția consumatorului, din landul brandenburg. & 1311 & medium & Medium & Socio-Economic & Socio-Economic & NA & 2017-06-22 & 2017 & 2 & ECO
Frame & low-medium & Regional & +1000 & 1.0255497 & 0.6327204 & -1.4131940 & 0.5015415 & -0.9523600 & 0.0 & -1.0024021 & 0.3314703 & Recipient & Domestic & Domestic & Domestic & Domestic|ECO & Positive\\
Romania & https://evz.ro/dancila-atacata-dur-de-la-bruxelles.html & 530 & evz.ro & Private/Non-Public & Online and Offline & National & very high = CP is most important issue + CP is mentioned in title/headline & Mismanagement & Negative & EU & No myth & NA & NA & NA & NA & NA & NA & NA & NA & Romania & dăncilă atacată dur de la bruxelles: constat la fiecare întâlnire cât de dezinformată sunteți & 2018-12-29 & politica regională & 13:28 vaclav klaus: nu există democraţie în parlamentul european. este doar o glumă comisarul european pentru politică regională, corina crețu, și-a cerut scuze în fața oficialilor europeni pentru faptul că guvernul de la bucurești a ales să ignore munca comisiei europene cu privire la atragerea de fonduri ue. aceasta a remarcat, cu părere de rău, că la fiecare întâlnire cu premierul viorica dăncilă constată că prim-ministru este "dezinformat". "am urmarit cu mare interes declaratiile autoritatilor privind absorbtia fondurilor europene de catre românia. daca aceste rezultate vor fi confirmate de serviciile mele, a caror evaluare pe tari va fi definitivata in ianuarie, nu pot decat sa felicit autoritatile nationale, in mod sincer. ceea ce este trist in toate aceste declaratii triumfaliste este ignorarea completa nu neaparat a muncii mele, dar a intregii echipe de experti de la dg regio, care au venit cu solutii, discutate la zecile de intalniri bilaterale, atat in ce priveste achizitionarea de ambulante, redirectionarea fondurilor catre imm-uri, etc. nu cred ca exista vreo tara, dupa cum vad in revista presei pe care o primesc in fiecare zi, in care sa nu fie mentionata comisia europeana atunci cand se vorbeste de fonduri europene. exploziv! de ce se teme florian coldea? "vânătoarea de plagiate" îl acuză pe fostul șef din sri că a șters un interviual treilea cutremur în românia, în mai puțin de 24 de ore. decembrie, aproape un cutremur pe zi. ce spun specialiștii poate ca din punct de vedere politic nu este tocmai cel mai nimerit, dar ma simt datoare fata de colegii mei de la dg regio, fata de serviciile mele care au venit cu idei si au salvat milioane de euro, sa imi cer scuze pentru faptul ca munca comisiei europene a fost ignorata in toate interviurile si declaratiile guvernantilor. doamna prim ministru, cu tot respectul, comisia a schimbat programul operational regional, nu guvernul. imi pare rau sa constat, in fiecare intalnire pe care am avut-o cu dvs, cat de dezinformata sunteti de catre cei din subordinea dvs, lucru pe care vi l-am transmis in mai multe randuri personal. va asigur in continuare de tot sprijinul meu si al comisiei europene, asa cum am declarat si la bruxelles, la conferinta de presa comuna cu dl jean-claude juncker, presedintele comisiei europene. vezi acum oferta! cat despre cei care incearca in fiecare zi sa ma atace, prin diverse voci, as vrea sa fac urmatoarele precizari: toate declaratiile, inclusiv in românia, s-au referit strict la portofoliul meu, al politicii regionale, la fondurile pe care le gestionez. in toate cele 28 de tari ale uniunii europene am dat interviuri, am facut conferinte de presa si am prezentat situatia la zi. care in unele tari e buna, in altele nu. nu numai ca nu e interzis (?), dar e obligatia mea sa incurajez folosirea la maxim a fondurilor europene si sa atrag atentia asupra riscurilor . românia nu este o tara care sa isi poata rata sanse la dezvoltare din orgolii. am spus si repet- am fost in ultimii patru ani la dispozitia autoritatilor romane. cu totii m-ati putut vedea, am venit in românia in prima saptamana imediat dupa instalarea fiecarui guvern, am inclus valea jiului in proiectul pilot pentru dezvoltarea regiunilor carbonifere, am inclus regiunile nord-est si nord-vest din românia in cadrul initiativei pe care am lansat-o pentru regiunile lasate in urma. am initiat grupul de lucru pentru o mai buna implementare, gratie caruia românia a ajuns la 91\% grad de absorbtie pentru perioada 2007-2013. la unele probleme am gasit solutii, la altele nu. răsturnare de situație în războiul de la granițele româniei. ce se întâmplă de anul noumesaj incendiar transmis de judecătoarea adriana stoicescu. ce îndemn are pentru magistraţi dorim in continuare sa ajutam: asteptam proiecte, mai ales in domeniul infrastructurii de transport, expertii nostri au lucrat enorm si vreau sa le multumesc, stiind ca vor face eforturi in continuare, indiferent daca munca lor e apreciata sau nu la bucuresti. voi reveni in ianuarie cu analizele comisiei europene", a transmis corina crețu, sâmbătă, pe facebook. ultima ora! ce se intampla cu salariile, in 2019! anunt bomba pentru toti bugetarii! pagina 1 din 1 tag-uri: dancila, viorica dancila, cretu, ce, corina cretu în lipsa unui acord scris din partea evenimentul zilei, puteţi prelua maxim 500 de caractere din acest articol dacă precizaţi sursa şi dacă inseraţi vizibil link-ul articolului dăncilă atacată dur de la bruxelles: constat la fiecare întâlnire cât de dezinformată sunteți. scandalul provocat de "lista lui hodor" ajunge în sondajul care schimbă tot ce știai. cea mai proastă veste pentru să-l vezi pe ludovic orban când trage concluzii...! iohannis în rolul ... umilință maximă pentru dragnea și dăncilă! atacul, de la cel mai ... șocant! povestea lui traian, un câine aruncat pentru că... băutura aproape miraculoasă care topește grăsimea și... la 11 ani de la disparitie, elodia rupe tacerea: "m-am... sâmbătă, 29 decembrie 2018 autor: cătălina iordache artileria nouă din guvernul dăncilă, atac fără precedent la iohannis. "locatarul palatului cotroceni, un adevărat arhanghel al haosului" sâmbătă, 29 decembrie 2018 autor: cristian tănase este oficial! florin iordache îi va lua locul lui liviu dragnea la camera deputaților sâmbătă, 29 decembrie 2018 autor: gabriel valentin ministru din guvernul dăncilă, acuzat de "un act de trădare fără precedent" capital.ro uniunea europeană s-ar putea rupe. care este rolul... oug prin care se introduc noi taxe a fost aprobată. ce se... iohannis e în stare de șoc. anunțul făcut de liviu dragnea ghid de buzunar pentru părinţi creaţie sau evoluţie. trebuie să alegem? codul personalităţii. matricea celor paisprezece tipuri de personalitate directorul azilului, anunț despre zina dumitrescu. s-a... bianca drăgușanu, pozată într-un moment intim! 'cât de... cum a ajuns să arate o actriță porno, după 10 ani de... se întorc ninsorile! cun va fi vremea de anul nou,... doliu înainte de noul an. a murit o mare cântăreaţă, avea... zile libere 2019. bucurie pentru bugetari, ar putea să... oana roman, pozata goala, in baie! imaginea a ajuns pe... cu ce se ocupa sotia lui mihai gadea! uite cat de... adevarul despre ozana barabancea! a slabit sau nu? ce se... ştefania de la puterea dragostei, prinsă în pat cu un... mariana de la exatlon, părăsită înainte de sărbători! andreea mantea, atacată dur! "foarte urât că permiţi aşa... bombă în lumea mondenă: relaţia neştiută dintre iubitele... scandal total la exatlon! acuzații grave aduse de alin... adio, alexandru arșinel! din păcate, maestrul ne-a... a murit! doliu uriaș în sportul românesc. "drum bun spre... accident înfiorător pe dn 1! din păcate, la volan se afla... anm a schimbat prognoza de revelion. fenomene bizare în... soția, prinsă cu amantul când nici nu bănuia. detaliul de... şerveţelele umede, un real pericol: de ce ar trebui să... dacă ai această grupă de sânge sunt șanse mai mari să mori o mare actrita de 75 de ani se iubeste cu o femeie care... mama isi aminteste momentul in care cei trei copii au... au facut nunta in secret chiar inainte de craciun. de... psd-alde, candidat comun la prezidențiale. ce spun analiștii credite pentru români în 2019. daniel zamfir distruge... iohannis, refuz nominalizare șef de stat major. h. d.... de ce refuză iohannis schimbarea șefului armatei române.... fmi, avertisment pentru românia rareș bogdan: e ultima ediție, cu o singură condiție & 1219 & very high & High & Governance & NA & NA & 2018-12-29 & 2018 & 3 & POL
Frame & high-very high & National & +1000 & 1.0255497 & 0.6327204 & -1.4131940 & 0.5015415 & -0.9523600 & 0.0 & -1.0024021 & 0.3314703 & Recipient & European & European & European & European|POL & Negative\\
Romania & http://www.aradon.ro/miza-alegerilor-din-26-mai/2235402 & 557 & aradon.ro & Private/Non-Public & Online only & Regional/Local & low = CP mentioned more times but NOT important part of story (mainly about others issues) & Political capital/interests & Negative & National & No myth & NA & NA & NA & NA & NA & NA & NA & NA & Romania & miza alegerilor din 26 mai & 2019-04-17 & fondul social european & pe 16 martie a avut loc la bucurești o puternică reuniune europeană mai puțin obișnuită în românia. gazda a fost pnl. au fost invitați lideri și aleși locali și regionali ai celei mai puternice familii din uniunea europeană, partidul popular european. la această reuniune și-a expus programul politic viitorul președinte al comisiei europene, manfred weber. la nici o lună distanță, alde european anunța alde românia că începe procedura de excludere a formațiunii lui tăriceanu. câteva zile mai târziu, socialiștii europeni au anunțat înghețarea relațiilor cu psd. de ce sunt importante aceste semnale pentru alegerile europarlamentare și, mai ales, pentru soarta româniei în europa? "pnl este unul dintre motoarele din cadrul partidului popular european". "pnl este puternic și reprezintă viitorul româniei". sunt aprecierile lui manfred weber, lider al grupului ppe din parlamentul european și viitorul președinte al comisiei europene, prezent la bucurești la summitul ppe din 16 martie 2019. cuvintele liderului ppe nu au fost simple amabilități. ele arată limpede că pnl este cel mai puternic și respectat partid din românia în uniunea europeană. de altfel, la summit-ul de la bucurești, manfred weber și-a prezentat în premieră programul pentru viitoarea președinție a comisiei europene, iar românia și românii ocupă un loc important în acest program. semnalul dat de reuniunea aleșilor locali și regionali ai ppe a fost de forță. cel mai important partid european are cei mai puternici aliați și parteneri în românia: klaus iohannis, președintele româniei, și partidul național liberal. sigur, vă puteți întreba ce câștigă românia de pe urma faptului că pnl are relații bune cu partidul popular european. mai întâi de toate, bunele relații sunt bazate pe principii și valori comune, în interesul româniei și al uniunii europene. interesele românilor și valorile naționale sunt în armonie cu valorile europene și asta spun atât pnl, cât și ppe. de altfel, cu sprijinul ppe, partidul național liberal a obținut în parlamentul european victorii pentru români: eliminarea taxei de roaming, creșterea subvențiilor pentru agricultori, despăgubiri pentru fermierii care au avut pierderi din cauza pestei porcine, salvarea combinatelor chimice din românia și a mii de locuri de muncă, apărarea intereselor transportatorilor români, creșterea fondurilor pentru bursele de studii erasmus, programul care oferă bilete gratuite de tren tinerilor care vor să viziteze uniunea europeană. repetenții europei după o lungă perioadă de îngheț în relațiile cu tovarășii săi socialiști europeni, liviu dragnea a încercat marea cu degetul la congresul acestora de la madrid, din 22 februarie. a încercat să trezească o fărâmă de simpatie și a înfierat dreapta europeană și pe cea din românia în culorile și cu teoriile pe care le folosește în țară. "trebuie să ieşim la ofensivă împreună!", i-a îndemnat dragnea pe socialiștii europeni. a rămas, însă, singur cu acest îndemn. socialiștii europeni au anunțat sec că îngheață relațiile cu psd, până când guvernul româniei își va clarifica angajamentul față de respectarea statului de drept. colegul de guvernare al lui dragnea, călin popescu tăriceanu, și partidul lui au primit, la rândul lor, o palmă din partea fostului lor prieten, foarte vocalul guy verhofstadt, liderul alde european: "subminarea sistematică a valorilor europene, printre care se numără și independența justiției, nu poate fi făcută în numele alde din pe și nici folosindu-se numele alde". socialiștii europeni și formațiunea lui verhofstadt au înțeles, până la urmă, că psd și alde vor să îi scape de dosarele penale pe liviu dragnea și gașca lui. pentru asta, psd și alde au realizat cel mai amplu asalt la adresa sistemului judiciar din românia, unul fără precedent de la căderea comunismului până în prezent, iar reacțiile partenerilor externi nu au întârziat să apară. teoria cu statul paralel nu a ținut, nici în fața românilor, nici în fața oficialilor europeni. înaintea deciziilor pes și alde european, aria reacțiilor externe se lărgise. au reacționat pe rând, comisia europeană, comisia de la veneția, grupul statelor împotriva corupției (greco), parlamentul european, ambasadele occidentale de la bucurești, inclusiv cea a statelor unite ale americii. când s-au văzut puși la zid și izolați de partenerii occidentali, psd și alde au renunțat la orice aparență și au atacat frontal uniunea europeană cu o retorică naționalist-ceaușistă. în linia ofensivei anti-europene a psd-alde, a urmat o premieră nefastă de la aderarea la ue și până astăzi - guvernele dragnea-tăriceanu sunt primele care refuză explicit fonduri europene - fonduri pentru spitale regionale, pentru autostrăzi. au lăsat la bruxelles necheltuite zeci de miliarde de euro, bani gratis pentru românia. de exemplu, din banii europeni pentru angajarea tinerilor (329 de milioane euro), guvernul psd-alde a cheltuit doar 0,3\%. din fondul social european (5,43 miliarde euro), guvernul psd-alde a folosit doar 0,16\%. din cele 12,5 miliarde de euro de la fondul de dezvoltare regională, au utilizat doar 2,15 miliarde - din acest fond se puteau construi primele 3 spitale regionale din cele 8 promise de psd. temele reale ale campaniei pentru parlamentul european dragnea și psd încearcă să impună o temă falsă pe agenda acestor alegeri - aceea că românii ar fi defavorizați în uniunea europeană. de fapt, psd îi tratează pe români drept cetățeni de mâna a doua în propria lor țară de vreme ce guvernarea lor este incapabilă sau refuză de-a dreptul să folosească avantajele oferite de apartenența la uniunea europeană. fondurile europene sunt principalul avantaj pentru români, avantaj de care psd și-a bătut joc. mai mult, psd a pus-o pe rovana plumb pe primul loc pe lista de candidați la europarlamentare, deși este unul din cei mai slabi miniștri ai fondurilor europene. tema reală a campaniei pentru alegerile europarlamentare este: ce s-a realizat cu banii europeni pentru românia? la această întrebare, primarii și președinții de consilii județene pnl pot să dea cele mai concrete răspunsuri. adevăratul minister al fondurilor europene este la oradea, la alba iulia, timișoara, sibiu, arad, suceava, brașov, sinaia, cluj-napoca. administrațiile liberale au reușit să utilizeze aproape un miliard de euro din fondurile europene în folosul comunităților lor. românia chiar merită mai mult decât psd! cum să ceri votul românilor pentru parlamentul european când tu îți bați joc sistematic de valorile europene, ești paria în uniunea europeană, nu mai reprezinți pentru nimeni un partener de dialog? orice vot pentru psd sau alde e un vot irosit. ei sunt rupți de uniunea europeană, sunt rupți de familiile lor politice și nu pot face nimic bun pentru românia în parlamentul european și ue. mai mult, orice vot pentru psd și alde este un vot pentru marginalizarea româniei și aruncarea țării la periferia uniunii europene. românia împlinește 30 ani de libertate. psd, de la iliescu la dragnea, a irosit cel puțin jumătate din acești ani. alegerile din 26 mai sunt despre românia care-și întărește identitatea, demnitatea și valorile naționale în armonie cu cele occidentale. tocmai pentru că este parte a europei, poate sta în primul rând al europei și nu în afara europei. românia chiar merită mai mult decât psd! (204137) & 1152 & low & Low & Power & NA & NA & 2019-04-17 & 2019 & 3 & POL
Frame & low-medium & Regional & +1000 & 1.0255497 & 0.6327204 & -1.4131940 & 0.5015415 & -0.9523600 & 0.0 & -1.0024021 & 0.3314703 & Recipient & Domestic & Domestic & Domestic & Domestic|POL & Negative\\
\addlinespace
Romania & https://www.monitorulsv.ro/Local/2018-05-16/Angelica-Fador-Comisarul-european-Corina-Cretu-confirma-ca-actualul-Guvern-este-o-frana-in-dezvoltarea-Romaniei & 592 & Monitorul de Suceava & Private/Non-Public & Online and Offline & Regional/Local & very high = CP is most important issue + CP is mentioned in title/headline & Economic development & Negative & EU + National & No myth & Infrastructure & Negative & National & No myth & Mismanagement & Negative & National & No myth & Romania & angelica fador comisarul european corina cretu confirma ca actualul guvern este o frana in dezvoltarea romaniei & 2018-05-16 & politica de coeziune & deputatul pnl de suceava angelica fădor a declarat, ieri, că prin scrisoarea transmisă mai multor ministere comisarul european corina creţu a confirmat faptul că actualul guvern al româniei, condus de viorica dăncilă şi girat de liderul psd, liviu dragnea, reprezintă o frână în dezvoltarea ţării. angelica fădor a făcut referire la scrisoarea pe care comisarul european român pentru dezvoltare regională, corina creţu, a transmis-o premierului viorica dăncilă, ministrului transporturilor, lucian şova, şi ministrului fondurilor europene, rovana plumb, prin care îşi exprimă îngrijorarea cu privire la implementarea proiectelor europene, în special a celor de infrastructură de transport, dar şi cu privire la slaba absorbţie a banilor de la uniunea europeană. "prin această scrisoare sunt demascate incompetenţa şi modul iresponsabil în care guvernează psd şi alde. scrisoarea nu este una de gratulare a celor doi miniştri din cabinetul dăncilă pentru <activitatea excepţională> pe care o desfăşoară, ci din contră, este un strigăt de disperare al partenerilor europeni, prin vocea doamnei comisar european, în legătură cu ritmul extrem de lent şi modul deficitar în care sunt planificate şi implementate proiectele de infrastructură de transport în românia", a spus angelica fădor, care a adăugat că "suntem convinşi că această scrisoare se află pe masa celor doi miniştri de două săptămâni, că a fost analizată, dar nu a fost făcută publică de frica reacţiilor ce pot apărea". deputatul pnl de suceava a atras atenţia că românia va pierde banii europeni pentru infrastructură pentru că există doar patru proiecte majore, iar altele noi, care să fie lansate în execuţie, nu există. "ne aflăm la mijlocul exerciţiului financiar 2014-2020, va fi o evaluare intermediară în anul 2018, iar faptul că pregătirea proiectelor este foarte slabă ar putea conduce la dezangajări imediate, periclitând astfel însuşi politica de coeziune în românia", a spus fădor. deputatul liberal a mai precizat că românia va pierde fonduri pentru că ministerul transporturilor, prin conducerea sa, este mai mult preocupat de politica de cadre şi de numiri pe funcţii "ale găştii politice", decât de a construi politici de transport eficiente şi sustenabile. angelica fădor atrage atenţia că românia va fi ocolită de investiţii din cauza lipsei unei reţele de transport de mare viteză angelica fădor a arătat că uniunea europeană solicită o implicare mai mare a celor doi miniştri în procesul de implementare a proiectelor de infrastructură, prin pregătirea unui portofoliu de proiecte mature care să se califice pentru a primi finanţare europeană. "se cere responsabilitate şi instituţii funcţionale pentru că ne aflăm la jumătatea perioadei de programare 2014-2020, iar cartea noastră de vizită cu care ne vom prezenta la negocieri pentru următorul exerciţiu financiar european are multe lacune şi ne creează mari dezavantaje. mai mult, vom primi mai puţini bani pentru infrastructură pe următorul exerciţiu financiar european 2020-2027", a declarat deputatul pnl de suceava. în aceste condiţii, angelica fădor a precizat că pnl îi solicită premierului viorica dăncilă să îşi depună mandatul din fruntea guvernului, iar odată cu ea să plece din funcţii şi miniştrii rovana plumb şi lucian şova. "eşecul guvernării psd-alde la acest capitol nu mai poate fi tolerat. românia pierde enorm din cauza acestei inacţiuni guvernamentale, din cauza incompetenţei manifestate în actul de guvernare în domeniul infrastructurii de transport. pierde din punct de vedere economic, pierde investitori, pierde bani europeni ieftini, pierde credibilitate externă şi devine o ţară nefrecventabilă din acest punct de vedere pentru potenţiali parteneri strategici", a afirmat angelica fădor. ea a mai adăugat că în aceste condiţii românia va fi ocolită de investiţii din cauza lipsei unei reţele de transport de mare viteză, rutieră şi feroviar, iar pe termen lung există riscul să plece şi investitorii existenţi, care au afaceri ce depind de existenţa unei infrastructuri extinse, moderne şi eficiente. & 618 & very high & High & Socio-Economic & Socio-Economic & Governance & 2018-05-16 & 2018 & 3 & ECO
Frame & high-very high & Regional & 500-1000 & 1.0255497 & 0.6327204 & -1.4131940 & 0.5015415 & -0.9523600 & 0.0 & -1.0024021 & 0.3314703 & Recipient & Domestic & European & Mixed & Domestic|ECO & Negative\\
Romania & https://www.dcnews.ro/comisia-europeana-ajutor-pentru-romania-in-problema-de-eurilor\_614697.html & 588 & dcnews.ro & Private/Non-Public & Online only & National & very high = CP is most important issue + CP is mentioned in title/headline & Environment/green/low-carbon & Positive & EU & No myth & NA & NA & NA & NA & NA & NA & NA & NA & Romania & comisia europeană, ajutor pentru românia în problema deșeurilor & 2018-09-24 & fondurile structurale & 'gestionarea deşeurilor ne afectează pe toţi. estimările recente ale băncii mondiale arată o creştere anuală a generării de deşeuri, de la 2,01 miliarde de tone în 2016 la 3,40 miliarde de tone în 2050. deşi în europa gestionarea şi reciclarea deşeurilor se îmbunătăţeşte, un lucru e clar: continuarea sistemului actual nu este o opţiune şi încă mai sunt multe de făcut', se arată în comunicatul ce. luni, executivul comunitar a publicat cea mai recentă revizuire a modului în care reglementările privind gestionarea şi reciclarea deşeurilor sunt aplicate în europa. în pofida progreselor înregistrate de statele membre, ce a identificat decalaje semnificative care trebuie rezolvate rapid, astfel încât europenii să poată beneficia de pe urma beneficiilor economice şi de mediu ale economiei circulare. ajutor prin fonduri structurale și asistență tehnică conform ce, 14 state membre (românia, bulgaria, croaţia, cipru, estonia, finlanda, grecia, ungaria, letonia, malta, polonia, portugalia, slovacia şi spania) au fost identificate ca riscând să nu-şi îndeplinească ţinta pentru anul 2020 de a recicla 50\% din deşeurile municipale. aceste ţări trebuie să facă mai mult astfel încât economiile şi locuitorii lor să poată beneficia de pe urma economiei circulare. de aceea, comisia europeană a prezentat un plan detaliat pentru a se asigura că aceste state acţionează pentru a satisface legislaţia ue privind deşeurile. comisarul pentru mediu, afaceri maritime şi pescuit, karmenu vella, a declarat: 'cu noile reglementări adoptate recent de parlamentul european şi de consiliul de miniştri, europa poate deveni lider mondial în gestionarea şi reciclarea deşeurilor şi îşi poate dezvolta suplimentar economia circulară. încă există diferenţe în europa, dar progresele sunt necesare şi posibile dacă autorităţile naţionale şi locale respective implementează acţiunile identificate în acest raport. comisia europeană le ajută, oferind asistenţă tehnică şi sprijin prin fondurile structurale şi în schimbul de informaţii privind cele mai bune practici'. & 302 & very high & High & Socio-Economic & NA & NA & 2018-09-24 & 2018 & 3 & ECO
Frame & high-very high & National & <500 & 1.0255497 & 0.6327204 & -1.4131940 & 0.5015415 & -0.9523600 & 0.0 & -1.0024021 & 0.3314703 & Recipient & European & European & European & European|ECO & Positive\\
Romania & https://www.realitatea.net/viorica-dancila-acuzata-ca-a-mintit-in-legatura-cu-planurile-pentru-centura-capitalei\_2172937.html & 529 & REALITATEA.NET & Private/Non-Public & Online only & National & high = CP is most important issue in story (can also cover other issues) & Mismanagement & Negative & EU & No myth & NA & NA & NA & NA & NA & NA & NA & NA & Romania & viorica dăncilă, acuzată că a minţit în legătură cu planurile pentru centura capitalei & 2018-12-06 & fondul de coeziune & într-un comunicat transmis agerpres, eurodeputatul pnl a prezentat un răspuns oficial al comisiei din 5 decembrie, care confirmă că guvernul n-a depus până în acest moment dosarul de finanţare. "serviciile comisiei nu au primit încă nicio cerere de finanţare pentru un proiect major referitor la centura ocolitoare a municipiului bucureşti. ministerul transporturilor a informat serviciile comisiei că cnair (compania naţională de administrare a infrastructurii rutiere) i-a transmis cererea de finanţare din fondul de coeziune. prin urmare, comisia înţelege că proiectul este încă în curs de elaborare la nivel naţional. ministerul transporturilor a informat comisia că, până la sfârşitul anului 2018, va fi trimisă comisiei cererea de finanţare pentru proiectul major referitor la centura de sud a municipiului bucureşti. se arată în răspunsul comisiei. "prim-ministrul dăncilă a minţit. comisia europeană a răspuns la o interpelare scrisă pe care i-am adresat-o în care spune că nu s-a depus nicio cerere de finanţare până în prezent", a precizat deputatul european siegfried mureşan. totodată, el susţine că de la depunerea proiectului de metrou 1 mai - otopeni, pe 24 decembrie 2017, "guvernele psd-alde n-au mai depus niciun proiect major de infrastructură de transport". "în momentul de faţă, în românia, fondurile europene sunt absorbite doar de către primarii şi preşedinţii de consilii judeţene. proiectele importante de infrastructură aflate în responsabilitatea guvernului se lasă aşteptate. ", a mai spus mureşan. pe 11 octombrie, premierul a anunţat că va fi solicitată finanţare din bani europeni pentru investiţiile care vizează centura capitalei. "dosarul este gata şi luni (15 octombrie 2018 - n.r,) va fi transmis comisiei europene. solicităm finanţarea din bani europen. vorbim despre fonduri estimate la peste 1,3 miliarde de euro, fără tva. guvernul româniei va oferi tot sprijinul pentru atragerea finanţării europene necesare realizării acestui proiect", a afirmat viorica dăncilă. & 302 & high & High & Governance & NA & NA & 2018-12-06 & 2018 & 3 & POL
Frame & high-very high & National & <500 & 1.0255497 & 0.6327204 & -1.4131940 & 0.5015415 & -0.9523600 & 0.0 & -1.0024021 & 0.3314703 & Recipient & European & European & European & European|POL & Negative\\
Romania & https://www.monitorulsv.ro/Local/2019-05-28/Lansare-proiect-Asigurarea-accesului-la-servicii-de-sanatate-in-regim-ambulatoriu-pentru-populatia-judetului-Suceava & 498 & Monitorul de Suceava & Private/Non-Public & Online and Offline & Regional/Local & low = CP mentioned more times but NOT important part of story (mainly about others issues) & Public services & Positive & EU + National + Subnational & No myth & NA & NA & NA & NA & NA & NA & NA & NA & Romania & lansare proiect asigurarea accesului la servicii de sanatate in regim ambulatoriu pentru populatia judetului suceava & 2019-05-28 & fondul european de dezvoltare regională & ministerul sanatatii/unitatea de implemenatre și coordonare programe în parteneriat cu: o spitalul judeţean de urgenţă "sfântul ioan cel nou" suceava o spitalul municipal fălticeni o spitalul municipal câmpulung moldovenesc o spitalul orăşenesc gura humor în calitate de beneficiar, anunță demararerea (semnarea) contractului de finanțare nerambursabilă nr. 3599/14.12.2018, cod smis 125387 pentru proiectul "asigurarea accesului la servicii de sănătate în regim ambulatoriu pentru populația județului suceava" finanțat prin programuloperaţional regional 2014-2020, axa prioritară 8 - dezvoltarea infrastructurii sanitare şi sociale, prioritatea de investiții 8.1 - investiţii în infrastructurile sanitare şi sociale care contribuie la dezvoltarea la nivel naţional, regional şi local, reducând inegalităţile în ceea ce priveşte starea de sănătate şi promovând incluziunea socială prin îmbunătăţirea accesului la serviciile sociale, culturale și de recreere, precum și trecerea de la serviciile instituționale la serviciile prestate de comunități, obiectiv specific 8.1 - creșterea accesibilității serviciilor de sănătate, comunitare și a celor de nivel secundar, în special pentru zonele sărace și izolate operațiunea: a - ambulatorii; apel proiecte: por/2018/8/8.1/1/8.1.a/7 regiuni-nefinalizate-cod apel 420/8. obiectivul general al proiectului este dezvoltarea infrastructurii sanitare, prin investiții în dotări sau lucrări necesare infrastructurilor sanitare, care contribuie la dezvoltarea operaționalității la nivel național, regional și local, reducând inegalitățile în ceea ce priveşte starea de sănătate publicăsi îmbunătăţind accesul la serviciile medicale de calitate. prin proiect se urmărește modernizarea si tehnologizarea la nivel european al infrastructurilor sanitare deficitare, care utilizează în continuare aparatura uzată fizic si moral, prin achiziția de echipamente noi si eficiente, ce aduc atât o creștere calitativă a serviciilor oferite, cât și o creștere cantitativă a numărului de persoane beneficiari de servicii medicale, îmbunătățind starea de sanitare publică la nivel național, regional si local. rezultate așteptate: prin implementarea proiectului activitatea medicală desfășurată în unități sanitare care beneficiază de dotări și/sau lucrări va fi eficientizată. aceasta constă în creșterea capacității de tratarea a persoanelor oferind servicii medicale îmbunătățite și de o calitate superioară celei existente, prin înnoirea gamei de echipamente și dispozitive medicale deținute sau prin îmbunătățirea infrastructurii unității. valoarea totală a proiectului este de 30.245.164,37 lei, din care finanțare nerambursabilă din fondul european de dezvoltare regională 21.171.615,07 lei. defalcarea pe parteneri a valorii totale a proiectului: o 29.769.578,23 lei pentru spitalul judeţean de urgenţă "sfântul ioan cel nou" suceava o 210.307,99 lei pentru spitalul municipal fălticeni o 152.543,77 lei pentru spitalul municipal câmpulung moldovenesc o 112.734,38 lei pentru spitalul orăşenesc gura humor perioada de implementare a proiectului este de 69 luni, respectiv între data de 17.03.2014 și data de 30.11.2019. date de contact: spitalul municipal câmpulung moldovenesc, str. sirenei nr. 25, tel. 0727/057184, e-mail: contabilsef@spitalcampulungmoldovenesc.ro investim în viitorul tău! proiect cofinanțat din fondul european de dezvoltare regională prin programul operațional regional 2014-2020 & 484 & low & Low & Socio-Economic & NA & NA & 2019-05-28 & 2019 & 3 & ECO
Frame & low-medium & Regional & <500 & 1.0255497 & 0.6327204 & -1.4131940 & 0.5015415 & -0.9523600 & 0.0 & -1.0024021 & 0.3314703 & Recipient & Domestic & European & Mixed & Domestic|ECO & Positive\\
Romania & http://www.euractiv.ro/fonduri-ue/romania-este-mult-in-urma-bulgariei-la-absorbtia-de-fonduri-ue-9694 & 494 & EurActiv | Știri, politici europene \& Actori UE online & Private/Non-Public & Online only & National & high = CP is most important issue in story (can also cover other issues) & Bureaucracy and/or delays & Negative & National + Other country & No myth & NA & NA & NA & NA & NA & NA & NA & NA & Romania & românia este mult în urma bulgariei la absorbția de fonduri ue @ euractivromania & 2017-12-22 & fondul de coeziune & statistica zileiromânia este mult în urma bulgariei la absorbția de fonduri ue românia a cheltuit doar 2\% din banii europeni alocaţi programării 2014-2020, respectiv 936,3 milioane euro, în timp ce state precum bulgaria şi polonia au avut 9\%, respectiv 8\%, potrivit datelor comisiei europene. cea mai mare parte a sumei cheltuite a fost prin fondul de coeziune, respectiv 469,6 milioane de euro, urmată fondul european de dezvoltare regională cu 440,2 milioane euro, relatează agerpres. doar austria și irlanda, fiecare cu câte 1\%, stau mai prost decât românia la absorbția fondurilor structurale şi de investiţii. ministrul delegat pentru fonduri europene, marius nica, declara săptămâna trecută că românia va termina anul cu o absorbţie de 5\% a banilor europeni, reprezentând 1,23 miliarde euro. bugetul pentru 2017 prevedea însă o rată de absorbție de 9\% la final de an. & 141 & high & High & Governance & NA & NA & 2017-12-22 & 2017 & 2 & POL
Frame & high-very high & National & <500 & 1.0255497 & 0.6327204 & -1.4131940 & 0.5015415 & -0.9523600 & 0.0 & -1.0024021 & 0.3314703 & Recipient & Domestic & European & Mixed & Domestic|POL & Negative\\
\addlinespace
Romania & http://www.romanialibera.ro/economie/fonduri-europene/proiectul-\%E2\%80\%9Cmodelling-the-new-europe--se-incheie-401295 & 508 & RomaniaLibera.ro & Private/Non-Public & Online and Offline & National & very low = CP mentioned once & Jobs & Factual & Subnational & No myth & NA & NA & NA & NA & NA & NA & NA & NA & Romania & proiectul "modelling the new europe" se încheie & 2015-12-08 & fondul social european & universitatea hyperion din bucurești anunță în data de 15.12.2015 închiderea proiectului: "modelling the new europe" program masteral în guvernanța economică europeană proiect cofinanţat din fondul social european prin programul operaţional sectorial dezvoltarea resurselor umane 2007-2013 axa prioritară nr. 1 "educaţia şi formarea profesională în sprijinul creşterii economice şi dezvoltării societăţii bazate pe cunoaştere", domeniul major de intervenție 1.2 "calitate în învăţământul superior", dezvoltarea unui program masteral interdisciplinar în domeniul guveranţei economice europene, pentru dezvoltarea calitativă a resurselor umane şi a comunicării dintre mediul academic şi cel economic, în scopul creşterii gradului de integrare a absolvenţilor pe piaţa muncii. & 102 & very low & Low & Socio-Economic & NA & NA & 2015-12-08 & 2015 & 1 & ECO
Frame & v.low & National & <500 & 1.0255497 & 0.6327204 & -1.4131940 & 0.5015415 & -0.9523600 & 0.0 & -1.0024021 & 0.3314703 & Recipient & Domestic & Domestic & Domestic & Domestic|ECO & Neutral\\
Romania & http://www.mediafax.ro/social/cinci-organizatii-ale-societatii-civile-cer-investigarea-proiectului-sii-analytics-al-sri-16144358 & 555 & Mediafax.ro & Private/Non-Public & Online only & National & high = CP is most important issue in story (can also cover other issues) & Political capital/interests & Negative & EU + National & 4.No added value & Mismanagement & Negative & EU & 8.Mismanaged & NA & NA & NA & NA & Romania & cinci organizaţii ale societăţii civile cer investigarea proiectului sii analytics al sri & 2017-01-31 & fondul european de dezvoltare regională & "dorim să vă semnalăm următoarele fapte în legătură cu proiectul sii analytics al sri, cu precizarea că ele au făcut obiectul unei sesizări a comisiei europene şi a mai multor articole de presă. considerăm că ar trebui să investigaţi cu prioritate faptele semnalate aici, existând suspiciunea că sri a intervenit la înalţi oficiali ai guvernului cioloş pentru aprobarea acestui proiect, ba chiar şi la primul ministru, şi că ultima tranşă a acestui proiect survine pe fondul unor încălcări constituţionale mai vechi, acoperite prin decizii ale csat a căror constituţionalitate nu a fost verificată", se arată în scrisoarea semnată de societatea academică din românia (sar), active watch, asociaţia miliţia spirituală, asociaţia pentru apărarea drapturilor omului în românia - comitetul helsinki (apador-ch), asociaţia pentru tehnologie şi internet (apti) şi centrul pentru resurse juridice, remisă mediafax. organizaţiile semnatare arată că tematica sesizării "se circumscrie exact misiunii" comisiei de control a sri, iar faptele asupra cărora parlamentarii sunt sesizaţi "crează un precedent periculos pentru democraţia noastră, contravine dreptului european şi bunei gestionări a fondurilor europene". în sesizarea propriu-zisă se arată că există suspiciunea că sri ar fi fost favorizat înaintea şi în cadrul procedurii de licitaţie a proiectului din fondul european de dezvoltare regională (fedr), licitaţie la care, de altfel, a existat un singur ofertant. "proiectul "sii analytics - sistem informatic de integrare şi valorificare operaţională şi analitică a volumelor mari de date" este destinat asigurării unei capacităţi superioare de analiză a bazelor de date ale principalelor instituţii din românia. obiectivul platformei este de a spori considerabil viteza de căutare a informaţiei relevante în bazele de date deja existente. practic, în loc să interogheze sisteme diferite, neuniformizate informatic şi procedural, instituţiile statului vor putea accesa informaţiile integrat, rapid şi eficient", transmitea serviciul român de informaţii (sri) printr-un comunicat de presă remis în august 2016. instituţia preciza că proiectul reprezintă crearea unei platforme moderne, similară altora existente în plan european.în ceea ce priveşte accesul la datele stocate, sursa citată menţiona că orice interogare a sistemului este înregistrată şi analizată pentru a se evita orice formă de abuz. valoarea proiectului a fost estimată la 142,071 milioane de lei, reprezentând aproximativ 31,5 milioane de euro, din care 84,3411\% finanţare din fonduri externe nerambursabile, anunţa sri. precizările au venit în contextul în care apador-ch a anunţat anterior că patru organizaţii neguvernamentale au trimis o scrisoare deschisă mai multor instituţii naţionale şi europene, prin care au cerut anularea proiectului sri finanţat din fonduri europene, despre care susţineau că ar fi unul "de supraveghere în masă". conținutul website-ului www.mediafax.ro este destinat exclusiv informării și uzului dumneavoastră personal. este interzisă republicarea conținutului acestui site în lipsa unui acord din partea mediafax. pentru a obține acest acord, vă rugăm să ne contactați la adresa vanzari@mediafax.ro. & 464 & high & High & Power & Governance & NA & 2017-01-31 & 2017 & 2 & POL
Frame & high-very high & National & <500 & 1.0255497 & 0.6327204 & -1.4131940 & 0.5015415 & -0.9523600 & 0.0 & -1.0024021 & 0.3314703 & Recipient & Domestic & European & Mixed & Domestic|POL & Negative\\
Romania & http://www.zvj.ro/articole-36480-Vineri\%2B\%2Bla\%2BBiblioteca\%2BUniversit\%2B\%2B\%2B\%2Bii\%2Bdin\%2BPetro\%2B\%2Bani\%2B\%2Bin\%2Bbaza\%2Bunui\%2Bproiect\%2Beuropean\%2Baflat\%2Bin\%2Bderular.html & 537 & Gazeta Vaii Jiului & Private/Non-Public & Online and Offline & Regional/Local & very low = CP mentioned once & Social justice & Positive & Subnational & No myth & NA & NA & NA & NA & NA & NA & NA & NA & Romania & vineri, la biblioteca universitatii din petrosani, in baza unui proiect european aflat in derulare, euro jobs srl petrosani a organizat un workshop in urma caruia am aflat despre beneficiile/avantajele participarii la activitatea de formare profesionala continua (fpc) & 2015-08-30 & fondul social european & câteva zeci de persoane au ţinut să participe vineri, 28 august, la workshopul intitulat "beneficiile/avantajele participării la activitatea de fpc" din cadrul proiectului european "casse - calificarea angajaţilor - servicii la standarde europene", proiect aflat în derulare şi implementat de sc brahms internaţional srl sibiu, în parteneriat cu fundaţia "şcoala comercială şi de servicii" bacău şi sc euro jobs srl petroşani, cofinanţat din fondul social european prin posdru 2007-2013 "investeşte în oameni!". & 72 & very low & Low & Socio-Economic & NA & NA & 2015-08-30 & 2015 & 1 & ECO
Frame & v.low & Regional & <500 & 1.0255497 & 0.6327204 & -1.4131940 & 0.5015415 & -0.9523600 & 0.0 & -1.0024021 & 0.3314703 & Recipient & Domestic & Domestic & Domestic & Domestic|ECO & Positive\\
Romania & https://www.desteptarea.ro/programul-comisarul-european-corina-cretu-bacau/ & 579 & Deșteptarea- Ziarul Bacăului & Private/Non-Public & Online only & Regional/Local & medium = CP is important part of story & Infrastructure & Balanced & EU & No myth & NA & NA & NA & NA & NA & NA & NA & NA & Romania & programul comisarul european corina crețu în bacău | deșteptarea- ziarul bacăului & 2018-05-19 & politica regională & oficialul european va fi însoțit de premierul viorica dăncilă și de ministrul fondurilor europene, rovana plumb comisarul european pentru politică regională, corina crețu, premierul viorica dăncilă și ministrul fondurilor europene, rovana plumb se vor afla în zilele de 21 și 22 mai în bacău. pe 21 mai, oficialul european și cei din guvernul româniei vor vizita, de la ora 13.30, proiectul insula de agrement din bacău, după care corina crețu și viorica dăncilă vor avea o întâlnire bilaterală. programul va continua cu o conferință pe tema "investițiile urbane în europa: provocare pentru schimbare" și se va încheia cu o vizită la proiectul cu finanțare europeană "restaurarea și valorificarea patrimoniului cultural: observatorul astronomic bacău". pe 22 mai, universitatea "vasile alecsandri" din bacău va fi gazda dialogului cu cetățenii despre viitorul europei și al politicii de coeziune, interlocutorii fiind corina crețu și rovana pumb. evenimentul va fi moderat de jurnalistul dan cărbunaru, directorul site-ului calea europeană.ro. comisarul european trage de urechi guvernanții români recent, corina crețu a trimis o scrisoare guvernanților din românia, în care face o evaluare a activității pe domeniul transporturilor pentru anul trecut. viorica dăncilă a anunțat că va discuta această temă tot la bacău cu oficialul european. "pot să vă spun - a precizat premierul - că am vorbit deja cu ministrul transporturilor pentru a vedea cum accelerăm procesul de absorbție a fondurilor europene". corina crețu se arată extrem de îngrijorată cu privire la planificarea şi implementarea proiectelor de infrastructură de transport, avertizând că există riscul major al dezangajării unor fonduri consistente alocate româniei în acest exerciţiu bugetar al uniunii europene. dar, ministrul transporturilor, lucian șova, a declarat că scrisoarea comisarului european corina creţu face referire la "perioade anterioare exerciţiului pe care mt îl are în ultimul an". până în prezent, numai patru proiecte majore au fost depuse într-un interval de timp de patru ani. recent, autoritățile române au anunțat o listă de proiecte majore de transport pe care doresc să le înainteze comisiei în 2018, printre care și varianta de ocolire bacău. "salut faptul că noi proiecte sunt în curs de pregătire spre a fi înaintate comisiei - a scris corina crețu -, dar îmi exprim îngrijorarea față de nivelul scăzut de maturitate al acestora". comisarul european mai precizează că din această cauză țara noastră riscă "dezangajări imediate". & 381 & medium & Medium & Socio-Economic & NA & NA & 2018-05-19 & 2018 & 3 & ECO
Frame & low-medium & Regional & <500 & 1.0255497 & 0.6327204 & -1.4131940 & 0.5015415 & -0.9523600 & 0.0 & -1.0024021 & 0.3314703 & Recipient & European & European & European & European|ECO & Neutral\\
Romania & http://ziuadecj.realitatea.net/administratie/probleme-pe-drumul-rachitele-ic-ponor-vezi-ce-reclama-constructorii--141342.html & 525 & ziuadecj.realitatea.net & Private/Non-Public & Online only & Regional/Local & very low = CP mentioned once & Mismanagement & Negative & Subnational & No myth & Bureaucracy and/or delays & Negative & Subnational & 10.Slow spend & Infrastructure & Factual & Subnational & No myth & Romania & probleme pe drumul rachitele-ic ponor. vezi ce reclama constructorii & 2015-09-24 & fondul european de dezvoltare regională & constructorii drumului turistic răchiţele ic ponor au trimis o adresă către raadpp şi consiliul judeţean în care vorbesc despre faptul că au apărut noi probleme legate de reabilitatea traseului. regia autonomă de administrare a domeniului public și privat a județului (raadpp) cluj, entitate subordonată consiliului județean (cj) şi care se ocupă de implementarea acestui proiect pe fonduri europene, informează că a fost înștiințată de asocierea drumuri şi poduri judeţene (dpj) cluj - cridov srl (foto) de faptul că ritmul lucrărilor la obiectivul "modernizarea infrastructurii de acces în zona turistică răchițele - prislop - ic ponor" va fi încetinit. motivul invocat de asociere cu privire la această măsură este neplata facturilor de către cj pentru lucrările aferente perioadelor 24 iulie - 15 august (1.807.906 lei, cu taxa pe valoarea adăugată inclusă) și 16-31 august (1.959.967 lei, tva inclusă), în condițiile în care, până în prezent, valoarea lucrărilor a fost suportată de către constructor. "raadpp a făcut către cj toate demersurile și adresele necesare pentru ca plățile lucrărilor executate de asociere la obiectivul "modernizarea infrastructurii de acces în zona turistică răchițele - prislop - ic ponor" să fie onorate la timp, potrivit unui comunicat de presă al regiei. regia a arătat că stadiul lucrărilor la 22 septembrie, verificate de către aceasta prin unitatea de implementare a proiectului, este următorul: pe sectorul de drum județean desemnat spre execuție societății drumuri şi poduri judeţene cluj (14 km) s-a turnat primul strat de covor asfaltic pe 9,77 km, pe ambele sensuri de circulație, și cel de al doilea strat de asfalt (de uzură) pe 6,57 km, pe ambele sensuri de circulație, iar pe sectorul de drum județean desemnat spre execuție cridov srl (6 km) erau executate lucrări de așternere a primul strat de asfalt pe 5,3 km, pe ambele sensuri de circulație. contactat de ziua de cluj, preşedintele cj cluj, mihai seplecan, a declarat că vicepreşedintele vakar istvan a primit atribuţii să se ocupe de plăţile pentru acest contract şi că legea va fi respectată, iar facturile vor fi plătite în termenul legal. vakar nu a putut fi contactat pentru a exprima un punct de vedere. el este plecat din ţară şi urmează să se întoarcă la sfârşitul acestei săptămâni. "factura a fost trimisă în data de 9 septembrie şi a ajuns la cj în 14 septembrie. din data de 14 şi până în 21 septembrie raadpp nu a dat "bun de plată". mă întreb atunci cine sabotează lucrările la drumul ic ponor? de ce nu a ajuns factura la noi în 10 septembrie şi tot atunci să fie dat şi "bun de plată"? între 14 şi 21 septembrie ce a făcut domnul mircea avram, directorul raadpp? totul a fost făcut concertat şi intenţionat de către domnul avram ca să pună presiune pe mine, ştiind că domnul vakar a fost împuternicit să semneze şi că în această perioadă este plecat din ţară. domnul vakar va veni în 28 septembrie la birou şi va semna ordinul de plată. oricum termenul limită pentru plata facturii este data de 9 octombrie", a mai declarat seplecan. mircea avram, directorul raadpp, afirmă că prima factură a fost trimisă de constructori direct la cj cluj în 8 septembrie, că documentul a primit "bun de plată" în aceeaşi dată. "noi avem tot interesul ca lucrările să se facă cât mai repede. fiecare zi contează în această perioadă. în data de 21 septembrie a fost trimisă într-adevăr o factură, dar cea de-a doua, pentru că sunt două facturi, dar nu a fost plătită nici prima, cea din 8 septembrie. noi, prin diriginţii de şantier, am semnat situaţiile de lucrări şi le-am trimis la cj cluj pentru ambele facturi. mai trebuiau semnate ordinele de plată de către cj cluj", a mai precizat directorul mircea avram. lucrările care au mai rămas de efectuat pe acest drum au o lungime de 20,5 km şi au fost evaluate la 18,2 milioane de lei. noul constructor a câştigat licitaţia organizată în acest sens cu o propunere financiară de 13,3 milioane de lei. licitaţia iniţială a fost câştigată în 2009 de asocierea dintre firmele mbs group turda, alfa rom satu mare şi nemzetkozi betonut kft. din ungaria, pentru 34 de milioane de lei. între constructor şi beneficiar (cj cluj) au apărut numeroase neînţelegeri legate de suplimentarea valorii lucrării, care au dus în 2013 la rezilierea contractului. cj a cerut o expertiză asupra lucrărilor efectuate şi în urma prezentării concluziilor acesteia, care arătau că ar fi fost făcute lucrări care nu au respectat standardele de calitate cerute, a depus şi o sesizare la departamentul de luptă antifraudă şi la direcția națională anticorupție. între timp, o parte din lucrările efectuate s-au degradat şi trebuie refăcute din banii cj cluj, prin regia de administrare a domeniului public şi privat. potrivit directorului raadpp cluj, mircea avram, valoarea acestora se ridică la 1,2 milioane lei, fără taxa pe valoarea adăugată, şi vor fi executate în prima etapă de către noul constructor. în a a doua etapă se va turna efectiv asfalt şi se vor face celelalte lucrări de amenajare: rigole, parapeţi, indicatoare. finanțarea investiției este asigurată prin programul operațional regional, din fondul european de dezvoltare regională, bugetul central de stat şi bugetul propriu al cj cluj. pentru a nu pierde banii europeni reabilitarea drumului turistic trebuie terminată până la sfârşitul anului. & 888 & very low & Low & Governance & Governance & Socio-Economic & 2015-09-24 & 2015 & 1 & POL
Frame & v.low & Regional & 500-1000 & 1.0255497 & 0.6327204 & -1.4131940 & 0.5015415 & -0.9523600 & 0.0 & -1.0024021 & 0.3314703 & Recipient & Domestic & Domestic & Domestic & Domestic|POL & Negative\\
\addlinespace
Romania & https://www.dcnews.ro/rovana-plumb-cere-la-bruxlles-buget-pe-masura-post-2020\_537905.html & 551 & dcnews.ro & Private/Non-Public & Online only & National & very low = CP mentioned once & Institutional bargaining over funding & Balanced & EU & No myth & NA & NA & NA & NA & NA & NA & NA & NA & Romania & rovana plumb cere la bruxlles buget pe măsură post-2020 & 2017-03-31 & politica de coeziune & rovana plumb, ministru delegat pentru fonduri europene, aflată la bruxelles, a avut o întâlnire cu jyrki katainen, vicepreședinte al comisiei europene, responsabil cu ocuparea forței de muncă, creștere economică, investiții și competitivitate. cei doi oficiali au discutat, printre altele, despre creșterea vizibilității fondului european de investiții strategice (efsi) în românia, îmbunătățirea accesului la finanțare pentru sectorul imm-urilor și despre utilizarea combinată a granturilor și a finanțării din efsi. rovana plumb a arătat că finanțarea imm-urilor din efsi este o prioritate pentru românia, în sectoare cheie precum agricultura, cel forestier sau turismul. referitor la politica de coeziune, ministrul român a arătat că țara noastră dorește o politică ambițioasă post-2020, cu un buget pe măsură. vicepreședintele katainen a subliniat, la rândul său, importanța continuării politicii de coeziune, efsi și fondurile esi fiind complementare. a adăugat că, pentru românia, îmbunătățirea capacității administrative și sustenabilitatea proiectelor de importanță strategică națională ar trebui să reprezinte o prioritate. [citeste si] la bruxelles, în programul ministrului delegat pentru fonduri europene a fost prevăzută o întâlnire și cu doamna elzbieta bieńkowska, comisar european pentru piața internă, industrie, antreprenoriat și imm-uri. doamna plumb a transmis, în cadrul întrevederii, angajamentul ferm al guvernului româniei în ceea ce privește implementarea măsurilor necesare pentru creșterea absorbției fondurilor europene, inclusiv cele legate de implementarea strategiei naționale în domeniul achizițiilor publice, agreată cu comisia europeană. comisarul bieńkowska a apreciat calitatea strategiei naționale pe achiziții publice pentru românia și a arătat că absorbția și buna utilizare a fondurilor ridică provocări majore legate de îndeplinirea condiționalităților ex-ante pe achiziții publice. dialogul între comisie și autoritățile române va continua pentru asigurarea de progrese semnificative până la autoevaluarea prevăzută pentru luna iunie 2017. & 280 & very low & Low & Power & NA & NA & 2017-03-31 & 2017 & 2 & POL
Frame & v.low & National & <500 & 1.0255497 & 0.6327204 & -1.4131940 & 0.5015415 & -0.9523600 & 0.0 & -1.0024021 & 0.3314703 & Recipient & European & European & European & European|POL & Neutral\\
Romania & http://www.euractiv.ro/video/Best-practice-made-in-Romania-Proiecte-de-cercetare-cu-fonduri-UE-708 & 568 & EurActiv | Știri, politici europene \& Actori UE online & Private/Non-Public & Online only & National & medium = CP is important part of story & Research \& innovation & Positive & EU + Subnational & No myth & NA & NA & NA & NA & NA & NA & NA & NA & Romania & best practice made in romania: proiecte de cercetare cu fonduri ue & 2015-06-10 & fondul european de dezvoltare regională & proiectul de cercetare privind bolile infecțioase derulat de institutul matei balș și proiectul eli al măgurele au fost vizitate de eurodeputații din comisia regi a parlamentului european. potrivit acestora, proiectele pot deveni exemple de succes ale utilizării finanțărilor ue în românia, de folos pentru alte țări precum noua membră ue, croația. constanze krehl, eurodeputat s\&d în comisia pentru dezvoltare regionala (regi) a parlamentului european, a explicat în cadrul unei conferințe de presă la bucurești că o serie de proiecte finanțate prin fondul european de dezvoltare regională pot fi date exemplu la nivelul ue. deputații s\&d din comisia regi au vizitat, în românia, două proiecte din domeniul cercetării: institutul matei balș - care derulează un proiect de cercetare medicală, proiectul eli -np măgurele dar și alte tipuri de proiecte, precum stația de tratare a apei reziduale de la glina și proiecte finanțate în rezervația biosferei delta dunării. mihnea costoiu, fost ministru român al educației și cercetării a spus că sunt foarte multe proiecte de cercetare in românia în domeniul sănătății, pe lângă cel de la matei balș. în plus, de dorește ca măgurele să devină un pol de creștere, "un oraș al științei". acesta a mai spus că programul de cercetare al româniei a fost "extrem de apreciat la nivelul comisiai europene" și că nu a suferit niciodată corecții financiare". tonino picula, de asemenea eurodeputat în comisia regi a spus că proiectele româniei pot deveni exemplu pentru croația, nou membră a uniunii europene, de rezultate ale utilizării finanțărilor europene. & 249 & medium & Medium & Socio-Economic & NA & NA & 2015-06-10 & 2015 & 1 & ECO
Frame & low-medium & National & <500 & 1.0255497 & 0.6327204 & -1.4131940 & 0.5015415 & -0.9523600 & 0.0 & -1.0024021 & 0.3314703 & Recipient & Domestic & European & Mixed & Domestic|ECO & Positive\\
Romania & http://www.business24.ro/libra-bank/investitii/libra-internet-bank-va-acorda-credite-avantajoase-pentru-imm-uri-de-pana-la-47-7-milioane-de-euro-1588568 & 505 & Business24.ro & Private/Non-Public & Online only & National & medium = CP is important part of story & Jobs & Positive & National & No myth & NA & NA & NA & NA & NA & NA & NA & NA & Romania & libra internet bank va acorda credite avantajoase pentru imm-uri de pana la 47,7 milioane de euro & 2017-10-23 & fondul european de dezvoltare regională & libra internet bank a semnat un acord cu fondul european de investitii (fei) in cadrul programului "initiativa pentru imm-uri". in baza acestui acord libra internet bank va acorda credite in valoare de pana la 47,7 milioane euro imm-urilor in conditii avantajoase, pentru finantarea activitatii curente sau a investitiilor. "ne bucura extinderea acordului cu fondul european de investitii si pentru programul initiativa pentru imm-uri, data fiind experienta deja acumulata in anii anteriori pentru programele easi, cosme si progress microfinance sau programul dedicat finantarii sectorului cultural si creativ. imm-urile reprezinta un segment important pentru libra internet bank si ne dorim sa dezvoltam in continuare produse de finantare si garantare accesibile, in acord cu nevoile acestora", a declarat emil bituleanu, director general al libra internet bank. tranzactiile imm beneficiaza de sprijinul uniunii europene prin programul initiativa pentru imm-uri, finantat de uniunea europeana prin fedr si horizon 2020 si de catre fondul european de investitii si banca europeana de investitii. prin programul initiativa pentru imm-uri este sustinuta cresterea accesului la finantare al companiilor, care beneficiaza astfel de garantii de 60\% din valoarea creditului de la fondul european de investiti, dar si de o dobanda redusa. despre fondul european de investitii fondul european de investitii (fei) face parte din grupul bancii europene de investitie. misiunea sa principala este sa sustina microintreprinderile si intreprinderile mici si mijlocii din europa, ajutandu-le sa obtina acces la finantare. fei proiecteaza si dezvolta capital de risc si de crestere, garantii si instrumente de microfinantare care se concentreaza exact pe acest segment de piata. in aceasta calitate, fei favorizeaza obiectivele ue pentru sustinerea inovatiei, cercetarii si dezvoltarii, antreprenoriatului, cresterii si ocuparii fortei de munca. despre libra internet bank libra internet bank are o cota de piata de 1\%, in primele sase luni ale anului 2017 inregistrand active totale in valoare de 4,15 miliarde de lei si un profit net de 23,96 de milioane de lei. libra internet bank are o retea teritoriala de 50 de sucursale care deservesc clienti din segmentele profesii liberale, imm, corporate, real estate si agribusiness, precum si un canal digital care ofera clientilor servicii online. despre initiativa pentru imm-uri programul initiativa pentru imm-uri al uniunii europene este gestionat de catre comisia europeana, prin intermediul bancii europene de investitii si fondului european de investitii. programul este finantat de uniunea europeana prin fondul european de dezvoltare regionala (fedr), programul de finantare a inovatiei si cercetarii horizon 2020, fondul european de investitii si banca europeana de investitii. ti-a placut acest articol? urmareste business24 si pe facebook! comenteaza si vezi in fluxul tau de noutati de pe facebook cele mai noi si interesante articole de pe business24. & 449 & medium & Medium & Socio-Economic & NA & NA & 2017-10-23 & 2017 & 2 & ECO
Frame & low-medium & National & <500 & 1.0255497 & 0.6327204 & -1.4131940 & 0.5015415 & -0.9523600 & 0.0 & -1.0024021 & 0.3314703 & Recipient & Domestic & Domestic & Domestic & Domestic|ECO & Positive\\
Romania & https://romanialibera.ro/politica/teodorovici-despre-autostrada-iasi-targu-mures-nu-se-exclude-nicio-varianta-de-finantare-760262 & 577 & RomaniaLibera.ro & Private/Non-Public & Online and Offline & National & medium = CP is important part of story & Infrastructure & Positive & EU + National & No myth & NA & NA & NA & NA & NA & NA & NA & NA & Romania & teodorovici, despre autostrada iaşi-târgu mureş: nu se exclude nicio variantă de finanţare | romania libera & 2018-11-08 & politica regională & întrebat dacă sunt bani pentru autostrada iaşi-târgu mureş, teodorovici a spus: "mesajul este foarte clar: avem bani pentru investiţii", citează agerpres. potrivit lui teodorovici, compania naţională de autostrăzi trebuie să definitiveze documentaţia de atribuire şi să o introducă în sistemul electronic de achiziţii publice. "ceea ce trebuie să se facă şi am spus-o şi celor care au propus şi au votat şi care solicită astfel de lucruri de infrastructură - să se concentreze foarte mult pe faptul că cei de la compania naţională de autostrăzi să definitiveze documentaţia de atribuire şi să o ridice în sistemul electronic de achiziţii publice. acela este cel mai bun argument, cea mai bună dovadă că într-adevăr se doreşte ca un astfel de obiectiv să se construiască", a explicat teodorovici. el a susţinut că nu încurcă cu nimic faptul că obiectivul construirii autostrăzii a fost stabilit de parlament, şi nu de guvern. "nu încurcă absolut deloc, dar vă spun, ca principiu, că există acea discuţie între trimiterea la comisie a unor cereri de finanţare(...), cel mai important este să faci documentaţia de atribuire, să o lansezi ca şi achiziţie publică, să dai drumul la lucrare şi este la fel de important să pregăteşti documentaţia pentru ca orice cheltuială se face pe acel obiectiv să nu rămână pe bugetul de stat şi ea să poată să fie rambursată pe fonduri europene. acesta este mecanismul pe care trebuie să-l aplicăm", a mai spus ministrul finanţelor. întrebat din ce bani ar urma să se construiască autostrada, din fonduri europene sau parteneriat public privat, teodorovici a precizat că se poate discuta asupra "montajului financiar", care poate fi definitivat în momentul în care documentaţia este gata să fie lansată în sistemul electronic de achiziţii publice. "finanţarea poate să difere de la surse publice, buget de stat, împrumuturi bei, berd sau alte instituţii financiare sau un concept de implicarea privatului. nu se exclude niciuna dintre variante, pot să fie toate prezente şi fonduri europene, şi buget de stat şi privat", a adăugat eugen teodorovici. el a mai spus că acest obiectiv se va face, dar nu poate da un termen. "obiectivul se face. nu ştiu când, până nu ai un caiet de sarcini şi un contract de lucrări semnate, ca să ştii foarte clar ce şi-a asumat constructorul din proiectul scos la licitaţie, vorbim dorinţe, sunt dorinţe", a adăugat acesta. "autostrada târgu mureş-iaşi-ungheni poate fi eligibilă pentru cofinanţare din fondurile europene structurale şi de investiţii şi este responsabilitatea autorităţilor române să pregătească un proiect în acest sens, a declarat, pentru agerpres, comisarul european pentru politică regională, corina creţu. ''autostrada târgu mureş-iaşi-ungheni poate fi eligibilă pentru cofinanţare din fondurile europene structurale şi de investiţii, având în vedere faptul că este situată în reţeaua ten-t de bază şi este totodată parte a master planului general de transport. mai mult decât atât, accesibilitatea regiunii de nord-est a româniei este foarte importantă, zona urbană iaşi este cea mai populată din românia, după bucureşti'', a precizat oficialul european. proiectul de lege privind aprobarea obiectivului de investiţii autostrada iaşi - târgu mureş (numită şi "autostrada unirii"), iniţiat de parlamentari ai partidului mişcarea populară (pmp) şi semnat de reprezentanţi ai tuturor formaţiunilor politice, a fost adoptat miercuri camera deputaţilor, care este for decizional. potrivit proiectului, autostrada unirii începe la graniţa româniei cu republica moldova printr-un nou pod peste râul prut şi se termină printr-o conexiune cu autostrada a3 braşov - borş, în apropierea oraşului târgu mureş. autostrada unirii se finanţează de la bugetul de stat, prin bugetul ministerului transporturilor, din credite externe şi din fonduri europene nerambursabile şi/sau prin parteneriat public-privat. în fiecare lege bugetară anuală, pe durata implementării proiectului, se vor aloca acestui obiectiv credite bugetare conform necesarului de finanţare comunicat de ministerul transporturilor. ministerul transporturilor este responsabil pentru coordonarea realizării autostrăzii unirii şi va demara, în termen de 30 de zile de la intrarea în vigoare a legii, procedurile necesare pentru acest obiectiv de investiţii. & 661 & medium & Medium & Socio-Economic & NA & NA & 2018-11-08 & 2018 & 3 & ECO
Frame & low-medium & National & 500-1000 & 1.0255497 & 0.6327204 & -1.4131940 & 0.5015415 & -0.9523600 & 0.0 & -1.0024021 & 0.3314703 & Recipient & Domestic & European & Mixed & Domestic|ECO & Positive\\
Romania & https://ziuadecj.realitatea.net/mica-publicitate/maiestria-metalului-in-ambientul-modern--184208.html & 502 & ziuadecj.realitatea.net & Private/Non-Public & Online only & Regional/Local & very low = CP mentioned once & Jobs & Positive & Subnational & No myth & NA & NA & NA & NA & NA & NA & NA & NA & Romania & "măiestria metalului în ambientul modern" & 2019-03-27 & fondul european de dezvoltare regională & societatea riela comimpex srl, cu sediul în cluj napoca, bulevardul muncii nr 18, specializată în productia de mobilier metalic, a implementat în perioada 16.03.2018 - 31.03.2019, proiectul "maiestria metalului in ambientul modern", cod smis 111399 în cadrul programului operaţional regional (por 2014-2020), axa prioritară îmbunatățirea competitivității întreprinderilor mici și mijlocii, operațiunea consolidarea poziției pe piată a imm-urilor în domeniile competitive identificate în snc si pdr-", componenta 2.2. sprijinirea creării și extinderea capacităților avansate de producție și dezvoltarea serviciilor. proiectul are o valoare totală de 7.573.650,30 lei, cu o finanţare nerambursabilă în valoare de 4.515.348.,35 lei, din care 3.838.046,10 lei prin programul operaţional regional din fondul european de dezvoltare regională şi 677.302,25 lei de la bugetul de stat, în baza contractului de finanţare încheiat cu "ministerul dezvoltarii regionale, administratiei publice si fondurilor europene" în calitate de autoritate de management, prin "agenția de dezvoltare regionala nord vest" în calitate de organism intermediar. obiectivele specifice ale proiectului, respectiv 1. diversificarea productiei, inclusiv extinderea capacitatii, prin investitii în 13 active fixe de ultima generatie, un sistem de panouri solare si un software pentru gestionarea productiei; cresterea vizibilitatii si a prezentei în piata prin participarea la un târg international, certificarea a 2 sisteme de management si certificarea produselor; crearea a 5 locuri noi de munca din care unul va fi ocupat de o persoana din categoria persoanelor defavorizateau fost realizate. rezultatele proiectului se reflectă în diversificarea și creșterii capacității de producție, cu ajutorul echipamentelor noi achizitionate. impactul proiectului: cele 5 noi locuri de muncă create, dintre care unul este ocupat de o persoană categoria persoanelor defavorizate contribuie, alături de creșterea productivității muncii începând cu al doilea an de la finalizarea implementării proiectului, la creșterea veniturilor bugetare și la un standard de viață mai bun. tel. +40 - 728.262717, e-mail: office@rielacomimpex.ro proiectul de investiţii a fost implementat în municipiul cluj napoca, bulevardul muncii nr 18. & 329 & very low & Low & Socio-Economic & NA & NA & 2019-03-27 & 2019 & 3 & ECO
Frame & v.low & Regional & <500 & 1.0255497 & 0.6327204 & -1.4131940 & 0.5015415 & -0.9523600 & 0.0 & -1.0024021 & 0.3314703 & Recipient & Domestic & Domestic & Domestic & Domestic|ECO & Positive\\
\addlinespace
Romania & https://ziuadecj.realitatea.net/administratie/consiliul-judetean-ar-putea-prelua-proiectul-spitalului-regional-de-urgenta-vrem-sa-accesam-in-mod-direct-acesti-bani-fara-sa-mai-tinem-de-guvern--180100.html & 517 & ziuadecj.realitatea.net & Private/Non-Public & Online only & Regional/Local & very high = CP is most important issue + CP is mentioned in title/headline & Ineffective goal achievement & Negative & EU + National + Subnational & No myth & Mismanagement & Negative & National & No myth & Improve governance & Negative & National & No myth & Romania & consiliul judeţean ar putea prelua proiectul spitalului regional de urgenţă: "vrem să accesăm în mod direct aceşti bani, fără să mai ţinem de guvern & 2018-11-21 & politica regională & în condiţiile în care românia ar putea pierde finanţarea europeană pentru realizarea celor trei spitale regionale, dat fiind faptul că ministerul sănătăţii şi guvernul psd nu arată nici un interes în demararea lor, consiliile judeţene din cluj, iaşi şi craiova ar putea trece la cârma proiectelor. clujul a făcut un prim pas în acest sens. comisarul european corina creţu afirmă, la finele lunii octombrie 2018, că guvernul dăncilă tărăgănează accesarea banilor europeni necesari pentru construcția spitalelor regionale din cluj, iași și craiova. "sunt mâhnită că discuţiile despre un obiectiv atât de important se duc într-o zonă a speculaţiilor şi disputelor. realizarea acestor spitale regionale reprezintă un proiect de suflet pentru mine, ca român, și tocmai de aceea am adus la masa negocierilor toți actorii cheie, încă din luna mai 2016, când am avertizat asupra întârzierilor și mi-am exprimat disponibilitatea să ajutăm autoritățile naționale și regionale cu absolut tot ce se poate". reacţia comisarului european venea după ce consilierul premierului viorica dăncilă, darius vâlcov, a scris pe facebook, că e mai avantajos ca unele proiecte să fie făcute în parteneriat-public privat în românia, în condiţiile în care banca europeană de investiţii (bei) elaborează studii în valoare de 120 milioane de euro. mai multe detalii aici! ulterior, în urma unei întâlniri între corina creţu şi viorica dăncilă, s-a convenit ca guvernul să transmită comisiei europene propunerea româniei în ceea ce privește modalitatea prin care se doreşte continuarea finanţării acestor 3 spitale."totodată, în cadrul întâlnirii noastre de la bucurești s-a propus şi ideea etapizării, adică realizarea acestor spitale în două etape, documentația în actuala perioadă de programare şi construcția efectivă în perioada 2021-2027, idee prezentată de ministru rovana plumb", a declarat, la început de noiembrie 2018, corina creţu. "consiliile judeţene din cele trei județe trebuie să preia inițiativa și să pregătească documentaţia" cum guvernul nu a mai revenit cu un răspuns pentru comisia europeană, europarlamentarul pnl marian-jean marinescu solicită consiliilor judeţene din cluj, iaşi şi craiova să preia inițiativa în cazul proiectelor privind spitalele regionale, considerând că autoritățile județene ar trebui să pregătească documentația în acest sens pentru a nu se mai pierde timp. "ministerul sănătăţii ar fi trebuit să întocmească studii de fezabilitate și să scrie proiecte pentru finanțarea construirii, prin programul operaţional regional (por) al ue, a unor spitale în craiova, iași şi cluj napoca. proiectele trenează, iar ministerul sănătății nu arată nici un interes în demararea lor. de aceea, solicit consiliilor judeţene din cele trei județe să preia inițiativa și să pregătească documentaţia pentru proiectele de spitale; solicit ministrului sănătății, sorina pintea, ministerului sănătăţii și guvernului să accepte ca aceste proiecte să fie derulate de beneficiari, adică de consiliile județene din craiova, iași și cluj napoca; solicit guvernului să modifice programul operațional regional, astfel încât proiectele să fie scrise și depuse direct de beneficiari", a precizat europarlamentarul într-o scrisoare transmisă ministrului sănătății și comisarului european pentru politică regională corina creţu. alin tişe: "am trimis solicitare către ministerul sănătăţii şi corina creţu. suntem capabili" clujul s-a arătat deschis acestei propuneri şi a făcut un prim pas. "nu numai că am dicutat, dar în calitate de preşedinte al consiliului judeţean (cj) cluj am şi semnat o adresă către ministerul sănătăţii şi către corina creţu prin care, văzând incompetenţa crasă a guvernului şi văzând că a abandonat din nou proiectele spitalelor regionale, am cerut ceea ce aţi citit dumneavoastră din acel comunicat de presă. adică, noi solicităm ca la nivelul judeţului cluj consiliul judeţean cluj să devină titularul finanţării pe programul operațional regional, ceea ce presupune o modificare a ghidului acceptat de comisia europeană, tocmai pentru ca noi, în calitate de proprietari ai terenului pe care îl avem acolo, să putem accesa în mod direct aceşti bani fără să mai ţinem de guvern", a declarat preşedintele cj cluj, alin tişe, în cadrul matinalului realitatea fm cluj. alin tişe a amintit că, în urmă cu puţin timp, cj cluj a predat terenul din floreşti către ministerul sănătăţii, în suparfaţă de 14 hectare, cu scopul clar precizat în hotărârea de consiliu judeţean de a fi folosit pentru construirea viitorului spital regional de urgenţă. "în condiţiile în care ministerul sănătăţii nu construieşte spitalului în termenul convenit, noi, în proxima şedinţă a forului judeţean, după ce avem răspuns de la corina creţu şi ministerul sănătăţii, analizăm posibilitatea de retrage dreptul de proprietate pe care l-am dat ministerului sănătăţii, astfel ca terenul să revină la cluj şi cu acestă modificare a ghidului programului operaţional regional să putem noi construi acest spital". alin tişe mai spune că administraţia judeţeană este capabilă să acceseze banii europeni, având în vedere că a reuşit să obţină finanţări europene de peste 250 de milioane de euro doar în ultimii 2 ani. ultimul termen înaintat - 2013, greu de respectat în vara lui 2018, ministrul sănătăţii promitea că spitalul regional de urgenţă din cluj ar urma să fie operaţional la finalul anului 2023, investiţia totală ridicându-se la o jumătate de miliard de euro, bani alocaţi din fonduri europene, guvernamentale, judeţene şi locale. deocamdată, au rămas doar promisiunile. aproape 900 de paturi conform planurilor, spitalul are o suprafaţă totală de 120.621 mp, din care cea mai mare parte (aprox. 49.000 mp) pentru servicii medicale şi chirurgicale. alţi 19.000 mp vor fi destinaţi serviciilor de tratament. unitatea medicală ce urmează să fie construită în cea mai mare comună din ţară va dispune de un număr de 849 de paturi de spitalizare continuă şi alte 64 de spitalizare de zi. de asemenea, va fi echipată cu aparatură de înaltă tehnologie care va permite efectuarea de proceduri minim invazive, chiar în cazuri dificile, complexe. organizarea fluxului de pacienţi în spital va permite creşterea calităţii îngrijirilor, a satisfacţiei pacientului, raţionalizarea resurselor, dar şi dezvoltarea şi optimizarea serviciilor de spitalizare de zi şi a celor ambulatorii. noul spital regional de urgenţă cluj va prelua rolul spitalului clinic judeţean de urgenţă şi servicii medicale din alte unităţi sanitare. investiţie de 500 milioane euro costurile totale ale spitalului din cluj se ridică la suma de 500 de milioane de euro cu tva, banii fiind alocaţi din fonduri europene, bugetul de stat, iar pentru utilităţi, de la autorităţi locale şi judeţene. conform ministrului sănătăţii, în costurile realizării spitalului au fost incluse şi banii necesari reabilitării şi reorganizării clădirilor în care funcţionează, în prezent, spitalul clinic judeţean de urgenţă. bani de la ue amintim că uniunea europeană acordă fonduri în valoare de 150 de milioane de euro pentru construirea celor trei spitale regionale din românia - de la cluj, iaşi şi craiova. pentru ca această finanţare să rămână valabilă este obligatoriu ca recepţia obiectivelor să fie finalizată până în anul 2023. trei drumuri de acces spre spitalul din cluj ziua de cluj scria, în mai 2018, că trei drumuri ar urma să lege clujul de noul sru: unul din dn1, unul dinspre metro, pe un drum median nou, și unul din centura de sud a floreștiului. propunerile au fost prezentate de arhitectul șef al județului cluj, claudiu salanță, la comisia tehnică de amenajare a teritoriului și urbanism (ctatu). problema nu este nu este clarificată nici în prezent. totul este la mâna administrației locale din florești. mai multe aici. proiectul spitalului din cluj, vechi din 2004 proiectul de construire a unui spital regional de urgenţă a fost lansat în mandatul de preşedinte al consiliului judeţean (cj) cluj al liberalului marius nicoară, începând din 2004, fiind vorba de o unitate care să deservească judeţele din regiunea de nord-vest. cj a identificat atunci un teren pentru acest spital în zona câmpeneşti, aparţinând comunei apahida, situată în estul municipiului cluj-napoca, dar proiectul nu a fost prins la finanţare din partea guvernului. după 2008, când s-a schimbat conducerea cj, noul preşedinte al instituţiei alin tişe (pdl) a schimbat şi amplasamentul spitalului, în comuna floreşti, situată în vestul municipiului cluj-napoca, dar lipsa finanţării guvernamentale a împiedicat şi de această dată începerea construirii sru & 1320 & very high & High & Socio-Economic & Governance & Governance & 2018-11-21 & 2018 & 3 & ECO
Frame & high-very high & Regional & +1000 & 1.0255497 & 0.6327204 & -1.4131940 & 0.5015415 & -0.9523600 & 0.0 & -1.0024021 & 0.3314703 & Recipient & Domestic & European & Mixed & Domestic|ECO & Negative\\
Romania & http://ziuadecj.realitatea.net/cultura/propuneri-pentru-identitatea-istorica-a-clujului-patinoar-pe-lacul-din-parc-revitalizarea-cimitirului-central-sauredactarea-unui-ghid-turistic--148542.html & 515 & ziuadecj.realitatea.net & Private/Non-Public & Online only & Regional/Local & low = CP mentioned more times but NOT important part of story (mainly about others issues) & Cultural development & Balanced & Subnational & No myth & NA & NA & NA & NA & NA & NA & NA & NA & Romania & propuneri pentru identitatea istorica a clujului: patinoar pe lacul din parc, revitalizarea cimitirului central sau...redactarea unui ghid turistic & 2016-06-30 & fondurile structurale & cultura este una din principalele direcţii de dezvoltare a municipiului cluj-napoca, atât din punct de vedere economic cât şi al imaginii oraşului la nivel european, cel puţin până în anul 2020. iar conturarea identităţii istorice a urbei joacă un rol important, estimat la 15 milioane de euro. în acest sens, specialiştii care au lucrat la redactarea strategiei de dezoltare a municipiului cluj-napoca pentru perioada 2014-2020, coordonaţi de călin hinţea, fost consilier de stat, în prezent decanul facultăţii de ştiinţe politice, administrative şi ale comunicării din cadrul universităţii "babeş-bolyai", au dedicat un întreg capitol clujului cultural: "oraş al excelenţei artistice şi al participării culturale". "clujul va fi un reper european prin viaţa sa culturală dinamică, vibrantă, care sprijină experimentarea şi iniţiativa. cultura va reprezenta un factor transversal în organizarea coumunităţii, devenit motorul transformării sociale şi regenerării urbane. prin cultură şi prin procesele pe care cultura le catalizează, oraşul poate oferi locuitorilor noi perspective de participare la viaţa publică, poate dezvolta noi mecanisme de solidaritate, poate revitaliza zonele periferice, îşi poate dezvolta infrastructura, poate să primească o mai largă deschide europeană şi poate genera colaborări şi parteneriate care să aducă beneficii economice şi sociale întregii comunităţi. cultura include arta şi toate formele sale de expresie artistică: muzică, teatru, dans, literatură, arhitectură, dar şi sistemul de valori, tradiţiile şi credinţele unui grup", se arată în documentul strategiei, aprobată de consiliul local în luna septembrie 2015, document ce numără 1.319 pagini. pe lângă proiectele propuse spre implementare de către specialişti - construcția centrului cultural "transilvania", reabilitarea, modernizarea și dotarea spațiilor culturale existente, etc. - municipalitatea propune şi 10 obiective care să fie incluse în "strategia pentru identitatea istorica a oraşului", subcapitol cultural. între acestea se numără: redactarea unui ghid turistic cultural al clujului, amenajarea unui patinoar pe lacul din parcul central, amenajarea unui muzeu al oraşului, revizalizarea cimitrului central sau retaurarea unor biserici precum sf. mihail din piaţa unirii, biserica şi mănăstirea franciscană din piaţa muzeului sau sinagoga "tempul memorial al deportaţilor" de pe strada horea. aceste propuneri au fost supuse spre dezbatere publică astfel că, persoanele interesate pot trimite sesizările, observaţiile, punctele de vedere la adresa de email:consultarepublica@primariaclujnapoca.ro sau pot înregistra adrese scrise la serviciul centrul de informare pentru cetăţeni, strada moţilor nr.7 sau la orice primărie de cartier până în data de 12 iulie 2016. o cartografiere a sectorului cultural clujean actual arată astfel: 19 instituţii publice de cultură, 60 de ong-uri, numeroase grupuri informale derulând activităţi cu componentă culturală, 6 uniuni de creaţie, 6 universităţi sau departamente cu profil artistic, 3 licee de artă, 7 centre culturale străine, 17 biblioteci ce oferă cursuri de limbi străine sau acces la cărţi în alte limbi, iar anul 2012 a numărat circa 90 de festivaluri, de mai mică sau mai mare dimensiune. conform specialiştilor care au redactat noua strategie de dezvoltare a clujului, pentru realizarea tuturor obiectivelor curpinse la capitolul cultură se recomandă o creştere a bugetului de alocări până la minim 4\% din bugetul local până în 2020 şi optimizarea sistemelor de finanţare a proiectelor culturale din bani publici. ar fi necesari 61 de milioane de euro pentru: extinderea şi modernizarea infrastructurii culturale - construcţia centrului cultural "transilvania" şi reabilitarea, modernizarea şi dotarea structurilor existente (teatrul/opera română, opera maghiară, ccs) - 30 milioane de euro, programul cluj capitală europeană - 15 milioane euro, consilidarea, reabilitarea şi introducerea în circuitul turistic al obiectivelor de patrimoniu cultural şi istoric - 15 milioane euro, city branding - 1 milion euro. - fondurile europene - 2014-2020 - europa creativă și fondurile structurale - activarea spațiilor publice (manual de utilizare a spațiului public) - cultura voluntariatului este în creștere - implicarea mai multor generații - dezvoltarea/punerea în valoare a pivnițelor din centrul orașului - existența specialiștilor it pentru proiecte de cultură digitală - siturile naturale și de patrimoniu din împrejurimile clujului (catelul bonțida, salina turda, cheile turzii etc.) - cluj arena - pentru evenimente de anvergură - mediul de afaceri clujean - redeschiderea apetitului privat pentru implicare în actul cultural - universurile paralele - găsirea de puncte de contact - pregătirea centrului pentru industrii creative (lomb) - proiectul cluj innovation city - fragilitatea sectorului în fața deciziei politice - incoerența în implementarea politicilor culturale - adâncirea izolării sectorului cultural față de cel politic și economic - pericolul ca noile megaproiecte (centrul cultural transilvania, cluj innovation city, centrul pentru industrii creative etc) să canalizeze toate resursele fără a aduce un beneficiu întregului spectru cultural - festivalizarea și megaevenimentizarea agendei culturale a orașului - infrastructura de acces terestră subdezvoltată - autostrada nefinalizată, calea ferată nemodernizată - pierderea oportunităților de a valorifica pentru comunitate spații (publice sau construite) reprezentive pentru experiența colectivă - ex. continental, parcul feroviarilor etc potrivit datelor prezentate de direcţia judeţeană de statistică, în perioada 1 ianuarie 2015 - 31 decembrie 2015, în judeţul cluj au fost cazaţi 428.239 turişti, din care 88.368 turişti străini, reprezentând 20,6\% din totalul turiştilor. cei mai mulţi turişti străini provin din: ungaria (14873), germania (9555), italia (9293), franţa (6344), polonia (5183). conform aceleeaşi surse, la nivelul municipiului cluj-napoca s-au cazat anul trecut 319.220 de turişti, în creştere cu 21,57 \% faţă de anul 2014, când s-au înregistrat 262.572 de turişti. & 845 & low & Low & Socio-Economic & NA & NA & 2016-06-30 & 2016 & 2 & ECO
Frame & low-medium & Regional & 500-1000 & 1.0255497 & 0.6327204 & -1.4131940 & 0.5015415 & -0.9523600 & 0.0 & -1.0024021 & 0.3314703 & Recipient & Domestic & Domestic & Domestic & Domestic|ECO & Neutral\\
Romania & https://www.dcnews.ro/valdis-dombrovskis-ne-preocupa-pierderea-de-competitivitate-a-romaniei\_646214.html & 564 & dcnews.ro & Private/Non-Public & Online only & National & very low = CP mentioned once & Economic development & Balanced & EU & No myth & NA & NA & NA & NA & NA & NA & NA & NA & Romania & valdis dombrovskis: ne preocupă pierderea de competitivitate a româniei & 2019-04-04 & fondurile structurale & valdis dombrovskis, vicepreședinte ce, a făcut o analiză a situției româniei prezentând evoluțiile, dar și îngrijorările pe anumite paliere. românia pierde la capitolul competitivitate şi ne preocupă, de asemenea, deficitul de cont curent al ţării care s-ar putea să afecteze eforturile de convergenţă, a declarat, joi, în cadrul unei conferinţe de specialitate, valdis dombrovskis, vicepreşedintele comisiei europene şi comisar responsabil pentru moneda euro şi dialogul social, precum şi pentru stabilitatea financiară, serviciile financiare şi uniunea pieţelor de capital. "analiza privind românia arată că este important că economia şi nivelul de trai al oamenilor au crescut. pib-ul româniei a crescut de la 39\% din media europeană la 66\%, în 2017. această tendinţă va continua numai dacă se vor face eforturi. ne preocupă pierderea de competitivitate şi deficitul de cont curent al româniei care s-ar putea să afecteze eforturile de convergenţă ale ţării. din cauza acestor dezechilibre, românia intră în procedura de dezechilibre macroeconomice, procedură pe care o lăsase în urmă acum doi ani. chiar şi după ani de creştere economică puternică au apărut dificultăţi. anul trecut s-a constatat că românia nu a adoptat măsuri de corectare. în acelaşi fel, limitările de introducere a reformelor şi-au spus cuvântul. românia a înregistrat doar un progres limitat sau chiar niciun fel de progres legat de recomandările transmise anul trecut, de la cadrul fiscal până la dialogul social, integrarea cetăţenilor de etnie romă şi consolidarea administraţiei publice", a spus oficialul european. acesta a adăugat că programul privind drepturile sociale a însemnat un număr de 40 de proiecte, în contextul în care semestrul european devine din ce în ce mai social. "semestrul european devine mai social. pilonul de drepturi sociale va continua să fie consolidat către convergenţă. scopul nostru trebuie să fie acela să ne asigure o creştere care va duce la avantaje pentru toţi cetăţenii, pentru toată societatea. românia are unul dintre nivelurile care trebuie să crească în uniunea europeană şi implementarea reformelor. programele de susţinere trebuie să fie sprijinite. în românia, programul a însemnat 40 de proiecte, promovarea bănci naţionale, servicii pentru construirea de locuinţe, copiii din familiile defavorizate să nu mai părăsească şcoala. discutăm cu toate ţările despre fondurile structurale şi de coeziune. au fost identificate proiecte de investiţii în infrastructură, cercetare-dezvoltare, politici publice şi dezvoltare rurală şi urbană", a afirmat valdis dombrovskis. citește și: proces kovesi. norica nicolai: era o strategie pusă în practică anterior dombrovskis a subliniat faptul că pentru ca semestrul european să se bucure de succes, toate ţările din uniunea europeană trebuie să fie realmente angajate în acest proces. "factorii politici din europa şi-au dat seama că piaţa unică trebuie susţinută de politici economice. uniunea economică şi monetară ne aduce şi mai aproape dar există şi riscurile unor efecte secundare. în momentul în care românia a aderat la uniunea europeană, în 2007, apăruseră deja dezechilibre în multe state membre. uniunea europeană a adoptat o serie de măsuri pentru a contracara criza şi pentru a preveni situaţii similare prin politici sustenabile fiscale şi macroeconomice. în acest context, am dezvoltat semestrul european, care impune elaborarea la nivel european permiţându-ne să identificăm semnele de avertizare din timp. trei priorităţi ale semestrului european sunt importante, şi anume investiţii, reforme structurale şi politici fiscale adoptate cu responsabilitate. şi eu, şi alţi comisari europeni care am vizitat diferite ţări, ne-am angajat în dialoguri cu guverne, parlamentele şi partenerii sociali. trebuie ca ţările să fie realmente angajate pentru ca semestrul european să se bucure de succes. este foarte importantă angajarea statelor membre la toate nivelurile. semestrul funcţionează chiar dacă nu întotdeauna este apreciat aşa cum se cuvine. de când a fost lansat semestrul european s-au înregistrat o serie de progrese pe partea fiscală. deficitul era de 2,9\%, acum este de 0,8\%. în acelaşi timp, trebuie să recunoaştem că ritmul este destul de lent şi uneori ţările revin asupra reformelor începute şi nu sunt folosite din plin momentele de avânt economic", a menţionat vicepreşedintele ce. vice-preşedintele comisiei europene, valdis dombrovskis, şi vice-prim ministrul româniei, viorel ştefan, participă, joi, la conferinţa "coordonarea politicilor economice la nivelul ue, un rol reînnoit pentru semestrul european", organizată în contextul deţinerii de către românia a preşedinţiei consiliului uniunii europenea & 701 & very low & Low & Socio-Economic & NA & NA & 2019-04-04 & 2019 & 3 & ECO
Frame & v.low & National & 500-1000 & 1.0255497 & 0.6327204 & -1.4131940 & 0.5015415 & -0.9523600 & 0.0 & -1.0024021 & 0.3314703 & Recipient & European & European & European & European|ECO & Neutral\\
Romania & http://www.cotidianul.ro/calin-georgescu-nu-recunosc-conducerea-acestei-tari-259196/ & 549 & Cotidianul & Private/Non-Public & Online only & National & very low = CP mentioned once & Lose sovereignty & Negative & EU + National & No myth & NA & NA & NA & NA & NA & NA & NA & NA & Romania & călin georgescu: "nu recunosc conducerea acestei ţări" & 2015-04-02 & politica regională & între o vizită recentă, la începutul lui martie, la washington, probabil ca membru al clubului de la roma internaţional, şi un comentariu simplu, direct, de recunoştere, însoţit de fotografia sa pe pagina a i-a a trustului de media al kremlinului, russia today, călin georgescu, tehnocrat european şi naţionalist român stabilit temporar la viena, s-a aflat la 13 martie în nord, în maramureşul descălecătorilor de ţări române din sec. (13?) 14. la biblioteca "petru dulfu" din baia mare, cu o colecţie impresionantă de carte veche românească şi străină, la distanţă de două uşi de prima lege a românilor scrisă în limba română (pravila de la govora, la 1640), de prima lor biblie în română (biblia lui şerban cantacuzino, bucureşti, 1680) şi - simbolic pentru românia semi-analfabetă de astăzi - de prima ei gramatică tipărită în româneşte (ghe. şincai, "elementa lingua daco-romanae..."), călin georgescu a reiterat calm, dar mai ferm ca niciodată, că fără trecut nu există prezent, şi a îndemnat la intoleranţă şi răzvrătire împotriva întregii clase politice post-decembrie1989, a spus, citat, "nu recunosc conducerea acestei ţări". deşi au trecut abia câteva luni între publicarea ultimei sale cărţi, "cumpăna româniei" - un proiect de refacere şi dezvoltare durabilă a ţării, unicul cu adevărat sustenabil din cei 25 de ani trecuţi - şi discursul naţionalist de acum câteva zile, în acest răstimp scurt călin georgescu s-a radicalizat enorm. ceea ce scria pe atunci în 148 de pagini a spus acum, plus altele, într-o formă revoluţionară, în 3 sau 4 pagini. a reabilitat şi enumerat la baia mare "arhaisme" ieşite din vocabularul românului modificat prin "ingineria genetică" a post-comunismului, cum ar fi cuvintele, idiomurile şi sintagmele izgonite din limba română: demnitate, libertate, ţara mea, neam, românitate, interesul general, binele comun, icoane, repere şi modele, patriot, patriotism, naţionalist, naţionalism, suflet, curaj, lumina din noi, răzvrătire, de bună seamă ultima în legalitate şi în respectul deplin pentru libertăţile cetăţeanului, democraţie şi ordinea publică. spune călin georgescu că, de la 1989 încoace, impostura a devenit "profesiunea" nr. 1 în românia şi că românii au acceptat trădători de ţară în fruntea lor. zice corect că "mioriţa", oricât de frumoasă ar fi şi oricât am iubi-o, ne învaţă să abdicăm, nu să luptăm pentru viaţă, pentru drepturile şi bunăstarea noastră. afirmă adevărul, că românii s-au condamnat singuri la sărăcie şi disperare, atunci când şi-au vândut resursele, pădurile, pământul, petromul, electrica, băncile etc. că privatizarea fără oprelişti a fost sinucidere. profund dezamăgiţi, la rândul nostru spunem că, în esenţă, călin georgescu a zugrăvit în nord o altă stemă a româniei, în centrul căreia însemnele heraldice reprezentând cele 5 regiuni istorice ale ţării - vulturul ţării româneşti, leul olteniei, bourul moldovei, acvila cu cele 7 cetăţi ale transilvaniei şi delfinii dobrogei - au fost înlocuite cu alte "blazoane": o bancă străină, o pereche de cătuşe, o farmacie, o maşină de păcănele şi o şaormerie cu tejgheaua la stradă. în continuare, cu vasta sa expertiză internaţională, călin georgescu apreciază că politica externă a fost cea mai slabă verigă din activitatea statului român de după 1989 şi a avut ponderea negativă cea mai mare în declinul abrupt al româniei; crede că nimic nu s-a negociat şi întâmplat în conformitate cu realitatea geo-politică regională, europeană şi globalistă, şi nici cu interesele economice vitale ale ţării, şi că, astăzi, redresarea generală trebuie neapărat să înceapă cu reformularea relaţiilor internaţionale ale bucureştiului. astfel, în conferinţa sa a afirmat că eliberarea din decembrie1989 afost iluzorie şi că românia a fost, ulterior, încarcerată într-un lagăr neo-liberal, că se află în prezent sub tripla dictatură a corporaţiilor fără ţară, a ue şi a fmi; că răul cel mare şi l-au făcut românii cu mâna lor, întrucât neo-liberalismul nu a venit şi s-a instalat aici numai prin forţe proprii, ci şi ajutat de lachei de serviciu locali, urmărind avantaje politice şi economice. a spus că, după 1989, românilor li s-a inoculat minciuna că statul este duşmanul tuturor şi că trebuie adus la dimensiuni neglijabile; că aşa-zisa reformare, de 25 de ani, a statului înseamnă distrugerea şi subordonarea lui unor interese străine. iar toată această minciună nefastă cu statul ca o specie în curs de extincţie, ticluită de slugi "reformatoare" locale şi de komisari externi de tot felul, adăugăm noi, nu este altceva decât doctrina falimentară a reaganismului, respinsă de ani buni de toată europa occidentală, dar aplicată în revoluta românie, hinterland şi "second hand" estic al ue. în rostirea sa liberă de la baia mare, şi în dialoguri trecute cu acest autor, călin georgescu a criticat stări de lucruri negative, dar, în primul rând, a vorbit de reconstrucţie. el crede corect că în politicile externe, înainte de orice, se impune resetarea cuprinzătoare a relaţiilor româniei cu statele unite, partenerul strategic şi aliatul ei principal. consideră că dialogul bucureştiului trebuie purtat la washington cu decidenţi raţionali şi rezonabili, interesaţi, în primul rând, de o imagine pozitivă a propriei lor ţări, america, în lume, promotori ai respectului reciproc, negocierii, diplomaţiei profesioniste, colaborării internaţionale şi păcii, şi nu cu "uliii" neoconservatori ai războiului, stărilor conflictuale şi destabilizării păcii în europa de est, zona unde se află românia, şi în alte părţi. este convins că şi america trebuie să ia în calcul faptul că stabilitatea şi validitatea demersului ei de politică externă în estul european şi la marea neagră vor fi puse la îndoială, că se vor pierde, dacă în românia nu se vor impune grabnic nu numai statul de drept, dar şi o nouă autoritate morală. în sfârşit, călin georgescu crede că economia de proximitate de care a vorbit în proiectul lui de ţară presupune o reformulare serioasă a relaţiilor externe regionale ale româniei cu toţi vecinii săi, toate ţările ex-socialiste estice, sau componente ale fostei urss şi, în primul rând, cu super-puterea regională, rusia lui putin. despre politica externă românească, aşa cum o vede tehnocratul european şi naţionalistul român călin georgescu, am mai aminti doar că a fost şi este decisă, prin constituţia acestei republici prezidenţiale, de către preşedintele ţării. de 40 de ani, din 1975, de când există instituţia respectivă, românia a avut 5 preşedinţi. unul a sfârşit în faţa plutonului de execuţie, după un proces dezgustător. doi dintre ei sunt, se zice, puşcăriabili. al patrulea nu a putut conduce ţara cum a voit domnia sa, pentru că, îl cităm, "a fost învins de fosta securitate", adică a fost învins în anul 2000 de o instituţie dispărută legal de 11 ani, adică iată o explicaţie cel puţin idioată. al cincilea, instalat recent la cotroceni, în materie de "externe" a călcat din prima zi, de la prima vizită afară, cu stângul. este închis ermetic în sine, opinia publică nu află nimic de la el despre conţinutul discuţiilor cu lideri străini, sau ştie câte ceva de la interlocutori şi presa lor (merkel, plevneliev). stăm foarte, foarte prost, la capitolul "instituţia prezidenţială", am rămas repetenţi... dacă vrem să ridicăm românia, înainte chiar să terminăm autostrada transilvaniei, trebuie să instaurăm în politica internă moralitatea, iar în politica externă transparenţa. & 1171 & very low & Low & Power & NA & NA & 2015-04-02 & 2015 & 1 & POL
Frame & v.low & National & +1000 & 1.0255497 & 0.6327204 & -1.4131940 & 0.5015415 & -0.9523600 & 0.0 & -1.0024021 & 0.3314703 & Recipient & Domestic & European & Mixed & Domestic|POL & Negative\\
Romania & http://www.cugetliber.ro/stiri-sanatate-noi-locuri-de-munca-pentru-personalul-medical-din-constanta-270850 & 506 & CugetLiber.ro & Private/Non-Public & Online and Offline & Regional/Local & very low = CP mentioned once & Jobs & Positive & Subnational & No myth & NA & NA & NA & NA & NA & NA & NA & NA & Romania & noi locuri de muncă pentru personalul medical din constanţa & 2015-10-15 & fondul social european & ştire online publicată vineri, 16 octombrie 2015. autor: andreea năstase asociaţia regională pentru dezvoltare socială, în parteneriat cu spitalul clinic de boli infecţioase constanţa, institutul naţional de boli infecţioase "prof. dr. matei balş", institutul pentru dezvoltarea resurselor umane constanţa şi expertmob srl bucureşti, organizează conferinţa regională privind promovarea principiului egalităţii de şanse în societatea românească, în cadrul proiectului "şanse noi pentru personalul medical. un acces egal la ocupaţii de actualitate de pe piaţa muncii din domeniul sănătăţii", proiect cofinanţat din fondul social european prin programul operaţional sectorial dezvoltarea resurselor umane 2007-2013. proiectul este implementat pe o perioadă de 20 luni şi îşi propune să contribuie la dezvoltarea şi promovarea principiului egalităţii de şanse şi de gen în societatea românească pentru 802 persoane, prin dezvoltare profesională, formare, consultanţă şi campanie de informare şi conştientizare. în cadrul proiectului, sunt furnizate servicii integrate de informare şi conştientizare, formare profesională şi consiliere antreprenorială. programele de formare profesională vizează calificarea a 564 dintre beneficiarele proiectului (asistente şi infirmiere) în domenii precum baby sitter (bonă) şi îngrijitoare bătrâni la domiciliu. după finalizarea programelor de formare, 55 de antreprenoare, selectate dintre absolventele care au obţinut certificate de calificare, vor fi susţinute şi asistate să înfiinţeze mici start-up-uri, contribuind astfel la extinderea portofoliului personal de competenţe, ceea ce va determina un acces egal şi facil la noile ocupaţii de pe piaţa muncii, din domeniul sănătăţii. & 230 & very low & Low & Socio-Economic & NA & NA & 2015-10-15 & 2015 & 1 & ECO
Frame & v.low & Regional & <500 & 1.0255497 & 0.6327204 & -1.4131940 & 0.5015415 & -0.9523600 & 0.0 & -1.0024021 & 0.3314703 & Recipient & Domestic & Domestic & Domestic & Domestic|ECO & Positive\\
\addlinespace
Romania & http://www.euractiv.ro/fonduri-ue/Studiu-Peste-230-milioane-de-euro-obtinuti-in-ultimii-cinci-ani-din-finantari-europene-returnati-la-UE-562 & 545 & EurActiv | Știri, politici europene \& Actori UE online & Private/Non-Public & Online only & National & medium = CP is important part of story & Fraud/Corruption & Positive & National & No myth & NA & NA & NA & NA & NA & NA & NA & NA & Romania & studiu: peste 230 milioane de euro obţinuţi în ultimii cinci ani din finanţări europene, returnaţi la ue & 2015-05-07 & fondul de coeziune & peste 230 de milioane de euro din banii cheltuiţi în ultimii cinci ani în cadrul unor proiecte europene au fost returnaţi la ue ca urmare a unor corecţii financiare, relevă un studiu al institutului pentru politici publice, citat de mediafax. cea mai crescută incidenţă de corecţii financiare raportat la numărul de contracte semnate a fost la programul operaţional sectorial (pos) mediu - 76\%. potrivit institutului pentru politici publice (ipp), datele au fost colectate în baza unor solicitări de informaţii de interes public transmise către 41 de consilii juudeţene, 45 de primării ale municipiilor reşedinţă de judeţ, inclusiv primaria capitalei şi toate primăriile de sector, şapte autorităţi de management, autoritatea pentru audit, departamentul pentru luptă antifraudă, direcţia naţională anticorupţie şi 15 parchete de pe lângă curţile de apel. în urma interpretării datelor obţinute, s-a stabilit că, în perioada 2011 - 2014, suma totală aproximativă a corecţiilor financiare aplicate în proiectele cu finanţare din fondul european de dezvoltare regională (feder), fondul social european (fse) şi fondul de coeziune este de 1.039.310.000 de lei, adică aproximativ 230 de milioane de euro. numărul total de contracte de finanţare a fost de 17.034, iar numărul total de note de constatare a corecţiilor emise în proiectele cu finanţare din fonduri structurale ajunge la 2.977. în acelaşi timp, potrivit studiului ipp, au fost soluţionate pe cale administrativă 495 de note de constatare, înregistrate 532 de suspiciuni de fraudă şi au existat 188 de sesizări către organele de anchetă. cea mai crescută incidenţă de corecţii financiare raportat la numărul de contracte semnate a fost la programul operaţional sectorial (pos) mediu - 76\%, urmat, în ordine, de programul operaţional dezvoltarea capacităţii administrative (podca) - 49\%, programul operaţional regional (por) - 29\%, programul operaţional sectorial dezvoltarea resurselor umane (posdru) - 16\%, programul operaţional sectorial creşterea competitivităţii economice (poscce) - 5\% şi programul operaţional asistenţă tehnică (poat) - 2 \%, ciyeaza mediafax. cea mai crescută incidenţă de suspiciune la fraudă s-a înregistrat în cazul proiectelor pos mediu - 9\% şi podca - 9\%, urmate de posdru - 5\%, por - 2\%, poscce - 2\% şi poat - 1\%. cela mai multe sesizări către organele de cercetare penală au fost făcute în legătură cu proiectele posdru - 65\%, por - 37\%, pos mediu - 34\% şi podca - 33\%. & 365 & medium & Medium & Governance & NA & NA & 2015-05-07 & 2015 & 1 & POL
Frame & low-medium & National & <500 & 1.0255497 & 0.6327204 & -1.4131940 & 0.5015415 & -0.9523600 & 0.0 & -1.0024021 & 0.3314703 & Recipient & Domestic & Domestic & Domestic & Domestic|POL & Positive\\
Romania & http://www.national.ro/news/sri-si-cabinetul-ciolos-s-au-ales-cu-o-reclamatie-la-oficiul-european-antifrauda-569555.html/ & 547 & national.ro & Private/Non-Public & Online and Offline & Regional/Local & very low = CP mentioned once & Fraud/Corruption & Negative & National & No myth & NA & NA & NA & NA & NA & NA & NA & NA & Romania & sri si cabinetul ciolos s-au ales cu o reclamatie la oficiul european antifrauda & 2017-01-31 & fondul european de dezvoltare regională & un grup de organizatii non-guvernamentale s-au adresat in aceeasi perioada si oficiului european antifrauda (olaf), precum si comisiei parlamentare care verifica activitatea sri-ului. acestea vor a fi lamurit modul in care serviciul roman de informatii a primit fonduri europene pentru programul sii analytics, descris ca sistemul informatic de integrare si valorificare operationala si analitica a volumelor mari de date. mai exact, societatea academica din romania (sar), active watch, asociatia militia spirituala, asociatia pentru apararea drepturilor omului in romania - comitetul helsinki (apador-ch), asociatia pentru tehnologie si internet (apti) si centrul pentru resurse juridice (crj). organizatiile in cauza au adresat comisiei de control a sri din parlament o scrisoare prin care cer investigarea proiectului sii analytics al sri, pe motiv ca "suspiciunea ca sri a intervenit la inalti oficiali ai guvernului ciolos pentru aprobarea acestui proiect, ba chiar si la primul ministru, si ca ultima transa a acestui proiect survine pe fondul unor incalcari constitutionale mai vechi, acoperite prin decizii ale csat a caror constitutionalitate nu a fost verificata". in scrisoarea cu pricina se mentioneaza ca exista suspiciunea ca sri ar fi fost favorizat inaintea si in cadrul procedurii de licitatie a proiectului din fondul european de dezvoltare regionala (fedr), licitatie la care a existat un singur ofertant. redam, mai jos, integral, sesizarea trimisa comisiei sri din parlament: "in atentia comisiei comune permanente a camerei deputatilor si senatului pentru exercitarea controlului parlamentar asupra activitatii sri stimate doamne, stimati domni. dorim sa va semnalam urmatoarele fapte in legatura cu proiectul sii analytics al sri, cu precizarea ca ele au facut obiectul unei sesizari a comisiei europene si a mai multor articole de presa. consideram ca ar trebui sa investigati cu prioritate faptele semnalate aici, existind suspiciunea ca sri a intervenit la inalti oficiali ai guvernului ciolos pentru aprobarea acestui proiect, ba chiar si la primul ministru, si ca ultima transa a acestui proiect survine pe fondul unor incalcari constitutionale mai vechi, acoperite prin decizii ale csat a caror constitutionalitate nu a fost verificata. consideram ca atit posibila interventie, cit si proiectul in sine, creaza un precedent periculos pentru democratia noastra, contravine dreptului european si bunei gestionari a fondurilor europene, si ca aceasta tematica se circumscrie exact misiunii dvs. asociatia pentru apararea drepturilor omului in romania - comitetul helsinki (apador-ch) & 380 & very low & Low & Governance & NA & NA & 2017-01-31 & 2017 & 2 & POL
Frame & v.low & Regional & <500 & 1.0255497 & 0.6327204 & -1.4131940 & 0.5015415 & -0.9523600 & 0.0 & -1.0024021 & 0.3314703 & Recipient & Domestic & Domestic & Domestic & Domestic|POL & Negative\\
Romania & https://www.mediafax.ro/politic/corina-cretu-despre-rezolutie-din-parlamentul-european-si-raportul-mcv-cred-ca-romania-trebuie-sa-invete-din-aceste-lectii-ce-spune-despre-posibilitatea-de-a-fi-numita-premier-17651670?utm\_source=feedburner\&utm\_medium=feed\&utm\_campaign=Feed\%3A+MediafaxSocial+\%28Mediafax+-+Social\%29 & 495 & Mediafax.ro & Private/Non-Public & Online only & National & medium = CP is important part of story & Bureaucracy and/or delays & Negative & EU + National & NA & NA & NA & NA & NA & NA & NA & NA & NA & Romania & corina creţu despre rezoluţie din parlamentul european şi raportul mcv: cred că românia trebuie să înveţe din aceste lecţii/ ce spune despre posibilitatea de a fi numită premier & 2018-11-14 & politica regională & "din păcate, nu e o săptămână uşoară pentru românia. în primul rând pe a adoptat această rezoluţie privind statul de drept în românia şi în aceeaşi zi a venit şi raportul mcv din partea ce. eu cred că românia trebuie să înveţe din aceste lecţii. nu cred că cineva ăşi doreşte răul româniei. reuşita fiecărei ţări e reuşita europei în general. pe de altă parte, s-a politzat foarte mult, s-a creat impresia că singura preocupare e pentru justiţie. eu nu sunt comisar pe justiţie. dl timmermans a făcut o conferinţă de presă a spus că evoluţiile pozitive din 2017 sunt puse sub semnul întrebării şi că au fost adăugat încă 8 recomandări, faţă de 2017. aş vrea ca partidele politce din românia să privească cu înţelepciune şi să pună în cheie complotistă faptul că cineva ne vrea răul. eu cred că trebuie să înţelegem că viitorul copiilor noştri depinde ceea ce facem în viaţa de zi cu zi", a afirmat corina creţu, într-un interviu acordat postului românia tv. în românia, adoptarea legilor justiţiei şi presiunile asupra independenţei sistemului judiciar, în special asupra dna, au generat dubii privind ireversibilitatea progreselor înregistrate de ţara noastră, a anunţat comisia europeană, recomandând în mcv suspendarea procedurilor în cazul procurorilor de rang înalt. parlamentul european a adoptat, marţi, proiectul de rezoluţie prin care exprimă preocupare privind modificarea legislaţiei penale şi judiciare în românia. textul rezoluţiei semnalează îngrijorarea pe privind modificarea legislaţiei penale şi judiciare, avertizând că există riscul subminării independenţei justiţiei şi a acţiunilor anticorupţie. creţu, întrebată dacă i s-a propus să fie premier: nu, şi îmi doresc să lucrez ca şi comisar corina creţu a afirmat, într-un interviu acordat miercuri românia tv, că nu i s-a propus să fie premier şi că această funcţie nu intră în calculele ei de viitor, aceasta dorindu-şi să continue să fie comisar european. "nu, şi îmi doresc să lucrez ca şi comisar. nu, nu intră în niciun caz în calculele mele de viitor. eu le mulţumesc tuturor care mă consideră capabilă de acest lucru. în primul rând, vreau să îi mulţumesc mult dl victor ponta. nu intră în calculele mele să fiu prim-ministru. vreau să îi urez succes. dar pe de altă parte am văzut că şi domnul băsescu a vehiculat numele meu. eu am o altă treabă importantă pentru românia. ştiţi bine că nu a agreat numirea mea. eu am această responsabilitate faţă de românia şi faţă de români, nu faţă de guvern, nui faţă de partide", a declarat comisarul european pentru politică regională, corina creţu, într-un interviu acordat postului românia tv. aceasta a transmis un mesaj pentru guvern, în contextul tensiunilor care au apărut în ultima perioadă între comisarul european şi executiv, precizând că vrea să reia discuţiile "cu cifrele pe masă". "vreau să dau un mesaj pentru guvern, haideţi să reluăm discuţiile cu cifrele pe masă. haideţi să vedem de ce investiţiile prin fonduri nerambursabile nu ar fi mai bune decât cele prin parteneriat-privat. o ţară trebuie să-şi epuizeze aceste resurse care vin gratuit şi apoi să meargă cu altele sau să meargă în paralel cu parteneriatele public-private", a completat creţu. întrebată de ce crede că a fost atacată de unii colegi de partid, corina creţu a răspuns: "nu ştiu, poate sunt nişte calcule politice. nu îmi dau seama. am fost foarte afectată. am fost numită trădătoare de ţară şi aşa mai departe. pe mine mă interesează ca aceste proiecte să se realizeze. vreau să ne aşezăm la masă şi să discutăm exact". fostul prim-ministru victor ponta a înaintat în mai multe rânduri numele corinei creţu pentru funcţia de premier, de la 1 ianuarie 2019, când românia va prelua preşedinţia consiliului ue. conținutul website-ului www.mediafax.ro este destinat exclusiv informării și uzului dumneavoastră personal. este interzisă republicarea conținutului acestui site în lipsa unui acord din partea mediafax. pentru a obține acest acord, vă rugăm să ne contactați la adresa vanzari@mediafax.ro. & 662 & medium & Medium & Governance & NA & NA & 2018-11-14 & 2018 & 3 & POL
Frame & low-medium & National & 500-1000 & 1.0255497 & 0.6327204 & -1.4131940 & 0.5015415 & -0.9523600 & 0.0 & -1.0024021 & 0.3314703 & Recipient & Domestic & European & Mixed & Domestic|POL & Negative\\
Romania & https://www.desteptarea.ro/peste-500-de-copii-inclusi-in-proiectul-educatia-o-sansa-pentru-fiecare/ & 561 & Deșteptarea- Ziarul Bacăului & Private/Non-Public & Online and Offline & Regional/Local & low = CP mentioned more times but NOT important part of story (mainly about others issues) & Social awareness/inclusion & Positive & Subnational & No myth & NA & NA & NA & NA & NA & NA & NA & NA & Romania & peste 500 de copii incluși în proiectul "educația, o șansă pentru fiecare!" | deșteptarea- ziarul bacăului & 2018-11-06 & fondul social european & sala mare a consiliului județean (cj) bacău a găzduit, marți, lansarea unui proiect derulat de școala specială "maria montessori" bacău, în parteneriat cu alte patru instituțiii: centrul școlar de educație incluzivă nr.1 bacău, școala "octavian voicu", direcția generală de asistență socială și protecția copilului și centrul român pentru educație și dezvoltare umană. proiectul se numește "educația, o șansă pentru fiecare!" și este finanțat din fondul social european, prin programul operațional capital uman. au participat la lansare reprezentanți ai instituțiilor publice locale și județente, ai organizațiilor neguvernamentale, părinți și copii beneficiari ai proiectului. activitățile vor fi derulate până în mai 2021 și urmăresc reducerea și prevenirea abandonului școlar timpuriu și promovarea accesului la o educație de calitate pentru 516 de copii dezavantajați socio-economic, dintre care 391 de copii cu cerințe educaționale speciale. concret, e vorba despre îmbunătățirea programelor și dotărilor educaționale, sprijinirea elevilor preșcolari și școlari să depășească dificultățile sociale, cognitive și emoționale cu care se confruntă, precum și stimularea politicilor educaționale și moblizarea comunității în susținerea procesului de desegregare, asigurarea unei educații incluzive și accesului la educație a tuturor copiilor, în special a celor cu cerințe educaționale speciale. valoarea totală a proiectului este de 5.034.601 de lei, din care 4.279.411 de lei reprezintă valoarea finanțării ue, iar 666.515 de lei, suma asigurată din bugetul național. & 222 & low & Low & Socio-Economic & NA & NA & 2018-11-06 & 2018 & 3 & ECO
Frame & low-medium & Regional & <500 & 1.0255497 & 0.6327204 & -1.4131940 & 0.5015415 & -0.9523600 & 0.0 & -1.0024021 & 0.3314703 & Recipient & Domestic & Domestic & Domestic & Domestic|ECO & Positive\\
Romania & http://ziuadecj.realitatea.net/educatie/investitii-de-23-milioane-lei-la-spitalul-pentru-copii-din-cluj--161852.html & 585 & ziuadecj.realitatea.net & Private/Non-Public & Online only & Regional/Local & medium = CP is important part of story & Environment/green/low-carbon & Positive & Subnational & No myth & Infrastructure & Positive & Subnational & No myth & NA & NA & NA & NA & Romania & investitii de 2,3 milioane lei la spitalul pentru copii din cluj & 2017-07-27 & fondul european de dezvoltare regională & plenul forului administrativ județean a adoptat proiectul de hotărâre privind aprobarea proiectului "creșterea eficienței energetice la clădirile secției pediatrie ii, corpurile c1 și c2 din cadrul spitalului clinic de urgență pentru copii cluj-napoca", proiect ce urmează a fi depus spre finanțare în cadrul programului operațional regional 2014-2020. necesitatea derulării acestei investiții în cadrul secției pediatrie ii a spitalului de copii este determinată de vechimea și starea de degradare accentuată a celor două corpuri de clădire situate pe strada crișan din municipiul cluj-napoca, acestea având, în prezent, consumuri energetice respectiv încălzire, apă caldă și iluminat, foarte mari. "prin reabilitarea termică a celor două imobile ale spitalului de urgență pentru copii ne dorim, pe de o parte, să contribuim la creșterea confortului pacienților și al personalului medical, și, pe de altă parte, să scădem presiunea bugetară rezultată ca urmare a cheltuielilor mari avansate pentru plata utilităților", a declarat alin tișe, președintele consiliului județean cluj. în mod concret, în vederea îndeplinirii obiectivului acestui proiect, respectiv cel de creștere a eficienței energetice, vor fi realizate o serie de lucrări care vizează în special izolarea termică a pereților exteriori ai celor două clădiri, izolarea conductelor de încălzire și a celor de apă caldă, înlocuirea corpurilor de iluminat interior existente cu lămpi de tip led care conferă o eficiență energetică ridicată și un consum redus de energie, instalarea de panouri fotovoltaice etc. toate aceste noi investiții vor contribui la reducerea consumurilor de energie din surse convenționale și diminuarea emisiilor de gaze cu efect de seră, astfel încât consumul anual specific de energie primară - încălzire, apă și iluminat este preconizat a scădea cu minim 30\%, până la valori cuprinse între 170 și 200 kwh/mp2/an. în ceea ce privește sursele de finanțare ale acestei investiții în valoare totală de 2.339.354,64 lei, acestea se constituie atât din asistența financiară nerambursabilă solicitată din fondul european de dezvoltare regională, în sumă de 1.892.701,69 lei, cât și din contribuția proprie a consiliului județean cluj, în valoare de 446.652,95 lei. menționăm în acest context faptul că spitalul clinic de urgență pentru copii este un spital universitar de prestigiu al medicinei românești care își desfășoară activitatea în opt locații diferite din municipiul cluj-napoca, fiind format din 15 secții care totalizează un număr de 506 de paturi. aici se desfășoară atât activitate medicală specifică de îngrijire a copiilor bolnavi din județul cluj și din alte peste 30 de județe ale țării cât și o intensă activitate didactică și științifică. în ceea ce privește clinica pediatrie ii a spitalului, în care se află cele două corpuri de clădire edificate acum mai bine de un secol, această unitate oferă servicii medicale nou-născuților, sugarilor, copiilor și adolescenților până la vârsta de 18 ani, într-o multitudine de domenii, care variază de la pediatrie generală la oncologie pediatrică. & 474 & medium & Medium & Socio-Economic & Socio-Economic & NA & 2017-07-27 & 2017 & 2 & ECO
Frame & low-medium & Regional & <500 & 1.0255497 & 0.6327204 & -1.4131940 & 0.5015415 & -0.9523600 & 0.0 & -1.0024021 & 0.3314703 & Recipient & Domestic & Domestic & Domestic & Domestic|ECO & Positive\\
\addlinespace
Romania & http://ziuadecj.realitatea.net/economie/zeci-de-milioane-de-euro-pentru-integrarea-in-munca-a-tinerilor-din-ardeal--166125.html & 534 & ziuadecj.realitatea.net & Private/Non-Public & Online only & Regional/Local & low = CP mentioned more times but NOT important part of story (mainly about others issues) & Social justice & Positive & EU + National & No myth & NA & NA & NA & NA & NA & NA & NA & NA & Romania & zeci de milioane de euro pentru integrarea in munca a tinerilor din ardeal & 2017-11-13 & fondul social european & autoritatea de management pentru programul operațional capital uman (am pocu) din ministerul dezvoltării regionale a lansat cinci ghiduri de finanțare în valoare totală de 551 de milioane de euro pentru proiecte care vizează reducerea șomajului și creșterea ocupării, în special în rândul tinerilor. "este cel mai mare pachet de măsuri finanțat din fonduri europene destinat șomerilor, inclusiv tinerilor neets, și îmbunătățirii situației de pe piața muncii din românia. sprijinim în același timp și angajatorii, și pe cei care doresc să-și găsească un loc de muncă, oferind stimulente care sa-i pună în valoare pe tineri, o resursă valoroasă a româniei", a afirmat ministrul marius nica. cele cinci apeluri de proiecte vizează inserția în piața muncii pe de o parte a tinerilor neets (persoane între 15 și 24 de ani care nu au loc de muncă, nu urmează niciun program educațional sau de formare), pe de altă parte a celorlalte categorii de șomeri și persoane inactive. cererile de proiecte pentru sprijinirea tinerilor neets au o valoare de 254 de milioane de euro. din această sumă, 94 de milioane sunt destinate sprijinirii inserției pe piața muncii a tinerilor neets șomeri, cu accent pe cei din mediul rural și minoritatea romă ("viitor pentru tineri neets ii") din regiunile nord-vest, nord-est, vest, sud-vest oltenia, bucureşti-ilfov, 57 de milioane pentru regiunile centru, sud-est, sud muntenia. 103 milioane sunt pentru scheme naționale - programe de ucenicie și stagii i pentru tinerii neets din regiunile centru, sud-est şi sud muntenia. apelurile care vizează celelalte categorii de șomeri și persoane inactive au valoarea de 297 de milioane de euro, din care 206 milioane scheme naționale - programe de ucenicie și stagii pentru șomeri și persoane inactive, de etnie romă, din mediul rural şi 91 de milioane pentru subvenţionarea locurilor de muncă, acordarea de prime de mobilitate pentru șomeri și persoane inactive, de etnie romă, din mediul rural (prime de încadrare, instalare, relocare, activare). pot depune cereri de finanțare în cadrul apelurilor "viitor pentru tineri" neets i și ii, începând cu 15 noiembrie până în 17 ianuarie, furnizori autorizați de formare profesională a adulților, furnizori acreditați de servicii specializate pentru stimularea ocupării, centre autorizate de evaluare și certificare a competențelor profesionale obținute pe alte căi decât cele formale, organizaţii sindicale şi nonguvernamentale, întreprinderi sociale de inserție, organizații de tineret, angajatori, camere de comerț și industrie. pentru celelalte apeluri de proiecte organizația care poate depune cereri de finanțare este agenția naţională pentru ocuparea forței de muncă, din 10 noiembrie în 28 decembrie 2017 pentru scheme naționale - programe de ucenicie și stagii i pentru tinerii neets din regiunile centru, sud-est, sud muntenia, respectiv până în 30 iulie 2018 în cazul apelurilor scheme naționale - programe de ucenicie și stagii pentru șomeri și persoane inactive, persoane de etnie romă, persoane din mediul rural, subvenţionarea locurilor de muncă și acordarea de prime de mobilitate pentru șomeri și persoane inactive, persoane de etnie romă, persoane din mediul rural. pocu 2014-2020 are o alocare de 4,3 miliarde de euro. programul stabilește prioritățile de investiții, obiectivele specifice și acțiunile asumate de către românia în domeniul resurselor umane, continuând investițiile realizate prin fondul social european în perioada 2007‐2013 și contribuind la reducerea disparităților de dezvoltare economică și socială dintre românia și statele membre ale uniunii europene. & 547 & low & Low & Socio-Economic & NA & NA & 2017-11-13 & 2017 & 2 & ECO
Frame & low-medium & Regional & 500-1000 & 1.0255497 & 0.6327204 & -1.4131940 & 0.5015415 & -0.9523600 & 0.0 & -1.0024021 & 0.3314703 & Recipient & Domestic & European & Mixed & Domestic|ECO & Positive\\
Romania & http://ziarulfaclia.ro/birocratia-fiscalitatea-excesiva-si-coruptia-pricipalele-obstacole-pentru-imm-uri/ & 590 & ziarulfaclia.ro & Private/Non-Public & Online and Offline & Regional/Local & medium = CP is important part of story & Bureaucracy and/or delays & Negative & National + Subnational & 7.Fraud & Fraud/Corruption & Negative & National + Subnational & No myth & NA & NA & NA & NA & Romania & birocraţia, fiscalitatea excesivă şi corupţia - pricipalele obstacole pentru imm-uri & 2016-07-15 & fondurile structurale & antreprenorii imm-urilor româneşti sunt pesimişti privind evoluţia acestui sector în anul în curs, doar 13,78\% dintre companii apreciind că evoluţia mediului de afaceri românesc va fi favorabilă. rezultatele cartei albe a imm-urilor prezintă un mediu antreprenorial caracterizat de pesimism cu privire la dezvoltarea mediului economic, care acuză dificultăţi în depăşirea problemelor de ordin birocratic şi lipsit de interes în accesarea de programe cu finanţare europeană, a declarat florin jianu, preşedintele consiliului naţional al întreprinderilor private mici şi mijlocii din românia (cnipmmr). astfel, 13,78\% dintre companii apreciază că evoluţia mediului de afaceri românesc va fi favorabilă imm-urilor, 61,41\% dintre imm-uri consideră că birocraţia este principala problemă în dezvoltarea afacerii şi 81,66\% dintre întreprinzători intenţionează să nu acceseze fondurile structurale în perioada următoare. birocraţia (61,41\%), fiscalitatea excesivă (54,74\%) urmată de corupţie (45,16\%) şi controalele excesive (44,98\%) sunt dificultăţile cu care se confruntă frecvent imm-urile şi doar 6,72\% dintre persoanele investigate consideră că evoluţia economică va fi pozitivă. conform concluziilor documentului, 33,58\% dintre întreprinderi găsesc evoluţia mediului economic autohton stânjenitoare dezvoltării, iar 48,27\% din imm-uri o apreciază neutră. concluziile lucrării cnipmmr arată că doar 13,78\% dintre companii estimează că evoluţia mediului de afaceri românesc pe întreg anul 2016 va fi favorabilă imm-urilor. dacă accesul mai bun pe pieţe (45,99\% dintre firme) constituie principalul efect pozitiv ale aderării româniei la ue asupra imm-urilor, 81,66\% dintre imm-uri nu intenţionează să acceseze fondurile structurale într-o perioadă viitoare, în condiţiile în care doar 0,18\% dintre întreprinderi au primit aprobarea pentru proiectul depus. aproximativ 72\% dintre imm-urile investigate îşi finanţează activităţile economice din surse proprii. "pregătim un program de guvernare a mediului de afaceri. nu mai aşteptăm partidele politice să iasă cu programe de guvernare şi să ne spună ce anume este necesar pentru mediul de afaceri ci ieşim noi cu o serie de măsuri. sperăm că şi alte organizaţii patronale vor da curs invitaţiei noastre de a deveni parteneri într-un astfel de program de guvernare", a afirmat jianu. cea de-a xiv-a ediţie a "cartei albe a imm-urilor din românia" are la bază interviuri cu 1.096 întreprinzători. & 370 & medium & Medium & Governance & Governance & NA & 2016-07-15 & 2016 & 2 & POL
Frame & low-medium & Regional & <500 & 1.0255497 & 0.6327204 & -1.4131940 & 0.5015415 & -0.9523600 & 0.0 & -1.0024021 & 0.3314703 & Recipient & Domestic & Domestic & Domestic & Domestic|POL & Negative\\
Romania & http://www.romanialibera.ro/economie/fonduri-europene/finalizarea-proiectului-\%E2\%80\%9Eparteneriat-pentru-o-cariera-de-succes-in-specialitatile-medicale-implicate-in-patologia-renala---perfmed--400305 & 513 & RomaniaLibera.ro & Private/Non-Public & Online and Offline & National & very low = CP mentioned once & Jobs & Positive & National & No myth & NA & NA & NA & NA & NA & NA & NA & NA & Romania & finalizarea proiectului "parteneriat pentru o carieră de succes în specialităţile medicale implicate în patologia renală" (perfmed) & 2015-12-08 & fondul social european & programul operaţional sectorial dezvoltarea resurselor umane 2007 - 2013 axa prioritară nr. 2 "corelarea învăţării pe tot parcursul vieţii cu piaţa muncii" domeniul major de intervenţie 2.1 "tranziţia de la şcoală la viaţă activă" titlul proiectului: "parteneriat pentru o carieră de succes în specialităţile medicale implicate în patologia renală" beneficiar: universitatea de medicina si farmacie "carol davila" bucuresti contract nr. posdru/161/2.1/g/135802 universitatea de medicină și farmacie "carol davila" din bucurești invită reprezentanții mass-media, studenții aparținând grupului tintă și reprezentanții instituțiilor și companiilor din domeniul medical să participe la conferința de presă organizată cu ocazia finalizării proiectului perfmed - "parteneriat pentru o carieră de succes în specialitățile medicale implicate în patologia renală", proiect cofinanțat din fondul social european prin programul operațional sectorial dezvoltarea resurselor umane 2007-2013, în baza contractului de finanțare posdru/161/2.1/g/135802. conferința de presă a proiectului perfmed va avea loc joi, 26 noiembrie 2015, ora 18:00, în sala de consiliu a decanatului facultății de medicină, din cadrul universității de medicină și farmacie "carol davila" din bucurești, bulevardul eroii sanitari nr. 8. implementat începând din data de 25 aprilie 2014 de universitatea de medicină și farmacie "carol davila" din bucurești în parteneriat cu institutul clinic fundeni, proiectul perfmed se apropie de final ! având o durată de implementare de 20 de luni (25 aprilie 2014-15 decembrie 2015) și o valoare totală a contractului de finanțare de 2.126.515,00 lei (din care 1.830.988,34 lei valoarea eligibilă nerambursabilă din fondul social european, 252.995,66 lei valoarea eligibilă nerambursabilă din bugetul național și 42.531,00 lei cofinantarea eligibilă a beneficiarului), proiectul perfmed - "parteneriat pentru o carieră de succes în specialitătile medicale implicate în patologia renală" a fost implementat în mediul urban, în regiunea bucurești-ilfov, s-a adresat studenților facultății de medicină din cadrul universității de medicină și farmacie "carol davila" bucurești, și a avut ca obiectiv susținerea tranziției studenților din grupul tintă către viața activă în specialitățile implicate în patologia renală, creșterea competitivității acestora pe piață muncii și asigurarea pentru acești studenți a unei cariere de succes pe termen lung, prin servicii de consiliere și orientare profesională în patologia renală și stagii de practică medicală cu un înalt nivel științific organizate de către institutul clinic fundeni. din grupul tintă al proiectului perfmed au făcut parte peste 360 de studenți ai facultății de medicină ce au beneficiat de activitățile de consiliere și orientare profesională pentru o carieră de succes în domeniul patologiei renale. la stagiile de practică organizate de către institutul clinic fundeni au participat 360 de studenți din grupul țintă. toți cei 360 de studenți care au participat la stagiile de practică organizate în cadrul proiectului perfmed au primit subvenții financiare, iar pentru 24 de studenți care au obținut cele mai bune rezultate în cadrul stagiilor de practică medicală s-au acordat premii. valoarea totală a subvențiilor financiare și premiilor acordate studenților participanți la stagiile de practică în cadrul proiectului perfmed a fost de 385.125,00 lei. pentru mai multe informații privind activitătile proiectului perfmed vă invităm să accesați site-ul web al proiectului: http://perfmed.umf.ro. persoană de contact: gabriela milea, expert pentru promovare și informare universitatea de medicina și farmacie "carol davila" din bucurești & 544 & very low & Low & Socio-Economic & NA & NA & 2015-12-08 & 2015 & 1 & ECO
Frame & v.low & National & 500-1000 & 1.0255497 & 0.6327204 & -1.4131940 & 0.5015415 & -0.9523600 & 0.0 & -1.0024021 & 0.3314703 & Recipient & Domestic & Domestic & Domestic & Domestic|ECO & Positive\\
Romania & http://www.mediafax.ro/social/dancila-despre-scrisoarea-corinei-cretu-in-care-se-se-vorbeste-despre-riscul-dezangajarii-fondurilor-ue-este-o-evaluare-pe-anul-trecut-vom-avea-o-discutie-17218200 & 522 & Mediafax.ro & Private/Non-Public & Online only & National & medium = CP is important part of story & Mismanagement & Balanced & EU + National & No myth & NA & NA & NA & NA & NA & NA & NA & NA & Romania & dăncilă, despre scrisoarea corinei creţu în care se vorbeşte despre riscul dezangajării fondurilor ue: este o evaluare pe anul trecut. vom avea o discuţie & 2018-05-16 & fondul de coeziune & premierul viorica dăncilă a declarat, miercuri, că va avea o întâlnire cu corina creţu, comisar pe politici regionale şi a discutat deja cu ministrul transporturilor, în contextul scrisorii transmise de creţu în care se se vorbeşte despre riscul dezangajării fondurilor ue. "ştiu despre scrisoarea doamnei comisar corina creţu, comisar pe politici regionale. acea scrisoare este o evaluare a activităţii pe domeniul transporturilor pentru anul trecut. o să am o întâlnire cu doamna corina creţu, pe data de 21, la bacău. pot să vă spun că am vorbit deja cu ministrul transporturilor pentru a vedea cum accelerăm procesul de absorbţie a fondurilor europene", a declarat viorica dăncilă, întrebată fiind scrisoarea corinei creţu. comisarul european, corina creţu, i-a transmis, pe 25 aprilie, premierului viorica dăncilă, miniştrilor lucian şova şi rovana plumb, o scrisoare în care îşi exprima îngrijorarea cu privire la riscul dezangajării fondurilor ue, potrivit documentului prezentat, luni, de pnl. "mă adresez dumneavoastră cu privire la situaţia actuală a investiţiilor din domeniul transporturilor, ca parte a programului operaţional infrastructură mare (poim) pentru perioada 2014-2020. având în vedere că ne apropiem de jumătatea actualei perioade de programare şi că este momentul în care serviciile comisiei încep pregătirile pentru următoarea perioadă de programare, sunt extrem de îngrijorată cu privire la planificarea şi implementarea proiectelor de infrastructură de transport care sunt cofinanţate de fondul de coeziune şi de fondul european de dezvoltare regională (feder) în românia. (...) în acest sens autorităţile române vor trebui să acorde o atenţie deosebită exerciţiilor bugetare pentru doua fază se vor apropia de finalizare iar noile proiecte riscă a fi încă în faza de demarare, fără să genereze cheltuielile necesare", afirmă comisarul european pentru politică regională, corina creţu, în scrisoarea adresată ministrului transporturilor, lucian şova, ministrului fondurilor europene, rovana plumb, şi premierului viorica dăncilă. creţu atrage atenţia că până în prezent doar patru noi mari proiecte majore pentru sectorul feroviar au fost depuse în intervalul 2014-2018, exprimându-şi îngrijorarea cu privire la faza incipientă în care se află proiectele. "pregătirea şi prezentarea de proiecte noi pentru perioada de programare actuală a fost mult întârziată. până în prezent, numai patru proiecte majore au fost depuse într-un interval de timp de patru ani: tronsonul de cale ferată radna-gurasada-simeria (1 306 milioane eur din fondul de coeziune, aprobat), autostrada sebeş-turda faza ii (272 de milioane eur din feder, aprobat), autostrada câmpia turzii-ogra-târgu-mureş (247 milioane eur din fondul de coeziune, în curs de evaluare) şi magistrala 6 a reţelei de metrou din bucureşti (657 milioane eur din fondul de coeziune, în curs de evaluare). recent, autorităţile române au anunţat o listă de proiecte majore de transport pe care doresc să le înainteze comisiei în 2018, printre care: podul peste dunăre de la brăila, dn73 braşov-piteşti (faza a doua), sibiupiteşti loturile 1 şi 5, eventual, craiova-piteşti, autostrada transilvania şi varianta de ocolire bacău. salut faptul că noi proiecte sunt în curs de pregătire spre a fi înaintate comisiei, dar în acelaşi timp îmi exprim îngrijorarea faţă de nivelul scăzut de maturitate al acestora", se mai arată în scrisoare. comisarul european mai precizează că această cauză ţara noastră riscă "dezangajări imediate". & 527 & medium & Medium & Governance & NA & NA & 2018-05-16 & 2018 & 3 & POL
Frame & low-medium & National & 500-1000 & 1.0255497 & 0.6327204 & -1.4131940 & 0.5015415 & -0.9523600 & 0.0 & -1.0024021 & 0.3314703 & Recipient & Domestic & European & Mixed & Domestic|POL & Neutral\\
Romania & http://www.mediafax.ro/economic/investitie-de-57-milioane-de-euro-pentru-infrastructura-de-banda-larga-din-romania-cum-se-va-desfasura-proiectul-ro-net-13445993?utm\_source=feedburner\&utm\_medium=feed\&utm\_campaign=Feed\%253A\%2Bmediafax\%252FQddx\%2B\%2528Mediafax\_ALL\%2529 & 573 & Mediafax.ro & Private/Non-Public & Online only & National & medium = CP is important part of story & Public services & Positive & EU + National & No myth & Solidarity to poor countries/regions & Positive & EU & No myth & NA & NA & NA & NA & Romania & investiţie de 57 milioane de euro pentru infrastructura de bandă largă din românia. cum se va desfăşura proiectul ro-net & 2014-10-23 & politica regională & proiectul ro-net va acoperi 783 din cele 2268 de localităţi identificate în românia ca "zone albe" şi va contribui la reducerea decalajului digital dintre zonele urbane şi cele rurale prin oferirea accesului la internet în bandă largă unui număr de aproximativ 130.000 de gospodării cu 400.000 de locuitori, 8.500 de întreprinderi şi 2.800 de instituţii publice. la începutul lunii iulie, ministerul pentru societatea informaţională a stabilit câştigătoare ofertele depuse de romtelecom şi cosmote, care acum operează sub brandul telekom, pentru concensionarea lucrărilor de construire a reţelei ro-net, contractul ridicându-se la 365,8 milioane de lei, tva inclusă (83,1 milioane euro). "proiectele de acest tip servesc la crearea unor condiţii-cadru îmbunătăţite pentru mediul de afaceri şi cetăţeni şi permit regiunilor mai puţin dezvoltate din românia să beneficieze de oportunităţile oferite de economia digitală", a declarat johannes hahn, comisarul european pentru politica regională. investiţia de 57,1 milioane de euro are loc în cadrul programului "creşterea competitivităţii economice", fiind finanţată pe axa prioritară "tic pentru sectoarele privat şi public", potrivit comunicatului comisiei europene. proiectul urmează a fi implementat până la sfârşitul anului 2015. proiectul ro-net va veni în completarea infrastructurii de telecomunicaţii aparţinând operatorilor existenţi în regiunile respective şi va conferi acces deschis şi nediscriminatoriu la furnizorii de servicii pentru întreprinderi sau consumatori. decizia de cofinanţare a proiectului se încadrează în perioada de programare 2007-2013. româniei i-au fost alocate fonduri în valoare de aproximativ 20 miliarde de euro în cadrul politicii de coeziune în perioada 2007-2013. pentru perioada 2014-2020, româniei îi sunt alocate 22,9 miliarde de euro (la preţurile actuale). & 273 & medium & Medium & Socio-Economic & Values & NA & 2014-10-23 & 2014 & 1 & ECO
Frame & low-medium & National & <500 & 1.0255497 & 0.6327204 & -1.4131940 & 0.5015415 & -0.9523600 & 0.0 & -1.0024021 & 0.3314703 & Recipient & Domestic & European & Mixed & Domestic|ECO & Positive\\
\addlinespace
Romania & http://www.magazinsalajean.ro/politica/voi-il-mai-credeti & 516 & magazinsalajean.ro & Private/Non-Public & Online only & Regional/Local & very low = CP mentioned once & Ineffective goal achievement & Negative & EU + National & No myth & NA & NA & NA & NA & NA & NA & NA & NA & Romania & voi îl mai credeți? & 2018-05-17 & fondul de coeziune & prăpastia dintre faliile care rup românia în două se adâncește de la o zi la alta. desigur, cu concursul nemijlocit al politrucilor români. "există o temere din ce în ce mai mare, o teamă din ce în ce mai mare, o supărare din ce în ce mai mare generată de faptul că, în scurt timp, adevărul economic, adevărul social, adevărul fiscal va ieşi atât de mult în evidenţă încât toate aceste minciuni nu vor mai putea fi folosite pentru a manipula oameni şi încearcă să facă orice pentru ca guvernarea să fie blocată", declara zilele trecute condamnatul şef al psd, președintele camerei deputaților, inculpat în câteva dosare penale pe rol, liviu dragnea, despre declaraţiile făcute de preşedintele klaus iohannis, şi nu numai, cu privire la situaţia economică a româniei, ajunsă în prag de colaps sub actuala guvernare. și, cum altfel?! în viziunea lui dragnea şi a guvenului dăncilă, cu sprijinul nemijlocit al televiziunilor de casă al psd-alde, unde mai pot avea acces acești indivizi, lucrurile sunt minunate, chiar fabuloase. indiferent de imaginea reală a ecomomiei țării, fiți conviși că acolo, în pepiniera infractorilor, totul va fi prezentat pe modelul fantasmagoriei. desigur, în politică, la fel ca în viața de zi cu zi, adevărul se află nu neapărat la mijloc, ci undeva între tabere. într-o dispută, o tabără va avea propriul ei "adevăr", contrar "adevărului" celeilalte tabere. însă, de data aceasta, statistica nu poate fi mințită. nici măcar de către acești politruci versați și versatili. cifrele statistice spun mereu adevărul, atâta timp cât nu se "umblă" la ele. iată-le, așadar: statul a cheltuit pe primul trimestru cu 1 miliard de euro mai mult decât a încasat (adică decât îşi permitea); contul curent al balanţei de plăţi a înregistrat un deficit de 967 milioane euro în primul trimestru al acestui an, cu 25,6 la sută mai mare faţă de ianuarie - martie 2017 (contul curent al balanţei de plăţi include sumele încasate şi cele plătite de românia în relaţiile comerciale cu celelalte state); ponderea veniturilor totale ale statului, exprimate ca procent din pib, a scăzut de la 7,3 la sută în 2016, la 6.9 la sută în 2017 şi la 6.9 la sută în 2018. sigur că nu toată lumea e la curent cu toți termenii de mai sus și, mai grav, mulți ar putea să ne plictisească ori să dobândească subit dureri de cap, iar bugetarii care deja primesc semnale că rămân fără salariile de curând mărite sunt niște mincinoși, de altfel, în timp ce liderii psd dețin deja formula magică prin care pot păstra bugetul doldora de bani, chiar dacă mereu rup din el și nu pun nimic la loc. fraților, lucrurile nu-s deloc roz. o spune chiar o pesedistă (m-am ferit, recunosc, de termenul acesta, în condițiile în care nu se prea "pupă" cu prototipul acetui tip de politruc) din sălaj, europarlamentarul corina crețu: "sunt extrem de îngrijorată cu privire la planificarea şi implementarea proiectelor de infrastructură de transport care sunt cofinanţate de fondul de coeziune şi de fondul european de dezvoltare regională (feder) în românia. pentru perioada de programare 2014-2020, 5.1 miliarde eur au fost rezervate pentru investiţiile în domeniul transportului. din bugetul disponibil de 5.1 miliarde eur pentru transport, proiectele fazate (proiecte care au început în perioada de programare 2007-2013, dar trebuie să fie finalizate în cadrul poim 2014-2020) necesită 1.8 miliarde eur, respectiv 35\% din această sumă. (...) pregătirea şi prezentarea de proiecte noi pentru perioada de programare actuală a fost mult întârziată. până în prezent, numai patru proiecte majore au fost depuse într-un interval de timp de patru ani (....) în acelaşi timp îmi exprim îngrijorarea faţă de nivelul scăzut de maturitate al proiectelor. pentru a permite absorbţia alocaţiilor bugetare disponibile în anii următori, este nevoie ca proiecte noi să fie lansate urgent pe teren. în acest sens autorităţile române vor trebui să acorde o atenţie deosebită exerciţiilor bugetare pentru 2019 şi anii ulteriori pentru a se evita dezangajarea de fonduri, întrucât proiectele aflate în a doua fază se vor apropia de finalizare iar noile proiecte riscă a fi încă în faza de demarare, fără să genereze cheltuielile necesare". acestea sunt câteva fragmente dintr-o scrisoare adresată ministerului transporturilor, de corina creţu, comisarul european pentru politică regională, membru cu vechime în psd. însă, dacă-l auzi pe dragnea, guvernul vioricăi deține cel mai mare grad de absorbţie de fonduri europene nerambursabile. evident, în realitate, situaţia este dezastruoasă - şi asta o spune chiar comisarul european corina creţu, membru de seamă al psd. şi nici că o putea spune cineva mai bine şi mai clar. am convingerea că, așa cum dragnea ne minte astăzi cu privire la marile realizări ale cabinetului dăncilă, la fel ne-a mințit și cu programul ăla de guvernare, cu care a reușit să câștige alegerile. și, de ce nu, pentru el este la fel de simplu să mintă în fața instanței cum că nici usturoi n-a mâncat, nici gura nu-i miroase. însă, hai să nu ne antepronunțăm cu privire la pseudo vinovăția "nevinovatului" condamnat liviu dragnea. o va face instanța, la ultimul termen din judecata de fond, în 29 mai, atunci când vom afla dacă este sau nu vinovat în dosarul în care este acuzat de instigare la abuz în serviciu și fals. el zice că-i nevinovat. voi îl mai puteți crede? & 896 & very low & Low & Socio-Economic & NA & NA & 2018-05-17 & 2018 & 3 & ECO
Frame & v.low & Regional & 500-1000 & 1.0255497 & 0.6327204 & -1.4131940 & 0.5015415 & -0.9523600 & 0.0 & -1.0024021 & 0.3314703 & Recipient & Domestic & European & Mixed & Domestic|ECO & Negative\\
Romania & http://ziarulfaclia.ro/stamatian-sceptic-in-privinta-construirii-spitalului-regional-de-urgenta/ & 531 & ziarulfaclia.ro & Private/Non-Public & Online only & Regional/Local & medium = CP is important part of story & Mismanagement & Balanced & EU + National + Subnational & No myth & NA & NA & NA & NA & NA & NA & NA & NA & Romania & stamatian, sceptic în privinţa construirii spitalului regional de urgenţă & 2018-02-10 & politica regională & deputatul liberal de cluj, florin stamatian a avertizat că dacă în acest an nu încep lucrările la spitalele regionale de urgenţă există riscul pierderii finaţării europene. parlamentarul a precizat că inclusiv comisarul european corina creţu a recunoscut recent că există riscul ca banii europeni pentru aceste spitale să fie pierduţi dacă nu se urgentează procedurile de execuţie. "doamna creţu ( corina creţu, comisarul european pentru politică regională. n.r.), în conferinţa de presă a zis un lucru, pe care noi îl ştiam, îl bănuiam, că sunt cei 150 de milioane de euro alocaţi pentru spitalele regionale de urgenţă, care vin de la comunitatea europeană şi care dacă anul acesta nu se începe construcţia celor trei spitale, se pierd banii. n-am auzit nimic de un studiu de fezabilitate care să fie gata, pentru că trebuie să fie apoi şi proiectarea gata, ca să începi. nu? sperăm să se mişte ceva mai repede decât proiectarea unei autostrăzi, că nu înţeleg ce se întâmplă", a declarat deputatul florin stamatian. medicul stamatian a mai arătat că mai există şi un alt program prin care pot fi accesaţi bani europeni pentru sănătate, dar că şi aici avem o problemă cu întocmirea proiectelor. "în acelaşi timp a zis că sunt 500 de milioane de euro pentru reabilitarea spitalelor mici. acum mi-e greu să spun în ce măsură directorii de spitale mici sunt capabili să facă proiecte europene, dar dacă cei de la ţară au reuşit, primăriile rurale, să facă programe europene, înseamnă că trebuie învăţaţi şi directorii, managerii de spitale mici, să facă proiecte europene prin care să-şi reabiliteze infrastructura şi să facă rost şi de aparatură, cât să poată să atragă personalul medical, în condiţiile în care suntem în criză de medici", a precizat deputatul stamatian. potrivit ministerului sănătăţii, până în luna martie ar urma să fie finalizată documentaţia tehnică pentru toate cele trei spitale regionale( cluj-napoca , iaşi, craiova ) ce urmează să fie construite în românia. "această documentaţie tehnică este parte componentă a aplicaţiei pentru finanţare din por a acestor obiective investiţionale. depunerea aplicaţiilor este estimată a se realiza în primul semestru al anului 2018 conform ghidului specific aferent programului operaţional regional (por) 2014-2020, axa prioritară 8: dezvoltarea infrastructurii sanitare şi sociale şi indicaţiilor mdrap'', a arătat ministerul sănătăţii. în toamna anului trecut, consiliul judeţean cluj a predat ministerului sănătăţii terenul în suprafaţă de 143.064 mp, situat în localitatea floreşti, strada avram iancu nr. 370 - 374, în scopul construirii spitalului regional de urgenţă cluj. valoarea de inventar a terenului este de 25.760.295 lei. & 425 & medium & Medium & Governance & NA & NA & 2018-02-10 & 2018 & 3 & POL
Frame & low-medium & Regional & <500 & 1.0255497 & 0.6327204 & -1.4131940 & 0.5015415 & -0.9523600 & 0.0 & -1.0024021 & 0.3314703 & Recipient & Domestic & European & Mixed & Domestic|POL & Neutral\\
Romania & http://www.romanialibera.ro/politica/uniunea-europeana/romania-are-la-dispozitie-peste-570-de-milioane-de-euro-pentru-combaterea-saraciei-si-a-excluziunii-sociale-419702 & 536 & RomaniaLibera.ro & Private/Non-Public & Online and Offline & National & high = CP is most important issue in story (can also cover other issues) & Social justice & Positive & EU & No myth & NA & NA & NA & NA & NA & NA & NA & NA & Romania & românia are la dispoziţie peste 570 de milioane de euro pentru combaterea sărăciei şi a excluziunii sociale | romania libera & 2016-06-14 & fondul social european & românia are la dispoziţie peste 572 de milioane de euro, în perioada 2014-2020, pentru proiecte ce vizează incluziunea socială şi reducerea sărăciei, a declarat, marţi, comisarul european pentru politică regională, corina creţu, citat de agerpres. aceasta a discutat marţi, la bruxelles, cu ministrul muncii, familiei, protecţiei sociale şi persoanelor vârstnice din românia, dragoş pîslaru, pe marginea pachetului naţional anti-sărăcie, a agendei urbane şi a integrării comunităţilor marginalizate, informează un comunicat al reprezentanţei comisiei europene în românia. "românia are la dispoziţie peste 572 de milioane de euro, finanţare din fondul european de dezvoltare regională (fedr) pentru perioada 2014-2020, pentru proiecte ce vizează incluziunea socială şi reducerea sărăciei. pentru a asigura o implementare eficientă este extrem de important să existe o bună coordonare între fedr şi fondul social european. ministerele de resort şi autorităţile de management trebuie să-şi accelereze eforturile pentru a demara cât mai curând intervenţiile planificate", a declarat corina creţu. comisarul european pentru politică regională corina creţu a discutat marţi, la bruxelles, cu ministrul muncii, familiei, protecţiei sociale şi persoanelor vârstnice din românia, dragoş pîslaru, pe marginea pachetului naţional anti-sărăcie, a agendei urbane şi a integrării comunităţilor marginalizate, informează un comunicat al reprezentanţei comisiei europene în românia. comisarul european corina creţu a subliniat faptul că pachetul naţional anti-sărăcie poate atrage fonduri din mai multe programe operaţionale, în special din programul operaţional regional şi programul operaţional "capital uman", pentru a găsi soluţii la provocări precum părăsirea timpurie a şcolii, sărăcia, excluziunea socială, cât şi pentru probleme din domeniul sănătăţii şi al locuinţelor. "evaluarea nevoilor este o condiţie prealabilă pentru prioritizarea investiţiilor noastre. în această privinţă românia a făcut progrese în identificarea domeniilor ce vor fi vizate de iniţiativele curente de dezinstituţionalizare a persoanelor cu dizabilităţi. ne aşteptăm ca aceste eforturi să fie extinse şi la domeniul protecţiei copilului", a adăugat comisarul european. conform eurostat, rata persoanelor expuse riscului de sărăcie sau de excluziune socială în românia este de aproape 40\%. de asemenea, cele mai afectate sunt categoriile vulnerabile, jumătate dintre copiii din românia fiind expuşi riscului de sărăcie, potrivit sursei citate. în cadrul perioadei de programare 2014-2020 se urmăreşte ca 75.000 de copii, 10.000 de persoane cu dizabilităţi şi 62.000 de persoane vârstnice să beneficieze de condiţii de viaţă mai bune, prin reabilitarea infrastructurii de servicii sociale. & 386 & high & High & Socio-Economic & NA & NA & 2016-06-14 & 2016 & 2 & ECO
Frame & high-very high & National & <500 & 1.0255497 & 0.6327204 & -1.4131940 & 0.5015415 & -0.9523600 & 0.0 & -1.0024021 & 0.3314703 & Recipient & European & European & European & European|ECO & Positive\\
Romania & http://www.ziuaconstanta.ro/diverse/stiri-calde/studenti-absorbiti-pe-piata-fortei-de-munca-proiecteaza-ti-cariera-la-universitatea-ovidius-518207.html & 538 & ZIUA de Constanta & Private/Non-Public & Online only & Regional/Local & very low = CP mentioned once & Social justice & Positive & Subnational & No myth & Jobs & Positive & Subnational & No myth & NA & NA & NA & NA & Romania & studenti absorbiti pe piata fortei de munca: proiecteaza-ti cariera!, la universitatea ovidius & 2014-10-29 & fondul social european & universitatea politehnica din bucureşti, în parteneriat cu universitatea "ovidius" din constanţa şi universitatea "dunărea de jos" din galaţi organizează evenimentul ce marchează debutul reuniunilor informale infostud din cadrul proiectului "proiectează-ţi cariera!" posdru/161/2.1/g/136624, cofinanţat din fondul social european, prin programul operaţional sectorial dezvoltarea resurselor umane 2007-2013. obiectivul general al proiectului este asigurarea tranziţiei a 360 de studenţi din regiunea sud est spre piaţa forţei de muncă, facilitând integrarea lor în societatea româneasca. proiectul se adresează studenţilor, masteranzilor şi absolvenţilor universităţii "ovidius" din constanţa, facultatea de inginerie mecanică, industrială şi maritimă, aflaţi atât în etapa tranziţiei de la şcoală la viaţa activă, cât şi celor aflaţi în etapa construirii carierei profesionale. în cadrul întâlnirii, se vor purta discuţii privind importanţa consilierii şi orientării profesionale în vederea construirii unei cariere de succes. invitatul special al acestei dezbateri este ing. ovidiu mocanu. evenimentul va avea loc în data de 30 octombrie, de la ora 17.00, la sala acn, b-dul mamaia nr.124. & 167 & very low & Low & Socio-Economic & Socio-Economic & NA & 2014-10-29 & 2014 & 1 & ECO
Frame & v.low & Regional & <500 & 1.0255497 & 0.6327204 & -1.4131940 & 0.5015415 & -0.9523600 & 0.0 & -1.0024021 & 0.3314703 & Recipient & Domestic & Domestic & Domestic & Domestic|ECO & Positive\\
Romania & http://ziuadecj.realitatea.net/educatie/inchiderea-oficiala-a-proiectuluieficientizarea-activitatilor-de-consiliere-si-orientare-profesionala-a-elevilor-prin-metode-computerizate--143125.html & 501 & ziuadecj.realitatea.net & Private/Non-Public & Online only & Regional/Local & very low = CP mentioned once & Jobs & Positive & Subnational & NA & NA & NA & NA & NA & NA & NA & NA & NA & Romania & inchiderea oficiala a proiectului"eficientizarea activitatilor de consiliere si orientare profesionala a elevilor prin metode computerizate" & 2015-11-27 & fondul social european & în românia, gradul de abandon la universitățile de stat a crescut considerabil, aproximativ 4 din 10 studenți abandonând studiile în primii ani de facultate. totodată, rata șomajului în rândul tinerilor este de 4 ori mai mare decât rata șomajului în rândul adulților, crescând în ultimii patru ani de la 18\% la 23\%. (p) această situație este generată, pe de o parte, de faptul că studiile nu se pliază pe cerințele pieței muncii, pe de altă parte, de o insuficientă orientare și consiliere profesională a elevilor. în vederea optimizării procesului de consiliere și orientare în carieră, s.c. cognitrom s.r.l., în parteneriat cu asociația națională a psihologilor școlari, a implementat în perioada aprilie 2014 - noiembrie 2015 proiectul strategic "eficientizarea activităților de consiliere și orientare profesională a elevilor prin metode computerizate". proiectul a beneficiat de cofinanțare din fondul social european prin programul operațional sectorial dezvoltarea resurselor umane 2007 - 2013, axa prioritară 2 - "corelarea învățării pe tot parcursul vieții cu piața muncii", domeniul major de intervenție 2.1 - "tranziția de la școală la viața activă". principalul obiectiv al proiectului a vizat creșterea șanselor de continuare a studiilor și de inserție pe piața muncii a 20.000 de elevi de liceu din regiunile vest și nord-vest, prin îmbunătățirea serviciilor de consiliere și orientare profesională, asistate de computer. "cercetările și experiența dobândită în cadrul acestui proiect ne-au permis să dezvoltăm cea mai competitivă aplicație software de orientare în carieră a elevilor de liceu din românia. această aplicație se bazează pe instrumente validate științific prin care se stabilește care sunt ocupațiile cele mai potrivite cu aptitudinile, interesele și valorile elevilor", au afirmat organizatorii proiectului. aplicația a fost testată pe un eșantion semnificativ și are o uzabilitate și eficiență dovedite științific. ea va intra în dotarea permanentă a unor licee din județele bihor, bistrița - năsăud, cluj, maramureș, hunedoara și timiș. * peste 21.500 de elevi consiliați și orientați profesional, din județele bihor, bistrița - năsăud, cluj, maramureș, hunedoara și timiș; * o analiză swot a metodelor și instrumentelor computerizate de consiliere și orientare care a stat la baza dezvoltării unei platformei intranet; * un chestionar de autoevaluare a intereselor ocupaționale și un chestionar de autoevaluare a valorilor ocupaționale, etalonate și validate pentru populația românească; * 101 de studii privind noile ocupații de pe piața muncii; * peste 200 de materiale utile în activitățile de consiliere în carieră ale elevilor (din care: 100 de activitățile independente destinate elevilor; 85 de exerciții pentru activități la clasă sub îndrumarea profesorilor consilieri; 10 ilustrări video ale unor ocupații și 10 înregistrări video de prezentare ai unor potențiali angajatori); * un studiu privind consilierea și orientarea profesională a generației digitale; * un raport privind eficiența platformelor multiuser online pentru consilierea și orientarea profesională a elevilor și un raport privind uzabilitatea platformelor multiuser online pentru consilierea și orientarea profesională a elevilor; * 28 de acorduri de parteneriat încheiate cu instituții de învățământ prin care acestea beneficiază cu titlu gratuit de platforma intranet de orientare în carieră dezvoltată în cadrul proiectului, pe o perioadă nedeterminată. & 497 & very low & Low & Socio-Economic & NA & NA & 2015-11-27 & 2015 & 1 & ECO
Frame & v.low & Regional & <500 & 1.0255497 & 0.6327204 & -1.4131940 & 0.5015415 & -0.9523600 & 0.0 & -1.0024021 & 0.3314703 & Recipient & Domestic & Domestic & Domestic & Domestic|ECO & Positive\\
\addlinespace
Romania & http://ziarulfaclia.ro/finantare-pentru-programe-de-ucenicie-si-stagiatura-pentru-tinerii-neets/ & 496 & ziarulfaclia.ro & Private/Non-Public & Online and Offline & Regional/Local & very low = CP mentioned once & Social awareness/inclusion & Positive & Subnational & No myth & Jobs & Positive & Subnational & No myth & NA & NA & NA & NA & Romania & finanţare pentru programe de ucenicie şi stagiatură pentru tinerii neets & 2018-06-25 & fondul social european & agenţia naţională pentru ocuparea forţei de muncă (anofm) a anunţat că acordă sprijin financiar angajatorilor care organizează programe de ucenicie și/sau stagii pentru tinerii neets. anofm implementează, pe o perioadă de patru ani, proiectul "unit 2 rmpd - ucenicie și stagii pentru tineri neets din regiunile mai puțin dezvoltate", proiect cofinanțat de uniunea europeană din fondul social european prin programul operațional capital uman 2014 - 2020. prin implementarea acestui proiect se urmăreşte îmbunătățirea posibilităților de încadrare prin programe de ucenicie și/sau stagii a cel puțin 900 tineri neets șomeri înregistrați la serviciul public de ocupare, cu rezidența în regiunile nord-vest (din care face parte şi judeţul cluj), nord-est, vest, sud-vest oltenia. obiectivele proiectului sunt stimularea ocupării a 636 tineri neets șomeri înregistrați la serviciul public de ocupare, prin acordarea sprijinului financiar angajatorilor, aferent încadrării prin ucenicie la locul de muncă, precum şi acordarea unui sprijin financiar pentru încadrarea prin stagii a 264 tineri neets șomeri, absolvenți de învățământ superior. concret, angajatorii care organizează programe de ucenicie și/sau programe de stagii în condițiile legii, pot beneficia de sprijin financiar nerambursabil astfel: * angajatorul care încheie, în condiţiile legii, un contract de ucenicie cu un tânăr neet din evidenţele anofm beneficiază, la cerere, pe întreaga perioadă de derulare a contractului de ucenicie (12 luni, 24 luni, 36 de luni, în funcţie de nivelul de calificare al programului de ucenicie), de 1.125 lei/lună * angajatorul care încheie, în condiţiile legii, un contract de stagiu cu un tânăr neet din evidenţele anofm beneficiază, la cerere, pe perioada derulării contractului de stagiu (6 luni), de 1.350 lei/lună. conform legii 76/2002 privind sistemul asigurărilor pentru şomaj şi stimularea ocupării forţei de muncă, tânărul neet este "persoana cu vârsta cuprinsă între 16 ani şi până la împlinirea vârstei de 25 de ani, care nu are loc de muncă, nu urmează o formă de învăţământ şi nu participă la activităţi de formare profesională". perioada de implementare a programului este de 48 de luni, respectiv 19 februarie 2018 - 18 februarie 2022, fiind prevăzut un buget total de 38.629.300,79 lei din care 5.794.395,11 lei finanţare naţională și 32.834.905,68 lei finanţare externă nerambursabilă. & 368 & very low & Low & Socio-Economic & Socio-Economic & NA & 2018-06-25 & 2018 & 3 & ECO
Frame & v.low & Regional & <500 & 1.0255497 & 0.6327204 & -1.4131940 & 0.5015415 & -0.9523600 & 0.0 & -1.0024021 & 0.3314703 & Recipient & Domestic & Domestic & Domestic & Domestic|ECO & Positive\\
Romania & http://zdbc.ro/comunicat-de-presa-privind-derularea-proiectului-consiliere-inovare-simulare-pentru-acces-real-la-piata-muncii-c-i-s-4/ & 512 & Ziarul de Bacau & Private/Non-Public & Online only & Regional/Local & very low = CP mentioned once & Jobs & Positive & Subnational & No myth & NA & NA & NA & NA & NA & NA & NA & NA & Romania & comunicat de presă privind derularea proiectului "consiliere, inovare, simulare - pentru acces real la piaţa muncii (c.i.s.)" & 2015-08-01 & fondul social european & proiectul consiliere, inovare, simulare - pentru acces real la piaţa muncii (c.i.s.), pos dru/161/2.1/g/139667, cofinanţat din fondul social european, prin programul operaţional sectorial dezvoltarea resurselor umane, axa prioritară 2, domeniul major de intervenţie 2.1., al cărui beneficiar este inspectoratul şcolar judeţean bacău în parteneriat cu centrul judeţean de resurse şi asistenţă educaţională bacău, colegiul tehnic "dumitru mangeron" bacău şi colegiul naţional "costache negri" târgu ocna, se află în perioada de implementare. activitate-nucleu a proiectului, târgul inter - regional al firmelor de exerciţiu a avut loc la colegiul tehnic "dumitru mangeron" bacău, în perioada 22-24.05.2015. s-au înscris 22 de firme de exerciţiu, dintre care 17 au fost cu participare directă şi 5 cu participare indirectă, doar pentru anumite secţiuni ale concursului. târgul inter - regional al firmelor de exerciţiu organizat în cadrul proiectului cis şi-a propus: dezvoltarea competenţelor antreprenoriale ale elevilor şi recunoaşterea activităţii lor; efectuarea de tranzacţii directe cu alte firme de exerciţiu; promovarea conceptului de firmă de exerciţiu pe plan local, judeţean şi regional; promovarea progresului şi a rezultatelor proiectului cis la nivel local, judeţean, regional şi naţional. au participat: firmele de exerciţiu înfiinţate în cadrul proiectului "consiliere, inovare, simulare - pentru un acces real la piața muncii (c.i.s.)", la cele 2 şcoli: colegiul tehnic "dumitru mangeron" bacău şi colegiul naţional ,,costache negri" târgu ocna; firme de exerciţiu din judeţ şi din alte regiuni de dezvoltare ale ţării; firme de exerciţiu din toată ţara; reprezentanţi ai i.s.j. bacău; profesori coordonatori; factori de decizie de la nivel local; mass-media; agenţi economici; profesori şi elevi de la cele două şcoli partenere în cadrul proiectului; profesori si elevi de la alte şcoli şi licee din judeţul bacău; vizitatori, persoane interesate. în cadrul firmelor de exerciţiu elevii îşi formează deprinderi şi abilitați, îşi însușesc competenţe cheie, cum ar fi capacitatea de a lucra în echipa, gândire interdisciplinara şi critică, competenţe de comunicare şi relaționare instituțională. mai mult de atât, obțin abilități profesionale, o flexibilitate necesară pe piața muncii şi pot descoperi o afinitate faţă de un anumit loc de muncă. competenţele obținute pot determina reducerea perioadei de acomodare la locul de muncă, familiarizarea cu sarcinile înscrise în fişa postului, lucru benefic atât pentru angajatori cât şi pentru viitorii angajați. din acest motiv orice participare a elevilor la evenimente care permit contactul direct cu mediul de afaceri este deosebit de benefică pentru dezvoltarea competenţelor antreprenoriale şi de comunicare ale acestora. experienţa de a conduce direct negocierile şi de a dezvolta spiritul de echipă se pot exersa cu succes numai în cadrul unui târg al firmelor de exerciţiu. aici pot fi întâlniţi parteneri de afaceri, se poate comunica direct cu ei, elevii pot conduce în mod real negocieri, pot învăța şi să ia decizii şi îşi dezvoltă încrederea în forţele proprii. & 471 & very low & Low & Socio-Economic & NA & NA & 2015-08-01 & 2015 & 1 & ECO
Frame & v.low & Regional & <500 & 1.0255497 & 0.6327204 & -1.4131940 & 0.5015415 & -0.9523600 & 0.0 & -1.0024021 & 0.3314703 & Recipient & Domestic & Domestic & Domestic & Domestic|ECO & Positive\\
Romania & http://www.romanialibera.ro/opinii/editorial/locul-i-la-exportul-de-tatism-424644 & 523 & RomaniaLibera.ro & Private/Non-Public & Online and Offline & National & very low = CP mentioned once & Mismanagement & Negative & EU + National & NA & NA & NA & NA & NA & NA & NA & NA & NA & Romania & locul i la exportul de ţăţism | romania libera & 2016-08-04 & politica regională & nu mă pricep la fonduri europene. am o vagă idee despre ce înseamnă. ştiu şi eu cam cât ştiţi şi dumneavoastră, cetăţenii obişnuiţi, care nu şi-au dat doctoratul în asta. citesc însă de ani de zile ştirile, sondajele şi statisticile potrivit cărora suntem codaşii europei la gradul de absorbţie. ciulesc urechea la specialiştii care ne spun, din când în când, de ce nu merge la noi treaba cu finanţările ue. de trei luni încoace, subiectul a revenit în actualitate. de bună seamă, pentru că avem un nou ministru al fondurilor europene. care, spre deosebire de înaintaşii săi, a reuşit măcar să provoace o mică dezbatere publică. asta pentru că domnul cristian ghinea, venit din zona societăţii civile, s-a hotărât să nu mai ţină cifrele sub preş. a ieşit la rampă şi ne-a spus cum stăm după guvernarea ponta. şi stăm prost, foarte prost! rezultatul? în loc să vină cu contraargumente, partidul fostului premier l-a chemat la ordine. mai întâi în parlament, la o şedinţă de înfierare în comisia de specialitate. la care a participat, pe post de invitat, însuşi deputatul sebastian ghiţă. chiar cel acuzat că a parazitat, pe partea de it, accesarea de fonduri europene. evident, din discuţie nu putea lipsi victor ponta. care a sărit şi el, de pe facebook, pe noul ministru. marţi, 2 august 2016, după o lungă tăcere, s-a trezit din somn şi corina creţu de la bruxelles. în caz că aţi uitat, dumneaei este comisar european pentru politică regională. într-o declaraţie de presă, distribuită tot pe facebook, îşi exprima îngrijorarea pentru faptul că n-am încasat niciun euro din banii alocaţi pentru 2014-2020. tonul scrisoricii pare un ciripit de rândunică, dar printre rânduri putem citi mesajele de propagandă ale partidului care a propulsat-o în înalta funcţie. a doua zi, au urmat o nouă postare a ministrului şi un nou răspuns al comisarului. nici nu vreau să mă gândesc ce-o fi în capul europenilor care citesc schimbul de replici dintre doamna comisar şi domnul ministru. replici date în văzul întregii lumi. se întreabă probabil, ca şi mine, dacă românia, ţara lui nenea iancu, nu este totuşi pe primul loc la ceva: exportul de ţăţism. & 368 & very low & Low & Governance & NA & NA & 2016-08-04 & 2016 & 2 & POL
Frame & v.low & National & <500 & 1.0255497 & 0.6327204 & -1.4131940 & 0.5015415 & -0.9523600 & 0.0 & -1.0024021 & 0.3314703 & Recipient & Domestic & European & Mixed & Domestic|POL & Negative\\
Romania & http://www.cotidianul.ro/sii-analytics-la-comisia-sri-297565/ & 546 & Cotidianul.ro & Private/Non-Public & Online only & National & low = CP mentioned more times but NOT important part of story (mainly about others issues) & Fraud/Corruption & Negative & National & No myth & NA & NA & NA & NA & NA & NA & NA & NA & Romania & sii analytics, la comisia sri & 2017-03-12 & fondul european de dezvoltare regională & comisia parlamentară pentru controlul activității sri a solicitat serviciului, dar şi altor instituţii, date privind sii analytics, după ce cinci organizaţii ale societăţii civile au cerut, în luna ianuarie, investigarea atribuirii proiectului sii analytics al serviciului român de informaţii, transmite mediafax în pagina electronică. "este în verificare, am cerut relaţii de la sri şi de la alte instituţii. aştept răspunsurile, nu pot să le decid eu că nu sunt organ de specialitate pe subiectul ăsta. acolo ridicau aspecte legate de legalitatea atribuirii contractului, de calificarea sri pe acest subiect, nu le puteam clarifica eu. am cerut relaţii de la sri şi de la alte instituţii. când o să am un răspuns o să anunţ", a declarat, pentru agenția citată, adrian ţuţuianu, şeful comisiai sri, solicitat să precizeze ce s-a întâmplat cu adresa a cinci organizaţii ale societăţii civile, transmisă în luna ianuarie. sesizarea, iniţiată de către societatea academică din românia şi semnată de apador ch, asociaţia pentru tehnologie şi internet, active watch, centrul de resurse juridice şi miliţia spirituală, vizează investigarea proiectului sii analytics al serviciului român de informaţii (sri). organizaţiile semnatare arată că faptele, asupra cărora parlamentarii sunt sesizaţi, duc la " un precedent periculos pentru democraţia noastră, contravine dreptului european şi bunei gestionări a fondurilor europene". în sesizarea propriu-zisă se arată că există suspiciunea că sri ar fi fost favorizat înaintea şi în cadrul procedurii de licitaţie a proiectului din fondul european de dezvoltare regională (fedr), licitaţie la care, de altfel, a existat un singur ofertant. mai există însă o mare temere în rândul organizaţiilor neguvernamentale, şi anume, că acest proiect ar putea fi un program de supraveghere în masă. "anul trecut, ministerul comunicaţiilor a acordat un proiect sri-ului care a provocat foarte mult scandal, în primul rând fiind problema de conţinutul acelui proiect. practic, temerile sunt că este vorba despre un program de supraveghere în masă. ce îşi propune să facă sri prin acel program este să agregheze baze de date conţinând date personale de la diverse instituţii ale statului, ceea ce încalcă în mod evident legislaţia protecţiei datelor cu caracter personal, pentru că aşa cum ştiţi, în momentul în care o instituţie pasează la o altă instituţie publică datele mele sau ale dvs personale trebuie să ne ceară acordul. există o singură excepţie pe această chestie, care este siguranţa naţională. când e vorba despre siguranţa naţională, nu mai trebuie acordul persoanei vizate. dar proiectul nu este unul de siguranţă naţională, ci unul pe e-guvernare, ceea ce ridică din nou semne de întrebare. ce treaba are sri cu e-guvernarea? în fine, explicaţii sunt", a declarat, pentru mediafax, victoria stoiciu, membru afiliat al societăţii academice din românia. de altfel, organizaţiile neguvernametale au făcut, în luna ianuarie a acestui an, o sesizare şi către oficiul european de luptă anti-fraudă (olaf) în care se precizează că există o "suspiciune rezonabilă de favorizare din partea autorităţii de management (ministerul fondurilor europene) şi a organismului intermediar pentru promovarea societăţii informaţionale (ministerul pentru societatea informaţională) a unei entităţi publice în procedura de atribuire a unui proiect finanţat prin fedr". "problema este următoarea, legată de alocarea acestui proiect. e publicat pe site în 7 iunie şi pe 8 iunie, a doua zi, sri depune proiectul. este o chestie foarte complicată, să depui un proiect pentru finanţare europeană, nu e ceva pe care l-ai făcut în câteva ore. adică aveau toată documentaţia pregătită şi este şi singurul care depune acel proiect, adică nu mai există competitor şi ia şi 80 la sută din fondul disponibil. este ceea ce se cheamă single bidder (singur ofertant) ceea ce ridică semne de întrebare. noi acum nu avem niciun fel de dovadă, dar tocmai de asta am sesizat olaf, pentru că situaţia cel puţin naşte suspiciuni rezonabile. după aceea mai e un aspect. legislaţia prevede foarte clar, e legea din 2001 privind tehnica legislativă, prevede foarte clar că un act intră în vigoare din momentul publicării sale în monitorul oficial. ori ordinul secretarului de stat de aprobare a ghidului solicitantului, trebuie aprobat prin ordin al ministrului sau al secretarului de stat care are un act normativ. se aprobă în 7 iunie, dar e publicat abia pe 27 iulie în monitorul oficial, adică la o luna şi jumătate, de unde eu şi ceilalţi care am semnat plângerea ne punem următoarea întrebare: oare nu cumva... adică noi credem că procedura normală care se urmează în toate proiectele europene este asta, se aprobă ordinul, se publică în monitorul oficial, după aceea se dă drumul la finanţări. or aici nu s-a făcut aşa, s-a aprobat ordinul, s-a dat drumul la finanţare, s-au luat banii şi după aia s-a publicat în monitorul oficial", spune victoria stoiciu. reprezentanţii asociaţiilor atrag atenţia că nicăieri în ghidul solicitantului nu se menţionează care sunt cele 36 de de evenimente de viaţă ale cetăţenilor a căror digitalizare ar fi trebuit să facă obiectul finanţării. tocmai de aceea sunt enumerate evenimentele care ar face obiectul: starea civilă (căsătorie, divorţ, deces); activitatea agenţilor economici (înfiinţarea sau închiderea unei firme); drepturile şi obligaţiile cetăţeneşti; munca, familia şi protecţia socială; afacerile externe; parcursul educaţional (gimnaziu, universitate sau chiar o înscriere a bibliotecă); serviciile medicale; informaţii de călătorie. "observând lista de evenimente, opinăm că sri nu este eligibil conform ghidului pentru că nu gestionează, nu coordonează şi nici nu susţine servicii publice ce vizează evenimentele de viaţă de mai sus, la fel cum nici nu contribuie la dezvoltarea acestora", se precizează în scrisoarea către olaf. şi apador-ch a anunţat, în vara anului trecut, că patru organizaţii neguvernamentale au trimis o scrisoare deschisă mai multor instituţii naţionale şi europene, prin care cer anularea proiectului sri finanţat din fonduri europene, despre care susţin că ar fi unul "de supraveghere în masă". "proiectul, numit "sii analytics", are un potenţial de supraveghere generalizată a întregii populaţii a româniei - un adevărat sistem informatic big brother - fără a avea prevăzută vreo măsură de garantare a drepturilor cetăţeneşti sau de limitare a accesului sri sau al altor instituţii publice la datele personale colectate şi integrate în acest sistem, scăpare ce încalcă serios drepturile fundamentale", se arată într-un comunicat de presă al asociaţiei pentru apărarea drepturilor omului în românia - comitetul helsinki. răspunsul din partea comisiei europene, prin reprezentanţa de la bucureşti, a venit abia la jumătatea lunii decembrie 2016. "în conformitate cu principiul gestiunii partajate care se aplică în politic de coeziune, autorităţile de management din statele membre sunt responsabile de aplicarea şi monnitorizarea proiectelor cofinanţate din fondurile uniunii europene. ele au responsabilitatea de a se asigura că asistenţa furnizată prin intermediul fondurilor este în concordantă cu activităţile, politicile şi priorităţile uniunii europene. potrivit datelor furnizate de autoritatea de management, proiecul a fost selecţionat în conformitate cu normele naţionale şi ale uniunii europene. beneficiarul are obligaţia de a respecta legile naţionale cu privire la prelucrarea datelor cu caracter personal şi toate celelalte legi aplicate la nivel naţional, dar şi în cadrul uniunii europene. suntem la curent în legătură cu preocupările exprimate în unele ong-uri din românia cu privire la acest proiect şi urmărim îndeaproape acest subiect", se arată în răspuns. proiectul "sii analytics - sistem informatic de integrare şi valorificare operaţională şi analitică a volumelor mari de date" este destinat asigurării unei capacităţi superioare de analiză a bazelor de date ale principalelor instituţii din românia. obiectivul platformei este de a spori considerabil viteza de căutare a informaţiei relevante în bazele de date deja existente. practic, în loc să interogheze sisteme diferite, neuniformizate informatic şi procedural, instituţiile statului vor putea accesa informaţiile integrat, rapid şi eficient", transmitea sri în august 2016. potrivit sursei citate, sistemul nu colectează date noi, ci le analizează, pe baza unor algoritmi, pe cele existente, măsurile de rapiditate în accesarea bazelor de date fiind impuse "ca necesitate de ameninţările specifice instituţiilor de intelligence din românia - terorism, migraţie ilegală, crimă organizată, pentru a căror contracarare este necesară o primă reacţie în timp foarte scurt". valoarea proiectului a fost estimată la 142,071 milioane de lei, reprezentând aproximativ 31,5 milioane de euro, din care 84,3411\% finanţare din fonduri externe nerambursabile, anunţa sri. & 1346 & low & Low & Governance & NA & NA & 2017-03-12 & 2017 & 2 & POL
Frame & low-medium & National & +1000 & 1.0255497 & 0.6327204 & -1.4131940 & 0.5015415 & -0.9523600 & 0.0 & -1.0024021 & 0.3314703 & Recipient & Domestic & Domestic & Domestic & Domestic|POL & Negative\\
Romania & http://www.bzi.ro/comisia-europeana-ingrijorata-de-ponderea-foarte-mare-a-suspiciunilor-de-achizitii-trucate-la-fondurile-ue-din-romania-479082 & 543 & BZI.ro & Private/Non-Public & Online only & National & medium = CP is important part of story & Fraud/Corruption & Balanced & EU + Subnational & NA & Ineffective goal achievement & Balanced & National & NA & NA & NA & NA & NA & Romania & comisia europeana, ingrijorata de ponderea "foarte mare" a suspiciunilor de achizitii trucate la fondurile ue din romania & 2015-01-30 & fondurile structurale şi de investiţii europene & comisia europeana este ingrijorata de ponderea "foarte mare" a plangerilor privind posibile achizitii publice trucate in proiectele finantate din fonduri europene, in totalul reclamatiilor pe achizitiile publice, se arata in raportul mecanismului de cooperare si verificare pe justitie, adoptat miercuri de executivul comunitar. "din plangerile legate de achizitiile publice din romania, aproape 40\% se refera la achizitii publice finantate din fonduri ue, dupa cum se mentioneaza cel mai recent raport al consiliului national pentru solutionarea contestatiilor. la nivel local si regional, potrivit dna cluj-napoca, fraude legate de proiectele mari pe fonduri ue rareori se refera la totalitatea proiectului, ci la parti din el. de exemplu, circa 20 - 30 de cazuri grave sunt descoperite in fiecare an numai in aceasta regiune", arata comisia in aparatul tehnic al raportului mcv. citeste si in curand apar noi programe europene pentru fermierii din iasi in raportul mcv adoptat miercuri de colegiul comisarilor europeni, sunt condamnate mai ales achizitiile publice frauduloase de la nivel local. "procedurile de achizitii publice, in special la nivel local, sunt in continuare grevate de acte de coruptie si de conflicte de interese - un fapt recunoscut pe scara larga de autoritatile romane in materie de integritate si de aplicare a legii. acest lucru a avut consecinte negative asupra absorbtiei fondurilor ue", precizeaza comisia. capacitatea administrativa slaba a autoritatilor publice romanesti este si ea mentionata printre neajunsurile care ingreuneaza absorbtia fondurilor europene. citeste si ieseni, ati fost prostiti din nou! iohannis si ponta gireaza afacerile oneroase ale camarilei "este de asemenea adevarat ca sunt multi alti factori in joc - inclusiv capacitatea administrativa a autoritatilor care organizeaza achizitii publice, lipsa de stabilitate si fragmentarea cadrului juridic, precum si calitatea concurentei in domeniul achizitiilor publice", se mai arata in raportul mcv pentru anul 2014. in context, comisia europeana indeamna guvernul sa indeplineasca conditionalitatile ex-ante in domeniul achizitiilor publice, pentru o mai buna absorbtie de fonduri structurale. "reinnoirea dialogului structurat dintre comisie si romania in contextul transpunerii noilor directive privind achizitiile publice si al conditionalitatii ex ante pentru fondurile structurale si de investitii europene ar trebui sa contribuie la identificarea deficientelor, inclusiv a aspectelor expuse riscului de coruptie si de conflicte de interese", scrie in raportul mcv. citeste si anunt important pentru afaceristii din iasi care solicita fonduri europene de asemenea, comisia europeana recomanda guvernului "sa recurga la strategia nationala anticoruptie pentru a identifica mai bine domeniile expuse riscului de coruptie si sa adopte masuri educative si preventive, cu sprijinul ong-urilor si profitand de oportunitatile oferite de fondurile ue". & 417 & medium & Medium & Governance & Socio-Economic & NA & 2015-01-30 & 2015 & 1 & POL
Frame & low-medium & National & <500 & 1.0255497 & 0.6327204 & -1.4131940 & 0.5015415 & -0.9523600 & 0.0 & -1.0024021 & 0.3314703 & Recipient & Domestic & European & Mixed & Domestic|POL & Neutral\\
\addlinespace
Romania & http://www.mediafax.ro/economic/ce-mobilizeaza-35-de-miliarde-de-euro-pentru-grecia-cretu-atena-are-nevoie-de-reforme-si-investitii-14596730?utm\_source=feedburner\&utm\_medium=feed\&utm\_campaign=Feed\%253A\%2Bmediafax\%252FQddx\%2B\%2528Mediafax\_ALL\%2529 & 565 & Mediafax.ro & Private/Non-Public & Online only & National & very low = CP mentioned once & Economic development & Positive & EU + Other country & No myth & NA & NA & NA & NA & NA & NA & NA & NA & Romania & ce mobilizează 35 de miliarde de euro pentru grecia. creţu: atena are nevoie de reforme şi investiţii & 2015-07-15 & fondurile structurale şi de investiţii europene & "la două zile după realizarea unui consens privind măsurile ce se impun pentru încheierea unui nou acord de împrumut acordat greciei, comisia europeană a dezvăluit astăzi planurile de ajutorare a greciei în vederea maximizării gradului de utilizare a fondurilor ue. conform mandatului conferit de summitul zonei euro din 12-13 iulie, vor fi astfel mobilizate peste 35 de miliarde de euro până în 2020 pentru a veni în sprijinul economiei greceşti", potrivit unui comunicat al ce. pachetul de măsuri privind crearea de locuri de muncă şi creşterea economică destinat greciei însoţeşte un set cuprinzător de reforme ce va face parte dintr-un program creat în cadrul mecanismului european de stabilitate. acest program urmează a fi negociat de grecia cu partenerii săi internaţionali în săptămânile următoare. ambele elemente - reformele şi mobilizarea de fonduri pentru investiţii şi coeziune - sunt condiţii preliminare esenţiale pentru crearea de noi locuri de muncă, redresarea economică a greciei şi revenirea ţării la prosperitate. "planul pentru locuri de muncă şi creştere economică îşi propune să ajute cetăţenii şi companiile greceşti să depăşească perioada dificilă de criză din ultimii ani şi să redea speranţa într-un viitor mai bun, în special tinerei generaţii. acesta vine ca o continuare a sprijinului acordat deja greciei de comisie pe perioada crizei, atât din punct de vedere financiar, cât şi tehnic", precizează ce. ca o măsură excepţională şi în lumina situaţiei unice din grecia, comisia propune îmbunătăţirea situaţiei lichidităţilor pentru a permite în continuare finanţarea investiţiilor din perioada de programare 2007 - 2013. acestea vor include: plata anticipată a restului de 5\% din finanţările europene, procent în mod normal reţinut până la închiderea programelor; şi aplicarea unei rate de cofinanţare de 100\% pentru perioada 2007 - 2013. aceste măsuri echivalează cu lichidităţi în valoare de 500 de milioane de euro şi economii la bugetul greciei de aproximativ 2 miliarde de euro. aceste sume vor fi disponibile pentru reluarea imediată a investiţiilor în sprijinul creşterii economice şi creării locurilor de muncă, cu condiţia ca autorităţile greceşti să se asigure că aceste fonduri vor fi utilizate în favoarea beneficiarilor şi operaţiunilor din cadrul programelor. comisia va propune, de asemenea, creşterea nivelului de pre-finanţare pentru programele din perioada 2014-2020 cu 7 puncte procentuale. prin această pre-finanţare suplimentară poate fi pus la dispoziţie încă un miliard de euro care poate fi utilizat exclusiv pentru lansarea proiectelor cofinanţate în cadrul politicii de coeziune cu respectarea pe deplin a articolului 81 (2) din regulamentul dispoziţiilor comune. executivul european reaminteşte că grecia "s-a bucurat de tratament preferenţial şi până acum: programele greceşti finanţate cu fonduri ue în cadrul perioadei de programare 2007-2013 primesc finanţare ue într-o proporţie mai mare". din acest motiv, grecia trebuie să cofinanţeze mai puţin decât celelalte ţări graţie cofinanţării suplimentare de 10\% din partea ue, asigurată până la jumătatea anului 2016. în multe cazuri, acest lucru înseamnă că ue plăteşte 95\% din costul total al investiţiei pentru perioada de finanţare 2007-2013 (spre deosebire de maximum 85\% aplicabil celorlalte ţări). în plus, în cazul politicii de coeziune, dacă sunt îndeplinite toate condiţiile, autorităţile greceşti pot primi, în continuare, rambursare până la atingerea pragului legal de 95\% pentru cheltuielile eligibile aferente programelor din perioada 2007-2013. anunţul de miercuri vine în continuarea înfiinţării unui grup la nivel înalt condus de vicepreşedintele dombrovskis. împreună cu autorităţile greceşti, grupul îşi propune să garanteze utilizarea banilor rămaşi din perioada de programare 2007-2013 înainte de atingerea termenului-limită, respectiv la sfârşitul acestui an, şi să ajute grecia să îndeplinească cerinţele necesare pentru a accesa toate fondurile ue disponibile în actuala perioadă de programare 2014-2020. totodată, grecia va continua să primească ajutor pentru implementarea reformelor din partea noului serviciu de sprijin pentru reforme structurale, care şi-a început activitatea la 1 iulie şi beneficiază de experienţa valoroasă a grupului de lucru pentru grecia şi de asistenţa tehnică oferită statelor membre. în urma summitului zonei euro din 12 iulie, comisiei i s-a cerut să contribuie la sprijinirea creării de locuri de muncă şi creşterii economice din grecia în următorii 3-5 ani. comisia a primit astfel sarcina de a "lucra îndeaproape cu autorităţile greceşti pentru a mobiliza fonduri de până la 35 de miliarde de euro (în cadrul diferitelor programe ale ue) în vederea finanţării investiţiilor şi activităţii economice, inclusiv a imm-urilor". cele peste 35 de miliarde de euro pe care grecia ar putea să le primească în cadrul perioadei de programare 2014-2020 cuprind 20 miliarde de euro din fondurile structurale şi de investiţii europene şi 15 miliarde de euro din fondurile agricole. acestea pot fi direcţionate către investiţii, combaterea şomajului, a sărăciei şi a condiţiilor sociale precare, cercetare şi educaţie, infrastructură. primele plăţi efectuate din aceste fonduri ue în 2014 şi 2015 însumează deja 4,4 miliarde de euro. "utilizarea fondurilor ue pentru grecia nu a fost simplă în ultima vreme. în ultimele luni, condiţiile financiare severe şi nesiguranţa privind situaţia economică au perturbat planurile de investiţii, punând sub semnul întrebării capacitatea greciei de a utiliza pe deplin şi corect fondurile ue disponibile", afirmă ce. "economia greacă a intrat din nou în recesiune, sistemul bancar este aproape de colaps (...) grecia are nevoie acum de stabilitate şi încredere. odată ce stabilitatea financiară este restaurată, ţara poate să se ocupe de redresarea economică, crearea de locuri de muncă şi un viitor mai bun, în special pentru cei mai vulnerabili. pentru a se întâmpla asta, autorităţile din grecia trebuie să reformeze urgent statul şi să creeze o administraţie eficientă şi fără de corupţie, să furnizeze sisteme de securitate sociale, care să fie sustenabile şi corecte, să creeze condiţii mult mai atractive pentru investiţii", a declarat vicepreşedintele comisiei europene pentru moneda euro şi dialog social, valdis dombrovskis. "implementarea strictă a acestor reforme este esenţială pentru recâştigarea încrederii în rândul celor 19 democraţii ale zonei euro", a spus dombrovskis. la rândul său, comisarul european pentru politici regionale, corina creţu, a afirmat că ue s-a angajat să sprijine în continuare "eforturile greciei de a-şi redresa economia şi de a reveni pe creştere economică". "în acest scop, reformele stabilite în cadrul reuniunii eurogrupului sunt necesare, dar am discutat astăzi în colegiul comisarilor şi toţi am căzut de acord că doar aceste reforme nu pot aduce grecia din nou pe calea creşterii şi a creării de locuri de muncă. ce este convinsă că reformele trebuie să fie însoţite de investiţii ambiţioase, care ar readuce economia greacă pe drumul cel bun din nou şi ar ajuta atât la gestionarea şomajului, cât şi a cauzelor sale", a precizat corina creţu. dacă ţi-a plăcut articolul, urmăreşte mediafax.ro pe facebook " conținutul website-ului www.mediafax.ro este destinat exclusiv informării și uzului dumneavoastră personal. este interzisă republicarea conținutului acestui site în lipsa unui acord din partea mediafax. pentru a obține acest acord, vă rugăm să ne contactați la adresa vanzari@mediafax.ro. & 1141 & very low & Low & Socio-Economic & NA & NA & 2015-07-15 & 2015 & 1 & ECO
Frame & v.low & National & +1000 & 1.0255497 & 0.6327204 & -1.4131940 & 0.5015415 & -0.9523600 & 0.0 & -1.0024021 & 0.3314703 & Recipient & European & European & European & European|ECO & Positive\\
Romania & https://romanialibera.ro/politica/tariceanu-se-opune-unei-europe-cu-mai-multe-viteze-758895 & 550 & RomaniaLibera.ro & Private/Non-Public & Online and Offline & National & high = CP is most important issue in story (can also cover other issues) & Political leverage & Balanced & National & No myth & NA & NA & NA & NA & NA & NA & NA & NA & Romania & tăriceanu se opune unei europe cu mai multe viteze | romania libera & 2018-10-30 & politica de coeziune & preşedintele senatului, călin popescu-tăriceanu, respinge "cu tărie" construirea unei europe cu mai multe viteze, subliniind că o astfel de abordare subminează rolul internaţional al ue ca un actor puternic şi legitim. "anul viitor va fi cu adevăr o provocare pentru noi toţi, dincolo de faptul că ne confruntăm cu consecinţele primului pas înapoi major pentru ue, mă refer la brexit, ne confruntăm cu o confruntare de ideologii şi, în acest context, ar trebui cu toţii să jucăm un rol mai predominant şi să încurajăm toate instituţiile europene să ia toate măsurile necesare pentru a menţine coeziunea în rândul societăţilor noastre, care sunt de diferite niveluri de dezvoltare şi ambiţii. datorită implicaţiilor macroeconomice şi macro-sociale, politica de coeziune europeană poate fi considerată cel mai important instrument în corectarea dezechilibrelor regionale şi în sporirea bunăstării cetăţenilor", a afirmat tăriceanu. el a spus că "politica de coeziune nu constituie un ajutor pentru vecini, ci o încercare de a construi o piaţă mai flexibilă şi mai profitabilă, mai prosperă care să sporească standardele de trai pentru toţi cetăţenii europeni". "de aceea, viitorul europei este implicit viitorul româniei, iar aceasta depinde de ancorarea fermă de polul stabilităţii, pe de o parte, şi, pe de altă parte, pe polul prosperităţii", a adăugat tăriceanu. în opinia sa, pentru românia, constituie aspecte de îngrijorare "menţinerea unor politici agricole comune neschimbate şi evaluarea intenţiei anunţate de unii de a introduce noi criterii de convergenţă pentru gestionarea următorului cadru financiar". "cred că ne confruntăm cu toţii cu o etapă crucială, trebuie ca solidaritatea să fie în centrul proiectului european, trebuie să învăţăm din propriile noastre dificultăţi pentru a soluţiona astfel de probleme. (...) resping cu tărie construirea unei europe cu mai multe viteze, că unii sunt mai europeni decât alţii.(...) dincolo de faptul că este contraproductivă şi aduce conflicte constante în rândul statelor membre, o astfel de abordare subminează rolul internaţional al ue ca un actor puternic şi legitim", a susţinut călin popescu-tăriceanu. el a subliniat că trebuie luată o decizie importantă în cadrul ue şi găsite răspunsuri la mai multe întrebări. "se pare că rolul nostru este să luăm o decizie importantă, vom alege unitatea sau izolarea, critica sau dialogul, vom fi o europă mai puternică financiar sau vom fi învinşi de greşelile trecutului nostru. cred că acestea sunt întrebările importante care necesită răspunsuri reale, pentru a ne asigura că ciclul financiar multianual va pune la dispoziţie resursele necesare pentru a consolida posibilitatea de a face provocărilor de astăzi şi de mâine", a completat tăriceanu. el s-a referit şi la preluarea de către românia a preşedinţiei consiliului ue. "suntem onoraţi că deţinem anul viitor preşedinţia consiliului ue. sibiu, unul dintre cele mai frumoase oraşe ale româniei, cu încărcătură istorică, va organiza summitul. sunt optimist în privinţa concluziilor. succesul preşedinţiei române depinde de noi toţi, pentru a oferi răspunsuri sincere la aspecte care au puterea de a crea succesul sau insuccesul oricărei efort. (...) cred că modalitatea cea mai uşoară de a oferi răspunsuri la întrebările de mai sus stă în eficienţa politicii europene de coeziune, acest instrument bine folosit are capacitatea de a acoperi decalajele actuale şi de a vindeca rănile trecutului. vă reamintesc că ue nu şi-a pierdut niciodată capacitatea de a genera speranţa, atât la nivel intern, cât şi mondial. solicit tuturor celor competenţi să acţioneze", a conchis tăriceanu. & 555 & high & High & Power & NA & NA & 2018-10-30 & 2018 & 3 & POL
Frame & high-very high & National & 500-1000 & 1.0255497 & 0.6327204 & -1.4131940 & 0.5015415 & -0.9523600 & 0.0 & -1.0024021 & 0.3314703 & Recipient & Domestic & Domestic & Domestic & Domestic|POL & Neutral\\
Romania & http://www.bzc.ro/conferinta-comisiei-coter-modele-de-bune-practici-in-absorbtia-fondurilor-europene-la-cluj-napoca-apreciate-de-oficialii-europeni-65406 & 562 & BZC.ro & Private/Non-Public & Online only & National & very high = CP is most important issue + CP is mentioned in title/headline & Economic development & Positive & EU + National & No myth & Infrastructure & Positive & Subnational & No myth & NA & NA & NA & NA & Romania & conferința comisiei coter: modele de bune practici în absorbția fondurilor europene la cluj-napoca... & 2019-03-27 & fondurile structurale & modele de bune practici în absorbția fondurilor europene la cluj-napoca apreciate de oficialii europeni: autobuze electrice, autobuze școlare, prioritizarea transportului în comun și modenizarea spațiilor publice. primarul municipiului cluj-napoca, emil boc, a prezentat în data de 25 martie a.c., în cadrul conferinței "susținerea și dezvoltarea unei politici de coeziune eficiente pornind de la autoritățile locale și regionale din uniunea europeană", modul în care clujul a beneficiat direct de politica de coeziune, în contextul unui panel construit pe marginea gradului de implementare al acestei politici la nivel european, în prezentul exercițiu bugetar. țara noastră este un beneficiar net al politicii de coeziune. în românia au venit 45 de miliarde de euro de la aderare. este o datorie să promovăm valorile europene pentru că de ele depinde soarta europei. de aceea, avem nevoie de o atitudine pro absorbție, pro fonduri europene. politica de coeziune este o politică profundă de care depinde viitorul europei. la nivelul municipiului cluj-napoca, impactul politicii de coeziune este semnificativ, aducând beneficii zilnice cetățenilor. proiecte majore precum achiziționarea de autobuze electrice, autobuzele școlare cu circuit închis sau reamenajarea spațiilor publice, precum și politica de prioritizare a transportului în comun au fost asimilate ca modele de bune practici de către oficialii europeni prezenți la cluj-napoca pentru conferința comisiei coter. de asemenea, la cluj-napoca, cetățenii au văzut concret impactul politicii de coeziune. printr-un proiect de mare succes pentru românia și pentru orașe în special: programul de reabilitare termică a blocurilor și a clădirilor publice, unde deja s-au epuizat resursele financiare și dacă, inclusiv în acest exercițiu financiar, se vor găsi mai mulți bani pentru acest program va fi o dovadă clară a faptului că uniunea europeană ajunge în fiecare casă. în privința clujului, în actualul exercițiu financiar există proiecte care au fost semnate, în fază de implementare sau în faza de evaluare în valoare de 420 de milioane de euro, unde sunt incluse și cele 150 de milioane de euro pentru centura metropolitană a orașului. "pentru cetățenii clujului, fonduri europene înseamnă creșterea calității vieții. de aceea, pe lângă componenta de inovare, ne-am concentrat pe componenta de calitate a vieții pentru a oferi cele mai bune servicii cetățenilor noștri. din această perspectivă, ne-am concentrat pe dimensiunea nepoluantă, pe dimensiunea verde și pe mobilitatea alternativă la transportul cu autoturismul personal astfel încât să oferim o calitate mai bună a vieții cetățenilor noștri prin mai multe spații pietonale, prin mai multe piste de biciclete, prin transport ecologic. o treime din transportul public din oraș este în acest moment electric și acest proces va continua într-un ritm foarte accelerat", a susținut primarul la conferință. în uniunea europeană, politica de coeziune este principala politică europeană de investiții cu ajutorul căreia se poate realiza obiectivul din tratatul de aderare referitor la coeziunea economică, socială și teritorială. în acest sens, politica de coeziune are o valoare adăugată clară în ceea ce privește crearea de locuri de muncă, creșterea durabilă și o infrastructură modernă, depășirea barierelor structurale, impulsionarea capitalului uman și îmbunătățirea calității vieții pentru toți cetățenii din întreaga ue. orașele și regiunile sunt principalii beneficiari ai politicii de coeziune a ue. noua politică de coeziune după 2020 este, prin urmare, un aspect esențial pentru orașele și regiunile europene. concluzia dezbaterilor este una clară: politica de coeziune este esențială pentru dezvoltarea economică și socială, de această politică se leagă, fără doar și poate, viitorul unității europene. --- întâlnirea comisiei cor coter este principalul eveniment al comitetului european al regiunilor privind politica de coeziune şi fondurile structurale și este organizat în aceste zile la cluj-napoca, în contextul preşedinţiei româniei la consiliul uniunii europene. & 604 & very high & High & Socio-Economic & Socio-Economic & NA & 2019-03-27 & 2019 & 3 & ECO
Frame & high-very high & National & 500-1000 & 1.0255497 & 0.6327204 & -1.4131940 & 0.5015415 & -0.9523600 & 0.0 & -1.0024021 & 0.3314703 & Recipient & Domestic & European & Mixed & Domestic|ECO & Positive\\
Romania & https://www.realitatea.net/romania-a-luat-cu-imprumut-1-miliard-de-euro-ce-va-face-cu-banii\_2085132.html & 582 & REALITATEA.NET & Private/Non-Public & Online only & National & high = CP is most important issue in story (can also cover other issues) & Infrastructure & Positive & EU + National & No myth & NA & NA & NA & NA & NA & NA & NA & NA & Romania & românia a luat cu împrumut 1 miliard de euro! ce va face cu banii & 2017-07-18 & fondurile structurale şi de investiţii europene & banca europenă de investiţii (bei) acordă româniei un împrumut de un miliard de euro pentru cofinanţarea cu fondurile structurale şi de investiţii europene a proiectelor prioritare de infrastructură în domeniul transporturilor, în valoare totală de 6,8 miliard de euro, care vor fi implementate în ţara noastră în perioada de programare 2014 - 2020. împrumutul este acordat pe o perioadă de până la 25 de ani, din care o perioadă de graţie de până la 7 ani, fiecare tragere fiind considerată un împrumut de sine stătător, cu propria sa maturitate şi perioadă de graţie. dobânda poate fi fixă sau variabilă, stabilită de comun acord cu bei şi nu se va percepe decât în cazul în care se efectuează trageri din împrumut. rambursarea fiecărei trageri poate fi efectuată în una sau mai multe tranşe, conform unui comunicat al ministerului finanţelor publice. "românia are calitatea de împrumutat şi este reprezentată de ministerul finanţelor publice, iar promotorul este ministerul transporturilor. beneficiarii proiectelor sunt autorităţile şi societăţile române relevante care implementează proiectele incluse în axele 1 şi 2 din poim", precizează mfp. fondurile bei vor acoperi contribuţia bugetului de stat care cofinanţează investiţii prioritare de infrastructură în cadrul programului operaţional infrastructură mare 2014 - 2020 (poim). aceste proiecte vor fi implementate în diverse regiuni, majoritatea fiind în zonele mai puţin dezvoltate din românia şi/sau pe axele prioritare ten-t (coridoarele pan-europene) . investiţiile vor beneficia de granturi ue şi se vor concentra pe promovarea unui transport sustenabil şi pe eliminarea blocajelor din infrastructurile cheie din reţea. de asemenea se va pune accent şi pe dezvoltarea unor sisteme de transport multimodal şi pe reducerea impactului transporturilor asupra mediului. românia este reprezentată de ministerul finanţelor publice, iar beneficiarii sunt autorităţile naţionale relevante şi companiile care implementează proiecte conform programului strategic de dezvoltare a infrastructurii naţionale în perioada de programare 2014 - 2020 (poim). "împrumutul de la banca europenă de investiţii va sprijini investiţiile strategice de infrastructură de transport, care sunt importante pentru creşterea mobilităţii, siguranţei şi a interconectivităţii în românia şi pentru întărirea competitivităţii ţării. acesta proiect reprezintă un bun exemplu al cooperării bei cu economia românească, căreia instituţia îi oferă servicii de finanţare şi consultanţă accesibile pe termen lung, accelerând absorbţia fondurilor ue şi stimulând avansul economiei şi scăderea şomajului, precum şi îmbunătăţirea vieţii de zi cu zi a cetăţenilor", a declarat andrew mcdowell, vicepreşedinte bei. ministrul finanţelor publice, ionuţ mişa, a afirmat la rândul său că "noul împrumut contractat de la bei este o dovadă a bunei colaborări stabilite între românia şi banca europenă de investiţii acând ca scop încurajarea investiţiilor în ţara noastră, punerea în practică a mater planului general de transport convenit cu comisia europeană şi creşterea absorbţiei fondurilor europene pentru perioada de programare 2014 - 2020. obţinută în condiţii financiare favorabile şi pe termen lung, acest tip de facilitate de creditare sprijină suplimentar eforturile guvernului româniei de a crea sau de a menţine locuri de muncă, de a reduce decalajele dintre diferitele zone şi de a asigura bunăstarea populaţiei într-un mediu benefic, toate ducând la o creştere sustenabilă a economiei. românia a beneficiat în trecut de patru facilităţi similare: două facilităţi legate de perioada anterioară de programare ue, în valoare de 1,3 miliarde de euro, şi alte două pentru perioada de programare curentă, în valoare de 660 milioane de euro, care acoperă proiectele de mediu din cadrul programului operaţional infrastructură mare, programul operaţional competitivitate şi programul operaţional capital uman". şi comisarul pentru politică regională corina creţu a subliniat că "dezvoltarea reţelei strategice de transport în românia va aduce beneficii semnificative comerţului şi turismului şi va alimenta direct economia reală, dar, la fel de important este şi impactul pozitiv al conexiunilor mai rapide, al drumurilor mai sigure şi al unor sisteme de transport mai sustenabile pentru cetăţeni. de aceea mă bucur să văd că fondurile de coeziune şi resursele bei sunt unificate pentru o mai bună conectivitate în românia." finanţarea bei se acordă sub forma unui împrumut pentru programe structurale (spl), care permite atât finanţarea proiectelor mari de infrastructură, cât şi a unor scheme mai mici, care, din cauza dimensiunilor lor limitate, nu ar putea beneficia de finanţare directă de către bei, se precizează într-un comunicat al reprezentanţei comisiei europene la bucureşti. iniţiativa de asistenţă tehnică jaspers (asistenţă comună pentru sprijinirea proiectelor în regiunile europene - furnizată în comun de bei şi comisia europeană) a contribuit la pregătirea unor strategii de dezvoltare pe termen lung care stau la baza selectării proiectelor care urmează a fi implementate cu fonduri din acest împrumut, verificând astfel şi asigurând calitatea consistentă a proiectelor care urmează a fi efectuate. banca europeană de investiţii este instituţia de creditare pe termen lung a uniunii europene, deţinută de statele membre ale ue. aceasta face ca finanţările pe termen lung să fie disponibile pentru investiţii solide, pentru a contribui la atingerea obiectivelor politicii ue. & 802 & high & High & Socio-Economic & NA & NA & 2017-07-18 & 2017 & 2 & ECO
Frame & high-very high & National & 500-1000 & 1.0255497 & 0.6327204 & -1.4131940 & 0.5015415 & -0.9523600 & 0.0 & -1.0024021 & 0.3314703 & Recipient & Domestic & European & Mixed & Domestic|ECO & Positive\\
Romania & http://www.evz.ro/aproape-50-de-intreprinzatori-au-accesat-subventii-de-peste-5-milioane-de-lei-si-au-creat-peste-100-de-locuri-de-munca.html & 511 & evz.ro & Private/Non-Public & Online and Offline & National & very low = CP mentioned once & Jobs & Positive & National & No myth & NA & NA & NA & NA & NA & NA & NA & NA & Romania & aproape 50 de întreprinzători au accesat subvenţii de peste 5 milioane de lei şi au creat peste 100 de locuri de muncă & 2015-11-25 & fondul social european & în cadrul proiectului "ideal - investim inteligent pentru dezvoltare economică, antreprenorială şi locuri de muncă", implementat de fundaţia centrul pentru educaţie economică şi dezvoltare din românia - ceed românia, în parteneriat cu ipa s.a. - societate comercială pentru cercetare, proiectare și producție de echipamente și instalații de automatizare, 47 de întreprinzători au beneficiat de subvenţii de până la 25.000 de euro pentru înfiinţarea propriei afaceri, suma totală a ajutorului de minimis acordat ridicându-se la 5.096.437 lei. proiectul, cofinanţat din fondul social european prin programul operaţional sectorial dezvoltarea resurselor umane 2007-2013, axa prioritară 3 "creşterea adaptabilităţii lucrătorilor şi a întreprinderilor", domeniul major de intervenţie 3.1 "promovarea culturii antreprenoriale", în cadrul cererii de propuneri de proiecte "românia start-up", s-a derulat în perioada 9 februarie - 8 decembrie 2015, în regiunile bucureşti-ilfov şi sud-vest oltenia şi s-a adresat unui număr de 457 de beneficiari. participanţii la proiect au urmat cursuri gratuite de perfecţionare a competenţelor antreprenoriale, au primit consultanţă de business şi au aplicat la concursurile de idei şi planuri de afaceri în vederea obţinerii ajutorului de minimis în valoare de până la 25.000 de euro. astfel, din cele 295 de idei de afaceri înscrise în competiţie, 247 au fost admise. ulterior, pe baza planurilor de afaceri depuse - în număr de 137, 84 au fost acceptate, iar cele mai bune 47 au primit finanţare. ca urmare a implementării proiectului ideal, au fost înfiinţate firme noi, câte 2 în judeţele gorj şi mehedinţi, câte 3 în olt şi vâlcea, 5 în dolj, 7 în ilfov şi 25 în municipiul bucureşti. aceste firme, a căror activitat este circumscrisă domeniilor turism şi ecoturism, textile şi pielărie, lemn şi mobilă, industrii creative, tehnologia informaţiilor şi comunicaţiilor, sănătate şi produse farmaceutice, au creat deja peste 100 locuri de muncă noi. "în românia, cultura antreprenoriatului este încă în formare şi nu beneficiază de o tradiţie solidă, aşa cum se întâmplă în ţările din vestul europei - spre exemplu, sau în statele unite. este important însă că avem foarte multe persoane interesate să deţină şi să gestioneze afaceri proprii. mulţi dintre români sunt atraşi de independenţa pe care ţi-o oferă propriul business, dar trebuie să înţeleagă foarte bine şi responsabilităţile care decurg de aici. noi am considerat că, pentru a-şi creşte şansele de reuşită, aceşti oameni cu înclinaţii antreprenoriale au nevoie de îndrumare, de consolidarea cunoştinţelor de afaceri şi - foarte important, de finanţare. acesta a fost rolul implementării proiectului ideal." - a declarat cristina mănescu, manager de proiect şi director executiv ceed românia. pentru a instaura un cadru civilizat de discuţii, de eliminare a "postacilor" de partid sau a celor plătiţi ca să blocheze un articol civilizat, am adoptat următoarele soluţii, în privinţa comentariilor: orice critică este acceptată pe site-ul evz.ro, cu condiţia păstrării unui limbaj civilizat, toate aceste măsuri fiind şi în sprijinul celor interesaţi să-şi expună punctele de vedere fără a mai fi hărţuiţi. & 488 & very low & Low & Socio-Economic & NA & NA & 2015-11-25 & 2015 & 1 & ECO
Frame & v.low & National & <500 & 1.0255497 & 0.6327204 & -1.4131940 & 0.5015415 & -0.9523600 & 0.0 & -1.0024021 & 0.3314703 & Recipient & Domestic & Domestic & Domestic & Domestic|ECO & Positive\\
\addlinespace
Romania & http://www.desteptarea.ro/sute-de-femei-vor-fi-ajutate-sa-si-gaseasca-un-loc-de-munca/ & 504 & Deșteptarea- Ziarul Bacăului & Private/Non-Public & Online and Offline & Regional/Local & very low = CP mentioned once & Jobs & Positive & Subnational & No myth & Social justice & Factual & Subnational & No myth & NA & NA & NA & NA & Romania & sute de femei vor fi ajutate sa-si gaseasca un loc de munca & 2015-02-18 & fondul social european & camera de comert si industrie bacau a prezentat, ieri, partea a doua a proiectului "egalitate de sanse pe piata muncii - egal", destinat promovarii egalitatii de sanse pe piata muncii. derulat inca de anul trecut, in parteneriat cu fundatia antreprenoriat social vâlcea, camera de comert harghita, fundatia centrul de pregatire profesionala vâlcea si indice consulting si management, proiectul are o valoare de peste 9,9 milioane de lei, fiind cofinantat din fondul social european prin programul operational sectorial dezvoltarea resurselor umane 2007-2013. grupul tinta sunt femeile care sunt in somaj sau au iesit din aceasta forma de sustinere din partea statului. aproximativ 450 de femei vor beneficia de cursuri de calificare pentru diverse meserii cautate pe piata muncii si 300 vor primi asistenta în vederea dezvoltarii de activitati independente sau a initierii unei afaceri. in total, vor fi 18 cursuri de perfectionare in meserii din alimentatia publica, estetica, ingrijirea batrânilor si a copiilor, precum si la cele 15 cursuri de antreprenoriat. totodata, participantele la cursuri vor primi si un sprijin financiar de circa 900 de lei pe luna pentru fiecare. "pentru judetul bacau, sunt repartizate 150 de locuri pentru tot atâtea femei, dintre care, pâna acum, s-au scolit 78 de persoane, in alimentatie publica si comert." mihai tulbure, director camera de comert si industrie bacau principalul obiectiv al proiectului este consolidarea principiului accesului egal pentru toti pe piata muncii cu scopul de a creste oportunitatile de angajare ale femeilor si asigurarea accesului egal la ocupare si la construirea unei cariere profesionale. "noi pregatim 451 de femei pe care le recalificam in meserii importante, in acest moment. acestea vor primi calificare atestata de autoritatile statului, ceea ce dovedeste ca aceste cursuri au calitatile similare cu cele din invatamântul profesional organizat de stat, daca nu chiar mai bun, daca ne gândim ca, la stat, exista unele structuri care nu mai corespund realitatii", a explicat si valentin cismaru, presedintele fundatiei antreprenoriat social vâlcea. proiectul se deruleaza pe durata a 18 luni, în mai multe judete din regiunile nord-est - inclusiv in bacau, sud-vest oltenia si centru, acesta urmând a fi incheiat in luna noiembrie a acestui an. & 356 & very low & Low & Socio-Economic & Socio-Economic & NA & 2015-02-18 & 2015 & 1 & ECO
Frame & v.low & Regional & <500 & 1.0255497 & 0.6327204 & -1.4131940 & 0.5015415 & -0.9523600 & 0.0 & -1.0024021 & 0.3314703 & Recipient & Domestic & Domestic & Domestic & Domestic|ECO & Positive\\
Romania & http://www.dcnews.ro/corina-cre-u-scrisoare-catre-ministrul-fondurilor-europene-pe-un-subiect-delicat\_487460.html & 559 & dcnews.ro & Private/Non-Public & Online only & National & very high = CP is most important issue + CP is mentioned in title/headline & Social awareness/inclusion & Balanced & EU & No myth & NA & NA & NA & NA & NA & NA & NA & NA & Romania & corina crețu, scrisoare către ministrul fondurilor europene, pe un subiect delicat & 2015-10-25 & fondurile structurale & comisarul european pentru politici regionale, corina crețu, i-a trimis o scrisoare ministrului fondurilor europene, marius nica. vă scriu aceste rânduri pentru a vă informa despre rolul pe care îl poate juca politica de coeziune în integrarea migranților și refugiaților. fondurile structurale pot într-adevăr ajuta considerabil statele membre pentru a răspunde nevoilor pe termen scurt sau lung privind integrarea în societate a migranților și refugiaților. după cum bine știți, comisia europeană a depus permanent eforturi semnificative pentru a oferi un răspuns coordonat la nivel european în materie de migrație și refugiați. în acest sens, la data de 23 septembrie, președintele juncker le-a prezentat șefilor de stat și de guvern o serie de măsuri operaționale concrete. în ceea ce privește politica noastră, fondul european de dezvoltare regională, de exemplu, poate cofinanța o gamă largă de acțiuni, de la crearea unor centre de primire, spitale mobile, furnizarea de apă și facilități sanitare, până la asigurarea infrastructurii sociale, de sănătate, educație, cazare și îngrijire a copilului; de la acțiuni de minimizare a izolării spațiale și educaționale a migranților, până la sprijinul acordat pentru înființarea de noi întreprinderi. eficacitatea acestor investiții depinde, în mare măsură, de coordonarea lor cu măsurile de integrare socială și de ocupare a forței de muncă cofinanțate din fondul social european. [citeste si] mă bucur să vă informez că o serie de măsuri concrete sunt deja în curs de aplicare ca urmare a efectului direct al acțiunilor noastre colective. de exemplu, aproximativ 220 de milioane de euro sub formă de sprijin din fondul european de dezvoltare regională au fost redirecționați în italia pentru a finanța facilități de primire și nave de patrulare și salvare. dar se poate face mai mult. suntem hotărâți să ajutăm statele membre să adapteze programele la noile circumstanțe, dacă acest lucru este necesar. deoarece statele membre sunt cele care pot evalua ce tip de investiții sunt necesare, doresc să vă invit să ne transmiteți rapid, mie și serviciilor mele, propunerile dumneavoastră. echipa mea se va asigura că veți beneficia de consiliere personalizată asupra investițiilor care pot fi finanțate, de sprijin pentru identificarea celor mai potrivite instrumente ue, precum și de ajutor în crearea unor pachete integrate. vă rog să primiți asigurările mele că vom lua toate măsurile ca propunerile să fie adoptate cât mai repede posibil", este mesajul corinei crețu pentru marius nica. & 389 & very high & High & Socio-Economic & NA & NA & 2015-10-25 & 2015 & 1 & ECO
Frame & high-very high & National & <500 & 1.0255497 & 0.6327204 & -1.4131940 & 0.5015415 & -0.9523600 & 0.0 & -1.0024021 & 0.3314703 & Recipient & European & European & European & European|ECO & Neutral\\
Romania & https://www.monitorulsv.ro/Local/2018-05-04/Deputatul-Dumitru-Mihalescul-a-interpelat-o-pe-ministrul-Rovana-Plumb-cu-privire-la-cresterea-eficacitatii-fondurilor-europene & 510 & Monitorul de Suceava & Private/Non-Public & Online and Offline & Regional/Local & low = CP mentioned more times but NOT important part of story (mainly about others issues) & Jobs & Negative & National & No myth & NA & NA & NA & NA & NA & NA & NA & NA & Romania & deputatul dumitru mihalescul a interpelat o pe ministrul rovana plumb cu privire la cresterea eficacitatii fondurilor europene & 2018-05-04 & fondul social european & deputatul pnl de suceava dumitru mihalescul i-a cerut precizări ministrului fondurilor europene, rovana plumb, cu privire la soluţii privind creşterea eficacităţii fondurilor europene utilizate prin pocu. parlamentarul a arătat că în cadrul de programare financiară 2007-2013, fondul social european a finanţat în românia numeroase proiecte derulate în cadrul programului operaţional sectorial dezvoltarea resurselor umane. "nu trebuie trecută cu vederea experienţa din ultimul exerciţiu financiar, din care românia nu numai că a pierdut sume consistente din cauza incapacităţii de a le utiliza, dar şi programele derulate au avut efecte modeste în piaţa muncii", a precizat mihalescul. deputatul a mai arătat că "creşterea economică înregistrată în ultimii ani a generat o criză tot mai consistentă a forţei de muncă bine calificată, deoarece ţara noastră a pierdut, pe calea migraţiei, o parte consistentă a forţei de muncă. în al doilea rând, ezitările şi experimentele nereuşite din domeniul educaţiei nu au adus în nici un fel soluţii salvatoare la criza de forţă de muncă ce ar trebui să fie calificată, în strictă legătură cu nevoile angajatorilor. aşa se explică tot mai puternic creşterea numărului de locuri de muncă vacante în economie, în condiţiile în care peste 450 de mii de români se află în căutarea unui loc de muncă". "agenţi economici importă forţă de muncă din state asiatice" criza forţei de muncă a generat demersuri disperate ale agenţilor economici, unii antreprenori au înfiinţat propriile şcoli de calificare, alţi agenţi economici "importă" forţă de muncă din state asiatice, a subliniat deputatul, iar "toate acestea se întâmplă în condiţiile în care sute de mii de români ar vrea să muncească, dar nu au calificarea necesară, iar românia dispune de miliarde de euro, fonduri europene nerambursabile, chiar cu destinaţia investiţiilor în capitalul uman". având în vedere faptul că "suntem în al patrulea an din cadrul de programare bugetară 2014-2020, iar, pentru componenta capital uman, ţara noastră nu a utilizat aproape nimic din fondurile disponibile (4,37 miliarde euro disponibili)", mihalescul a solicitat clarificări cu privire la gradul de contractare a sumelor ce pot fi utilizate în exerciţiul financiar 2014-2020, la data de 1 mai 2018. în acelaşi timp i-a cerut ministrului plumb să precizeze cum intenţionează să încurajeze utilizarea fondurilor nerambursabile disponibile în axele prioritare: locuri de muncă pentru toţi sau locuri de muncă pentru tineri. parlamentarul pnl i-a mai solicitat ministrului fondurilor europene să precizeze dacă în cadrul axelor prioritare ale pocu a prevăzut şi proiecte care să stimuleze integrarea elitelor ştiinţifice româneşti, nu numai în educaţie şi cercetare, ci şi în ciclurile de producţie economică. & 426 & low & Low & Socio-Economic & NA & NA & 2018-05-04 & 2018 & 3 & ECO
Frame & low-medium & Regional & <500 & 1.0255497 & 0.6327204 & -1.4131940 & 0.5015415 & -0.9523600 & 0.0 & -1.0024021 & 0.3314703 & Recipient & Domestic & Domestic & Domestic & Domestic|ECO & Negative\\
Romania & http://www.dcnews.ro/andreea-paul-criteriul-meritocra-iei-e-real-la-ce-duce\_486023.html & 500 & dcnews.ro & Private/Non-Public & Online only & National & very low = CP mentioned once & Jobs & Negative & National & No myth & Economic development & Balanced & National & No myth & NA & NA & NA & NA & Romania & andreea paul: criteriul meritocrației e real? la ce duce? & 2015-10-06 & fondurile structurale & andreea paul, prim-vicepreședintă și deputată pnl, a scris, pentru dcnews.ro, un articol în care îi dă replica președintelui camerei deputaților, valeriu zgonea, care susține că trebuie să se aplice politicienelor principiul meritocrației. andreea paul arată că scuzele pentru subreprezentarea politică a femeilor nu fac decât să aducă pierderi umane și economice româniei. situația economică ar fi alta dacă femeile ar fi egale cu bărbații în politică. iată analiza politică scrisă de andreea paul pentru dcnews.ro: ce-ar fi să aplicăm criteriul meritocrației tuturor politicienilor? dl zgonea cere meritocrație doar când vine vorba despre femeile din politică. să ne lămurim: bărbații au fost până acum aleși tot pe merite, nu-i așa?! primul care ar trebui să plece ar fi chiar președintele camerei deputaților aplicând criteriul cerut în reprezentarea femeilor, aplicat și bărbaților. dar câți alți domni au ocupat nemeritat locurile în ierarhia politică românească și nu am auzit să fie deranjați de absența criteriilor meritocratice. domnilor, când vine vorba despre femei, ele musai să fie meritocrate! nu vă dați seama cât de misogini sunteți? nu vă dați seama ce scuze ridicole folosiți pentru grava subreprezentare politică a femeilor? nu vă dați seama că pierdem resurse umane și economice consistente de dezvoltare? vorba aceea, chiar din vârful camerei deputaților. românia poate obține un plus de 30 de miliarde de euro la pib până în anul 2025 prin diminuarea dezechilibrelor de gen românia poate obține un plus de 11\% la pib până în anul 2025 prin diminuarea dezechilibrelor de gen până la nivelul celor mai bune practici europene. asta înseamnă încă 30 de miliarde de euro la pib-ul obținut în anul 2025. estimările mele au luat în calcul prognozele pib în condițiile standard de creștere economică, la care am aplicat metodologia pe care o propune compania mckinsey în raportul recent publicat, în septembrie 2015, și care atrage atenția unui potențial de 12 trilioane de dolari la creșterea economică globală până în anul 2025: 'how advancing women's equality can add \$12 trillion to global growth'. același raport indică românia printre cele trei țări cu cel mai ridicat potențial de creștere, alături de polonia și slovacia, în regiunea europei de est și a asiei centrale. potențialul de creștere economică suplimentară de 30 de miliarde de euro, adică un plus de 1.500 de euro la pib pe locuitor în românia anului 2025, poate fi obținut printr-o politică echilibrată de gen în economie, în politică și în societate, la standardele cele mai ridicate europene. [citeste si] acest potențial echivalează cu fondurile structurale și de coeziune pe care nu am reușit să le absorbim în integralitate pentru perioada 2007-2014. echivalează cu eforturile pentru absorbția a trei sferturi din fondurile alocate româniei pentru ciclul financiar 2014-2020. acest potențial al femeilor încă insuficient valorificat economic și politic echivalează cu pib-ul generat în anul 2014 de oricare două regiuni cumulate ale româniei din cele 8 regiuni de dezvoltare. excepție face doar regiunea bucurești-ilfov, care are ea însăși un aport la pib mai mare decât oricare celelalte două regiuni cumulate. acest potențial al politicilor mai echilibrate de gen echivalează cu contribuția a trei județe - timiș, cluj și brasov - de 11\%, la formarea pib în anul 2025. așadar, puterea economică a parității de gen nu mai poate fi în niciun chip neglijată. nu vorbim despre situații teoretice, ideale, de echilibre perfecte de gen, ci despre diminuarea dezechilibrelor grave de gen prin aplicarea celor mai bune rețete care funcționeaza deja în europa. notă: titlul de homepage și lead-ul aparțin redacției. titlul și textul aparțin autoarei. & 594 & very low & Low & Socio-Economic & Socio-Economic & NA & 2015-10-06 & 2015 & 1 & ECO
Frame & v.low & National & 500-1000 & 1.0255497 & 0.6327204 & -1.4131940 & 0.5015415 & -0.9523600 & 0.0 & -1.0024021 & 0.3314703 & Recipient & Domestic & Domestic & Domestic & Domestic|ECO & Negative\\
Romania & http://ziuadecj.realitatea.net/educatie/peste-15-milioane-de-lei-din-fonduri-europene-pentru-reabilitarea-unei-scoli-speciale-clujene--163907.html & 533 & ziuadecj.realitatea.net & Private/Non-Public & Online only & Regional/Local & very low = CP mentioned once & Social justice & Positive & Subnational & No myth & Environment/green/low-carbon & Factual & Subnational & No myth & NA & NA & NA & NA & Romania & peste 15 milioane de lei din fonduri europene pentru reabilitarea unei scoli speciale clujene & 2017-09-22 & fondul european de dezvoltare regională & plenul forului administrativ județean a adoptat vineri, 22 septembrie 2017, proiectul de hotărâre inițiat de președintele consiliului județean cluj, domnul alin tișe, privind aprobarea proiectului "creșterea eficienței energetice în clădirea școlii gimnaziale speciale - centru de resurse și documentare privind educația incluzivă/integrată", proiect ce urmează a fi depus spre finanțare în cadrul programului operațional regional 2014-2020. necesitatea derulării acestei investiții este determinată de vechimea și starea de degradare accentuată a corpului de clădire situat pe strada bucurești din cartierul clujean mărăști. imobilul, format din peste 40 de săli de clasă, cabinete medicale, birouri administrative și spații conexe este, în prezent, clasificat în clasa "c" din punct de vedere al performanței energetice, având costuri cu utilitățile - încălzire, apă caldă și iluminat, foarte mari. "este un nou demers dintr-un proiect mult mai complex prin care urmărim să reabilităm, cu fonduri europene, cât mai multe imobile care găzduiesc școli speciale sau unități medicale de sub autoritatea consiliului județean. creșterea eficienței energetice a acestor clădiri va contribui în mod direct atât la creșterea confortului elevilor, respectiv al pacienților, cât și la scăderea presiunii bugetare determinate de cheltuielile mari avansate pentru plata utilităților", a declarat președintele consiliului județean cluj, alin tișe. în mod concret, în vederea îndeplinirii obiectivului acestui proiect, respectiv cel de creștere a eficienței energetice, vor fi realizate o serie de lucrări care vizează în special izolarea termică a pereților exteriori ai clădirii școlii, refacerea învelitorilor și a stratificației acoperișului, instalarea unei centrale termice și de panouri solare, modernizarea instalațiilor electrice și termice. toate aceste investiții vor contribui semnificativ la reducerea consumurilor de energie din surse convenționale și la diminuarea emisiilor de gaze cu efect de seră, astfel încât consumul anual specific de energie primară - încălzire, apă și iluminat se va înjumătăți, reducându-se de la 202 kwh/mp2/an la sub 100 kwh/mp2/an. în ceea ce privește sursele de finanțare ale acestei investiții în valoare totală de 17.102.208,10 lei cu tva, acestea se constituie atât din asistența financiară nerambursabilă solicitată din fondul european de dezvoltare regională, în sumă de 15.358.898,25 lei, cât și din contribuția proprie a consiliului județean cluj, în valoare de 1.743.309,85 lei. amintim faptul că școala gimnazială specială - centru de resurse și documentare privind educația incluzivă/integrată este o instituție de învățământ special care asigură accesul la educație copiilor cu cerințe educative speciale, în scopul atingerii nivelului lor optim de dezvoltare individuală, al recuperării, reabilitării, adaptării și integrării lor școlare, profesionale și sociale. școala oferă o instruire secvențială, continuă, structurată și individualizată într-un mediu stimulativ și protector care corespunde nevoilor specifice de dezvoltare ale elevilor cu deficiență mintală ușoară, deficiență mintală moderată și deficiență mintală severă/ profundă, respectând zona proximei dezvoltări. & 456 & very low & Low & Socio-Economic & Socio-Economic & NA & 2017-09-22 & 2017 & 2 & ECO
Frame & v.low & Regional & <500 & 1.0255497 & 0.6327204 & -1.4131940 & 0.5015415 & -0.9523600 & 0.0 & -1.0024021 & 0.3314703 & Recipient & Domestic & Domestic & Domestic & Domestic|ECO & Positive\\
\addlinespace
Romania & https://www.euractiv.ro/fonduri-ue-politica-de-coeziune/corina-cretu-marinescu-spitale-regionale-13139 & 528 & EurActiv | Știri, politici europene \& Actori UE online & Private/Non-Public & Online only & National & very high = CP is most important issue + CP is mentioned in title/headline & Mismanagement & Negative & EU & No myth & Bureaucracy and/or delays & Negative & EU & No myth & Mismanagement & Negative & Subnational & No myth & Romania & corina crețu: românia nu a trimis niciun document privind spitalele regionale @ euractivromania & 2019-01-10 & politica regională & \#politicadecoeziunecorina crețu: românia nu a trimis niciun document privind spitalele regionale comisarul european pentru politică regională a declarat că guvernul româniei nu a depus până acum nici un proiect la comisia europeană pentru cele 3 spitale și nu a finalizat vreun studiu de fezabilitate. corina crețu a confirmat că autoritățile locale pot construi spitalele regionale din iași, craiova și cluj, dar guvernul insistă să le facă ministerul sănătății. deși cele trei spitale au asigurată o finanțare de 150 de milioane de euro din fonduri europene, guvernul a anunțat că vrea să le facă prin parteneriat public-privat. spitalele, incluse încă din 2012 în programele de guvernare ale psd, nu au avansat nici măcar pe hârtie, deși ue a alocat cele 150 de milioane de euro încă de la începutul cadrului financiar 2014-2020. cea mai recentă propunere a guvernului vizează construcția acestora în două etape: prima etapă, finanţată din actualul cadru financiar, va avea ca obiective întocmirea proiectelor tehnice şi a celorlalte documente, în timp ce a doua etapă, finanţată din programul operaţional regional 2021-2027, va avea ca obiective executarea lucrărilor de construire şi operaţionalizarea celor trei spitale regionale de urgenţă. guvernul susține că amânarea este datorată faptului că, potrivit studiilor realizate de experţii băncii europene de investiții, costurile totale pentru construcţia şi dotarea celor trei spitale sunt de aproximativ 400 de milioane de euro pentru fiecare în parte și banii de la ue ar fi insuficienți. exasperate de amânări, autoritățile locale ar vrea să preia ele construcția. consiliul județean cluj a solicitat oficial preluarea construcției spitalului regional de urgență, dar ministrul sănătății a spus că începerea lucrărilor după 2021 are drept scop obținerea mai multor bani de la ue. cele 3 spitale regionale "pot fi finanțate din fonduri europene în perioada curentă de programare", spune comisarul corina crețu într-un răspuns la o interpelare pe această temă formulată de europarlamentarul marian-jean marinescu (pnl, grupul popularilor). "guvernul român trebuie să ia deciziile necesare pentru a evita întârzierile suplimentare care ar submina posibilitatea lansării cu succes a celor trei proiecte și, ca urmare, utilizarea fondurilor uniunii europene disponibile în acest scop până la sfârșitul perioadei actuale de finanțare, în 2023", susține comisarul pentru dezvoltare regională. ea adaugă că primul pas ar fi finalizarea studiilor de fezabilitate pentru cele trei spitale și aprobarea acestora de către autoritățile române înainte ca o cerere de "proiect fazat" să fie adresată comisiei. "comisia este pregătită pentru discuții tehnice privind modul de adaptare a bugetului disponibil la estimările costurilor, luând în considerare un scenariu realist de timp", a mai spus corina crețu. eurodeputatul marian jean marinescu atrage atenția că amânarea pentru următorul exercițiu financiar ar costa românia mai mult și nu mai puțin cum susține guvernul acum. "în calitate de coordonator al grupului ppe din parlamentul eurpean pentru elaborarea cadrului financiar multianual 2021-2027, le atrag atenția premierului viorica dăncilă, precum și miniștrilor rovana plumb și sorina pintea că după 2021 românia va beneficia de mai puțini bani de la ue. acum românia poate beneficia de până la 85\% cofinanțare din partea ue, iar din anul 2021 doar de 70\%", precizează marinescu. de aceea, eurodeputatul cere guvernului, și miniștrilor responsabili să solicite comisiei europene modificarea programului operațional regional si desemnarea ca beneficiari consiliile județene din cluj, dolj și iași. "cele trei consilii județene sunt pregătite, vor să depună proiectele acum, și nu să le amâne", mai spune marinescu. corina crețu a explicat că este posibilă modificarea beneficiarilor, dar că guvernul tebuie să facă această solicitare. "în prezent, beneficiarul desemnat al proiectului este ministerul sănătății. totuși, programul operațional regional, la fel ca oricare al program operațional, poate fi modificat în cursul unei perioade de programare la cererea autorităților naționale. nu este de competența comisiei să decidă asupra beneficiarilor individuali care fac obiectul programului; acest aspect intră în responsabilitatea guvernului româniei", arată crețu. ea menționează, în răspunsul la interpelarea adresată de eurodeputatul pnl că, în privința propunerii de a desemna consiliile locale din cele 3 județe ca beneficiari ai proiectelor, este oricum necesară o implicare serioasă în dezvoltarea spitalelor regionale atât la nivel național, cât și la nivel local. "recomandarea mea este ca toți actorii locali să se angajeze într-un dialog tehnic cu ministerul sănătății pentru elaborarea planurilor locale de dezvoltare în domeniul sănătății", arată comisarul european. de cealaltă parte, ministrul sănătății a acuzat chiar lipsa de implicare a comunităților locale pentru unele întârzieri în derularea proiectelor. "nu s-a întârziat pentru că n-a vrut guvernul sau n-a vrut ministerul sănătății, ci pentru că există niște proceduri pe care acești consultați le-au respectat și pe mine mă mâhnește că anumite comunități locale care nu au fost atât de implicate spun că ministerul sănătății și guvernul nu vor", spunea sorina pintea în decembrie 2018. în schimb, șeful consiliului județean cluj acuză guvernul că nu înțelege prioritatea acestor spitale. "am dat ministerului sănătății un teren de 14 hectare, cu titlu gratuit, să facă spitalul, iar acum nu mai vor să îl facă din diverse motive, cu toate că comisarul corina crețu a spus explicit că există bani, ei doar trebuie să fie trași. să stau să mă uit cum ei nu depun pentru finanțare, când avem banii care ne așteaptă, este inadmisibil", declara alin tișe anul trecut pentru hotnews. e greu să nu-l crezi pe liderul liberal al cj cluj, dat fiind că psd a inclus în programele de guvernare încă din 2012 a nu mai puțin de 8 (opt) spitale regionale și a unuia republican, dar nici măcar documentele nu sunt gata aproape șapte ani mai târziu. & 925 & very high & High & Governance & Governance & Governance & 2019-01-10 & 2019 & 3 & POL
Frame & high-very high & National & 500-1000 & 1.0255497 & 0.6327204 & -1.4131940 & 0.5015415 & -0.9523600 & 0.0 & -1.0024021 & 0.3314703 & Recipient & European & European & European & European|POL & Negative\\
Romania & http://www.euractiv.ro/fonduri-ue/rata-de-absorbtie-a-fondurilor-ue-va-depasi-90-pentru-perioada-2007-2013-7046 & 524 & EurActiv | Știri, politici europene \& Actori UE online & Private/Non-Public & Online only & National & medium = CP is important part of story & Mismanagement & Balanced & EU & No myth & NA & NA & NA & NA & NA & NA & NA & NA & Romania & rata de absorbție a fondurilor ue va depăși 90\% pentru perioada 2007-2013 @ euractivromania & 2017-03-15 & politica regională & comisarul european pentru politică regională corina crețu a declarat miercuri că în românia rata de absorbție a fondurilor alocate pentru exercițiul financiar anterior va depăși 90\%. "am făcut toate eforturile posibile pentru a mări rata de absorbție. când am devenit comisar european, românia avea 60\% rata de absorbție. românia nu a mai pierdut niciun euro pe perioada 2015-2016 din alocațiile pe care le-a avut", a spus corina crețu într-o declarație în fața comisiilor reunite de afaceri europene din parlamentul româniei. "vreau să vă informez că până la 31 martie mai primim facturi pentru perioada 2007-2013, după care comisia are la dispoziție cinci luni să le analizeze. într-adevăr, estimările noastre sunt că rata de absorbție va depăși 90\%". potrivit comisarului european, românia a pierdut circa 2 miliarde de euo din fondurile alocate în principal datorită întârzierilor din perioada 2007-2011. "am făcut toate eforturile de a faza proiectele care nu au putut fi terminate", a mai spus corina crețu, citată de agerpres. datele publicate de ministerul de finanțe arată că românia a primit fonduri europene în valoare de 40,87 miliarde de euro în cei 10 de ani de când a devenit membru al uniunii europene. în schimb, contribuția româniei la bugetele ue s-a ridicat la 13,78 miliarde de euro până la sfârșitul lui 2016. ca urmare, suma netă de care a beneficiat românia a fost de peste 27 de miliarde de euro în perioada 2007-2016. în exercițiul financiar 2007-2013, românia a încasat peste 37 miliarde de euro de la ue. dintre acestea, 2,7 miliarde de euro au fost fonduri pre-aderare, restul fiind fonduri post-aderare. & 276 & medium & Medium & Governance & NA & NA & 2017-03-15 & 2017 & 2 & POL
Frame & low-medium & National & <500 & 1.0255497 & 0.6327204 & -1.4131940 & 0.5015415 & -0.9523600 & 0.0 & -1.0024021 & 0.3314703 & Recipient & European & European & European & European|POL & Neutral\\
Romania & http://ziarulunirea.ro/conferinta-de-inchidere-a-proiectului-stagii-de-practica-si-consiliere-pentru-studenti-vineri-la-universitatea-din-alba-iulia-350331/ & 509 & Ziarul Unirea & Private/Non-Public & Online and Offline & Regional/Local & very low = CP mentioned once & Jobs & Factual & Subnational & No myth & NA & NA & NA & NA & NA & NA & NA & NA & Romania & conferința de închidere a proiectului "stagii de practică și consiliere pentru studenți", vineri, la universitatea din alba iulia & 2015-11-19 & fondul social european & universitatea "1 decembrie 1918" din alba iulia în parteneriat cu fundația progpers - alba iulia, vă invită la conferinţa de finalizare a proiectului "stagii de practică și consiliere pentru studenți", adresat studenţilor universităţii albaiuliene. managerul proiectului este domnul prof.univ.dr. daniel breaz, activităţile din cadrul proiectului incluzându-se în sfera mai largă de preocupări care vizează îmbunătăţirea activităţilor de instruire practică în toate domeniile. proiectul "stagii de practică și consiliere pentru studenți" a fost cofinanţat din fondul social european prin programul operaţional sectorial dezvoltarea resurselor umane 2007-2013, contract nr. posdru/189/2.1/g/156113 şi a avut o durată de implementare de 5 luni. obiectivul-cadru al proiectului l-a reprezentat creșterea calitativă a stagiilor de practică ale studenților și facilitarea inserţiei absolvenţilor de studii universitare pe piaţa muncii, contribuind, astfel, la creşterea gradului de ocupare şi la reducerea şomajului. proiectul vizează corelarea cunoştinţelor teoretice acumulate în timpul studiilor superioare cu cerinţele pieţei muncii și îmbunătăţirea competenţelor studenţilor în sprijinul tranziţiei de la şcoală la viaţa activă. conferinţa de închidere a proiectului va avea loc vineri, 20 noiembrie 2015, ora 9.30, în amfiteatrul a9 al universităţii "1 decembrie 1918" din alba iulia. (v.n.) & 197 & very low & Low & Socio-Economic & NA & NA & 2015-11-19 & 2015 & 1 & ECO
Frame & v.low & Regional & <500 & 1.0255497 & 0.6327204 & -1.4131940 & 0.5015415 & -0.9523600 & 0.0 & -1.0024021 & 0.3314703 & Recipient & Domestic & Domestic & Domestic & Domestic|ECO & Neutral\\
Romania & http://jurnalgiurgiuvean.ro/eveniment-de-informare-privind-programul-de-formare-antreprenoriala-si-selectia-de-grup-tinta-si-planuri-de-afaceri-in-cadrul-proiectului-visul-tau-afacerea-de-maine-id10/ & 499 & jurnalgiurgiuvean.ro & Private/Non-Public & Online only & Regional/Local & medium = CP is important part of story & Jobs & Factual & Subnational & No myth & NA & NA & NA & NA & NA & NA & NA & NA & Romania & eveniment de informare privind programul de formare antreprenoriala si selectia de grup tinta si planuri de afaceri in cadrul proiectului "visul tau, afacerea de maine!" - id105949 & 2018-07-07 & fondul social european & universitatea "valahia" din targoviste implemeteaza proiectul cu titlul "visul tau, afacerea de maine!", proiect finantat din fondul social european prin programul operationa capital uman 2014-2020. proiectul se implementeaza in perioada 11 ianuarie 2018 - 10 ianuarie 2021. obiectivul general al proiectului este cresterea ocuparii in regiunea sud muntenia prin promovarea activitatilor independente, a antreprenoriatului si infiintarea de intreprinderi cu profil nonagrocol in zona urbana. proiect cofinantat din fondul social european prin programul operational capital uman 20214-2020 in data de 11.07.2018, universitatea "valahia" din targoviste, organizeaza evenimentul de informare privind programul de formare antreprenoriala si selectia de grup tinta si planuri de afaceri in cadrul proiectului. evanimentul va avea loc in dolce cafe, începând cu ora 11,00. adresa: municipiul targoviste, b-dul regele carol 1 nr. 2, judetul dambovita, romania & 133 & medium & Medium & Socio-Economic & NA & NA & 2018-07-07 & 2018 & 3 & ECO
Frame & low-medium & Regional & <500 & 1.0255497 & 0.6327204 & -1.4131940 & 0.5015415 & -0.9523600 & 0.0 & -1.0024021 & 0.3314703 & Recipient & Domestic & Domestic & Domestic & Domestic|ECO & Neutral\\
Romania & http://www.aradon.ro/platiti-sa-faca-angajari/2157671 & 539 & aradon.ro & Private/Non-Public & Online only & Regional/Local & very low = CP mentioned once & Social justice & Positive & Subnational & No myth & NA & NA & NA & NA & NA & NA & NA & NA & Romania & plătiți să facă angajări & 2018-12-05 & fondul social european & o nouă măsură activă de integrare a șomerilor pe piața muncii a intrat în vigoare: 2.250 lei pentru angajatorii care încadrează în muncă șomeri. agenția națională pentru ocuparea forței de muncă a anunțat o nouă serie de măsuri active ce vizează integrarea pe piața muncii a șomerilor și persoanelor în căutarea unui loc de muncă. printre măsurile destinate creșterii șanselor de ocupare a șomerilor se numără acordarea de subvenții în vederea stimulării angajatorilor de a încadra pe piața muncii persoane aparţinând unor categorii dezavantajate sau cu acces mai dificil pe piaţa muncii. "angajatorii care încadrează în muncă, pe perioadă nedeterminată, şomeri peste 45 ani sau părinți unici susţinători ai familiilor monoparentale, şomeri de lungă durată sau tineri neets primesc lunar, pe o perioadă de 12 luni, pentru fiecare persoană angajată din aceste categorii, o sumă în cuantum de 2.250 lei, cu obligaţia menţinerii raporturilor de muncă sau de serviciu cel puţin 18 luni. de aceeași subvenție beneficiază și angajatorii care încadrează în muncă, potrivit legii, şomeri care, în termen de 5 ani de la data angajării îndeplinesc, conform legii, condiţiile pentru a solicita pensia anticipată parţială sau de acordare a pensiei pentru limită de vârstă. de facilităţile menționate anterior (subvenție în cuantum de 2.250 lei) beneficiază şi angajatorii care, în raport cu numărul de angajaţi, şi-au îndeplinit obligaţia, potrivit legii, de a încadra în muncă persoane cu handicap, precum şi angajatorii care nu au această obligaţie legală, dacă încadrează în muncă pe durată nedeterminată persoane cu handicap şi le menţin raporturile de muncă sau de serviciu cel puţin 18 luni", au transmis reprezentații agenției de ocupare. pentru absolvenți în situația încadrării în muncă pe durată nedeterminată a absolvenților, anofm acordă angajatorilor subvenții în cuantum de 2.250 de lei/lună pe o perioadă de 12 luni, pentru fiecare absolvent angajat. în cazul în care sunt încadrați absolvenţi din rândul persoanelor cu handicap subvențiile se acordă pe o perioadă de 18 luni. casetă: cu finanțare europeană toate aceste subvenții sunt asigurate de anofm prin implementarea de proiecte finanțate din fondul social european prin programul operațional capital uman 2014-2020, urmărindu-se astfel creșterea oportunităților de încadrare în muncă a persoanelor aflate în dificultate din punct de vedere al ocupării. facilităţile se acordă la solicitarea angajatorilor pentru persoanele din categoriile menţionate și înregistrate în evidenţele agenţiilor pentru ocuparea forţei de muncă. & 393 & very low & Low & Socio-Economic & NA & NA & 2018-12-05 & 2018 & 3 & ECO
Frame & v.low & Regional & <500 & 1.0255497 & 0.6327204 & -1.4131940 & 0.5015415 & -0.9523600 & 0.0 & -1.0024021 & 0.3314703 & Recipient & Domestic & Domestic & Domestic & Domestic|ECO & Positive\\
\addlinespace
Romania & http://www.mediafax.ro/economic/comisia-europeana-a-aprobat-fazarea-unor-proiecte-majore-de-infrastructura-pentru-romania-16016162 & 583 & Mediafax.ro & Private/Non-Public & Online only & National & high = CP is most important issue in story (can also cover other issues) & Infrastructure & Positive & EU + National & No myth & NA & NA & NA & NA & NA & NA & NA & NA & Romania & comisia europeană a aprobat fazarea unor proiecte majore de infrastructură pentru românia & 2016-12-05 & fondul de coeziune & comisarul european pentru politică regională corina creţu a aprobat fazarea a şapte proiecte majore pentru românia. cele şapte proiecte sunt cofinanţate prin fondul european de dezvoltare regională (fedr) sau prin fondul de coeziune, se anunţă într-un comunicat transmis luni de reprezentanţa comisiei europene în românia. "prin aceste decizii de eşalonare a unor proiecte majore, românia beneficiază de o şansă în plus în utilizarea fondurilor europene ce i-au fost alocate - 147 de milioane de euro pentru infrastructura de transport şi aproximativ 355 de milioane de euro pentru proiecte de infrastructură de apă şi apă uzată. sper ca aceste proiecte să fie implementate cu succes în termenele prevăzute, iar la finalizarea lor să constatăm că investiţiile cofinanţate cu bani europeni au dus la îmbunătăţirea condiţiilor de viaţă a românilor", a declarat comisarul pentru politică regională corina creţu. proiectele de infrastructură de transport sunt: - drumul naţional piteşti-câmpulung-braşov: modernizarea dn 73; - autostrada sebeş-turda: noua autostradă de 70 km va face legătura între autostrăzile a1 (orăştie-sibiu) şi a3 (gilău-câmpia turzii); - drumul naţional dn 76: reabilitarea drumului naţional între deva şi oradea. alte patru proiecte aprobate vizează extinderea şi reabilitarea sistemelor de alimentare cu apă şi canalizare în judeţele braşov, hunedoara, olt şi tulcea. o bună parte din fondurile europene alocate în cadrul vechiului buget al ue (2007-2013) pentru infrastructură riscă să fie pierdute fiindcă lucrările nu vor fi terminate până la sfârşitul acestui an. fazarea acestor proiecte, cu aprobarea primită de la comisia europeană, este o metodă prin care autorităţile susţin că încearcă să nu piardă fondurile europene, care mai pot fi cheltuite până la 31 decembrie. conținutul website-ului www.mediafax.ro este destinat exclusiv informării și uzului dumneavoastră personal. este interzisă republicarea conținutului acestui site în lipsa unui acord din partea mediafax. pentru a obține acest acord, vă rugăm să ne contactați la adresa vanzari@mediafax.ro. & 312 & high & High & Socio-Economic & NA & NA & 2016-12-05 & 2016 & 2 & ECO
Frame & high-very high & National & <500 & 1.0255497 & 0.6327204 & -1.4131940 & 0.5015415 & -0.9523600 & 0.0 & -1.0024021 & 0.3314703 & Recipient & Domestic & European & Mixed & Domestic|ECO & Positive\\
Romania & https://www.euractiv.ro/espresso/oug-pentru-linistea-politicienilor-infractori-12180 & 548 & EurActiv | Știri, politici europene \& Actori UE online & Private/Non-Public & Online only & National & very low = CP mentioned once & Poor communication of funding/rules & Negative & National & No myth & NA & NA & NA & NA & NA & NA & NA & NA & Romania & oug pentru liniștea politicienilor infractori @ euractivromania & 2018-10-16 & politica de coeziune & chiar în ziua plecării în turcia, premierul a dat o oug privind legile justiției abia adoptate. textul ordonanței n-a apărut încă în monitorul oficial. viorica dăncilă ne-a anunțat că oug, al cărei text a fost finalizat peste noapte, a fost dată pentru a amorniza prevederile legilor justiției abia intrate în vigoare, după care a plecat în turcia "să ia lumină de la însuși sultanul erdogan", lăsându-l pe tudorel toader să explice conținutul oug. ministrul justiției n-a fost în stare să dea o explicație privind necesitatea acestei ordonanțe, prin care și-a acordat un nou atribut: acela de a sesiza inspecția judiciară pentru anchetarea procurorilor. e doar una dintre prevederile cuprinse în oug care a provocat îngrijorarea procurorilor încă dinainte de a fi adoptată. deocamdată, textul ordonanței nu e public și nici nu a fost publicat în monitorul oficial. una dintre "victimele" ordonanței va fi - potrivit lui toader - laura-codruța kovesi. dacă are grad de tribunal, fosta șefă a dna nu poate să mai rămână la parchetul general. nici măcar social-democrații nu sunt prea mulțumiți de oug, deoarece, ca de obicei, nu au fost consultați. insensibil la opinia csm, care a respins-o în mod clar pe adina florea, tudorel toader a trimis propunerea pentru șefia dna la președintele klaus iohannis. pe fondul disputelor dintre comisarul corina crețu și guvern privind lipsa de interes a bucureștiului pentru fonduri europene, merită amintit că politica de coeziune este una dintre cele mai eficiente unelte ale ue pentru o mai bună integrare europeană. instrumentele acesteia nu sunt însă suficient cunoscute și explorate de majoritatea cetățenilor. \#politicadecoeziune recomandarea de lectură a redacției: the new yorker - as america's élite abandons a reckless saudi prince, will trump join them? & 288 & very low & Low & Governance & NA & NA & 2018-10-16 & 2018 & 3 & POL
Frame & v.low & National & <500 & 1.0255497 & 0.6327204 & -1.4131940 & 0.5015415 & -0.9523600 & 0.0 & -1.0024021 & 0.3314703 & Recipient & Domestic & Domestic & Domestic & Domestic|POL & Negative\\
Romania & http://www.evz.ro/curtea-de-conturi-a-gasit-prapad-la-autoritatea-nationala-pentru.html & 532 & evz.ro & Private/Non-Public & Online and Offline & National & medium = CP is important part of story & Mismanagement & Negative & National & No myth & NA & NA & NA & NA & NA & NA & NA & NA & Romania & prăpăd la autoritatea naţională pentru turism & 2017-04-09 & fondul european de dezvoltare regională & 00:00 început de săptămână cu vreme în general frumoasă, dar de miercuri se întoarce frigul | prognoza meteo turismul românesc nu are o strategie, iar investiţiile nu sunt monitorizate. autoritatea nici măcar nu știe care este patrimoniul turistic al țării, iar fondurile sunt cheltuite aiurea veniturile scăzute din turism înregistrate de românia, comparativ cu state cu un potenţial mult mai scăzut decât al nostru, reflectă incapacitatea factorilor decidenţi din sector de a crea politici durabile care să genereze o dezvoltare continuă, este concluzia unui audit realizat de curtea de conturi la autoritatea naţională pentru turism în perioada 2013-2015. patrimoniul turistic, o necunoscută cum la nivelul ant nu a existat o evidenţă a poliţelor de asigurare încheiate de agenţiile de turism, în 2016 mii de turişti au fost în pericol de a-şi pierde banii achitaţi ca urmare a intrării în insolvenţă a două agenţii importante. ant nu s-a învrednicit nici să supună aprobării guvernului o strategie de dezvoltare a turismului pe termen mediu şi lung. mai mult, ant nu cunoaşte patrimoniul turistic al româniei, deşi are atribuţii în ceea ce privește atestarea, evidenţierea, monitorizarea şi valorificarea acestuia. curtea de conturi a constatat un grad redus de absorbţie a fondurilor europene, de circa 39 la sută, şi un grad ridicat de reziliere a unor contracte de finanţare, adică 16 contracte în valoare de 65 de milioane de lei. tarife supraevaluate legat de participarea la expoziţiile de turism, s-a stabilit că ant a acceptat tarife supraevaluate de la operatorii care amenajează standurile, aceştia fundamentându- şi ofertele în funcţie de bugetul alocat de autoritate. controlorii au descoperit pe stoc materiale de promovare mai vechi de trei ani, hărţi, broşuri şi albume, care nu mai erau de actualitate sau aveau greşeli de editare în valoare de peste 4,6 milioane de lei. aprobate prin hotărâre de guvern în 2006 şi finanţate cu peste 20 de milioane de lei pentru realizarea studiilor de fezabilitate, şase proiecte de investiţii erau în acelaşi stadiu, adică nu se făcuse nimic, după 10 ani. la data finalizării auditului, 118 din 186 de proiecte aprobate prin hg în 2010 nu fuseseră începute, autorităţile locale cărora le fuseseră repartizate neluând nicio măsură pentru realizarea lor. pe de altă parte, la investiţiile în turism, unităţile administrativteritoriale au efectuat plăţi suplimentare de 44 de milioane de lei. asta s-a putut întâmpla şi pentru că ant nuşi face datoria de a monitoriza anual lucrările de investiţii aflate în curs de execuţie. din 109 centre de promovare turistică finalizate în perioada 2012-2015 doar pentru 32 consiliile locale sau cele judeţene solicitaseră acreditatarea. oricum, ant nu avea un sistem de evaluare a eficienţei acestora. "amenajare plaja sulina", peste 7 milioane de lei cheltuiţi degeaba în perioada 2009-2015, curtea de conturi a efectuat verificări privind eficienţa modului în care au fost derulate proiectele de investiţii în turism finanţate de la buget. la obiectivul "amenajare plaja sulina", de exemplu, s-a constatat că s-a făcut recepţia unor lucrări nerealizate, în valoare de 56.692 lei. această amenajare care a costat peste şapte milioane de lei s-a degradat văzând cu ochii după ce i-a fost închiriată unui agent economic privat pentru o chirie de 25.000 de lei pe an. proiectul s-a dovedit ineficient, neputând fi vorba de venituri obţinute suplimentar de comunitatea locală. de la data dării în administrare către uato sulina, obiectivul nu a funcţionat şi nu fost exploatat conform destinaţiei sale. scopul investiţiei de dezvoltare economico-socială a oraşului sulina şi zonelor limitrofe a rămas doar pe hârtie. foto: foto: ziuadetulcea.ro primul venit, primul servit referitor la înscrierea şi aprobarea participării unor firme în calitate de co-expozanţi la târgurile din străinătate, auditul a constatat că modul în care a fost făcută selecţia participanţilor este netransparent, neasigurând o selecţie eficientă. ordinele preşedintelui ant prin care au fost aprobate selecţia şi lista manifestărilor expoziţionale nu au fost publicate în monitorul oficial, singurul criteriu de participare fiind ordinea înregistrării cererilor de înscriere. astfel, au existat operatori, ca polar managament srl şi karpaten turism srl, care au participat în mod constant la expoziţii. autoritatea a acceptat ca operatorii să facă dovada că nu au datorii şi nu sunt în executare silită în baza unor declaraţii pe proprie răspundere, aşa că cinci firme care au solicitat să meargă la eibtm barcelona 2015 înregistrau debite la buget. la târgul cmt germania 2015 a participat o societate care avea activitatea suspendată din 2013. 884.315 lei, gaură la buget dintr-o lovitură în ceea ce priveşte contractarea şi decontarea serviciilor de promovare şi publicitate, auditul a constatat achiziţia acestora la preţuri supraevaluate. realizarea de acţiuni de promovare şi publicitate cu rol de creştere a notorietăţii destinaţiilor turistice putea fi realizată din fondul european de dezvoltare regională, dar, din cauza gradului redus de realizare a proiectelor, au fost refuzate la plată integral, astfel că toate cheltuielile au rămas decontate de la bugetul de stat, rezultând o gaură de 884.315 lei. capital.ro test opel insignia: echipamente premium pentru o nouă... doctorulzilei.ro ce medicamente pot duce la orbire. dr. monica pop:... evzmonden.ro eu de căsătorit nu mă căsătoresc fiindcă nu văd sensul tag-uri: turism, ant, romania, strategie, investitii, bani nota: pentru a instaura un cadru civilizat de discuţii, de eliminare a "postacilor" de partid sau a celor plătiţi ca să blocheze un articol civilizat, am adoptat următoarele soluţii, în privinţa comentariilor: 1) moderarea comentariilor lăsate în formularul de la finalul articolelor o dată la o oră - în acest caz, comentariile nu vor apărea instant. 2) postarea instant a comentariilor lăsate prin intermediul contului de facebook - în acest caz comentariile vor fi postate imediat. puteţi să vă faceţi cont de facebook aici. orice critică este acceptată pe site-ul evz.ro, cu condiţia păstrării unui limbaj civilizat, toate aceste măsuri fiind şi în sprijinul celor interesaţi să-şi expună punctele de vedere fără a mai fi hărţuiţi. sperăm că veţi înţelege adevărata valoare a demersului evz.ro şi vă veţi asuma responsabilitatea alături de noi. în lipsa unui acord scris din partea evenimentul zilei, puteţi prelua maxim 500 de caractere din acest articol dacă precizaţi sursa şi dacă inseraţi vizibil link-ul articolului curtea de conturi a găsit prăpăd la autoritatea naţională pentru turism . alte articole din categoria: justiţie duminică, 09 aprilie 2017 autor: maria anghel unul dintre violatorii de la vaslui a făcut cerere de revizuire a dosarului. ce au decis magistrații duminică, 09 aprilie 2017 autor: dan andronic o noapte cu kovesi și coldea. alegerile prezidențiale din 2009 și legătura cu operațiunea "noi suntem statul!" sâmbătă, 08 aprilie 2017 autor: maria anghel un controversat om de afaceri, din cercul lui mazăre, condamnat la 13 ani de închisoare evz.ro scene halucinante în penitenciar. gardienii s-au înarmat... unul dintre violatorii de la vaslui a făcut cerere de... cum mi-am petrecut o noapte cu laura codruța kovesi și... libertatea.ro decizie șocantă a lui vladimir putin! rusia ripostează... un consilier pnl a fost găsit mort în râul olt! este... băsescu, umilit de igor dodon. ce a pățit fostul președinte rtv.net concediu de odihnă de 40 zile!!! decizia a fost validată... minivacanta de 5 zile pentru toti romanii dupa paste... horoscop mariana cojocaru: veşti în plan sentimental,... wowbiz.ro nadine e de nerecunoscut! adio, bomba sexy de altadata!... o mai tii minte pe iulia fratila? e incredibil cum arata... fiul ilenei ciuculete, acuzat ca se distreaza, la mai... b1.ro valentina pelinel, despre noul copil al lui borcea: mă... la 47 de ani, dana săvuică a rămas goală puşcă, după ce a... un bărbat obsedat de extratereștri a dispărut. ce a lăsat... cancan.ro a nascut!!! avem prima imagine cu bebelusul, mamica si... e insarcinata!!! nu au anuntat oficial, dar avem dovada... elena merisoreanu a primit mesajul in miez de noapte si... ziare.com casa alba e o afacere de familie. de ce asta nu e... ziua in care america s-a trezit cu trump si s-a culcat cu... astazi sunt floriile - ce obiceiuri si traditii avem unica.ro sunt cele mai folosite medicamente pentru dureri sau... cum arată și cu ce se ocupă fiicele lui vladimir putin o mai ții minte pe selena de la "candy"? cum arată după... fanatik.ro o mai ții minte pe analia selis? ce destin trist! ce s-a... antonia i-a născut al doilea copil, dar el calcă pe bec!... s-a aflat adevărul! câți bani și ce job are, de fapt,... dcnews.ro update: protest la psd. scandal cu jandarmii: acesta nu... celebrarea săptămânii sfinte în lume. paştele este... ministrul cazanciuc, protagonistul unei întâmplări... & 1435 & medium & Medium & Governance & NA & NA & 2017-04-09 & 2017 & 2 & POL
Frame & low-medium & National & +1000 & 1.0255497 & 0.6327204 & -1.4131940 & 0.5015415 & -0.9523600 & 0.0 & -1.0024021 & 0.3314703 & Recipient & Domestic & Domestic & Domestic & Domestic|POL & Negative\\
Romania & http://ziarulunirea.ro/783-de-localitati-vor-beneficia-de-acoperire-cu-internet-in-banda-larga-pana-in-noiembrie-zonele-din-alba-vizate-380390/ & 584 & Ziarul Unirea & Private/Non-Public & Online and Offline & Regional/Local & medium = CP is important part of story & Infrastructure & Positive & EU + National & No myth & NA & NA & NA & NA & NA & NA & NA & NA & Romania & 783 de localități vor beneficia de acoperire cu internet în bandă largă, până în noiembrie. zonele din alba vizate & 2016-07-03 & fondul european de dezvoltare regională & proiectul ro-net privind construirea unei rețele naționale de internet în bandă largă în zonele dezavantajate, prin folosirea fondurilor structurale, a fost finalizat în peste 200 de localități, iar în alte 220 se află în fază avansată de implementare, se arată într-un comunicat de presă al ministerului comunicațiilor pentru societatea informațională (mcsi), remis vineri. potrivit țintei asumate prin proiect, până în noiembrie 2016 un număr 783 de localități din zone greu accesibile vor beneficia de acoperire cu internet în bandă largă. în județul alba, sunt vizate 30 de localități:albac, fața, dealu lămășoi, cionești, bărăști, rimetea, colțești , ocoale, gârda seacă, ghețari, munună, tonea, pleși, horea, tătârlaua, costești, poiana vadului, turdaș, dealu bajului, bubești, cobleș , fața cristesei, băcăinți, sohodol, mogoș, ciugudu de jos, dumbrava, hopârta, mescreac, pleși și vale în jos. ministerul comunicațiilor și pentru societatea informațională (mcsi) a prezentat vineri, 1 iulie, amplasamentele finalizate și beneficiile conectării la internet locuitorilor din localitățile tomești și mascurei, din județul vaslui. "elevii și profesorii vor avea acces mai ușor la informație. producătorii locali și meșteșugarii vor putea utiliza internetul pentru afacerea lor și pot intra chiar în sistemul național de licitații publice cu produsele lor. autoritățile locale pot comunica mai rapid cu omologii lor și pot cere susținere pentru dezvoltarea utilităților din zona, iar persoanele în vârstă vor putea ține legătura prin intermediul internetului cu cei dragi aflați departe. astfel, întreaga comunitate beneficiază în urma disponibilității accesului la internet. cu alte cuvinte, chiar dacă unii se vor întreba poate de ce am adus internet în sate unde, de exemplu, nu există o rețea de canalizare bine pusă la punct, răspunsul este că, dacă ai internet și îl folosești inteligent pentru comunitate, poți să dezvolți și utilitățile din zonă', a declarat horațiu anghelescu, secretar de stat în cadrul ministerului, în cadrul vizitei în județul vaslui. ro-net este un proiect major finanțat prin fonduri structurale europene. la finalul lunii mai, comisia europeană a decis modificarea acestui proiect și, astfel, împărțirea sa în două etape. valoarea totală aprobată pentru prima etapă a proiectului este de 18,8 milioane de euro, din care 12,5 milioane reprezintă co-finanțarea prin fondul european de dezvoltare regională. bugetul proiectului pentru cea de-a doua etapă este de 53 milioane de euro. 'ro-net va acoperi 783 din cele 2.268 de localități identificate ca 'zone albe'. acest lucru va contribui la reducerea decalajului digital dintre zonele urbane și zonele rurale aducând internetul în bandă largă mai aproape de 130.000 de gospodării cu 400.000 de locuitori, 8.500 de întreprinderi și 2.800 de instituții publice', informează mcsi. până acum s-a decontat de la ce aproximativ 22\% din valoarea totală, iar stadiul fizic de realizare este de peste 25\%, rezultat care s-a atins în circa două luni. & 463 & medium & Medium & Socio-Economic & NA & NA & 2016-07-03 & 2016 & 2 & ECO
Frame & low-medium & Regional & <500 & 1.0255497 & 0.6327204 & -1.4131940 & 0.5015415 & -0.9523600 & 0.0 & -1.0024021 & 0.3314703 & Recipient & Domestic & European & Mixed & Domestic|ECO & Positive\\
Romania & https://evz.ro/cretu-mesaj-guvern.html & 527 & evz.ro & Private/Non-Public & Online and Offline & National & very high = CP is most important issue + CP is mentioned in title/headline & Mismanagement & Negative & EU & No myth & NA & NA & NA & NA & NA & NA & NA & NA & Romania & corina crețu, mesaj dur pentru guvernul româniei & 2018-10-08 & politica regională & 00:00 divergență electorală. hăcuit de nevastă de față cu cei patru copii comisarul european pentru politică regională, corina crețu, a afirmat, ieri, că nu mai acceptă insulte din partea guvernului româniei, în contextul în care a avertizat de nenumărate ori autoritățile de la bucureşti că românia pierde bani din lipsa proiectelor, însă a fost apostrofată că spune lucrurilor pe nume. "după cum ştiți, de când am devenit comisar european pentru politică regională, am încercat să ajut cu rata de absorbție pentru regiunile mai puțin dezvoltate. în cazul româniei, am făcut eforturi extraordinare, în sensul că am fazat foarte multe proiecte în valoare de trei miliarde de euro, practic am degrevat bugetul național de trei miliarde de euro, bani care ar fi fost pierduți", a spus oficialul european, într- o conferință de presă susținută la bruxelles. caz revoltător! bebeluș ținut în pungă 10 ore. doctorii au crezut că este mortscandal în social media. ce a ascuns google+ până acum. sute de mii de utilizatori au fost afectați ea a arătat că în românia s-a ajuns la o rată de absorbție, pentru perioada 2007 - 2013, de peste 90\%, în condițiile în care în 2014 s-a plecat de la sub 50\%. pagina 1 din 1 tag-uri: corina cretu, mesaj, dur, guvern nota: pentru a instaura un cadru civilizat de discuţii, de eliminare a "postacilor" de partid sau a celor plătiţi ca să blocheze un articol civilizat, am adoptat următoarele soluţii, în privinţa comentariilor: 1) moderarea comentariilor lăsate în formularul de la finalul articolelor o dată la o oră - în acest caz, comentariile nu vor apărea instant. 2) postarea instant a comentariilor lăsate prin intermediul contului de facebook - în acest caz comentariile vor fi postate imediat. puteţi să vă faceţi cont de facebook aici. orice critică este acceptată pe site-ul evz.ro, cu condiţia păstrării unui limbaj civilizat, toate aceste măsuri fiind şi în sprijinul celor interesaţi să-şi expună punctele de vedere fără a mai fi hărţuiţi. sperăm că veţi înţelege adevărata valoare a demersului evz.ro şi vă veţi asuma responsabilitatea alături de noi. în lipsa unui acord scris din partea evenimentul zilei, puteţi prelua maxim 500 de caractere din acest articol dacă precizaţi sursa şi dacă inseraţi vizibil link-ul articolului corina crețu, mesaj dur pentru guvernul româniei. stirile zilei de ce a eșuat referendumul? pentru că psd a avut interesul să eșueze. ... cutremur la înalta curtea de casație și justiție. gestul lui dragnea ... dragnea avea un plan fabulos dacă ieșea bine la referendum. și iohannis ... atac neașteptat de la bruxelles. corina crețu șterge pe jos cu lideri ... animalzoo.ro achizitionarea unui animal de companie doctorulzilei.ro intoleranța la gluten și semnele adeseori neglijate, care... evzmonden.ro cel mai bine ținut secret din lumea artiștilor a fost... alte articole din categoria: politica alte articole din categorie luni, 08 octombrie 2018 autor: adam popescu umilință pentru psd! victor ponta dă de pământ cu partidul lui dragnea marţi, 09 octombrie 2018 autor: razvan gheorghe, cătălina iordache, andreea pâslaru eșecul referendumului cutremură psd și pnl! orban, lovit mai tare decât dragnea luni, 08 octombrie 2018 autor: adam popescu veste șoc despre ion iliescu! lovitură dură pentru psd capital.ro se schimbă harta europei: "independenţa va fi o... adevărata lovitură pentru liviu dragnea abia acum vine:... detalii neștiute! ce pensie are nadia comăneci? corabia de gheata 5,00lei misterul din insula 5,00lei ce se întâmplă cu mine? (pentru fete) 14,00lei poveştile reginei maria a româniei 40,00lei libertatea.ro părintele necula, reacție dură după referendum. atac la... vecinul din costa rica rupe tăcerea. ce spune despre... ion iliescu, lovitură dură pentru dragnea. incredibl ce a... rtv.net din păcate, veşti triste despre maria ciobanu. anunţul... imagini fierbinţi la insula iubirii, prima noapte de amor... noi dezvăluiri incendiare despre elena udrea şi alina... fanatik.ro rezultate referendum! votul s-a incheiat! soc! a murit in urma cu putin timp, invins de boala! ce... udrea, mesaj din inchisoarea din costa rica! ce se... wowbiz.ro cine este, de fapt, bărbatul care le-a ajutat pe elena... adevărul despre războinicul poliţist de la exatlon! cat... motivul pentru care adrian alexandrov nu a luat-o de... b1.ro vești cumplite pentru mirabela dauer: a murit după o... halucinant! cristian pomohaci a comis-o chiar înainte de... din păcate medicii au făcut anunţul trist: e vorba de... cancan.ro doliu in presa din romania: a murit! "condoleanțe... iubitul elenei udrea, mesaj cutremurător cu bebelușul în... bombă! tudor chirilă, filmat în timp ce se săruta cu o... playtech.ro cel mai grav accident rutier din istoria româniei. 48 de... cum rezolvă marina rusă problema piraților somalezi... video: și-a urmărit soția cu o dronă, iar imaginile l-au... unica.ro un mare sportiv, suspectat ca a platit o prostituata... cum arata corpul femeii care si-a scos 6 coaste. a vrut... cantareata a facut transplant de rinichi. nimeni nu stia... dcnews.ro cosette chichirău, fugărită pe stradă: voia să mă bată și... problemă generală la ing. clienții băncii s-au trezit... "să-l găsim pe tom". câine pierdut, mobilizare pentru... stiridiaspora.ro iubitul elenei udrea, detalii incredibile din momentul... ce spune presa străină despre referendum oreste: "dragnea e frustrat erotic" & 866 & very high & High & Governance & NA & NA & 2018-10-08 & 2018 & 3 & POL
Frame & high-very high & National & 500-1000 & 1.0255497 & 0.6327204 & -1.4131940 & 0.5015415 & -0.9523600 & 0.0 & -1.0024021 & 0.3314703 & Recipient & European & European & European & European|POL & Negative\\
\addlinespace
Romania & http://ziarulunirea.ro/foto-eveniment-de-inchidere-a-proiectului-work-i-t-la-domeniile-martinuzzi-vintu-de-jos-rezultatele-obtinute-prezentate-de-expertii-proiectului-351609/ & 540 & Ziarul Unirea & Private/Non-Public & Online and Offline & Regional/Local & very low = CP mentioned once & Social justice & Positive & National & No myth & NA & NA & NA & NA & NA & NA & NA & NA & Romania & foto: eveniment de închidere a proiectului "work i.t.!", la domeniile martinuzzi - vințu de jos. rezultatele obținute, prezentate de experții proiectului & 2015-11-27 & fondul social european & în perioada 1 aprilie 2014 - 30 noiembrie 2015, xerom service s.r.l. din alba iulia a derulat, proiectul "work i.t.!", în parteneriat cu sc industrial euro star srl alba iulia și sc gami clean srl alba iulia. serviciile proiectului s-au adresat unui număr de 834 de persoane: șomeri tineri și de lungă durată, persoane inactive, aflate în căutarea unui loc de muncă, persoane ocupate în agricultura de subzistență, precum și angajați și manageri din mediul rural, din regiunile centru (județele alba, sibiu, mureș) și vest (hunedoara, caraș-severin). rezultatele finale au fost prezentate în cadrul conferinței finale din regiunea centru. evenimentul a avut loc vineri, 27 noiembrie 2015, în comuna vințu de jos (județul alba), la domeniile martinuzzi și a reunit 155 de persoane - reprezentanți ai instituțiilor publice locale și regionale, ai firmelor și ong-urilor, experți ai beneficiarului și partenerilor din proiect, reprezentanți ai mass media, precum și membri ai grupului țintă. participanții au aflat de la experții proiectului rezultatele obținute prin implementarea activităților proiectului, gradul de atingere al indicatorilor, dar și măsurile care asigură sustenabilitatea proiectului. în municipiul sibiu, str. înfrățirii, nr.47, va funcționa un centru de informare și promovare amăsurilor active deocupare. proiectul "work i.t. !" a vizat combaterea șomajului și creșterea ratei de ocupare pe piața muncii din zonele rurale, din cele două regiuni. proiectul este cofinanțat din fondul social european, prin programul operațional sectorial dezvoltarea resurselor umane 2007 - 2013 "investeşte în oameni!". (a.m.) & 243 & very low & Low & Socio-Economic & NA & NA & 2015-11-27 & 2015 & 1 & ECO
Frame & v.low & Regional & <500 & 1.0255497 & 0.6327204 & -1.4131940 & 0.5015415 & -0.9523600 & 0.0 & -1.0024021 & 0.3314703 & Recipient & Domestic & Domestic & Domestic & Domestic|ECO & Positive\\
Romania & http://www.mediafax.ro/economic/lucrarile-din-master-plan-vor-fi-finantate-si-din-imprumuturi-publice-si-acciza-pe-carburanti-13893546?utm\_source=feedburner\&utm\_medium=feed\&utm\_campaign=Feed\%253A\%2Bmediafax\%252FQddx\%2B\%2528Mediafax\_ALL\%2529 & 580 & Mediafax.ro & Private/Non-Public & Online only & National & medium = CP is important part of story & Infrastructure & Factual & National & No myth & NA & NA & NA & NA & NA & NA & NA & NA & Romania & lucrările din master plan vor fi finanţate şi din împrumuturi publice şi acciza pe carburanţi & 2015-02-25 & fondul de coeziune & plafonul de 7 miliarde euro a fost aprobat, de asemenea, în şedinţa de miercuri a executivului. valoarea totală a proiectelor identificate în document se ridică la 45,4 miliarde euro, pentru sectorul de transport rutier, feroviar, aerian, naval şi multimodal, ministrul de resort, ioan rus, explicând că finanţarea disponibilă în prezent acoperă proiectele prioritare. astfel, în strategia de finanţare a master planului, suma indicată ca fiind disponibilă este de 2,7 miliarde de euro din fondul de coeziune, inclusiv cofinanţarea de la bugetul de stat, şi 2,1 miliarde de euro din fondul european de dezvoltare regională, pentru sectorul rutier. "suma totală pe care românia o are la dispoziţie, în perioada de programare 2014-2020, pentru cele patru sectoare de transport (rutier, feroviar, naval, multimodal) este de 6,843 miliarde euro. la acestea se adaugă un plafon suplimentar de 7 miliarde euro pe care guvernul l-a aprobat astăzi, care vor proveni din împrumuturi publice - ifi- şi din acciza pe carburanţi, în vederea asigurării surselor de finanţare necesare implementării proiectelor din sectorul rutier, cu prioritate pentru autostrăzi. astfel, valoarea totală a finanţării pentru sectorul infrastructurii rutiere, în perioada de programare 2014-2020, este de 13,8 miliarde de euro", se arată într-un comunicat al guvernului. ministrul transporturilor, ioan rus, a precizat că procentul care va fi alocat pentru master plan din acciza la carburanţi va fi stabilit de executiv în perioada următoare. "lângă sumele pe care le-am avut, se adaugă, prin hotărârea de astăzi, încă un plafon suplimentar de 7 miliarde de euro, pentru care ministerul finanţelor se va ocupa să găsească cadrul fiscal, adică spaţiu în interiorul actualului regim fiscal în 2014-2020, bani care vor proveni din împrumuturi berd, bei, instituţii financiare, alte resurse financiare şi, de asemenea, din acciza pe carburanţi. parte din această acciză, pe care o vom stabili împreună în guvern, va intra direct în finanţarea master planului general de transport al româniei", a spus rus. întrebat cum va fi acoperită diferenţa de finanţare până la suma totală de 45 miliarde euro, reprezentând valoarea proiectelor identificate, ministrul a spus doar că la acest moment sunt acoperite proiectele prioritizate, dar că există discuţii cu bănci comerciale. "în acest moment, avem pentru tot ce avem prioritizat şi în faza de a începe, avem resursele financiare asigurate, la care am suplimentat prin decizia guvernului de azi încă 7 miliarde cu care noi ne descurcăm. am discutat şi dimineaţă cu alte bănci decât bei, berd sau banca mondială, să spunem bănci comerciale, deci, toată lumea, în momentul în care va exista acest document aprobat de comisie şi vor şti toate instituţiile financiare internaţionale că românia are un document cu priorităţi aprobate şi în guvern, şi în comisiile de specialitate ale parlamentului, şi la comisia europeană, toată lumea va veni alături de noi în a ne sprijini financiar, iar ministerul finanţelor are sarcină să găsească spaţiu fiscal pentru aceşti bani. vom reuşi să rostogolim cu siguranţă sumele pe care le avem astfel încât să ne putem finanţa până în 2030 toate aceste proiecte", a spus rus. planul financiar al master planului general de transport urmează să fie prezentat în şedinţele reunite ale comisiilor de transport din parlament înainte de trimiterea acestuia la comisia europeană, pentru discuţii programate în intervalul 9-13 martie. în sinteză, master plan-ul de transprt prevede construcţia a 1.300 kilometri de autostradă, în valoare 13,7 miliarde euro, 1.825 kilometri de drumuri expres (9,9 miliarde euro), 2.874 kilometri de drumuri transregio (1,69 miliarde euro), 343 kilometri de drumuri transeuro (190 milioane euro), 2.883 kilometri reabilitaţi de cale ferată (10,7 miliarde euro), 1.001 kilometri de linii de cale ferată cu viteze sporite (274,1 milioane euro). sunt prevăzute, printre altele, investiţii şi la porturile constanţa (865,3 milioane euro), galaţi (110,7 milioane euro), drobeta turnu severin (20,2 milioane euro) şi aeroporturile bucureşti-henri coandă (668,9 milioane euro), timişoara (111,5 milioane euro), cluj-napoca (131 milioane euro), iaşi (93,5 milioane euro). dacă ţi-a plăcut articolul, urmăreşte mediafax.ro pe facebook " conținutul website-ului www.mediafax.ro este destinat exclusiv informării și uzului dumneavoastră personal. este interzisă republicarea conținutului acestui site în lipsa unui acord din partea mediafax. pentru a obține acest acord, vă rugăm să ne contactați la adresa vanzari@mediafax.ro. & 720 & medium & Medium & Socio-Economic & NA & NA & 2015-02-25 & 2015 & 1 & ECO
Frame & low-medium & National & 500-1000 & 1.0255497 & 0.6327204 & -1.4131940 & 0.5015415 & -0.9523600 & 0.0 & -1.0024021 & 0.3314703 & Recipient & Domestic & Domestic & Domestic & Domestic|ECO & Neutral\\
Romania & http://www.evz.ro/investitii-de-5-miliarde-de-euro-pentru-dezvoltarea-romanilor.html & 541 & evz.ro & Private/Non-Public & Online and Offline & National & high = CP is most important issue in story (can also cover other issues) & Social justice & Positive & EU & No myth & Jobs & Positive & EU & No myth & Social awareness/inclusion & Positive & No actor & No myth & Romania & investiţii de 5 miliarde de euro pentru dezvoltarea românilor & 2015-02-25 & fondul social european & comisia europeană a adoptat astăzi programul operațional al româniei "capital uman" și finanțat din fondul social european pentru perioada 2014-2020. programul urmărește să investească aproximativ 5 miliarde de euro, din care 4,3 miliarde de euro din bugetul ue, pentru a ajuta cetățenii români, inclusiv tinerii, să își găsească un loc de muncă, să își îmbunătățească nivelul de educație și de competențe, precum și pentru a contribui la reducerea sărăciei și a excluziunii social. o atenție deosebită este acordată tinerilor, romilor și populației din mediul rural. marianne thyssen, comisarul pentru ocuparea forței de muncă, afaceri sociale, competențe și mobilitatea forței de muncă, a declarat: "românia se confruntă în prezent cu importante provocări pe piața muncii, atât în ceea ce privește ocuparea forței de muncă, cât și sărăcia. mă bucur că programul se concentrează pe principalele priorități identificate în cazul româniei și sunt convinsă că, dacă este utilizat în mod corespunzător, acest ajutor financiar va constitui o importantă sursă de investiții în ocuparea forței de muncă, în educație și în combaterea sărăciei sau în sprijinirea serviciilor sociale". programul cuprinde 7 axe prioritare, care dispun de următoarele alocări din bugetul ue: axele prioritare 1 și 2 sunt dedicate punerii în aplicare a garanției pentru tineret în românia, beneficiind de 211 milioane de euro pentru regiunile eligibile pentru inițiativa privind ocuparea forței de muncă în rândul tinerilor (centru, sud-est și sudul munteniei) și de 362 milioane de euro pentru restul țării. axa prioritară 3, având ca temă locurile de muncă pentru toți, dispune de o alocare de 1,1 miliarde de euro din bugetul ue. această axă vizează sprijinirea accesului la ocuparea forței de muncă, acordând o atenție deosebită șomerilor și persoanelor inactive, șomerilor de lungă durată, lucrătorilor în vârstă, persoanelor cu handicap și persoanelor cu un nivel mai scăzut de educație. axa prioritară 4, care dispune de 940 milioane de euro, vizează să promoveze incluziunea socială și să combată sărăcia. printre alte acțiuni se vor număra și o serie de măsuri integrate, care vor ajuta persoanele dezavantajate, inclusiv romii, să aibă acces la piața forței de muncă, prin îmbunătățirea competențelor și sprijinirea spiritului antreprenorial și a întreprinderilor sociale. axa prioritară 5, care dispune de 201 milioane de euro, sprijină dezvoltarea locală aflată în responsabilitatea comunităților. sunt vizate zonele urbane (orașe cu peste 20 000 de locuitori), punându-se un accent deosebit pe comunitățile defavorizate. prin urmare, această măsură va veni în completarea sprijinului pentru zonele rurale și orașele mai mici, acordat în cadrul programului de dezvoltare rurală. axa prioritară 6, care vizează educația și competențele, va investi 1,2 miliarde de euro pentru sprijinirea educației de a doua șansă a tinerilor care nu sunt încadrați profesional și care nu urmează niciun program educațional sau de formare, pentru reducerea ratei de părăsire timpurie a școlii, pentru îmbunătățirea accesului la învățământul terțiar și a calității acestuia, pentru înființarea de stagii și pentru învățarea pe tot parcursul vieții. axă prioritară 7 dispune de restul de 258 de milioane de euro pentru asistență tehnică în vederea punerii în aplicare a programului. & 507 & high & High & Socio-Economic & Socio-Economic & Socio-Economic & 2015-02-25 & 2015 & 1 & ECO
Frame & high-very high & National & 500-1000 & 1.0255497 & 0.6327204 & -1.4131940 & 0.5015415 & -0.9523600 & 0.0 & -1.0024021 & 0.3314703 & Recipient & European & European & European & European|ECO & Positive\\
Romania & http://www.ziuaconstanta.ro/informatii/administratia-bazinala-de-apa-dobrogea-litoral/comunicat-de-presa-proiectul-de-reabilitare-a-litoralului-romanesc-al-marii-negre-realizat-in-proportie-de-90-566579.html & 587 & ZIUA de Constanta & Private/Non-Public & Online only & Regional/Local & medium = CP is important part of story & Environment/green/low-carbon & Positive & National & No myth & Jobs & Positive & No actor & No myth & NA & NA & NA & NA & Romania & comunicat de presa: proiectul de reabilitare a litoralului romanesc al marii negre realizat in proportie de 90\% & 2015-09-30 & fondul de coeziune & constanta, 30 septembrie 2015: proiectul de reabilitare a litoralului romanesc al marii negre, unul dintre cele mai ambitioase proiecte din romania, inregistreaza mari progrese, indeplinind cu succes obiectivele asumate. proiectul de interes national, unic, este realizat in proportie de 90\% si va fi finalizat la timp, pe 30 octombrie 2015. in luna august 2015 a fost demarata prima etapa de innispare in zona eforie nord din cadrul proiectului - "protectia si reabilitarea partii sudice a litoralului romanesc al marii negre (mamaia sud, tomis nord, tomis centru, tomis sud, eforie nord), finantat prin programul operational sectorial mediu, din fondul de coeziune", primul pas in cadrul actiunilor de stopare a fenomenului de eroziune de pe tarmul romanesc al marii negre. romair consulting, companie 100\% romaneasca, in calitate de prestator al serviciilor de asistenta tehnica pentru managementul si supravegherea lucrarilor, coordoneaza cel mai ambitios proiect de protectie a litoralului românesc - "protectia si reabilitarea partii sudice a litoralului romanesc al marii negre (mamaia sud, tomis nord, tomis centru, tomis sud, eforie nord), finantat prin programul operational sectorial mediu, din fondul de coeziune." "romair consulting va finaliza cu succes un proiect unic, de interes national, prin care teritoriul romaniei se mareste cu 33 de hectare ajungand la 72 de hectare de plaja la nivelul celor 5 loturi din proiect. este un moment important pentru tara noastra ce a implicat o echipa de ingineri de top care a reusit sa modernizeze plajele litoralului romanesc cu peste 7,3 km si sa le largeasca in medie cu aproximativ 100 m. mai mult de 3,5 milioane de metri cubi de nisip au fost aduse din mare pentru loturile mamaia sud, tomis nord, tomis centru, tomis sud si eforie nord. ne bucuram ca suntem parte dintr-un proiect de mediu care vine in sprijinul populatiei, dar si un proiect in care facem apa mai curata si turistii mai fericiti. astfel, peste 278.000 de locuitori vor fi protejati de efectele eroziunii costiere si peste 50.000 de turisti anual vor beneficia de plaje moderne.", declara gheorghe boeru, presedinte romair consulting. pana la aceasta data au fost executate 85\% din lucrarile proiectate pentru lotul mamaia sud, 98\% pentru lotul tomis nord, 99\% din lotul tomis centru, 99\% din lotul tomis sud si 75\% din lotul eforie nord. valoarea totala a proiectului este de 170.450.084 euro inclusiv tva. finantarea nerambursabila din fondul de coeziune este in valoare de 145.680.660 euro. investitiile finantate vor fi administrate de beneficiarul proiectului, administratia natională "apele romane", in calitate de institutie nationala cu competente exclusiv in gestionarea zonei costiere a romaniei, si vor fi exploatate de catre acesta prin administratia bazinala de apa dobrogea-litoral. reprezentantii administratiei bazinale de apa dobrogea-litoral, in calitate de beneficiar, au depus toate eforturile pentru obtinerea finantarii celui mai mare proiect din romania, iar din 2013 sunt alaturi de antreprenori si de supervizorul lucrarilor pentru a se asigura ca se ating obiectivele proiectului. despre proiect: comisia europeana a aprobat in data de 22.03.2013 unul dintre cele mai ambitioase proiecte din punct de vedere al reducerii eroziunii costiere, respectiv "protectia si reabilitarea partii sudice a litoralului romanesc al marii negre (mamaia sud, tomis nord, tomis centru, tomis sud, eforie nord)". beneficiarii proiectului sunt cei aproximativ 278.000 de locuitori care vor fi protejati in urma reducerii riscului de inundare a proprietatilor, precum si peste 50.000 de turisti care se vor putea bucura de o plaja extinsa. de asemenea, proiectul aduce avantaje majore in domeniul socio-economic pentru cei peste 122 de operatori economici identificati (hoteluri, restaurante, magazine, firme de pescuit etc.) care vor fi, de asemenea, protejati impotriva riscului de eroziune costiera. proiectul va genera 250 de locuri de munca temporare si 10 permanente. proiectul, finantat prin programul operational sectorial mediu, din fondul de coeziune, isi propune realizarea de masuri de protectie a plajei impotriva riscului de eroziune accelerate care afecteaza zonele mamaia sud, tomis nord, tomis centru, tomis sud si eforie nord, pe o lungime de 7,3 km. proiectul consta in lucrari de reparatie a digurilor de larg existente, constructia de noi diguri de larg, reabilitarea si prelungirea digurilor de protecţie, constructia de diguri de protectie a zonei innisipate si a unor diguri ingropate (pinteni), demolarea unor structuri existente si lucrari de innisipare artificiala pentru largirea plajei litoralului romanesc cu peste 60 de metri. ca urmare a implementarii proiectului, se urmaresc: protectia si imbunatatirea calitatii mediului si a standardelor de viata in romania, urmarindu-se conformarea cu prevederile acquis-ului de mediu. respectarea in totalitate a directivelor ue si a legislatiei romanesti in domeniul constructiei infrastructurii de protectie la eroziune a coastei marii negre din zona mamaia sud, tomis nord, tomis centru, tomis sud si eforie nord; cresterea nivelului de protectie a bunurilor si cresterea sigurantei locuintelor din aceasta zona prin stoparea procesului de eroziune si crearea infrastructurii de prevenire a fenomenului in zona afectata de proiect; proiectul va contribui la cresterea capacitatii institutionale locale de implementare a proiectelor cu finantare publica europeana si nationala lucrarile prevad atat executia de structuri noi, reabilitarea/demolarea/extinderea celor existente, cat si reinnisiparea artificiala a plajelor, toate fiind proiectate pentru a asigura stabilitate pe o perioada de 50 ani. proiectul face parte dintr-o initiativa mai ampla, pe termen lung, in contextul recomandarii privind managementul integrat al zonelor costiere si al directivei-cadru privind stategia ue pentru mediul marin. beneficiar: anar, prin administratia bazinala de apa dobrogea litoral antreprenor: asocierea porr bau - dinamica-franco giuseppe pentru lotul mamaia sud, respectiv van oord - societatea de constructii in transporturi bucuresti pentru loturile tomis nord, centru si sud si eforie nord romair consulting - coordonatorul proiectului asistenta tehnica pentru managementul proiectului si supervizarea lucrarilor - "protectia si reabilitarea partii sudice a litoralului romanesc al marii negre in zona mamaia sud, tomis nord, tomis centru, tomis sud, eforie nord" despre romair consulting compania romair consulting este o companie 100\% romaneasca, lider de piata in construirea si implementarea proiectelor finantate de uniunea europeana. infiintata in 1997, compania romair consulting implementeaza astazi 147 de proiecte majore finantate de catre uniunea europeana, cu o valoare totala de investitie de peste 1,8 mld euro, care au contribuit semnificativ la dezvoltarea romaniei, in domenii prioritare precum: infrastructura, mediu si energie regenerabila, agricultura, gis si topogeodezie, program operational sectorial cresterea competitivitatii economice (poscce). romair a realizat aprox. 18\% din proiectele cu care romania a intrat in ue. "de peste 18 ani construim proiecte de succes care demonstreaza ca romania merita sa fie tara membra ue. in prezent, compania romair consulting are aprobate 109 proiecte majore de catre comisia uniunea europeana, fiecare cu o valoare totala a investitiei mai mare de 50 milioane de euro, care schimba in fiecare zi viata locuitorilor romaniei. sunt mandru sa demonstrez cu fiecare proiect pe care il derulez alaturi de echipa romair consulting ca participam efectiv la gradul de absorbtie a banilor ue in romania, iar munca noastra are un impact real pentru milioane de romani: fie ca sunt fermieri care invata sa scrie aplicatii pentru cereri de finantare; sateni care sunt protejati impotriva inundatiilor; locuitori care se bucura de centre istorice restaurate, parcuri reabilitate sau sunt protejati de inundatii; sau tineri antreprenori romani care pun bazele unor branduri romanesti puternice, cu ajutorul nostru.", declara gheorghe boeru, presedinte romair consulting. romair consulting este singura companie romaneasca ce deruleaza proiecte pe 5 axe importante: program operational sectorial mediu (pos mediu), program operational regional (por), program operational sectorial transport, autoritatea de management program operational sectorial dezvoltarea resurselor umane (posdru) si planul national pentru agricultura si dezvoltarea rurala (pnadr). de asemenea, romair consulting este singura companie romaneasca ce are expertiza si echipa competenta de a realiza intregul proiect de la faza intocmirii cererii de finantare, pana la implementarea fazelor de supervizare. & 1287 & medium & Medium & Socio-Economic & Socio-Economic & NA & 2015-09-30 & 2015 & 1 & ECO
Frame & low-medium & Regional & +1000 & 1.0255497 & 0.6327204 & -1.4131940 & 0.5015415 & -0.9523600 & 0.0 & -1.0024021 & 0.3314703 & Recipient & Domestic & Domestic & Domestic & Domestic|ECO & Positive\\
Romania & http://www.romanialibera.ro/economie/fonduri-europene/strategia-ocuparii--proiect-penru-reintegrarea-pe-piata-muncii-a-somerilor--401240 & 493 & RomaniaLibera.ro & Private/Non-Public & Online and Offline & National & low = CP mentioned more times but NOT important part of story (mainly about others issues) & Social awareness/inclusion & Positive & EU + Subnational & No myth & NA & NA & NA & NA & NA & NA & NA & NA & Romania & strategia ocupării, proiect penru reintegrarea pe piața muncii a șomerilor & 2015-12-08 & fondul social european & începând cu data de 28 februarie 2015 şomerii, șomerii de lungă durată și persoanele aflate în căutarea unui loc de muncă din judeţele tulcea, galați și botoșani au avut oportunitatea de a participa gratuit la un set de măsuri de suport şi sprijin, menite să-i ajute să se reintegreze pe piaţa muncii, în cadrul proiectului "strategia ocupării" cofinanțat de fondul social european prin programul operațional sectorial dezvoltarea resurselor umane 2007-2013, implementat de asociația valori dobrogene în parteneriat cu s.c. infotrust-design srl. aceste măsuri au inclus : scopul acestor acțiuni este facilitarea accesului și angajării pe un loc de muncă sau înființarea propriului loc de muncă prin deschiderea propriei afaceri. până în acest moment: 258 de persoane şi-au exprimat interesul de a beneficia de oportunităţile oferite prin proiect, 200 persoane au beneficiat de informare ocupațională și consiliere psihoprofesională; 179 persoane au început și finalizat formarea în ocupații precum: coafor stilist, manichiurist-pedichiurist, croitor, stilist protezist, machior, competențe antreprenoriale; 177 persoane au fost certificate anc; 178 persoane au beneficiat de servicii de asistență și consultanță privind demararea unei afaceri sau activități pe cont propriu și 8 persoane vor fi sprijinite să-și creeze propriul loc de muncă prin dezvoltarea propriei afaceri. în perioada următoare vor fi organizate și derulate conferințele de închidere a proiectului. informaţii suplimentare cu privire la proiect, în acest moment se pot obţine de la beneficiar, asociația valori dobrogene, punct de lucru proiect: localitatea tulcea, aleea crucea roșie, nr.1, parter, telefon 0240 506300, fax 0240 506301, email: contact@asvd.ro, persoană de contact: liviu sirotencu - manager proiect. & 263 & low & Low & Socio-Economic & NA & NA & 2015-12-08 & 2015 & 1 & ECO
Frame & low-medium & National & <500 & 1.0255497 & 0.6327204 & -1.4131940 & 0.5015415 & -0.9523600 & 0.0 & -1.0024021 & 0.3314703 & Recipient & Domestic & European & Mixed & Domestic|ECO & Positive\\
\addlinespace
Romania & http://ziarulfaclia.ro/calendar-pentru-realizarea-spitalului-de-urgenta-din-cluj/ & 571 & ziarulfaclia.ro & Private/Non-Public & Online and Offline & Regional/Local & medium = CP is important part of story & Public services & Positive & EU & No myth & NA & NA & NA & NA & NA & NA & NA & NA & Romania & calendar pentru realizarea spitalului de urgenţă din cluj & 2017-09-07 & politica regională & în cadrul întâlnirii de miercuri, 6 septembrie, de la bruxelles, privind stadiul implementării proiectelor finanţate cu fondurile politicii de coeziune în domeniul sănătăţii, între comisarul european pentru politică regională corina crețu şi ministrul român al sănătăţii florian bodog, s-au pus bazele unui calendar de implementare a proiectelor celor 3 spitale regionale din cluj, iaşi şi craiova, cofinanțate prin programul operațional regional. românia beneficiază de o alocare de peste 22 de miliarde de euro în cadrul politicii de coeziune pentru perioada 2014-2020. în programul operațional regional, aproximativ 326 de milioane de euro sunt dedicate investiţiilor în domeniul sănătăţii, jumătate din această sumă fiind alocată construcţiei celor 3 spitale regionale menționate. spitalele urmează a fi realizate cu expertiza băncii europene de investiţii şi a băncii mondiale, avându-se în vedere finalizarea studiului de fezabilitate până în martie 2018 şi depunerea proiectelor până în luna aprilie a aceluiaşi an. acordul vine ca urmare a validării desemnării sistemelor de management şi control ale programului operaţional regional de către autoritatea naţională de audit, cât şi a îndeplinirii condiţionalităţii privind sănătatea de către românia. "aceste progrese sunt îmbucurătoare, fiind condiţii esenţiale pentru începerea implementării proiectelor în domeniul sănătăţii. susţinem autorităţile naţionale în pregătirea următorilor paşi pentru proiectele deja prevăzute. am discutat şi oportunitatea dezvoltării unor proiecte în domenii de mare necesitate, precum combaterea epidemiei de rujeolă, care din păcate a făcut atâtea victime în ultimele luni", a declarat corina creţu. & 236 & medium & Medium & Socio-Economic & NA & NA & 2017-09-07 & 2017 & 2 & ECO
Frame & low-medium & Regional & <500 & 1.0255497 & 0.6327204 & -1.4131940 & 0.5015415 & -0.9523600 & 0.0 & -1.0024021 & 0.3314703 & Recipient & European & European & European & European|ECO & Positive\\
Romania & https://www.dcnews.ro/fonduri-ue-corina-cretu-strategia-de-la-bruxelles-ar-trebuie-schimbata\_617768.html & 519 & dcnews.ro & Private/Non-Public & Online only & National & high = CP is most important issue in story (can also cover other issues) & Solidarity to poor countries/regions & Positive & EU & No myth & NA & NA & NA & NA & NA & NA & NA & NA & Romania & fonduri ue. corina crețu: "strategia de la bruxelles ar trebui schimbată" & 2018-10-12 & politica de coeziune & comisarul european pentru politica regională, corina crețu, admite că strategia bruxelles-ului trebuie schimbată. într-un interviu acordat revistei parlamentului european, corina crețu spune că "politica de coeziune ar trebui simplificată". politica ar trebui să fie "mai flexibilă și mai orientată spre rezultate. în multe state membre există proiecte doar pentru a avea un proiect și nu pentru a promova locuri de muncă și creșterea economică". corina crețu spune că politica regională "este concepută pentru toate regiunile noastre, în număr de 300. din punct de vedere social, este foarte importantă, pentru că investim în zone pe care sectorul privat nu le consideră profitabile. de exemplu, transportul public". oficialul român a adăugat că unele regiuni au încă nevoie de ajutor pentru lucruri de bază cum ar fi apa, tratarea apelor reziduale, transport etc, în timp ce altele au nevoie de ajutor pentru cercetare si inovație. și, deși există inegalități enorme între regiunile europei, de politica de coeziune beneficiază toate statele. politica de coeziune, una din cele mai mari realizări ale ue într-un moment în care unele critici se îndreaptă spre câți bani sunt cheltuiți în ue, comisarul insistă că "politica de coeziune este necesară mai mult ca niciodată". "voi lupta pentru ea - este una dintre cele mai importante realizări ale acestui proiect unic". privind spre viitor, există temeri că politica de coeziune ar putea fi în pericol, deoarece ue urmărește să își reprogrameze modelul de finanțare înainte de brexit. cu câteva luni în urmă, comisia a publicat un document privind viitorul europei și, astfel, spune corina crețu, există cinci scenarii din care, din păcate, în patru politica de coeziune este redusă. în aceste condiții devine foarte important modul în care vor fi gestionate resursele financiare. o altă opțiune ar fi contribuția cu o sumă mai mare de bani din partea statelor membre. trecând peste aspectul financiar al politicii de coeziune, corina crețu a precizat că a depus eforturi pentru simplificarea accesului la fonduri. în prezent beneficiarii fondurilor trebuie să completeze "mii de pagini de formulare și solicitări pentru accesarea fondurilor ue, dar nu există condiții pentru accesarea fondurilor guvernamentale. avem nevoie de o legislație simplificată". o comisie specială lucrează la acest aspect. & 362 & high & High & Values & NA & NA & 2018-10-12 & 2018 & 3 & ECO
Frame & high-very high & National & <500 & 1.0255497 & 0.6327204 & -1.4131940 & 0.5015415 & -0.9523600 & 0.0 & -1.0024021 & 0.3314703 & Recipient & European & European & European & European|ECO & Positive\\
Romania & https://www.mediafax.ro/politic/corina-cretu-comisar-european-romania-nu-si-permite-luxul-sa-se-gandeasca-la-iesirea-din-ue-17574351 & 526 & Mediafax.ro & Private/Non-Public & Online only & National & very high = CP is most important issue + CP is mentioned in title/headline & Mismanagement & Negative & EU & No myth & NA & NA & NA & NA & NA & NA & NA & NA & Romania & corina creţu, comisar european: românia nu-şi permite luxul să se gândească la ieşirea din ue & 2018-10-29 & politica regională & "românia nu-şi poate permite luxul de a se gândi la ideea de a ieşi din uniunea europeană. mai ales acum, când are în faţă preşedinţia consiliului uniunii europene, când are posibilitatea ca lider al uniunii europene să schimbe lucrurile în bine, din interior", a declarat, luni, la bucureşti, la conferinţa euroimpact, comisarul european corina creţu. ea a adăugat că ieşirea marii britanii din ue reprezintă un proces care nu este înţeles pe deplin pentru că marea britanie contribuia cu peste 10\% la fondurile de coeziune alocate statelor membre. comisarul european pentru politică regională, corina creţu, a declarat luni, la bucureşti, că românia a pierdut deja fonduri europene în valoare de aproape 2 miliarde de euro pe care le putea folosi în infrastructura de transport. totodată, aceast a adăugat că absorbţia "greoaie" a fondurilor europene de către românia este o slăbiciune pe care bucureştiul o are încă de la aderarea la uniunea europeană, şi că îşi propune "un dialog permanent cu guvernul". comisarul european pentru politică regională corina creţu efectuează, luni şi marţi, o vizită oficială în românia, ea urmând să aibă o întrevedere cu premierul viorica dăncilă. & 188 & very high & High & Governance & NA & NA & 2018-10-29 & 2018 & 3 & POL
Frame & high-very high & National & <500 & 1.0255497 & 0.6327204 & -1.4131940 & 0.5015415 & -0.9523600 & 0.0 & -1.0024021 & 0.3314703 & Recipient & European & European & European & European|POL & Negative\\
Romania & https://www.monitorulbt.ro/national/2018/10/15/corina-cretu-cere-sa-nu-se-mai-dea-vina-pe-ue/ & 556 & Monitorul de Botosani & Private/Non-Public & Online only & Regional/Local & high = CP is most important issue in story (can also cover other issues) & Political capital/interests & Negative & EU + National & 4.No added value & Mismanagement & Negative & EU & 8.Mismanaged & NA & NA & NA & NA & Romania & corina creţu cere să nu se mai dea vina pe ue - monitorul de botoșani & 2018-10-15 & politica de coeziune & comisarul european corina creţu susţine că are conştiinţa împăcată în legătură cu lucrurile bune pe care le-a făcut, dar afirmă că autorităţile europene nu se pot substitui celor naţionale şi locale. "mă doare să văd că unele neîmpliniri ale statelor membre sunt puse pe seama bruxelles-ului, a comisiei europene. este absolut nedrept", a scris corina creţi, ieri, pe facebook. "am conştiinţa împăcată în legătură cu lucrurile bune pe care le-am făcut: încheierea programelor pentru perioada 2007-2014 în toate statele membre, încheierea acordurilor de parteneriat şi a programelor operaţionale pentru perioada 2014-2020, dar mai ales negocierea bugetului european pentru politică regională, 2021-2027, când pentru prima dată în istoria uniunii europene suma cea mai mare, 373 miliarde de euro, vor fi alocaţi către politică de coeziune, iar din această suma 80\% vor fi investiţi în regiunile mai puţin dezvoltate din europa", a scris corina creţu. aceasta a explicat că a lansat grupul pentru simplificarea procedurilor de accesarea fondurilor europene, dar şi că un alt grup de experţi lucrează pentru strategii adaptate la nevoile cetăţenilor din regiunile rămase în urmă. "lucrăm pentru a adapta regiunile carbonifere la cerinţele secolului xxi. mă doare să văd că unele neîmpliniri ale statelor membre sunt puse pe seama bruxelles-ului, a comisiei europene. este absolut nedrept. nu ne putem substitui autorităţilor naţionale şi locale", a mai scris comisarul european. aceasta a afirmat că şi-a dori o implementare accelerată a unor proiecte "de calitate". "singurul meu obiectiv este de a vedea că regiunile se dezvoltă, viaţa oamenilor se îmbunătăţeşte considerabil graţie solidarităţii europene şi a fondurilor pe care le gestionez", a mai scris creţu. pe 8 octombrie, comisarul european corina creţu a declarat, vizibil iritată, că nu mai acceptă "insulte" din partea guvernului român faţă de munca pe care o face, subliniind că la comisia europeană se fac "eforturi supraomeneşti" pentru a evita dezangajările de fonduri pentru românia, însă fără proiecte nu se poate finanţa. se încarcă...evaluați articolul - se încarcă... & 330 & high & High & Power & Governance & NA & 2018-10-15 & 2018 & 3 & POL
Frame & high-very high & Regional & <500 & 1.0255497 & 0.6327204 & -1.4131940 & 0.5015415 & -0.9523600 & 0.0 & -1.0024021 & 0.3314703 & Recipient & Domestic & European & Mixed & Domestic|POL & Negative\\
Romania & https://ziuadecj.realitatea.net/economie/20-de-fonduri-de-investitii-se-bat-pe-un-proiect-it-din-cluj-google-si-microsoft-interesate--179382.html & 569 & ziuadecj.realitatea.net & Private/Non-Public & Online only & Regional/Local & medium = CP is important part of story & Research \& innovation & Positive & Subnational & No myth & NA & NA & NA & NA & NA & NA & NA & NA & Romania & 20 de fonduri de investiții se bat pe un proiect it din cluj. google și microsoft, interesate & 2018-10-30 & fondul european de dezvoltare regională & antreprenorul clujean paul brie a obținut o finanțare nerambursabilă de 1 milion de euro din fonduri europene și de la stat pentru proiectul său teleporthq, informează portalul start-up cafe. acesta a fost invitat la conferința euroimpact de la bucurești pentru a-și povesti experiența de business și de cercetare în tehnologie. "beneficiile pe care le avem sunt absolut incredibile, nu mă așteptam. era prima oară când am făcut apel la fondurile europene și ne era foarte, foarte frică de aspectul administrativ, dar în final s-a dovedit că beneficiile sunt absolut gigantice", a spus brie. el i-a sfătuit pe antreprenori să se apuce de un proiect european când dispun de anumite resurse materiale, de care nu depinde funcționarea firmei în mod curent. "când aveți mai mulți bani - și nu depindeți de banii pe care îi obțineți de la clienți să supraviețuiți până luna viitoare - aceasta generează timp pentru a face cercetare-dezvoltare. banii pe care i-am primit prin programul operațional competitivitate (poc) ne-au permis nouă, unui grup de oameni dintr-o firmă, să nu mai lucrăm ca să mâncăm, ci să dedicăm timp fundamental pe zona de tehnolgie", a spus brie. "am depus în 2016 proiectul și am primit răspunsul în 2017. am făcut o propunere tehnologică ce, în 2016, era coerentă, dar în momentul în care am primit răspunsul pozitiv. viața s-a schimbat foarte mult. a fost o mare provocare pentru noi, pentru că eram pe punctul de a refuza banii. prin contractul semnat, deși 80\% vin de la fondurile europene, noi avem obligația ca în cinci ani să avem o cifră de afaceri de 1 milion de euro, iar dacă nu putem, să dăm banii înapoi. pentru o firmă mică, a da 1 milion de euro înapoi este destul de stresant. ne-a luat 2 luni să analizăm dacă mai suntem în măsură să răspundem la acele condiții. pe final am ales să o facem, adică să livrăm și proiectul vizat, deși financiar nu mai era viabil, și să-l îmbogățim cu planuri de cercetare-dezvoltare. noi am început să implementăm din vara lui 2017, după ce am reușit să ne reorganizăm ca să facem produsul promis. acum, la sfârșitul lui 2018, suntem la 1 an până la lansarea unui produs pe piață". potențiali investitori mari au fost atrași de un videoclip de prezentare lansat pe linkedin. "vreme de două săptămâni, 2,5 milioane de persoane l-au văzut. am fost contactați de 20 de fonduri de investiții, de google, de microsoft, care au apreciat potențialul unei astfel de tehnologii. făceam un fel de cercetare - n-aș spune fundamentală, dar am explorat liber niște agregări de tehnologii care, deși existau pe piață de câțiva ani, nu au fost combinate în felul în care am făcut-o noi". el crede că antreprenorii români trebuie să lase în urmă afacerile în care fabricau pentru mari branduri din străinătate și să se dedice creării de proprietate intelectuală "made in romania". proiectul a fost propus de un parteneriat constituit din evo forge și corebuild, membri ai clusterului itech transilvania, cu scopul de a facilita trecerea de la outsourcing la dezvoltarea bazată pe inovare. teleporthq a obținut, în 2017, o finanțare europeană prin poc 2014-2020. valoarea este de 4,5 milioane de lei, din care eligibilă nerambursabilă din fondul european de dezvoltare regională, 3,2 milioane, iar eligibilă nerambursabilă din bugetul naţional, 0,56 milioane. proiectul se implementează în cluj, pe o durată de doi ani. rezultatul său va fi un produs inovator de it. & 588 & medium & Medium & Socio-Economic & NA & NA & 2018-10-30 & 2018 & 3 & ECO
Frame & low-medium & Regional & 500-1000 & 1.0255497 & 0.6327204 & -1.4131940 & 0.5015415 & -0.9523600 & 0.0 & -1.0024021 & 0.3314703 & Recipient & Domestic & Domestic & Domestic & Domestic|ECO & Positive\\
\addlinespace
Romania & http://www.ziuaconstanta.ro/stiri/actualitate/arhiepiscopia-tomisului-cumpara-servicii-de-formare-profesionala-si-de-catering-557260.html & 560 & ZIUA de Constanta & Private/Non-Public & Online only & Regional/Local & low = CP mentioned more times but NOT important part of story (mainly about others issues) & Social awareness/inclusion & Factual & Subnational & No myth & NA & NA & NA & NA & NA & NA & NA & NA & Romania & arhiepiscopia tomisului cumpara servicii de formare profesionala si de catering & 2015-07-20 & fondul social european & arhiepiscopia tomisului, partener în proiectul "egalitate pe piața muncii! proiect pilot pentru sprijinirea persoanelor vulnerabile" va organiza şi desfăşura licitaţie pentru atribuirea contractului "servicii de formare profesională, inclusiv servicii de catering". ofertele de participare pot fi depuse până pe 23 iulie, criteriul de atribuire aplicat pentru stabilirea ofertei câştigătoare fiind "preţul cel mai scăzut". cursurile de formare profesională se vor desfăşura între 1.08.2015 şi 15.11.2015. potrivit prevederilor contractului de finanţare posdru aferent proiectului "egalitate pe piaţa muncii! proiect pilot pentru sprijinirea persoanelor vulnerabile", din cadrul programului operaţional sectorial dezvoltarea resurselor umane 2007- 2013, program cofinanţat din fondul social european este necesară achiziţionarea unor servicii de cursuri de formare profesională pentru 143 de persoane aparţinând grupurilor vulnerabile, inclusiv servicii de catering. valoarea totală estimată a serviciilor care urmează să fie achiziţionate a fost estimată la 279.413,28 lei. obiectul principal al contractului constă în prestarea de servicii de formare profesională, inclusiv servicii de catering şi săli de curs complet utilitate pentru cursanţi, pentru următoarele cursuri de calificare - nivel 1: cofetar, manichiurist, pedichiurist, patiser, asamblor articole textile, barman ospătar, îngrijitor bătrâni la domiciliu, lucrător în comerţ. seriile de formare sunt organizate în cadrul proiectului "egalitate pe piaţa muncii! proiect pilot pentru sprijinirea persoanelor vulnerabile" şi se adresează unui număr de 700 persoane vulnerabile, urmărind dezvoltarea competenţelor şi abilităţilor acestora, dar şi asigurarea unor măsuri active pentru creşterea motivaţiei acestora pentru dezvoltarea profesională şi personală. beneficiarul proiectului "egalitate pe piața muncii! proiect pilot pentru sprijinirea persoanelor vulnerabile" este agenția națională antidrog, în parteneriat cu direcția generală de asistență socială și protecția copilului bucurești, romani criss - centrul romilor pentru intervenție socială și studii, fundația de sprijin comunitar, arhiepiscopia tomisului, agenția națională pentru romi. "egalitate pe piaţa muncii!": agenția națională antidrog şi arhiepiscopia tomisului vin în sprijinul persoanelor vulnerabile (galerie foto) & 301 & low & Low & Socio-Economic & NA & NA & 2015-07-20 & 2015 & 1 & ECO
Frame & low-medium & Regional & <500 & 1.0255497 & 0.6327204 & -1.4131940 & 0.5015415 & -0.9523600 & 0.0 & -1.0024021 & 0.3314703 & Recipient & Domestic & Domestic & Domestic & Domestic|ECO & Neutral\\
Romania & http://www.bzi.ro/proiect-important-lansat-de-usamv-iasi-522553 & 514 & BZI.ro & Private/Non-Public & Online only & National & low = CP mentioned more times but NOT important part of story (mainly about others issues) & Jobs & Positive & EU + Subnational & NA & NA & NA & NA & NA & NA & NA & NA & NA & Romania & proiect important lansat de usamv iasi & 2015-10-19 & fondul social european & universitatea de stiinte agricole si medicina veterinara "ion ionescu de la brad" (usamv) iasi, in parteneriat cu sc brainer consulting srl lanseaza proiectul "sp² = sanse profesionale prin stagii de practica". maine, 20 octombrie 2015, in aula magna harlamb vasiliu are loc conferinta de lansare a proiectului, care va avea loc la ora 13:00. proiectul se deruleaza pe o perioada de sase luni si beneficiaza de finantare nerambursabila din fondul social european prin programul operational sectorial pentru dezvoltarea resurselor umane 2007-2013, axa prioritara nr. 2: "corelarea invatarii pe tot parcursul vietii cu piata muncii", domeniul major de interventie 2.1 - "tranzitia de la scoala la viata activa". grupul tinta este reprezentat de 180 de studenti de la facultatea de agricultura, care vor beneficia de servicii de orientare in cariera, din care 150 studenti vor participa la stagii de practica. vezi si simpozion international la usamv iasi despre sol, hrana si resurse pentru o viata sanatoasa & 156 & low & Low & Socio-Economic & NA & NA & 2015-10-19 & 2015 & 1 & ECO
Frame & low-medium & National & <500 & 1.0255497 & 0.6327204 & -1.4131940 & 0.5015415 & -0.9523600 & 0.0 & -1.0024021 & 0.3314703 & Recipient & Domestic & European & Mixed & Domestic|ECO & Positive\\
Romania & http://www.magazinsalajean.ro/eveniment/sisteme-de-management-performant-la-consiliul-judetean & 572 & magazinsalajean.ro & Private/Non-Public & Online and Offline & Regional/Local & low = CP mentioned more times but NOT important part of story (mainly about others issues) & Public services & Positive & National + Subnational & No myth & NA & NA & NA & NA & NA & NA & NA & NA & Romania & "sisteme de management performant" la consiliul județean & 2018-04-11 & fondul social european & judeţul sălaj, în calitate de beneficiar, implementează începând din 29 martie 2018 proiectul cu denumirea "sisteme de management performant pentru consiliul judeţean sălaj". contractul de finanţare pentru acest proiect a fost semnat de către preşedintele consiliului judeţean sălaj, tiberiu marc, în calitate de reprezentant legal al beneficiarului şi paul stănescu, viceprim-ministru, ministrul dezvoltării regionale şi administraţiei publice, în calitate de reprezentant legal al ministerului dezvoltării regionalea - autoritatea de management a programul operaţional capacitate administrativă. proiectul este co-finanţat din fondul social european prin programul operaţional capacitate administrativă 2014-2020, obiectivul specific 2.1 - introducerea de sisteme și standarde comune în administrația publică locală ce optimizează procesele orientate către beneficiari în concordanță cu strategia pentru consolidarea administrației publice 2014 - 2020. potrivit reprezentanților consiliului județean, obiectivul principal al proiectului urmăreşte îmbunătăţirea performanţei organizaţionale prin implementarea instrumentului de autoevaluare a modului de funcţionare a instituţiilor administraţiei publice (caf). implementarea proiectului va avea următoarele rezultate finale: o procedură caf implementată instituţional cuprinzând planul de acţiuni elaborat în vederea îmbunatăţirii şi eficientizării modului de funcţionare a consiliului judeţean sălaj; 50 de angajați de la nivelul consiliui județean sălaj instruiţi în domeniile managementul şi implementarea caf, precum şi evaluarea şi monitorizarea implementării caf, la nivelul instituţiei consiliului judeţean sălaj. valoarea totală a proiectului se ridică la 522.988,78 lei, din care: 444.540,46 lei - finanţare din fondul social european; 67.988,54 lei - finanţare din bugetul naţional; 10.459,78 lei - cofinanţarea eligibilă a consiliului judeţean sălaj. durata de implementare a proiectului este de 12 luni, cu începere din 29 martie a.c.. acesta este primul proiect finanţat în cadrul programului operaţional capacitate administrativă 2014-2020 şi se doreşte a fi un exemplu de bună practică pentru alte instituţii publice din judeţ, susțin sursele amintite. & 292 & low & Low & Socio-Economic & NA & NA & 2018-04-11 & 2018 & 3 & ECO
Frame & low-medium & Regional & <500 & 1.0255497 & 0.6327204 & -1.4131940 & 0.5015415 & -0.9523600 & 0.0 & -1.0024021 & 0.3314703 & Recipient & Domestic & Domestic & Domestic & Domestic|ECO & Positive\\
Romania & https://jurnalul.antena3.ro/stiri/politica/corina-cretu-va-participa-saptamana-viitoare-la-conferinta-la-nivel-inalt-privind-viitorul-politicii-de-coeziune-de-la-dubrovnik-801920.html & 520 & jurnalul.antena3.ro & Private/Non-Public & Online and Offline & National & very high = CP is most important issue + CP is mentioned in title/headline & Improve governance & Factual & EU & No myth & NA & NA & NA & NA & NA & NA & NA & NA & Romania & corina creţu va participa săptămâna viitoare la conferinţa la nivel înalt privind viitorul politicii de coeziune, de la dubrovnik & 2019-03-01 & politica de coeziune & comisarul european pentru politică regională, corina creţu, va participa, în perioada 4 - 5 martie, la o conferinţă la nivel înalt privind viitorul politicii de coeziune în următoarea perioadă bugetară 2021 - 2027, ce va avea loc la drubrovnik, în croaţia. discuţiile se vor concentra în mod special pe consolidarea conexiunilor dintre politica de coeziune şi reformele structurale. "aştept cu deosebit interes vizita la dubrovnik în vederea participării la conferinţa la nivel înalt organizată de prietenii politicii de coeziune (friends of cohesion). astfel de forumuri de dezbatere sunt esenţiale pentru conturarea viitorului acestei politici şi pentru asigurarea succesului investiţiilor sale", a declarat corina creţu înaintea vizitei. în croaţia, înaltul oficial european se va întâlni cu ministrul croat pentru dezvoltare regională şi fonduri europene, gabrijela zalac, cu ministrul polonez pentru investiţii şi dezvoltare economică, jerzy kwiecinski, cu ministrul sloven pentru politica de coeziune, iztok puric, precum şi cu alţi reprezentanţi ai autorităţilor din 26 de state membre ue. agerpres & 156 & very high & High & Governance & NA & NA & 2019-03-01 & 2019 & 3 & POL
Frame & high-very high & National & <500 & 1.0255497 & 0.6327204 & -1.4131940 & 0.5015415 & -0.9523600 & 0.0 & -1.0024021 & 0.3314703 & Recipient & European & European & European & European|POL & Neutral\\
Romania & https://ziuadecj.realitatea.net/administratie/comisia-europeana-spitalul-regional-de-urgenta-poate-fi-construit-de-cj-cluj--181786.html & 593 & ziuadecj.realitatea.net & Private/Non-Public & Online only & Regional/Local & high = CP is most important issue in story (can also cover other issues) & Public services & Balanced & EU + National + Subnational & No myth & Improve governance & Negative & EU + National + Subnational & No myth & NA & NA & NA & NA & Romania & comisia europeană: spitalul regional de urgență poate fi construit de cj cluj & 2019-01-15 & politica regională & comisia europeană (ce) oferă județului cluj o soluție pentru a merge mai departe în realizarea spitalului regional de urgență, însă decizia stă în mâinile guvernului. consiliul județean (cj) cluj ar putea lua locul ministerului sănătății, ca beneficiar al proiectului pentru construirea spitalului regional de urgență, potrivit comisarului european corina crețu. în acest sens, autoritățile județene vor solicita guvernului româniei să declare consiliul ca fiind eligibil în depunerea proiectului pentru construirea spitalului: "în răspunsul primit recent de la comisarul european pentru politici regionale, doamna corina crețu, se admite faptul că programul operațional regional - p.o.r. 2014/2020 poate fi modificat în cursul perioadei de programare, la cererea autorităților naționale. altfel spus, consiliul județean cluj poate solicita guvernului româniei să declare consiliul ca fiind eligibil în depunerea proiectului pentru construirea spitalului regional de urgență, adică beneficiar în locul ministerului sănătății. în realizarea pasului instituțional următor, în imediata ședință a consiliului județean, voi cere consilierilor să își însușească solicitarea, astfel încât demersul nostru să fie unul unitar, considerând, astfel, că sănătatea nu are culoare politică. documentul va fi prezentat în ședință publică și va solicita modificarea p.o.r. 2014 - 2020 în sensul înlocuirii ministerului, ca aplicant eligibil, cu unitățile administrativ-teritoriale pe raza cărora s-a decis amplasarea spitalelor regionale", a transmis președintele consiliului județean, alin tișe. de asemenea, cj cluj va cere sprijinul județelor din regiunea de nord-vest în solicitarea care va fi adresată guvernului. "ținând cont de importanța proiectului, voi solicita și consiliilor județene din regiunea de nord-vest să sprijine demersul nostru", a adăugat tișe. consiliul judeţean a solicitat preluarea proiectului spitalului regional de urgenţă în condiţiile în care românia ar putea pierde finanţarea europeană pentru realizarea celor trei spitale regionale, dat fiind faptul că ministerul sănătăţii şi guvernul psd nu arată niciun interes în demararea lor, consiliile judeţene din cluj, iaşi și craiova ar putea trece la cârma proiectelor. în luna noiembrie 2018, președintele cj cluj, alin tișe, a transmis ce - direcția generală de politică regională și urbană precum și autorității de management pentru por din cadrul ministerului dezvoltării regionale și administrației publice o adresă oficială prin care solicită preluarea de către forul județean a proiectului de construire a spitalului regional de urgență de la cluj. "nu numai că am discutat, dar în calitate de preşedinte al cj cluj am şi semnat o adresă către ms şi către corina creţu prin care, văzând incompetenţa crasă a guvernului şi văzând că a abandonat din nou proiectele spitalelor regionale, am cerut ceea ce aţi citit dumneavoastră din acel comunicat de presă. adică noi solicităm ca la nivelul judeţului, cj cluj să devină titularul finanţării pe por, ceea ce presupune o modificare a ghidului acceptat de ce, tocmai pentru ca noi, în calitate de proprietari ai terenului pe care îl avem acolo, să putem accesa în mod direct aceşti bani, fără să mai ţinem de guvern", a declarat preşedintele cj cluj, alin tişe, la realitatea fm cluj. răspunsul ministerului sănătăţii a fost unul tranşant. ministrul sorina pintea le-a replicat reprezentanţilor cj cluj că acest proiect va fi făcut de către guvernul româniei, şi l-a contrazis comisarul european corina creţu în privința posibilității de suplimentare a fondurilor pentru aceste unităţi regionale. "sru cluj se va face de către guvernul româniei, cu atragerea de fonduri europene. am înţeles că sunt nişte discuţii în spaţiul public. cj cluj ar trebui să se ocupe de spitalul de pediatrie, pentru că am nişte referinţe nu tocmai plăcute despre ceea ce se întâmplă acolo din punctul de vedere al structurii", a declarat ministrul sănătăţii, în noiembrie, la bistriţa. corina creţu: "am putea avea alte trei clinici private la care vor avea acces doar oameni bogaţi" după ce în octombrie 2018, consilierul premierului viorica dăncilă, darius vâlcov, afirma că e mai avantajos ca unele proiecte să fie făcute în parteneriat-public privat în românia, comisarul european corina creţu a reacţionat afirmând că dacă guvernul decide să meargă pe ideea parteneriatului public-privat în cazul spitalelor regionale de urgență de la cluj, iași și craiova, am putea avea " alte trei clinici private la care vor avea acces doar oameni bogaţi". detalii aici. proiectul spitalului regional, vechi din 2004 proiectul de construire a unui spital regional de urgenţă a fost lansat în mandatul de preşedinte al consiliului judeţean (cj) cluj al liberalului marius nicoară, începând din 2004, fiind vorba de o unitate care să deservească judeţele din regiunea nord-vest. cj a identificat atunci un teren pentru acest spital în zona câmpeneşti, aparţinând comunei apahida, la est de cluj-napoca, dar proiectul nu a fost prins la finanţare din partea guvernului. după 2008, când s-a schimbat conducerea cj, noul preşedinte al instituţiei, alin tişe (pdl), a schimbat amplasamentul spitalului, în comuna floreşti, situată în vestul municipiului, dar lipsa finanţării guvernamentale a împiedicat şi de această dată începerea construirii acestuia. pe de altă parte, ziua de cluj scria, la finele lunii august 2018, că problema drumurilor care ar urma să deservească noul spital regional de urgență nu este clarificată. guvernul psd-alde tărăgănează construcția spitalului de urgență cluj în octombrie, comisarul european corina creţu atrăgea atenția, într-o postare pe facebook, că actualul guvern tărăgănează accesarea banilor europeni necesari pentru construcția spitalelor regionale din cluj, iași și craiova. "sunt mâhnită că discuţiile despre un obiectiv atât de important se duc într-o zonă a speculaţiilor şi disputelor", scria crețu după ce consilierul premierului viorica dăncilă, darius vâlcov, a scris pe aceeaşi reţea de socializare, că e mai avantajos ca unele proiecte să fie făcute în parteneriat-public privat în românia, în condiţiile în care banca europeană de investiţii (bei) elaborează studii în valoare de 120 milioane de euro. bani de la ue uniunea europeană acordă fonduri de 150 de milioane de euro pentru construirea celor trei spitale regionale din românia - de la cluj, iaşi, craiova. pentru ca această finanţare să rămână valabilă este obligatoriu ca recepţia obiectivelor să fie finalizată până în 2023. din 2017, ministerul sănătăţii beneficiază de asistenţă tehnică de la bei pentru elaborarea documentaţiei pentru studiile de fezabilitate, depunerea aplicaţiilor de finanţare şi lansarea procedurilor de licitaţie pentru finalizarea investiţiilor aferente celor trei spitale regionale. din iunie 2018 a fost aprobată în ministerul sănătăţii o structură specifică pentru implementarea şi monitorizarea proiectelor spitalelor regionale. & 1041 & high & High & Socio-Economic & Governance & NA & 2019-01-15 & 2019 & 3 & ECO
Frame & high-very high & Regional & +1000 & 1.0255497 & 0.6327204 & -1.4131940 & 0.5015415 & -0.9523600 & 0.0 & -1.0024021 & 0.3314703 & Recipient & Domestic & European & Mixed & Domestic|ECO & Neutral\\
\addlinespace
Romania & http://www.mediafax.ro/stirile-zilei/comisarul-corina-cretu-apel-la-ministerul-transporturilor-sa-depuna-proiecte-mature-pentru-finantare-europeana-critici-fata-de-intarzierea-construirii-spitalelor-regionale-17519775?utm\_source=feedburner\&utm\_medium=feed\&utm\_campaign=Feed\%3A+MediafaxEconomic+\%28Mediafax+-+Economic\%29 & 591 & Mediafax.ro & Private/Non-Public & Online only & National & very high = CP is most important issue + CP is mentioned in title/headline & Mismanagement & Negative & EU & No myth & Infrastructure & Factual & EU & No myth & NA & NA & NA & NA & Romania & comisarul corina creţu, apel la ministerul transporturilor să depună proiecte mature pentru finanţare europeană/ critici faţă de întârzierea construirii spitalelor regionale & 2018-09-14 & politica regională & comisarul european pentru politică regională, corina creţu, face apel la ministerul transporturilor să depună proiecte mature pentru finanţare europeană, cum ar sibiu - piteşti, piteşti - constanţa sau podul de la brăila. corina creţu a declarat, vineri, la cluj-napoca, la un dialog cu cetăţenii organizat la facultatea de studii europene a ubb, că ultimul proiect depus de românia pe programul operaţional infrastructură mare este metroul gara de nord - otopeni, neexistând proiecte noi, în afară de fazarea proiectelor de apă şi canalizare. "sunt probleme privind capacitatea administrativă, nu s-a redus birocraţia, iar domeniul trasporturilor este unul în care românia a pierdut bani. analizăm proiectul de metrou gara de nord - otopeni, ultimul proiect depus de românia pe programul operaţional infrastructură mare şi nu mai avem proiecte noi, în afară de fazarea proiectelor de apă şi canalizare. este important să se vină cu proiecte noi pentru că, începând de anul viitor, se termină proiectele fazate până în 2020. vestea bună este că aceste proiecte se vor termina, vestea proastă este că deja consumă din ceea ce ar fi trebuit să fie alocat pe proiecte noi din perioada 2014 - 2020. fac un apel la ministerul transporturilor să pregătească proiecte mature, sibiu - piteşti, piteşti - constanţa, podul de la brăila. avem bani să începem din această perioadă de programare studiul de fezabilitate pentru autostrada montana. avem bani din programul operaţional de infrastructură mare, sunt surprinsă că toţi primarii îmi spun că nu sunt bani, dar la ce ultimul proiect sub analiză este metroul gara de nord - otopeni", a spus creţu. potrivit acesteia, deşi se spune că reprezentanţii ce îngreunează lucrurile privind absorbţia fondurilor europene, s-a constatat un fenomen şi anume faptul că unele state membre adaugă, prin ghiduri, tot felul de condiţionalităţi în plus faţă de regulamentele europene. "avem întârzieri în toate statele membre, datorită adoptării târzii a legislaţiei privind perioada 2014 - 2020 şi spun că trebuie să învăţăm din lecţiile trecutului", a explicat corina creţu. "sper ca lucrările la cele 3 spitale regionale să înceapă până la sfârşitul anului 2019" comisarul european pentru dezvoltare regională, corina creţu, speră ca lucrările de construire la cele trei spitale regionale de urgenţă de la cluj, iaşi şi craiova să înceapă până la sfârşitul anului viitor, transmite corespondentul mediafax. corina creţu a spus că este trist faptul că s-a prevăzut ca studiile de fezabilitate pentru aceste proiecte să fie gata în 2016, dar nu au finalizate nici în prezent. "au fost întârzieri şi nu mă voi feri niciodată să avertizez statele membre atunci când sunt întârzieri în ceea ce priveşte infrastructura pentru investiţii importante, dar acum bei a realizat studiile de fezabilitate pentru spitalele de la cluj şi iaşi, urmează craiova. proiectele vor sosi la ce până la sfârşitul acestui an, noi sperăm să le aprobăm cât mai repede. vor începe după aceea licitaţiile, speranţa mea este ca până a la sfârşitul anului 2019 să înceapă construcţia efectivă, banii pot fi cheltuiţi până în 2023. este trist că au fost semnate în iulie 2015 şi s-a prevăzut ca studiile de fezabilitate să fie gata în 2016, suntem în 2018 şi nu sunt gata. sperăm că construcţiile vor fi mai rapide decât elaboarea documentaţiilor", a spus creţu. corina creţu participă, vineri, alături de comisarul european pentru agricultură, phil hogan, la facultatea de studii europene a ubb la un dialog cu cetăţenii intitulat "un buget modern pentru o uniune mai puternică", alături de ministrul agriculturii, petre daea, şi europarlamentarul daniel buda. cei doi comisari europeni vor fi prezenţi, vineri, la festivalul zilele recoltei, organizat pe platoul din faţa sălii sporturilor din cluj-napoca, unde participă peste 200 de expozanţi cu produse agricole de sezon, după care vor vizita o cramă la turda şi o fermă de vite din comuna mociu. conținutul website-ului www.mediafax.ro este destinat exclusiv informării și uzului dumneavoastră personal. este interzisă republicarea conținutului acestui site în lipsa unui acord din partea mediafax. pentru a obține acest acord, vă rugăm să ne contactați la adresa vanzari@mediafax.ro. & 664 & very high & High & Governance & Socio-Economic & NA & 2018-09-14 & 2018 & 3 & POL
Frame & high-very high & National & 500-1000 & 1.0255497 & 0.6327204 & -1.4131940 & 0.5015415 & -0.9523600 & 0.0 & -1.0024021 & 0.3314703 & Recipient & European & European & European & European|POL & Negative\\
Romania & http://www.monitorulcj.ro/actualitate/70428-clujul-devine-smart-territory-coziile-nesfarsite-la-ghiseu-atentiile-dispar-promite-cj-cluj-intr-un-proiect-de-4-milioane-de-lei & 574 & monitorulcj.ro & Private/Non-Public & Online only & Regional/Local & very low = CP mentioned once & Public services & Positive & National & No myth & NA & NA & NA & NA & NA & NA & NA & NA & Romania & clujul devine  smart territory :  coziile nesfârșite la ghișeu, atențiile, dispar! , promite cj cluj într-un proiect de 4 milioane de lei & 2019-04-13 & fondul social european & consiliul judeţean şi-a propus implementarea unui proiect, denumit "smart territory", în valoare de aproape 4 milioane de lei, care presupune ca prin folosirea unor soluţii it să reducă la 60 de zile termenul de obţinere a unei autorizaţii de construire. datele făcute publice de banca mondială arată că românia se află pe ultimele locuri în uniunea europeană în privinţa uşurinţei cu care se obţine o autorizaţie de construire, respectiv 287 de zile, media termenului de procesare, faţă de 169 de zile care este termenul mediu în celelalte state membre. "unul dintre obiectivele principale ale proiectului smart territory este reducerea acestui termen la maxim 60 de zile, prin realizarea unui atlas teritorial complex şi implementarea unor soluţii it de ghişeu unic şi de arhivare. consiliul judeţean cluj este singura unitate administrativ-teritorială de acest tip din ţară cu un astfel de program aflat în implementare. practic, timpul pierdut pentru plata unor taxe de emitere şi obţinerea de documente de la mai multe instituţii, completări suplimentare/subiective ale dosarelor, cozile nesfârşite la ghişeu, atenţiile, toate acestea dispar", se arată într-un comunicat remis de consiliul județean cluj. proiectul "judeţul cluj - smart territory" are ca termen de implementare sfârşitul anului 2019 pentru arhivă şi ghişeul unic, respectiv 2020 pentru atlasul teritorial. valoarea totală a proiectului este de 3.901.772,00 lei din care 3.316.506,20 lei este finanţare nerambursabilă din fondul social european (85\%), 507.230,36 lei este finanţare nerambursabilă din bugetul naţional (13\%) şi 78.035,44 lei este cofinanţare eligibilă din bugetul consiliului judeţean cluj (2\%). atlasul teritorial este o aplicaţie informatică care va reuni într-o singură platformă seturi şi baze de date ale diferitelor instituţii, referitoare la activităţi de amenajare a teritoriului, urbanism, locuire, dezvoltare economică, în vederea fundamentării deciziilor referitoare la investiţii şi la planificarea strategică a judeţului. "impactul acestei aplicaţii la nivelul cetăţeanului rezidă din faptul că cele mai bune decizii vor putea fi implementate având la bază date reale, pentru a creşte calitatea vieţii în judeţ. conceptul nu este unul nou, dar nu există implementat la nivel de instituţii publice, motiv care conferă proiectului un caracter inovator", se mai arată în comunicat. în prezent, interogarea informaţiilor se face la cerere, fără a avea acces permanent la informaţii actualizate şi fără un real suport în realizarea bazei de date integrate, transpuse spaţial. sistemul informaţional va include reprezentări cartografice care să permită urmărirea evoluţiei în timp şi în spaţiu a unui număr mare de indicatori din domeniile de activitate relevante pentru dezvoltarea teritorială provenind de la entităţi publice precum agenţia naţională de cadastru şi publicitate imobiliară (e-terra), direcţia naţională de statistică (fişele localităţilor), registrul comerţului, direcţiile de evidenţă a persoanelor, inspectoratele teritoriale de muncă, inspectoratul şcolar judeţean, primării etc. proiectul presupune şi crearea unui ghişeu unic. în procesul de obţinere a avizelor aferente autorizaţiei de construire fiecare aviz presupune o deplasare a cetăţeanului la o instituţie competentă. soluţia pentru eliminarea acestui inconvenient ar fi, potrivit cj cluj, stabilirea unui singur punct de depunere şi ridicare a documentelor, sub forma unui ghişeu unic care să asigure preluarea şi coordonarea obţinerii avizelor de la instituţiile abilitate. în acest mod, cetăţeanul ar putea prezenta toate documentele necesare procesului de autorizare (ante şi post autorizare) printr-o singură deplasare la consiliul judeţean care să obţină toate avizele necesare obţinerii autorizaţiei de construire. se propune astfel realizarea unei platforme electronice care să asigure posibilitatea încărcării şi transmiterii on-line a tuturor documentelor necesare în proces, reducând semnificativ timpii de obţinere a documentaţiilor. soluţia de gestionare electronică a documentelor în cadrul instituţiei, scanarea acestora pentru arhivare şi retro-digitalizarea arhivei se impun, în contextul activităţii extinse şi complexe a uat judeţul cluj, care îşi dublează, anual, actele cu care operează. teritoriul se aglomerează şi se extinde, iar gestiunea acestuia de către instituţie presupune consultarea multor informaţii arhivate sau a unor informaţii curente dar care nu există în format electronic. potrivit comunicatului, inovarea proiectului rezidă din implementarea efectivă a acestor soluţii informatice care, deşi există ca şi concepte la nivelul ţării, nu sunt implementate practic nici la nivel naţional, nici la nivel judeţean. "noţiunea de smart city a devenit în ultimii ani, instrumentul de măsurare a dezvoltării municipiilor şi oraşelor, atât din punctul de vedere al informatizării cât şi al oferirii de facilităţi locuitorilor lor. ţinând cont de faptul că acest concept nu se poate aplica judeţelor care doresc să ofere servicii similare tuturor cetăţenilor judeţului, proiectul propus vine să dezvolte noţiunea de smart territory, adică judeţ - teritoriu smart", mai menţionează comunicatul citat. & 754 & very low & Low & Socio-Economic & NA & NA & 2019-04-13 & 2019 & 3 & ECO
Frame & v.low & Regional & 500-1000 & 1.0255497 & 0.6327204 & -1.4131940 & 0.5015415 & -0.9523600 & 0.0 & -1.0024021 & 0.3314703 & Recipient & Domestic & Domestic & Domestic & Domestic|ECO & Positive\\
Romania & http://www.magazinsalajean.ro/economic/comunicat-de-presa-33 & 503 & magazinsalajean.ro & Private/Non-Public & Online and Offline & Regional/Local & very low = CP mentioned once & Jobs & Factual & Subnational & No myth & NA & NA & NA & NA & NA & NA & NA & NA & Romania & comunicat de presă & 2018-04-03 & fondul european de dezvoltare regională & sc diandra images srl, cu sediul în oras simleu silvaniei, str. partizanilor, nr. 1/a, judet salaj, romania, derulează, începând cu data de 05.03.2018 si avand ca termen de finalizare data de 31.01.2019, proiectul "achizitie de echipamente la sc diandra images srl", cod smis 107374, cofinanțat prin fondul european de dezvoltare regională, în baza contractului de finanțare nr. 1340 / 05.03.2018. proiectul este finanțat prin regio - programul operaţional regional 2014-2020, axa prioritară 2 "imbunatatirea competitivitatii intreprinderilor mici si mijlocii", prioritatea de investitii 2.1 - "promovarea spiritului antreprenorial, in special prin facilitarea exploatarii economice a ideilor noi si prin incurajarea crearii de noi intreprinderi, inclusiv prin incubatoare de afaceri". autoritatea de management: ministerul dezvoltării regionale și administrației publice. organismul intermediar: agenția de dezvoltare regională nord-vest. valoarea totală a proiectului este de 1.310.214,38 lei, din care asistența financiară nerambursabilă este de 873.215,49 lei. obiectivul general al proiectului este acela de a creste competitivitatea societatii prin achizitia de echipamente noi si moderne, crearea si mentinerea de noi locuri de munca care sa dezvolte si sa extinda capacitatea de productie, sa automatizeze principalele procese tehnologice si sa duca la cresterea cifrei de afaceri. detalii suplimentare puteţi obţine de la: nume persoană contact: diana georgiana pop funcţie: administrator tel. 0747 966 646, e-mail: contact@diandraimages.ro & 224 & very low & Low & Socio-Economic & NA & NA & 2018-04-03 & 2018 & 3 & ECO
Frame & v.low & Regional & <500 & 1.0255497 & 0.6327204 & -1.4131940 & 0.5015415 & -0.9523600 & 0.0 & -1.0024021 & 0.3314703 & Recipient & Domestic & Domestic & Domestic & Domestic|ECO & Neutral\\
Romania & https://www.euractiv.ro/fonduri-ue-politica-de-coeziune/corina-cretu-nu-am-o-confirmare-oficiala-privind-constructia-spitalelor-regionale-prin-ppp-12346 & 553 & EurActiv | Știri, politici europene \& Actori UE online & Private/Non-Public & Online only & National & high = CP is most important issue in story (can also cover other issues) & Institutional bargaining over funding & Balanced & EU & No myth & Political capital/interests & Negative & EU & No myth & Public services & Balanced & EU & No myth & Romania & corina crețu: nu am o confirmare oficială privind construcția spitalelor regionale prin ppp @ euractivromania & 2018-10-29 & politica regională & \#politicadecoeziunecorina crețu: nu am o confirmare oficială privind construcția spitalelor regionale prin ppp comisarul ue pentru politică regională a declarat luni că va avea o întâlnire cu premierul viorica dăncilă și alți membri ai guvernului pentru a discuta despre informațiile privind construirea spitalelor regionale prin parteneriat public-privat. "pentru mine vine ca o surpriză această nouă poziționare pentru care nu am o confirmare. la ora 14:00 am o întâlnire cu doamna prim-ministru și cu membri ai cabinetului său. sigur, ne dorim să avem clarificări în această privință, noi discutăm de ani de zile, avem corespondențe între serviciile noastre și autoritățile naționale. de pildă, în ceea ce privește spitalele regionale, având în vedere progresele care s-au făcut, nivelul foarte înaintat al studiilor de fezabilitate, am înțeles inclusiv din declarațiile ministrului pintea că proiectele celor trei spitale regionale de la cluj, iași și craiova vor fi trimise la comisie până la sfârșitul anului. de aceea o voi ruga pe doamna prim-ministru să ne confirme dacă acest lucru este valabil", a declarat, la o conferință de presă în cadrul evenimentului euroimpact, comisarul european corina crețu, întrebată de declarațiile unor oficiali privind construirea celor trei spitale regionale prin parteneriat public-privat, în loc de fonduri europene. amintind că a fost decizia guvernului ponta în 2015 ca cele trei spitale să se realizeze prin fonduri europene - 150 de milioane de euro bani europeni, 150 de milioane de euro finanțare națională și locală, corina crețu a precizat că e greșită declarația potrivit căreia studiul de fezabilitate al băncii europene de investiții ar costa 120 de milioane de euro, în realitate costul fiind de 1,8 milioane de euro. corina crețu a mai spus că va transmite o scrisoare oficială, după cea din aprilie, deoarece comisia europeană dorește să știe dacă guvernul de la bucurești mai vrea să folosească fonduri europene pentru spitale regionale. & 311 & high & High & Power & Power & Socio-Economic & 2018-10-29 & 2018 & 3 & POL
Frame & high-very high & National & <500 & 1.0255497 & 0.6327204 & -1.4131940 & 0.5015415 & -0.9523600 & 0.0 & -1.0024021 & 0.3314703 & Recipient & European & European & European & European|POL & Neutral\\
Romania & http://www.business24.ro/finantare/finantare-afaceri/ai-un-start-up-in-it-un-fond-de-investitii-este-interesat-sa-ti-finanteze-ideea-1591545 & 570 & Business24.ro & Private/Non-Public & Online only & National & medium = CP is important part of story & Research \& innovation & Factual & EU + National & No myth & NA & NA & NA & NA & NA & NA & NA & NA & Romania & ai un start-up in it? un fond de investitii este interesat sa-ti finanteze ideea! & 2018-01-22 & fondul european de dezvoltare regională & gapminder, un fond de investitii cu capital de risc care vizeaza start-up-uri inovatoare din romania, intra pe piata locala, acesta avand finantare europeana, se arata intr-un comunicat de presa al institutiei financiare remis, luni, agerpres. 'cu un buget initial de 26 milioane de euro, gapminder este primul fond de capital de risc creat ca urmare a selectiei derulate de fondul european de investitii. acesta este finantat in majoritate prin programul operational competitivitate 2014 - 2020, cu cofinantare prin fondul european de dezvoltare regionala. gapminder va investi in urmatorii ani in companii aflate in faza de seed sume de pana la un milion de euro si, potential, inca trei milioane de euro per companie in runde ulterioare, precum si in start-up-uri aflate la prima runda de investitii sau in faza de accelerare, in care investitiile vor fi de pana la 100.000 euro', se arata in document. potrivit acestuia, sunt vizate companii inovatoare de tehnologie cu potential ridicat de expansiune internationala, din domenii precum cele ale serviciilor it si software, securitate cibernetica, inteligenta artificiala, transformare digitala, solutii it pentru sanatate, fintech etc. pentru selectia start-up-urilor eligibile, gapminder va sustine si programul de accelerare techcelerator in bucuresti si cluj. 'pentru ca prima runda de evaluari in vederea potentialelor investitii sa aiba loc cat mai rapid, in perioada urmatoare va incepe preselectia start-up-urilor cu ajutorul techcelerator. orice start-up de tehnologie in cautare de finantare se poate inscrie in cadrul acestui program', informeaza fondul de invstitii. fondul european de investitii (fei) este parte a grupului bancii europene de investitii si este principalul finantator de risc pentru imm din europa, prin instrumente de capital de risc, garantii si microfinantare. in romania, fei implementeaza instrumente financiare finantate prin fonduri structurale si de investitii, precum jeremie 2007-2013, initiativa pentru imm si instrumentele financiare din cadrul po competitivitate, po regional si pndr. ti-a placut acest articol? urmareste business24 si pe facebook! comenteaza si vezi in fluxul tau de noutati de pe facebook cele mai noi si interesante articole de pe business24. & 344 & medium & Medium & Socio-Economic & NA & NA & 2018-01-22 & 2018 & 3 & ECO
Frame & low-medium & National & <500 & 1.0255497 & 0.6327204 & -1.4131940 & 0.5015415 & -0.9523600 & 0.0 & -1.0024021 & 0.3314703 & Recipient & Domestic & European & Mixed & Domestic|ECO & Neutral\\
\addlinespace
Romania & http://www.euractiv.ro/video/muresan-despre-bugetul-ue-7940 & 554 & EurActiv | Știri, politici europene \& Actori UE online & Private/Non-Public & Online only & National & high = CP is most important issue in story (can also cover other issues) & Institutional bargaining over funding & Positive & EU + National & No myth & NA & NA & NA & NA & NA & NA & NA & NA & Romania & mureșan dezbate bugetul ue cu reprezentați ai mfp, ins și patronatul investitorilor autohtoni @ euractivromania & 2017-06-16 & politica de coeziune & suntem și noi foarte mulțumiți de creșterea bugetului ue pentru că prevede niște majorări importante pentru niște politici importante pentru noi cum ar fi politica de coeziune la dezbaterea "bugetul uniunii europene pentru anul 2018 - prioritățile", parlamentul european este reprezentat de către eurodeputatul siegfried mureșan (grupul ppe), raportor al parlamentului european pentru bugetul uniunii europene pentru anul 2018. participă, de asemenea, reprezentanți ai autorităților naționale, mediului de afaceri, organizațiilor patronale, organizațiilor sindicale, organizațiilor și grupurilor de expertiză etc. vor fi luate în discuţie prioritățile parlamentului european pentru 2018, cum ar fi creșterea economică, locurile de muncă și siguranța cetățenilor, în raport cu prioritățile româniei (cristian pârvan, președintele patronatului investitorilor autohtoni, raluca zamfirescu, director, ministerul finanțelor publice, vladimir alexandrescu, insitutul național de statistică). & 122 & high & High & Power & NA & NA & 2017-06-16 & 2017 & 2 & POL
Frame & high-very high & National & <500 & 1.0255497 & 0.6327204 & -1.4131940 & 0.5015415 & -0.9523600 & 0.0 & -1.0024021 & 0.3314703 & Recipient & Domestic & European & Mixed & Domestic|POL & Positive\\
Romania & https://www.mediafax.ro/politic/acuzatii-grave-pnl-dragnea-si-psd-au-ascuns-faptul-ca-au-platit-datoria-externa-din-fonduri-europene-teodorovici-somat-sa-raspunda-public-17596687?utm\_source=feedburner\&utm\_medium=feed\&utm\_campaign=Feed\%3A+MediafaxPolitic+\%28Mediafax+-+Politic\%29 & 521 & Mediafax.ro & Private/Non-Public & Online only & National & very high = CP is most important issue + CP is mentioned in title/headline & Mismanagement & Negative & EU + National & 8.Mismanaged & NA & NA & NA & NA & NA & NA & NA & NA & Romania & acuzaţii grave| pnl: dragnea şi psd au ascuns faptul că au plătit datoria externă din fonduri europene/ teodorovici, somat să răspundă public & 2018-11-03 & politica regională & "bombă: datoria externă plătită din fonduri europene şi nu din creşterea economică trâmbiţată de psd+alde! iar i-am prins. liviu dragnea, guvernele lui, împreună cu toate trompetele psd ne-au ascuns un fapt deosebit de grav. psd+alde, la unison, înjură ue de doi ani de zile. tot ce este capital străin sau are legătură cu ue este înfierat cu mândrie proletară. dar ne-au ascuns ceva. ne-au ascuns că o parte importantă a datoriei externe a statului român a fost plătită exclusiv din fonduri ue. bombă, nu!? hoţii ăştia, au luat bani din fonduri europene şi au plătit împrumuturile la bei şi berd. nu încercaţi să înţelegeţi pentru că nicio ţara membră nu ai mai făcut asta până acum. să fie clar pentru toată lumea. de dimineaţă până seară înjură ue în timp ce pe şest se milogesc de oficialii ue să le dea bani gratis să poată să-şi plătească datoriile", a scris florin cîţu, sâmbătă într-o postare pe facebook. liberalul a adăugat că îi cere public ministrului eugen teodorovici să spună dacă e adevărat că împrumuturile de la banca europeană de investiţii (bei) şi banca europeană poentru reconstrucţie şi dezvoltare (berd) au fost plătite cu fonduri ue. "îi cer public lui eugen teodorovici să răspundă la următoarele întrebări: este adevărat că aţi achitat cele două credite de la bei şi berd cu bani din fonduri europene? prin această acţiune aţi diminuat fondurile destinate investiţiilor în românia? aţi informat parlamentul româniei despre această "operaţiune" financiară secretă?", a completat cîţu. comisarul european pentru politică regională corina creţu a afirmat, într-o postare pe facebook, că nu înţelege de ce intervenţiile sale publice au fost extrem de politizate şi a fost acuzată că ar fi "vândută intereselor străine", în condiţiile în care a făcut eforturi pentru a prelua pe fonduri europene toate creditele pe care românia le avea de plătit către bei şi berd. "jignirile pornite că la un semnal la adresa mea şi mai ales atacurile la adresa uniunii europene, injuriile prin care sunt acuzată că sunt vândută intereselor străine, în condiţiile în care toţi ştiu câte eforturi am făcut preluând pe fonduri europene toate creditele româniei pe care le avea de plătit către banca europeană de investiţii şi banca europeană de dezvoltare, salariile funcţionarilor din românia care lucrează pe fonduri europene, văzând zeci de proiecte care nu s-au terminat la timp şi pentru care statul pierdea aproximativ 3 miliarde de euro, sincer, toate acestea mă lasă fără cuvinte", a scris corina creţu, pe facebook. conținutul website-ului www.mediafax.ro este destinat exclusiv informării și uzului dumneavoastră personal. este interzisă republicarea conținutului acestui site în lipsa unui acord din partea mediafax. pentru a obține acest acord, vă rugăm să ne contactați la adresa vanzari@mediafax.ro. & 462 & very high & High & Governance & NA & NA & 2018-11-03 & 2018 & 3 & POL
Frame & high-very high & National & <500 & 1.0255497 & 0.6327204 & -1.4131940 & 0.5015415 & -0.9523600 & 0.0 & -1.0024021 & 0.3314703 & Recipient & Domestic & European & Mixed & Domestic|POL & Negative\\
Romania & http://www.aradon.ro/anunt-de-finalizare-a-proiectului-10/2196404 & 507 & aradon.ro & Private/Non-Public & Online only & Regional/Local & very low = CP mentioned once & Jobs & Positive & Subnational & No myth & NA & NA & NA & NA & NA & NA & NA & NA & Romania & anunț de finalizare a proiectului & 2019-02-14 & fondul european de dezvoltare regională & advertorial. sc trim finance srl, cu sediul în localitatea giroc, str. orhideea, nr. 34, mansardă, camera 1, ap. 11, județul timiș, românia, a derulat, începând cu data de 14.03.2018 și având ca termen de finalizare data de 28.02.2019, proiectul "achiziție de utilaje la sc trim finance srl", cod smis 111864, cofinanțat prin fondul european de dezvoltare regională, în baza contractului de finanțare nr. 1253 / 14.03.2018. proiectul este finanțat prin regio - programul operaţional regional 2014-2020, axa prioritară 2 "îmbunătățirea competitivității intreprinderilor mici și mijlocii", prioritatea de investiții 2.1 - "promovarea spiritului antreprenorial, în special prin facilitarea exploatării economice a ideilor noi și prin încurajarea creării de noi intreprinderi, inclusiv prin incubatoare de afaceri". autoritatea de management: ministerul dezvoltării regionale și administrației publice. organismul intermediar: agenția pentru dezvoltare regională vest. valoarea totală a proiectului este de 1.935.908,71 lei, din care asistența financiară nerambursabilă este de 894.407,52 lei din care valoare eligibilă nerambursabilă din fedr este 760.246,39 lei, iar valoare eligibilă nerambursabilă din bugetul național este 134.161,13 lei. obiectivul general al proiectului este acela de a crește competitivitatea societății prin achiziția de echipamente noi și moderne. rezultatele proiectului sunt: achiziția de utilaje și echipamente. impactul proiectului la nivelul municipiului timișoara, județul timiș, regiunea vest este: crearea și menținerea de noi locuri de muncă care să dezvolte și să extindă capacitatea de producție a societății, reușindu-se crearea a 5 noi locuri de muncă precum și să automatizeze principalele procese tehnologice care să ducă la creșterea cifrei de afaceri. detalii suplimentare puteţi obţine de la: nume persoană contact: matei serban stoica funcţie: administrator tel. 0760 567 575, e-mail: matei.stoica@gmail.com "conținutul acestui material nu reprezintă în mod obligatoriu poziția oficială a uniunii europene sau a guvernului româniei" sc trim finance srl www.inforegio.ro | facebook.com/inforegio.ro investim în viitorul tău! proiect cofinanțat din fondul european de dezvoltare regională prin programul operațional regional 2014-2020 (177836) & 332 & very low & Low & Socio-Economic & NA & NA & 2019-02-14 & 2019 & 3 & ECO
Frame & v.low & Regional & <500 & 1.0255497 & 0.6327204 & -1.4131940 & 0.5015415 & -0.9523600 & 0.0 & -1.0024021 & 0.3314703 & Recipient & Domestic & Domestic & Domestic & Domestic|ECO & Positive\\
Romania & https://ziuadecj.realitatea.net/eveniment/cnair-receptie-pe-un-lot-din-tronsonul-campia-turzii-targu-mures-maine-pe-inca-unul--180834.html & 576 & ziuadecj.realitatea.net & Private/Non-Public & Online only & Regional/Local & very low = CP mentioned once & Infrastructure & Positive & National & No myth & NA & NA & NA & NA & NA & NA & NA & NA & Romania & cnair, recepţie pe un lot din tronsonul câmpia turzii - târgu mureş. mâine, pe încă unul & 2018-12-11 & fondul european de dezvoltare regională & compania națională de administrare a infrastructurii rutiere (cnair) a recepţionat, marţi, lotul de 10 kilometri din autostrada târgu mureş - câmpia turzii construit de austriecii de la strabag. compania naţională de administrare a infrastructurii rutiere (cnair) vrea să dea miercuri în trafic circa 14 kilometri din autostrada câmpia turzii - târgu mureş. după ce azi a recepţionat lotul ungheni - ogra construit de austriecii de la strabag, mâine comisia merge pe lotul ogra - iernut al nemţilor de la geiger, au spus pentru economica.net reprezentanţii cnair. cnair a postat imagini de la recepţia lotului de peste 10 kilometri, construit de strabag. "acum ne întoarcem de la recepţia pe lotul ungheni - ogra şi analizăm datele. mâine mergem pe lotul ogra - iernut şi la iernut", a spus alin şerbănescu, purtătorul de cuvânt al cnair. directorul general al cnair, narcis neaga, a anunţat încă de luni că marţi şi miercuri urmează să fie recepţionaţi cei 14 kilometri de autostradă. "mâine şi poimâine se mai dau în trafic 14 kilometri. acolo autostrada se termină într-un drum naţional. apoi este o trecere la nivel cu calea ferată. am luat legătura cu cei de la cfr, am găsit o soluţie inteligentă la barieră. timpii de aşteptare vor fi foarte scăzuţi. este o soluţie temporară. sperăm că la anul vom avea pasaj peste calea ferată şi peste drum", declara luni narcis neaga, directorul general al cnair, într-un interviu pentru dcnews. şi asociaţia pro infrastructură a transmis marţi că "astăzi și mâine (11 şi 12 decembrie 2018, n.red) comisiile de recepție vor fi prezente pe cele două loturi de 13,7 km între iernut, ogra și aeroport târgu mureș, construite de geiger, respectiv strabag. dacă nu apar modificări sau surprize, probabil că mâine se va deschide traficul. nimic nu este sigur 100\%". reprezentanţii api au spus şi că poliția rutieră și autoritățile implicate sunt de acord cu soluția giratoriului temporar executat fix în capătul autostrăzii, pe aliniament, la câțiva pași de trecerea (cu barieră) la nivel cu calea ferată. "solicităm tuturor șoferilor să circule cu prudență și să respecte limitele de viteză impuse. cerem responsabililor din compania naţională de administrare a infrastructurii rutiere să se îngrijească de întreținerea semnalizării și de aderența corespunzătoare a asfaltului (deszăpezire, materiale antiderapante), astfel încât riscul de accidente (generat de cozile de vehicule care așteaptă la barieră) să fie menținut la un nivel cât mai scăzut", cere ong-ul pentru infrastructură. autostrada câmpia turzii - ogra -târgu mureș: 51,7 km cu finanţare europeană obiectivul construcţiei autostrăzii târgu mureș-ogra-câmpia turzii este de a realiza o legătură între municipiul târgu mureș și municipiul cluj-napoca cu asigurarea continuității sectorului de autostradă 2b câmpia turzii-cluj napoca vest (gilău), cu sectoarele 2a și 1c. "obiectivele specifice ale proiectului se referă la construcția, până în 2019, a 51.796 km de autostradă nouă 2x2 împreună cu 4,7 km de drum de legătură 2x2 spre târgu mureș, a 13 poduri, 30 de pasaje, 2 viaducte, o parcare, un centru de mentenață și coordonare, un spațiu de service, un punct de lucru pentru mentenață și 5 intersecții rutiere (dintre care 4 intersecții rutiere și 1 intersecție semi-rutieră)", se arată într-un document al cadnr. valoarea totală a proiectului este de 1.826.928.177,05 lei (inclusiv tva), finanţat prin programul operaţional infrastructura mare 2014-2020 astfel: 75\% contribuția uniunii europene din fondul european de dezvoltare regională -1.513.061.739,95 lei, 25\% contribuția națională (bugetul de stat) -378.265.434,99lei, iar restul de 313.866.437,10lei reprezintă valoarea tva aferentă cheltuielilor eligibile. & 591 & very low & Low & Socio-Economic & NA & NA & 2018-12-11 & 2018 & 3 & ECO
Frame & v.low & Regional & 500-1000 & 1.0255497 & 0.6327204 & -1.4131940 & 0.5015415 & -0.9523600 & 0.0 & -1.0024021 & 0.3314703 & Recipient & Domestic & Domestic & Domestic & Domestic|ECO & Positive\\
Romania & http://ziarulfaclia.ro/cluj-napoca-a-primit-finantare-pentru-actiuni-urbane-inovatoare/ & 567 & ziarulfaclia.ro & Private/Non-Public & Online and Offline & Regional/Local & medium = CP is important part of story & Research \& innovation & Positive & EU & No myth & NA & NA & NA & NA & NA & NA & NA & NA & Romania & cluj-napoca a primit finanţare pentru acţiuni urbane inovatoare & 2018-10-11 & politica regională & comisia europeană (ce) a anunţat miercuri cele 22 de oraşe câştigătoare în cadrul celei de-a treia cereri de proiecte privind acţiunile urbane inovatoare, se arată într-un comunicat al executivului comunitar. astfel, 92 de milioane de euro din partea fondului european de dezvoltare regională vor finanţa soluţii inovatoare pentru abordarea provocărilor urbane, cum ar fi calitatea aerului, schimbările climatice, locuinţele, locurile de muncă şi competenţele pentru economia locală. oraşele care au primit finanţare şi domeniile sunt: locuri de muncă şi competenţe în cadrul economiei locale: aveiro (portugalia), cluj-napoca (românia), cuenca (spania), eindhoven (olanda), vantaa (finlanda), ventspils (letonia); adaptarea la schimbările climatice: amsterdam (olanda), barcelona (spania), manchester (marea britanie), riba-roja de túria (spania), sevilla (spania), paris (franţa); calitatea aerului: metropola aix-marseille provence (franţa), breda (olanda), helsinki (finlanda), ostrava (cehia), portici (italia); locuinţe: bruxelles (belgia), budapesta (ungaria), gent (belgia), mataró (spania), metropola lyon (franţa). printre exemplele de proiecte se numără noi soluţii pentru a reduce riscul de incendii periurbane în riba-roja de túria, spania, o nouă generaţie de sisteme de alimentare cu energie bazate pe baterii electrice în breda, olanda, locuri de joacă şcolare rezistente în cazul valurilor de căldură în paris, franţa, locuinţe sociale eficiente din punct de vedere energetic în budapesta, ungaria şi orientare profesională inovatoare în ventspils, letonia. de asemenea, miercuri, ce a lansat cea de-a patra cerere de propuneri în cadrul acţiunilor urbane inovatoare. 100 de milioane de euro sunt acum disponibile pentru oraşe pentru a finanţa proiecte inovatoare de protejare şi reducere a vulnerabilităţii spaţiilor publice, după cum s-a anunţat în planul de acţiune din 2017 în cadrul agendei europene privind securitatea. de asemenea, în urma cererii de propuneri, vor fi finanţate proiecte digitale, de mediu şi de incluziune. "prin aceste cereri de proiecte, transformăm oraşele din ue în laboratoare reale pentru a testa soluţii care au potenţialul de a îmbunătăţi calitatea vieţii în toate oraşele lumii. şi deoarece securitatea urbană şi siguranţa spaţiilor publice au devenit o preocupare crucială pentru cetăţeni, după tragicele atacuri teroriste din ultimii ani, acum sprijinim oraşele în eforturile lor de a-şi proteja locuitorii", a declarat comisarul pentru politică regională, corina creţu. & 358 & medium & Medium & Socio-Economic & NA & NA & 2018-10-11 & 2018 & 3 & ECO
Frame & low-medium & Regional & <500 & 1.0255497 & 0.6327204 & -1.4131940 & 0.5015415 & -0.9523600 & 0.0 & -1.0024021 & 0.3314703 & Recipient & European & European & European & European|ECO & Positive\\
\addlinespace
Romania & https://www.bzi.ro/demersuri-pentru-incurajarea-achizitiilor-publice-ecologice-in-regiunea-nord-est-695004 & 589 & BZI.ro & Private/Non-Public & Online and Offline & Regional/Local & medium = CP is important part of story & Environment/green/low-carbon & Balanced & Subnational & No myth & NA & NA & NA & NA & NA & NA & NA & NA & Romania & demersuri pentru incurajarea achizitiilor publice ecologice in regiunea nord-est & 2019-05-25 & politica regională & pentru a adopta achizitiile publice ecologice (green public procurement - gpp) si de a le insera in instrumentele de politica regionala, adr nord-est s-a alaturat initiativei gpp stream, un proiect destinat autoritatilor publice din 5 tari partenere care vor coopera avand obiectivul comun de a promova eco-inovarea, eficienta resurselor si cresterea verde prin implementarea achizitiilor publice verzi. vezi si: arbusti si butasi de flori, distribuiti gratuit asociatiilor si institutiilor proiectul vizeaza imbunatatirea programului operational regional, in vederea dezvoltarii durabile si tranzitiei catre o economie cu emisii scazute de dioxid de carbon, in special prin achizitionarea pieselor de mobilier urban ecologic. la momentul aprobarii proiectului, romania nu adoptase un plan national cu privire la achizitiile publice verzi, desi circa 20 la sta din pib este alocat spre achizitii publice (media europeana fiind de 16 procente). in aprilie 2019, parteneri din bulgaria, spania, franta, italia si romania, cele 5 tari participante in proiect, s-au intalnit in cadrul unui eveniment transnational de invatare si schimb de bune practici, care a avut loc la alzira, in spania. vezi si: moloz si gunoaie aruncate in cantitati mari pe spatiile verzi din iasi "vor urma alte 3 evenimente transnationale organizate in romania, franta si bulgaria. provocarile majore pe care proiectul le vizeaza constau in integrarea achizitiilor publice verzi in cadrul a 8 instrumente de politica din cele 5 tari, precum si convingerea a altor 40 de autoritati de management din uniunea europeana sa aplice setul de instrumente creat in cadrul proiectului gpp-stream", au precizat reprezentantii adr - nord est. & 256 & medium & Medium & Socio-Economic & NA & NA & 2019-05-25 & 2019 & 3 & ECO
Frame & low-medium & Regional & <500 & 1.0255497 & 0.6327204 & -1.4131940 & 0.5015415 & -0.9523600 & 0.0 & -1.0024021 & 0.3314703 & Recipient & Domestic & Domestic & Domestic & Domestic|ECO & Neutral\\
Romania & https://www.mediafax.ro/economic/rovana-plumb-despre-stadiul-absortiei-de-fonduri-europene-au-contribuit-cu-8-6-miliarde-infuzie-in-economie-17964228 & 563 & Mediafax.ro & Private/Non-Public & Online only & National & very low = CP mentioned once & Economic development & Balanced & National & No myth & NA & NA & NA & NA & NA & NA & NA & NA & Romania & rovana plumb, despre stadiul absorţiei de fonduri europene: au contribuit cu 8.6 miliarde infuzie în economie & 2019-03-22 & fondurile structurale & plumb: fondurile structurale au contribuit cu 8,6 miliarde euro infuzie în economie "din bugetul total pentru apelurile lansate, din cei 28 de miliarde de euro alocaţi româniei pe politica de coeziune, sunt linii de finanţare deschise pentru toate domeniile în cuantum de 25 de miliarde de euro, până în momentul de faţă. avem contracte de finanţare semnate: 73\%, într-o perioadă de mai puţin de doi ani, în sumă de 21,2 miliarde de euro" a spus, vineri, rovana plumb, ministrul fondurilor europene, în cadrul conferinţei "priorităţi de investiţii în românia 2021-2027". în plus, plăţile către beneficiari sunt în valoare de 8,4 miliarde de euro, conform sursei citate. "în total, politica de coeziune cu programul naţional de dezvoltare rurală (pndr - n.red.), care face parte din politica agricolă comună, dar reprezintă fondurile structurale şi de investiţii - împreună au contribuit cu 8,6 miliarde de euro infuzie în economie, care cu efectul de levier se duce la 11-12 miliade de euro. vorbesc ca plus valoare. suntem în media uniunii europene (n.red. - 28\%). aşteptăm ca pe solicitările de autorizare transmise să intre şi alţi bani", a completat plumb. de exemplu, pe sănătate, românia a accesat: 325 de milioane de euro pentru screening-uri şi tratamente precoce pentru diverse boli (tbc, hiv - sida etc.), beneficiind 500.000 de persoane; 37 de proiecte pentru unităţile de primiri urgenţe în modernizare şi 20 de ambulatorii. "aşteptăm în continuare facturi la bruxelles pentru lucrări efectuate, (...) încă 80\%", a spus comisarul european pentru politică regională corina creţu. ministrul român al fondurilor europene a răspuns că românia mai aşteaptă aprobări pentru cinci proiecte majore, fiind în analiză centura de sud a bucureştiului, eroziunea costieră, ecluzele (agidea, ovidiu) şi două proiecte de apă şi canalizare. "aşteptăm sibiu - piteşti. aşteptăm să ne spuneţi cum procedăm în continuare cu spitalele regionale. este decizia dumneavoastră. suntem la dispoziţia dumneavoastră", a adăugat creţu. & 315 & very low & Low & Socio-Economic & NA & NA & 2019-03-22 & 2019 & 3 & ECO
Frame & v.low & National & <500 & 1.0255497 & 0.6327204 & -1.4131940 & 0.5015415 & -0.9523600 & 0.0 & -1.0024021 & 0.3314703 & Recipient & Domestic & Domestic & Domestic & Domestic|ECO & Neutral\\
Romania & http://www.business24.ro/macroeconomie/stiri-macroeconomie/corina-cretu-a-aprobat-o-finantare-de-246-de-milioane-de-euro-pentru-constructia-a-50-de-kilometri-de-autostrada-1595437 & 575 & Business24.ro & Private/Non-Public & Online only & National & very high = CP is most important issue + CP is mentioned in title/headline & Infrastructure & Positive & EU & No myth & Economic development & Positive & Subnational & No myth & NA & NA & NA & NA & Romania & corina cretu a aprobat o finantare de 246 de milioane de euro pentru constructia a 50 de kilometri de autostrada & 2018-06-19 & fondul de coeziune & comisarul european pentru politica regionala corina cretu a aprobat o finantare de peste 246 milioane de euro din fondul de coeziune, pentru constructia sectiunii de autostrada ce leaga localitatile targu mures, ogra si campia turzii, se arata intr-un comunicat al reprezentantei comisiei europene in romania. "acest proiect de mare importanta, situat pe un coridor european strategic, va contribui la crearea unei retele de transport mai rapide si mai sigure in romania. de asemenea, va face regiunea mai atractiva atat pentru investitori, cat si pentru turism si va inlesni dezvoltarea economica, permitand locuitorilor sa reduca timpul necesar deplasarii intre targu mures si campia turzii la mai putin de ora", a declarat inaltul oficial european, potrivit unui material publicat pe site-ul reprezentantei comisiei europene in romania. proiectul are in vedere constructia unei portiuni de 51,8 km de autostrada, cu doua benzi pe sens de circulatie, intre orasele sus-amintite, precum si a unei sectiuni de 4,7 km in apropierea orasului targu mures. sectiunea de autostrada va face parte din coridorul 9 - rin-dunare -, gandit sa lege strasbourg de constanta in cadrul retelei de transport pan-european. proiectul urmeaza a fi finalizat pana in luna octombrie 2019. potrivit datelor de pe site-ul cnair, lucrarile la a3, intre targu mures si campia turzii sunt impartite in cinci loturi si ar urma sa coste, per total, peste 400 de milioane de euro fara tva. la acest moment, lucrarile s-au inceput pe patru dintre cele 5 loturi. cele mai avansate sunt pe lotul 2 din sectorul targu mures-ogra, unde asocierea strabag srl si sc straco group a realizat 65\% din lucrari, dar unde exista acum o serie de dificultati cu relocarea unor retele electrice. romania beneficiaza de o alocare de peste 22 miliarde de euro in cadrul politicii de coeziune pentru perioada 2014-2020. proiectele de infrastructura mare din domeniul transportului urmeaza a beneficia de 5,1 miliarde de euro. ...citeste mai departe despre "corina cretu a aprobat o finantare de 246 de milioane de euro pentru constructia a 50 de kilometri de autostrada" pe ziare.com ti-a placut acest articol? urmareste business24 si pe facebook! comenteaza si vezi in fluxul tau de noutati de pe facebook cele mai noi si interesante articole de pe business24. & 378 & very high & High & Socio-Economic & Socio-Economic & NA & 2018-06-19 & 2018 & 3 & ECO
Frame & high-very high & National & <500 & 1.0255497 & 0.6327204 & -1.4131940 & 0.5015415 & -0.9523600 & 0.0 & -1.0024021 & 0.3314703 & Recipient & European & European & European & European|ECO & Positive\\
Romania & http://ziuadecj.realitatea.net/administratie/continua-lucrarile-la-drumul-turistic-rachitele---ic-ponor--141868.html & 578 & ziuadecj.realitatea.net & Private/Non-Public & Online only & Regional/Local & very low = CP mentioned once & Infrastructure & Positive & Subnational & No myth & NA & NA & NA & NA & NA & NA & NA & NA & Romania & continua lucrarile la drumul turistic rachitele - ic ponor & 2015-10-14 & fondul european de dezvoltare regională & regia autonomă de administrare a domeniului public și privat a județului cluj, entitate subordonată consiliului județean cluj, urmărește execuția lucrărilor la proiectul "modernizarea infrastructurii de acces în zona turistică răchițele - prislop - ic ponor". la data de 13 octombrie 2015, asocierea drumuri şi poduri judeţene cluj sa - sc cridov srl realiza, în continuare, lucrări de asfaltare, montare stâlpi de parapeţi şi construirea unui zid de căptuşire. astfel, pe sectorul de drum județean desemnat spre execuție societății drumuri şi poduri judeţene cluj sa s-a turnat primul strat de covor asfaltic între km 76+170 şi km 65+200, precum şi al doilea strat de asfalt între km 76+170 şi 66+500. s-a executat un zid (de captuşire) cu lungimea de 33 m (urmând ca acesta să fie continuat cu încă 8 m lungime) şi s-au montat stâlpi de parapeţi fără lise metalice pe o lungime de 3 km. pe sectorul de drum județean desemnat spre execuție societății sc cridov srl, până marţi erau finalizate lucrările de așternere a primului strat de asfalt între km 76+170 şi 82+330. totodată, în localitatea răchiţele, la km 63+300 s-a amplasat o staţie de betoane (cu bilă) a sc cridov srl necesară lucrărilor de turnarea betonului la rigole, la ziduri şi la permeabilizarea acostamentelor cu beton. lucrările care au mai rămas de efectuat pe acest drum au o lungime de 20,5 km şi au fost evaluate la 18,2 milioane de lei. noul constructor a câştigat licitaţia organizată în acest sens cu o propunere financiară de 13,3 milioane de lei. licitaţia iniţială a fost câştigată în 2009 de asocierea dintre firmele mbs group turda, alfa rom satu mare şi nemzetkozi betonut kft. din ungaria, pentru suma de 34 de milioane de lei. între constructor şi beneficiar (cj cluj) au apărut numeroase neînţelegeri legate de suplimentarea valorii lucrării, care au dus în 2013 la rezilierea contractului. cj a cerut o expertiză asupra lucrărilor efectuate şi în urma prezentării concluziilor acesteia, care arătau că ar fi fost făcute lucrări care nu au respectat standardele de calitate cerute, a depus şi o sesizare la departamentul de luptă antifraudă şi la direcția națională anticorupție. finanțarea investiției este asigurată prin programul operațional regional, din fondul european de dezvoltare regională, bugetul central de stat şi bugetul propriu al cj cluj. pentru a nu pierde banii europeni reabilitarea drumului turistic trebuie terminată până la sfârşitul anului. & 402 & very low & Low & Socio-Economic & NA & NA & 2015-10-14 & 2015 & 1 & ECO
Frame & v.low & Regional & <500 & 1.0255497 & 0.6327204 & -1.4131940 & 0.5015415 & -0.9523600 & 0.0 & -1.0024021 & 0.3314703 & Recipient & Domestic & Domestic & Domestic & Domestic|ECO & Positive\\
Austria & http://diepresse.com/home/innenpolitik/5231645/Vorwahlkampf\_Koestinger-zu-Gast-bei-OeVP-Niederoesterreich & 33 & Die Presse & Private/Non-Public & Online and Offline & National & very low = CP mentioned once & Institutional bargaining over funding & Positive & Subnational & No myth & NA & NA & NA & NA & NA & NA & NA & NA & Austria & vorwahlkampf: köstinger zu gast bei övp niederösterreich & 2017-06-08 & regionalpolitik & elisabeth köstinger sei in partnerschaft mit sebastian kurz "symbol der erneuerung der volkspartei", betont klubchef klaus schneeberger bei der klubklausur. mehr als nur dem landesbudget 2018 ist am donnerstag eine klausur des landtagsklubs der volkspartei niederösterreich im stift heiligenkreuz im wienerwald gewidmet gewesen. die reform des wahlrechts und das demokratiepaket seien ebenfalls diskutiert worden, sagte klubchef klaus schneeberger. weil vorwahlzeit ist, war zudem övp-generalsekretärin elisabeth köstinger zu gast. unter hinweis auf eine pressekonferenz des neuen finanzlandesrates ludwig schleritzko (övp) am mittwoch kommender woche streifte schneeberger den haushalt nur kurz. 8,6 milliarden euro an einnahmen würden 8,8 milliarden an ausgaben gegenüberstehen. nach maastricht betrage der abgang 61 millionen euro. der klubchef kündigte zudem drei resolutionen an. die volkspartei nö spreche sich für eine verlängerung der eu-regionalpolitik nach 2020 aus, weil diese "wichtig für das land" sei. feuerwehren soll die mehrwertsteuer für jene fahrzeuge zurückerstattet werden, "die gesetzlich vorgeschrieben sind, um den betrieb zu gewährleisten". nicht zuletzt, so schneeberger, müssten sonderschulen erhalten bleiben. beim wahlrecht gehe es darum, rechtssicherheit zu gewährleisten, erinnerte der klubobmann an die diskussion um zweitwohnsitzer. "es bedarf einer neuen regelung." gespräche mit der spö hätten noch nicht zu einem ergebnis geführt. komme man noch zu einer gemeinsamen lösung und somit zu einer zwei-drittel-mehrheit, werde es ein verfassungsgesetz geben, sagte schneeberger. bei der spö herrsche zum wahlrecht aktuell die einstellung, ja am ordentlichen wohnsitz, nein am zweitwohnsitz. ohne gemeinsamen nenner kündigte der övp-klubobmann eine wählerevidenz-änderung an. "wir verhandeln weiter." das demokratiepaket mit u.a. stärkung der minderheitenrechte im landtag soll in der juli-sitzung beschlossen werden. elisabeth köstinger sei in partnerschaft mit sebastian kurz "symbol der erneuerung der volkspartei", sagte schneeberger im zusammenhang mit dem besuch der generalsekretärin bei der klausur. dass die landesgruppe den weg des neuen övp-chefs unterstütze, "ist eine selbstverständlichkeit". trotz ihres jungen alters hätten kurz und köstinger mehr politische erfahrung "als der momentane bundeskanzler", merkte der klubobmann außerdem an. & 323 & very low & Low & Power & NA & NA & 2017-06-08 & 2017 & 2 & POL
Frame & v.low & National & <500 & -0.7948903 & -0.1706634 & 1.0035124 & 0.5015415 & -0.3119516 & 9.0 & 0.2648514 & -0.9910023 & Payer & Domestic & Domestic & Domestic & Domestic|POL & Positive\\
\addlinespace
Austria & http://derstandard.at/2000059831441/EU-Hilfsgelder-Sein-und-Schein & 82 & der Standard & Private/Non-Public & Online and Offline & National & very high = CP is most important issue + CP is mentioned in title/headline & Political leverage & Negative & EU + Other country & No myth & NA & NA & NA & NA & NA & NA & NA & NA & Austria & eu-hilfsgelder: sein und schein & 2017-06-26 & kohäsionsfonds & alles über werbung, stellenanzeigen und immobilieninserate die ministerpräsidenten der vier visegrád-staaten weigern sich, die beschlossene richtlinie zur umverteilung der flüchtlinge anzuwenden laut der deutschen bundeskanzlerin angela merkel habe ein "geist neuer zuversicht" dank der engen abstimmung mit dem neuen französischen staatspräsidenten" das eu-gipfeltreffen geprägt. bei dem ungewöhnlichen gemeinsamen auftritt mit der kanzlerin vor der presse sagte auch emmanuel macron: "wenn deutschland und frankreich mit einer stimme reden, dann kann europa vorankommen." auch der "europäische trauerakt" am nächsten samstag am sitz des europäischen parlaments in straßburg für den verstorbenen deutschen bundeskanzler helmut kohl symbolisiert den handlungswillen zur stärkung des europäischen geistes. dass auf französischem boden der französische staatspräsident und die deutsche bundeskanzlerin gemeinsam an seinem sarg, zusammen auch mit dem früheren demokratischen us-präsidenten bill clinton und dem ehemaligen sozialistischen regierungschef spaniens, felipe gonzáles, stehen und sprechen werden, das wird bleiben. so wie das bild kohls, hand in hand mit dem französischen präsidenten françois mitterrand im september 1984 bei der gedenkfeier von verdun. während macron betont, dass es darum gehe, "die menschen zu vereinen", um die bürger europas glaubwürdig zu beschützen, gibt es auch politiker, die europäische werte nur so lange kennen, bis sie nicht mit den eigenen machtinteressen kollidieren. das gilt zum beispiel für die ministerpräsidenten der vier visegrád-staaten - polen, slowakei, tschechische republik und ungarn -, die sich weigern, die beschlossene richtlinie zur umverteilung der flüchtlinge anzuwenden. mit der bemerkung "europa ist kein supermarkt" hatte macron beklagt, dass man nicht die hand für eu-hilfsgelder aufhalten und andererseits die solidarität bei der verteilung von flüchtlingen verweigern könne. der ungarische ministerpräsident viktor orbán, unterstützt von seiner polnischen kollegin szydlo, konterte sofort: "der start des jungen neulings ist nicht vielversprechend verlaufen. er hat gestern geglaubt, dass er die mitteleuropäischen länder treten könne. so funktioniert das hier nicht." drei tage vorher hatte der ehemalige reichsverweser miklós horthy, der für den eintritt ungarns in den zweiten weltkrieg und für den mord an 560.000 ungarische juden mitverantwortlich gewesen war, ein besseres öffentliches zeugnis von orbán erhalten: horthy sei ein "außergewöhnlicher staatsmann" gewesen. wie standard-korrespondent thomas mayer berichtete, wird sicherheit in einem umfassenden sinn das hauptthema der eu-politik sein. sicherheit von einem effizienten schutz der außengrenzen bis zur terrorbekämpfung würde weit mehr als drei prozent des eu-budgets derzeit kosten. nicht nur macron, auch die deutsche regierung will die auszahlung der gelder aus dem eu-kohäsionsfonds (63 mrd. euro 2014-2020 für osteuropa, vor allem für polen und ungarn) an neue bedingungen knüpfen. da viele mitgliedstaaten (auch österreich) die aufstockung des eu-budgets ablehnen und der brexit einen ausfall von 13 milliarden euro für die eu jährlich bedeutet, dürften die solidaritätsverweigerer in der flüchtlingskrise größerem druck ausgesetzt werden. (paul lendvai, 26.6.2017) weitere kolumne von paul lendvai lesen sie hier. & 460 & very high & High & Power & NA & NA & 2017-06-26 & 2017 & 2 & POL
Frame & high-very high & National & <500 & -0.7948903 & -0.1706634 & 1.0035124 & 0.5015415 & -0.3119516 & 9.0 & 0.2648514 & -0.9910023 & Payer & European & European & European & European|POL & Negative\\
Austria & https://www.sn.at/panorama/international/oettinger-einschnitte-fuer-landwirte-im-neuen-eu-haushalt-23792512 & 96 & sn.at & Private/Non-Public & Online and Offline & Regional/Local & very low = CP mentioned once & Institutional bargaining over funding & Negative & EU + National & No myth & NA & NA & NA & NA & NA & NA & NA & NA & Austria & oettinger: einschnitte für landwirte im neuen eu-haushalt & 2018-02-04 & kohäsionsfonds & auf landwirte und die regionen in europa kommen nach aussagen von eu-haushaltskommissar günther oettinger einschnitte im nächsten eu-haushalt zu. es werde keinen kahlschlag geben, sagte oettinger der "welt am sonntag" laut vorabbericht. aber auch in deutschland müssten sich landwirte und regionen auf kürzungen einstellen. die eu-kommission plane, die mittel für die agrar- und kohäsionsfonds im nächsten mehrjährigen haushalt um jeweils fünf bis zehn prozent zu verringern. im frühjahr beginnen die verhandlungen übe den siebenjährigen finanzrahmen der europäischen union nach 2020. im budget fehlen dann wegen des brexits voraussichtlich bis zu 14 milliarden euro an britischen beiträgen. gleichzeitig will oettinger für einige aufgaben wie verteidigung oder migrationspolitik mehr geld einplanen. im haushalt soll deshalb umgeschichtet werden. zudem sollen die eu-länder zehn bis 20 prozent mehr einzahlen. es gebe bereits vorschläge, wie die kürzungen im agrarsektor gestaltet werden könnten, sagte oettinger. so werde erwogen, die direktzahlungen pro hektar fläche künftig degressiv zu gestalten. die landwirte erhielten dann ab einer bestimmten schwelle weniger finanzielle unterstützung pro hektar. auf deutschland komme insgesamt eine mehrbelastung im einstelligen milliardenbereich zu. oettinger hofft auch auf neue eigenmittel für den eu-haushalt. "wir erwägen auch, dass künftig ein kleiner teil der gewinne, die die europäische zentralbank mit der ausgabe von banknoten macht, als eigenmittel in den eu-haushalt fließt." der kommissar setzt zudem weiter auf seinen vorschlag einer "plastiksteuer" auf verpackungen. mit dem künftigen haushaltsrahmen befassen sich am 23. februar erstmals die eu-staats- und regierungschefs. oettinger will seinen entwurf im mai vorlegen. über einzelheiten dürfte danach monatelang mit den mitgliedsländern und dem europaparlament gestritten werden. & 262 & very low & Low & Power & NA & NA & 2018-02-04 & 2018 & 3 & POL
Frame & v.low & Regional & <500 & -0.7948903 & -0.1706634 & 1.0035124 & 0.5015415 & -0.3119516 & 9.0 & 0.2648514 & -0.9910023 & Payer & Domestic & European & Mixed & Domestic|POL & Negative\\
Austria & http://www.ots.at/presseaussendung/OTS\_20160309\_OTS0170/st-poelten-auftaktveranstaltung-des-eu-foerderprogrammes-oesterreich-tschechische-republik & 97 & APA-OTS & Private/Non-Public & Online only & National & high = CP is most important issue in story (can also cover other issues) & Territorial cooperation & Positive & National + Other country & No myth & NA & NA & NA & NA & NA & NA & NA & NA & Austria & st. pölten: auftaktveranstaltung des eu-förderprogrammes österreich - tschechische republik & 2016-03-09 & regionalpolitik & st. pölten (ots/nlk) - heute erfolgt in st. pölten die kick-off-veranstaltung zum grenzüberschreitenden förderprogramm österreich - tschechische republik. mag. barbara schwarz, landesrätin für eu-regionalpolitik, und zdeněk semorád, tschechiens vizeminister für regionalentwicklung, konnten bei der eröffnung gemeinsam mit vertreterinnen und vertretern der projektpartner oberösterreich, wien, südmähren, südböhmen und vysočina sowie der europäischen kommission über 400 interessierte teilnehmerinnen und teilnehmer im nö landtagssaal begrüßen. insgesamt stehen für die grenzüberschreitende regionalentwicklung in der programmregion in den kommenden jahren rund 98 millionen euro an eu-geldern zur verfügung. "niederösterreich kann dabei 17 millionen euro an eu-mitteln für die kooperation mit tschechischen partnern bis zum jahr 2020 abholen. dieses gemeinsame interreg-programm, auf das wir uns mit unseren programmpartnern geeinigt haben, muss vielen unterschiedlichen bedürfnissen gerecht werden und gleichzeitig ein beitrag zu nachhaltiger, integrierter regionalentwicklung sein", erklärte landesrätin schwarz. "mit der umsetzung dieses programmes wartet also wieder eine große herausforderung, aber auch eine spannende, vielseitige aufgabe auf unsere projektträger und ihre partner", so schwarz. nach der neuen eu-verordnung muss das programm einen beitrag zur umsetzung der eu 2020-strategie leisten und daher auf deren fokus "smart, sustainable and inclusive growth" abzielen. deshalb werden grenzüberschreitende projekte zwischen österreichischen und tschechischen partnern vor allem in den programmschwerpunkten forschung, technologische entwicklung und innovation, umwelt und ressourcen, entwicklung von humanressourcen sowie nachhaltige netzwerke und institutionelle kooperation unterstützt. tschechiens vizeminister für regionalentwicklung zdeněk semorád betonte in diesem zusammenhang die bisherige gute zusammenarbeit: "gerade in den strategisch wichtigen bereichen krankenhäuser und rettungsdienste, in der gemeinsamen wirtschaftsentwicklung und innovationspolitik sowie in der aus- und weiterbildung gilt es, den begonnen weg der erfolgreichen zusammenarbeit fortzusetzen und weiterzuentwickeln. aber auch zu den aktuellen herausforderungen im umweltbereich - etwa im klimaschutz und den öko-innovationen - sind grenzüberschreitende kooperationen gefragt." das fördergebiet des aktuellen programmes ist mit dem programmgebiet aus der förderperiode 2007 bis 2013 identisch und umfasst die folgenden regionen: mostviertel-eisenwurzen, sankt pölten, waldviertel, weinviertel, wiener umland-nordteil, wien, innviertel, linz-wels, mühlviertel, steyr-kirchdorf sowie jihomoravský kraj (südmähren), jihočeský kraj (südböhmen) und kraj vysočina (vysočina). nähere informationen: büro lr schwarz, mag. (fh) dieter kraus, telefon 02742/9005-12655, e-mail dieter.kraus@noel.gv.at, www.at-cz.eu. & 359 & high & High & Socio-Economic & NA & NA & 2016-03-09 & 2016 & 2 & ECO
Frame & high-very high & National & <500 & -0.7948903 & -0.1706634 & 1.0035124 & 0.5015415 & -0.3119516 & 9.0 & 0.2648514 & -0.9910023 & Payer & Domestic & European & Mixed & Domestic|ECO & Positive\\
Austria & http://www.wienerzeitung.at/themen\_channel/wissen/forschung/945249\_Vorarbeiten-fuer-IST-Park-Klosterneuburg-gestartet.html & 25 & Wiener Zeitung & Private/Non-Public & Online and Offline & Regional/Local & very low = CP mentioned once & Research \& innovation & Positive & National + Subnational & No myth & NA & NA & NA & NA & NA & NA & NA & NA & Austria & vorarbeiten für ist park klosterneuburg gestartet & 2018-02-05 & strukturfonds & 15 millionen euro investitionen und fertigstellung für mitte 2019 geplant. klosterneuburg. nach mehr als vier jahren planungsphase beginnen die vorarbeiten für das technologiezentrum ist park in klosterneuburg. in das gebäude werden circa 15 mio. euro investiert, in einem ersten schritt sollen rund 50 arbeitsplätze entstehen, teilte die niederösterreichische wirtschaftsagentur ecoplus am montag in einer aussendung mit. die fertigstellung ist für mitte 2019 geplant. ist park ist eine gemeinsame initiative von ecoplus und dem institute of science and technology (ist) austria. gebaut wird ein zentrum für dem ist austria nahestehende forschungseinrichtungen, spin-offs und technologieorientierte unternehmen. auf zwei ebenen werden 2.500 quadratmeter forschungs- und büroflächen zur verfügung stehen. einer der zukünftigen mieter ist laut aussendung das unternehmen crystalline mirror solutions, das speziell beschichtete spiegel herstellt. der bau wird aus efre-strukturfonds- und ecoplus-regionalfördermitteln finanziert. georg schneider, managing director des ist austria, erklärte: "bei der kommunikation unserer arbeit sehen wir oft die herausforderung, dass der nutzen unserer grundlagenforschung für die bevölkerung auf den ersten blick nicht erkennbar ist. durch das neue technologiezentrum wird der nutzen sofort sichtbar. hier entstehen arbeitsplätze, die zum teil in direkter verbindung mit unserer forschungstätigkeit stehen." bürgermeister stefan schmuckenschlager (övp) sagte, klosterneuburg werde "als stadt der wissenschaft aufwertet". das fünfte technologie- und forschungszentren (tfz) in niederösterreich wurde im jänner in seibersdorf eröffnet, weitere standorte befinden sich in krems, tulln, wiener neustadt und wieselburg land. rund 120 mio. euro wurden bisher laut ecoplus-geschäftsführer helmut miernicki in die tfz investiert, dadurch seien rund 900 arbeitsplätze entstanden. & 249 & very low & Low & Socio-Economic & NA & NA & 2018-02-05 & 2018 & 3 & ECO
Frame & v.low & Regional & <500 & -0.7948903 & -0.1706634 & 1.0035124 & 0.5015415 & -0.3119516 & 9.0 & 0.2648514 & -0.9910023 & Payer & Domestic & Domestic & Domestic & Domestic|ECO & Positive\\
Austria & http://www.ots.at/presseaussendung/OTS\_20161201\_OTS0177/jahresbericht-2015-des-eu-rechnungshofs-im-bundesrat & 56 & APA-OTS & Private/Non-Public & Online only & National & low = CP mentioned more times but NOT important part of story (mainly about others issues) & Mismanagement & Negative & National + Other country & No myth & NA & NA & NA & NA & NA & NA & NA & NA & Austria & jahresbericht 2015 des eu-rechnungshofs im bundesrat & 2016-12-01 & strukturfonds & wien (pk) - die fehlerquote bei der verwendung der eu-mittel ist weiterhin hoch. das zeigt einmal mehr der europäische rechnungshof (erh) auf, dessen jahresbericht 2015 vom bundesrat diskutiert wurde. der bericht wurde vom bundesrat einstimmig zur kenntnis genommen. wie der eu-rechnungshof ausführt, waren die fehlerhaften ausgaben nur in den wenigsten fällen auf betrug, ineffizienz oder verschwendung zurückzuführe vielmehr handelt es sich dabei um eine schätzung der mittel, die nicht hätten ausgezahlt werden dürfen, weil sie nicht vollständig im einklang mit den eu-vorschriften verwendet wurden. grundsätzlich ist dem erh zufolge die fehlerquote bei förderregelungen, die auf der erstattung von kosten der begünstigten basieren, tendenziell höher als bei regelungen, die auf zahlungsansprüchen beruhen. der eu-ausschuss des bundesrates nehme seine verantwortung wahr, das zeige sich auch mit der intensiven befassung mit dem bericht des erh, hielt edgar mayer (v/v) fest. der bericht zeige eine fehlerquote auf, die mit 3,8\% noch immer deutlich über der toleranzgrenze von 2\% lag. immerhin gebe es auch eine positive tendenz. oskar herics, österreichs vertreter im erh, habe gegenüber dem eu-ausschuss des bundesrates vor allem auch die entwicklung in der eu kritisiert, immer mehr finanzinstrumente zu schaffen, wie den europäischen stabilitätsmechanismus (esm) oder die europäische investitionsbank (eib), mit eingeschränkter kontrolle des erh. der erh ist etwa bei der eib nur dort zuständig, wo diese programme im auftrag der kommission durchführt. herics plädiere auch dafür, bei den prüfungen der finanzinstrumente einen stärkeren fokus auf die wirkung zu legen. einen unbefriedigenden zustand hat der prüfer auch in hinblick auf die europäische zentralbank (ezb) geortet, erinnerte mayer. es sei nicht sache des erh, fragen der geldpolitik zu prüfen, zu kritisieren sei aber der mangelnde zugang zu den unterlagen. in österreich gebe es systemmängel bei der abwicklung von projekten, was zu korrekturen führte. die eu verwalte ein großes budget, das brauche entsprechende kontrolle, betonte stefan schennach (s/w). ein problem bestehe darin, dass zu oft gelder unter falschen titeln ausgezahlt werden, was dann zu korrekturen führt. schennach hob hervor, dass die rate der aufdeckung von fehlern durch den erh weit höher liegt als in den nationalstaaten. systemmängel bei der abwicklung von eu-projekten machten letztes jahr auch hierzulande finanzkorrekturen von 10 mio. € erforderlich. speziell die auszahlungen aus den eu-strukturfonds an österreich waren korrekturbedürftig. der erh stellte bei der republik in diesem bereich für den zeitraum 2009 bis 2015 mit 53\% eine deutlich über dem eu-schnitt (42\%) liegende fehlerquote fest, das sei sehr hoch. österreich müsse hier dringend korrekturen vornehmen. ein großes problem des eu-budgets liegt aus schennachs sicht darin, dass aufgrund der bestimmungen gelder oft nicht abgerufen werden können, zuletzt etwa zehn prozent der mittel. hier liege ein systemfehler vor, da gerade staaten mit wirtschaftlichen schwierigkeiten hier weitere nachteile erleiden. es sei wichtig, dass die regierungen der eu-staaten auf ihre rechnungshöfe hören, meinte christoph längle (f/v). so zeige der österreichische rechnungshof immer wieder hohe sparpotenziale auf. was die eu betreffe, so sei zwar positiv zu vermerken, dass die fehlerquote rückläufig sei, es gebe aber verbesserungsbedarf, auch für österreich. besonders die hohe fehlerquote in den strukturfonds, die über dem eu-schnitt liege, müsse gesenkt werden. österreich als nettozahler zahle rund 1,1 mrd. € mehr an die eu ein, als es aus dem budget erhalte, erinnerte längle. für ihn ist es dabei besonders wichtig, auf die österreichische landwirtschaft zu achten. die freiheitlichen stehen klar zu europa und zum miteinander, hielt er fest. sie betonten aber gleichzeitig, dass eine bessere kontrolle der finanzpolitik der eu notwendig ist. heidelinde reiter (g/s) hinterfragte, wie aussagekräftig die 3,8\% einer durchschnittlichen fehlerquote ist, das sie in den verschiedenen bereich sehr unterschiedlich ist. immerhin sei die zahl der fälle, wo tatsächlich betrügerische absicht vorliege, sehr niedrig. erfreulich ist aus ihrer sicht, dass die empfehlungen des erh von der eu-kommission sehr konsequent nachvollzogen werden. eine schwachstelle der förderpolitik der eu ist für reiter, dass es gerade für schwächere gruppen schwierig sei, mittel tatsächlich abzuholen. daher sei es kein wunder, wenn förderungen an große betriebe und große projekte fließen, meinte sie. auch für strukturschwache länder sei es schwer, mittel abzuholen. dem erh sei das problem bekannt und er versuche, hilfestellungen zu leisten. handlungsbedarf für österreich ortete die bundesrätin in der zielerreichung von projekten und der darstellung des nutzens von projekten. das gelte auch beim einsatz vom mitteln gegen den klimawandel, hier werden die eu-ziele nicht erreicht. ein sehr negativer punkt ist für reiter, dass in der frage der finanztransaktionssteuer sich nichts bewegt hat. martin preineder (v/n) betonte, dass gerade die österreichische landwirtschaft von den eu-zahlungen abhängig ist, da die geringen lebensmittelpreise gerade die produktionskosten abdecken. die ausgleichszahlungen der eu sind damit notwendig, um tatsächlich ein einkommen zu erzielen. österreich hat im vorjahr 1,5 mrd. € aus dem eu-haushalt zurückerhalten, davon 1,1 mrd. € für die landwirtschaft. gerade im bereich der landwirtschaft gebe es allerdings auch eine fehlerquote von 33\%, was unter dem eu-durchschnitt von 45\% liegt. allerdings wäre mit mehr kontrolle aus sicht preineders nicht mehr effizienz zu erreichen, da die kontrollen bereits jetzt eine hohe belastung der landwirtinnen darstellen. ein grundproblem der förderungen liegt laut dem bundesrat in der koppelung des großteils der ausgleichszahlungen an die fläche. diese sei aber notorisch schwierig zu bestimmen und schwanke je nach angewandter messmethode beträchtlich. insgesamt habe das aber finanztechnisch für die eu keine weiteren auswirkungen, gab er zu bedenken. ein typisches beispiel für überbürokratisierung der förderrichtlinien stellt für preineder auch die feststellung der almflächen dar. für die bäuerinnen und bauern sei eine senkung der bürokratie vordringlich. preineder appellierte, mehr augenmaß bei den förderrichtlinien walten zu lassen und klare vorgaben zu schaffen, was bei anträgen förderrelevant ist, und was nicht. (schluss bundesrat) sox & 946 & low & Low & Governance & NA & NA & 2016-12-01 & 2016 & 2 & POL
Frame & low-medium & National & 500-1000 & -0.7948903 & -0.1706634 & 1.0035124 & 0.5015415 & -0.3119516 & 9.0 & 0.2648514 & -0.9910023 & Payer & Domestic & European & Mixed & Domestic|POL & Negative\\
\addlinespace
Austria & https://www.sn.at/politik/weltpolitik/eu-gipfel-kurz-lehnt-staerkere-belastung-der-nettozahler-ab-24579931 & 46 & sn.at & Private/Non-Public & Online and Offline & Regional/Local & very low = CP mentioned once & Institutional bargaining over funding & Negative & EU + National & 2.Rich countries pay & NA & NA & NA & NA & NA & NA & NA & NA & Austria & eu-gipfel - kurz lehnt stärkere belastung der nettozahler ab & 2018-02-23 & strukturfonds & verhandlungen über den milliardenschweren eu-finanzrahmen gehören zu den schwierigsten, die es in der eu gibt. diesmal wird es noch schwieriger, denn mit großbritannien verliert die union auch ihren drittgrößten beitragsleister zum eu-budget. bundeskanzler sebastian kurz (övp) sprach sich vor den beratungen trotzdem gegen eine stärkere belastung der nettozahler aus. "denn die nettozahler leisten jetzt schon einen sehr großen beitrag", sagte kurz am freitag vor dem eu-gipfel in brüssel. österreich habe eine klare position, so der övp-chef. österreich wolle eine starke eu, aber auch, dass die eu sparsam mit dem geld der steuerzahler umgehe. wo es möglich sei, sollte die eu schlanker werden, um mehr budget für wichtige aufgaben wie sicherheitspolitik zu haben, "wo es notwendig ist, an einem strang ziehen. was wir nicht wollen, ist eine ständige stärkere belastung de nettozahler." österreich sei mit dieser forderung in einer gruppe von fünf staaten, die sich gut akkordiere und auch die richtigen argumente habe. "wenn die europäische union durch den brexit kleiner wird und auch ein wichtiger nettozahler wegfällt, dass es dann natürlich notwendig ist, dass man sich die frage stellt, wo kann man sparsamer werden?" es gebe zusätzliche aufgaben auf eu-ebene und bereiche, wo österreich wolle, dass sich die eu engagiere, "wo wir bereit sind, dafür zu bezahlen, aber es muss auch bereiche geben, wo wir schlanker und sparsamer werden". zu der forderung der deutschen kanzlerin angela merkel, eu-mittel an die flüchtlingsaufnahme zu knüpfen, sagte kurz, man stehe erst am anfang des verhandlungsprozesses, es gebe noch nicht einmal einen konkreten vorschlag. "die verhandlungen werden sehr lange dauern", erwartet der kanzler. am ende werde das ergebnis ein kompromiss sein. "viele der vorschläge, die derzeit am tisch liegen, gefallen uns als nettozahler so nicht." er könne grundsätzlich nachvollziehen, dass man gewisse konditionalitäten festsetze. "ich würde nur bitten, nicht ständig auf flüchtlinge zu fokussieren. denn solidarität ist weit mehr als nur die aufnahme von flüchtlingen." in der migrationspolitik müsse es "unser ziel sein, die menschen an der außengrenze zu stoppen, und nicht ständig die verteilung zu diskutieren". auch der niederländische ministerpräsident mark rutte forderte eine kürzung des eu-haushalts nach dem brexit. "wir müssen den haushalt modernisieren. damit können wir geld freimachen und für neue prioritäten ausgeben wie den gemeinsamen digitalen markt, für die sicherung der außengrenzen, dem kampf gegen cyber-kriminalität", sagte rutte am freitag in brüssel. "wir müssen also neue gelder freischaufeln und gleichzeitig das budget senken, da großbritannien austritt." dadurch fallen für den eu-haushalt zahlungen in milliardenhöhe weg. dem widersprach der präsident des europäischen parlaments, antonio tajani. es gebe zahlreiche neue aufgaben in der eu, die bezahlt werden müssten. "wir brauchen deshalb mehr geld", sagte der konservative italienische politiker. er forderte eine eigene einnahmequelle der eu, etwa durch eine "web-tax", also eine steuer auf internet-unternehmen in europa. zugleich stellte sich tajani hinter die forderung der deutschen kanzlerin angela merkel, künftig zahlungen aus den eu-strukturfonds mit der einhaltung der eu-regeln zu verbinden. man könne sich nicht einzelne teile aus den eu-vereinbarungen herauspicken, die man wolle und andere ablehnen. hintergrund ist die weigerung etwa polens, sich an den in der eu vereinbarte verteilung von flüchtlingen zu halten. "wenn man die regeln nicht befolgen wird, muss man etwas zahlen", sagte tajani. polen lehnt dies ab. & 543 & very low & Low & Power & NA & NA & 2018-02-23 & 2018 & 3 & POL
Frame & v.low & Regional & 500-1000 & -0.7948903 & -0.1706634 & 1.0035124 & 0.5015415 & -0.3119516 & 9.0 & 0.2648514 & -0.9910023 & Payer & Domestic & European & Mixed & Domestic|POL & Negative\\
Austria & https://www.ots.at/presseaussendung/OTS\_20170607\_OTS0094/hammerschmid-kostenlose-basisbildung-fuer-tausende-erwachsene-im-ministerrat-beschlossen & 38 & APA-OTS & Private/Non-Public & Online only & National & low = CP mentioned more times but NOT important part of story (mainly about others issues) & Jobs & Positive & National & No myth & Public services & Positive & National & No myth & NA & NA & NA & NA & Austria & hammerschmid: kostenlose basisbildung für tausende erwachsene im ministerrat beschlossen & 2017-06-07 & europäischer sozialfonds & wien (ots) - "ich freue mich sehr, dass wir auch in zukunft vielen tausend menschen in österreich die chance auf bessere basisbildung und einen pflichtschulabschluss geben können. mehr als 111,5 millionen euro stellen bund, länder und europäischer sozialfonds gemeinsam in den jahren 2018 bis 2021 bereit", erklärte bildungsministerin sonja hammerschmid anlässlich eines beschlusses im ministerrat, in dem die entsprechende bund-länder-vereinbarung auf den weg gebracht wurde. damit geht das förderprogramm - besser bekannt als "initiative erwachsenenbildung" - in die dritte programmperiode. die "initiative erwachsenenbildung" wurde im jahr 2012 ins leben gerufen, damit gering qualifizierte menschen bessere chancen am arbeitsmarkt erhalten. in österreich lebende jugendliche und erwachsene können unentgeltlich grundlegende kompetenzen und bildungsabschlüsse auch nach beendigung der schulischen ausbildungsphase erwerben. rund 27.000 personen sollen zwischen 2018 und 2021 erreicht werden - 18.000 im bereich der basisbildung und 9.000 im programmbereich des nachholens des pflichtschulabschlusses. der europäische sozialfonds finanziert erstmals auch das nachholen des pflichtschulabschlusses. die gesamte förderperiode wird auch von drei auf vier jahre ausgeweitet und bietet damit noch mehr qualitativ hochwertige angebote für die betroffenen menschen sowie einen besseren planungshorizont für die beteiligten institutionen. "die lernenden profitieren durch kostenfreie, qualitativ hochwertige und erwachsenengerechte bildungsangebote von professionellen bildungsträgern", so hammerschmid abschließend. & 201 & low & Low & Socio-Economic & Socio-Economic & NA & 2017-06-07 & 2017 & 2 & ECO
Frame & low-medium & National & <500 & -0.7948903 & -0.1706634 & 1.0035124 & 0.5015415 & -0.3119516 & 9.0 & 0.2648514 & -0.9910023 & Payer & Domestic & Domestic & Domestic & Domestic|ECO & Positive\\
Austria & https://aktien-portal.at/shownews.html?id=52848 & 70 & aktien-portal.at & Private/Non-Public & Online only & National & very low = CP mentioned once & Solidarity to poor countries/regions & Balanced & EU + Subnational & No myth & NA & NA & NA & NA & NA & NA & NA & NA & Austria & allgemeines & 2018-10-10 & kohäsionspolitik & im wohlstands-ranking der oecd ist die österreichische bundeshauptstadt in den vergangenen jahren zurückgefallen. die - medial thematisierte - zuwanderung ist aber nur eine von vier erklärungsansätzen. die oecd veröffentlichte soeben ein update ihrer regelmäßigen publikation "regions and cities at a glance". diese publikation enthält eine fülle von informationen zu den sozialen und wirtschaftlichen unterschieden zwischen den regionen bzw. großstädten der oecd und deren entwicklung. in den detailergebnissen für österreich wird das abschneiden für die drei österreichischen metropolregionen wien, graz und linz dargestellt. wien weist danach für 2016 unter den drei heimischen metropolen das höchste bip pro kopf auf. allerdings hat sich die stadt nach oecd-berechnungen im ranking unter 329 berücksichtigten oecd-metropol¬regionen zwischen 2000 und 2016 um zwanzig plätze (auf rang 104) verschlechtert. dies bedeutet eine positionsverschlechterung um etwa 6 prozent innerhalb von 16 jahren. in der medialen diskussion wurde dies insbesondere mit einem hohen bevölkerungszuwachs in wien aus dem inland wie ausland in verbindung gebracht. über diesen eindimensionalen und verkürzten erklärungsansatz hinaus sind jedoch drei weitere erklärungen für diesen rückfall zu nennen. 1. schwächere metropolregionen holen auf tatsächlich sind im gesamten europäischen städtesystem konvergenzprozesse zu beobachten, im ausgangszeitpunkt "schwächere" metropolregionen holen also auf. ähnliches ist auch innerhalb österreichs zu beobachten - eine entwicklung, die übrigens auch politisch erwünscht ist und erhebliche anstrengungen der eu-kohäsionspolitik bzw. der nationalen regionalpolitik zur grundlage hat. 2. außereuropäische länder entwickelten sich teils besser dazu bezieht das oecd-ranking eine beträchtliche zahl von städten in ländern mit ähnlichem entwicklungsniveau außerhalb europas mit ein, die im vergleichszeitraum ein deutlich höheres wachstum aufwiesen als die eu (vor allem australien, südkorea, usa). dies dürfte auch die position der städte dieser länder im ranking verbessert haben. welche städte wien im ranking tatsächlich überholt haben, geht aus den bisherigen oecd-informationen nicht hervor, da die detaildaten noch nicht veröffentlicht wurden. 3. preisbereinigung auf nationaler ebene verzerrt das ergebnis letztlich ist auch ein statistischer effekt denkbar: das ranking basiert auf "kaufkraftbereinigten" werten für das bip pro kopf. die dafür notwendigen preiserhebungen basieren in allen ländern auf nationalen stichproben. diese sind nur für die nationale ebene repräsentativ, nicht aber für einzelne städte oder regionen. bei erheblichen regionalen preisunterschieden innerhalb der länder führt die bewertung zu nationalen kaufkraftparitäten zu erheblichen verzerrungen zugunsten der "teuren" zentren. dies lässt sich beispielhaft an bratislava zeigen: laut eurostat lag sie 2014 beim bip pro kopf in kaufkraftstandards unter den metropolregionen der eu auf rang 7 - weit vor wien (rang 40), dem großteil der deutschen stadtregionen, aber auch vor london oder brüssel. & 415 & very low & Low & Values & NA & NA & 2018-10-10 & 2018 & 3 & ECO
Frame & v.low & National & <500 & -0.7948903 & -0.1706634 & 1.0035124 & 0.5015415 & -0.3119516 & 9.0 & 0.2648514 & -0.9910023 & Payer & Domestic & European & Mixed & Domestic|ECO & Neutral\\
Austria & http://text.derstandard.at/2000068237749/EU-Austritt-Briten-wollen-Bruessel-mehr-Geld-zahlen?ref=rss & 71 & Der Standard & Private/Non-Public & Online and Offline & National & very low = CP mentioned once & Institutional bargaining over funding & Balanced & EU + Other country & No myth & NA & NA & NA & NA & NA & NA & NA & NA & Austria & brexit - eu-austritt: briten wollen brüssel mehr geld zahlen & 2017-11-22 & kohäsionsfonds & in london ist hinter vorgehaltener hand von einer summe von rund 40 milliarden euro die rede - 1 foto während in der britischen regierung erstmals realistisch über die eigenen finanzverpflichtungen gegenüber der eu gesprochen wird, setzen prominente eu-hasser in der konservativen partei auf neue konfrontation mit brüssel. die regierungskrise in berlin habe die brexit-verhandlungen ins chaos gestürzt, glaubt der frühere parteichef iain duncan smith. deshalb wäre zum jetzigen zeitpunkt ein angebot weiterer zahlungen in die clubkasse "töricht", pflichtet ihm der hardliner jacob rees-mogg bei. genau diese offerte will premierministerin theresa may offenbar am freitag dem eu-ratspräsidenten donald tusk unterbreiten. wie in ihrer florentiner rede angekündigt, werde ihr land seine "während der mitgliedschaft eingegangenen verpflichtungen einhalten". gemeint damit ist nun nicht mehr nur die erfüllung aller zahlungen im laufenden eu-haushaltsrahmen, also bis ende 2020. offenbar will london auch darüber hinausgehende verpflichtungen, etwa für projekte des eu-kohäsionsfonds oder der investmentbank eib, einhalten. hinter vorgehaltener hand ist in london von einer gesamtsumme von rund 40 milliarden euro die rede. dafür holte sich die regierungschefin am dienstag die zustimmung des kabinetts. freilich wollen die briten ihr entgegenkommen an baldige verhandlungen über die zukünftigen handelsbeziehungen mit dem kontinent knüpfen. diese werden von den eu-partnern bisher blockiert, was in london für unverständnis sorgt: insbesondere bei der frage die künftige grenze zwischen der republik irland und das britische nordirland betreffend kann es, so die meinung vieler experten, keinen fortschritt geben, ohne dass wenigstens die konturen der künftigen handelsbeziehungen erkennbar sind. zur ernüchterung der hardliner im kabinett hat womöglich die am montag in brüssel endgültig beschlossene abwanderung wichtiger eu-behörden aus london gesorgt. brexit-minister david davis hatte dies noch vor wenigen monaten für keineswegs ausgemacht gehalten. am ende des abstimmungsverfahrens lagen je zwei städte gleichauf, sodass im losverfahren entschieden werden musste. bei der eba zog dublin den kürzeren, die bankenaufsicht geht nun nach paris; für die mit mehr als 1000 hochspezialisierten mitarbeitern deutlich größere arzneimittelbehörde ema erhielt amsterdam den zuschlag vor mailand. dass das mit seinen bewerberstädten bonn (ema) und frankfurt (eba) leer ausgegangene deutschland künftig großbritannien stärker unterstützen könnte, schloss landwirtschaftsminister christian schmidt (csu) in der bbc aus. auch warnte er vor der von eu-feinden favorisierten möglichkeit, die insel könne im märz 2019 ohne jede vereinbarung aus binnenmarkt und zollunion ausscheiden. "das wäre ein desaster für die britische wirtschaft", sagte schmidt. der brexit sei "kein spiel mit gewinnern und verlierern". die niederlande und frankreich, wo ema bzw. eba nun ihre zelte aufschlagen, sind glücklich. "das ist eine anerkennung für die attraktivität und das engagement frankreichs für europa", twitterte beispielsweise präsident emmanuel macron. weniger glücklich zeigte man sich in österreich. als "höchst bedauerlich" bezeichnete etwa wirtschaftskammerpräsident christoph leitl die doppelte niederlage. doppelt deshalb, weil sich wien hoffnung auf den zuschlag für beide eu-institutionen machen konnte. österreich müsse sich künftig frühzeitig verbündete und allianzpartner suchen und etwa mit den kleinen und mittelgroßen ländern langfristige gemeinsame strategien schmieden. (sebastian borger aus london, 22.11.2017) & 494 & very low & Low & Power & NA & NA & 2017-11-22 & 2017 & 2 & POL
Frame & v.low & National & <500 & -0.7948903 & -0.1706634 & 1.0035124 & 0.5015415 & -0.3119516 & 9.0 & 0.2648514 & -0.9910023 & Payer & European & European & European & European|POL & Neutral\\
Austria & http://www.ots.at/presseaussendung/OTS\_20170202\_OTS0146/spoe-kaernten-aufwind-in-allen-bereichen & 29 & APA-OTS & Private/Non-Public & Online only & National & very low = CP mentioned once & Jobs & Positive & Subnational & No myth & NA & NA & NA & NA & NA & NA & NA & NA & Austria & spö kärnten: aufwind in allen bereichen & 2017-02-02 & europäischer sozialfonds & klagenfurt (ots) - "seit april 2016 kontinuierlich steigende beschäftigung und sinkende arbeitslosenquote sowie ein aufwärtstrend beim auftragsbestand der kärntner industrie zeigen deutlich, dass es in kärnten wieder bergauf geht", zeigt sich spö klubobmann herwig seiser erfreut über die aktuellen entwicklungen und die heute präsentierten ergebnisse der regionalisierten iv-konjunkturumfrage. der aufschwung zeige deutlich, dass die gemeinsam von koalition und sozialpartnern getroffenen maßnahmen wirken. "einer der job-motoren ist die bauwirtschaft, die mehrheitlich durch aufträge der öffentlichen hand erfreuliche steigerungsraten vorweisen kann. 2017 werden öffentliche auftraggeber in kärnten bauinvestitionen von mehr als 550 millionen euro tätigen. damit werden 7.000 arbeitsplätze in der bauwirtschaft ganzjährig abgesichert", so seiser. um den arbeitsmarkt weiter zu stärken, investieren land, arbeitsmarktservice und europäischer sozialfonds über den territorialen beschäftigungspakt heuer 34,4 millionen euro in kärnten. rund 4.000 maßnahmenplätze werden damit finanziert. vor allem die gruppe der älteren arbeitssuchenden und langzeitarbeitslosen stehen im fokus dieser maßnahmen. investiert werde auch weiterhin in die technologisierung und modernisierung des landes, unter anderem über die plattform industrie 4.0, die digitalisierungsstrategie des landes und den mikroelektronik-cluster silicon alps. seiser weist darauf hin, dass die ergebnisse der konjunkturumfrage der iv sich auch mit dem bank-austria-konjunkturbarometer, das für kärnten das zweithöchste wirtschaftswachstum aller bundesländer ausweist, decken. zu den in der iv-konjunkturumfrage zitierten berechnungen des ehemaligen rechnungshofpräsidenten josef moser, wonach allein im gesundheitswesen ein einsparungspotenzial von 140,5 millionen euro liege, stellt seiser folgendes klar: "wir werden die gesundheitsversorgung vor leichtfertigen und fatalen blutigen einschnitten beschützen". zudem sei es unredlich wie moser und andere immer wieder äpfel mit birnen vergleichen. "kärnten ist das einzige bundesland mit voll ausgebauter geriatrie - das erhöht natürlich stürmisch den bettenstand", so seiser. "unter der spö wird in kärnten niemand wirklich schmerzhafte einschnitte erleiden müssen, wie sie immer wieder von selbsternannten experten am reißbrett gezeichnet werden. denn diese maßnahmen hätten und haben auswirkungen auf die menschen - anders als bei den experten, die nie in die verlegenheit kommen werden, die verantwortung für ihre blutigen pläne übernehmen zu müssen, stehen für diese spö-geführte regierung die menschen an erster stelle", schließt seiser. & 345 & very low & Low & Socio-Economic & NA & NA & 2017-02-02 & 2017 & 2 & ECO
Frame & v.low & National & <500 & -0.7948903 & -0.1706634 & 1.0035124 & 0.5015415 & -0.3119516 & 9.0 & 0.2648514 & -0.9910023 & Payer & Domestic & Domestic & Domestic & Domestic|ECO & Positive\\
\addlinespace
Austria & http://www.kleinezeitung.at/politik/aussenpolitik/5188574/ & 14 & Kleine Zeitung & Private/Non-Public & Online and Offline & Regional/Local & medium = CP is important part of story & Solidarity to poor countries/regions & Balanced & EU & No myth & NA & NA & NA & NA & NA & NA & NA & NA & Austria & 60 jahre römische verträge: wer oder was ist die eu - und was wird aus ihr? & 2017-03-24 & kohäsionsfonds & am samstag begeht die europäische union in der italienischen hauptstadt "60 jahre römische verträge" mit einem sondergipfel - aber was gibt es eigentlich zu feiern? es begann im sagenhaften prunk des saals der horatier und curiatier im konservatorenpalast in rom - und führte in den brüsseler beton. die herren im dunklen anzug, die am 25. märz 1957 auf dem kapitolshügel die römischen verträge zur gründung der europäischen gemeinschaften zeichneten, hätten sich wohl kaum träumen lassen, was daraus in 60 jahren wurde. mittlerweile ist es ein gebilde mit 28 staaten und einer halbe milliarde menschen, mit 44.000 bediensteten und zehntausenden regeln, mit hochfliegenden ambitionen auf frieden, freiheit und wohlstand - und kleinteiligen zuständigkeiten von der glühbirne bis zum buntstift. es ist ein riese, mit dem die kleinen leute fremdeln. nur gut ein drittel der europäer hatte 2016 ein eindeutig positives bild von der eu oder vertrauen in die institutionen. und noch weniger menschen verstehen, was eu-kommission, rat und parlament in brüssel, straßburg, luxemburg genau tun. wer oder was also ist die eu? und was soll aus ihr werden? parlamentskantine, dienstagfrüh um neun. bas eickhout weiß genau, dass er seine zuhörer leicht mal abhängt. "jetzt wird es kompliziert", sagt der niederländer vorsorglich bei croissants und kaffee im abgeordnetenrestaurant des gigantischen parlamentskomplexes an der brüsseler rue wiertz. der grünen-abgeordnete plagt sich mit einer reform des europäischen emissionshandels, der eigentlich helfen soll, das klima zu retten, aber nie wirklich funktionierte. es geht um co2-zertifikate und zementwerke, um "carbon leakage" und den "cross sectoral correction factor". eickhout wirkt beseelt. die wortungetüme verpackt der 40-jährige mit charme, hinter den brillengläsern blitzt der schalk. er dürfte unter den 500 millionen europäern einer von vielleicht 500 sein, die das nebulöse handelssystem durchdringen. doch schon im parlament kommt er mit seinen grünen ambitionen für optimalen klimaschutz nicht durch. die mehrheit will eine weichere reform. und die 28 eu-mitgliedsländer im ministerrat eine noch windelweichere. jetzt hat jede institution ihre version - die eu-kommission, das parlament und der rat: zeit für den "trilog". das heißt, alle drei parteien werden über wochen oder monate in verhandlungsrunden dampfgegart, bis irgendeine art von kompromiss steht, der das europa der 500 millionen ein kleines bisschen besser macht. vielleicht. es ist das normale demokratische hin und her, aber eben in brüsseler dimensionen: langatmig, undurchsichtig und in 24 sprachen. auch erzeuropäer sind von dem zähen prozedere entnervt. "unser politisches system funktioniert nicht", sagt der fraktionschef der liberalen im europaparlament, guy verhofstadt. der ehemalige belgische regierungschef predigt eine generalüberholung der eu und hat dabei - für einen mann des parlaments nicht weiter verwunderlich - vor allem den rat der mitgliedsländer und die kommission auf dem kieker. statt rat will er eine zweite parlamentskammer. und die 28-köpfige kommission sei einfach groß. es gebe ja noch nicht mal genug ressorts für je einen kommissar pro mitgliedsland, witzelt verhofstadt. zwölf reichten auch. die betroffenen sehen das naturgemäß anders. kommissionsgebäude berlaymont, dienstagnachmittag um drei. corina cretu ist sicher nicht die prominenteste der 28 kommissare - sie scheint irgendwie sympathisch beglückt, dass journalisten zu ihrem hintergrundgespräch zur eu-regionalpolitik gekommen sind. dabei ist cretu zuständig für ein drittel des eu-budgets, sie ist herrin über die vielen fonds, die den zusammenhalt der gemeinschaft stärken sollen. kohäsion heißt das im jargon. zwischen 2014 und 2020 stellt die eu dafür 454 milliarden euro bereit. wie so viele in brüssel verbringt die rumänin mit der sonoren stimme ihre tage im dornigen gestrüpp verquerer abkürzungen, zwischen esi, efre, esf, eusf und eff. aber auch sie zweifelt keinen moment an der wichtigkeit ihres tuns: "kohäsionspolitik alleine wird europa nicht heilen, aber europa wird nicht heil ohne kohäsionspolitik", ist ihr credo. cretu ist stolz auf diese riesige umverteilungsmaschine zwischen gebern und nehmern in der eu. sie verweist auf das verdoppelte bruttoinlandsprodukt der ostdeutschen bundesländer seit 1991, auf knapp eine million geschaffene jobs in der eu in den sieben jahren des förderzeitraums 2007 bis 2013, auf die politische mission: "für mich gibt es keinen besseren ausdruck der eu-solidarität." doch trotz dieser politik bleibt das wirtschaftliche gefälle in der eu erschreckend groß. die luxemburger kommen nach zahlen der deutschen bundesregierung auf ein jährliches bruttoinlandsprodukt von 81.000 euro pro kopf, während die bulgaren gerade mal 5.500 euro pro kopf erwirtschaften. seit den letzten großen eu-erweiterungsrunden 2004 und 2007 sitzen hier einfach sehr ungleiche kumpane in einem boot, und nach der finanz- und eurokrise schaukelt die jolle weiter gewaltig. bei aller von eu-kritikern unterstellten machtfülle - um das ganz große rad zu drehen, fehlen brüssel oft die hebel. selbst eu-kommissionspräsident jean-claude juncker, der 2014 mit großen ambitionen antrat, zeigt immer häufiger seinen frust, dass die eu manchmal so ohnmächtig erscheint. mit vielen im parlament ist er sich einig, woran es liegt: die mitgliedsländer sind es, die den laden aufhalten und in brüssel erst dinge mitbeschließen, von denen sie dann zuhause nichts mehr wissen wollen. ratsgebäude justus lipsius, freitagnachmittag um zwei. es ist das ende des frühjahrsgipfels der eu-staats- und regierungschefs und die deutsche bundeskanzlerin angela merkel zieht bilanz im brüsseler ratsgebäude, wie immer. es waren zwei tage im krisenmodus. die polen haben den aufstand gegen die übrigen 27 länder geprobt. der brexit droht. populisten in den niederlanden, frankreich, deutschland wollen das ende der eu. und das thema des letzten gipfeltages - das 60. jubiläum der römischen verträge und die zukunft der eu - entzweit die gemeinschaft. merkels europa der verschiedenen geschwindigkeiten wollen längst nicht alle, die osteuropäer fürchten sogar einen neuen "eisernen vorhang", wie juncker sagt - das ultimative abgehängtsein jenseits aller kohäsionsfonds. bitterkeit brodelt. beschwichtigung ist merkels antwort. konstruktiv habe man geredet, sagt die kanzlerin. ein signal der geschlossenheit sei geplant beim sondergipfel in rom am 25. märz, ein bekenntnis zu binnenmarkt, wettbewerbsfähigkeit, zum friedenswerk der eu. unbegründet auch die aufregung um die verschiedenen geschwindigkeiten: das gebe es doch längst. beim euro oder beim verzicht auf grenzkontrollen machten ja auch nicht alle mit. von radikalen veränderungen wie bei einem verhofstadt, von einem brüssel 2.0 ist bei merkel nichts zu hören. und zwar so gar nichts, dass ein reporter der "sunday times" die kanzlerin fragt, was sich in europa denn überhaupt ändern werde nach den feiern in rom zum 60. geburtstag? merkels antwort legt nahe: eigentlich nichts. denn bei allen problemen, das betont merkel an diesem sonnigen nachmittag, sei die eu doch ein gelungenes modell, "wo wirtschaftliche stärke und soziale sicherheit in einem maße verwirklicht werden, wie man das auf der welt selten findet". & 1067 & medium & Medium & Values & NA & NA & 2017-03-24 & 2017 & 2 & ECO
Frame & low-medium & Regional & +1000 & -0.7948903 & -0.1706634 & 1.0035124 & 0.5015415 & -0.3119516 & 9.0 & 0.2648514 & -0.9910023 & Payer & European & European & European & European|ECO & Neutral\\
Austria & http://derstandard.at/2000054723084/Turbo-fuer-Verkehrsprojekte-in-Suedosteuropa & 52 & der Standard & Private/Non-Public & Online and Offline & National & very low = CP mentioned once & Infrastructure & Positive & EU & No myth & NA & NA & NA & NA & NA & NA & NA & NA & Austria & turbo für verkehrsprojekte in südosteuropa & 2017-03-24 & kohäsionsfonds & nach dem milliardenschweren juncker-plan will die eu-kommission private investitionen anstoßen "wir dürfen bei der donau nicht die gleichen fehler machen wie beim rhein", sagt die eu-koordinatorin für den rhein-donau-verkehrskorridor, karla peijs. "wäre schade um den schönen fluss." teil des korridors, einer von neun großen transportrouten in europa, seien auch sechs balkanländer, und die will die frühere niederländische verkehrsministerin beim juncker-plan, wie das milliarden-investitionsprogramm connecting europe facility (cef) im volksmund heißt, an bord holen. "wir würden uns ins eigene bein schießen, wenn wir die nicht-eu-länder nicht mitdotieren", sagt peijs. der korridor hätte dann keinen wert mehr für südosteuropa. um die donau durch warentransporte nicht (weiter) zu verschmutzen, müssen die zu 60 prozent aus dem eu-kohäsionsfonds finanzierten projekte für die region relevant sein, reif für die umsetzung und den klimazielen von paris zumindest nicht abträglich. was drei jahre nach dotierung noch immer nicht begonnen wurde, verliert die förderung wieder. und: auch für betrieb und erhaltung ist finanziell vorzusorgen. "die instandhaltung kann jedes tolle projekt umbringen", warnt eu-koordinatorin peijs im gespräch mit dem standard. jedes vorhaben werde von drei externen experten evaluiert. damit will man verhindern, dass monströse autobahnprojekte forciert werden, wie dies in den 1990er-jahren der fall war. straßen bauten regierungen von sich aus als erstes, sagte peijs. dabei brauchten sie keine unterstützung. bei wasserwegen und eisenbahn sei dies anders. mehr action bei investments bei der eu-regionalkonferenz in sofia versuchte eu-verkehrskommissarin violeta bulc optimismus zu versprühen. sie wünscht sich mehr action bei investments in verkehrsinfrastruktur und -technologie, vor allem seitens des privatsektors. um die ob ihrer größe traditionell staatlichen investitionen mit privatem geld aufzupeppen, hat die eu-kommission von den 24,9 milliarden euro des "juncker-plans" eine milliarde abgezwackt und mit der europäischen investitionsbank (eib) sowie dem europäischen fonds für strategische investments zusammengespannt. "blending" nennt die eu-kommission maßnahmen, bei denen kredite mit förderungen, und/oder haftungen zusammengespannt werden, um projekte bank- und kapitalmarktfähig zu machen, damit sie als hebel für milliardeninvestitionen wirken können. die größenordnung bei der wirkungsfolgenabschätzung scheint inzwischen verloren, sie changiert von 350 milliarden bis 500 milliarden euro. die neue fördermilliarde wird im wettbewerb vergeben. der erste call läuft seit februar, bis mitte juli können projekte eingereicht werden, der zuschlag erfolgt im jänner 2018. eine zweite tranche startet im herbst. "mit staatlichen investitionen allein kommen wir nicht weiter", betonte eib-vizepräsident pim van ballekom, "wir brauchen private investoren", sonst explodiere die staatsverschuldung. (luise ungerboeck aus sofia, 24.3.2017) die teilnahme am eu-seminar in sofia erfolgte auf einladung der eu-kommission. & 428 & very low & Low & Socio-Economic & NA & NA & 2017-03-24 & 2017 & 2 & ECO
Frame & v.low & National & <500 & -0.7948903 & -0.1706634 & 1.0035124 & 0.5015415 & -0.3119516 & 9.0 & 0.2648514 & -0.9910023 & Payer & European & European & European & European|ECO & Positive\\
Austria & https://www.ots.at/presseaussendung/OTS\_20170516\_OTS0221/nationalrat-lueger-reform-macht-eu-foerderungen-rascher-besser-und-effizienter & 57 & APA-OTS & Private/Non-Public & Online only & National & very high = CP is most important issue + CP is mentioned in title/headline & Institutional bargaining over funding & Factual & EU + National & No myth & NA & NA & NA & NA & NA & NA & NA & NA & Austria & nationalrat - lueger: reform macht eu-förderungen rascher, besser und effizienter & 2017-05-16 & strukturfonds & wien (ots/sk) - in der nationalratssitzung am dienstag ging spö-abgeordnete angela lueger auf die reform bei den eu-förderprogrammen - im zuge geänderter eu-rahmenbedingungen - ein, die die förderungen der eu rascher, besser und effizienter gestalten werde. "die neun länderprogramme werden zu einem österreichweiten programm zusammengeführt und die 36 förderstellen werden auf 16 reduziert", erklärte die abgeordnete. **** die vereinbarung regelt die aufgabenverteilung von bund und ländern bei der abwicklung der strukturfonds, darunter der europäische fonds für regionale entwicklung efre und der europäische sozialfonds esf, sowie die etz-programme (für europäische territoriale zusammenarbeit). lueger führte aus, dass es im bereich des efre-fonds in der letzten periode ein fördervolumen von 680 millionen euro gab, beim esf-programm lag dieses bei 524 millionen euro. "bei den programmen zur territorialen zusammenarbeit, mit dem etwa der grenzüberschreitende infrastrukturausbau gefördert wird, konnte das fördervolumen sogar zu mehr als 100 prozent ausgeschöpft werden", freut sich die abgeordnete abschließend.(schluss) up/jab & 155 & very high & High & Power & NA & NA & 2017-05-16 & 2017 & 2 & POL
Frame & high-very high & National & <500 & -0.7948903 & -0.1706634 & 1.0035124 & 0.5015415 & -0.3119516 & 9.0 & 0.2648514 & -0.9910023 & Payer & Domestic & European & Mixed & Domestic|POL & Neutral\\
Austria & http://text.derstandard.at/2000044244059/EU-Staaten-wollen-EU-Budget-um-sieben-Prozent-kuerzen?ref=rss & 36 & Der Standard & Private/Non-Public & Online and Offline & National & low = CP mentioned more times but NOT important part of story (mainly about others issues) & Institutional bargaining over funding & Balanced & EU + National + Subnational & No myth & NA & NA & NA & NA & NA & NA & NA & NA & Austria & haushalt - eu-staaten wollen eu-budget um sieben prozent kürzen & 2016-09-12 & kohäsionspolitik & brüssel - die eu-staaten wollen beim eu-budget für 2017 einsparungen von 7,04 prozent gegenüber dem budget für das laufende jahr durchsetzen. dies geht aus der gemeinsamen position des eu-ministerrates hervor, welche am montag in brüssel veröffentlicht wurde. demnach sollen sich die tatsächlich geleisteten zahlungen von 143,5 milliarden euro in 2016 auf 133,4 milliarden euro im nächsten jahr reduzieren. in verpflichtungsermächtigungen, bei denen die tatsächliche zahlung später erfolgt, sieht der eu-ministerrat einen leichten anstieg vor, nämlich von 154,5 milliarden euro heuer auf 155,8 milliarden euro in 2017. zusammen mit den beiden außerhalb des budgets angesiedelten instrumenten, dem globalisierungsanpassungsfonds und dem eu-solidaritätsfonds, sollen die zahlungen 2017 bei 133,8 milliarden euro liegen, die verpflichtungsermächtigungen bei 156,4 milliarden euro. die meisten einsparungen wollen die eu-staaten in der eu-kohäsionspolitik für ärmere regionen erzielen. hier habe die technische analyse gezeigt, dass die eu-kommission die tatsächlichen bedürfnisse überschätzt habe, erklärte der rat. außerdem rufen die eu-staaten die anderen eu-institutionen auf, ihr personal bis 2017 um 5 prozent zu reduzieren, so wie dies 2013 vereinbart worden sei. die zahlungen für justiz, sicherheit, asyl, verbraucherschutz, jugend, kultur und den bürgerdialog sollen gegenüber 2016 um rund 800 millionen euro (24,42 prozent) steigen. um 8,89 prozent steigen soll der bereich der strategischen investitionen, eu-programme zur erhöhung der wettbewerbsfähigkeit und das bildungsprogramm "erasmus +". der chefverhandler der slowakischen eu-ratspräsidentschaft, vazil hudak, äußerte sich zuversichtlich, dass mit dem eu-parlament ein "nachhaltiges budget" gefunden werde. die jährlichen eu-haushalte müssen vom eu-ministerrat und vom europaparlament bis spätestens dezember gemeinsam beschlossen werden. (apa, 12.9.2016) & 272 & low & Low & Power & NA & NA & 2016-09-12 & 2016 & 2 & POL
Frame & low-medium & National & <500 & -0.7948903 & -0.1706634 & 1.0035124 & 0.5015415 & -0.3119516 & 9.0 & 0.2648514 & -0.9910023 & Payer & Domestic & European & Mixed & Domestic|POL & Neutral\\
Austria & https://www.ots.at/presseaussendung/OTS\_20171114\_OTS0028/lh-mikl-leitner-und-ungarns-minister-balog-bei-generalversammlung-des-instituts-fuer-den-donauraum-und-mitteleuropa-idm & 84 & APA-OTS & Private/Non-Public & Online only & National & medium = CP is important part of story & Solidarity to poor countries/regions & Positive & National + Other country & No myth & Territorial cooperation & Positive & National + Other country & No myth & NA & NA & NA & NA & Austria & lh mikl-leitner und ungarns minister balog bei generalversammlung des instituts für den donauraum und mitteleuropa (idm) & 2017-11-14 & regionalpolitik & st. pölten (ots/nlk) - am rande der generalversammlung des instituts für den donauraum und mitteleuropa (idm) trafen landeshauptfrau johanna mikl-leitner und der ungarische minister für humanressourcen zoltán balog am gestrigen montag zu einem arbeitsgespräch in st. pölten zusammen. außerdem hielten die beiden bei der idm-generalversammlung ebenso wie univ.-prof. ulrike guérot, leiterin des departments für europapolitik und demokratieforschung an der donau-universität krems, einen vortrag zum thema "eu-regionalpolitik als schlüssel zur europäischen integration". die vorträge und eine anschließende podiumsdiskussion standen nach der begrüßung von idm-vorsitzenden vizekanzler a.d. erhard busek und einer keynote von franz fischler, präsident des europäischen forums alpbach und ehemaliger eu-kommissar, am programm. landeshauptfrau mikl-leitner bedankte sich bei idm-vorsitzenden vizekanzler a.d. busek, dass die generalversammlung des idm zum wiederholten mal in niederösterreich stattfindet. das sei eine "große wertschätzung" und unterstreiche, "dass uns regionalpolitik ein wichtiges thema ist". mikl-leitner betonte, dass das miteinander ganz wichtig sei "zwischen den nationalstaaten und den regionen sowie zwischen den regionen und der europäischen union". "europa ist wichtig und notwendig, gerade in zeiten wie diesen", führte die landeshauptfrau aus, dass durch terrorismus, den brexit und separatistische tendenzen in den vergangenen jahren sehr viel an vertrauen verloren gegangen sei und, dass es wichtig sei, dieses vertrauen wiederzubekommen. das könne man erreichen, "indem wir dem bürger signalisieren, was europa bewegen kann und das kann man am besten mit projekten zeigen", so mikl-leitner. in niederösterreich habe man eine starke regionalpolitik und diese sei nun verlängert worden, betonte die landeshauptfrau: "was im kleinen funktioniert, funktioniert auch im großen." das habe die europäische union bereits gezeigt, erinnerte sie etwa an die fahrrad-und fußgängerbrücke über die march zwischen schloss hof und devínska nová ves und die vielen hochwasserschutzprojekte. "danke für die großartige arbeit", bedankte sich mikl-leitner beim idm. das institut sei "zulieferant von wissen und knowhow in den verschiedensten bereichen." der ungarische minister für humanressourcen balog betonte, dass es wichtig sei, im gespräch zu bleiben. es sei eine illusion gewesen, zu denken, nach dem fall des eisernen vorhangs werde sich schon "alles ändern". "es ist anders geworden und deshalb ist es wichtig, dass wir das gespräch nicht nur fortsetzen, sondern auch neu aufnehmen", so balog. regionalpolitik sei ein bindeglied, eine brücke und eine verstehenshilfe von zentraler politik, gesteuert von brüssel, und der mitgliedstaats-isolierung. "regionalismus ist - um in hauptstädten zu sprechen - ein bindeglied zwischen brüssel und budapest", so balog. es gebe nicht nur eine demografische krise, sondern auch eine identitätskrise und eine krise der wertvorstellungen. man müsse sich auch mit der frage auseinandersetzen: "wie verändert sich unsere gesellschaft, wenn wir arbeitskräfte aus anderen ländern zu uns hereinlassen?" alpbach-präsident fischler stellte in seiner keynote einige überlegungen über die europäische struktur an und spannte dabei einen bogen von den römerverträgen bis heute. "es ist an der zeit, dass sich an der europäischen regional- und strukturpolitik einiges ändert. jetzt ist der richtige zeitpunkt, etwas zu ändern", führte fischler aus, dass die 2020-strategie auslaufe und man noch nicht wisse, was an diese stelle treten werde. es gehe um zentrale zukunftsfragen, die sehr komplex und nicht einfach zu lösen seien, hob er die frage des umgangs mit der überalterung hervor. man müsse sich außerdem wesentlich stärker als bisher mit der frage der integration auseinandersetzen und ein weiteres zentrales thema sei die nachhaltigkeit. univ.-prof. guérot von der donau-universität krems betonte, "dass es an der zeit ist europa neu zu denken". man müsse die regionen strukturell so fördern, damit sie nicht mehr abgehängt werden und das bedeute, dass man auf die richtigen fragen schauen müsse. reelle probleme, die strukturelle vernachlässigung schafften, würden oft gar nicht thematisiert werden. es brauche eine "neue debatte über ein europa, in dem wir leben wollen", betonte guérot. als lösungen führte sie die folgenden beiden an: "verabschieden wir uns von subsidiarität und reden wir über souveränität", damit würde man die politische teilhabe der regionen erhöhen. und: "wir müssen die interregionale zusammenarbeit stärken." in der anschließenden podiumsdiskussion diskutierten der ungarische minister zoltán balog, univ.-prof. ulrike guérot von der donau-universität krems und alpbach-präsident fischler gemeinsam mit idm-geschäftsführer georg krauchenberg zum thema "eu-regionalpolitik als schlüssel zur europäischen integration". nähere informationen: institut für den donauraum und mitteleuropa (idm), telefon 01/319 72 58-0, e-mail idm@idm.at, www.idm.at & 717 & medium & Medium & Values & Socio-Economic & NA & 2017-11-14 & 2017 & 2 & ECO
Frame & low-medium & National & 500-1000 & -0.7948903 & -0.1706634 & 1.0035124 & 0.5015415 & -0.3119516 & 9.0 & 0.2648514 & -0.9910023 & Payer & Domestic & European & Mixed & Domestic|ECO & Positive\\
\addlinespace
Austria & http://derstandard.at/2000055543247/Kaerntner-SW-Umwelttechnik-steigert-Gewinn & 37 & der Standard & Private/Non-Public & Online and Offline & National & very low = CP mentioned once & Infrastructure & Factual & National + Other country & No myth & NA & NA & NA & NA & NA & NA & NA & NA & Austria & kärntner sw umwelttechnik steigert gewinn & 2017-04-07 & kohäsionsfonds & alles über werbung, stellenanzeigen und immobilieninserate heuer werden im öffentlichen bereich schleppende, im privaten sektor aber starke geschäfte erwartet wien/klagenfurt - die in wien börsennotierte sw umwelttechnik ag hat 2016 einen gewinn von 1,4 mio. euro (2015: 0,4 mio.) erzielt. das ergebnis der gewöhnlichen geschäftstätigkeit (egt) stieg im vorjahresvergleich von 0,9 auf 2,1 mio. euro, das betriebsergebnis (ebit) von 3,1 auf 4,1 mio. euro. der umsatz sank von 64,5 auf 60,7 mio. euro. das finanzergebnis blieb mit minus 2,1 mio. euro (-2,2 mio. euro) negativ. die firma mit sitz in klagenfurt ist vor allem im südosteuropäischen raum tätig. der umsatz in ungarn, der im vorjahr um sechs prozent auf 28,3 mio. euro zurückging und rumänien (2016 stabil bei etwas mehr als 17 mio. euro) übersteigt jenen in österreich. auch im heimatland gingen die erlöse im vorjahr zurück - um sechs prozent auf 13,5 mio. euro. im durchschnitt wurden im vorjahr insgesamt 429 mitarbeiter beschäftigt. erholung in ungarn und rumänien "erfreulicher weise führte die wirtschaftliche erholung in ungarn und rumänien zu einer steigenden bautätigkeit in industrie und gewerbe", wurde vorstandsmitglied klaus einfalt am freitag in einer unternehmensmitteilung zitiert. so habe man den gutteil eines rückganges bei öffentlichen projekten kompensiert. die überarbeiteten bestimmungen des eu-kohäsionsfonds hätten voriges jahr zu starken verzögerungen bei öffentlichen ausschreibungen und beim abruf von eu-fördermitteln geführt. dadurch sei der öffentlich finanzierte tiefbausektor insbesondere im zweiten halbjahr hinter den erwartungen zurückgeblieben, so das kärntner unternehmen. als "meilenstein" im jahr 2016 bezeichnet die firma die errichtung eines produktionsstandortes im nordosten rumäniens. um 1,5 mio. euro wird ein vergleichsweise kleines werk gebaut, durch das die marktabdeckung verbessert werde. heuer soll das werk in betrieb gehen. für das laufende jahr erwartet das unternehmer im öffentlichen bereich weiterhin schleppende, im privaten sektor aber starke geschäfte. im bereich infrastruktur werde man die umsätze voraussichtlich halten, so einfalt. im bereich umweltschutz erwartet er aber erst fürs zweite halbjahr 2017 eine erholung. (apa, 7.4.2017) & 334 & very low & Low & Socio-Economic & NA & NA & 2017-04-07 & 2017 & 2 & ECO
Frame & v.low & National & <500 & -0.7948903 & -0.1706634 & 1.0035124 & 0.5015415 & -0.3119516 & 9.0 & 0.2648514 & -0.9910023 & Payer & Domestic & European & Mixed & Domestic|ECO & Neutral\\
Austria & http://orf.at/stories/2298846/ & 76 & newsORF.at & Public & Online and Offline & National & medium = CP is important part of story & Political leverage & Balanced & EU + Other country & No myth & NA & NA & NA & NA & NA & NA & NA & NA & Austria & deutschland: ruf nach "plan b" in flüchtlingskrise & 2015-09-15 & strukturfonds & in deutschland mehrt sich der ruf nach einem konzept, wie es in der flüchtlingskrise weitergehen soll. niedersachsens regierungschef stephan weil (spd) forderte heute die deutsche regierung auf, einen "plan b" für die bewältigung der probleme durch den hohen zustrom an flüchtlingen vorzulegen. "wir sind gut beraten, nicht allzu viele hoffnungen auf europa zu setzen", sagte er gegenüber dem "spiegel" (onlineausgabe). die deutsche regierung müsse die fragen beantworten, wie sich deutschland angesichts des flüchtlingselends einerseits und der verweigerungshaltung europas andererseits verhalten werde. der deutsche innenminister thomas de maiziere forderte heute finanzielle sanktionen für eu-staaten, die eine aufnahme von flüchtlingen und eine entsprechende quote ablehnen. "die länder, die sich verweigern, denen passiert nichts. an ihnen gehen die flüchtlinge eben vorbei", sagte der cdu-politiker heute im "morgenmagazin" des zdf. "deswegen müssen wir, glaube ich, über druckmittel reden." es handle sich oft um länder, die viele strukturmittel von der eu erhielten. de maiziere unterstütze den vorstoß von eu-kommissionspräsident jean-claude juncker, diesen ländern die mittel aus den fonds zu kürzen. die eu-kommission plant jedoch aktuell keine kürzungen von strukturfonds gegenüber jenen eu-staaten, die sich einer verbindlichen verteilungsquote von flüchtlingen widersetzen. das sei derzeit rechtlich nicht möglich und würde eine änderung des mehrjährigen finanzrahmens erfordern, teilte die brüsseler behörde heute mit. & 211 & medium & Medium & Power & NA & NA & 2015-09-15 & 2015 & 1 & POL
Frame & low-medium & National & <500 & -0.7948903 & -0.1706634 & 1.0035124 & 0.5015415 & -0.3119516 & 9.0 & 0.2648514 & -0.9910023 & Payer & European & European & European & European|POL & Neutral\\
Austria & https://diepresse.com/home/wirtschaft/economist/5407016/EUFoerderprojekt-in-Kroatien\_Strabag-klagt-gegen-Antidumping-aus & 69 & Die Presse & Private/Non-Public & Online and Offline & National & very low = CP mentioned once & Infrastructure & Negative & EU + Other country & No myth & NA & NA & NA & NA & NA & NA & NA & NA & Austria & eu-förderprojekt in kroatien: strabag klagt gegen antidumping aus china & 2018-04-17 & kohäsionsfonds & ein chinesisches konsortium erhält den zuschlag für den bau der geostrategisch wichtigen peljesac-brücke in kroatien. die strabag zieht vor kroatiens verwaltungsgericht - und schaltet die eu-kommission ein. profitiert österreich von dem wachsenden chinesischen engagement in südosteuropa? so wollte es wirtschaftskammer präsident christoph leitl noch während des österreichischen staatsbesuchs in china verstanden wissen. österreich sei für china wegen seiner vernetzung nach südosteuropa interessant. doch was der vormarsch chinesischer unternehmen in der region tatsächlich bedeutet, veranschaulicht ein brückenbauprojekt in kroatien: der österreichische konzern strabag verlor das wettrennen gegen chinesisches konsortium unter leitung des staatlichen bauunternehmens china road and bridge corporation. der bauriese geht nun rechtlich gegen die kroatische entscheidung vor - und schaltet die eu-kommission ein. die strabag habe beschwerde vor dem kroatischen verwaltungsgericht eingereicht, heißt es aus dem österreichischen baukonzern gegenüber der "presse". zugleich habe die strabag einen antrag auf einstweilige verfügung gestellt, der die vertragsunterzeichnung des staatlichen auftraggebers hrvatske ceste mit dem chinesischen anbieter verhindern soll. außerdem ruft das unternehmen mit sitz in wien die eu-kommission an: die kroatischen behörden haben bei der vergabe des projektes die anti-dumping-regelungen der eu, die vor billigimporten aus china schützen sollen, nicht eingehalten, heißt es. das chinesische konsortium erhielt den zuschlag für den bau der 2,4 kilometer langen und 55 meter hohen peljesac-brücke ende jänner. die chinesen legten mit knapp 280 millionen euro das günstigste angebot vor. das offert der strabag lag mit 352, 4 millionen euro um mehr als 70 millionen euro über dem der chinesischen firma. auch die italienische astaldi und die türkische ictas waren leer ausgegangen. die strabag argumentiert, dass das niedrige angebot - konkret auch beim verwendeten stahl - nur mit hilfe staatlicher subventionen aus china ermöglicht werde. so kalkulierte das chinesische staatsunternehmen etwa für die stahlseile der schrägseilbrücke 4,5 millionen euro ein. die strabag rechnete mit 9,5 millionen mit mehr als dem doppelten, bestätigt das unternehmen gegenüber der "presse". eine beschwerde der drei erfolglosen bewerber vor der staatlichen kroatischen kommission für die aufsicht öffentlicher beschaffungsverfahren (dkom) blieb erfolglos. die behörde wies das ansuchen ende märz zurück: es gebe keine beweise für staatliche subventionen oder dumping-preise. was die dumpingpreise betreffe, liege die zuständigkeit bei der eu-kommission. das brisante an dem fall: der brückenbau ist zu etwa drei viertel aus eu-geldern finanziert. 357 millionen euro sollen aus dem eu-kohäsionsfonds in das projekt fließen, für das inklusive der begleitenden infrastruktur 420 millionen euro anberaumt wurden. zudem ist die brücke in süddalmatien von geostrategischer bedeutung: sie soll es ermöglichen, vom norden kommend vom kroatischen festland auf die halbinsel peljesac zu gelangen, ohne bosnien und herzegowina zu queren. für kroatien ist die für 2022 geplante fertigstellung somit ein wichtiger schritt, um der schengenzone beizutreten und ein passfreies reisen bis an die südspitze des landes - bis hin zum tourismusmagnet dubrovnik - zu ermöglichen. & 467 & very low & Low & Socio-Economic & NA & NA & 2018-04-17 & 2018 & 3 & ECO
Frame & v.low & National & <500 & -0.7948903 & -0.1706634 & 1.0035124 & 0.5015415 & -0.3119516 & 9.0 & 0.2648514 & -0.9910023 & Payer & European & European & European & European|ECO & Negative\\
Austria & http://www.kleinezeitung.at/wirtschaft/wirtschaftaufmacher/5365802/ & 20 & Kleine Zeitung & Private/Non-Public & Online and Offline & Regional/Local & very low = CP mentioned once & Institutional bargaining over funding & Negative & EU & No myth & NA & NA & NA & NA & NA & NA & NA & NA & Austria & bin zu zehn prozent weniger?: oettinger: einschnitte für landwirte im neuen eu-haushalt & 2018-02-04 & kohäsionsfonds & eu-haushaltskommissar: direktzahlungen pro hektar fläche können degressiv gestaltet werden. auf landwirte und die regionen in europa kommen nach aussagen von eu-haushaltskommissar günther oettinger einschnitte im nächsten eu-haushalt zu. es werde keinen kahlschlag geben, sagte oettinger der "welt am sonntag" laut vorabbericht. aber auch in deutschland müssten sich landwirte und regionen auf kürzungen einstellen. die eu-kommission plane, die mittel für die agrar- und kohäsionsfonds im nächsten mehrjährigen haushalt um jeweils fünf bis zehn prozent zu verringern. im frühjahr beginnen die verhandlungen übe den siebenjährigen finanzrahmen der europäischen union nach 2020. im budget fehlen dann wegen des brexits voraussichtlich bis zu 14 milliarden euro an britischen beiträgen. gleichzeitig will oettinger für einige aufgaben wie verteidigung oder migrationspolitik mehr geld einplanen. im haushalt soll deshalb umgeschichtet werden. zudem sollen die eu-länder zehn bis 20 prozent mehr einzahlen. oettinger äußerte sich konkret auf deutschland bezogen. selbes dürfte allen bauern in der eu bevorstehen, wie bereits mehrfach angedeutet wurde. es gebe bereits vorschläge, wie die kürzungen im agrarsektor gestaltet werden könnten, sagte oettinger. so werde erwogen, die direktzahlungen pro hektar fläche künftig degressiv zu gestalten. die landwirte erhielten dann ab einer bestimmten schwelle weniger finanzielle unterstützung pro hektar. auf deutschland komme insgesamt eine mehrbelastung im einstelligen milliardenbereich zu. oettinger hofft auch auf neue eigenmittel für den eu-haushalt. "wir erwägen auch, dass künftig ein kleiner teil der gewinne, die die europäische zentralbank mit der ausgabe von banknoten macht, als eigenmittel in den eu-haushalt fließt." der kommissar setzt zudem weiter auf seinen vorschlag einer "plastiksteuer" auf verpackungen. mit dem künftigen haushaltsrahmen befassen sich am 23. februar erstmals die eu-staats- und regierungschefs. oettinger will seinen entwurf im mai vorlegen. über einzelheiten dürfte danach monatelang mit den mitgliedsländern und dem europaparlament gestritten werden. & 292 & very low & Low & Power & NA & NA & 2018-02-04 & 2018 & 3 & POL
Frame & v.low & Regional & <500 & -0.7948903 & -0.1706634 & 1.0035124 & 0.5015415 & -0.3119516 & 9.0 & 0.2648514 & -0.9910023 & Payer & European & European & European & European|POL & Negative\\
Austria & https://www.ots.at/presseaussendung/OTS\_20170626\_OTS0106/vana-keine-kuerzung-von-eu-foerderungen-nach-2020 & 89 & APA-OTS & Private/Non-Public & Online only & National & high = CP is most important issue in story (can also cover other issues) & Institutional bargaining over funding & Balanced & EU + National & No myth & Civic participation/collaboration & Balanced & EU + National & No myth & Environment/green/low-carbon & Balanced & EU + National & No myth & Austria & vana: "keine kürzung von eu-förderungen nach 2020" & 2017-06-26 & kohäsionspolitik & brüssel (ots) - die europäische kommission lädt am 26. und 27. juni 2017 zu einem breit angelegten kohäsionsforum in brüssel. dort werden erste weichen für die künftige ausgestaltung von eu-förderungen für die periode nach 2020 gestellt. die grünen im europaparlament haben ein positionspapier mit ihren vorstellungen zur zukunft der kohäsionspolitik erstellt. mit einem gesamtvolumen von mehr als 350 mrd. euro machen fördergelder bisher rund ein drittel des eu-budgets aus. derzeit wird vor allem die kürzung von fördermitteln diskutiert, kritisiert monika vana, grüne europaabgeordnete und regionalpolitische sprecherin der grünen/efa-fraktion: "die eu-kohäsionspolitik ist eine der wichtigsten fördersäulen der europäischen union und für die österreichischen bundesländer und städte besonders bedeutend. deshalb lehnen wir grüne ganz klar jede kürzung dieser fördergelder ab. wir fordern, dass die vorhandenen mittel zielgerichteter eingesetzt werden und vermehrt in klimaschutz und soziales fließen, aber keinesfalls in nuklearenergie oder den aufbau einer verteidigungsunion. auch das partnerschaftsprinzip muss weiter gestärkt werden, denn städte und regionen oder auch ngos brauchen mehr mitsprache. das brexit-votum hat gezeigt, dass sich viele menschen nicht mehr mit der europäischen union identifizieren. eu-fördermittel wirken dem entgegen, denn sie kommen den europäerinnen direkt zu gute und sind in manchen regionen die einzigen öffentlichen investitionen seit ausbruch der krise. dass förderungen einerseits gekürzt und andererseits in junkers efsi-fonds umgeschichtet werden sollten, ist schlecht für europa und schlecht für österreich." & 227 & high & High & Power & Socio-Economic & Socio-Economic & 2017-06-26 & 2017 & 2 & POL
Frame & high-very high & National & <500 & -0.7948903 & -0.1706634 & 1.0035124 & 0.5015415 & -0.3119516 & 9.0 & 0.2648514 & -0.9910023 & Payer & Domestic & European & Mixed & Domestic|POL & Neutral\\
\addlinespace
Austria & https://www.tt.com/politik/europapolitik/14338874/premier-keine-verbindungen-zwischen-mafia-und-regierung & 9 & Tiroler Tageszeitung Online & Private/Non-Public & Online and Offline & Regional/Local & low = CP mentioned more times but NOT important part of story (mainly about others issues) & Political leverage & Negative & EU + Other country & No myth & NA & NA & NA & NA & NA & NA & NA & NA & NA & premier: keine verbindungen zwischen mafia und regierung & 2018-09-05 & NA & NA & 409 & low & Low & Power & NA & NA & 2018-09-05 & 2018 & 3 & POL
Frame & low-medium & Regional & <500 & -0.7948903 & -0.1706634 & 1.0035124 & 0.5015415 & -0.3119516 & 9.0 & 0.2648514 & -0.9910023 & Payer & European & European & European & European|POL & Negative\\
Austria & http://www.tt.com/politik/europapolitik/14049557-91/merkel-fordert-neuausrichtung-der-eu-finanzen.csp & 26 & Tiroler Tageszeitung Online & Private/Non-Public & Online and Offline & Regional/Local & low = CP mentioned more times but NOT important part of story (mainly about others issues) & Political leverage & Balanced & EU + Other country & No myth & NA & NA & NA & NA & NA & NA & NA & NA & Austria & merkel fordert neuausrichtung der eu-finanzen | tiroler tageszeitung online - nachrichten von jetzt! & 2018-02-22 & kohäsionsfonds & berlin, brüssel - kurz vor dem gipfeltreffen zur eu-haushaltspolitik hat die deutsche bundeskanzlerin angela merkel eine harte position deutschlands angekündigt. der austritt großbritanniens aus der europäischen union sei eine chance, die eu-finanzen insgesamt auf den prüfstand zu stellen, sagte merkel am donnerstag in einer regierungserklärung im bundestag in berlin. sie pochte mit blick auf den mittelfristigen eu-finanzrahmen von 2021 bis 2027 auf die finanzierung neuer aufgaben wie den europäischen grenzschutz. zudem will sie finanzhilfen etwa an osteuropäische länder mit deren mitarbeit bei der flüchtlingspolitik verknüpfen. die 27 staats- und regierungschefs der nach dem brexit verbleibenden eu-staaten treffen sich am freitag in brüssel, um über die mittelfristige verteilung von geld in der eu zu sprechen. fast ein dutzend regierungschefs, darunter merkel, wollte bereits am donnerstagabend in brüssel auf einladung des belgischen ministerpräsidenten charles michel zu einem abendessen zusammenkommen. klar ist, dass nach dem brexit ende märz 2019 mehrere milliarden euro im jahr fehlen, weil großbritannien wie deutschland oder österreich mehr geld in den haushalt einzahlt als es zurückbekommt. die eu-kommission hat vorgeschlagen, dass ein teil der bisherigen zahlungen im eu-haushalt gekürzt werden sollen, die anderen eu-staaten aber auch mehr zahlen. cdu/csu und spd haben sich bereit erklärt, dass die bundesrepublik mehr geld nach brüssel überweist. andere staaten wie die niederlande und österreich lehnen dies ab. agrarländer fürchten kürzungen die debatte über den finanzrahmen für die jahre ab 2020 ist im allgemeinen bisher unübersichtlich. streit gibt es auch über mögliche kürzungen der förderung für die landwirtschaft, die vor allem agrarländer fürchten. eu-kommissionspräsident jean-claude juncker fasste die diskussionen in einer rede am donnerstag so zusammen: "bei der zukunft des eu-haushalts gibt es die länder, die nicht mehr zahlen wollen, und jene, die nicht weniger bekommen wollen. zuerst müssen wir uns über die prioritäten einigen, dann können wir über zahlen reden." beschlüsse oder vorfestlegungen für die finanzplanung werden vom gipfel noch nicht erwartet. das gilt auch für das zweite große gipfelthema, nämlich die auswahl des nächsten eu-kommissionspräsidenten nach der europawahl im mai 2019. das europaparlament beschloss kürzlich, nur einen kandidaten zu bestätigen, der vorher bei der europawahl als spitzenkandidat einer partei angetreten ist. der rat der mitgliedsländer lehnt aber einen "automatismus" mehrheitlich ab. einigkeit besteht wohl zumindest darüber, das europaparlament nach dem brexit von heute 751 auf 705 abgeordnete zu verkleinern. merkel sagte, für diesen vorschlag erwarte sie breite unterstützung der staats- und regierungschefs. "wir brauchen einen neuen aufbruch für europa", sagte merkel. 2018 sei das jahr, in dem die weichen für die zukunft gestellt werden müssten. grünen-fraktionschefin katrin göring-eckardt forderte merkel auf, mehr leidenschaft und anstrengung für europa an den tag zu legen. vertreter der afd und der linkspartei forderten dagegen in der debatte, dass der eu-haushalt nach dem brexit verringert werden sollte. fdp-chef christian lindner sprach sich zumindest für eine überprüfung der eu-kohäsionsfonds aus. spd und linkspartei forderten in der eu-debatte eine stärkere konzentration auf soziale themen. finanzhilfe nach verteilung der migranten staffeln die deutsche kanzlerin pochte darauf, dass diese milliardenschweren töpfe weiter unterentwickelten regionen in allen eu-staaten zur verfügung stehen müssten. bei den strukturfonds, von denen ärmere staaten etwa beim bau von infrastrukturprojekten profitieren, forderte sie neue kriterien. dies dürfte zu konflikten mit den betroffenen regierungen führen. streit dürfte es auch bei der von merkel verlangten verknüpfung von finanzhilfen mit der asylpolitik geben. "bei der neuverteilung der strukturfondsmittel müssen wir darauf achten, dass die verteilungskriterien auch das engagement vieler region und kommunen bei der aufnahme und integration von migranten widerspiegeln", sagte merkel. "solidarität kann in der eu keine einbahnstraße sein." hintergrund ist die weigerung osteuropäischer staaten, an der umverteilung von flüchtlingen aus anderen eu-staaten teilzunehmen. schutz der außengrenzen soll ausgeweitet werden merkel forderte zugleich einen effektiven schutz der eu-außengrenze mit einer länge von 14.000 kilometern. die personalausstattung der grenzschutzbehörde frontex müsse massiv verbessert werden. sie pochte zudem darauf, dass bei der reform der eurozone haftung und kontrolle in einer hand bleiben müssten. vor allem müsse die wettbewerbsfähigkeit verbessert werden. ohne einen digitalen binnenmarkt werde es den eu-staaten schwerfallen, international wettbewerbsfähig zu bleiben. laut einer umfrage des allensbach-instituts für die "frankfurter allgemeine zeitung" steigt die zustimmung der deutschen zu einer vertieften eu-integration. dies finden 51 prozent richtig. 14 prozent der wähler halten die eu für überflüssig - bei den afd-wählern sind dies aber 50 prozent. (apa/reuters/dpa) angela merkel deutschland europäischen union österreich & 736 & low & Low & Power & NA & NA & 2018-02-22 & 2018 & 3 & POL
Frame & low-medium & Regional & 500-1000 & -0.7948903 & -0.1706634 & 1.0035124 & 0.5015415 & -0.3119516 & 9.0 & 0.2648514 & -0.9910023 & Payer & European & European & European & European|POL & Neutral\\
Austria & https://www.ots.at/presseaussendung/OTS\_20190327\_OTS0127/kwf-bilanz-2018-bild & 32 & OTS.at & Private/Non-Public & Online only & National & low = CP mentioned more times but NOT important part of story (mainly about others issues) & Economic development & Positive & Subnational & No myth & NA & NA & NA & NA & NA & NA & NA & NA & Austria & kwf bilanz 2018 & 2019-03-27 & europäischer fonds für regionale entwicklung & klagenfurt (ots) - das jahr 2018 stand im fokus der vergabe von eu-mitteln beziehungsweise akquise von entsprechenden projekten, die den herausfordernden kriterien der eu-kofinanzierung entsprechen, so kwf vorstand sandra venus einleitend. die gute konjunktur hat den kwf dabei unterstützt, lag doch das wachstum des jahres 2018 nach den aktuellsten prognosen über dem von 2017. die gesamtaktivitäten des kwf im jahr 2018 umfassten 602 förderfälle (-24 \% gegenüber 2017) mit einem fördervolumen von 31,4 mio. eur (+31 \% zum vorjahr). das damit verbundene investitionsvolumen (projektkosten) belief sich auf 293,6 mio. eur (+69 \% zu 2017) mit dem plan, 764 neue arbeitsplätze zu schaffen und zudem 13.285 bestehende zu sichern. der rückgang der förderfälle beruht auf einer programmänderung beim "kleinunternehmerzuschuss" und der damit zusammenhängenden abwicklung im sogenannten "vereinfachten verfahren". dies bedingt die verschiebung der förderzusagen in die jahre 2019 und 2020. die anzahl der beim kwf eingelangten anträge war im jahr 2018 annähernd gleich hoch wie in den vorjahren. bezeichnend für das jahr 2018 im vergleich zum vorjahr war die hohe anzahl an projekten mit sehr hohem investitionsvolumen, so vorstandskollege erhard juritsch. die betrachtung nach sektoren zeigt, dass auch 2018 das gewerbe bezogen auf die anzahl der förderfälle (57 \%) am stärksten vertreten war und erneut für die meisten neuen arbeitsplätze (391) sorgte. das höchste investitionsvolumen löste 2018 die industrie (115 mio. eur), gefolgt vom tourismus (69 mio. eur) und den bildungs- und außeruniversitären forschungseinrichtungen (sektor "sonstige": 52,9 mio. eur) aus. als größtes industrieprojekt ist jenes der klh massivholz wiesenau gmbh - sie errichtet im bezirk wolfsberg ein hochmodernes brettsperrholzwerk - mit einem investitionsvolumen von 93,6 mio. eur und einem fördervolumen von 6,9 mio. eur hervorzuheben. dem bereich "sonstige" sind unter anderem die förderungen für das build! gründerzentrum-projekt "go2market coachingprogramm" in der höhe von 1,7 mio. eur und das kompetenzzentrum holz-projekt "erweiterung f\&e-infrastruktur reinraum" in der höhe von 0,6 mio. eur zuzuzählen. unternehmensgrößen: entsprechend der kärntner wirtschaftsstruktur richtet sich das förderangebot des kwf primär an kmu. 90 \% der fälle betrafen diese und sie konnten 11,7 mio. eur an fördermittel binden, so kwf kuratoriumsvorsitzender kr werner kruschitz. im abgelaufenen jahr sehr erfreulich sind die zahlen des technologiefonds kärnten, der qualitativ hochwertige und anspruchsvolle projekte fördert, um kärnten als hightech-standort weiterzuentwickeln. mit 16,9 mio. eur konnte das fördervolumen gegenüber dem vorjahr fast verdoppelt und auch im vergleich zum durchschnitt der vergleichsperiode 2014-2017 um 82 \% gesteigert werden. im zuge des kwf wettbewerbs "td|ikt technologische dienstleistungen und informations- und kommunikationstechnologien" wurden 2018 die unternehmen alturos, augmensys, sepin, g\&p schadenlogistik und symvaro ausgezeichnet und gefördert. erfreulich auch die verlängerung des comet (competence centers for excellent technologies) k1 assic (austrian smart systems integration research center) zentrums der ctr carinthian tech research ag aus villach mit einem investitionsvolumen von 20,4 mio. eur und einem kwf förderzuschuss von 2,5 mio. eur. im bundesländervergleich hervorzuheben sind die positiven zahlen kärntens im bereich gründungen und insolvenzen. mit 2.560 unternehmensgründungen (vorläufige zahl) im jahr 2018 konnte kärnten um +2,8 \% zulegen (nur salzburg hatte eine größere dynamik), wohingegen in österreich die gründerzahlen rückläufig waren (-4,5 \%). im jahr 2018 wurden 317 insolvenzen gemeldet, der niedrigste stand seit 2002. mit 109 mio. eur an passiva wurde in kärnten ein neuer tiefststand erreicht, während diese österreichweit stiegen. über die bundesförderstellen aws, ffg, kpc und öht konnten bei 1.741 projekten (+611 projekte gegenüber dem vorjahr) zusätzlich 86,7 mio. eur (+49,5 mio. eur gegenüber 2017) an fördermittel für projekte von kärntner unternehmen akquiriert werden. an eu-fördermittel aus dem iwb | efre-fonds (europäischer fonds für regionale entwicklung) gab es 2018 förderzusagen für 13 projekte in der höhe von 12,5 mio. eur. weitere 17 projekte aus der erfolgten ausschreibung "efre offensive für wachstum und beschäftigung" befinden sich im genehmigungsprozess. die zusammen 30 projekte lösen durch den multiplikatoreffekt der eu kofinanzierung ein gesamtinvestitionsvolumen von 225,7 mio. eur aus. die digitalisierung hat auch vor dem kwf nicht haltgemacht. mit dem im jahr 2018 gestarteten projekt "kwf.digital" wird an einem kundenportal sowie einer elektronischen ver- und bearbeitung der förderabwicklung für unternehmen gearbeitet, um die vielfalt und zunehmende menge an informationen bestmöglich zu verarbeiten und vor allem die abwicklung und kommunikation mit den unternehmen zu verbessern und bestmöglich zu unterstützen. die monetären förderungen werden an bedeutung verlieren, der wirkungsorientierte steuerungsansatz wird die bestimmung der handlungsfelder wesentlich beeinflussen. wettbewerbs- und zukunftsfähigkeit, f\&e, neue unternehmen sowie die wirtschaftsentwicklung definieren als solche die leitlinie und damit das zukünftige tun. 2018 konnten im noch jungen segment der "wirtschaftsentwicklung" bereits 43 förderfälle mit einem fördervolumen von 4,9 mio. eur (+149 \% zum vorjahr) genehmigt werden. die dominierenden handlungsfelder waren "lieferantenentwicklung", "arbeit der zukunft" und "umsetzung innovativer gründungsvorhaben". diese und ähnlich gelagerte zukunftsthemen werden weiter mit dem ziel forciert, die verbindung von wissenschaft und wirtschaft mit breiten interessierten bevölkerungsschichten zu vertiefen. "die südachse" - gemeint sind damit die bundesländer steiermark und kärnten - überwindet nicht nur fiskalpolitische grenzen, sondern ist neben der alpen-adria-region zum synonym für institutionelle und politische zusammenarbeit geworden. durch am weltmarkt führende leitbetriebe, die forschungseinrichtung silicon austria labs oder den silicon alps cluster bestehen beste voraussetzungen, den standort kärnten im bereich der ebs electronic based systems international als relevanten spieler zu etablieren, für unternehmen interessant zu machen und langfristig auszurichten. für das jahr 2019 zeichnen sich weiterhin sehr gute voraussetzungen für anspruchsvolle projekte im investiven bereich quer durch alle sektoren ab. weitere informationen: www.kwf.at/jahr-2018 kostenfreier fotobezug (pressekonferenz 27. märz 2019): fritz press gmbh | t: 0676-3434040 | office@fritzpress.net & 921 & low & Low & Socio-Economic & NA & NA & 2019-03-27 & 2019 & 3 & ECO
Frame & low-medium & National & 500-1000 & -0.7948903 & -0.1706634 & 1.0035124 & 0.5015415 & -0.3119516 & 9.0 & 0.2648514 & -0.9910023 & Payer & Domestic & Domestic & Domestic & Domestic|ECO & Positive\\
Austria & http://text.derstandard.at/2000059831441/EU-Hilfsgelder-Sein-und-Schein?ref=rss & 85 & Der Standard & Private/Non-Public & Online and Offline & National & low = CP mentioned more times but NOT important part of story (mainly about others issues) & Political leverage & Balanced & National + Other country & 6.Does not defend EU values (eg.gender/law/democracy) & NA & NA & NA & NA & NA & NA & NA & NA & Austria & paul lendvai - eu-hilfsgelder: sein und schein & 2017-06-26 & kohäsionsfonds & die ministerpräsidenten der vier visegrád-staaten weigern sich, die beschlossene richtlinie zur umverteilung der flüchtlinge anzuwenden laut der deutschen bundeskanzlerin angela merkel habe ein "geist neuer zuversicht" dank der engen abstimmung mit dem neuen französischen staatspräsidenten" das eu-gipfeltreffen geprägt. bei dem ungewöhnlichen gemeinsamen auftritt mit der kanzlerin vor der presse sagte auch emmanuel macron: "wenn deutschland und frankreich mit einer stimme reden, dann kann europa vorankommen." auch der "europäische trauerakt" am nächsten samstag am sitz des europäischen parlaments in straßburg für den verstorbenen deutschen bundeskanzler helmut kohl symbolisiert den handlungswillen zur stärkung des europäischen geistes. dass auf französischem boden der französische staatspräsident und die deutsche bundeskanzlerin gemeinsam an seinem sarg, zusammen auch mit dem früheren demokratischen us-präsidenten bill clinton und dem ehemaligen sozialistischen regierungschef spaniens, felipe gonzáles, stehen und sprechen werden, das wird bleiben. so wie das bild kohls, hand in hand mit dem französischen präsidenten françois mitterrand im september 1984 bei der gedenkfeier von verdun. während macron betont, dass es darum gehe, "die menschen zu vereinen", um die bürger europas glaubwürdig zu beschützen, gibt es auch politiker, die europäische werte nur so lange kennen, bis sie nicht mit den eigenen machtinteressen kollidieren. das gilt zum beispiel für die ministerpräsidenten der vier visegrád-staaten - polen, slowakei, tschechische republik und ungarn -, die sich weigern, die beschlossene richtlinie zur umverteilung der flüchtlinge anzuwenden. mit der bemerkung "europa ist kein supermarkt" hatte macron beklagt, dass man nicht die hand für eu-hilfsgelder aufhalten und andererseits die solidarität bei der verteilung von flüchtlingen verweigern könne. der ungarische ministerpräsident viktor orbán, unterstützt von seiner polnischen kollegin szydlo, konterte sofort: "der start des jungen neulings ist nicht vielversprechend verlaufen. er hat gestern geglaubt, dass er die mitteleuropäischen länder treten könne. so funktioniert das hier nicht." drei tage vorher hatte der ehemalige reichsverweser miklós horthy, der für den eintritt ungarns in den zweiten weltkrieg und für den mord an 560.000 ungarische juden mitverantwortlich gewesen war, ein besseres öffentliches zeugnis von orbán erhalten: horthy sei ein "außergewöhnlicher staatsmann" gewesen. wie standard-korrespondent thomas mayer berichtete, wird sicherheit in einem umfassenden sinn das hauptthema der eu-politik sein. sicherheit von einem effizienten schutz der außengrenzen bis zur terrorbekämpfung würde weit mehr als drei prozent des eu-budgets derzeit kosten. nicht nur macron, auch die deutsche regierung will die auszahlung der gelder aus dem eu-kohäsionsfonds (63 mrd. euro 2014-2020 für osteuropa, vor allem für polen und ungarn) an neue bedingungen knüpfen. da viele mitgliedstaaten (auch österreich) die aufstockung des eu-budgets ablehnen und der brexit einen ausfall von 13 milliarden euro für die eu jährlich bedeutet, dürften die solidaritätsverweigerer in der flüchtlingskrise größerem druck ausgesetzt werden. (paul lendvai, 26.6.2017) & 446 & low & Low & Power & NA & NA & 2017-06-26 & 2017 & 2 & POL
Frame & low-medium & National & <500 & -0.7948903 & -0.1706634 & 1.0035124 & 0.5015415 & -0.3119516 & 9.0 & 0.2648514 & -0.9910023 & Payer & Domestic & European & Mixed & Domestic|POL & Neutral\\
Austria & https://www.wienerzeitung.at/nachrichten/wirtschaft/international/826372\_EU-Gegner-verschweigen-Folgen.html & 10 & Wiener Zeitung & Private/Non-Public & Online and Offline & Regional/Local & low = CP mentioned more times but NOT important part of story (mainly about others issues) & Institutional bargaining over funding & Negative & EU + Other country & No myth & NA & NA & NA & NA & NA & NA & NA & NA & NA & eu-gegner verschweigen folgen & 2016-06-21 & NA & NA & 956 & low & Low & Power & NA & NA & 2016-06-21 & 2016 & 2 & POL
Frame & low-medium & Regional & 500-1000 & -0.7948903 & -0.1706634 & 1.0035124 & 0.5015415 & -0.3119516 & 9.0 & 0.2648514 & -0.9910023 & Payer & European & European & European & European|POL & Negative\\
\addlinespace
Austria & https://www.ots.at/presseaussendung/OTS\_20180321\_OTS0265/koestinger-oesterreichs-eu-ratsvorsitz-hat-klimaschutz-im-fokus & 90 & APA-OTS & Private/Non-Public & Online only & National & low = CP mentioned more times but NOT important part of story (mainly about others issues) & Bureaucracy and/or delays & Balanced & National & No myth & NA & NA & NA & NA & NA & NA & NA & NA & Austria & köstinger: österreichs eu-ratsvorsitz hat klimaschutz im fokus & 2018-03-21 & kohäsionspolitik & wien (pk) - klimaschutz, eu-energiepaket, gemeinsame agrarpolitik -diese schlagworte sorgten für eine lebhafte diskussion im nationalrat, als der eu-vorhabensbericht des nachhaltigkeitsministeriums auf der tagesordnung der heutigen sitzung stand. während die opposition klare aussagen zur österreichischen klimapolitik vermisst, werten die regierungsfraktionen die umweltpolitischen pläne, nicht zuletzt in hinblick auf den eu-vorsitz österreichs im 2. halbjahr 2018, als richtungsweisend. nachhaltigkeitsministerin elisabeth köstinger unterstrich mit hinweis auf die integrierte klima- und energiestrategie, umwelt- und klimaschutz stünden im mittelpunkt der arbeit ihres hauses. immerhin müssten heuer auf eu-ebene noch viele umweltagenden beschlossen werden. den bericht nahm das plenum mehrheitlich an, abgelehnt wurde ein spö-entschließungsantrag gegen das geplante eu-freihandelsabkommen mit südamerika. maßnahmen gegen klimaerwärmung steigern zur umsetzung der vorgaben aus dem pariser un-klimaabkommen von 2015 müssen die eu-mitgliedsstaaten bis zur nächsten klimakonferenz der vereinten nationen zu jahresende noch eine gemeinsame position finden, schreibt das nachhaltigkeitsministerium im bericht über eu-klimaschutzvorhaben. entscheidende frage bei den verhandlungen ist demnach, wie maßnahmen, die den temperaturanstieg der erderwärmung auf 1,5 grad begrenzen, gesteigert werden können. besonders hinsichtlich der notwendigen reduktion von treibhausgas-emissionen gibt es viel zu tun: gemäß aktuellem reduktionsziel will die eu 40\% an co2 bis 2030 im vergleich zu 1990 einsparen. köstinger sieht dichtes arbeitsprogramm im umweltbereich bundesministerin köstinger umriss das programm der eu-ratspräsidentschaft österreichs im 2. halbjahr 2018. extrem viele vorhaben gelte es zu bewältigen, nannte sie beispielsweise den abschluss der brexit-verhandlungen und den mehrjährigen finanzrahmen der europäischen union. zahlreiche vorhaben beziehungsweise eu-legislativvorschläge gebe es auch in den bereichen umwelt, klima- und energiepolitik, landwirtschaft und fischerei sowie in der kohäsionspolitik. neben vorschlägen der eu-kommission zur regelung der co2-emissionen bei pkws und lkws würden bei den ratssitzungen unter anderem die zukunft der abfallwirtschaft und eine strategie zur vermeidung von plastik sowie eine waldstrategie im zusammenhang mit dem klimaschutz debattiert. das eu-energiepaket ist für köstinger ein wichtiger meilenstein zur umsetzung der klimaziele unter dem dach der energieunion, die ebenfalls unter österreichs vorsitz finalisiert werden soll. im agrarbereich plane die eu-kommission legislativvorschläge im april beziehungsweise mai vorzulegen, so die ministerin, eine der wichtigsten fragen für das eu-agrarmodell der zukunft sei die finanzierung. in der eu-kohäsionspolitik werde eine thematische ausrichtung im sinne einer effizienten mittelverwendung bei vereinfachter verwaltung angestrebt. opposition verlangt mehr einsatz... klaus uwe feichtinger (spö) übersetzte die ziele der heimischen umweltpolitik mit "more of the same". besonders hinsichtlich budgetierung vermisst feichtinger genaue angaben über zukunftsprojekte, zumal auch bei der budgetrede von finanzminister hartwig löger und bei kanzler sebastian kurz' schwerpunktsetzung zur eu-ratspräsidentschaft klimafragen keine rolle gespielt hätten. dabei brauche es für eine erfolgreiche integrierte klima- und energiestrategie unbedingt ausreichend budgetmittel. eine konkrete österreichische positionierung in der umwelt- und klimapolitik erkennt pamela rendi-wagner (spö) nicht, verwies sie auf den bericht des ministeriums über diesbezügliche eu-vorhaben. sie verlangt hier mehr transparenz, das schlagwort nachhaltigkeit allein reiche nicht aus. "nachhaltigkeit bedeutet, dass wir künftigen generationen einen lebenswerten planeten hinterlassen", verdeutlichte die ehemalige gesundheitsministerin und kritisierte, die regierung denke vorrangig an die vorteile für wirtschaft und landwirtschaft. neos-abgeordneter michael bernhard meinte ebenfalls, die zuletzt im umweltausschuss behandelten berichte des ressorts ließen kaum schlüsse auf die haltung österreichs in der umweltpolitik zu. "in den letzten jahren war die umweltpolitik in der geiselhaft des bauernbundes", erboste sich bernhard über mangelhafte maßnahmen zur reduktion der co2-emmissionen, zur zurückdrängung von ölheizungen oder für mehr thermische sanierung. für die liste pilz zitierte martha bißmann aus dem vorhabensbericht "atomkraft ist keine antwort auf den klimawandel" und rief dazu auf, eu-subventionen für nuklearenergie zu streichen. entsprechende zahlungen an den atommeiler hinkley point c in großbritannien etwa böten österreich die gelegenheit, sich für einen gesamteuropäischen atomausstieg einzusetzen. energiesparen und erneuerbare energien müssten forciert werden, 45\% mehr erneuerbare energieträger seien nötig, um die erderwärmung tatsächlich auf weniger als 2°c zu beschränken, unterstrich bißmann. sie erwartet von den abgeordneten, beim klimaschutz über parteipolitische grenzen hinweg an einem strang zu ziehen. vorreiter solle österreich auch beim verbot von mikroplastik sein. große bedenken äußerte schließlich doris margreiter (spö) in bezug auf das anvisierte freihandelsabkommen mit der südamerikanischen wirtschaftsgemeinschaft mercosur, das die eu-kommission aktuell verhandelt, um importe über rohstoffe hinaus auszuweiten. österreichs standards zum schutz der umwelt und des ländlichen raums würden dabei aufs spiel gesetzt, warnt margreiter in einem eigenen antrag vor der unterzeichnung des abkommens. als gründe für ihre ablehnung nennt sie praktiken wie pestizidnutzung und brandrodung in der südamerikanischen landwirtschaft, hygienemängel der produkte und nicht zuletzt die erhöhung der einfuhrkontingente in die eu, worunter die heimische landwirtschaft massiv leiden würde. "es braucht fairen und gerechten handel", appellierte die sozialdemokratin, die durch das freihandelsabkommen einen schaden für den standort österreich befürchtet. regierungsparteien sehen zukunftsmodell für umwelt und wirtschaft klar positiv werteten hingegen die rednerinnen der övp die umweltpolitischen pläne der regierung. budgetäre angaben im bereich umwelt hätten vor der präsentation des budgets gar nicht gegeben werden können, richtete johannes schmuckenschlager der spö aus, wobei er anmerkte, eine mittelerhöhung garantiere keine automatische verbesserung: "mehr geld ist nicht mehr effizienz". anlässlich des heutigen internationalen tag des waldes würdigte schmuckenschlager den hohen mehrwert der waldbewirtschaftung für klimaschutz und wirtschaft. nur mit einer ökonomisch, ökologisch und sozial verträglichen herangehensweise ließen sich die klimaziele realisieren, betonte schmuckenschlager, der sich außerdem nachdrücklich gegen den einsatz von palmöl, zum beispiel in biodiesel, aussprach. ein klares bekenntnis zum klimaschutz und zur ressourcenschonung sowie gegen atomkraft liest martina diesner-wais (övp) aus dem bericht des nachhaltigkeitsministeriums. "wir in österreich sind im bereich der erneuerbaren energien vorreiter", erinnerte sie außerdem an den 70\%igen anteil dieser energieformen bei der stromproduktion. bis 2030 wolle die regierung den anteil auf 100\% steigern. der schutz von wasser werde den ratsvorsitz ebenso prägen wie der erhalt der biodiversität, für die man eine eigene strategie ausarbeite, kündigte diesner-wais an. maßgeblich sei in all diesen bereichen die landwirtschaft, die schon deswegen unterstützt werden müsse. aus dem breiten themenspektrum des berichts griff ihr fraktionskollege josef smolle folglich die landwirtschaft heraus. die eu stehe in der agrarpolitik vor intensiven verhandlungen über die förderungen. als "wesentlicher träger des umweltschutzes", der wiederum die grundlage für erfolgreichen tourismus bilde, spielten landwirtschaftliche betriebe in österreich eine herausragende rolle, sagte smolle, der wie johann rädler (övp) drohende kürzungen bei eu-agrarförderungen dezidiert ablehnt. in sachen nachhaltigkeit gibt es laut rädler im energiebereich und im verkehrssektor noch "luft nach oben". den einsatz der regierung gegen atomkraftwerke in der eu begrüßte namens der fpö walter rauch. es gehe für österreich darum, beim kampf gegen nuklearenergie "ein zeichen zu setzen". das arbeitsprogramm der regierung zur eu-ratspräsidentschaft nannte rauch hinsichtlich klima-und umweltschutz sowie landwirtschaft "ambitioniert", aber wichtig. gerade bei der kreislaufwirtschaft sieht er viel zukunftspotential, ebenso im schutz der biodiversität. der naturschutz sollte jedenfalls in mehreren politikbereichen forciert werde, schenkte er in diesem zusammenhang dem schutz der ressource wasser besonderes augenmerk. (fortsetzung nationalrat) rei & 1133 & low & Low & Governance & NA & NA & 2018-03-21 & 2018 & 3 & POL
Frame & low-medium & National & +1000 & -0.7948903 & -0.1706634 & 1.0035124 & 0.5015415 & -0.3119516 & 9.0 & 0.2648514 & -0.9910023 & Payer & Domestic & Domestic & Domestic & Domestic|POL & Neutral\\
Austria & http://text.derstandard.at/2000017804708/Griechenland-einst-geliebt-und-gebraucht?ref=rss & 73 & Der Standard & Private/Non-Public & Online and Offline & National & very low = CP mentioned once & Solidarity to poor countries/regions & Positive & EU & No myth & NA & NA & NA & NA & NA & NA & NA & NA & Austria & essay - griechenland, einst geliebt und gebraucht & 2015-06-21 & strukturfonds & griechenland wurde aus politischen gründen in den euroraum geholt. es lebte lange seine träume aus, trickste mit zahlen, war ein beliebtes ziel der bürger. die weltwirtschaftskrise 2008 traf das land am meisten. seither läuft eine schwierige rettungsaktion - 1 foto als der euro 1998 nach europa kam, hatte griechenland kein problem mit den partnern. für millionen europäer war die "wiege der abendländischen kultur" das gegenbild zu den sorgen des alltags: ein sehnsuchtsland. mit traumhaften inseln und stränden. besonders galt das für die deutschen und die österreicher, menschen aus den zentralen wohlstandsländern der union. der italienurlaub war schon etwas fad geworden. aber griechenland, erst seit 1974 eine demokratie mit militärdiktaturvergangenheit, seit 1981 eu-mitglied, hatte stark aufgeholt - aus eu-töpfen der agrarpolitik wie strukturfonds großzügig subventioniert. ein glück. an den küsten ex-jugoslawiens war es nicht so gemütlich. da tobten seit 1991 vier bürgerkriege. die nachtseite des kontinents. 1999 führte die nato den ersten heißen krieg ihrer geschichte gegen serbien wegen des kosovo, als hunderttausende albaner auf der flucht waren. griechenland hingegen war ein ort der schönen stabilität geworden. udo jürgens hatte es mit griechischer wein vorausgesungen, essen und trinken "beim griechen", die sehnsucht, die liebe, der sirtaki. die austropopper sts zogen 1985 nach, besangen den traum, dass wir "irgendwann dann durt" bleiben, auf der insel. das waren schlager, die anfang 2002, als der euro in elf eu-staaten nicht nur buchgeld war, sondern in geldscheinen aus bankomaten sprudelte, oft gespielt wurden. heute schimpft die mehrheit auf "die griechen", die bild sogar auf "die faulen griechen". wie konnte das geschehen? griechenland war nicht von anfang an dabei beim euro. der beschluss fiel verzögert im jahr 2000 bei einem eu-gipfel in portugal, obwohl das land die haushaltskriterien gemäß maastricht-vertrag nicht erfüllte. die staatsschuld war mehr als doppelt so hoch wie vorgesehen. aber das nahm man damals nicht so genau, weil andere - politische - faktoren wichtiger genommen wurden. auch belgien war mit mehr als 100 prozent der wirtschaftsleistung verschuldet, so wie italien auch. unter der hand hieß es: wenn man das hohe ausmaß an schwarzarbeit berücksichtige, sei deren wirtschaftskraft eh deutlich höher, die schulden also kleiner. deutschland, seit ende 1998 von einer rot-grünen koalition unter gerhard schröder regiert, lenkte ein. sein land wurde vom economist wegen schwachen wachstums als "kranker mann europas" tituliert, lag hinter frankreich, strukturell unbeweglich. gemeinsam brachen sie 2003 erstmals den eurostabilitätspakt. erst nach schröders "hartz-4-reform" ging es nach 2005 aufwärts. kanzlerin angela merkel hat all das geerbt. ein anderes argument, drachme durch euro zu ersetzen, war solider. zu der zeit zeichnete sich ab, dass die eu-erweiterung um die von sowjetdiktatur befreiten osteuropäischen länder breit ausfallen sollte. dafür brauchte es einen einstimmigen beschluss. die südländer fürchteten, dass die eu-subventionen für sie kleiner werden würden. und sie wollten vom "kern" der union nicht abgehängt werden. griechenland machte zusätzlich druck: zustimmung zur erweiterung nur, wenn auch die geteilte insel zypern eu-mitglied werde. euro statt drachme war auch ein sicherheitspolitischer deal, 2004 realisiert. die eu-kommission warnte damals schon davor, dass griechenland bei den gemeldeten zahlen trickste. den antrag, eurostat einblick in die bücher zu gewähren, lehnten die finanzminister ab. es war eine euphorische zeit. griechenland wurde fußballeuropameister. es richtete im sommer 2004 prächtige olympische spiele in athen aus. der staat gab milliarden und abermilliarden aus. als teil der eurozone bekam das land großzügig kredite. die zahl der beamten stieg ständig, noch mehr ihre gehälter - und die militärausgaben. strukturreformen wie privatisierungen unterblieben. der "klientelismus" blieb. die weltwirtschaftskrise 2008 bereitete der traumwelt ein ende. von den schlecht aufgestellten eurostaaten wurde griechenland als erstes hart getroffen. im mai 2010 stand es vor der pleite. mit eigener währung wäre es wohl sofort in die pleite geschickt worden. im verbund der eurozone war das jedoch nicht so einfach zu lösen. in den eu-verträgen war ein solcher fall nicht vorgesehen. sie verbieten den eurostaaten strikt, schulden von griechenland bei banken direkt zu übernehmen. die deutsche regierung muss damit rechnen, dass klagen beim bundesgerichtshof in karlsruhe gegen einen "schuldenrauskauf" als grundgesetzwidrig erkannt werden. zum anderen musste ein "überspringen" auf andere "wackel-eurostaaten" verhindert werden. die märkte spekulierten gegen ganze länder. man behalf sich zunächst mit bilateralen krediten und garantien von knapp 110 milliarden euro von eurostaaten (siehe grafik), teils vom währungsfonds (iwf). ein jahr später wurden aus dem gemeinsamen fonds (efsf) 143 milliarden euro zugesagt. die letzte tranche ist noch offen. die privaten gläubiger steuerten bei einem schuldenschnitt rund 55 milliarden bei. ein großteil des geldes ging in die stabilisierung der banken, etwa ein viertel an die regierung. mit mühsamen kompromissen tastete sich die euro zone an das system eines "eurorettungsmechanismus" (esm) heran. die hilfsformel lautet: billige kredite, wenn reformen gemacht werden, bis das vertrauen der märkte zurückkehrt. bei portugal oder irland hat das funktioniert. auch sie haben milliardenkredite bekommen. aber die bürger der eurozone grollten den portugiesen und iren kaum. bei griechenland lief es anders. alle regierungen seit premier giorgos papan dreou, den merkel 2011 zum rücktritt zwang, haben die versprochenen reformen nie geliefert. griechenland wurde zum sonderfall - auch einer enttäuschten liebe. (thomas mayer, 21.6.2015) & 846 & very low & Low & Values & NA & NA & 2015-06-21 & 2015 & 1 & ECO
Frame & v.low & National & 500-1000 & -0.7948903 & -0.1706634 & 1.0035124 & 0.5015415 & -0.3119516 & 9.0 & 0.2648514 & -0.9910023 & Payer & European & European & European & European|ECO & Positive\\
Austria & https://text.derstandard.at/2000073601862/EU-Kommissar-Bauern-werden-weniger-Mittel-bekommen?ref=rss & 31 & Der Standard & Private/Non-Public & Online and Offline & National & very low = CP mentioned once & Institutional bargaining over funding & Negative & Other country & No myth & NA & NA & NA & NA & NA & NA & NA & NA & Austria & landwirtschaft - eu-kommissar: bauern werden weniger mittel bekommen & 2018-02-04 & kohäsionsfonds & deutsche bauern und bundesländer müssten sich ab 2021 auf weniger geld aus brüssel einstellen - 1 foto berlin - die selbe sorge geht auch bei heimischen bauern um, auf deutschland bezogen hat sich eu-haushaltskommissar günter oettinger nun konkret geäußert: "auch" deutsche bauern und die bundesländer müssen sich nach seinen worten auf weniger geld aus brüssel im nächsten finanzrahmen ab 2021 einstellen. selbes dürfte allen bauern in eu bevorstehen, wie bereits mehrfach angedeutet wurde. beim neuen mehrjährigen eu-haushalt werde es zwar "keinen kahlschlag geben, wie einige befürchten", sagte oettinger der "welt am sonntag". "aber auch in deutschland werden sich landwirte und regionen auf finanzielle kürzungen einstellen müssen." die eu-kommission plane, "die agrar- und kohäsionsfonds im neuen mehrjährigen haushalt jeweils um fünf bis zehn prozent zu verkleinern", sagte oettinger. demnach gibt es bereits überlegungen, wie die kürzungen im landwirtschaftssektor aussehen könnten: "in der agrarpolitik erwägen wir, die direktzahlungen pro hektar fläche künftig degressiv zu gestalten. das bedeutet: ab einer gewissen fläche gibt es dann pro hektar weniger finanzielle unterstützung als für den ersten hektar." die eu-kommission will im mai einen konkreten vorschlag zum nächsten finanzrahmen vorlegen, der ab 2021 gilt. die landwirtschaft und die strukturförderung machen zusammen bisher fast drei viertel aller eu-ausgaben aus. im frühjahr beginnen die verhandlungen über den nächsten siebenjährigen finanzrahmen der europäischen union. im budget fehlen dann wegen des brexits voraussichtlich bis zu 14 milliarden euro an britischen beiträgen. gleichzeitig will oettinger für einige aufgaben wie verteidigung oder migrationspolitik mehr geld einplanen. im haushalt soll deshalb umgeschichtet werden. zudem sollen die eu-länder zehn bis 20 prozent mehr einzahlen. in der "welt am sonntag" reagierte oettinger auch auf die drohung von us-präsident donald trump, möglicherweise strafzölle auf produkte aus europa zu erheben: "wenn europäische exporteure zölle zahlen müssen, wird eine zweibahnstraße daraus. dann werden auch us-exporteure bei uns zölle zahlen müssen", sagte der eu-kommissar. "wer das instrument zückt, muss wissen, dass wir es auch haben. und der europäische markt ist mindestens so groß wie der amerikanische. (apa, 4.2.2018) & 338 & very low & Low & Power & NA & NA & 2018-02-04 & 2018 & 3 & POL
Frame & v.low & National & <500 & -0.7948903 & -0.1706634 & 1.0035124 & 0.5015415 & -0.3119516 & 9.0 & 0.2648514 & -0.9910023 & Payer & European & European & European & European|POL & Negative\\
Austria & https://www.ots.at/presseaussendung/OTS\_20170328\_OTS0184/bio-crime-projekt-ist-grenzueberschreitende-kampfansage-gegen-illegalen-heimtierhandel & 1 & OTS.at & Private/Non-Public & Online only & National & very low = CP mentioned once & Public services & Positive & EU + National + Subnational & No myth & Territorial cooperation & Positive & EU + National + Subnational & No myth & NA & NA & NA & NA & NA & bio crime-projekt ist grenzüberschreitende kampfansage gegen illegalen heimtierhandel & 2017-03-28 & NA & NA & 860 & very low & Low & Socio-Economic & Socio-Economic & NA & 2017-03-28 & 2017 & 2 & ECO
Frame & v.low & National & 500-1000 & -0.7948903 & -0.1706634 & 1.0035124 & 0.5015415 & -0.3119516 & 9.0 & 0.2648514 & -0.9910023 & Payer & Domestic & European & Mixed & Domestic|ECO & Positive\\
Austria & https://www.tt.com/wirtschaft/wirtschaftspolitik/10743296-91/eu-rechnungshof-kritisiert-schlamperei-im-eu-budget.csp & 98 & Tiroler Tageszeitung Online & Private/Non-Public & Online and Offline & Regional/Local & high = CP is most important issue in story (can also cover other issues) & Mismanagement & Negative & EU + National + Subnational & 8.Mismanaged & NA & NA & NA & NA & NA & NA & NA & NA & Austria & eu-rechnungshof kritisiert schlamperei im eu-budget & 2015-11-10 & europäischer fonds für regionale entwicklung & brüssel - der eu-rechnungshof hat wie in den vergangenen jahren milliardenfehler durch schlampereien und unregelmäßigkeiten im eu-budget gerügt. für den gesamten eu-haushalt 2014 von 142,5 milliarden euro oder rund 300 euro je bürger ermittelte der eu-rechnungshof in seinem am dienstag vorgelegten jahresbericht eine fehlerquote von 4,4 prozent, gegenüber 4,5 prozent im jahr 2013. der rechnungshof betont, dass die fehlerquote nicht mit betrug, ineffizienz oder verschwendung gleichzusetzen sei. das geld sei aber aus dem eu-budget nicht im einklang mit den eu-regeln ausbezahlt worden. der europäische rechnungshof (eurh) rief in seinem jahresbericht zu einem völlig neuen ansatz für die verwaltung von investitionen und ausgaben der eu auf. es sind erhebliche veränderungen notwendig - und zwar auf der ebene sämtlicher für die verwaltung von eu-mitteln verantwortlicher akteure. mehr flexibilität im krisenfall notwendig in seiner rede zur vorstellung des berichts vor dem europäischen parlament wies der präsident des eurh, vitor caldeira, darauf hin, dass die eu-entscheidungsträger den haushalt besser an die langfristigen strategischen prioritäten der eu anpassen und für mehr flexibilität im krisenfall sorgen müssen. die eu müsse daher "ihr geld besser investieren", forderte caldeira. außerdem müssen die gesetzgeber der eu sicherstellen, dass bei ausgabenregelungen eindeutig festgelegt wird, welche ergebnisse erzielt werden sollen und welche risiken hinnehmbar sind. die mittelbewirtschafter müssen sicherstellen, dass die ausgaben im einklang mit den vorschriften getätigt werden und die geplanten ergebnisse erzielen. österreich weist nach einer analyse des eu-rechnungshofes mehr fehler bei eu-projekten auf als der europäische durchschnitt. so wurden laut oskar herics, österreichs vertreter am europäischen rechnungshof, für 2014 insgesamt 18 transaktionen geprüft. "jede zweite war fehlerhaft", sagte herics am dienstag in brüssel. österreich mit mehr fehler als der europäische durchschnitt bis auf das burgenland wurden im vorjahr alle bundesländer geprüft. "fehler gab es in sieben bundesländern", erklärte herics. damit gebe es in österreich quasi eine "sehr gute flächendeckung", konstatierte er. lediglich in niederösterreich fanden die prüfer keine fehlerhaften transaktionen. zehn überprüfungen ausgewählter transaktionen für eu-förderungen betrafen den bereich landwirtschaft. zwei davon wiesen fehler auf, diese betrafen unter anderem flächenbezogene beihilferegelungen. fehlende belege, falsch berechnete kosten im bereich sozialfonds wurden sieben transaktionen geprüft, davon waren gleich sechs fehlerhaft. dabei ging es etwa um beschäftigungsprojekte mit einer fördersumme von bis zu 900.000 euro aus eu-fonds. hier fehlten belege, kosten wurden falsch berechnet, es kam zu unregelmäßigkeiten bei der auftragsvergabe, erklärte herics. eine weitere fehlerhafte transaktion betraf den europäischer fonds für regionale entwicklung, hier wurden zahlungen zu spät geleistet. laut herics ist in österreich der anteil der fehler "wesentlich höher als sein anteil am budget", insgesamt seien mit diesem rechnungshofbericht die ziele der eu "massiv gefährdet". "die wirksamkeit der kontrollsysteme muss erhöht und die haushaltsführung verbessert werden", forderte der experte. schlecht schneidet österreich auch im bereich der öffentlichen ausschreibungen ab. während im jahr 2012 insgesamt 3,1 prozent des bruttoinlandsprodukts eu-weit ausgeschrieben wurden, waren es hierzulande lediglich 1,5 prozent. "in österreich wird nicht umfassend genug ausgeschrieben", sagte herics. planlos eingesetzt die övp-europaabgeordnete claudia schmidt kritisierte die eu-mitgliedstaaten für "zu viele fehler im umgang mit eu-geldern". schmidt begrüßte, dass der rechnungshofs im haushaltsjahr 2014 neben der fehlerquote erstmals auch geprüft hat, ob die finanzierten projekte überhaupt ihre politikziele erreichen. demnach erreicht ein viertel der projekte keines der politikziele. "der rechnungshof bestätigt unsere befürchtung, dass die mitgliedstaaten gerade in den strukturfonds geld planlos einsetzen. (apa) kommentieren kommentar schreiben schlagworte brüssel budget europäischen rechnungshof europäischen union haushalt der europäischen union landwirtschaft schlamperei österrreich & 579 & high & High & Governance & NA & NA & 2015-11-10 & 2015 & 1 & POL
Frame & high-very high & Regional & 500-1000 & -0.7948903 & -0.1706634 & 1.0035124 & 0.5015415 & -0.3119516 & 9.0 & 0.2648514 & -0.9910023 & Payer & Domestic & European & Mixed & Domestic|POL & Negative\\
\addlinespace
Austria & http://www.oe-journal.at/Aktuelles/\%212016/0416/W2/21504ApkFoerderungen.htm & 40 & oe-journal.at & Private/Non-Public & Online only & National & very high = CP is most important issue + CP is mentioned in title/headline & Mismanagement & Negative & EU + National & No myth & Institutional bargaining over funding & Negative & National + Other country & No myth & NA & NA & NA & NA & Austria & konstant hohe fehlerquote bei eu-förderungen & 2016-04-18 & kohäsionsfonds & rechnungshofausschuss debattiert europäische aspekte der finanzkontrolle brüssel/wien (pk) - die zahlungen aus eu-förderprogrammen sind nach wie vor stark mit fehlern behaftet. oskar heric, der als österreichisches mitglied des europäischen rechnungshofs am 14.04. im rechnungshofausschuss gemeinsam mit seinen mitarbeiterinnen margit spindelegger und thomas obermayer den abgeordneten rede und antwort stand, bezifferte die für 2014 geschätzte fehlerquote mit 4,4\% und leitete daraus die forderung nach einer grundlegenden veränderung des eu-finanzmanagements - von der vergabe bis zur kontrolle - ab. betroffen von den unregelmäßigkeiten ist auch österreich, das vor allem im bereich der mittelauszahlung aus dem struktur- und kohäsionsfonds im negativen europäischen spitzenfeld liegt. im rahmen der sitzung, die vor allem der europäischen perspektive der prüftätigkeit gewidmet war, beklagten die abgeordneten auch prüfungslücken, etwa im zusammenhang mit der ezb und dem europäischen stabilitätsmechanismus (esm). heric, aber auch rechnungshofpräsident josef moser bestätigten kontrolldefizite, sahen in dieser frage aber die nationalen parlamente am zug. ein antrag der fpö auf prüfung von eu-fördermitteln, die direkt an förderungsempfänger ausbezahlt werden, fand keine mehrheit. struktur- und kohäsionsfonds: österreich musste 1,5\% der förderungen zurückzahlen die ausgaben sind nicht immer im einklang mit den eu-vorschriften erfolgt, das eu-finanzmanagement erfüllt noch nicht die ansprüche der union und ihrer mitgliedstaaten, skizzierte oskar heric den vom europäischen rechnungshof festgestellten grundtatbestand. am höchsten war die fehlerquote mit 5,7\% im bereich der regional- und sozialfonds, knapp dahinter folgte der bereich wachstum und beschäftigung mit 5,6\%. auszahlungen beim programm für ländliche entwicklung wiesen eine fehlerhäufigkeit von 3,6\% auf. wie heric präzisierte, bestanden die fehler im wesentlichen in der erfassung von nicht förderfähigen kosten, der verletzung von vergabevorschriften oder etwa der angabe von falschen flächengrößen bei landwirtschaftlichen förderungen. von den fehlern war auch österreich nicht ausgenommen. so waren von den 18 transaktionen, die im jahr 2014 stichprobenartig geprüft wurden, neun fehlerhaft. eine besonders hohe fehlerquote wiesen etwa zahlungen im bereich des struktur- und kohäsionsfonds auf, wo österreich mit 64\% im eu-vergleich den drittletzten platz einnimmt. besser schnitt österreich im bereich der förderungen aus dem programm für den ländlichen raum ab. hier betrug die fehlerquote bei stichproben 39\% und lag damit deutlich unter dem eu-durchschnitt von 47\%. insgesamt musste österreich im zeitraum von 2007 bis 2013 1,5\% der erhaltenen eu-zahlungen im bereich struktur- und kohäsionsfonds zurückzahlen. unbehagen über komplexe förderrichtlinien in der debatte wiesen die abgeordneten auf einen möglichen zusammenhang zwischen der förderbürokratie und der fehlerquote hin. so beklagten etwas spö-agrarsprecher erwin preiner und bruno rossmann von den grünen sowie jessi lintl (f) die hohe komplexität der vorgaben. man frage sich, "ob man da überhaupt noch mitmachen soll", brachte övp-abgeordneter johann singer das unbehagen zahlreicher förderungswerberinnen auf den punkt. vereinfachung ist ein thema in der eu, versicherte oskar heric, wies aber gleichzeitig auf die notwendige balance zwischen einem erleichterten zugang zu eu-förderungen auf der einen seite und der entfaltung der gewünschte wirkung der ausgezahlten mittel auf der anderen seite hin. opposition sieht prüfungslücken bei esm und eu-förderungen neos-mandatarin claudia gamon lenkte den blick auf die bankenaufsicht und ortete eine prüfungslücke bei der europäischen zentralbank ezb, wobei sie bemerkte, hier werde macht ohne kontrolle ausgeübt. diesen auch von bruno rossmann (g) aufgegriffenen themenbereich nahm rechnungshofpräsident josef moser zum anlass für seine anregung, die europäische kontrollarchitektur auf neue beine zu stellen. es müsse jedenfalls verhindert werden, dass im zuge der verschiebung von kompetenzen auf die eu-ebene oder etwas bei der abwanderung von systemrelevanten banken zur europäischen zentralbank prüfungslücken entstehen, mahnte er. gleiches gelte im zusammenhang mit direktzahlungen, die derzeit nur im falle einer geteilten mittelverantwortung der kontrolle durch die nationalen rechnungshöfe unterliegen. auch oskar heric sprach das problem der prüfungslücken an und verwies in diesem zusammenhang auf die bestrebungen einzelner eu-länder in richtung einer gemeinsamen prüfung des stabilitätsmechanismus. eu-haushalt 2013 - am meisten geld bekamen polen und ungarn das anliegen einer größeren transparenz bei den zahlungen der eu an österreich, bei der einordnung österreichs in den eu?haushalt und bei der verwendung von eu?mitteln in österreich bestimmte die ausschussdebatte auch bei der einstimmigen vertagung des eu-finanzberichts 2013 ( iii-204 d.b.). österreich leistete im jahr 2013 3,191 mrd. € an eu-beiträgen und verbuchte 1,862 mrd. € an rückflüssen, teilte rechnungshofpräsident josef moser den abgeordneten bei seiner präsentation des umfangreichen zahlenwerks mit. österreichs nettobeitrag lag 2013 somit bei 1,329 mrd. € und damit über dem durchschnitt von 795,67 mio. € in den jahren 2007 bis 2013, rechneten die prüfer den abgeordneten vor. von 2007 bis ende 2013 nutzte österreich mittel des europäischen landwirtschaftsfonds für die entwicklung des ländlichen raums (eler) zu 89,5\%, des europäischen fonds für regionale entwicklung (efre) zu 52,9\% und des europäischen sozialfonds (esf) zu 85,6\%. österreich nützte die ihm in beiden strukturfonds zugewiesenen mittel zu 67,1\%. im auslaufzeitraum bis ende 2015 kann österreich die ihm zugewiesenen eler- und esf-mittel vollständig auszuschöpfen, schreibt der rechnungshof, für nicht realistisch halten die prüfer hingegen eine vollständige ausschöpfung der efre-mittel. die größten rückflüsse an mitteln aus dem eu-haushalt erzielten in absoluten zahlen polen und relativ zum bnp ungarn, die geringsten schweden, dänemark und deutschland. in der reihe der länder mit den absolut größten rückflüssen nahm österreich den 18. platz ein. 17 eu-mitgliedstaaten erhielten mehr und 10 weniger rückflüsse als österreich. auf der liste der 11 nettozahler der eu lag österreich 2013 an 9. stelle, geht aus dem rechnungshofbericht hervor. der eu-beitrag österreichs wird 2013 zu 73,6\% vom bund, zu 22,5\% von den ländern und zu 3,9\% von den gemeinden getragen. zu der seit 2011 diskutierten reform des eu-eigenmittelsystems wurde eine interinstitutionelle arbeitsgruppe eingesetzt. für 2016 ist eine interparlamentarische konferenz geplant. die finanztransaktionssteuer soll im rahmen der verstärkten zusammenarbeit eingeführt werden, berichtete der rh-präsident. kontrolllücken ortete rechnungshofpräsident moser bei den direkt-förderungen der eu, die 2013 245 mio. € ausmachten. davon gingen 48,7\% an öffentliche, 50,8\% an private einrichtungen und 0,5\% an natürliche personen, wobei private grundsätzlich nicht geprüft werden können, kritisierte der rechnungshofpräsident. weitere prüfungslücken bestehen laut moser im bereich der europäischen banken-union. bei der investitionsoffensive sei die abstimmung auf die bedürfnisse der realwirtschaft sicherzustellen, sagte moser. die fehlerquote, die bis 2009 auf 3,3\% sank, stieg seither wieder auf 4,7\%. daher verweigerte das europäische parlament dem generalsekretär des rates auch 2015 die entlastung für den haushaltsplan 2013, ebenso dem direktor des eu-innovations- technologieinstituts und dem exekutivdirektor des gemeinsamen unternehmens ecsel. in der debatte schlug bruno rossmann (g) vor, die eu-finanzen neu auszurichten und bedauerte, dass niemand wisse ob und wann eine finanztransaktionssteuer komme, die diesen namen verdient. laut rossmann liegt der fokus der eu-finanzen zu sehr auf der ersten säule der gemeinsamen agrarpolitik, wo die fehlerquote besonders hoch sei. europa brauche mehr mittel für den arbeitsmarkt und für die finanzierung der flüchtlingsbewegungen. eine geplante exkursion des ausschusses zum europäischen rechnungshof unterstützte elmar mayer (s) ebenso wie die gemeinsame feststellung von lücken bei der prüfung der bankenaufsicht. angesichts der relation zwischen finanziellen rückflüssen europäischer finanzmittel nach polen und ungarn und den beiträgen dieser beiden länder zur lösung der flüchtlingsproblematik hielt mayer maßnahmen für notwendig. auf fehlende sanktionen nach der verweigerten entlastung von der verantwortung für haushaltmittel der eu drängte claudia angela gamon (n). in antworten auf fragen der abgeordneten jessi lintl (f) und andreas ottenschläger (v) empfahl rechnungshofpräsident josef moser wirtschaftlichkeits-prüfungen, um die fehlerquote bei der verwendung von eu-mitteln zu verringern. es gelte zu klären, warum fehler auftreten und konsequenzen daraus zu ziehen. ausschussmehrheit lehnt rh-prüfung von eu-direktförderungen ab von seiten der fpö lag dem ausschuss der bereits einmal vertagte antrag vor, die prüfzuständigkeit des rechnungshofs auf direkt an empfänger ausbezahlte eu-fördermittel zu erweitern ( 411/a(e)). zur begründung erinnerte edith mühlberghuber (f) an den eu-finanzbericht 2011, demzufolge österreich 1,876 mrd. € an eu-mitteln erhielt, von denen 1,481 mrd. € über den bundeshaushalt nach österreich flossen. der rest ging direkt an forschungseinrichtungen oder energieunternehmen. diese 395 mio. € wurden ohne konkrete prüfung ausbezahlt, sagte mühlberghuber (f) und sprach die erwartung einer zustimmung aus, wie dies "dem verlauf der debatte entspricht". diese zustimmung verweigerte claudia durchschlag (v) mit dem argument, eine prüfung privater entspreche nicht dem system des öffentlichen finanzkontrollsystems. überdies seien sowohl der europäische als auch der österreichische rechnungshof ausgelastet. die oppositionsparteien reagierten mit unverständnis auf diese haltung. es sei nicht einzusehen, dass der rechnungshof nicht prüfen soll, wo kontrolllücken bestehen und nachwiesenermaßen öffentliche mittel verschwendet werden, sagten bruno rossmann (g) und claudia gamon (n). die "bremsende haltung der övp" sei unverständlich, weil die kontrolle von eu-direktzahlungen in der vergangenheit auch von övp-abgeordneten gefordert wurde. überdies sollen nicht private kontrolliert werden, sondern öffentliche hände bei der vergabe von direktförderungen, erklärte gabriela moser (g). die ablehnung des antrags erfolgte mit der mehrheit der regierungsparteien. rechnungshofbericht iii-247 d.b. vertagte der ausschuss - zur fristwahrung - einstimmig. & 1474 & very high & High & Governance & Power & NA & 2016-04-18 & 2016 & 2 & POL
Frame & high-very high & National & +1000 & -0.7948903 & -0.1706634 & 1.0035124 & 0.5015415 & -0.3119516 & 9.0 & 0.2648514 & -0.9910023 & Payer & Domestic & European & Mixed & Domestic|POL & Negative\\
Austria & https://diepresse.com/home/wirtschaft/economist/5368409/Ministerin-Koestinger\_Kleine-Betriebe-nicht-dem-freien-Markt & 42 & Die Presse & Private/Non-Public & Online and Offline & National & very low = CP mentioned once & Institutional bargaining over funding & Negative & EU + National & No myth & NA & NA & NA & NA & NA & NA & NA & NA & Austria & ministerin köstinger: kleine betriebe nicht dem freien markt überlassen & 2018-02-08 & kohäsionsfonds & wien. grundsätzlich steht die türkis-blaue regierung subventionen ja eher kritisch gegenüber. anders ist das mit dem eu-geld für österreichs bauern. diese förderungen dürfen nicht gekürzt werden, so elisabeth köstinger (övp), die als nachhaltigkeitsministerin auch die landwirtschaft zu ihren agenden zählt, am donnerstag. hintergrund ist die ankündigung von eu-haushaltskommissar günther oettinger, dass sich die bauern in der eu auf weniger geld aus brüssel einstellen müssten. wegen des brexit fehlen im eu-budget ab 2021 voraussichtlich bis zu 14 milliarden euro an britischen beiträgen. die kommission plane, die agrar- und kohäsionsfonds jeweils um fünf bis zehn prozent zu kürzen, sagte oettinger zur "welt am sonntag" . es werde zwar keinen kahlschlag geben. aber "auch" deutsche bauern müssten sich auf weniger geld einstellen, so der politiker mit blick in sein herkunftsland. dasselbe dürfte früheren ankündigungen zufolge allen landwirten in der eu blühen. köstinger plädiert für eine umverteilung der mittel von groß- zu kleinbetrieben. und vertritt damit eine traditionell österreichische position. weil österreichs landwirtschaft klein strukturiert ist, setzen sich heimische agrarpolitiker stets für obergrenzen für großbetriebe ein: ein durchschnittlicher österreichischer betrieb bewirtschaftet 19, ein deutscher 58 hektar land. etwa zwei drittel der bauerneinkommen in österreich kommen von der öffentlichen hand. vor allem viele bergbauern müssten ohne subventionen zusperren. "diese betriebe können wir nicht dem freien markt überlassen", sagt köstinger. aktuell bekommt die heimische landwirtschaft rund zwei milliarden euro jährlich an nationalen und eu-förderungen. zu recht, findet köstinger. zumal die agrarzahlungen in hohem ausmaß eine förderung für die konsumenten seien und weniger für die bauern. nur wegen der subventionen sei es möglich, dass lebensmittel zu so günstigen preisen verkauft würden, wie es derzeit der fall sei. "das ist eine systemfrage." man könne schon darüber diskutieren, das system umzustellen. aber dann müsse man auch die frage stellen, wie lebensmittel dauerhaft leistbar bleiben. apropos preise: wie die meisten funktionäre in der branche hat köstinger wenig freude mit den österreichischen handelsketten. sie kritisiert die marktmacht der drei großen supermarktketten (rewe, spar und hofer) und wirft ihnen "unlautere geschäftspraktiken" vor. sinke der milchpreis, sei das sofort auf dem abrechnungszettel der bauern zu sehen. steige er, dauere es oft monate, bis die bauern das spüren. irgendwo seien da die margen, an denen jemand verdient. "die bauern sind es nicht", so köstinger. rewe und spar kommen zusammen auf einen marktanteil von 64 prozent, inklusive hofer sind es rund 85 prozent. auch die eu-kommission habe sich dieses themas angenommen, so die ministerin. die behörde werde im mai vorschläge präsentieren, wie sich diese marktmacht eindämmen lasse, so köstinger. sie appelliert auch an die konsumenten: die "geiz ist geil"-mentalität sei in österreich stark verankert. & 435 & very low & Low & Power & NA & NA & 2018-02-08 & 2018 & 3 & POL
Frame & v.low & National & <500 & -0.7948903 & -0.1706634 & 1.0035124 & 0.5015415 & -0.3119516 & 9.0 & 0.2648514 & -0.9910023 & Payer & Domestic & European & Mixed & Domestic|POL & Negative\\
Austria & https://www.ots.at/presseaussendung/OTS\_20180924\_OTS0024/eu-ratsvorsitz-kaernten-organisiert-eigenstaendig-konferenzen & 63 & APA-OTS & Private/Non-Public & Online only & National & medium = CP is important part of story & Jobs & Positive & EU + Subnational & No myth & Economic development & Positive & EU + Subnational & No myth & Territorial cooperation & Positive & EU + Subnational & No myth & Austria & eu-ratsvorsitz: kärnten organisiert eigenständig konferenzen & 2018-09-24 & kohäsionspolitik & klagenfurt (ots) - nachdem das bundesland kärnten im rahmen des österreichischen eu-ratsvorsitzes seitens der bundesregierung nicht als austragungsort für entsprechende veranstaltungen und konferenzen berücksichtigt wurde, hat sich österreichs südlichstes bundesland in eigenregie darum gekümmert - "mit erfolg", wie landeshauptmann peter kaiser, heute montag, bekannt gibt. "neben einer bereits fixierten, hochkarätig besetzten konferenz des ausschusses der regionen (adr) am 09. november im lakeside park in klagenfurt, sind wir auch dabei, einen bürgerdialog mit eu-digitalkommissarin mariya gabriel zu organisieren, die anlässlich einer von infineon organisierten veranstaltung ebenfalls nach kärnten kommt", so kaiser. "kärnten hat sich als eine region im herzen europas vor allem in den letzten jahren durch gemeinsame grenzüberschreitende erfolgsprojekte und durch europa-politisches engagement auch in brüssel einen hervorragenden ruf erarbeitet. brüssel, viele europäische staaten und regionen sind auf kärnten und seine weiterentwicklung im alpe adria raum aufmerksam geworden, man spricht in positiven tönen über uns", verdeutlicht kaiser. daher sei es ihm wichtig gewesen, gemeinsam mit dem von martina rattinger geleiteten kärntner verbindungsbüro in brüssel, sich darum zu bemühen, in der zeit des österreichischen eu-ratsvorsitzes, kärnten im rahmen einer selbst organisierten veranstaltung zu präsentieren. es freue ihn als ständiges und aktives mitglied im adr für kärnten besonders, dass es gelungen ist, eine hochkarätig besetzte adr-konferenz unter der leitung von adr-präsident karl-heimz lambertz zu gewinnen. thema der am 09. november im lakeside park in klagenfurt stattfindenden konferenz: "territorialer zusammenhalt in europa nach 2020 im klartext: die territoriale dimension der eu-kohäsionspolitik und die herausforderungen in den städtischen und ländlichen gebieten". gerade für kärnten sei das thema der gemeinsamen wirtschaftlichen weiterentwicklung von regionen von besonderer bedeutung. "daher ist auch alles zu tun, damit die dafür vorgesehenen eu-finanzmittel und fördertöpfe weiter und ohne grobe einschnitte im rahmen des kommenden mehrjährigen finanzrahmens zur verfügung stehen", betont kaiser. er verweist darauf, dass zwischen 2014 und 2020 aus dem europäischen fonds für regionale entwicklung (efre) rund 81 millionen euro in kärntner und in grenzüberschreitende projekte mit kärntner beteiligung geflossen sein werden. beispielsweise in die internationalisierung exportwilliger kmus, die grenzüberschreitende entwicklung wichtiger technologien für die digitalisierung und den übergang auf eine co2-arme wirtschaft. "damit sind direkt und indirekt tausende arbeitsplätze verbunden", macht kaiser deutlich. zur konferenz eingeladen wurden unter anderen eu-regionalkommissarin corina cretu, der ministerpräsident der deutschsprachigen gemeinschaft in belgien oliver paasch, ilona raugze, direktorin der espon (european territorial observatory network) sowie mitglieder des europaparlaments aus unterschiedlichen europäischen ländern. nach dieser konferenz findet am darauffolgenden tag (10. november) ein termin mit vertretern der eu und der bundesregierung zum 1,6 milliarden-investment von infineon in villach statt. "im vorfeld versuchen wir noch mit der an dieser veranstaltung teilnehmenden eu-digitalkommissarin mariya gabriel ein zusammentreffen mit den mitgliedern der kärntner landesregierung sowie einen bürgerdialog zu fixieren", so kaiser. in jedem fall werde sich kärnten auch im rahmen dieser selbst organisierten termine, von der besten und fortschrittlichsten seite zeigen. "wir nutzen jede chance und bemühen uns, um kärnten als lebenswerten mittelpunkt für unternehmen, industrie und familien zu präsentieren", so kaiser abschließend. & 499 & medium & Medium & Socio-Economic & Socio-Economic & Socio-Economic & 2018-09-24 & 2018 & 3 & ECO
Frame & low-medium & National & <500 & -0.7948903 & -0.1706634 & 1.0035124 & 0.5015415 & -0.3119516 & 9.0 & 0.2648514 & -0.9910023 & Payer & Domestic & European & Mixed & Domestic|ECO & Positive\\
Austria & http://derstandard.at/2000028939290/Asylwerber-Schwierige-Jobsuche-im-Tourismusland & 100 & der Standard & Private/Non-Public & Online and Offline & National & very low = CP mentioned once & Jobs & Positive & Subnational & No myth & NA & NA & NA & NA & NA & NA & NA & NA & Austria & asylwerber: schwierige jobsuche im tourismusland & 2016-01-13 & europäischer sozialfonds & qualifizierungsprojekte werden in vorarlberg pflicht und sollen chancen auf arbeitsplatz erhöhen bregenz - koch, kellner und metzger sind die mangelberufe in vorarlberg. in diesen berufen wäre es für asylsuchende jugendliche möglich, eine lehrstelle zu bekommen. die voraussetzungen für eine beschäftigungsbewilligung: alter unter 25 jahre, seit drei monaten im asylverfahren, keine gleich qualifizierten anderen bewerber. 15 asylsuchende junge menschen haben auf diesem weg 2015 eine lehre im tourismus begonnen, sagt anton strini, geschäftsführer des arbeitsmarktservice (ams) vorarlberg. über das sonderkontingent für saisonarbeitskräfte kamen weitere 16 flüchtlinge im tourismus unter. strini: "obwohl wir den betrieben gesagt haben, dass wir asylsuchende vorrangig behandeln würden." kein großes interesse das interesse, saisonkräfte aus dem kreis der vorgemerkten 427 flüchtlinge anzustellen, war laut strini nicht groß. die betriebe würden lieber auf ihre stammsaisonniers aus deutschland und ungarn zurückgreifen. angeboten bekämen flüchtlinge meist niedrig qualifizierte jobs. der grund: mangelnde deutschkenntnisse. "der schlüssel zur ausbildung oder zum guten job heißt sprache", sagt strini. "ohne gute deutschkenntnisse hat man keine chancen." auch nicht als hoch qualifizierter techniker mit guten englischkenntnissen: "nur in einzelfällen gelingt eine vermittlung in betriebe mit unternehmenssprache englisch." mangel an trainern spracherwerb wird flüchtlingen in vorarlberg aber nicht einfach gemacht. "es fehlt an trainerinnen und trainern, viele flüchtlinge müssen mit nur wenigen deutschstunden pro woche auskommen." auch wenn es in der umgangssprache bereits gut klappt, kommen die schwierigkeiten in der berufsschule, zeigt die praxis. malermeister wolfgang hoch, der seit zwei jahren einen aus afghanistan stammenden lehrling ausbildet: "die hürde ist die fachsprache, sie bereitet in allen branchen probleme." er achte darauf, dass sein lehrling weiterführende sprachkurse besuche, sagt hoch. bei der finanzierung unterstütze die wirtschaftskammer. qualifizierung als pflicht 2016 wird die zahl der arbeitsuchenden flüchtlinge um 800 steigen, schätzt das ams. mit qualifizierungsprojekten, die in diesem monat beginnen werden, versuche man die menschen auf den arbeitsmarkt vorzubereiten, sagt strini. ziel der landesregierung ist es, die menschen nicht auf dauer zu mindestsicherungsbeziehern zu machen. "das wollen auch die wenigsten", sagt strini, "aus den ersten kompetenzchecks wissen wir, dass die leute unglaublich biss haben und arbeiten wollen." 3,3 millionen euro investieren europäischer sozialfonds, ams und land in drei projekte für menschen unterschiedlicher altersgruppen und qualifikationen. 1600 menschen sollen in den nächsten zwei jahren in diesen projekten auf den ersten arbeitsmarkt vorbereitet werden. durch arbeitstraining, weiterbildung und einzelcoaching. die angebote sieht landeshauptmann markus wallner (vp) als pflicht: "verweigerung wird konsequenzen haben." wenn nötig, eine kürzung der mindestsicherung. (jutta berger, 13.1.2016) & 403 & very low & Low & Socio-Economic & NA & NA & 2016-01-13 & 2016 & 2 & ECO
Frame & v.low & National & <500 & -0.7948903 & -0.1706634 & 1.0035124 & 0.5015415 & -0.3119516 & 9.0 & 0.2648514 & -0.9910023 & Payer & Domestic & Domestic & Domestic & Domestic|ECO & Positive\\
Austria & https://www.wienerzeitung.at/meinungen/gastkommentare/961487\_Die-Rache-der-Kanzlerin.html & 23 & Wiener Zeitung & Private/Non-Public & Online and Offline & Regional/Local & very low = CP mentioned once & Political leverage & Negative & EU + Other country & No myth & NA & NA & NA & NA & NA & NA & NA & NA & Austria & die rache der kanzlerin & 2018-04-26 & kohäsionsfonds & die eu will jetzt polen oder ungarn de facto bestrafen, weil sie keine migranten nehmen. klug ist das nicht. es klingt vorerst wie eine überschaubar interessante nachricht aus dem maschinenraum der europäischen union: brüssel, so schlägt die eu-kommission neuerdings vor, soll die zahlungen aus dem sogenannten "kohäsionsfonds" an ärmere mitgliedstaaten künftig nicht mehr bloß anhand der wirtschaftsleistung berechnen, sondern auch die höhe der jugendarbeitslosigkeit sowie die "belastung durch migrationswellen" berücksichtigen. dabei geht es um sehr viel geld. in der laufenden budgetperiode (2014-2020) flossen und fließen unter diesem titel insgesamt 350 milliarden euro an die mitgliedstaaten. und weil dabei die ärmeren weit überproportional berücksichtigt werden, ging bisher sehr viel geld an die ehemals kommunistischen staaten im osten der union, allen voran polen und ungarn. setzt sich die kommission mit ihrem plan durch, wäre dies de facto die von der deutschen bundeskanzlerin seit über einem jahr zwischen den zeilen immer wieder geforderte bestrafung jener osteuropäer, die sich beharrlich weigern, illegal eingereiste migranten auf ihrem staatsgebiet aufzunehmen, zugunsten von griechen oder italienern. vor allem warschau und budapest würden dann gleichsam die quittung dafür bekommen, sich dem diktat der berliner regierungschefin nicht gebeugt zu haben, die zuerst millionen migranten ins land gelassen hat und die daraus, no na, resultierenden probleme aller art per zwangsquote den anderen eu-staaten umhängen wollte. dass berlin meint, eine art züchtigungsrecht gegenüber dem rest europas zu besitzen, ist leider offenbar eine art kontinuität der deutschen geschichte. doch die anderen eu-staaten wären schlecht beraten, frau merkel das ausleben dieser düsteren neigung mit dem geld der europäischen steuerzahler zu gewähren. denn damit würde die heute schon sichtbare spaltung europas zwischen ost und west dramatisch vertieft; was ja nicht wirklich im sinne des vermeintlichen "friedensprojektes" sein kann. "eigentlich sind die westeuropäer auf dem weg, eine teilung europas durchzuführen", fürchtet ex-vizekanzler und osteuropa-kenner erhard busek nicht ohne grund, eine massive finanzielle benachteiligung als disziplinierungsmaßnahme würde diese teilung wohl endgültig machen. dazu kommt, dass eine koppelung der transfermilliarden an die jugendarbeitslosigkeit jene belohnt, die eine schlechte bis desaströse wirtschaftspolitik betreiben oder jedenfalls betrieben haben, während jene bestraft werden, die trotz ihrer relativen armut vernünftig und somit erfolgreicher wirtschaften, wie etwa die balten, tschechen, ungarn oder polen. damit entstünde letztlich ein geradezu perverses anreizsystem, jugendarbeitslosigkeit in kauf zu nehmen, um nicht die kohle aus brüssel zu verlieren. noch erstaunlicher ist schließlich, dass mehr migranten in einem land zu höheren subventionen führen sollen. hat uns nicht gerade frau merkel erklärt, welche bereicherung die migranten für den arbeitsmarkt und das pensionssystem seien, "wertvoller als gold", wie herr schulz von der spd zu sagen pflegte? wenn das stimmt, müssten doch eher jene eu-staaten subventioniert werden, die nicht das glück haben, dieses segens teilhaftig geworden zu sein. aber diese debatte führt berlin leider nicht; warum auch immer. & 464 & very low & Low & Power & NA & NA & 2018-04-26 & 2018 & 3 & POL
Frame & v.low & Regional & <500 & -0.7948903 & -0.1706634 & 1.0035124 & 0.5015415 & -0.3119516 & 9.0 & 0.2648514 & -0.9910023 & Payer & European & European & European & European|POL & Negative\\
\addlinespace
Austria & http://www.salzburg.com/nachrichten/oesterreich/politik/sn/artikel/kanzler-kern-empfing-rumaenischen-kollegen-grindeanu-246895/ & 59 & Salzburger Nachrichten & Private/Non-Public & Online and Offline & Regional/Local & very low = CP mentioned once & Economic development & Positive & National + Other country & No myth & NA & NA & NA & NA & NA & NA & NA & NA & Austria & kanzler kern empfing rumänischen kollegen grindeanu & 2017-05-09 & kohäsionsfonds & gemeinsame pressekonferenz im bundeskanzleramt. bild: sn/apa (bka)/andy wenzel dies sei wichtig für die wirtschaftliche kooperation, betonte kern am dienstag in einer gemeinsamen pressekonferenz mit grindeanu in wien. der rumänische premier seinerseits dankte kern für die unterstützung seines landes beim angestrebten beitritt zum schengenraum. ein weiteres thema waren die geplanten kürzungen der familienbeihilfe für im ausland lebende kinder sowie die flüchtlingspolitik. kern betonte: "wir erwarten, dass es innerhalb europas solidarität gibt und dass jeder seine verpflichtungen wahrnimmt." eine frage sei, wie man mit flüchtlingen und migranten umgehe, und auch die frage des unterhalts - "der ist in österreich immer noch auf einem niveau, das viel höher ist als in fast allen anderen ländern". grindeanu betonte im hinblick auf die pläne zur familienbeihilfe, dass nationale gesetze jedenfalls der eu-gesetzgebung entsprechen müssten. die eu war neben den wirtschaftlichen beziehungen das hauptthema der beiden regierungschefs, weil rumänien im ersten halbjahr 2019 den eu-ratsvorsitz von österreich übernimmt. beide seien sich einig, dass die entwicklung der kohäsionsfonds und infrastrukturmaßnahmen priorität habe und deutliche wachstumsimpulse geben könne, sagte kern. grindeanu ergänzte, dass beide ministerpräsidenten wollten, dass europa ein "erfolgsprojekt bleibt, eine institution, die effizient und solidarisch vorgeht und die auf den prinzipien von demokratie und freiheit beruht". beide regierungschefs lobten außerdem die wirtschaftliche zusammenarbeit zwischen österreich und rumänien. kern sagte, dass sich die rumänische wirtschaft "exzellent" entwickle und "österreichische firmen dort florieren". grindeanu betonte, dass österreich bei den investitionen in rumänien an zweiter stelle stehe. "unser land ist eines der profitabelsten in mitteleuropa", ebenso eines der "wirtschaftsfreundlichsten". die gebühren und abgaben seien sehr niedrig, die wachstumsrate sei im vergangenen jahr die höchste in europa gewesen. laut kern haben österreichische firmen zehn milliarden euro in rumänien investiert und exporte über zwei milliarden euro getätigt. es gibt mehr als 7.000 firmen mit österreichischem kapital in rumänien, "70.000 arbeitsplätze hängen davon ab", sagte der kanzler. & 310 & very low & Low & Socio-Economic & NA & NA & 2017-05-09 & 2017 & 2 & ECO
Frame & v.low & Regional & <500 & -0.7948903 & -0.1706634 & 1.0035124 & 0.5015415 & -0.3119516 & 9.0 & 0.2648514 & -0.9910023 & Payer & Domestic & European & Mixed & Domestic|ECO & Positive\\
Austria & http://text.derstandard.at/2000031920456/EU-Sozialkommissarin-gegen-Zusatzhilfen-fuer-Deutschland?ref=rss & 86 & Der Standard & Private/Non-Public & Online and Offline & National & very low = CP mentioned once & Social awareness/inclusion & Negative & EU + Other country & No myth & NA & NA & NA & NA & NA & NA & NA & NA & Austria & finanzen - eu-sozialkommissarin gegen zusatzhilfen für deutschland & 2016-02-28 & europäischer sozialfonds & thyssen: europäischer sozialfonds langfristig angelegt - deutsches ministerium hält grenzkontrollkosten für überschaubar berlin - eu-sozialkommissarin marianne thyssen hat sich gegen zusätzliche finanzielle hilfen für deutschland wegen der flüchtlingskrise ausgesprochen. "wir können nicht einfach mehr geld nach deutschland überweisen - zu lasten von ländern, die weniger unter der flüchtlingssituation leiden", sagte die belgierin den zeitungen der funke mediengruppe (sonntag-ausgaben). die christdemokratin sieht deutschland in der pflicht, in der flüchtlingskrise einen größeren beitrag zu leisten als andere staaten. sie wies auf die geringere arbeitslosigkeit und eine funktionierende soziale marktwirtschaft hin. der europäische sozialfonds (esf), der für die integration von flüchtlingen genutzt werden könne, sei "langfristig angelegt, und wir müssen uns an die regeln halten", sagte sie. deutschland bekomme 7,5 milliarden euro für den zeitraum von 2014 bis 2021, "und dabei wird es zunächst auch bleiben". der sozialfonds werde 2017 überprüft. "frühestens dann könnte deutschland ein kleines bisschen mehr bekommen. aber das ist nicht einfach", betonte die sozialkommissarin. "dazu wäre ein beschluss des europäischen rates erforderlich." das deutsche wirtschaftsministerium rechnet unterdessen einem medienbericht zufolge auch im falle dauerhafter grenzkontrollen nicht mit einer massiven steigerung der kosten. die ökonomischen auswirkungen dürften insgesamt überschaubar bleiben, zitierte die "welt am sonntag" im voraus aus einem internen papier der behörde. es sei vom 22. februar datiert und trage den titel "wirtschaftliche folgen einer wiedereinführung von systematischen grenzkontrollen für deutschland". zwar könnte es "warteschlangen vor grenzübergängen" und "auswirkungen auf die lieferketten" von unternehmen geben. dennoch halte das ressort des deutschen vizekanzlers sigmar gabriel befürchtungen aus der wirtschaft für unbegründet, dass ein stockender warenverkehr kosten in milliardenhöhe verursachen würde. "belastbare informationen" darüber lägen nicht vor. der deutsche industrie- und handelskammertag (dihk) befürchtet, dass der wirtschaft kosten in einer höhe von zehn milliarden euro pro jahr entstehen könnten. die regierung in berlin hatte am 13. september 2015 vorübergehend grenzkontrollen an den deutschen grenzen eingeführt. den schwerpunkt bildet dabei die deutsch-österreichische grenze. die kontrollen sind bis zum 13. mai dieses jahres befristet und müssten dann verlängert werden. (apa/dpa, 28.2.2016) & 332 & very low & Low & Socio-Economic & NA & NA & 2016-02-28 & 2016 & 2 & ECO
Frame & v.low & National & <500 & -0.7948903 & -0.1706634 & 1.0035124 & 0.5015415 & -0.3119516 & 9.0 & 0.2648514 & -0.9910023 & Payer & European & European & European & European|ECO & Negative\\
Austria & http://derstandard.at/2000045015150/Laender-wollen-jaehrlich-500-Millionen-mehr-vom-Bund & 41 & der Standard & Private/Non-Public & Online and Offline & National & low = CP mentioned more times but NOT important part of story (mainly about others issues) & Political leverage & Negative & National + Other country & No myth & Social justice & Balanced & National + Subnational & No myth & NA & NA & NA & NA & Austria & länder wollen jährlich 500 millionen mehr vom bund & 2016-09-27 & kohäsionsfonds & graz/wien - die landesfinanzreferenten fordern vom bund 500 millionen euro jährlich im neuen finanzausgleich. "das ist angesichts der bundesgesetzlichen mehrbelastungen keine überbordende forderung", sagte der vorsitzende, der steirische landeshauptmann-stellvertreter michael schickhofer (spö) am dienstag nach dem treffen in graz. für die flüchtlingsbetreuungskosten könnte man nicht abgerufene mittel aus dem eu-kohäsionsfonds holen. schickhofer sagte nach dem treffen im grazer landhaus, wie schon in der steiermark üblich wehe der reformgeist: "alle länder ziehen an einem strang". man habe sich geeinigt, "einige kernkapitel mit dem bund abzuarbeiten, wie gesundheit, soziales, flüchtlingsbetreuung. das brennt den menschen unter den nägeln". es gehe auch darum, etwa das ärztearbeitszeitgesetz (äazg) abzuändern und kostengünstiger zu gestalten und ein bündel an maßnahmen bei der pflege vorzuschlagen. "der bund soll respektieren, dass die länder dynamische ausgabenbereiche haben, eben in der gesundheit, der pflege, im sozialen. seit 2008 ist das bip um 21 prozent gestiegen, die ausgaben in diesen bereichen aber um 62 prozent. und diese bereiche haben wir in unseren haushalten", sagte schickhofer. bund strenger als brüssel schickhofer sagte nach der konferenz, die länder seien reformbereit: "aber wir sind damit konfrontiert, dass der bund dinge beschließt und regelt, die schärfer sind als die eu fordert. es ist mehr als gerechtfertigt, dass die länder 500 millionen euro mehr pro jahr erhalten und dass der pflegefonds aufgestockt wird". weiters sei für die länder das thema wohnen zentral: "die vom bund zugesagten 180 millionen euro müssen unkompliziert fließen", sagte der steirische finanzreferent. die länder seien auch bereit, die finanzierung der modellprojekte kinderbetreuung und ganztagsschule zu diskutieren, sie wollten aber keine noch schwerfälligeren systeme. "wir wollen einen schlanken, einfachen und gerechten finanzausgleich", sagte schickhofer. der oö-lh und finanzreferent josef pühringer (övp) bekräftigte, dass die forderungen von allen ländern einheitlich getragen würden und rechtfertigte die forderung nach mehr mitteln für die länder: "wirtschaftswachstum und inflation sind niedrig, kann man da mehr verlangen? wir müssen mehr verlangen. die ausgaben für die aufgaben wachsen überproportional, flüchtlingsbetreuung, soziales". wenn die eu nicht eine faire verteilung von flüchtlingen zustande bringe, müssten die die hauptlast tragenden länder unterstützt werden. "im kohäsionsfonds sind nicht alle mittel abgeholt. durch kommission und rat könnten sie einer anderen verwendung zugeführt werden. was bisher gezahlt wurde, ist ein eher symbolischer beitrag", so oberösterreichs landeshauptmann. bekenntnis zum sparen die länder zeigten beim fag große reformbereitschaft. er hoffe, dass man haftungsobergrenzen für die länder zusammenbringe und ein gemeinsames haushaltsrecht - "damit der vorwurf weg ist, man kann budgets nicht vergleichen. wir bekennen uns alle zum sparen, aber es gibt herausforderungen und das beste beispiel ist die pflege. der pflegefonds muss bis 2021 entsprechend dotiert werden". wenn man jetzt eine pflegeversicherung einführe, so seien das lohnnebenkosten: "das ist nicht das signal, das wir jetzt brauchen". schickhofer stieß ins gleiche horn: "allen ist klar, dass im bereich pflege, soziales, gesundheit mehr geld ausgegeben wird als im jahr vorher. die massiven steigerungsraten müssen sich auf ein vernünftiges maß einpendeln. die länder müssen handlungsfähig bleiben, um investitionsfähig zu bleiben." bezüglich der gespräche zu einer neuen bemessung des verteilungsschlüssels sagte schickhofer auf journalistenfragen, die strategie sei festgelegt. am 21. oktober gebe es weitere verhandlungen mit dem bund und danach noch einmal ein außerordentliches treffen der referenten in der steiermark. (apa, 27.9.2016) & 533 & low & Low & Power & Socio-Economic & NA & 2016-09-27 & 2016 & 2 & POL
Frame & low-medium & National & 500-1000 & -0.7948903 & -0.1706634 & 1.0035124 & 0.5015415 & -0.3119516 & 9.0 & 0.2648514 & -0.9910023 & Payer & Domestic & European & Mixed & Domestic|POL & Negative\\
Austria & http://kaernten.orf.at/news/stories/2701954/ & 16 & kaernten.orf.at & Public & Online only & Regional/Local & low = CP mentioned more times but NOT important part of story (mainly about others issues) & Jobs & Positive & Subnational & No myth & NA & NA & NA & NA & NA & NA & NA & NA & Austria & 33 mio. euro für arbeitsmarktmaßnahmen & 2015-03-26 & europäischer sozialfonds & 3.300 personen werden heuer durch den territorialen beschäftigungspakt (tep) unterstützt. 33 mio. euro stehen für maßnahmen im kampf gegen arbeitslosigkeit und arbeitsplatzsicherung zur verfügung. der tep ist eine kooperation zwischen land und ams. vom land kommen 7,5 mio. euro, vom ams 20 mio. euro. der differenzbetrag wird über dritte, gemeinden, betriebe und eigenerwirtschaftung aufgebracht. ein schwerpunkt der unterstützung liege in der integration älterer arbeitnehmer und auf personen mit länger andauernder arbeitslosigkeit, so eine aussendung des landes kärnten. aber auch lehrwerkstätten werden unterstützt, insgesamt mit zwei mio. euro. eine von ihnen ist die produktionsschule am wifi in klagenfurt. landesrätin gaby schaunig (spö) und der landesgeschäftsführer des arbeitsmarktservice, franz zewell, betonten den schulterschluss zwischen land, ams und sozialpartnern. es gehe darum, maßnahmen für arbeitsmarkt und wirtschaftspolitik eng aufeinander abzustimmen. bei vergaberichtlinien etwa habe man zuschlagskriterien zugunsten älterer arbeitnehmer sowie auch für lehrlinge gestaltet. das thema der integration älterer arbeitnehmer wie auch altersgerechte arbeitsmodelle wird noch eine viel größere rolle spielen müssen, so schaunig. insgesamt würden beim arbeitsmarkt für 28.000 personen an die 76,4 mio. euro aufgewendet. die integrationsquote bezifferte zewell mit 45 prozent, das ziel seien jedoch 50 prozent. er sagte auch, dass man für gemeinnützige arbeit eintrete, diese müsse aber bezahlt werden. leider würde die zahl von menschen auf arbeitssuche zunehmen, ende februar seien es 35.000 gewesen. im baubereich habe ein großer strukturwandel stattgefunden, seit 2009 seien 1.400 arbeitsplätze verlorengegangen. daher unterstütze man das land auch in spezifischen bereichen, wie etwa der althaussanierung. einig ist man sich darin, dass ausbildung und bildung von zentraler bedeutung sind, insbesondere für die jugend. "die lehrlingszahlen seien erfreulicherweise leicht steigend", so zewell, der die kombination lehre und matura als zukunftsweisend erachtet. "jedoch müsste es auch eine art anlehre oder eine teilqualifizierung für jugendliche geben, die gefördert und als solche auch anerkannt werden sollte." ingesamt entfallen auf die bereiche beschäftigungsmaßnahmen (eingliederungsbeihilfen, sozialökonomische betriebe, gemeinnützige beschäftigungsprojekte) 12,35 mio. euro, auf qualifizierungsmaßnahmen 17,4 mio. euro, auf unterstützungsmaßnahmen (kinderbetreuung, berufsorientierung usw.) 2, 3 mio. euro und projekte des esf (europäischer sozialfonds) machen 0,5 mio. euro aus. & 346 & low & Low & Socio-Economic & NA & NA & 2015-03-26 & 2015 & 1 & ECO
Frame & low-medium & Regional & <500 & -0.7948903 & -0.1706634 & 1.0035124 & 0.5015415 & -0.3119516 & 9.0 & 0.2648514 & -0.9910023 & Payer & Domestic & Domestic & Domestic & Domestic|ECO & Positive\\
Austria & http://www.ots.at/presseaussendung/OTS\_20170217\_OTS0043/eu-projekte-fuer-nachhaltige-stadtentwicklung-in-wien & 67 & APA-OTS & Private/Non-Public & Online only & National & high = CP is most important issue in story (can also cover other issues) & Infrastructure & Positive & Subnational & No myth & NA & NA & NA & NA & NA & NA & NA & NA & Austria & eu-projekte für nachhaltige stadtentwicklung in wien & 2017-02-17 & kohäsionspolitik & wien (ots) - in wien stehen bis zum jahr 2020 über 24 mio. euro für städtische projekte zur verfügung, finanziert aus dem efre-programm für investitionen in wachstum und beschäftigung. "zur bewältigung der anstehenden herausforderungen sind die städte bei der eu-politikgestaltung stärker als bisher involviert. die im mai 2016 durch den sogenannten 'pakt von amsterdam' beschlossene eu-städteagenda bietet dazu einen ausgezeichneten rahmen. erste ergebnisse aus dieser neuen form der zusammenarbeit sollten auch in die neuausrichtung der eu-kohäsionspolitik nach 2020 einfließen." streicht martin pospischill, leiter der abteilung für europäische angelegenheiten der stadt wien, bei einer informationsveranstaltung am 16. februar hervor. neben der revitalisierung öffentlicher räume liegen weitere förderschwerpunkte in den nächsten jahren auf forschungsprojekten, wie dem wasserbaulabor wien der universität für bodenkultur, oder auf projekten zur städtischen co2-reduktion. im bereich neugründungen führt die wirtschaftsagentur wien mit seinem "start-up modul" ein efre-kofinanziertes projekt für die beratung potentieller gründerinnen und gründer durch. an die 100 teilnehmer verfolgten bei der von stadt wien und der verwaltungsbehörde des förderprogramms organisierten veranstaltung die präsentation aktueller eu-projekte und eine lebhafte diskussion von projektträgern über ihre erfahrungen mit efre-förderungen. miguel avila albez von der europäischen kommission, generaldirektion regio, zeigte den kontext zwischen eu-regionalpolitik und den efre-förderungen in wien auf. nähere infos auf www.efre.gv.at und eu.wien.at. auskünfte zu fördermöglichkeiten erteilt die ma 27 unter post@ma27.wien.gv.at. & 234 & high & High & Socio-Economic & NA & NA & 2017-02-17 & 2017 & 2 & ECO
Frame & high-very high & National & <500 & -0.7948903 & -0.1706634 & 1.0035124 & 0.5015415 & -0.3119516 & 9.0 & 0.2648514 & -0.9910023 & Payer & Domestic & Domestic & Domestic & Domestic|ECO & Positive\\
\addlinespace
Austria & http://www.kleinezeitung.at/politik/innenpolitik/5260305/ & 28 & Kleine Zeitung & Private/Non-Public & Online and Offline & Regional/Local & very low = CP mentioned once & Political leverage & Balanced & Other country & No myth & NA & NA & NA & NA & NA & NA & NA & NA & Austria & wegen kern-kritik an warschau: österreichs botschafter ins polnische außenministerium zitiert & 2017-07-28 & strukturfonds & der bundeskanzler hatte polen und ungarn mit kürzung von eu-mitteln gedroht, wenn sich weigerten, europäische grundsätze einzuhalten. regierung in warschau reagiert verstimmt. österreichs botschafter in polen, thomas buchsbaum, ist im zusammenhang mit einem interview von bundeskanzler christian kern (spö) in das polnische außenministerium in warschau zitiert worden. das berichtete am freitag die nachrichtenagentur pap. kern hatte polen und ungarn mit einer kürzung von eu-mitteln gedroht, wenn sie sich weigerten, europäische grundsätze einzuhalten. kritik gab es laut pap vor allem an der aussage kerns in dem interview mit der "frankfurter allgemeinen zeitung" (faz), "das nächste budget" werde "der d-day" sein. zudem wurde bei dem gespräch in warschau, das der apa am freitag vom außenministerium (bmeia) bestätigt wurde, unmut über die forderung des bundeskanzlers geäußert, nettotransfers innerhalb der eu als hebel einzusetzen, um länder wie polen zu zwingen, sich "in eine gemeinsame europäische politik" einzufügen. neben der umstrittenen justizreform, welche die nationalkonservative regierung in polen durchsetzen wollte - präsident andrzej duda legte freilich teilweise sein veto ein - hatte der spö-bundeskanzler auch die weigerung polens kritisiert, sich trotz eu-beschlüssen an der umverteilung von flüchtlingen zu beteiligen. von polnischer seite hieß es dem pap-bericht zufolge, dass die frage der "zwangsumsiedlungen" von flüchtlingen "in keiner weise im zusammenhang mit dem thema der eu-strukturfonds" stünde. kern hatte gegenüber der "faz" (donnerstagsausgabe) unter anderem erklärt: "wer sich nicht an die regeln hält, der kann nicht nettoempfänger von 14 oder 6 mrd. euro sein". das werde weder wien noch berlin mittragen. das derzeitige eu-budget ist bis 2019 gültig. auch eu-kommissarin vera jourova hatte am dienstag "ganz allgemein" - und nicht nur auf polen bezogen - darauf gedrängt, künftig rechtsstaatlichkeit zur bedingung für eu-fördermittel zu machen. & 284 & very low & Low & Power & NA & NA & 2017-07-28 & 2017 & 2 & POL
Frame & v.low & Regional & <500 & -0.7948903 & -0.1706634 & 1.0035124 & 0.5015415 & -0.3119516 & 9.0 & 0.2648514 & -0.9910023 & Payer & European & European & European & European|POL & Neutral\\
Austria & https://www.ots.at/presseaussendung/OTS\_20180604\_OTS0006/tiroler-wellnesskongress-2018-bild & 99 & APA-OTS & Private/Non-Public & Online only & National & very low = CP mentioned once & Public services & Positive & Subnational & No myth & NA & NA & NA & NA & NA & NA & NA & NA & Austria & tiroler wellnesskongress 2018 & 2018-06-04 & europäischer fonds für regionale entwicklung & innsbruck (ots) - das geschäft mit gesundheit und wohlbefinden ist eine goldgrube. eine reihe von angeboten zur förderung der körperlichen und geistigen gesundheit wurde von der wellnessbranche bereits entwickelt und erfolgreich auf den markt gebracht. diese angebote sind angepasst an die ansprüche von gästen, die sich heute ebenso wie deren auffassung von gesundheit, wohlbefinden und glück laufend ändern. aktuell ist es die digitalisierung, die neue türen öffnet - durch die aber nicht jeder gesundheitsbewusste gast automatisch gehen will, manche streben ganz im gegenteil nach "digital detox". vor diesem hintergrund eröffnen sich speziell für die hotellerie neue chancen, konzepte und geschäftsmodelle. antworten auf diese fragen suchen wir gemeinsam mit ihnen beim größten netzwerktreffen der wellness-community in tirol, am tiroler wellness kongress 2018. außerdem erfahren sie den aktuellen stand des eu-projekts winhealth, in dem die auswirkungen von wintersport auf die gesundheit untersucht und darauf aufbauend neue touristische angebote entwickelt werden. der cluster wellness der standortagentur tirol lädt in kooperation mit der wirtschaftskammer tirol alle interessierten herzlich zum tiroler wellnesskongress 2018 ein: tiroler wellnesskongress 2018 - der branchentreff für wellness, gesundheit und innovative hotellerie am 14. juni 2018 in innsbuck: expertise durch kompetente fachreferenten, interessante produkt- und dienstleistungsinnovationen, raum zum kennenlernen und vernetzen sowie inspiration für neue geschäftsmodelle. u.a. mit vortrag von dr. david bosshart (gottlieb duttweiler institut zürich. das erwartet sie: - expertise durch kompetente fachreferente - interessante produkt- und dienstleitungsinnovatione - raum zum kennenlernen und vernetze - inspiration für neue geschäftsmodelle programm und anmeldung: www.standort-tirol.at/wellnesskongress18 die cluster der standortagentur tirol werden aus mitteln des efre (europäischer fonds für regionale entwicklung) ko-finanziert. & 261 & very low & Low & Socio-Economic & NA & NA & 2018-06-04 & 2018 & 3 & ECO
Frame & v.low & National & <500 & -0.7948903 & -0.1706634 & 1.0035124 & 0.5015415 & -0.3119516 & 9.0 & 0.2648514 & -0.9910023 & Payer & Domestic & Domestic & Domestic & Domestic|ECO & Positive\\
Austria & http://derstandard.at/2000008594217/Budget-EU-Parlament-blockiert-britische-Ratenzahlung?ref=rss & 64 & der Standard & Private/Non-Public & Online and Offline & National & very low = CP mentioned once & Bureaucracy and/or delays & Negative & EU & No myth & NA & NA & NA & NA & NA & NA & NA & NA & Austria & nachzahlung - eu-parlament blockiert britische ratenzahlung & 2014-11-25 & regionalpolitik & alles über onlinewerbung, stellenanzeigen und immobilieninserate laut övp-abgeordnetem rübig muss großbritannien damit die nachzahlung in der höhe von 2,1 milliarden euro bis nächsten montag begleichen straßburg - das europaparlament hat am dienstag in straßburg eine teillösung für großbritannien im laufenden eu-budgetstreit abgelehnt. die abgeordneten stimmten gegen ein schnellverfahren, dass es london gestatten würde, seine nachzahlungen durch mehrwertsteuer- und bip-korrekturen in höhe von 2,1 milliarden euro in raten zu zahlen. eine folgewirkung der nicht-einigung zwischen den eu-regierungen und dem europaparlament über das budget 2015 sei, dass großbritannien diese zahlung - eine nachzahlung für das laufende budget 2014 - mit dem 1. dezember leisten müsse, erläuterte der övp-europaabgeordnete paul rübig. einigung über budget am 15. dezember bestehe für den eu-ministerrat, am 17. dezember für das parlament der letzte mögliche zeitpunkt, sich zu einigen, sagte rübig. sollte dies nicht gelingen, müsste der eu-haushalt 2015 auf der grundlage von monatlichen zwölftelzahlungen auf basis des diesjährigen eu-budgets erfolgen. die höhe eines solchen zwölftelbudgets wäre deutlich niedriger als ein eigens budget für 2015, sagte die spö-europaabgeordnete karin kadenbach. zwischen den eu-staaten und dem parlament besteht eine kluft von 6 mrd. euro über den haushalt 2015. die grüne europaabgeordnete monika vana sagte, die eu-budgetkommissarin kristalina georgiewa werde wahrscheinlich am freitag einen neuen eu-budgetvorschlag für 2015 machen, der unter dem bisherigen kommissionsentwurf von ausgaben in der höhe von insgesamt 142,137 milliarden liege. das europaparlament beklagt überdies zahlungsrückstände der eu in höhe von rund 25 milliarden euro vor allem im bereich der eu-regionalpolitik. für die abgeordneten ist die begleichung dieser offenen rechnungen durch die eu eine weitere bedingung, um in dem budgetstreit zu einer einigung zu kommen. (apa, 25.11.2014) & 286 & very low & Low & Governance & NA & NA & 2014-11-25 & 2014 & 1 & POL
Frame & v.low & National & <500 & -0.7948903 & -0.1706634 & 1.0035124 & 0.5015415 & -0.3119516 & 9.0 & 0.2648514 & -0.9910023 & Payer & European & European & European & European|POL & Negative\\
Austria & https://kurier.at/chronik/oberoesterreich/fuer-vereinigte-staaten-von-europa/305.136.884 & 22 & Kurier & Private/Non-Public & Online and Offline & National & very low = CP mentioned once & Institutional bargaining over funding & Negative & National + Other country & 2.Rich countries pay & NA & NA & NA & NA & NA & NA & NA & NA & Austria & "für vereinigte staaten von europa" & 2018-01-06 & kohäsionsfonds & nach seinem rücktritt als obmann des wirtschaftsbundes konzentriert sich christoph leitl auf europa. christoph leitl hat im dezember seine funktion als obmann des wirtschaftsbundes abgegeben. der 68-jährige präsident der wirtschaftskammer wurde ein zweites mal zum präsidenten von eurochambres gewählt, der vereinigung der europäischen wirtschaftskammern. ende februar begleitet er antonio tajani, den präsidenten des europaparlaments, bei dessen serbien-besuch. kurier: wann soll serbien der eu beitreten? christoph leitl: die eu ist zurückhaltend, weil es genügend innere probleme gibt. aber serbien ist auf dem besten weg. es könnte in drei bis fünf jahren mitglied werden. katalonien will ein selbstständiger staat werden, was spanien vehement ablehnt. wie soll man den konflikt lösen? es mangelt auf beiden seiten an gesprächsfähigkeit und gesprächsbereitschaft. europa definiert sich als ein europa der regionen und vielfalt. ein autonomiestatus ähnlich dem von südtirol würde es nicht mehr notwendig machen, einen eigenen staat zu gründen. die eu muss position beziehen, denn spanien ist mitglied. die eu lehnt das mit dem argument ab, es handle sich um einen innerspanischen konflikt. die eu soll nicht die details regeln, sondern mit best-practise-beispielen arbeiten. ein beispiel: österreich ist bekannt für sein duales ausbildungssystem der lehrlinge. deswegen haben wir eine der niedrigsten jugendarbeitslosenraten.damit wachsen auch die fachkräfte heran, an denen es europaweit mangelt. die bundesregierung will den südtirolern auch die österreichische staatsbürgerschaft geben. finden sie das gut oder ist das eine schnapsidee? ich hüte mich, in ausschließlich politische fragen einzugreifen. ich bin anhänger einer europäischen staatsbürgerschaft als zeichen der zusammengehörigkeit. europa klingt für österreich manchmal weit weg. aber wenn europa nicht erfolgreich ist oder gar zerfällt, würde das für österreich ungeheuer negative konsequenzen haben. wir wären dann wie eine nußschale im ozean. wir sind in einem globalen wettbewerb. china ist weltweit in der offensive. sie haben die osteuropäischen länder für ihre strategie der seidenstraße gewonnen.china besitzt in afrika die wichtigsten rohstoffvorkommen. die europäer schlafen. wir streiten uns mit russland anstatt sich mit ihm zu verbünden. russland hat daran erheblichen anteil, denn es hat mit der krim und der ostukraine einfach teile der ukraine besetzt und annektiert. amerika zeigt uns den weisel und wir kommen nicht drauf, dass wir in die selbstständigkeit gehen müssen. europa muss eine eigenständige politik des ausgleichs betreiben. natürlich hat russland gegen das völkerrecht verstoßen, aber haben das nicht auch die amerikaner gemacht? ein unrecht rechtfertigt nicht das andere. ich will das auch nicht hochrechnen, ich will nur, dass wir von den ständigen drohgebärden und den unsinnigen sanktionen, wo die wirtschaft als waffe missbraucht wird, wegkommen. was soll denn die eu außer den wirtschaftssanktionen machten? mit gutem zureden wird man putin nicht beeindrucken. es gibt politische und diplomatische möglichkeiten. ich halte nichts vom boykott sportlicher veranstaltungen, nichts vom abbrechen kultureller brücken und nichts von wirtschaftlichen sanktionen. glauben sie, dass reden bei einem ehemaligen geheimdienstagenten wie putin wirklich genügt? haben die dreijährigen wirtschaftssanktionen irgendeine lösung gebracht? lösung haben sie keine gebracht, aber sie zeigen wirkung. die russen haben dadurch einen vorteil. putin hat genau das erreicht, was er wollte. aus einem import, der russland früher devisen gekostet hat, wurde eine investition vor ort. das kann man am beispiel der firma backaldrin sehen, die jetzt in moskau ein werk gebaut hat. beide haben profitiert, backaldrin und russland. österreich zahlt drauf, weil der kornspitz in russland produziert wird. aber die grundware wird aus österreich angeliefert. österreich zahlt drauf. der schaden für ganz europa geht jährlich in dreistellige milliardenbeträge. mein vorschlag wäre, schritt für schritt auf beiden seiten abzurüsten und abzubauen. und am schluss eine kooperation mit russland zu suchen. wir wären dann im globalen wettbewerb unschlagbar. deshalb sind die amerikaner dagegen. der freihandel ist generell bedroht. trump sagt, america first, die chinesen sagen china strong, was sagen wir europäer? wir geben keine antwort und lassen uns die anteile am weltmarkt nehmen. so ist es. mein motto ist europa competitive. wir sind wettbewerbsfähig,weil wir innovativ sind, weil wir auf qualität setzen. wir brauchen verbündete in einer europäischen freihandelszone, die langfristig auch länder wie die türkei, die ukraine, den nahen osten und nordafrika umfasst. sollen wir zuschauen, wie sich die anderen bei uns verankern und sich festkraulen? gleichzeitig sollten wir die eu, ausgehend von der eurozone, vertiefen. wir brauchen ein unternehmerisches europa, das sich nicht fürchtet zwischen amerika und china zerrieben zu werden, sondern das unsere werte und unseren wohlstand sichert. europa zählt nur sieben prozent der weltbevölkerung, mit sinkender tendenz. es hat noch 25 prozent des welthandels, das wird auch immer weniger, aber es stellt auch 50 prozent der weltausgaben für soziales und umwelt. das, was für uns selbstverständlich ist, ist nicht selbstverständlich. wie soll die krise des euro gelöst werden? der euro hat sich viel besser gehalten als es die düsteren prognosen der amerikaner prognostiziert haben. ich habe damals bei einem vortrag in princeton gesagt, dass der dollar seinen thron zu teilen hat. den wirklichen durchbruch gibt es erst dann, wenn die währung durch eine gemeinsame wirtschafts- und währungspolitik fundiert ist. dass bedeutet, dass sich europa weiter vertiefen muss. es muss die wirtschafts-, die finanz- und fiskalpolitik stärker gemeinsam gestaltet werden. soll es einen gemeinsamen finanzminister geben? ich halte diesen vorschlag für sehr gut. wir sollen auch bei der unternehmensbesteuerung eine gemeinsame bandbreite finden, um dumping zu verhindern. ungarn reduziert die unternehmensbesteuerung auf neun prozent, wir in österreich haben 25 prozent. auf der anderen seite erhält ungarn aus den kohäsionsfonds der eu mittel auch aus österreich. das passt nicht zusammen. in welchem bereich soll sich die bandbreite bewegen? zwischen 15 und 25 prozent. soll es auch einen gemeinsamen europäischen wirtschaftsminister geben? sicher. das bedeutet, dass die nationalstaaten kompetenzen an brüssel abtreten müssen. ja. der wille ist dafür weder in der bevölkerung noch bei den politikern vorhanden. es ist eine bewusstseinsfrage. der euro ist basis unseres wohlstandes. wenn wir diesen wohlstand erhalten wollen, müssen wir alles tun, dass die wirtschafts- und währungspolitik funktionieren. welche kompetenzen sollen an brüssel abgegeben werden? wir haben jetzt schon die finanzpolitik abgegeben. jetzt sollten wir das in der steuerpolitik machen. mit spielräumen? durchaus mit spielräumen. es ist auch in der schweiz so, das die kantone spielräume haben. auch in den usa haben die bundesstaaten diese spielräume. ihre vorstellung läuft auf die vereinigten staaten von europa hinaus. langfristig gesehen ja, denn das ist ohne alternative. das soll aber kein zentralstaat sein, der alles im detail regelt, sondern ein starker staat nach außen, der unser schutzmantel gegen die bedrohungen der globalisierung ist. nach innen soll er auf dem prinzip der subsidiarität aufgebaut sein. alles, was länder und regionen selbst regeln können, sollen sie regeln. der nationalstaat wird auf der einen seite kompetenzen an die regionen abgeben. siehe oberösterreich, das mehr autonomie fordert. das bedeutet eine stärkung der regionen auf der einen seite und eine stärkung von brüssel auf der anderen seite. beides geht auf kosten des nationalstaates. so ist es. sie sind präsident der bundeswirtschaftskammer, eine funktion, die sie bis zum ende der eu-ratspräsidentschaft österreich, also bis ende 2018, behalten wollen. das ist eine mögliche variante. es ist derzeit noch nichts bestimmt. ich möchte das im gespräch mit sebastian kurz und harald mahrer definieren. ich werde das thema europa, in welcher funktion auch immer, aktiv begleiten. sie haben vor einigen jahren in der österreichischen verwaltung ein einsparpotenzial von vier milliarden euro definiert. wie viel davon ist bereits gehoben? (lacht). von den damals von mir genannten 3,5 milliarden euro ist das auf vier milliarden angestiegen. es läuft in die falsche richtung. ich setze große erwartungen in minister josef moser, den ehemaligen rechnungshofpräsidenten. er kennt sich aus. wir reduzieren in der wirtschaftskammer durch die möglichkeiten der digitalisierung um 20 prozent. warum kann der staat nicht um zehn prozent, das wären 16 milliarden euro, reduzieren? das muss heute jedes unternehmen machen. der staat könnte kompetenzzentren einrichten und nicht alles in wien zentralisieren. oberösterreich könnte kompetenzzentrum für innovation, für erneuerbare energie und umwelttechnik werden. & 1305 & very low & Low & Power & NA & NA & 2018-01-06 & 2018 & 3 & POL
Frame & v.low & National & +1000 & -0.7948903 & -0.1706634 & 1.0035124 & 0.5015415 & -0.3119516 & 9.0 & 0.2648514 & -0.9910023 & Payer & Domestic & European & Mixed & Domestic|POL & Negative\\
Austria & https://aktien-portal.at/shownews.html?id=55190 & 17 & aktien-portal.at & Private/Non-Public & Online only & National & very low = CP mentioned once & Infrastructure & Balanced & National + Other country & No myth & Bureaucracy and/or delays & Balanced & EU & No myth & NA & NA & NA & NA & Austria & porr & 2019-04-29 & kohäsionsfonds & baukonzern erfreut sich hoher auslastung: nachfrage in heimmärkten ungebrochen - umfeld aber "herausfordernd" - ceo pariert kritik an kostensteigerungen - deutschland "nachhaltig positiv" - bild grafik österreichs zweitgrößter baukonzern porr kann sich über eine vollauslastung beinahe aller kapazitäten freuen. 2019 und 2020 stelle man aber jedenfalls ertrag vor umsatzwachstum, betonte generaldirektor karl-heinz strauss am montag im bilanzpressegespräch. voriges jahr markierten bauleistung und auftragsstand neue rekorde. der börsennotierte bauriese zählt mittlerweile fast 20.000 mitarbeiter. in den kernmärkten seien bauindustrie und baugewerbe "sehr ausgelastet", die nachfrage dort sei ungebrochen, so strauss. überhitzt sei der sektor in polen: dort kämen viele projekte gleichzeitig auf den markt, weil jeder die kofinanzierungen des eu-kohäsionsfonds ausnutzen wolle. deutschland habe allgemein ein kapazitätsproblem, weil der markt von kleinen und mittleren playern geprägt sei. das marktumfeld sei "herausfordernd", so porr im ausblick - wegen des fachkräftemangels, der engpässe bei subunternehmern sowie steigenden baupreisen und lohnkosten. vor allem in polen habe sich dieser trend 2018 verstärkt, dessen ende vorerst noch nicht absehbar sei. in dem land gebe es zudem eine knappheit an baumaterial. porr selbst nehme in polen nur mehr sehr gut kalkulierte projekte herein. preisanstiege seien auch für die porr als baukonzern nicht immer gut, verwies strauss etwa auf kritik heimischer wohnbauträger, die sich generell wiederholt über einen starken kostenschub und verzögerungen bei bauprojekten beklagen. der porr-chef rechnete vor, dass bauindustrie und -gewerbe im zeitraum 2011/12 bis 2018 den großteil der kostensteigerungen selbst "geschluckt" hätten: die kosten seien in der zeit um fast 30 prozent gestiegen, den firmen weitergegeben habe man aber nur 13 prozent, 17 prozent selbst aufgefangen. in deutschland, wo porr 2018 beim ebt wieder in die gewinnzone gekommen ist, sieht strauss diese entwicklung als nachhaltig an; nahezu alle bereiche seien positiv, auch der hochbau. für 2019 rechne man im nachbarland mit einem deutlich besseren ergebnis als 2018. bei der bauleistung erreichte porr 2018 zum dritten mal in folge ein zweistelliges plus, sie stieg um 18,0 prozent auf 5,59 mrd. euro - ein neuer rekordwert. deutliche anstiege verzeichneten hier hauptsächlich der industrie- und ingenieurbau sowie internationale infrastrukturprojekte. heuer soll die produktionsleistung über dem vorjahr liegen. der auftragsstand erreichte ein all-time-high, er wuchs 11,5 prozent auf 7,10 mrd. euro. neben vielen hochbauvorhaben habe man vor allem neue infrastruktur-großprojekte akquiriert. gemessen an der leistung war der heimmarkt österreich, an dem sich porr als nummer 1 sieht, mit 2,3 mrd. euro oder 42 prozent anteil weiterhin klar der größte der porr-märkte, wie finanzvorstand andreas sauer sagte - gefolgt von deutschland mit 1,5 mrd. euro (27 prozent) sowie polen (12 prozent) und dann der schweiz (4 prozent) und tschechien (3,6 prozent). beim nachbarn deutschland sei man, je nach betrachtung, die nummer 4 oder 5, sagte straus. rumänien und slowakei könnten 2020 oder 2021 zu kernmärkten aufrücken. anders bei norwegen: dort plane man aus heutiger sicht nicht, ins flächengeschäft zu gehen, das projektgeschäft sei aber nachhaltig. in uk beobachte man, was mit dem "brexit" sei, so der ceo. man habe dort ein projekt, das laufe gut, und man beobachte zwei weitere projekte. wie norwegen und uk seien auch katar und dubai stabile projektmärkte, sagte strauss. in katar und den vereinigten arabischen emiraten (vae) bleibe porr weiter engagiert und evaluiere vorsichtig aufkommende marktpotenziale. in der übergabe steht in katar das al-wakrah-stadion - austragungsort der fußball-wm 2022 -, am 16. mai finde das stadion-eröffnungsspiel statt. davor zeichnete porr in katar für den bau der fast 17 km langen "green line" des u-bahn-systems von doha verantwortlich. in dubai realisiert porr für die stadt das "deep tunnel storm water system", eine mehr als 10 km lange tunnelkonstruktion, die grund- und niederschlagswasser in richtung eines pumpwerks am meer leiten soll. nach der bis zur expo 2020 geplanten fertigstellung werden rund 40 prozent des stadtgebiets von dubai über dieses system entwässert. mit dem projekt schaffte porr den markteintritt in vae. seine geschäftsfeld-aufteilung hat porr mit jahresbeginn neu aufgestellt, also verschlankt. bildeten früher österreich/schweiz/tschechien sowie deutschland, internationales und umwelt die vier business-units, sind es jetzt drei: 1 österreich/schweiz/umwelttechnik, 2 deutschland, 3 polen\&norwegen/tschechien\&slowakei/tunnelbau/katar\&vae/rumänien/uk. beim ergebnis vor steuern (ebt) erreichte porr 2018 mit plus 3,3 prozent auf 88,1 (85,3) mio. euro das bisher zweitbeste ergebnis. der jahresüberschuss lag mit 66,2 (63,7) mio. euro 3,9 prozent höher - je aktie bei 2,17 (2,09) euro. als dividende sind erneut 1,10 euro je aktie geplant. trotz zweistelligen wachstums stieg die nettoverschuldung - 150 (147) mio. euro - kaum. das eigenkapital wuchs im jahresabstand auf 618 (597) mio. euro, die eigenkapitalquote betrug 19,9 (20,7) prozent. ende 2018 gab es 320 mio. euro liquide mittel. der mitarbeiterstand erhöhte sich - vor allem akquisitionsbedingt - um 7,3 prozent auf 19.014 (17.719). & 801 & very low & Low & Socio-Economic & Governance & NA & 2019-04-29 & 2019 & 3 & ECO
Frame & v.low & National & 500-1000 & -0.7948903 & -0.1706634 & 1.0035124 & 0.5015415 & -0.3119516 & 9.0 & 0.2648514 & -0.9910023 & Payer & Domestic & European & Mixed & Domestic|ECO & Neutral\\
\addlinespace
Austria & http://www.wienerzeitung.at/nachrichten/wirtschaft/international/880001\_Der-Strukturfonds-steht-in-voller-Bluete.html & 65 & Wiener Zeitung & Private/Non-Public & Online and Offline & Regional/Local & very high = CP is most important issue + CP is mentioned in title/headline & Solidarity to poor countries/regions & Positive & EU & No myth & Institutional bargaining over funding & Balanced & EU + Other country & No myth & NA & NA & NA & NA & Austria & der strukturfonds steht in voller blüte - wiener zeitung online & 2017-03-16 & strukturfonds & wien. (wak) mittelfristig scheinen die aussichten für die länder mittel-, ost- und südeuropas blendend. vor allem in den eu-mitgliedsländern der region stehen die zeichen auf wachstum: das liegt allerdings zum teil daran, dass in den kommenden jahren - der prognosezeitraum ist von 2017 bis 2019 - die fördertöpfe von der eu voll schlagend werden. denn da ist die hauptzeit der investitionen durch den langfristigen strukturfonds der eu, der unter anderem infrastrukturprojekte in der region fördert. die derzeitige finanzperiode läuft von 2014 bis 2020. die anfangszeit dieses siebenjährigen zeitraums lässt sich immer ein wenig langsam an, projekte müssen eingereicht werden, partner müssen gefunden werden - aber ab 2017 wird mit den vollen effekten gerechnet. zumindest vom wiener institut für internationale wirtschaftsvergleiche (wiiw). das institut rechnet 2017 für die eu-länder mittel- und osteuropas mit drei prozent wachstum (1,2 prozentpunkte vorsprung zum durchschnitt der eurozone). in der periode 2018-2019 soll das wachstum der wirtschaft mittel- und osteuropas sogar auf 3,2 prozent ansteigen. riskant wird dann, was in den jahren darauf folgt. stichwort: brexit. denn der rückzug des vereinigten königreichs aus der eu wird sich besonders bei den transferleistungen in der region bemerkbar machen, sagt wiiw-ökonom mario holzner: "man darf nicht vergessen, dass das vereinigte königreich bisher 20 prozent zum nettobudget beiträgt." damit wird der strukturfonds - und die region - in zukunft mit "erheblichen einbußen" zu rechnen haben. "paradoxe gruppe visegrad" die sich gegenüber anderen mitgliedern oft kämpferisch gebenden visegrad-staaten - polen, tschechien, slowakei und ungarn - sind laut holzner eine "paradoxe gruppe", denn bei denen ist die "renaissance des nationalen stolzes" zu beobachten, dabei sind sie "extrem abhängig von den transferzahlungen". ihr verhalten gegenüber brüssel schaut demnach "oft inkohärent aus", formuliert es holzner. die neuverhandlungen des eu-budgets werden demnach auch "interessant" für diese staaten werden, wo doch westeuropa mitverhandeln wird. man dürfe nicht vergessen, dass die eu-transferzahlungen je nach land zwei bis vier prozent des bruttoinlandsprodukts ausmachen, und damit für das land "fundamental" sind, urteilt holzner. trotzdem: vor allem die visegrad-staaten scheinen die förderungen gezielt in industrieproduktionen anzulegen; hier streicht holzner die stattfindende reindustrialisierung hervor. orientiert man sich rein an den wirtschaftlichen daten, so sieht es auch im rest der untersuchten länder besser aus als zuletzt. auf dem westbalkan zieht die wirtschaft an, und sogar die gemeinschaft unabhängiger staaten sowie die ukraine "scheinen langsam aus dem sumpf herauszukommen und an fahrt aufzunehmen", erklärt holzner. die arbeitsmarktlage hat sich verbessert - die arbeitslosenzahl ist gesunken, die erwerbsbevölkerung stagniert. man bewege sich langsam in richtung vollbeschäftigung. in großen teilen der region gebe es sogar einen arbeitskräftemangel, sagt holzner. aus heutiger sicht ist das gebiet "wieder auf dem weg der konvergenz" mit dem euroraum. allerdings dürfe man nicht vergessen, dass parallel die politische unsicherheit steigt. neben dem brexit nennt holzner unter anderem die wahl von donald trump: "es ist nicht berechenbar, was trump konkret bedeutet." die tatsache, dass der us-präsident ein vorreiter des protektionismus zu sein scheint, könnte etwa auch ein schlaglicht auf die zukunft des welthandels und damit auf die länder süd-, mittel- und osteuropas werfen: denn diese seien extrem vom export abhängig. & 512 & very high & High & Values & Power & NA & 2017-03-16 & 2017 & 2 & ECO
Frame & high-very high & Regional & 500-1000 & -0.7948903 & -0.1706634 & 1.0035124 & 0.5015415 & -0.3119516 & 9.0 & 0.2648514 & -0.9910023 & Payer & European & European & European & European|ECO & Positive\\
Austria & https://www.nachrichten.at/wirtschaft/Suedosteuropa-Markt-mit-EU-Perspektive;art15,1505679\#ref\%3Drss & 5 & Oberosterreichische Nachrichten & Private/Non-Public & Online and Offline & Regional/Local & medium = CP is important part of story & Economic development & Balanced & EU + Other country & No myth & Bureaucracy and/or delays & Negative & Other country & 10.Slow spend & NA & NA & NA & NA & NA & zwölf milliarden euro für kroatien und slowenien im eu-kohäsionsfonds & 2014-09-23 & NA & NA & 247 & medium & Medium & Socio-Economic & Governance & NA & 2014-09-23 & 2014 & 1 & ECO
Frame & low-medium & Regional & <500 & -0.7948903 & -0.1706634 & 1.0035124 & 0.5015415 & -0.3119516 & 9.0 & 0.2648514 & -0.9910023 & Payer & European & European & European & European|ECO & Neutral\\
Austria & http://www.tt.com/wirtschaft/wirtschaftspolitik/11967590-91/nowotny-für-londoner-city-steht-sehr-viel-am-spiel.csp & 43 & Tiroler Tageszeitung Online & Private/Non-Public & Online and Offline & Regional/Local & very low = CP mentioned once & Institutional bargaining over funding & Negative & National + Other country & No myth & NA & NA & NA & NA & NA & NA & NA & NA & Austria & nowotny: & 2016-09-02 & kohäsionsfonds & wien, alpbach, london - für die londoner city, dem größten finanzplatz in europa, stehe bei einem brexit "sehr viel am spiel", betonte oenb-gouverneur am freitag in alpbach bei einem pressegespräch. die frage der clearinghäuser sei daher ein ganz entscheidender bereich für die austrittsverhandlungen und für die city of london. der andere wichtige bereich seien die wegfallenden zahlungen ins eu-budget. derzeit sei die city of london der wichtigste handelsplatz für transaktionen in euro. von den euro-fremdwährungs-transaktionen werden 45 prozent in london abgewickelt, das waren im jahr 2013 täglich durchschnittlich 1.009,3 mrd. us-dollar. paris und frankfurt stehen in den startlöchern der anteil an den transaktionen von euro-zinsderivaten ist noch höher: da werden dreiviertel aller euro-transaktionen in london abgewickelt. das tägliche volumen lag 2013 bei 927,8 mrd. us-dollar. "das clearing hier ist ein ganz wichtiger geschäftszweig der londoner city", betonte nowotny. wenn - im schlimmsten fall für london - es zu einem verlust des passporting kommt, die britischen finanzinstitute also keinen zugang mehr zu den europäischen märkten haben, dann werde davon ausgegangen, dass 70 prozent dieses geschäfts nicht mehr von london ausgehen kann, sondern von finanzplätzen in der eu - paris, frankfurt zum beispiel. "da sind schon erhebliche dinge im spiel." die ezb habe zur abwicklung von euro-transaktionen eine klare position und deswegen bereits einen rechtsstreit mit der englischen regierung über euro-clearinghäuser gehabt. die ezb sei der meinung, dass clearinghäusern außerhalb der eu kein euro-clearing ermöglicht sein soll. der rechtsstreit sei verloren worden, weil für den eugh der europäische binnenmarkt von größerem wert gewesen sei. "wenn england aber nicht mehr im binnenmarkt ist, schauen die dinge anders aus." dann sind wir genau dort, wo die ezb ursprünglich sein wollte, so nowotny. die frage der clearinghäuser sei daher ein ganz entscheidender bereich für die austrittsverhandlungen und für die city of london. "da steht schon für die city sehr viel am spiel". briten sind zweitgrößter nettozahler was den erleichterten zugang zum europäischen finanzmarkt mittels passporting betreffe, sei die position der city auch unterschiedlich. für die banken sei dies eine ganz entscheidende sache. die city sei aber auch private equity-fonds oder hedgefonds, die seien eher pro-brexit gewesen, weil sie sich erhoffen, eine leichtere regulierung zu haben, und so "als piraten", in den europäischen binnenmarkt hineingehen können. "das wird auch eine herausforderung sein, sich gegen eine solche strategie "regulierung light" zu wehren. ebenfalls zu wenig gesehen werde in der öffentlichkeit der umstand, dass das vereinte königreich der zweitgrößte zahler im eu-budget sei, trotz rabatts. die zahlungen finanzierten vor allem die struktur- und kohäsionsfonds, die für die neuen mitgliedsstaaten von erheblicher bedeutung seien. derzeit gebe es keine bereitschaft, den ausfall dieser zahlungen durch höhere zahlungen der anderen mitgliedsstaaten zu kompensieren. das sei ein sehr sensibler bereich, der auch österreich indirekt betreffe, weil davon die osteuropäischen staaten betroffen seien. (apa) clearing: allgemein: die gegenseitige auf- und verrechnung von forderungen und verbindlichkeiten zwischen geschäftspartnern. zahlungsverkehr (interbanken-clearing) : verfahren der übermittlung, der abstimmung und in einigen fällen auch die bestätigung von zahlungsaufträgen vor dem zahlungsausgleich; dies entspricht im deutschen sprachgebrauch dem begriff der zahlungsverkehrsabwicklung. industrie-clearing: in der klassischen variante einigen sich zwei industrieunternehmen über ein vorziehen von zukünftigen zahlungen für lieferungen und leistungen, um einen auftretenden zahlungsmittelbedarf auszugleichen. wertpapierclearing: im wertpapierhandel findet die abwicklung und abrechung von wertpapierkäufen und -- verkäufen über ein clearinghaus statt. die mitglieder eines clearingsystems müssen bestimmte anforderungen erfüllen und für die zulassung sicherheiten leisten. kommentieren kommentar schreiben schlagworte alpbach brexit city clearing finanzmärkte finanzplatz hedgefonds kohäsionsfonds london private equity walter nowotny & 588 & very low & Low & Power & NA & NA & 2016-09-02 & 2016 & 2 & POL
Frame & v.low & Regional & 500-1000 & -0.7948903 & -0.1706634 & 1.0035124 & 0.5015415 & -0.3119516 & 9.0 & 0.2648514 & -0.9910023 & Payer & Domestic & European & Mixed & Domestic|POL & Negative\\
Austria & https://www.ots.at/presseaussendung/OTS\_20180123\_OTS0012/lh-mikl-leitner-in-bruessel-arbeitsgespraeche-mit-kommissionspraesident-juncker-und-budgetkommissar-oettinger & 80 & APA-OTS & Private/Non-Public & Online only & National & medium = CP is important part of story & Institutional bargaining over funding & Balanced & EU + National + Subnational & No myth & NA & NA & NA & NA & NA & NA & NA & NA & Austria & lh mikl-leitner in brüssel: arbeitsgespräche mit kommissionspräsident juncker und budgetkommissar oettinger & 2018-01-23 & kohäsionspolitik & st. pölten (ots/nlk) - landeshauptfrau johanna mikl-leitner traf am gestrigen montag in brüssel mit eu-kommissionspräsident jean-claude juncker und eu-kommissar günther h. oettinger zusammen. in den intensiven arbeitsgesprächen warb die landeshauptfrau für eine verlängerung der eu-regionalförderung nach 2020 für alle regionen. die landeshauptfrau, die dem kommissionspräsidenten und dem budgetkommissar ein aktualisiertes positionspapier der partnerregionen niederösterreichs zur eu-regionalförderung übergab, betonte dazu: "ein starkes europa kann es nur dann geben, wenn es die regionen europa auch spüren. dies wird am besten durch die eu-regionalförderung sichtbar, die projekte direkt beim bürger eu-kofinanziert. für den bürger sind eine effiziente eu-außengrenze, sicherheit, terrorismus-bekämpfung, verhinderung der illegalen migration ein eu-thema. aber europa muss auch bei den menschen ankommen. das geschieht am besten und sichtbarsten durch die eu-regionalpolitik." die aktuelle eu-förderperiode läuft ende 2020 aus und jetzt befindet man sich in der "heißen phase" der vorbereitung auf die neue siebenjährige förderperiode ab 2021. dabei geht es um ein budgetvolumen von rund derzeit 352 milliarden euro für sieben jahre - das entspricht einem drittel des gesamten eu-haushaltes. unter der federführung niederösterreichs haben sich die regionen unter neuen voraussetzungen abermals zusammengetan und ein aktualisiertes positionspapier nach vorbild der vergangenen regionen-initiative entwickelt. der zentrale inhalt des papiers: die eu-regionalpolitik soll weiterhin alle regionen berücksichtigen - nicht nur die schwächeren regionen, auch die übergangsregionen und stärkeren regionen sollen weiterhin regionalfördermittel erhalten. neue standpunkte des aktuellen papiers sind etwa die stärkung des subsidiaritätsprinzips und der bedeutung der europäischen union als entscheidende faktoren für die zukunft europas, das bekenntnis zur konzentration der förderung auf die wichtigsten regionalen herausforderungen und prioritäten im einklang mit den eu-zielen sowie die forderung nach vereinfachung und entbürokratisierung der eu-förderabwicklung. die landeshauptfrau zum ziel des positionspapiers: "die regionen müssen möglichst rasch planungssicherheit haben, weil jedes projekt national und regional kofinanziert werden muss. und das verlangt stabilität über projektperioden hinweg." mikl-leitner weiters: "hinter dieser initiative stehen aktuell 342 regionen, städte und organisationen. das sind 84 prozent der eu-27-bevölkerung, die für den bestand einer erneuerten und vereinfachten, aber starken eu-kohäsionspolitik eintreten". die unterschriften seien "ein auftrag an die eu-institutionen, auf die stimme der regionen zu hören", betont sie: "in unseren arbeitsgesprächen konnten wir unser anliegen nochmals mit nachdruck verstärkten, und zwar bei allen wesentlichen vertretern der eu-institutionen." neben der initiative zur eu-regionalförderung wurde seitens der landeshauptfrau auch ein positionspapier der nö landesregierung als beitrag zum diskussionsprozess zum "weißbuch zur zukunft der eu" in die arbeitsgespräche eingebracht. demnach gibt es einen beschluss der landesregierung, der angeregt, dass sich die eu u. a. für die schaffung einer umfassenden und gemeinsamen europäischen migrationspolitik, den effizienten schutz der eu-außengrenzen und die einhaltung des subsidiaritätsprinzips als wesentliches element der europäischen integration einsetzen soll. auch diese positionen wurden den höchsten vertretern der kommission übergeben. die gespräche verliefen sehr positiv. sowohl der präsident der europäischen kommission also auch kommissar oettinger, zuständig für die planung des künftigen eu-budgets, können einen großteil der positionen teilen und als kommissionsposition ende mai vorlegen. die entscheidung liegt dann beim rat und dem europäischen parlament, bestmöglich vor den europawahlen im mai 2019. & 522 & medium & Medium & Power & NA & NA & 2018-01-23 & 2018 & 3 & POL
Frame & low-medium & National & 500-1000 & -0.7948903 & -0.1706634 & 1.0035124 & 0.5015415 & -0.3119516 & 9.0 & 0.2648514 & -0.9910023 & Payer & Domestic & European & Mixed & Domestic|POL & Neutral\\
Austria & http://www.salzburg.com/nachrichten/meinung/standpunkt/sn/artikel/wo-oesterreichs-kanzler-recht-hat-165692/ & 13 & Salzburger Nachrichten & Private/Non-Public & Online and Offline & Regional/Local & very low = CP mentioned once & Political leverage & Balanced & National + Other country & No myth & NA & NA & NA & NA & NA & NA & NA & NA & Austria & where austria chancellor is right & 2015-09-13 & strukturfonds & mitten im asylchaos beginnt die suche nach einer neuen europäischen flüchtlingspolitik. die alte ist glorios gescheitert. heute, montag, treffen einander die innenminister der 28 eu-staaten in brüssel zum auftakt eines abschieds. verabschiedet wird die bisherige asylpolitik. funktioniert hat sie sowieso nie recht. in den 1990er-jahren wurde in dublin ein übereinkommen beschlossen, das den staaten am rand der eu die höchste last auferlegt. jener staat, den ein schutzsuchender zuerst betritt, ist für dessen registrierung und sein asylverfahren zuständig. wird ein flüchtling in einem anderen eu-land aufgegriffen, kann er in diesen erststaat zurückgeschoben werden. so weit die theorie. die realität: griechenland registriert flüchtlinge - und schickt sie weiter. dann registriert ungarn dieselben flüchtlinge - und schickt sie wieder weiter. österreich registriert sie nicht, schickt sie aber auch weiter. das ziel ist deutschland oder schweden, die diese menschen eigentlich wieder nach griechenland oder ungarn zurückschicken könnten, was sie aber nicht tun. dafür beginnt deutschland wieder mit grenzkontrollen und pocht auf dublin. der vorschlag der eu-kommission, über den die innenminister heute zu befinden haben, sieht den absprung aus diesem wahnsinn vor. eine verbindliche quote von flüchtlingen, die jedes eu-land aufzunehmen hat, also eine art umsiedlungsmechanismus, hebelt dublin aus. viel wichtiger aber: ein ja aller 28 staaten zu dieser neuen grundsatzlösung ist gleichzeitig ein bekenntnis zur gemeinsamen verantwortung. die flüchtlinge, die an europas grenzen aus welchen gründen auch immer anklopfen, sind nicht sache und problem deutschlands, österreichs oder schwedens, sondern der europäischen union. das wäre schritt eins. schritt zwei wäre der ebenfalls von der eu-kommission vorgeschlagene notfallplan: sollte es erneut zu einer akuten überlastung eines landes kommen, kann der verteilungsschlüssel wieder in kraft gesetzt werden. viele weitere maßnahmen müssen folgen, bis ein stabiles europäisches asylrecht steht. die verteilung von flüchtlingen wird das problem nicht lösen. doch die längste reise beginnt mit dem ersten schritt. genau den wollen länder wie das vom rechtsnationalen viktor orbán regierte ungarn nicht tun. sie verweigern solidarität. österreichs kanzler werner faymann hat vor diesem hintergrund darauf hingewiesen, dass in diesem fall auch eine andere solidarität hinterfragt werden könnte - nämlich die der nettozahler in die strukturfonds, die vorwiegend den osteuropäern zugutekommen. ihm ist recht zu geben. & 358 & very low & Low & Power & NA & NA & 2015-09-13 & 2015 & 1 & POL
Frame & v.low & Regional & <500 & -0.7948903 & -0.1706634 & 1.0035124 & 0.5015415 & -0.3119516 & 9.0 & 0.2648514 & -0.9910023 & Payer & Domestic & European & Mixed & Domestic|POL & Neutral\\
\addlinespace
Austria & https://www.ots.at/presseaussendung/OTS\_20190507\_OTS0163/lsth-ruedisser-vorarlberg-profitiert-von-europa & 58 & OTS.at & Private/Non-Public & Online only & National & low = CP mentioned more times but NOT important part of story (mainly about others issues) & Institutional bargaining over funding & Balanced & Subnational & No myth & NA & NA & NA & NA & NA & NA & NA & NA & Austria & lsth. rüdisser: vorarlberg profitiert von europa & 2019-05-07 & europäischer sozialfonds & bregenz (ots) - (vlk) - der traditionelle europatag am 9. mai steht heuer unter besonderen vorzeichen: zum einen treffen an diesem tag die eu-staats- und regierungschefs im rumänischen sibiu zusammen, um über die zukunft der union zu beraten. zum anderen finden in wenigen wochen die wahlen zum europäischen parlament statt. im vorfeld dieser ereignisse erläuterte landesstatthalter karlheinz rüdisser im heutigen (dienstag, 7. mai) pressefoyer die positiven auswirkungen der eu-mitgliedschaft auf die entwicklung vorarlbergs. die eu-kofinanzierten projekte im lande seien eine erfolgsgeschichte, einige davon werden im rahmen eines tages der offenen tür unter dem titel "europa im ländle" am 16. mai der interessierten öffentlichkeit vorgestellt. vorarlberg hat in der eu von vielfältigen faktoren profitiert - durch offene binnengrenzen, gemeinsame währung, personenfreizügigkeit, handelserleichterungen sowie impulse aus gemeinschaftsprogrammen zur stärkung der länderübergreifenden vernetzung und zusammenarbeit in grenzregionen. landesstatthalter rüdisser verwies auf die wirtschaftsdaten. so hat sich das vorarlberger exportvolumen innerhalb von zwei jahrzehnten mehr als vervierfacht, auf mehr als zehn milliarden euro im jahr. auch die zahl der beschäftigten ist kontinuierlich gestiegen und hat im jahresdurchschnitt 2018 einen höchststand von 176.000 erreicht. besonders markant ist die entwicklung des außenhandels mit den im rahmen der eu-osterweiterung 2004 beigetretenen ländern. deren anteil am vorarlberger exportvolumen ist innerhalb von zwölf jahren von 8 auf 15 prozent gestiegen. die vorteile der eu-mitgliedschaft beschränken sich aber nicht auf die wirtschaftliche entwicklung, sondern die eu habe sich auch als friedensprojekt bewährt, so rüdisser. rein finanziell ergibt aus der eu-mitgliedschaft für vorarlberg im wesentlichen eine ausgeglichene bilanz. die einzahlungen des landes in den eu-haushalt betrugen im jahr 2018 rund 25 millionen euro. in der förderperiode 2014-2020 fließen bis dato aus eu-fördertöpfen (europäischer fonds für regionale entwicklung, interreg-programm "alpenrhein-bodensee-hochrhein", europäischer sozialfonds, agrarförderungen usw.) im schnitt jährlich ca. 32 millionen euro zurück. insgesamt kann vorarlberg in der laufenden programmperiode 2014-2020 mehr als 220 millionen euro an eu-förderungen lukrieren. beim tag der offenen tür am donnerstag, 16. mai, werden 19 ausgewählte projekte eindrucksvoll die vielfalt und den mehrwert von eu-förderungen für die region und die regionale bevölkerung aufzeigen. mehr dazu siehe auf www.vorarlberg.at/eu. & 356 & low & Low & Power & NA & NA & 2019-05-07 & 2019 & 3 & POL
Frame & low-medium & National & <500 & -0.7948903 & -0.1706634 & 1.0035124 & 0.5015415 & -0.3119516 & 9.0 & 0.2648514 & -0.9910023 & Payer & Domestic & Domestic & Domestic & Domestic|POL & Neutral\\
Austria & http://www.tt.com/politik/innenpolitik/12066182-91/länder-wollen-500-millionen-euro-pro-jahr-mehr-vom-bund.csp & 30 & Tiroler Tageszeitung Online & Private/Non-Public & Online and Offline & Regional/Local & very low = CP mentioned once & Political leverage & Negative & Subnational & No myth & Institutional bargaining over funding & Negative & Subnational & No myth & NA & NA & NA & NA & Austria & länder wollen 500 millionen euro pro jahr mehr vom bund | tiroler tageszeitung online - nachrichten von jetzt! & 2016-09-27 & kohäsionsfonds & graz, innsbruck - die landesfinanzreferenten fordern vom bund 500 millionen euro jährlich im neuen finanzausgleich. "das ist angesichts der bundesgesetzlichen mehrbelastungen keine überbordende forderung", sagte der vorsitzende, der steirische lhstv. michael schickhofer (spö) am dienstag nach dem treffen in graz. auch tirols landeshauptmann günther platter (övp) betonte in einer aussendung: ""es geht hier vor allem um mehrkosten in der pflege, der kinderbetreuung, im gesundheitsbereich, insbesondere durch das krankenanstaltenarbeitszeitgesetz.". auch bei der verwendung der bankenabgabe neu sehen sich die länder benachteiligt, soll diese doch eine reine bundesabgabe sein. das widerspreche dem bestehenden finanzausgleich. "bisher kam die bankenabgabe bund und ländern zugute, eine einseitige änderung können wir nicht akzeptieren", so platter. die bundesländer verlangen daher eine beteiligung an den einnahmen aus der bankenabgabe bzw. einen entsprechenden ersatz. 500 millionen "mehr als gerechtfertigt" schickhofer sagte nach der konferenz, die länder seien reformbereit: "aber wir sind damit konfrontiert, dass der bund dinge beschließt und regelt, die schärfer sind als die eu fordert. es ist mehr als gerechtfertigt, dass die länder 500 millionen euro mehr pro jahr erhalten und dass der pflegefonds aufgestockt wird". weiters sei für die länder das thema wohnen zentral: "die vom bund zugesagten 180 millionen euro müssen unkompliziert fließen", sagte der steirische finanzreferent. die länder seien auch bereit, die finanzierung der modellprojekte kinderbetreuung und ganztagsschule zu diskutieren, sie wollten aber keine noch schwerfälligeren systeme. "wir wollen einen schlanken, einfachen und gerechten finanzausgleich", sagte schickhofer. ober- und niederösterreich wollen mehr geld zulasten tirols auffassungsunterschiede zwischen den ländern gab es im vorfeld des treffens in graz auch zwischen den bundesländen. die bevölkerungsreichen länder wie ober- und niederösterreich wollen den sogenannten horizontalen aufteilungsschlüssel ändern. sie pochten darauf, dass auch die einwohnerzahl verstärkt berücksichtigt wird. das würde ihnen mehreinnahmen von bis zu 70 millionen euro pro jahr garantieren. demgegenüber müssten tirol und vorarlberg mit einbußen von rund 25 millionen euro rechnen. diese frage sei, so platter bei der sitzung in graz, nicht weiter vertieft worden: "ich habe immer betont, dass es strategisch richtig ist, sich zuerst auf eine gemeinsame verhandlungsposition gegenüber dem bund festzulegen. das ist jetzt passiert, die länder sprechen mit einer stimme." schickhofer meinte dazu auf journalistenfragen, die strategie sei festgelegt. am 21. oktober gebe es weitere verhandlungen mit dem bund und danach noch einmal ein außerordentliches treffen der referenten in der steiermark. die frage der kosten der flüchtlingsbetreuung sprach oberösterreichs landeshauptmann josef püringer an. wenn die eu nicht eine faire verteilung von flüchtlingen zustande bringe, müssten die die hauptlast tragenden länder unterstützt werden. "im kohäsionsfonds (daraus finanziert die eu vorhaben in den bereichen umwelt und transeuropäische netze , anm. d. red.) sind nicht alle mittel abgeholt. durch kommission und rat könnten sie einer anderen verwendung zugeführt werden. was bisher gezahlt wurde, ist ein eher symbolischer beitrag", so pühringer. (apa, tt.com) mehr zum thema: streit zwischen den ländern: http://bit.ly/2dzujwo kommentieren kommentar schreiben schlagworte günther platter innsbruck kohäsionsfonds länderfinanzausgleich michael schickhofer tirol vorarlberg & 481 & very low & Low & Power & Power & NA & 2016-09-27 & 2016 & 2 & POL
Frame & v.low & Regional & <500 & -0.7948903 & -0.1706634 & 1.0035124 & 0.5015415 & -0.3119516 & 9.0 & 0.2648514 & -0.9910023 & Payer & Domestic & Domestic & Domestic & Domestic|POL & Negative\\
Austria & http://www.salzburg.com/nachrichten/meinung/kolumne/hinter-den-zahlen/sn/artikel/der-brexit-und-die-finanzen-der-eu-247912/ & 83 & Salzburger Nachrichten & Private/Non-Public & Online and Offline & Regional/Local & low = CP mentioned more times but NOT important part of story (mainly about others issues) & Institutional bargaining over funding & Balanced & EU + Other country & No myth & NA & NA & NA & NA & NA & NA & NA & NA & Austria & der brexit und die finanzen der eu & 2017-05-16 & kohäsionsfonds & bei den verhandlungen über den brexit gilt die aufmerksamkeit derzeit den einmaleffekten, sprich den scheidungskosten. das sind jene verpflichtungen, die großbritannien als eu-mitglied eingegangen ist wie pensionen für eu-beamte oder zahlungsverpflichtungen aus mehrjährigen projekten. laut kommission sind das ca. 100 mrd. euro, der endgültige betrag wird erst nach den verhandlungen feststehen. daneben sind die strukturellen effekte des brexit auf den eu-finanzrahmen die langfristig größere herausforderung. die briten sind nicht nur mit 10 mrd. euro pro jahr einer der großen nettozahler in der eu, durch ihr ausscheiden verschieben sich auch die finanzierungsanteile der mitgliedsstaaten maßgeblich. das liegt zum einen in dem mit ausnahmen durchsetzten eu-finanzrahmen und zum anderen in der festlegung des rates, dass der eu-haushalt auf 1 prozent des eu-bruttonationaleinkommens zu beschränken ist. das hat zur folge, dass die nettozahler durch den austritt großbritanniens relativ stärker zur kasse gebeten würden als nettoempfänger. vor allem die niederlande, schweden, deutschland und österreich müssten durch den wegfall bisheriger rabatte signifikante höhere beiträge in kauf nehme, österreich knapp 400 mill. euro. daher ist die reaktion der nettozahler, der ausfall der britischen beiträge solle durch einsparungen kompensiert werden, prima vista verständlich. doch einsparungen von 10 mrd. euro sind höchst unrealistisch. zur besseren veranschaulichung: 10 mrd. euro sind mehr als die gesamten verwaltungskosten der eu (8,2 mrd. euro), oder das budget für die eu-außenpolitik "globales europa" (9,2), oder die eu-forschungsförderung "horizon 2020" (9,5). und es sind immerhin 20 prozent der ausgaben für die struktur- und kohäsionsfonds oder rund ein fünftel des haushalts für die gemeinsame agrarpolitik. mit einsparungen allein wird man also keine kompensation schaffen. und selbst wenn man die finanzierungslücke je zur hälfte durch einsparungen und beitragserhöhung schließen würde, ändert das nichts an der grundsätzlichen problematik einer notwendigen reform des eu-finanzierungssystems. der brexit bietet dazu eine chance. die nettozahler könnten bei den 2018 beginnenden verhandlungen über den neuen eu-finanzrahmen einer vorläufigen erhöhung zustimmen und im gegenzug eine tiefgreifende reform bei einnahmen und ausgaben fordern. das böte die einmalige chance, ein finanzierungssystem zu entwickeln, das den gemeinsamen europäischen zielen rechnung trägt und die dafür nötigen mittel bereitstellt. einnahmenanteile aus einer eu-weiten co2-abgabe oder einer steuer für emissionsüberschreitungen würden sich dafür ebenso eignen wie eine finanztransaktionssteuer. vorschläge gibt es genug, aber es braucht politischen mut und kreativität, sie umzusetzen. vielleicht schafft das eine neue deutsch-französische achse. & 395 & low & Low & Power & NA & NA & 2017-05-16 & 2017 & 2 & POL
Frame & low-medium & Regional & <500 & -0.7948903 & -0.1706634 & 1.0035124 & 0.5015415 & -0.3119516 & 9.0 & 0.2648514 & -0.9910023 & Payer & European & European & European & European|POL & Neutral\\
Austria & https://kurier.at/wirtschaft/asylberechtigte-leitl-denkt-an-praemie-fuer-firmen/309.219.112 & 49 & Kurier & Private/Non-Public & Online and Offline & National & medium = CP is important part of story & Jobs & Balanced & National & No myth & Political leverage & Balanced & National & No myth & NA & NA & NA & NA & Austria & asylberechtigte: leitl denkt an prämie für firmen & 2018-02-01 & kohäsionsfonds & wkö-chef: unternehmen sollen 1.000 euro pro monat als "integrationsprämie" erhalten, wenn sie einen asylberechtigten anstellen. statt um verteilungsquoten zu ringen, sollten die eu-mitgliedstaaten die kohäsionsfonds dazu nutzen, migranten in unternehmen auszubilden und zu beschäftigen: wkö-chef christoph leitl denkt in einem gespräch mit "frankfurter allgemeinen zeitung" (faz) 1.000 euro pro monat für jeden eingestellten asylberechtigten an - drei jahre lang. "die integration kann nur über die betriebe laufen, deshalb müsste man sie aus den kohäsionsfonds fördern." unternehmen in der eu sollen für jeden eingestellten asylberechtigten drei jahre lang 1.000 euro pro monat erhalten, so leitl, der auch präsident der vereinigung der europäischen wirtschaftskammern eurochambres ist. damit sich einheimische nicht benachteiligt fühlen, sollten neben den ausländern auch langzeitarbeitslose hilfen aus den fonds für arbeit und soziales erhalten. eine solche "integrationsprämie", für die die europäische gemeinschaft einen großteil der kosten trüge, würde die aufnahmebereitschaft für asylberechtigte deutlich erhöhen, glaubt leitl laut "faz". zugleich könnte es die einbeziehung in den arbeitsmarkt und in die gesellschaft der gastländer fördern. "die integrationsprämie könnte helfen, europäische solidarität nicht über eine quote zu erreichen, was immer verärgerung und bevormundungsängste auslöst, sondern über ein anreizmodell", resümierte leitl. finanzieren ließe sich das programm aus nicht abgerufenen mitteln am ende einer förderperiode. der plan stehe im einklang mit der systematik und den richtlinien der europäischen kohäsionsfonds. leitl (övp) sagte zur deutschen zeitung auch, dass es unbehagen über die fpö-regierungsbeteiligung gebe: in brüssel werde er darauf angesprochen. "der unterschied zu 2000 (schwarz-blau 1) ist, dass man sich heute in europa damit abgefunden hat, dass rechtsparteien in regierungen sind", sagte leitl. "ich sehe das sehr gelassen." es sei gut, rechte kräfte einzubinden. "wenn sie draußen populistisch agieren, werden sie immer stärker. wenn sie aber in der verantwortung sind, müssen sie konstruktiv mitwirken und nicht demagogisch." & 299 & medium & Medium & Socio-Economic & Power & NA & 2018-02-01 & 2018 & 3 & ECO
Frame & low-medium & National & <500 & -0.7948903 & -0.1706634 & 1.0035124 & 0.5015415 & -0.3119516 & 9.0 & 0.2648514 & -0.9910023 & Payer & Domestic & Domestic & Domestic & Domestic|ECO & Neutral\\
Austria & http://orf.at/stories/2424843/ & 18 & newsORF.at & Public & Online only & National & medium = CP is important part of story & Social awareness/inclusion & Positive & National & No myth & Political leverage & Balanced & National & No myth & Jobs & Balanced & National & No myth & Austria & leitl will "prämie" für beschäftigung von asylberechtigten & 2018-02-01 & kohäsionsfonds & statt um verteilungsquoten zu ringen, sollten die eu-mitgliedstaaten die kohäsionsfonds dazu nutzen, migranten in unternehmen auszubilden und zu beschäftigen, so wirtschaftskammer-chef christoph leitl in einem gespräch mit der "frankfurter allgemeinen zeitung" ("faz", donnerstag-ausgabe). "die integration kann nur über die betriebe laufen, deshalb müsste man sie aus den kohäsionsfonds fördern." unternehmen in der eu sollen für jeden eingestellten asylberechtigten drei jahre lang 1.000 euro pro monat erhalten, so leitl, der auch präsident der vereinigung der europäischen wirtschaftskammern eurochambres ist. damit sich einheimische nicht benachteiligt fühlen, sollten auch langzeitarbeitslose hilfen aus den fonds für arbeit und soziales erhalten. eine solche "integrationsprämie", für die die europäische gemeinschaft einen großteil der kosten trüge, würde die aufnahmebereitschaft für asylberechtigte deutlich erhöhen, glaubt leitl laut "faz". zugleich könnte es die einbeziehung in den arbeitsmarkt und in die gesellschaft der gastländer fördern. "die integrationsprämie könnte helfen, europäische solidarität nicht über eine quote zu erreichen, was immer verärgerung und bevormundungsängste auslöst, sondern über ein anreizmodell", resümierte leitl. finanzieren ließe sich das programm aus nicht abgerufenen mitteln am ende einer förderperiode. der plan stehe im einklang mit der systematik und den richtlinien des kohäsionsfonds der europäischen union. & 192 & medium & Medium & Socio-Economic & Power & Socio-Economic & 2018-02-01 & 2018 & 3 & ECO
Frame & low-medium & National & <500 & -0.7948903 & -0.1706634 & 1.0035124 & 0.5015415 & -0.3119516 & 9.0 & 0.2648514 & -0.9910023 & Payer & Domestic & Domestic & Domestic & Domestic|ECO & Positive\\
\addlinespace
Austria & http://www.kleinezeitung.at/politik/aussenpolitik/5437574/ & 45 & Kleine Zeitung & Private/Non-Public & Online and Offline & Regional/Local & very high = CP is most important issue + CP is mentioned in title/headline & Institutional bargaining over funding & Factual & EU + Other country & No myth & NA & NA & NA & NA & NA & NA & NA & NA & Austria & neuer schwerpunkt: kommission will mehr eu-geld in krisenländer wie italien leiten & 2018-05-29 & kohäsionsfonds & osteuropäische länder werden weniger geld aus dem milliardenschweren regional- und kohäsionsfonds erhalten. die eu-kommission hat die im nächsten langfrist-budget geplante umschichtung von hilfen für strukturschwache regionen richtung südeuropa verteidigt. die bisherigen hauptempfänger in osteuropa hätten die unterstützungen richtig eingesetzt, sagte haushaltskommissar günther oettinger am dienstag im europaparlament in straßburg. "etwa die slowakei, das baltikum oder polen bekommen in unserem vorschlag weniger geld, weil sie wettbewerbsstärker geworden sind." andere länder, die in den letzten jahren länger in der stagnation gewesen seien wie italien, bekämen dann mehr geld. darüber hinaus gehe er fest davon aus, dass einige der osteuropäischen eu-länder im nächsten jahrzehnt von der wirtschaftskraft her zum europäischen durchschnitt aufholen und erstmals auch geld in das budget einzahlen müssten. "die kohäsionspolitik war erfolgreich", sagte oettinger. die eu-kommission beschließt am dienstag ihre pläne für die zukunft der regional- und kohäsionsfonds. sie sollen eine angleichung der lebensverhältnisse in der eu fördern und sind nach den agrarausgaben der größte posten im eu-budget. die eu-kommission hat vorgeschlagen, dafür im nächsten eu-finanzrahmen von 2021 bis 2027 rund 373 milliarden euro zur verfügung zu stellen. die eu-kommission unterstützt dabei auch den deutschen vorschlag, gebiete mit einer hohen zahl von flüchtlingen künftig stärker zu berücksichtigen. dies könnte zu lasten osteuropäischer staaten gehen, welche die flüchtlingsaufnahme verweigern. hauptankunftsländer für flüchtlinge wie italien oder griechenland, aber auch deutsche regionen könnten davon profitieren. allerdings wird deutschland wegen seiner großen wirtschaftskraft nach dem brüsseler vorschlag insgesamt weniger bekommen. denn wegen des eu-austritts großbritanniens und neuer aufgaben bei verteidigung, grenzschutz und forschung sollen bei der kohäsionspolitik einsparungen erfolgen. den plänen müssen das europaparlament und die mitgliedstaaten noch zustimmen. & 274 & very high & High & Power & NA & NA & 2018-05-29 & 2018 & 3 & POL
Frame & high-very high & Regional & <500 & -0.7948903 & -0.1706634 & 1.0035124 & 0.5015415 & -0.3119516 & 9.0 & 0.2648514 & -0.9910023 & Payer & European & European & European & European|POL & Neutral\\
Austria & http://www.tt.com/politik/europapolitik/12802345-91/london-erwartet-eine-gesalzene-brexit-rechnung.csp & 60 & Tiroler Tageszeitung Online & Private/Non-Public & Online and Offline & Regional/Local & very low = CP mentioned once & Institutional bargaining over funding & Factual & EU + Other country & No myth & NA & NA & NA & NA & NA & NA & NA & NA & Austria & london erwartet eine & 2017-03-29 & kohäsionsfonds & martin trauth, afp brüssel - sie gilt als eines der schwierigen kapitel in den austrittsverhandlungen mit großbritannien: die austrittsrechnung. die eu will von london milliarden, wenn das land die union verlässt. eu-kommissionspräsident jean-claude juncker hat bereits angekündigt, die rechnung werde "sehr gesalzen" ausfallen. warum fordert die eu von london geld ? die eu plant ihr budget über zeiträume von sieben jahren. der aktuelle zeitraum läuft bis einschließlich 2020 - also bis fast zwei jahre nach dem geplanten brexit im märz 2019. bis dahin sind praktisch alle eu-mittel schon verplant - und ein teil der eingegangenen verpflichtungen für bestimmte projekte bezieht sich auch noch auf die jahre nach 2020. "das ist keine strafe", sagte ein eu-vertreter zu der austrittsforderung. "es geht nur darum, dass großbritannien das bezahlt, was es zugesagt hat." um was für posten geht es? betroffen ist letztlich jegliche form von eu-ausgaben einschließlich des eu-kohäsionsfonds zur angleichung der lebensverhältnisse zwischen ärmeren und reicheren mitgliedstaaten. ein zentraler punkt sind auch die pensionszahlungen für eu-beamte. "es geht dabei nicht um die pensionen für britische beamte, sondern den anteil an den pensionen für alle eu-beamte", sagt ein diplomat. als daneben besonders problematisch gilt der britische anteil an krediten, die über die europäische investitionsbank (eib) vergeben werden. wie hoch könnte die rechnung ausfallen? offizielle zahlen gibt es nicht. aus eu-kreisen heißt es aber, die kommission komme auf einen betrag von 55 bis 60 milliarden euro. andere quellen nennen 40 bis 60 milliarden euro. könnte london die forderung verringern? das wird die britische regierung in den verhandlungen sicher versuchen. premierministerin theresa may sagte anfang märz, london werde nach dem austritt sicher nicht "gewaltige summen" an die eu zahlen. in brüssel wird aber schon im vorhinein versuchen eine absage erteilt, etwa den anteil großbritanniens an eu-gebäuden abzuziehen. "das ist wie mit einer mitgliedschaft im golfclub", sagt ein diplomat. "wenn man austritt, bekommt man auch nicht den anteil am clubhaus ausgezahlt." und wenn london einfach geht, ohne zu zahlen? eine anfang märz veröffentlichte studie des britischen oberhauses kommt zu dem schluss, dass großbritannien die eu auch ohne weitere zahlungen verlassen könnte. london scheine hier "eine starke rechtliche position" zu haben, sagte die vorsitzende des finanzausschusses, kishwer falkner. es gebe aber auch "mögliche gewinne aus anderen bereichen der verhandlungen", auf die großbritannien dann verzichten würde. hier könnte es etwa um den status von britischen bürgern in der eu gehen oder den künftigen zugang für unternehmen zum europäischen binnenmarkt. "wenn es keine einigung auf die austrittsrechnung gibt, gibt es überhaupt kein austrittsabkommen", sagt ein eu-vertreter. kommentieren kommentar schreiben schlagworte brexit brüssel großbritannien jean-claude juncker london theresa may & 437 & very low & Low & Power & NA & NA & 2017-03-29 & 2017 & 2 & POL
Frame & v.low & Regional & <500 & -0.7948903 & -0.1706634 & 1.0035124 & 0.5015415 & -0.3119516 & 9.0 & 0.2648514 & -0.9910023 & Payer & European & European & European & European|POL & Neutral\\
Austria & https://derstandard.at/2000078493159/EU-will-in-Osteuropa-Milliarden-einsparen & 87 & der Standard & Private/Non-Public & Online and Offline & National & low = CP mentioned more times but NOT important part of story (mainly about others issues) & Institutional bargaining over funding & Balanced & EU + Other country & No myth & Political leverage & Factual & EU + Other country & No myth & NA & NA & NA & NA & Austria & eu will in osteuropa milliarden einsparen & 2018-04-23 & kohäsionsfonds & mit dem brexit verliert die eu milliarden im budget. ab 2020 soll es deshalb kürzungen bei regionalförderungen geben "restrukturierung" - das ist eines jener zauberworte, die experten des für budget und personal zuständigen eu-kommissars günther oettinger in brüssel am häufigsten über die lippen kommen, wenn es um das künftige gemeinschaftsbudget geht. sie haben derzeit die schwierige aufgabe, pläne für den nächsten "mittelfristigen finanzrahmen" zu erstellen, der ende 2021 beginnt. der geltende, auf sieben jahre angelegte budgetplan war 2014 noch unter der bedingung gestartet, dass großbritannien als großer nettozahler neben deutschland mit von der partie war. nach dem vorgesehenen eu-austritt des landes im märz 2019 werden im haushalt pro jahr netto zwischen zehn und vierzehn milliarden euro fehlen, schätzt oettinger, je nachdem, wie die austrittsbedingungen ausfallen. ausgabenstruktur verändern deutschland und frankreich sind zwar bereit, ihre beiträge zu erhöhen, aber nur, wenn die ausgabenstruktur verändert wird. kanzlerin angela merkel und präsident emmanuel macron haben zuletzt den druck beim vorschlag erhöht, auszahlungen aus den eu-kohäsionstöpfen zur förderung einkommensschwacher regionen an die "innere solidarität" bei der eu-weiten aufteilung von flüchtlingen zu knüpfen. die staaten sind total uneinig. das problem der finanzierung der eu-budgets wird zudem erschwert, weil viele kleinere nettozahlerländer wie österreich, die niederlande, schweden und die slowakei nicht bereit sind, den brexit durch höhere beiträge auszugleichen. sie drängen auf knappe sparbudgets, kürzungen bei subventionen für die osteuropäer, wie bundeskanzler sebastian kurz im standard gefordert hatte. auf der anderen seite drohen jene osteuropäischen länder wie polen oder ungarn, die am meisten profitieren, aber gegen umverteilung von flüchtlingen sturm laufen, mit blockaden. die kommission bereitet aus all dem nun einen gesamtvorschlag vor, den oettinger am 2. mai vorlegen will. "britische budgetlücke" der budgetkommissar wird zur bedeckung der "britischen budgetlücke" vorschlagen, dass die hälfte durch beitragserhöhungen der mitglieder, die andere hälfte durch sparen ausgeglichen wird. die "financial times" nannte montag weitere details zur deutschen forderung, die kohäsionsfonds zu reformieren. zwischen 2014 und 2020 werden daraus rund 350 milliarden euro verteilt werden, 77 milliarden an polen, 22 an ungarn. derzeit zählt vor allem die schwache kaufkraft in einer region (so wie früher das burgenland ein ziel-a-gebiet war) als kriterium, um in den genuss von förderungen zu kommen. in zukunft soll die beschäftigungslage ebenso eine rolle spielen wie der umweltschutz, oder inwieweit länder die rechtsstaatlichkeit achten (ein klarer hinweis auf polen); ob sie "werte" der union einhalten und ob sie bereit sind, flüchtlinge aufzunehmen. das bedeutete in der praxis, dass milliardenbeträge in zweistelliger höhe, die bisher nach osteuropa flossen, in zukunft wieder mehr in die südeuropäischen länder wie griechenland, italien und spanien fließen würden. nach der eu-erweiterung 2004 war es genau umgekehrt. ob das alles je so kommt, steht freilich in den sternen: um den budgetrahmen zu beschließen, braucht es die einstimmigkeit der eu-staaten. die verhandlungen sollen unter österreichischem eu-vorsitz starten. (thomas mayer aus brüssel, 23.4.2018) & 479 & low & Low & Power & Power & NA & 2018-04-23 & 2018 & 3 & POL
Frame & low-medium & National & <500 & -0.7948903 & -0.1706634 & 1.0035124 & 0.5015415 & -0.3119516 & 9.0 & 0.2648514 & -0.9910023 & Payer & European & European & European & European|POL & Neutral\\
Austria & https://www.derstandard.at/story/2000010299807/madeiras-ewiger-praesident-nach-37-jahren-zurueckgetreten & 4 & der Standard & Private/Non-Public & Online and Offline & National & very low = CP mentioned once & Economic development & Positive & Other country & No myth & Fraud/Corruption & Negative & Other country & No myth & NA & NA & NA & NA & NA & madeiras ewiger präsident nach 37 jahren zurückgetreten & 2015-01-12 & NA & NA & 208 & very low & Low & Socio-Economic & Governance & NA & 2015-01-12 & 2015 & 1 & ECO
Frame & v.low & National & <500 & -0.7948903 & -0.1706634 & 1.0035124 & 0.5015415 & -0.3119516 & 9.0 & 0.2648514 & -0.9910023 & Payer & European & European & European & European|ECO & Positive\\
Austria & https://diepresse.com/home/ausland/eu/5437500/Kommission-will-mehr-EUGeld-in-Krisenlaender-wie-Italien-leiten & 78 & Die Presse & Private/Non-Public & Online and Offline & National & high = CP is most important issue in story (can also cover other issues) & Solidarity to poor countries/regions & Positive & EU + Other country & No myth & Institutional bargaining over funding & Balanced & EU + Other country & No myth & NA & NA & NA & NA & Austria & kommission will mehr eu-geld in krisenländer wie italien leiten & 2018-05-29 & kohäsionsfonds & osteuropäische länder werden weniger geld aus dem milliardenschweren regional- und kohäsionsfonds erhalten - sie seien "wettbewerbsstärker" geworden, so haushaltskommissar oettinger. die eu-kommission hat die im nächsten langfrist-budget geplante umschichtung von hilfen für strukturschwache regionen richtung südeuropa verteidigt. die bisherigen hauptempfänger in osteuropa hätten die unterstützungen richtig eingesetzt, sagte haushaltskommissar günther oettinger am dienstag im europaparlament in straßburg. "etwa die slowakei, das baltikum oder polen bekommen in unserem vorschlag weniger geld, weil sie wettbewerbsstärker geworden sind." andere länder, die in den letzten jahren länger in der stagnation gewesen seien, wie italien, bekämen dann mehr geld. darüber hinaus gehe er fest davon aus, dass einige der osteuropäischen eu-länder im nächsten jahrzehnt von der wirtschaftskraft her zum europäischen durchschnitt aufholen und erstmals auch geld in das budget einzahlen müssten. "die kohäsionspolitik war erfolgreich", sagte oettinger. die eu-kommission beschließt am dienstag ihre pläne für die zukunft der regional- und kohäsionsfonds. sie sollen eine angleichung der lebensverhältnisse in der eu fördern und sind nach den agrarausgaben der größte posten im eu-budget. die eu-kommission hat vorgeschlagen, dafür im nächsten eu-finanzrahmen von 2021 bis 2027 rund 373 milliarden euro zur verfügung zu stellen. die eu-kommission unterstützt dabei auch den deutschen vorschlag, gebiete mit einer hohen zahl von flüchtlingen künftig stärker zu berücksichtigen. dies könnte zu lasten osteuropäischer staaten gehen, welche die flüchtlingsaufnahme verweigern. hauptankunftsländer für flüchtlinge wie italien oder griechenland, aber auch deutsche regionen könnten davon profitieren. allerdings wird deutschland wegen seiner großen wirtschaftskraft nach dem brüsseler vorschlag insgesamt weniger bekommen. denn wegen des eu-austritts großbritanniens und neuer aufgaben bei verteidigung, grenzschutz und forschung sollen bei der kohäsionspolitik einsparungen erfolgen. den plänen müssen das europaparlament und die mitgliedstaaten noch zustimmen. & 281 & high & High & Values & Power & NA & 2018-05-29 & 2018 & 3 & ECO
Frame & high-very high & National & <500 & -0.7948903 & -0.1706634 & 1.0035124 & 0.5015415 & -0.3119516 & 9.0 & 0.2648514 & -0.9910023 & Payer & European & European & European & European|ECO & Positive\\
\addlinespace
Austria & https://diepresse.com/home/meinung/gastkommentar/5602959/Its-the-emigration-stupid\_Wo-Viktor-Orban-irrt & 77 & Die Presse & Private/Non-Public & Online and Offline & National & very low = CP mentioned once & Mismanagement & Negative & Other country & No myth & NA & NA & NA & NA & NA & NA & NA & NA & Austria & it's the emigration, stupid: wo viktor orbán irrt & 2019-03-27 & strukturfonds & gastkommentare und beiträge von externen autoren müssen nicht der meinung der redaktion entsprechen. wahr ist in vielen punkten das gegenteil. ungarn ist nicht gefährdet durch einwanderung, sondern durch fehlende investitionen und arbeitsplätze für junge ungarn, die höhere lebenseinkommen wollen. viele ungarn haben in den fünfzigerjahren aus angst vor dem kommunismus ihr land verlassen. sie wurden in österreich willkommen geheißen, weil ihre vorfahren bei der integration halfen, aber auch, weil ungarn den ruf hatten, fleißig, fröhlich und einfallsreich zu sein. es war natürlich eine gefährliche "umvolkung für österreich", statt reiner germanen kam da ein fremdes volk mit einer sprache, die weder deutsch noch romanisch war, sondern "finnisch-ugrisch". das hätte die verblassende "deutsche" oder die aufkeimende österreichische identität stören können. aber die hilfsbereitschaft war größer, das talent der ungarn wurde erkannt, und die nationale einheit österreichs war nicht gefährdet. auch heute kommen ungarn nach österreich. es ist "wirtschaftliche migration". die menschen wollen ein besseres leben, mehr geld für ihre familien. kellner und bauarbeiter pendeln täglich oder wöchentlich. sie helfen in gastronomie, bauwirtschaft, fremdenverkehr und im burgenländischen gesundheitstourismus. ungarn hat ein emigrationsproblem, keine von soros gesteuerte immigration. die eu prognostiziert, dass die zahl der 20- bis 30-jährigen in ungarn von 1,6 millionen auf eine million im jahr 2050 sinken wird. bei diesem trend gibt es keine neuen betriebe, schulen müssen schließen, die universitäten verlieren an qualität, falls sie nicht vertrieben werden (ceu). ausländer aus dem westen kommen nicht, exilungarn kehren nicht zurück, anträge bei strukturfonds werden nicht gestellt oder für korruption genutzt. 2018 haben nur 600 personen den neuen stacheldraht überwunden. nicht jede gemeinde kann und will migranten als strategie gegen die drohende verödung aufnehmen. aber jede sollte ein konzept entwickeln, wie die region 2050 ausschauen soll, welche stärken sie entwickelt, mit jobs und zukunftsaussicht. typisch für populisten ist, nur zu sagen, was nicht sein soll, einen auslandsfeind aufzubauen und sich die guten alten zeiten mit einem homogenen, christlichen und größeren ungarn zurückzuwünschen. zu orbáns entlastung: dies stimmt nicht nur für ungarn. im baltikum und in südeuropa geht die junge bevölkerung auch um die hälfte zurück. ebenso in zehn bezirken in österreich (vor allem burgenland, kärnten, steiermark). hier sind zukunftsgerichtete "querdenkerkonzepte" gefragt. ein ansatzpunkt ist die qualifizierte migration. und wenn die region dazu zu wenig attraktiv ist, eine qualifizierung der integrationswilligen, die zunächst mit mittlerer qualifikation kommen. sicher falsch ist die abschiebung von lehrlingen. heterogenität ist ein vorteil, wie wir dank der exilungarn in österreich wissen. wer sagt das orbán? & 410 & very low & Low & Governance & NA & NA & 2019-03-27 & 2019 & 3 & POL
Frame & v.low & National & <500 & -0.7948903 & -0.1706634 & 1.0035124 & 0.5015415 & -0.3119516 & 9.0 & 0.2648514 & -0.9910023 & Payer & European & European & European & European|POL & Negative\\
Austria & http://derstandard.at/2000068237749/EU-Austritt-Briten-wollen-Bruessel-mehr-Geld-zahlen & 88 & der Standard & Private/Non-Public & Online and Offline & National & very low = CP mentioned once & Institutional bargaining over funding & Negative & EU + Other country & No myth & NA & NA & NA & NA & NA & NA & NA & NA & Austria & eu-austritt: briten wollen brüssel mehr geld zahlen & 2017-11-22 & kohäsionsfonds & in london ist hinter vorgehaltener hand von einer summe von rund 40 milliarden euro die rede während in der britischen regierung erstmals realistisch über die eigenen finanzverpflichtungen gegenüber der eu gesprochen wird, setzen prominente eu-hasser in der konservativen partei auf neue konfrontation mit brüssel. die regierungskrise in berlin habe die brexit-verhandlungen ins chaos gestürzt, glaubt der frühere parteichef iain duncan smith. deshalb wäre zum jetzigen zeitpunkt ein angebot weiterer zahlungen in die clubkasse "töricht", pflichtet ihm der hardliner jacob rees-mogg bei. "eingegangene verpflichtungen einhalten" genau diese offerte will premierministerin theresa may offenbar am freitag dem eu-ratspräsidenten donald tusk unterbreiten. wie in ihrer florentiner rede angekündigt, werde ihr land seine "während der mitgliedschaft eingegangenen verpflichtungen einhalten". gemeint damit ist nun nicht mehr nur die erfüllung aller zahlungen im laufenden eu-haushaltsrahmen, also bis ende 2020. offenbar will london auch darüber hinausgehende verpflichtungen, etwa für projekte des eu-kohäsionsfonds oder der investmentbank eib, einhalten. hinter vorgehaltener hand ist in london von einer gesamtsumme von rund 40 milliarden euro die rede. dafür holte sich die regierungschefin am dienstag die zustimmung des kabinetts. freilich wollen die briten ihr entgegenkommen an baldige verhandlungen über die zukünftigen handelsbeziehungen mit dem kontinent knüpfen. diese werden von den eu-partnern bisher blockiert, was in london für unverständnis sorgt: insbesondere bei der frage die künftige grenze zwischen der republik irland und das britische nordirland betreffend kann es, so die meinung vieler experten, keinen fortschritt geben, ohne dass wenigstens die konturen der künftigen handelsbeziehungen erkennbar sind. eu-behörden wandern ab zur ernüchterung der hardliner im kabinett hat womöglich die am montag in brüssel endgültig beschlossene abwanderung wichtiger eu-behörden aus london gesorgt. brexit-minister david davis hatte dies noch vor wenigen monaten für keineswegs ausgemacht gehalten. am ende des abstimmungsverfahrens lagen je zwei städte gleichauf, sodass im losverfahren entschieden werden musste. bei der eba zog dublin den kürzeren, die bankenaufsicht geht nun nach paris; für die mit mehr als 1000 hochspezialisierten mitarbeitern deutlich größere arzneimittelbehörde ema erhielt amsterdam den zuschlag vor mailand. jubel in paris, trauer in wien dass das mit seinen bewerberstädten bonn (ema) und frankfurt (eba) leer ausgegangene deutschland künftig großbritannien stärker unterstützen könnte, schloss landwirtschaftsminister christian schmidt (csu) in der bbc aus. auch warnte er vor der von eu-feinden favorisierten möglichkeit, die insel könne im märz 2019 ohne jede vereinbarung aus binnenmarkt und zollunion ausscheiden. "das wäre ein desaster für die britische wirtschaft", sagte schmidt. der brexit sei "kein spiel mit gewinnern und verlierern". die niederlande und frankreich, wo ema bzw. eba nun ihre zelte aufschlagen, sind glücklich. "das ist eine anerkennung für die attraktivität und das engagement frankreichs für europa", twitterte beispielsweise präsident emmanuel macron. weniger glücklich zeigte man sich in österreich. als "höchst bedauerlich" bezeichnete etwa wirtschaftskammerpräsident christoph leitl die doppelte niederlage. doppelt deshalb, weil sich wien hoffnung auf den zuschlag für beide eu-institutionen machen konnte. österreich müsse sich künftig frühzeitig verbündete und allianzpartner suchen und etwa mit den kleinen und mittelgroßen ländern langfristige gemeinsame strategien schmieden. (sebastian borger aus london, 22.11.2017) & 505 & very low & Low & Power & NA & NA & 2017-11-22 & 2017 & 2 & POL
Frame & v.low & National & 500-1000 & -0.7948903 & -0.1706634 & 1.0035124 & 0.5015415 & -0.3119516 & 9.0 & 0.2648514 & -0.9910023 & Payer & European & European & European & European|POL & Negative\\
Austria & https://www.ots.at/presseaussendung/OTS\_20181128\_OTS0065/br-praesidentin-posch-gruska-lud-zur-veranstaltung-centrope-vernetzung-im-europa-der-regionen & 39 & APA-OTS & Private/Non-Public & Online only & National & medium = CP is important part of story & Territorial cooperation & Positive & EU + National + Subnational & No myth & NA & NA & NA & NA & NA & NA & NA & NA & Austria & br-präsidentin posch-gruska lud zur veranstaltung centrope - vernetzung im europa der regionen & 2018-11-28 & kohäsionspolitik & posch-gruska: kulturelle, sprachliche und gesellschaftliche vernetzung kann helfen, eine gemeinsame mitteleuropäische identität zu finden centrope (central europe), ein zusammenschluss von österreich, ungarn, der slowakei und tschechien, feiert sein 15-jähriges gründungsjubiläum mit der präsentation eines buches von hans peter graner über diese organisation. bundesratspräsidentin inge posch-gruska und das "urban forum - egon matzner-institut für stadtforschung" luden zu diesem jubiläum ins parlament in der hofburg ein. "centrope liegt im herzen der europäischen union. die gemeinsamen grenzregionen zwischen österreich, ungarn, der slowakei und tschechien sind seit september 2003 in dieser ursprünglich als 'europa region mitte' gegründeten mitteleuropäischen vorzeigeregion zusammengefasst", ging posch-gruska einleitend auf die geschichte dieser vereinigung ein. "der name centrope entstand aus einem schülerinnen- und schülerwettbewerb in den vier ländern - ein wahrlich völkerverbindendes symbol." vieles sei in diesen 15 jahren auf wirtschaftlichem, gesellschaftlichem und kulturellem gebiet bereits entstanden, sagte die bundesratspräsidentin. vieles sei allerdings noch zu tun, um centrope in den köpfen und vor allem den herzen der bewohnerinnen und bewohner der region zu verankern. die burgenländerin posch-gruska wies insbesondere auf die besondere lage ihres bundeslands hin: "die kulturhistorisch bedingte position des burgenlandes als vermittler in diesem "europa der regionen" hat in der eu-kohäsionspolitik, der schaffung von fördermöglichkeiten für grenzüberschreitende verkehrsinfrastruktur, in bildungs- und forschungsprojekten, in kultur und tourismus sowie in der verhinderung von lohn- und sozialdumping eine bedeutende funktion", betonte sie und forderte gleichzeitig in diesem zusammenhang auch die verantwortung der eu ein: "hier muss allerdings auch die rolle der europäischen union in diesem 'europa der regionen' neu gedacht werden, um das vertrauen der menschen in dieses völkerverbindende europa zu rechtfertigen". die kulturelle, sprachliche und gesellschaftliche vernetzung könne dabei helfen, eine gemeinsame - einst selbstverständliche, jetzt wieder heranreifende - mitteleuropäische identität zu finden. "initiativen in kunst, kultur, forschung, bildung und umwelt sind die bausteine, die eine lebendige, optimistische, zukunftsorientierte und weltoffene region jenseits aller grenzen entstehen lassen", so die bundesratspräsidentin. der bundesrat als länderkammer des österreichischen parlaments bringe in dieses europa der regionen die regionalen bezugspunkte in die europäischen entscheidungsprozesse ein und sei mit seiner expertise ein starkes sprachrohr für die städte und gemeinden in der centrope-region. deshalb werde der bundesrat auch als "europakammer" des parlaments bezeichnet. hans peter graner, autor des bild-text-bandes "fließende grenzen in centrope" gab in einem kurzen vortrag einblick in natur und kultur dieser gemeinsamen lebens- und wirtschaftsregion. "was gibt es verbindenderes als die natur - tiere und pflanzen kennen einfach keine grenzen, bzw. halten sie sich einfach nicht daran." graner zeigte die gemeinsamkeiten und die unterschiede in der region centropa auf. in einer von bernhard müller, dem generalsekretär des urban forum, moderierten talkrunde, an der die botschafterin der republik slowenien ksenija škrilec, der botschafter von ungarn in wien andor nagy, der botschafter der slowakischen republik in wien peter mišík, mojmír jeřábek, erster botschaftssekretär der tschechischen republik, sowie bela hollos, von der österreichischen gesellschaft für politikberatung und politikentwicklung teilnahmen, sollte das erreichte und die zukunft von centrope im herzen mitteleuropas erörtert werden. dabei wurden zahlreiche themen debattiert: der prozess ungarns vom land hinter dem eisernen vorhang bis zu heute war ebenso thema wie die grenzüberschreitende regionalarbeit als eines der wichtigsten elemente der eu-arbeit. auch die grundsätzliche idee der gründung 2003 mit einer eu-anschubfinanzierung und die damaligen wünsche der länder im bereich der vernetzung wurden thematisiert. slowenien ist bis dato kein mitgliedsland von centrope, deshalb unterstrich die botschafterin ksenija škrilec seitens ihres landes: "wir grenzen aneinander und in meinem verständnis sind wir alle ein teil zentraleuropas". die aus jungen musikerinnen und musikern aus den donauländern bestehende internationale donauphilharmonie begleitete die festveranstaltung mit volksliedern und werken renommierter komponisten aus den centrope-ländern musikalisch. entwicklung eines mitteleuropäischen zentralraums die region centrope besteht als eine der jüngsten grenzüberschreitenden europaregionen zwischen "alten" und "neuen" mitgliedstaaten der europäischen union seit dem jahr 2003. sie umfasst regionen und städte ostösterreichs, der westslowakei, südtschechiens und westungarns. die europaregion zeichnet sich durch eine spezifische geographische situation aus: sie beinhaltet grenzregionen von vier mitgliedstaaten, die mit dem fall des eisernen vorhangs im jahr 1989, der eu-osterweiterung im mai 2004 und dem wegfall der grenzkontrollen im dezember 2007 infolge des übereinkommens von schengen ihre jahrzehntelange periphere lage zugunsten einer position in einem grenzüberschreitenden zentralraum wandelten. zudem befinden sich zwei hauptstädte in einer europaweit einzigartigen distanz zueinander: wien und bratislava liegen nur knapp 50 kilometer voneinander entfernt. in der region liegt eines der dynamischsten transnationalen wirtschaftsgebiete europas. das besondere des raums centrope ist seine lage zwischen den etablierten westeuropäischen wirtschaftszentren und den schnell wachsenden märkten im osten europas. die beteiligten regionen weisen heute ein überdurchschnittlich hohes wirtschaftswachstum auf. insgesamt leben rund 15 millionen menschen in der region centrope. in mehreren politischen konferenzen, beginnend mit der gründungskonferenz von september 2003 in kittsee, legten die politisch verantwortlichen der bundesländer, kreise und komitate sowie der städte die zukünftigen leitlinien und gemeinsamen arbeitsfelder für die entwicklung dieses mitteleuropäischen zentralraumes fest. ziel der europaregion ist eine institutionalisierte zusammenarbeit der regionen und städte in den bereichen wirtschaft, infrastruktur, bildung und kultur sowie ein gemeinsames lobbying. (schluss) mar & 831 & medium & Medium & Socio-Economic & NA & NA & 2018-11-28 & 2018 & 3 & ECO
Frame & low-medium & National & 500-1000 & -0.7948903 & -0.1706634 & 1.0035124 & 0.5015415 & -0.3119516 & 9.0 & 0.2648514 & -0.9910023 & Payer & Domestic & European & Mixed & Domestic|ECO & Positive\\
Austria & http://text.derstandard.at/2000019555772/Wifo-Chef-fuer-Recht-auf-Teilzeit-nach-zehn-Jahren-im?ref=rss & 53 & Der Standard & Private/Non-Public & Online and Offline & National & low = CP mentioned more times but NOT important part of story (mainly about others issues) & Mismanagement & Negative & Other country & 8.Mismanaged & NA & NA & NA & NA & NA & NA & NA & NA & Austria & arbeitsmarkt - wifo-chef für recht auf teilzeit nach zehn jahren im betrieb & 2015-07-22 & strukturfonds & karl aiginger: auszeit erleichtern, überstunden nicht mehr steuerlich begünstigen, sechste urlaubswoche nicht vordringlich - 3 fotos wien - wifo-chef karl aiginger übt zwar kritik an den sparauflagen für griechenland, er hält den verbleib des landes in der eurozone aber für alternativlos. damit die eu-politik von den mitgliedsländern besser mitgetragen wird, schlägt aiginger im standard-interview eine machtverschiebung vom rat der regierungschefs zur kommission und zum eu-parlament vor. der ökonom kann sich aber auch halbzeit-abgeordnete vorstellen. bei der arbeitsmarktpolitik plädiert er für eine debatte darüber, "dass rund 500.000 menschen weniger arbeiten wollen, als sie es tatsächlich tun". standard: viele experten haben vorgerechnet, es sei günstiger, griechenland im euro zu halten. kann man das angesichts der politischen unwägbarkeiten in griechenland überhaupt seriös behaupten? aiginger: ich glaube, es ist sowohl für europa als auch für griechenland günstiger. für europa, weil das zerstören einer idee natürlich langfristig kosten verursacht. die chinesen überlegen die ganze zeit, ob sie statt nur in dollar auch in euro investieren sollen. wenn ein land aus dem euro austritt, wären wir von einer reservewährung wieder weit weg. außerdem geht es um geopolitische überlegungen: der ganze balkan orientiert sich entweder an europa oder wird ein loch, das offen ist für politische, wirtschaftliche oder auch militärische instabilität. der friedensschluss im kosovo wurde nur möglich, weil serbien nähere beziehungen zur eu will. wenn europa im 21. jahrhundert eine rolle in der globalisierten welt spielen will, muss es ein geordnetes verhältnis zu seinen nachbarn haben. griechenland und der balkan sind die brücke zu diesen ländern. standard: aber wäre es nicht für griechenland selbst einfacher, mit einer neuen währung den neustart zu wagen? aiginger: wenn es aus dem euro austritt, würde die währung stark abwerten. griechenland würde also schlagartig sehr viel mehr für produkte aus dem ausland zahlen. das trifft vor allem kleine einkommensbezieher und würde das soziale problem noch vergrößern. auch politische unruhen wären nicht auszuschließen. normalerweise würde das land mit einer abgewerteten währung zwar konkurrenzfähiger werden, dafür bräuchte es aber einen industriellen kern, den griechenland derzeit nicht hat. standard: die eu könnte aber auch im falle eines euroausstiegs massive hilfen zusagen: bei der etablierung der neuen währung, für den gesundheitssektor, für hilfsprogramme für arme. aiginger: ich glaube aber eben nicht, dass der neustart dann leichter wäre. der reformdruck wäre geringer, und den bürgern in den anderen eu-ländern wäre es noch weniger zuzumuten, gelder für jemanden bereitzustellen, der nicht euromitglied ist. außerdem ist die frage, ob dann nicht radikale kräfte wie die morgenröte noch mehr zuspruch bekämen. standard: was sagen sie zum konkreten deal mit den geldgebern? aiginger: der deal ist sicher nicht optimal. wieder wurden vor allem budgetziele genannt und keine beschäftigungs- und investitionsziele. es braucht aber vor allem impulse auf der aktivseite. ich schlage daher einen fonds der auslandsgriechen vor. sie sind eine der erfolgreichsten ökonomischen gruppen der welt und finanzieren derzeit durch sammelaktionen kriegsschiffe. wenn europa beim aufbau des fonds hilft, könnten sicher einige milliarden für investitionen organisiert werden. mit ländern wie der schweiz müssen abkommen zur besteuerung von auslandsvermögen geschlossen werden, und auch die griechischen reeder sowie die orthodoxe kirche sollten stärker besteuert werden. brüssel wiederum muss helfen, damit griechenland mittel aus den europäischen strukturfonds abrufen kann. sie schaffen es derzeit nicht, richtige anträge zu schreiben. standard: aber ist es nicht absurd, wenn es nach acht jahren krise noch immer nicht möglich war, die griechische bürokratie zu modernisieren? aiginger: es zeigt, wie langsam diese prozesse vor sich gehen und wie sehr sie stocken, wenn es keinen druck gibt. tsipras ist sicher keiner, der dafür gesorgt hat, dass die verwaltung kleiner und effizienter geworden ist. die meisten gewerbe sind auch noch immer stark reguliert. wenn man zumindest in der tourismussaison sagen würde: jeder jugendliche, der zwei jahre unfallfrei gefahren ist, darf taxi fahren, transporte und zustellungen tätigen. auch kulturführungen sollten leichter möglich sein. das alles würde sofort 50.000 arbeitsplätze bringen. die monopolisten sind aber sehr gut organisiert in griechenland. standard: ist nicht ein grunddilemma in europa: für eine funktionierende währungsunion bräuchte es mehr integration bei wirtschafts-, sozial- und steuerpolitik, das wollen aber weder die bürger noch die politiker. aiginger: ich bin mir nicht sicher, ob die europäer nicht wollen, dass manche sachen europäisch geregelt werden. der widerstand findet meistens bei den kleinen dingen statt - wie den dummen led-lampen. die eu hat leider nie gelernt, welche bereiche auf gemeinschaftsebene geregelt werden sollten und was man besser den mitgliedern überlässt. aus meiner sicht bräuchte es auch eine verschiebung vom rat der regierungschefs zur eu-kommission und zum parlament. und ich kann mir auch vorstellen, dass abgeordnete die erste hälfte der periode im eu-parlament arbeiten und die zweite im jeweiligen nationalen parlament. dann würde kein politiker nur an seine heimische klientel denken, wie sie es jetzt tun. standard: in österreich gilt vor allem der arbeitsmarkt als sorgenkind. kann die regierung kurzfristig überhaupt gegensteuern? aiginger: man kann sicher gegensteuern, indem man den faktor arbeit billiger macht. bei der steuerreform wäre es viel besser gewesen, die lohnnebenkosten zu senken und dafür ökologische steuern einzuführen. österreich könnte sich auch mehr geld beim europäischen fonds für strategische investitionen abholen. wir haben alte sachen wie den koralmtunnel eingereicht. man könnte aber auch eine energieeffizienzoffensive oder eine qualifikationsoffensive einreichen. standard: diskutiert wird aber mehr über arbeitszeitflexibilisierung. geht das an den realen problemen vorbei? aiginger: wenn es irgendwo ein problem gibt, kann man darüber reden. das war immer die stärke der sozialpartnerschaft. generell ist österreich bei der arbeitszeitflexibilisierung aber nicht schlecht aufgestellt. da habe ich mehr sorge um zu eng definierte gewerbe, dass man für zu vieles konzessionen braucht, dass es noch immer schwer und teuer ist, ein unternehmen zu gründen. standard: was halten sie von der debatte über die sechste urlaubswoche? zeitgemäß? aiginger: wir werden in einer phase niedrigen wachstums eine bessere verteilung der arbeit brauchen. ich denke dabei aber zuerst nicht an die sechste urlaubswoche für jeden. aber jeder, der kürzer arbeiten oder eine auszeit nehmen will, sollte das auch können. standard: wie kann man das noch fördern? aiginger: eine möglichkeit wäre, überstunden nicht mehr steuerlich zu begünstigen. man könnte aber auch sagen: nach zehn jahren betriebszugehörigkeit hat jeder ein anrecht darauf, in teilzeit zu gehen oder eine auszeit zu nehmen. wir brauchen eine debatte darüber, dass rund 500.000 menschen weniger arbeiten wollen, als sie es tatsächlich tun. viele davon wären auch bereit, finanzielle einbußen hinzunehmen. (günther oswald, 22.7.2015) & 1060 & low & Low & Governance & NA & NA & 2015-07-22 & 2015 & 1 & POL
Frame & low-medium & National & +1000 & -0.7948903 & -0.1706634 & 1.0035124 & 0.5015415 & -0.3119516 & 9.0 & 0.2648514 & -0.9910023 & Payer & European & European & European & European|POL & Negative\\
Austria & http://orf.at/stories/2425123/ & 48 & newsORF.at & Public & Online and Offline & National & very low = CP mentioned once & Institutional bargaining over funding & Negative & EU + Other country & No myth & NA & NA & NA & NA & NA & NA & NA & NA & Austria & eu-kommissar: bauern werden weniger mittel bekommen & 2018-02-04 & kohäsionsfonds & dieselbe sorge geht auch bei heimischen bauern um, auf deutschland bezogen hat sich eu-haushaltskommissar günter oettinger nun konkret geäußert: "auch" deutsche bauern und die deutschen bundesländer müssen sich nach seinen worten auf weniger geld aus brüssel im nächsten finanzrahmen ab 2021 einstellen. selbiges dürfte allen bauern in der eu bevorstehen, wie bereits mehrfach angedeutet wurde. beim neuen mehrjährigen eu-haushalt werde es zwar "keinen kahlschlag geben, wie einige befürchten", sagte oettinger der "welt am sonntag". "aber auch in deutschland werden sich landwirte und regionen auf finanzielle kürzungen einstellen müssen." die eu-kommission plane, "die agrar- und kohäsionsfonds im neuen mehrjährigen haushalt jeweils um fünf bis zehn prozent zu verkleinern", sagte oettinger. es gäbe bereits überlegungen, wie die kürzungen im landwirtschaftssektor aussehen könnten. "in der agrarpolitik erwägen wir, die direktzahlungen pro hektar fläche künftig degressiv zu gestalten. das bedeutet: ab einer gewissen fläche gibt es dann pro hektar weniger finanzielle unterstützung als für den ersten hektar." die eu-kommission will im mai einen konkreten vorschlag zum nächsten finanzrahmen vorlegen, der ab 2021 gilt. die landwirtschaft und die strukturförderung machen zusammen bisher fast drei viertel aller eu-ausgaben aus. & 188 & very low & Low & Power & NA & NA & 2018-02-04 & 2018 & 3 & POL
Frame & v.low & National & <500 & -0.7948903 & -0.1706634 & 1.0035124 & 0.5015415 & -0.3119516 & 9.0 & 0.2648514 & -0.9910023 & Payer & European & European & European & European|POL & Negative\\
\addlinespace
Austria & http://orf.at/stories/2359830/ & 81 & newsORF.at & Public & Online and Offline & National & low = CP mentioned more times but NOT important part of story (mainly about others issues) & Institutional bargaining over funding & Factual & National + Subnational & No myth & NA & NA & NA & NA & NA & NA & NA & NA & Austria & länder fordern 500 millionen euro mehr vom bund  - news.orf.at & 2016-09-27 & kohäsionsfonds & die landesfinanzreferenten fordern vom bund 500 millionen euro jährlich im neuen finanzausgleich. "das ist angesichts der bundesgesetzlichen mehrbelastungen keine überbordende forderung", sagte der vorsitzende, der steirische landeshauptmann-stellvertreter michael schickhofer (spö)heute nach dem treffen in graz. für die flüchtlingsbetreuungskosten könnte man nicht abgerufene mittel aus dem eu-kohäsionsfonds holen. schickhofer betonte: "alle länder ziehen an einem strang". man habe sich geeinigt, einige kernkapitel mit dem bund abzuarbeiten, wie gesundheit, soziales, flüchtlingsbetreuung. die ausgaben in diesen bereichen, die alle bei den ländern ressortierten, seien seit 2008 um 62 prozent gestiegen, das bip aber nur um 21 prozent, sagte schickhofer. die länder seien auch bei den themen wohnen und neue projekte bei ganztagesbetreuung und -schulen reformbereit, wollten aber keine noch schwerfälligeren systeme. "wir wollen einen schlanken, einfachen und gerechten finanzausgleich", sagte schickhofer. der oberösterreichische landeshauptmann und finanzreferent josef pühringer (övp) vor allem in der flüchtlingsfrage auf mehr eu-gelder: "im kohäsionsfonds sind nicht alle mittel abgeholt. durch kommission und rat könnten sie einer anderen verwendung zugeführt werden. was bisher gezahlt wurde, ist ein eher symbolischer beitrag", so oberösterreichs lh. weiters forderte pühringer eine neue dotierung des pflegefonds bis 2021 statt einer pflegeversicherung, die nur die lohnnebenkosten steigere. die länder seien jedenfalls reformbereit, was den finanzausgleich betreffe. am 21. oktober wird mit dem bund weiterverhandelt. & 213 & low & Low & Power & NA & NA & 2016-09-27 & 2016 & 2 & POL
Frame & low-medium & National & <500 & -0.7948903 & -0.1706634 & 1.0035124 & 0.5015415 & -0.3119516 & 9.0 & 0.2648514 & -0.9910023 & Payer & Domestic & Domestic & Domestic & Domestic|POL & Neutral\\
Austria & https://kurier.at/wirtschaft/strabag-hat-weniger-aber-mit-mehr-gewinn-gebaut/260.792.572 & 24 & Kurier & Private/Non-Public & Online and Offline & National & very low = CP mentioned once & Infrastructure & Positive & National + Other country & No myth & NA & NA & NA & NA & NA & NA & NA & NA & Austria & strabag hat weniger, aber mit mehr gewinn gebaut & 2017-04-27 & kohäsionsfonds & bilanz: aktionären winkt die höchste dividende seit dem börsengang der strabag. ein neubau für den axel-springer-verlag in berlin, ein autobahnabschnitt in polen, ein tunnel in chile - das sind nur drei beispiele für großaufträge, die der heimische baukonzern strabag im vorjahr an land gezogen hat. aktuell werkt die strabag an rund 12.400 baustellen gleichzeitig. ein auftragsbestand, der im vorjahr um 13 prozent auf den rekordwert von 14,8 milliarden euro gestiegen ist, sorgt dafür, dass dem konzern die arbeit nicht ausgeht. im vorjahr hat die strabag aber doch weniger gebaut: die bauleistung ging um sechs prozent auf knapp 13,5 milliarden euro zurück. strabag-chef thomas birtel sieht den grund dafür vor allem darin, dass 2015 mittel aus dem eu-kohäsionsfonds ausgelaufen sind. mit diesen geldern fördert die eu umwelt- und infrastrukturprojekte in ländern, die bei der wirtschaftskraft (pro kopf gerechnet) noch nachhinken. 2015 ist dadurch das geschäft für die strabag in ländern wie der slowakei, polen und tschechien außergewöhnlich gut gelaufen, im vorjahr dann nicht mehr so. für den deutschen markt kann birtel dagegen auf einen boomenden hochbau und einen wachsenden straßenbau verweisen. beim gewinn hat die strabag im vorjahr kräftig ausgebaut: das konzernergebnis zog um 78 prozent auf 278 millionen euro an. die aktionäre dürfen sich über eine erhöhung der dividende von 0,65 auf 0,95 euro je aktie freuen. "das ist die höchste dividende seit dem börsengang", betont birtel. warum glaubt der strabag-boss, dass die bauleistung heuer steigen wird? birtel sieht mehrere gründe: die aufträge, die heuer schon hereingekommen sind, geben grund für optimismus. so verlasse etwa der baumarkt in russland schön langsam wieder "das tal der tränen". der auftrag für eine luxuswohnanlage in moskau über 50 millionen euro sei immerhin die erste größere order aus russland seit eineinhalb jahren gewesen. auch die totalübernahme der projektentwicklungsgesellschaft raiffeisen evolution im vorjahr werde zusätzliche bauaufträgen bringen. alles in allem sollte die bauleistung heuer um vier prozent auf rund 14 milliarden euro steigen, lautet birtels prognose. & 330 & very low & Low & Socio-Economic & NA & NA & 2017-04-27 & 2017 & 2 & ECO
Frame & v.low & National & <500 & -0.7948903 & -0.1706634 & 1.0035124 & 0.5015415 & -0.3119516 & 9.0 & 0.2648514 & -0.9910023 & Payer & Domestic & European & Mixed & Domestic|ECO & Positive\\
Austria & https://www.ots.at/presseaussendung/OTS\_20181010\_OTS0046/staedtebund-ueber-grenzen-hinaus-denken-beim-6-stadtregionstag-in-wels & 50 & APA-OTS & Private/Non-Public & Online only & National & low = CP mentioned more times but NOT important part of story (mainly about others issues) & Social awareness/inclusion & Positive & EU + National & No myth & Institutional bargaining over funding & Balanced & EU & No myth & NA & NA & NA & NA & Austria & städtebund: über grenzen hinaus denken beim 6. stadtregionstag in wels & 2018-10-10 & kohäsionspolitik & wien (ots/rk) - beim 6. österreichischen stadtregionstag in wels (oö), der heute, mittwoch, eröffnet wurde, dreht sich alles um erfahrungen aus der praxis stadtregionalen handelns - und den beitrag, den eu-fördermittel hier leisten können bzw. in der nächsten eu-förderpolitik ab 2020 künftig leisten werden. die vielfältigen prozesse auf der politischen ebene, in der verwaltung, planung und prozessbegleitung sowie fördermodelle hinter stadtregionalen kooperationen, werden beim 6. österreichischen stadtregionstag -eine kooperation von städtebund, land oö und stadt wels - vor den vorhang geholt. der heutige vormittag ist dem erfahrungsaustausch der oberösterreichischen stadtregionen gewidmet, wo bereits einige kooperationen erfolgreich umgesetzt wurden, wobei sich die städte und ihr umland des mehrwerts von gemeinsamer entwicklung und positionierung immer stärker bewusst werden: "städte wachsen und verändern sich. jede veränderung in der stadt hat auswirkungen auf das umland - das sehen wir auch bei uns in wels. wirtschaft, technologie, innovation oder verkehr kennen keine grenzen. hier braucht es gemeinsame lösungen und nachhaltige strukturen. wels wächst mit seinen regionen zunehmend stärker zusammen, ohne dabei die eigene individualität zu verlieren", sagte andreas rabl, bürgermeister der stadt wels. auch bürgermeister hans hingsamer, präsident des oberösterreichischen gemeindebundes, unterstreicht das miteinander von stadt und land: "70 prozent der eu-bürger leben in städten und stadtregionen. stadt und umland brauchen einander - es geht um attraktive lebens-, arbeits-, wirtschafts- und wohnbedingungen. nicht denken in gemeindegrenzen ist gefragt - sondern denken in lebensräumen. österreichs gemeinden leben seit vielen jahren interkommunale zusammenarbeit." neue eu-kohäsionspolitik 2020+ bietet fördermöglichkeiten für innovative stadtregionale projekte jörg wojahn, vertreter der europäischen kommission in österreich betont die rolle der städte für die umsetzung der ziele der europäischen union: "rund zwei drittel der eu-bevölkerung lebt in städten. dort konzentrieren sich wirtschaftliche, umweltpolitische und soziale herausforderungen. konsequenterweise müssen auch lösungsstrategien in den urbanen gebieten ansetzen. der vorschlag der europäischen kommission für die regional- und kohäsionspolitik in der neuen finanzperiode ab 2021 trägt dem rechnung: eine europäische stadtinitiative soll dafür sorgen, dass die städte beispielsweise bei der integration von migranten, dem wohnbau und der energiewende enger zusammenarbeiten. zudem tritt die kommission dafür ein, 6 prozent der mittel aus dem europäischen fonds für regionale entwicklung - er soll von 2021 bis 2027 insgesamt rund 200 milliarden euro umfassen - in nachhaltige stadtentwicklung zu investieren. das setzt freilich voraus, dass die bürgermeisterinnen förderwürdige projekte in der schublade haben. wer von eu-mittel in der neuen finanzperiode profitieren will, sollte sich bereits jetzt um das programmdesign kümmern. frühzeitige planung macht sich bei eu-förderungen bezahlt", so wojahn. claudia schmidt, abgeordnete zum europäischen parlament und mitglied des ausschusses für regionale entwicklung hält dazu fest: "städte und stadtregionen müssen das umsetzen, was in brüssel beschlossen wird. stichwort elektrifizierung, stichwort digitalisierung und stichwort integration. von der kompetenz der handelnden bürgermeister und beamtenschaft hängt der erfolg der eu ziele schlussendlich ab. mehr rückkoppelung zwischen theorie in brüssel und praxis in den städten ist meiner meinung der schlüssel für den künftigen erfolg der eu. wichtiger als der finale verteilungsschlüssel der fördertöpfe." forderung der städte: 20 prozent für nachhaltige stadtentwicklung diese sichtweise wird nur teilweise von thomas weninger, generalsekretär des städtebundes, geteilt: "fördermittel sind immer ein essentieller anreiz dafür, dass über die stadtgrenze an problemlösungen gearbeitet wird. die realität zeigt, dass sich städtische probleme sei es im bereich mobilität, energieraumplanung oder betriebsansiedlungen nur im stadtregionalen schulterschluss lösen lassen. eine zentrale forderung im positionspapier der städte und stadtregionen lautet daher, stadt-(um)land-partnerschaften und projekte neben integrierten stadtentwicklungsstrategien als bestandteil der städtischen dimension im efre stärker zu verankern." so weninger weiter. dazu soll die mittelbindung für integrierte nachhaltige stadtentwicklung im efre aus sicht des städtebundes auf 20 prozent erhöht werden. "hier gilt es noch verstärkt lobbying zu betreiben, sehen die derzeitigen entwürfe doch lediglich eine mittelbindung von mindestens 6 prozent für nachhaltige stadtentwicklung vor - viel zu wenig angesichts der enormen herausforderungen der städte bei umsetzung der eu-ziele von klimawandel, über migration bis hin zum sozialen zusammenhalt". doch das eigentliche match um die fördermittel wird ohnehin innerhalb österreichs ausgetragen werden, meint weninger: "inwieweit stadtregionale themen beim einsatz von eu-mitteln adäquat berücksichtigung finden, wird auf nationaler ebene vereinbart, brüssel ist hierbei ein partner. bisher ist die mangelnde berücksichtigung stadtregionaler herausforderungen hausgemacht gewesen. daher gilt es für den städtebund auf bundesebene für die mittel aus dem europäischen fonds für die ländliche entwicklung (eler) und auf landesebene für efre-mittel stärker auf die rolle und möglichkeiten von stadtregionen und städten zur umsetzung der eu-ziele hinzuweisen, auch im sinne eines produktiven miteinanders von stadt und land!" so weninger. josef plank, generalsekretär im bundesministerium für nachhaltigkeit und tourismus, betont seinerseits die notwendigkeit, städte und ihr umland als funktionale räume handlungsfähiger zu machen: "ob es um den erhalt von zusammenhängenden landschafts- und erholungsräumen geht, um einen attraktiven öffentlichen verkehr, kompakte und klimafreundliche siedlungsräume oder die tragfähigkeit sozialer infrastruktur: all das erfordert eine partnerschaftliche zusammenarbeit der zentralen orte und ihrer umlandgemeinden im weiteren sinne." in seiner neuen verantwortung für die eu-regional-und stadtentwicklungspolitik in österreich kann das bundesministerium für nachhaltigkeit und tourismus, die synergien, die sich zwischen den instrumenten der ländlichen entwicklung und der regionalpolitik der eu in österreich ergeben, optimal nutzen. der zweite tag des österreichischen stadtregionstages in wels am 11.10.18. wird anhand von praxisbeispielen aus deutschland und österreich die bedeutende rolle der eu-fördermittel für das zustandebringen stadtregionaler kooperationen in den vordergrund stellen. auch werden gute beispiele verdeutlichen, wie innerhalb der stadtregion gemeinsam an der entwicklung von standortfaktoren gearbeitet werden kann. zum stadtregionstag der stadtregionstag ist eine initiative des österreichischen städtebundes. er dient als informationsdrehscheibe, zum know-how-transfer und als lern- und kooperationsplattform. durch die abhaltung von nunmehr sechs stadtregionstagen wurde erhöhtes bewusstsein für stadtregionales denken, planen und entscheiden geschaffen und ein österreichweiter erfahrungsaustausch angeregt. weitere details zum programm unter: www.staedtebund.gv.at, www.stadtregionen.at; www.staedtebund.gv.at/fileadmin/userdata/themenfelder/europa/position spapier\_policy-paper\_final\_deutsch\_kurz.pdf twittern sie mit unter: \#stadtregionstag fotos zur veranstaltung finden sie zum download unter: www.picdrop.de/markuswache/stadtregionstag+2018 & 983 & low & Low & Socio-Economic & Power & NA & 2018-10-10 & 2018 & 3 & ECO
Frame & low-medium & National & 500-1000 & -0.7948903 & -0.1706634 & 1.0035124 & 0.5015415 & -0.3119516 & 9.0 & 0.2648514 & -0.9910023 & Payer & Domestic & European & Mixed & Domestic|ECO & Positive\\
Austria & http://www.salzburg.com/nachrichten/welt/politik/sn/artikel/borissow-erhielt-regierungsauftrag-in-bulgarien-245084/ & 92 & Salzburger Nachrichten & Private/Non-Public & Online and Offline & Regional/Local & very low = CP mentioned once & Solidarity to poor countries/regions & Factual & Other country & No myth & NA & NA & NA & NA & NA & NA & NA & NA & Austria & borissow erhielt regierungsauftrag in bulgarien & 2017-04-27 & kohäsionspolitik & borrisow hat nun eine woche zeit. bild: sn/apa (afp)/dimitar dilkoff dazu schloss seine partei eine koalitionsvereinbarung mit der nationalistischen formation vereinigte patrioten ab. "ich wünsche ihnen erfolg und hoffe, sie werden die erwartungen der bulgarischen bürger erfüllen", sagte präsident radew bei der erteilung des regierungsauftrags an den designierten ministerpräsidenten borissow. die stellt nach den neuwahlen vom 26. märz mit 95 der 240 parlamentsabgeordneten und ist auf einen koalitionspartner angewiesen. hinter den vereinigten patrioten stehen drei parteien. die formation ist die drittstärkste kraft im parlament mit 27 abgeordneten. die bürgerlich-nationalistische regierungsmehrheit ist mit 122 knapp. beide seiten unterzeichneten am donnerstag eine entsprechende koalitionsvereinbarung. die postenverteilung in der künftigen regierung muss aber noch in den nächsten sieben tagen erfolgen. es wird erwartet, dass das neue kabinett mit 17 ministern am 4. mai steht. ein koalitionsrat aus sechs mitgliedern soll die wichtigsten regierungsentscheidungen vorbereiten, sagte er weiter. einer von vier vizeregierungschefs wird voraussichtlich eigens mit den vorbereitungen für die erste, halbjährliche eu-ratspräsidentschaft bulgariens ab jänner 2018 beauftragt werden, sagte der co-vorsitzende der vereinigten patrioten, krassimir karakatschanow, gegenüber medien. im mittelpunkt der ratspräsidentschaft soll die fortsetzung der eu-kohäsionspolitik zum ausgleich zwischen ärmeren und reicheren mitgliedstaaten nach 2020 stehen. die versuchte einmischung der benachbarten türkei in den wahlkampf hat insofern niederschlag im regierungsprogramm der neuen koalition in sofia gefunden, indem die souveränität bulgariens explizit erwähnt wird. darauf haben die vereinigten patrioten bestanden. die weitere enge anbindung an eu und nato ist eine der außenpolitischen prioritäten der neuen regierung. die nationalisten setzten sich bei den rund zweiwöchigen koalitionsverhandlungen mit borissows gerb offensichtlich auch in der flüchtlingspolitik durch: so fordert die künftige regierungskoalition in sofia eine "radikale reform" des dublin-abkommens, um eine fairere lastenverteilung bei asylanträgen in der eu zu erreichen und den ländern entlang der eu-außengrenze entgegenzukommen. außenpolitisches ziel der neuen bürgerlich-nationalistische koalition ist auch der beitritt bulgariens zur schengen-zone. darüber hinaus formulierten beide partner die einführung des euro als ihr wirtschaftspolitisches ziel. "der beitritt zur eurozone ist unser natürlicher weg der weiteren eu-integration", heißt es in der 21 seiten starken koalitionsvereinbarung. was die nato betrifft, sollen die zwei prozent des bruttoinlandsprodukts (bip) für verteidigungsausgaben, auf welche die us-regierung pochen, erreicht werden. die steuerpolitik soll beibehalten werden. in bulgarien gilt eine flat tax in höhe von zehn prozent. das habe sich positiv für auslandsinvestitionen erwiesen. die korruptionsbekämpfung soll verstärkt, die mindestpension stufenweise angehoben werden, um der ärmsten bevölkerungsschicht in bulgarien unter die arme zu greifen. die neue populistische partei wolja (wille) will mit ihren zwölf abgeordneten die bürgerlich-nationalistische koalition zunächst unterstützen. die bulgarische sozialistische partei (bsp) hatte als zweitgrößte partei im parlament eine große koalition mit gerb entschieden abgelehnt. im neuen parlament in sofia ist auch die etablierte, liberale türkenpartei dps (bewegung für rechte und freiheiten) mit 26 abgeordneten vertreten. zu den vorgezogenen parlamentswahlen kam es, nachdem borissows kandidatin die präsidentenwahl gegen den von den sozialisten unterstützten radew klar verloren hatte. borissow reichte daraufhin überraschend den rücktritt seiner zweiten regierung (2014-16) ein. bei der parlamentswahl setzte sich die gerb-partei deutlicher durch, als die wahlforscher erwartet hatten. nun bildet borissow sein drittes kabinett, nachdem die gerb-partei 2009 zum ersten mal die parlamentswahlen gewonnen hatte. borissows erste regierung scheiterte jedoch wenige monate vor dem ende ihrer amtszeit an sozialen protesten. & 548 & very low & Low & Values & NA & NA & 2017-04-27 & 2017 & 2 & ECO
Frame & v.low & Regional & 500-1000 & -0.7948903 & -0.1706634 & 1.0035124 & 0.5015415 & -0.3119516 & 9.0 & 0.2648514 & -0.9910023 & Payer & European & European & European & European|ECO & Neutral\\
Austria & http://derstandard.at/2000047663547/Rechnungshof-Oesterreich-ist-fuenftgroesster-EU-Nettozahler & 19 & der Standard & Private/Non-Public & Online and Offline & National & low = CP mentioned more times but NOT important part of story (mainly about others issues) & Institutional bargaining over funding & Negative & National + Other country & No myth & Mismanagement & Negative & National + Other country & 10.Slow spend & NA & NA & NA & NA & Austria & rechnungshof: österreich ist fünftgrößter eu-nettozahler & 2016-11-16 & strukturfonds & födertöpfe wurden wegen mängeln in der verwaltung nicht voll ausgeschöpft wien - österreich hat im jahr 2014 rund 1,3 milliarden euro mehr an die eu überwiesen, als es von dort zurückbekam. das zeigt ein am mittwoch veröffentlichter rechnungshofbericht über die eu-finanzen. gemessen am bruttonationaleinkommen sind die niederlande der größte nettozahler (0,71 prozent der wirtschaftsleistung), gefolgt von deutschland, schweden und finnland. österreich liegt in diesem vergleich auf platz fünf (0,38 prozent). der größte profiteur der europäischen umverteilungsmechanismen ist demnach ungarn vor bulgarien und litauen. deutsche zahlen 15,5 milliarden mehr in absoluten zahlen ist natürlich deutschland der größte nettozahler. die zahlungen an die eu waren dort 15,5 milliarden euro höher als die rückflüsse. in diesem vergleich ist polen der größte nettoempfänger (plus 13,7 milliarden euro). wie der rechnungshof vorrechnet, hätte österreichs nettobeitrag durchaus etwas niedriger ausfallen können. die ausnutzung der europäischen agrar- und sozialfonds lag nämlich nur bei 95 prozent. bei den strukturfonds lag der ausnutzungsgrad sogar nur bei 63 prozent. der grund dafür: die eu-kommission stellte seit 2012 immer wieder fest, dass es mängel im heimischen verwaltungs- und kontrollsystem gab, und setzte deshalb zahlungen an österreich aus. die folge davon: in drei bundesländern - steiermark, tirol und vorarlberg - lag die quote sogar unter 50 prozent. (go, 16.11.2016) & 214 & low & Low & Power & Governance & NA & 2016-11-16 & 2016 & 2 & POL
Frame & low-medium & National & <500 & -0.7948903 & -0.1706634 & 1.0035124 & 0.5015415 & -0.3119516 & 9.0 & 0.2648514 & -0.9910023 & Payer & Domestic & European & Mixed & Domestic|POL & Negative\\
\addlinespace
Austria & http://www.nachrichten.at/nachrichten/wirtschaft/OEsterreich-mit-1-3-Milliarden-Euro-EU-Nettozahler;art15,2404895 & 34 & Oberosterreichische Nachrichten & Private/Non-Public & Online and Offline & Regional/Local & low = CP mentioned more times but NOT important part of story (mainly about others issues) & Institutional bargaining over funding & Negative & National + Subnational & No myth & Mismanagement & Negative & EU + National & No myth & NA & NA & NA & NA & Austria & österreich mit 1,3 milliarden euro eu-nettozahler & 2016-11-16 & strukturfonds & brüssel / wien. 2014 war österreich unter den zehn nettozahlern, 18 länder der eu waren nettoempfänger, allen voran polen. österreich hat im jahr 2014 um 1,297 milliarden euro mehr an die eu bezahlt als es von dort bekommen hat. damit liegt es unter den zehn nettozahlern in der eu auf rang 8. das geht aus der jüngsten statistik hervor, die der österreichische rechnungshof zu den eu-finanzen gestern, mittwoch, veröffentlicht hat. 18 länder waren nettoempfänger. am meisten kassierte ausgerechnet das eu-kritische polen. mit abstand größter nettozahler ist deutschland, das auch in bezug auf die wirtschaftsleistung ex aequo mit den niederlanden und schweden am meisten zahlt, gefolgt von finnland, österreich und belgien. österreich hat in den vergangenen jahren seinen bruttobeitrag bis 2013 stetig erhöht. damals wurden 3,19 milliarden euro nach brüssel überwiesen, 2014 waren es nur noch 2,87 milliarden euro. davon entfielen 672,8 millionen euro auf die länder und 120,8 millionen euro auf die gemeinden. diese zahlen ergaben sich aus dem schlüssel des finanzausgleichs. österreich hat aber 2014 nicht nur weniger nach brüssel überwiesen, sondern auch weniger von dort zurückbekommen, weil seit 2014 eine neue finanzperiode begonnen hat und nun neue vorgaben in kraft sind. 1,573 milliarden euro, das war ein rückgang um gut 15 prozent. über die agrarfonds erhielt österreich 1,2 milliarden euro, über strukturfonds 137,9 millionen euro. von den bundesländern entfielen die höchsten zahlungen auf niederösterreich (34,9 prozent), oberösterreich (19,6 prozent) und die steiermark (14,3 prozent). der europäische rechnungshof wiederum stellte fest, dass die fehlerquote bei der verwendung von eu-mitteln nach wie vor hoch sei. 3,8 prozent der eu-mittel wurden 2015 fehlerhaft aufgeteilt, das liege über der toleranzgrenze von zwei prozent. in seinem bericht führt der eu-rechnungshof aus, die fehlerhaften ausgaben seien nicht auf betrug, ineffizienz oder verschwendung zurückzuführen. vielmehr handle es sich dabei um eine schätzung der mittel, die nicht hätten ausgezahlt werden dürfen, weil sie nicht vollständig im einklang mit den eu-vorschriften verwendet wurden. die prüfungen des rechnungshofs haben aber offenbar positive wirkung. im jahr 2006 hätte die fehlerquote noch mehr als sieben prozent betragen, heißt es in dem bericht. & 359 & low & Low & Power & Governance & NA & 2016-11-16 & 2016 & 2 & POL
Frame & low-medium & Regional & <500 & -0.7948903 & -0.1706634 & 1.0035124 & 0.5015415 & -0.3119516 & 9.0 & 0.2648514 & -0.9910023 & Payer & Domestic & Domestic & Domestic & Domestic|POL & Negative\\
Austria & http://diepresse.com/home/ausland/eu/5285738/Durchbruch-fuer-Behindertenrechte & 12 & Die Presse & Private/Non-Public & Online and Offline & National & very low = CP mentioned once & Social awareness/inclusion & Balanced & EU & No myth & NA & NA & NA & NA & NA & NA & NA & NA & Austria & durchbruch für behindertenrechte & 2017-09-28 & kohäsionsfonds & fast jeder vierte europäer ist gebrechlich oder behindert. das europaparlament will diesen menschen die benützung von bankomaten, fahrkartenautomaten und anderen maschinen erleichtern. im schatten des rummels um die jährliche rede von kommissionspräsident jean-claude juncker zur lage der union hat das europaparlament am donnerstag fast unbemerkt einen beschluss von bedeutung für das tägliche leben von millionen behinderter und älterer menschen gefällt. mit 537 zu zwölf stimmen (bei 89 enthaltungen) beschlossen die parlamentarier ihren entwurf für den european accessibility act, also ein eu-gesetz zur barrierefreiheit. barrierefreiheit bedeutet, dass produkte und dienstleistungen von menschen mit behinderungen mehr oder weniger genau so verwendet werden können wie von nichtbehinderten. in einer gesellschaft, deren durchschnittsalter steigt, ist das eine wesentliche frage. im jahr 2020, also demnächst, soll es nach berechnungen der europäischen kommission rund 120 millionen europäer geben, die körperlich zumindest geringfügig beeinträchtigt sind. sorge für allzu hohen umbaukosten vor mehr als zwei jahren legte die kommission daher den entwurf der gegenständliche richtlinie vor, doch die arbeit an der darauf fußenden version des parlaments stockte lange. der dänische liberale morten løkkegaard, der im binnenmarktausschuss federführend für dessen verfassung zuständig war, musste sich von behindertenverbänden die kritik gefallen lassen, den bedenken der industrie- und gemeindelobbies, die vor untragbaren kosten warnen, zu stark nachzugeben. denn der gesetzesentwurf sieht vor, dass so gut wie jedes gerät, über das man etwas bezahlt, bucht oder sich informiert, barrierefrei benutzbar ist: vom bankomaten über das fernsehgerät bis zum fahrkartenautomaten am bahnhof oder der maschine zum eigenständigen check-in am flughafen. binnen zwei jahren nach inkrafttreten der richtlinie müssen die mitgliedstaaten sie umsetzen; binnen sechs jahren müssen die maßnahmen zur erreichung der barrierefreiheit ergriffen worden sein. hürden für handel mit geräten beseitigen die kommission war dabei in erster linie von der sorge getrieben, dass die bestehenden unterschiedlichen vorschriften für barrierefreiheit in den einzelnen mitgliedstaaten den handel und damit die verbreitung solcher produkte und dienste behinderten. "anbieter, die grenzüberschreitend tätig sind, müssen zusätzliche produktionskosten auf sich nehmen, um die unterschiedlichen barrierefreiheitsanforderungen zu erfüllen. der wettbewerb, die wettbewerbsfähigkeit und das wirtschaftliche wachstum werden behindert", hielt die kommission in ihren erläutenden bemerkungen fest. der text, auf den sich die parlamentarier nach einigem tauziehen bis knapp vor der abstimmung einigten, sieht vor, dass die barrierefreiheit bei der neuerrichtung oder kernsanierung von gebäuden zu berücksichtigen ist. "sämtliche bestehende gebäude umzubauen wäre unrealistisch gewesen", sagte othmar karas, chef der övp-delegation im parlament, zur "presse". keine neuen pflichten für kleinstbetriebe zudem sind kleinstunternehmen, also solche mit weniger als zehn mitarbeitern, von der anwendung der richtlinie ausgenommen. kleine und mittelgroße betriebe können ausnahmen wegen unzumutbarkeit erwirken. dort, wo ein mitgliedstaat bereits eine vorschrift zur durchsetzung der barrierefreiheit erwirkt hat, zwingt ihm die richtlinie kein weiteres tun auf. bei der vergabe öffentlicher aufträge und der entscheidung über die ausschüttung von mitteln aus den struktur- und kohäsionsfonds der eu sollen diese vorschriften verpflichtend werden. "es gibt zum beispiel zwischen schweden und bulgarien enorme unterschiede", sagte karas. "darum ist es wichtig, die mindeststandards in europa zu erhöhen. dabei geht es darum, die balance zwischem dem sinnvollen und dem theoretisch machbaren zu finden." industrielobby unzufrieden auch der spö-mandatar josef weidenholzer, der dem zuständigen binnenmarktausschuss angehört, begrüßt den erzielten kompromiss. seine fraktion hatte gegen den widerstand der konservativen und liberalen "durchgesetzt, dass produkte und dienstleistungen, die behindertengerecht sind, auch klar als solche gekennzeichnet werden", erklärte er. wann dieses gesetz in kraft tritt und somit den handlungsspielraum für behinderte und ältere im alltagsleben erhöht, ist offen. gemeinsam mit dem parlament ist der rat, also das gremium der nationalen regierungen, für den beschluss zuständig. doch sei man noch weit davon entfernt, eine eigene position für die verhandlungen mit dem parlament zu formulieren, hieß es am donnerstagnachmittag auf anfrage der "presse" aus dem sekretariat des rates. "die ausstehenden verhandlungen mit dem rat werden aber noch schwieriger, da die mitgliedsstaaten versuchen werden, die richtlinie wieder weiter zu verwässern", warnte weidenholzer. die reaktion von business europe, dem wichtigsten lobbyverband der europäischen wirtschaftstreibenden, scheint ihn zu bestätigen. "wir sind vor allem über die änderungen betreffend die öffentliche auftragsvergabe und die anforderungen an die barrierefreiheit in den annexen besorgt", heißt es dort. "wir fordern den rat auf, in diesem bereichen weitere verbesserungen zu machen." & 695 & very low & Low & Socio-Economic & NA & NA & 2017-09-28 & 2017 & 2 & ECO
Frame & v.low & National & 500-1000 & -0.7948903 & -0.1706634 & 1.0035124 & 0.5015415 & -0.3119516 & 9.0 & 0.2648514 & -0.9910023 & Payer & European & European & European & European|ECO & Neutral\\
Austria & https://www.ots.at/presseaussendung/OTS\_20180614\_OTS0111/fassmann-mahrer-egerth-burtscher-auf-dem-weg-zu-horizon-europe-100-milliarden-euro-fuer-forschung-und-innovation-bild & 27 & APA-OTS & Private/Non-Public & Online only & National & very low = CP mentioned once & Improve governance & Balanced & EU & No myth & NA & NA & NA & NA & NA & NA & NA & NA & Austria & faßmann, mahrer, egerth \& burtscher: auf dem weg zu "horizon europe" - 100 milliarden euro für forschung und innovation & 2018-06-14 & strukturfonds & wien (ots) - auf dem weg zu "horizon europe": die vorbereitungen für das 9. eu-rahmenprogramm für forschung und innovation (2021 - 2027) laufen auf hochtouren und standen heute im mittelpunkt eines gemeinsamen pressegesprächs von wissenschafts- und forschungsminister univ.-prof. dr. heinz faßmann, wkö-präsident dr. harald mahrer, ffg-geschäftsführerin dr. henrietta egerth sowie dem stv. generaldirektor dr. wolfgang burtscher (europäische kommission). nachdem die europäische kommission vergangene woche ihren vorschlag für "horizon europe" vorgelegt hat, wird es insbesondere auch an österreich liegen, diesen während der österreichischen eu-ratspräsidentschaft im zweiten halbjahr 2018 gemeinsam mit den anderen eu-mitgliedsstaaten und weiteren akteuren mit leben zu erfüllen. dabei sind im nächsten eu-mehrjahresbudget (2021 - 2027) rund 100 milliarden euro für forschung und innovation vorgesehen. in der anschließenden veranstaltung in der wirtschaftskammer österreich wurde auch der "überblicksbericht zu österreich in horizon 2020" präsentiert, der die erfolgreiche bisherige performance der forscherinnen und unternehmen im aktuellen eu-forschungsrahmenprogramm dokumentiert. "horizon 2020" hat sich zu einer erfolgsgeschichte mit europäischem mehrwert und nachweisbarem nutzen entwickelt. darauf aufbauend wird "horizon europe" weiterhin den gesamten forschungs- und innovationskreislauf unterstützen und gezielt dazu beitragen, die wissenschaftliche, wirtschaftliche und gesellschaftliche wirkung der europäischen forschungsförderung zu erhöhen. das programm beruht wie auch bereits "horizon 2020" auf drei säulen: säule i ("open science") konzentriert sich auch künftig auf exzellente wissenschaft, maßgeblich durch den europäischen forschungsrat (erc). säule ii legt den fokus auf globale herausforderungen und industrielle wettbewerbsfähigkeit und sieht u.a. "missionen" vor, um ziele mit hoher gesellschaftlicher relevanz und eine verstärkte sichtbarkeit zu erreichen. die säule iii ("open innovation") ist neu und soll insbesondere mit dem europäischen innovationsrat (eic) dazu beitragen, dass die eu bei bahnbrechenden marktschaffenden innovationen führend wird. faßmann: verhandlungen zu "horizon europe" so weit wie möglich voranbringen "es ist erfreulich, dass die europäische kommission mit dem neuen finanzrahmen einen deutlichen schwerpunkt in forschung und innovation setzt. zudem sollen die synergien der einzelnen programme stärker genutzt werden. das gilt insbesondere für die komplementäre verwendung von mitteln aus rahmenprogramm und strukturfonds, aber auch für die synergien mit erasmus oder der gemeinsamen agrarpolitik. dies erweitert zusätzlich die bedeutung von forschung und innovation", betonte wissenschaftsminister dr. heinz faßmann. "mit 'horizon europe' wird das bislang ehrgeizigste förderprogramm für forschung und innovation vorgeschlagen. neu ist, dass jene bereiche fokussiert werden, die unser tägliches leben betreffen. beispiele dafür könnten von der bekämpfung von krebs über den sauberen verkehr bis zu plastikfreien meeren reichen. wir möchten die verhandlungen zu 'horizon europe' während unseres eu-vorsitzes so weit wie möglich voranbringen", so der wissenschaftsminister. burtscher: "horizon europe" - eine gelungene mischung aus bewährtem und neuem "der von der europäischen kommission - trotz enger budgetvorgaben - vorgelegte 100 milliarden vorschlag macht deutlich, welchen zentralen stellenwert europa der forschung und innovation für die zukunft unseres kontinents beimisst", so dr. wolfgang burtscher, stv. generaldirektor forschung und innovation der europäischen kommission. "es ist sehr zu hoffen, dass in den nun bevorstehenden verhandlungen mit dem rat und dem europäischen parlament ein großzügiger haushalt für forschung und innovation beschlossen wird." die im rahmen dieses budgets vorgeschlagenen förderinstrumente und -maßnahmen führen zum einen bewährte instrumente wie erc und marie skłodowska-curie maßnahmen fort, beinhalten aber auch neuerungen wie die missionsorientierte forschung und den europäischen innovationsrat, der für eine impactorientierte forschung von zentraler bedeutung ist. "damit wollen wir dem anspruch einer nahtlosen förderung von forschung und innovation und der überbrückung des 'valley of death' endlich gerecht werden." mahrer: "innovation rules" - wirtschaftspolitische agenda konsequent auf innovationen ausrichten "'innovation rules' - diese erkenntnis müssen wir konsequent in der wirtschaftspolitischen agenda verankern. daher ist 'horizon europe', das größte forschungs- und innovationsförderprogramm weltweit, der wichtigste hebel mit dem die europäische union wohlstand und wachstum unterstützen kann. für diese hebelwirkung ist die beteiligung der wirtschaft entscheidend. davon profitieren wir alle - kleinbetriebe, großunternehmen und die gesellschaft als ganzes. die gleichung lautet: starke budgets für innovation und forschung + unternehmerische leistung = mehr zukunfts- und wettbewerbsfähigkeit für österreich, um unser ziel, in die gruppe der innovation leader zu kommen, zu erreichen", betonte wkö-präsident dr. harald mahrer. österreichs unternehmen können für das noch laufende programm "horizon 2020" eine sehr positive zwischenbilanz ziehen. bis märz 2018 konnten mehr als 460 österreichische unternehmen eu-mittel in höhe von gesamt 325 mio. euro für ihre innovationsvorhaben lukrieren. insgesamt wurden so bisher 871 mio. euro nach österreich geholt. durch die neuauflage des eu-forschungs- und innovationsförderprogramms soll europa stärker werden und im konzert der wirtschaftsregionen mit innovationen führen. für österreich ist ein erfolgreicher verhandlungsprozess im rahmen der eu-ratspräsidentschaft eine chance, 'horizon europe' praxisnah zu gestalten und das maximum herauszuholen, damit europa und österreich ab 2021 noch besser profitieren können", so mahrer. "wir müssen dafür sorgen, dass markt und wissenschaft noch enger zueinander rücken. dafür nötig ist, dass wissenschaft, wirtschaft und förderunternehmen noch besser kooperieren. hier wird die wirtschaft entsprechende initiativen starten. meine erwartungshaltung an 'horizon europe' ist daher hoch: am ende sollen zwischen 2021 und 2027 mehr als 2 mrd. euro nach österreich fließen, davon ca. 800 mio. an unternehmen." ffg-gf egerth: österreich ist messbar erfolgreich "aus ffg-sicht ist zentral, dass 'horizon europe' den bogen von der grundlagenforschung bis zur disruptiven innovation fördert und schlüsseltechnologien sowie die digitalisierung adressiert", so dr. henrietta egerth, geschäftsführerin der forschungsförderungsgesellschaft ffg. im europäischen innovationsrat (eic) sieht sie "die chance, dass die viel beklagte lücke, die sich in europa zwischen labor und markt auftut, geschlossen wird". die ffg hat als zentrale beratungsstelle auch das umfangreiche regelwerk im blick. der vorschlag zu "horizon europe" sieht dabei weitere vereinfachungen in der antragstellung und abwicklung vor. "damit sollen die rechtssicherheit erhöht und der verwaltungsaufwand verringert werden, das ist im interesse aller", so egerth. weiters betonte sie, dass "horizon europe" hand in hand gehen müsse mit dem eu-programm "digital europe", das investitionen von rund 9,2 milliarden euro (2021 - 2027) zur bewältigung der digitalen herausforderungen in europa vorsieht. "als nationale kontaktstelle für das eu-forschungsrahmenprogramm sowie dach für die digitalisierungsagentur dia sind wir auf nationaler wie europäischer ebene ein starker forschungs-, innovations- und digitalisierungs-partner." abschließend ging die ffg-geschäftsführerin auf den von der ffg im auftrag der ministerien erstellten "überblicksbericht zu österreich in horizon 2020" ein, der die bisher überdurchschnittlich erfolgreiche performance von forscherinnen und unternehmen belegt: "österreich ist messbar erfolgreich!" bereits in drei programmen (erc, ict, transport) kann österreich jeweils mehr als 100 millionen euro an rückflüssen verzeichnen. weitere informationen und dokumente zu "horizon europe" https://www.era.gv.at/directory/293 https://www.ffg.at/europa/fp9 beispiele erfolgreicher "horizon 2020"-projekte https://www.ffg.at/europa/erfolgsgeschichten "überblicksbericht zu österreich in horizon 2020" https://www.ffg.at/monitoring & 1082 & very low & Low & Governance & NA & NA & 2018-06-14 & 2018 & 3 & POL
Frame & v.low & National & +1000 & -0.7948903 & -0.1706634 & 1.0035124 & 0.5015415 & -0.3119516 & 9.0 & 0.2648514 & -0.9910023 & Payer & European & European & European & European|POL & Neutral\\
Austria & http://www.wienerzeitung.at/nachrichten/wirtschaft/international/809389\_Ein-F-an-rechter-Stelle.html & 61 & Wiener Zeitung & Private/Non-Public & Online and Offline & Regional/Local & medium = CP is important part of story & Bureaucracy and/or delays & Negative & EU + Other country & No myth & NA & NA & NA & NA & NA & NA & NA & NA & Austria & ein "f" an rechter stelle - wiener zeitung online & 2016-03-29 & kohäsionsfonds & der juncker-plan kann für die länder mittel- und osteuropas den kohäsionsfonds nicht ersetzen, warnt sandor richter vom wiiw. wien. die position des "f" macht einen milliardenschweren unterschied. jedenfalls für die eu-länder mittel- und osteuropas. denn ist das f an letzter stelle des akronyms, nämlich esif, steht das für europäischen struktur- und investitionsfonds, der vor allem für die neuen eu-mitgliedsländer ins leben gerufen worden ist und über transferzahlungen die sogenannte kohäsionspolitik vorantreiben soll. im prinzip, auch wenn das nicht ganz so deutlich ausgesprochen wird, soll dadurch der lebensstandard in den neuen an den der alten eu-mitgliedsländer angepasst werden. jedenfalls profitieren die neuen eu-mitgliedsländer überproportional von den geldern des kohäsionsfonds esif. steht das f an zweiter stelle, handelt es sich hingegen um den efsi, den europäischen fonds für strategische investitionen, besser bekannt als "juncker-plan". denn der kommissionspräsident jean-claude juncker hat diesen 315 milliarden euro schweren topf erdacht, um in dem zeitraum 2015 bis 2017 die investitionen in den 28 ländern der europäischen union anzukurbeln. die mission: öffentliche gelder gezielter zu verwenden und private investoren anzuziehen. während sich laut dem wiener institut für internationale wirtschaftsvergleiche (wiiw) das wachstum in den ländern mittel- und osteuropas, abgesehen von den krisenstaaten ukraine und russland, im vergangenen jahr wieder etwas erholt hat, droht dieser konjunkturaufschwung in den kommenden jahren erneut abzunehmen. der grund ist ganz einfach der zyklus, in dem die gelder des esif, des kohäsionsfonds, schlagend werden. der esif operiert jeweils in einem sieben-jahres-zeitraum. die spitze der investitionstätigkeit wurde bisher erst am ende des siebenjährigen finanzrahmens erreicht. einerseits aufgrund der fristen zur projekteinreichung und sicher andererseits auch wegen des am ende sich herauskristallisierenden spielraums bei der ausschöpfung der mittel - also wenn noch zweckgewidmetes geld übrig ist. der finanzierungsrahmen für cee steht wieder am anfang konkret haben die mittel des kohäsionsfonds in der ersten periode 2007-2008 erst 1,35 prozent des gesamten bruttoinlandsprodukts der damals zehn ländern der eu-cee ausgemacht, also bulgariens, estlands, lettlands, litauens, polens, rumäniens, der slowakei, sloweniens, tschechiens, ungarns - kroatien wurde erst 2013 aufgenommen und ist in diesen berechnungen noch nicht dabei. im folgenden zeitraum machte das eu-engagement bereits 2,51 prozent des aggregierten bip der region aus, am ende, 2012 bis 2014 waren es schon 3,20 prozent. "wenn man sich die daten der früheren jahre ansieht", schreibt der wiiw-experte sandor richter im aktuellen frühjahrsreport des instituts, dann kann man davon ausgehen, dass sich das finanzielle eu-engagement in den betroffenen mitgliedsländern mehr als halbieren wird und heuer auf "eins bis 2,5 prozent des bip herabsinken" kann. ein starker abfall in den transferleistungen könnte, so richter, negative auswirkungen auf den binnenkonsum haben und sich auch auf die wachstumsprognosen auswirken. "damit erhebt sich die frage, ob der juncker-plan die momentane abnahme in den kohäsions-transfers kompensieren kann", schreibt sandor richter, und gibt sich wenig später die antwort selbst: nein. "am 21. jänner wurden 42 größere projekte (die für den juncker-plan in frage kommen, anm.) auf der homepage der europäischen investitionsbank vorgestellt. die allokation von großen projekten war vor allem enttäuschend für die eu-cee-länder: bei den 42 projekten waren nur kroatien, polen und die slowakei mit jeweils einem genehmigten projekt vertreten, die anderen eu-länder aus mittel-und osteuropa kamen gar nicht vor." juncker-plan erreicht vor allem bereits "entwickelte" mitglieder das ist auch heute noch so. allerdings ist bei einem update der eib vom februar erklärt worden, dass mit den bereits gewidmeten 9 milliarden euro aus dem juncker-plan immerhin 22 von 28 eu-mitgliedsländern erreicht werden - sei es über großprojekte oder über klein- und mittelbetriebe. von den sechs ländern, die noch nicht über die juncker-initative erreicht worden sind, sind fünf aus mittel- und osteuropa. & 620 & medium & Medium & Governance & NA & NA & 2016-03-29 & 2016 & 2 & POL
Frame & low-medium & Regional & 500-1000 & -0.7948903 & -0.1706634 & 1.0035124 & 0.5015415 & -0.3119516 & 9.0 & 0.2648514 & -0.9910023 & Payer & European & European & European & European|POL & Negative\\
Austria & http://burgenland.orf.at/news/stories/2915785/ & 93 & burgenland.orf.at & Public & Online and Offline & National & high = CP is most important issue in story (can also cover other issues) & Institutional bargaining over funding & Positive & Subnational & No myth & NA & NA & NA & NA & NA & NA & NA & NA & Austria & eu-budget: burgenland könnte wieder profitieren & 2018-05-29 & kohäsionsfonds & das burgenland, das derzeit fördergeld als übergangsregion der eu bekommt, bemüht sich schon länger darum, auch weiter geld von der eu zu lukrieren und es sieht gut aus, wie die eu-kommission am dienstag signalisierte. die eu-kommission will im rahmen des nächsten mehrjahresbudgets 2021-2027 beim kohäsionsfonds "niemand zurücklassen". eu-regionalkommissarin corina cretu sagte am dienstag im eu-parlament in straßburg, es gehe um alle regionen. mit flexibleren neuen prioritäten würden alle bürger geschützt. generell bleibt es bei drei kategorien für die regionen. aber es werden neue sätze für die strukturförderungen eingeführt. für die weniger entwickelten regionen soll weiter eine grenze von 75 prozent des bruttonationalprodukts gelten. für die übergangsphase wird aber die obergrenze von 90 auf 100 prozent des durchschnittlichen eu-bruttoinlandsproduktes angehoben. damit würde auch das burgenland wieder geld beziehen können. außerdem sollen grenzüberschreitende regionalprogramme mit 9,5 milliarden euro gefördert werden. auch hier will das burgenland profitieren. & 151 & high & High & Power & NA & NA & 2018-05-29 & 2018 & 3 & POL
Frame & high-very high & National & <500 & -0.7948903 & -0.1706634 & 1.0035124 & 0.5015415 & -0.3119516 & 9.0 & 0.2648514 & -0.9910023 & Payer & Domestic & Domestic & Domestic & Domestic|POL & Positive\\
\addlinespace
Austria & https://www.ots.at/presseaussendung/OTS\_20190507\_OTS0162/52-mrd-euro-aus-eu-toepfen-fliesst-in-oesterreichische-projekte-bild & 54 & OTS.at & Private/Non-Public & Online only & National & very high = CP is most important issue + CP is mentioned in title/headline & Economic development & Positive & Subnational & No myth & NA & NA & NA & NA & NA & NA & NA & NA & Austria & 5,2 mrd. euro aus eu-töpfen fließt in österreichische projekte & 2019-05-07 & regionalpolitik & wien (ots) - mit der europawahl bekommt die europäische union ein neues politisches gesicht. für die bürgerinnen und bürger sichtbar und erfahrbar zeigt sich die union ihr gesicht aber in den vielen tausend projekten in den städten, dörfern und regionen europas, die dank der förderungen durch die europäischen strukturfonds möglich werden. etwa im projekt "start2work" der caritas vorarlberg, das bleibeberechtigte flüchtlinge beim einstieg in den arbeitsmarkt unterstützt. oder in der wiener werkstätte von gabarage upcycling design, in der chronisch suchtkranke menschen einen arbeitsplatz und professionelle betreuung bekommen, um sich wieder in einen arbeitsalltag einzugewöhnen, und dabei individuelle designprodukte aus entsorgten materialien herstellen. oder in der transformation des einstigen klassischen kurorts bad tatzmannsdorf zu burgenlands führender wellness- und gesundheitsdestination. bei einem pressegespräch in der gabarage-werkstätte in wien präsentierten die projektträger am dienstag diese projekte. burgenlands landesrat christian illedits und martina rüscher, die vizepräsidentin des vorarlberger landtags, gaben dabei auch den medialen startschuss zur kampagne "europa in meiner region", die am 9. und 16. mai jeweils im burgenland und in vorarlberg den bürgerinnen und bürgern gelegenheit gibt, viele eu-geförderte projekte zu besuchen und zu erleben, welche rolle sie in ihrer region spielen. "für uns ist es wichtig, dass die eu mit ihren förderungen in allen regionen präsent ist, auch wenn es rufe gibt, die reichen regionen davon auszuschließen. das gehört auch zum solidarischen prinzip", erklärte jörg wojahn, der repräsentant der europäischen kommission in österreich. man müsse aber künftig bei den förderungen insgesamt mit kürzungen rechnen, vor allem wenn der brexit kommt. das geld müsse daher stärker dort konzentriert werden, wo es reformbedarf gibt. für österreich schlägt die kommission eine noch stärkere ausrichtung auf die förderung von digitalen kompetenzen, energiewende, kreislaufwirtschaft und soziale inklusion vor. für das burgenland, das seit dem eu-beitritt unter allen bundesländern von der höchsten förderintensität profitiert, zog dessen landesrat christian illedits eine erfolgsbilanz über die teilhabe an der eu-regionalpolitik: "seit dem eu-beitritt sind 1,5 mrd. euro von der eu ins land geflossen und haben gesamtinvestitionen von 6 mrd. euro ausgelöst." wichtig sei aber was davon bei den menschen ankomme. so sei das durchschnittliche jahreseinkommen im burgenland in dem zeitraum von 22.000 auf 32.000 euro gestiegen. und das bip pro kopf ist um 22 prozentpunkte auf nunmehr 91 \% des eu-durchschnitts gestiegen. "entscheidend ist dabei aber, dass das von unten aufbauend passiert. jede der 171 burgenländischen gemeinden hat durch ihre initiativen von der eu mitgliedschaft profitiert," so illedits. auch für martina rüscher, vizepräsidentin des vorarlberger landtags, ist klar: "österreichs eu-beitritt hat sich für vorarlberg bis heute sehr positiv ausgewirkt." in zahlen: die beschäftigung im land ist von 128.000 auf 176.000 gestiegen. und: "der exportumsatz hat heuer die 10-mrd.-euro-grenze überschritten. zehn prozent davon gehen nach osteuropa. wir haben also auch von der eu-osterweiterung profitiert", erklärte rüscher, die aber betont: "über all dem steht dabei die politische idee: die eu als das erfolgreichste politische friedensprojekt der neueren geschichte. offene grenzen und der gleichberechtigte zugang zum eu-binnenmarkt sind für unsere jungen menschen in ausbildung und unsere exportorientierte wirtschaft ein zentraler faktor." 5,2 mrd. euro fließen in der laufenden periode 2014-2020 aus den europäischen struktur- und investitionsfonds (esi-fonds) in projekte in österreich. mit 3,9 mrd. euro kommt der großteil der mittel aus dem landwirtschaftsfonds eler und geht an maßnahmen zur entwicklung des ländlichen raums. 536 mio. euro aus dem regionalfonds efre gehen im rahmen des programms "investitionen in wachstum und beschäftigung" an projekte von forschung und entwicklung, co2-reduktion oder nachhaltigen stadtentwicklung. mit 442 mio. euro werden aus dem sozialfonds esf initiativen für bildung, lebenslanges lernen und soziale integration unterstützt. zudem fließen auch noch 257 mio. euro aus dem efre in projekte grenzüberschreitender zusammenarbeit mit nachbarstaaten. zu den eu-förderungen addiert sich jeweils eine nationale kofinanzierung, wodurch die induzierten gesamtinvestitionen noch wesentlich höher liegen. & 639 & very high & High & Socio-Economic & NA & NA & 2019-05-07 & 2019 & 3 & ECO
Frame & high-very high & National & 500-1000 & -0.7948903 & -0.1706634 & 1.0035124 & 0.5015415 & -0.3119516 & 9.0 & 0.2648514 & -0.9910023 & Payer & Domestic & Domestic & Domestic & Domestic|ECO & Positive\\
Austria & https://www.sn.at/panorama/medien/reporter-ohne-grenzen-besorgt-ueber-medien-entwicklung-in-europa-27649939 & 95 & sn.at & Private/Non-Public & Online and Offline & Regional/Local & low = CP mentioned more times but NOT important part of story (mainly about others issues) & Fraud/Corruption & Negative & Other country & 7.Fraud & NA & NA & NA & NA & NA & NA & NA & NA & Austria & reporter ohne grenzen besorgt über medien-entwicklung in europa & 2018-05-07 & kohäsionsfonds & die österreich-sektion von reporter ohne grenzen (rog) zeigt sich besorgt über die mediensituation in europa. die lage in ungarn und in der slowakei stand sonntagabend in wien im zentrum einer diskussion. in einer pessimistischen bestandsaufnahme hieß es, dass "die medienlandschaft in ungarn kaputt gemacht wurde" und dass in der slowakei "auch die heimische mafia gefährlich ist". im internationalen rog-medienranking sind zuletzt fünf europäische staaten abgestürzt; neben ungarn sind dies die slowakei, tschechien, malta und serbien. der frühere vizechefredakteur der von premier viktor orban geschlossenen ungarischen traditionszeitung "nepszabadsag", marton gergely, sprach klartext: "nicht nur unsere wirtschaftliche grundlage wird kaputt gemacht, sondern die ganze medienlandschaft." 2016 wurde "die kritische zeitung nummer eins eliminiert, seither veränderte sich alles zum schlechten". nach den worten gergelys hat die presse in ungarn ein problem mit der glaubwürdigkeit. die regierungsseite inklusive öffentlicher rundfunk und fernsehen verbreite in den linientreuen medien fake news. es gebe xenophobe und islam-feindliche aufrufe. die medien reagierten oft falsch darauf, gingen in die falle der regierung. "wir journalisten müssen nachdenken, was wir falsch gemacht haben." gergely bedauerte auch mangelnde kollegialität unter den journalisten, selbst bei den wenigen noch relativ unabhängigen medien. bei den jüngsten wahlen hätten die ängste der menschen orban in die hände gespielt. anders verhält es sich in der slowakei. die medienbeschimpfung durch den inzwischen abgetretenen premier robert fico und die ermordung des aufdeckungsjournalisten jan kuciak mit seiner verlobten habe die öffentlichkeit mobilisiert, so der radiojournalist tibor macak von rtv. die journalisten stünden nun im fokus des interesses. es gab eine welle der solidarität und "kein ende des investigativen journalismus". eine gruppe von internationalen journalisten formierte sich, die mit kuciaks online-portal aktuality kooperiere und dessen recherchen fortsetze. macak ist überzeugt: "die machenschaften der politiker und oligarchen kommen ans licht." in der slowakei sei "nicht nur die italienische mafia gefährlich, sondern auch die heimische". es gehe um die abzweigung von geldern aus dem brüsseler kohäsionsfonds und die damit verbundenen schweigegelder. in der regierung seien nur einige köpfe ausgetauscht worden, der forderung nach neuwahlen wurde nicht nachgekommen. macak erwartet demnach kein ende der proteste. der glaube an die justiz ist in ungarn und der slowakei gleichermaßen erschüttert. "die qualität der journalisten steht zur debatte", betonte macak. es gelte, die mörder kuciaks zu finden und vor allem die auftraggeber für den mord. er erinnerte daran, dass in der slowakei schon vorher zwei journalisten verschwanden, über deren schicksal man nichts wisse. gergely betonte: "wir arbeiten gut, denn wir können uns klagen nicht leisten." nach ausschaltung der opposition suche die regierung orban immer nach gegnern. als beispiel nannte er den us-millionär george soros, der zusammen mit einigen ngos zum feindbild gestempelt worden sei. den europäischen institutionen stellen gergely und macak ein relativ gutes zeugnis aus. in die slowakei sei rasch eine eu-parlamentarier-delegation entsandt worden, so macak. gergely erinnerte daran, dass orban sofort nach seinem ersten sieg 2010 den umbau der medien in angriff genommen habe. in sachen mediengesetz sei ungarns premier zweimal von der eu-kommission gestoppt worden. auf österreichischer seite kamen der frühere medienminister thomas drozda (spö) und "falter"-chefreporterin nina horaczek zu wort. drozda führte auf der negativliste an, dass regierungsmitglieder oft nicht zu interviews beim orf erscheinen, und verwies auf die turbulenzen im stiftungsrat. horaczek sprach die fpö-angriffe auf orf-moderator armin wolf an. dem "falter" habe schon werner faymann (spö) als bundeskanzler interviews verweigert, ergänzte sie. & 562 & low & Low & Governance & NA & NA & 2018-05-07 & 2018 & 3 & POL
Frame & low-medium & Regional & 500-1000 & -0.7948903 & -0.1706634 & 1.0035124 & 0.5015415 & -0.3119516 & 9.0 & 0.2648514 & -0.9910023 & Payer & European & European & European & European|POL & Negative\\
Austria & https://kurier.at/politik/ausland/premier-orban-verlangt-von-spd-chef-schulz-mehr-respekt-fuer-ungarn/305.511.022 & 35 & Kurier & Private/Non-Public & Online and Offline & National & very low = CP mentioned once & Political leverage & Negative & Other country & No myth & NA & NA & NA & NA & NA & NA & NA & NA & Austria & premier orban verlangt von spd-chef schulz mehr respekt für ungarn & 2018-01-08 & kohäsionsfonds & ungarn werde auch künftig keine flüchtlinge mehr aufnehmen, sagte orban. der rechtskonservative ungarische ministerpräsident viktor orban hat sich von spd-chef martin schulz "mehr respekt" für sein land erbeten. in anspielung auf schulz' früheres amt als präsident des europäischen parlaments sagte orban der bild-zeitung: "was gut und nett in brüssel war - wo es keine offensichtlichen konsequenzen gab - ist eine andere geschichte, als in deutschland parteichef zu sein und mit anderen ländern zu kommunizieren. wir finden, wir verdienen mehr respekt." der sozialdemokrat schulz hatte den csu-vorsitzenden horst seehofer aufgefordert, orban, der am freitag ehrengast der csu bei der winterklausur im oberbayerischen kloster seeon war, die grenzen aufzuzeigen. vor allem in der flüchtlingspolitik verfolge orban eine "gefährliche logik", hatte schulz kritisiert. "ich erwarte, dass herr seehofer ihm bei diesem thema und auch bei den themen presse- und meinungsfreiheit ganz klare grenzen aufzeigt." orban verwahrte sich in dem interview gegen den vorwurf, ungarn nehme geld von der eu, weigere sich aber, flüchtlinge aufzunehmen. der sogenannte kohäsionsfonds, welcher der ungarischen wirtschaft zugutekomme, sei kein geschenk. "er ist ein fairer ausgleich, da wir unseren markt dem freien wettbewerb geöffnet haben. das hat absolut nichts mit der flüchtlingsfrage zu tun." orban bekräftigte, dass ungarn auch künftig keine flüchtlinge aufnehmen werde. "wir glauben, dass eine hohe zahl an muslimen notwendigerweise zu parallelgesellschaften führt", sagte er. "so etwas möchten wir nicht. und wir möchten uns nichts aufzwängen lassen." & 232 & very low & Low & Power & NA & NA & 2018-01-08 & 2018 & 3 & POL
Frame & v.low & National & <500 & -0.7948903 & -0.1706634 & 1.0035124 & 0.5015415 & -0.3119516 & 9.0 & 0.2648514 & -0.9910023 & Payer & European & European & European & European|POL & Negative\\
Austria & https://diepresse.com/home/wirtschaft/international/3878816/Strassen\_Leichter-von-Ungarn-nach-Portugal-als-durch-Rumaenien?from=rss & 2 & Die Presse & Private/Non-Public & Online and Offline & National & very low = CP mentioned once & Ineffective goal achievement & Negative & Other country & 10.Slow spend & Infrastructure & Balanced & Other country & No myth & NA & NA & NA & NA & NA & straßen: leichter von ungarn nach portugal als durch rumänien & 2014-10-01 & NA & NA & 784 & very low & Low & Socio-Economic & Socio-Economic & NA & 2014-10-01 & 2014 & 1 & ECO
Frame & v.low & National & 500-1000 & -0.7948903 & -0.1706634 & 1.0035124 & 0.5015415 & -0.3119516 & 9.0 & 0.2648514 & -0.9910023 & Payer & European & European & European & European|ECO & Negative\\
Austria & https://www.ots.at/presseaussendung/OTS\_20190213\_OTS0157/vana-vergabe-von-eu-foerderungen-kuenftig-an-grundrechte-und-gleichstellung-koppeln & 72 & OTS.at & Private/Non-Public & Online only & National & high = CP is most important issue in story (can also cover other issues) & Political leverage & Balanced & EU + National & No myth & NA & NA & NA & NA & NA & NA & NA & NA & Austria & vana: "vergabe von eu-förderungen künftig an grundrechte und gleichstellung koppeln" & 2019-02-13 & kohäsionsfonds & straßburg (ots) - das europäische parlament hat heute über die zukunft der regionalpolitik 2021-2027 abgestimmt, das wichtigste regionalpolitische projekt dieser legislatur. die abgeordneten stimmten für das mandat für die verhandlungen mit dem rat und der europäischen kommission über die ausgestaltung von sieben strukturfonds, darunter der europäische sozialfonds und der regional- und kohäsionsfonds. erstmals soll die zahlungen von eu-fördergeldern an die einhaltung der grundrechte und gleichstellung gebunden sein. das europäische parlament fordert auch, dass regionalförderungen nicht mehr ausgesetzt werden können, wenn eu-mitgliedstaaten das haushaltsdefizit überschreiten. die finalen verhandlungen werden nach den europawahlen geführt, ein starkes verhandlungsmandat hat große bedeutung für die nächste legislaturperiode. monika vana, stellvertretende vorsitzende und regionalpolitische sprecherin der grünen/efa-fraktion im europäischen parlament, betont: "wir grüne/efa haben es geschafft, dass eu-projekte nur noch dann umgesetzt werden sollen, wenn die gleichstellung von frauen und männern sowie die einhaltung der grundrechte gewährleistet sind. das ist ein großer erfolg. projekte sollen in zukunft der nachhaltigen entwicklung von regionen dienen und der klimaschutz wird bedeutend aufgewertet. auf unsere initiative sollen alle finanzierten projekte einen umweltbericht vorlegen müssen. wir grüne/efa konnten unsere jahre lange forderung durchsetzen, die vergabe von fördergeldern nicht nur an die wirtschaftsleistung, sondern auch an kriterien wie jugendarbeitslosigkeit und bildungsniveau zu knüpften. das europäische parlament hat heute auch dafür gestimmt, das partnerschaftsprinzip zugunsten der regionen, städte und nichtregierungsorganisationen zu stärken. neben den regionalen und lokalen behörden sollen nichtregierungsorganisationen künftig als vollwertige partner anerkannt und in die planung von förderprogrammen eingebunden werden. es ist ein starkes signal des europäischen parlaments für kohäsionspolitik, dass weiterhin alle städte und regionen von den förderungen profitieren sollen. ein wichtiger erfolg ist, dass sich die mehrheit des europäischen parlaments dafür ausgesprochen hat, dass regionalförderungen in zukunft nicht mehr eingefroren werden können, wenn eu-mitgliedstaaten das haushaltsdefizit überschreiten. es würde die falschen treffen, wenn regionen für fehlverhalten ihrer regierungen bestraft werden." & 310 & high & High & Power & NA & NA & 2019-02-13 & 2019 & 3 & POL
Frame & high-very high & National & <500 & -0.7948903 & -0.1706634 & 1.0035124 & 0.5015415 & -0.3119516 & 9.0 & 0.2648514 & -0.9910023 & Payer & Domestic & European & Mixed & Domestic|POL & Neutral\\
\addlinespace
Austria & http://vorarlberg.orf.at/news/stories/2849545/ & 55 & vorarlberg.orf.at & Public & Online and Offline & Regional/Local & very low = CP mentioned once & Public services & Positive & National + Subnational & No myth & NA & NA & NA & NA & NA & NA & NA & NA & Austria & gratis schulabschluss weiterhin möglich & 2017-06-18 & europäischer sozialfonds & der ministerrat hat beschlossen, dass der pflichtschulabschluss in österreich weiterhin kostenlos nachgeholt werden kann. in vorarlberg kann der abschluss bei der vhs götzis und beim bfi der ak absolviert werden. für die jahre 2018 bis 2021 stellen bund, länder und europäischer sozialfonds mehr als 111,5 millionen euro für die "initiative erwachsenenbildung" zur verfügung, so bildungsministerin sonja hammerschmid (spö). die initiative wurde 2012 ins leben gerufen, damit gering qualifiziert menschen bessere chancen am arbeitsmarkt erhalten. in österreich lebende jugendliche und erwachsene können kostenlos den pflichtschulabschluss nachholen. in vorarlberg kann der nachträgliche pflichtschulabschluss bei der volkshochschule götzis und beim bfi der arbeiterkammer vorarlberg nachgeholt werden. das land stellt laut bildungslandesrätin bernadette mennel (övp) dafür jährlich 75.000 euro zur verfügung. & 119 & very low & Low & Socio-Economic & NA & NA & 2017-06-18 & 2017 & 2 & ECO
Frame & v.low & Regional & <500 & -0.7948903 & -0.1706634 & 1.0035124 & 0.5015415 & -0.3119516 & 9.0 & 0.2648514 & -0.9910023 & Payer & Domestic & Domestic & Domestic & Domestic|ECO & Positive\\
Austria & http://derstandard.at/2000025453738/Oesterreich-macht-besonders-viele-Fehler-im-Umgang-mit-EU-Geldern & 15 & der Standard & Private/Non-Public & Online and Offline & National & high = CP is most important issue in story (can also cover other issues) & Mismanagement & Negative & National + Other country & 4.No added value & Ineffective goal achievement & Negative & National + Other country & 4.No added value & NA & NA & NA & NA & Austria & österreich macht besonders viele fehler im umgang mit eu-geldern & 2015-11-10 & europäischer fonds für regionale entwicklung & laut analyse des eu-rechnungshofes weist österreich mehr fehler bei eu-projekten auf als der europäische durchschnitt brüssel - österreich weist nach einer analyse des eu-rechnungshofes mehr fehler bei eu-projekten auf als der europäische durchschnitt. so wurden laut oskar herics, österreichs vertreter am europäischen rechnungshof, für 2014 insgesamt 18 transaktionen geprüft. "jede zweite war fehlerhaft", sagte herics am dienstag in brüssel. bis auf das burgenland wurden im vorjahr alle bundesländer geprüft. "fehler gab es in sieben bundesländern", erklärte herics. damit gebe es in österreich quasi eine "sehr gute flächendeckung", konstatierte er. lediglich in niederösterreich fanden die prüfer keine fehlerhaften transaktionen. zehn überprüfungen ausgewählter transaktionen für eu-förderungen betrafen den bereich landwirtschaft. zwei davon wiesen fehler auf, diese betrafen unter anderem flächenbezogene beihilferegelungen. im bereich sozialfonds wurden sieben transaktionen geprüft, davon waren gleich sechs fehlerhaft. dabei ging es etwa um beschäftigungsprojekte mit einer fördersumme von bis zu 900.000 euro aus eu-fonds. hier fehlten belege, kosten wurden falsch berechnet, es kam zu unregelmäßigkeiten bei der auftragsvergabe, erklärte herics. eine weitere fehlerhafte transaktion betraf den europäischer fonds für regionale entwicklung, hier wurden zahlungen zu spät geleistet. überraschendes mehrfachversagen insgesamt waren im bereich der landwirtschaft zwischen 2009 und 2014 eu-weit 47 prozent der transaktionen fehlerhaft, in österreich wiederum lag der schnitt bei 39 prozent. in österreich wiesen von 51 überprüften transaktionen 20 fehler auf. massiv höher im eu-vergleich ist die fehlerquote österreichs im bereich sozialfonds, strukturhilfe und regionale entwicklung (esf und efre). hier lag der eu-schnitt bei 44 prozent, in österreich betrug die fehlerquote 64 prozent. von 22 überprüften fällen wiesen gar 14 fehler auf - damit waren zwei drittel aller transaktionen fehlerhaft, sagte herics. in diesem bereich befindet sich österreich hinter luxemburg und den niederlanden an drittletzter stelle. dieses "mehrfachversagen" ist auch für herics "überraschend". ein fall wies im jahr 2014 in österreich eine fehlerquote von über 20 prozent auf, er wird im bericht des eu-rechnungshofs extra aufgeführt. dieses esf-projekt betraf die eingliederung schwer vermittelbarer arbeitsloser in den arbeitsmarkt durch zeitweilige beschäftigung in gemeinnützigen einrichtungen in vorarlberg. die einnahmen, die bei diesem projekt erwirtschaftet wurden, wie etwa einnahmen aus verkäufen in geschäften, wurden nicht mit den aus dem esf finanzierten posten verrechnet, erklärte herics. laut herics ist in österreich der anteil der fehler "wesentlich höher als sein anteil am budget", insgesamt seien mit diesem rechnungshofbericht die ziele der eu "massiv gefährdet". "die wirksamkeit der kontrollsysteme muss erhöht und die haushaltsführung verbessert werden", forderte der experte. öffentliche ausschreibungen schlecht schneidet österreich auch im bereich der öffentlichen ausschreibungen ab. während im jahr 2012 insgesamt 3,1 prozent des bruttoinlandsprodukts eu-weit ausgeschrieben wurden, waren es hierzulande lediglich 1,5 prozent. "in österreich wird nicht umfassend genug ausgeschrieben", sagte herics. bezüglich der wirtschaftlichkeitsprüfung gab es für das jahr 2014 auch sechs sonderberichte des europäischen rechnungshofs mit österreich-bezug. einer betraf die förderung von erzeugern von erneuerbarer energie, er stellte österreich eine gute umsetzung aus. zwei weitere berichte betrafen die verhütung und behebung von waldschäden. hier wurden in österreich eindeutige mängel festgestellt. so wurden etwa beihilfen für forststraßen ohne nutzen für die waldbrandverhütung ausgezahlt, erklärte herics. auch wurden etwa in kärnten und niederösterreich die höchstsätze für wiederaufforstung massiv erhöht. betrug der höchstsatz in kärnten etwa im jahr 2008 noch 1.800 euro pro hektar, waren es 2010 bereits 6.000 euro. diese sätze wurden "erhöht für die bestmögliche ausschöpfung", konstatierte herics. die övp-europaabgeordnete claudia schmidt kritisierte die eu-mitgliedstaaten für "zu viele fehler im umgang mit eu-geldern". schmidt begrüßte, dass der rechnungshofs im haushaltsjahr 2014 neben der fehlerquote erstmals auch geprüft hat, ob die finanzierten projekte überhaupt ihre politikziele erreichen. demnach erreicht ein viertel der projekte keines der politikziele. "der rechnungshof bestätigt unsere befürchtung, dass die mitgliedstaaten gerade in den strukturfonds geld planlos einsetzen. die projekte müssen für wachstum und beschäftigung etwas bringen", so schmidt. (apa, 10.11.2015) & 641 & high & High & Governance & Socio-Economic & NA & 2015-11-10 & 2015 & 1 & POL
Frame & high-very high & National & 500-1000 & -0.7948903 & -0.1706634 & 1.0035124 & 0.5015415 & -0.3119516 & 9.0 & 0.2648514 & -0.9910023 & Payer & Domestic & European & Mixed & Domestic|POL & Negative\\
Austria & http://derstandard.at/2000050938229/Deutschland-und-Italien-wollen-bei-Digitalisierung-enger-kooperieren & 62 & der Standard & Private/Non-Public & Online and Offline & National & low = CP mentioned more times but NOT important part of story (mainly about others issues) & Infrastructure & Positive & Other country & No myth & NA & NA & NA & NA & NA & NA & NA & NA & Austria & deutschland und italien wollen bei digitalisierung enger kooperieren & 2017-01-18 & strukturfonds & deutscher wirtschaftsminister will eu-fonds stärker nutzen deutschland und italien wollen die digitalisierung ihrer wirtschaft vorantreiben. "deutschland und italien sind gemessen am produktionswert die beiden größten industrienationen in europa. das soll auch im digitalen zeitalter so bleiben", sagte der deutsche wirtschaftsminister sigmar gabriel bei einer deutsch-italienischen wirtschaftskonferenz am mittwoch in berlin. an der tagung nahm auch gabriels italienischer amtskollege carlo calenda teil. der europäische fonds für strategische investitionen (efsi) und die europäischen strukturfonds sollten intensiver genutzt werden, um investitionen nicht zuletzt im digitalbereich zu stärken, forderte gabriel. gemeinsame eckpunkte deutschland und italien wollten für den digitalen wandel in der industrie vor allem in den bereichen standardisierung, unterstützung von kleinen und mittleren unternehmen sowie qualifizierung von arbeitnehmern enger zusammenarbeiten. hierzu habe man gemeinsame eckpunkte vereinbart. bei der konferenz wurden auch die deutsche kanzlerin angela merkel und italiens regierungschef paolo gentiloni erwartet. (apa, 18.1.2017) link efsi & 147 & low & Low & Socio-Economic & NA & NA & 2017-01-18 & 2017 & 2 & ECO
Frame & low-medium & National & <500 & -0.7948903 & -0.1706634 & 1.0035124 & 0.5015415 & -0.3119516 & 9.0 & 0.2648514 & -0.9910023 & Payer & European & European & European & European|ECO & Positive\\
Austria & http://www.salzburg.com/nachrichten/dossier/brexit/sn/artikel/der-brexit-und-die-finanzen-der-eu-247912/ & 68 & Salzburger Nachrichten & Private/Non-Public & Online and Offline & Regional/Local & very low = CP mentioned once & Institutional bargaining over funding & Negative & EU + National & No myth & NA & NA & NA & NA & NA & NA & NA & NA & Austria & der brexit und die finanzen der eu & 2017-05-17 & kohäsionsfonds & bei den verhandlungen über den brexit gilt die aufmerksamkeit derzeit den einmaleffekten, sprich den scheidungskosten. das sind jene verpflichtungen, die großbritannien als eu-mitglied eingegangen ist, wie pensionen für eu-beamte oder zahlungsverpflichtungen aus mehrjährigen projekten. laut kommission sind das zirka 100 mrd. euro, der endgültige betrag wird erst nach den verhandlungen feststehen. daneben sind die strukturellen effekte des brexit auf den eu-finanzrahmen die langfristig größere herausforderung. die briten sind nicht nur mit 10 mrd. euro pro jahr einer der großen nettozahler in der eu, durch ihr ausscheiden verschieben sich auch die finanzierungsanteile der mitgliedsstaaten maßgeblich. das liegt zum einen in dem mit ausnahmen durchsetzten eu-finanzrahmen und zum anderen in der festlegung des rates, dass der eu-haushalt auf 1 prozent des eu-bruttonational einkommens zu beschränken ist. das hat zur folge, dass die nettozahler durch den austritt großbritanniens relativ stärker zur kasse gebeten würden als nettoempfänger. vor allem die niederlande, schweden, deutschland und österreich müssten durch den wegfall bisheriger rabatte signifikant höhere beiträge in kauf nehmen, österreich knapp 400 mill. euro. daher ist die reaktion der nettozahler, der ausfall der britischen beiträge solle durch einsparungen kompensiert werden, prima vista verständlich. doch einsparungen von 10 mrd. euro sind höchst unrealistisch. zur besseren veranschaulichung: 10 mrd. euro sind mehr als die gesamten verwaltungskosten der eu (8,2 mrd. euro) oder das budget für die eu- außenpolitik "globales europa" (9,2) oder die eu-forschungsförderung "horizon 2020" (9,5). und es sind immerhin 20 prozent der ausgaben für die struktur- und kohäsionsfonds oder rund ein fünftel des haushalts für die gemeinsame agrarpolitik. mit einsparungen allein wird man also keine kompensation schaffen. und selbst wenn man die finanzierungslücke je zur hälfte durch einsparungen und beitragserhöhung schließen würde, ändert das nichts an der grundsätzlichen problematik einer notwendigen reform des eu-finanzierungssystems. der brexit bietet dazu eine chance. die nettozahler könnten bei den 2018 beginnenden verhandlungen über den neuen eu-finanzrahmen einer vorläufigen erhöhung zustimmen und im gegenzug eine tiefgreifende reform bei einnahmen und ausgaben fordern. das böte die einmalige chance, ein finanzierungssystem zu entwickeln, das den gemeinsamen europäischen zielen rechnung trägt und die dafür nötigen mittel bereitstellt. einnah men anteile aus einer eu-weiten co2-abgabe oder einer steuer für emissionsüberschreitungen würden sich dafür ebenso eignen wie eine finanztransaktionssteuer. vorschläge gibt es genügend, aber es braucht politischen mut und kreativität, sie umzusetzen. vielleicht schafft das eine neue deutsch-französische achse. & 398 & very low & Low & Power & NA & NA & 2017-05-17 & 2017 & 2 & POL
Frame & v.low & Regional & <500 & -0.7948903 & -0.1706634 & 1.0035124 & 0.5015415 & -0.3119516 & 9.0 & 0.2648514 & -0.9910023 & Payer & Domestic & European & Mixed & Domestic|POL & Negative\\
Austria & https://www.kleinezeitung.at/politik/aussenpolitik/5441822/ & 8 & Kleine Zeitung & Private/Non-Public & Online and Offline & Regional/Local & medium = CP is important part of story & Fraud/Corruption & Negative & EU + Other country & 7.Fraud & NA & NA & NA & NA & NA & NA & NA & NA & NA & eu finanziert ein luftschloss in italien & 2018-06-06 & NA & NA & 526 & medium & Medium & Governance & NA & NA & 2018-06-06 & 2018 & 3 & POL
Frame & low-medium & Regional & 500-1000 & -0.7948903 & -0.1706634 & 1.0035124 & 0.5015415 & -0.3119516 & 9.0 & 0.2648514 & -0.9910023 & Payer & European & European & European & European|POL & Negative\\
\addlinespace
Austria & https://www.sn.at/anzeige/leitl-firmen-fuer-anstellen-von-asylberechtigten-foerdern-24590815 & 66 & sn.at & Private/Non-Public & Online and Offline & Regional/Local & medium = CP is important part of story & Social awareness/inclusion & Positive & National & No myth & Political leverage & Balanced & National & No myth & Jobs & Balanced & National & No myth & Austria & leitl: firmen für anstellen von asylberechtigten fördern & 2018-02-23 & kohäsionsfonds & (sn, apa) statt um verteilungsquoten zu ringen, sollten die eu-mitgliedstaaten die kohäsionsfonds dazu nutzen, migranten in unternehmen auszubilden und zu beschäftigen: wkö-chef christoph leitl denkt in einem gespräch mit "frankfurter allgemeinen zeitung" 1.000 euro pro monat für jeden eingestellten asylberechtigten an - drei jahre lang. "die integration kann nur über die betriebe laufen, deshalb müsste man sie aus den kohäsionsfonds fördern." unternehmen in der eu sollen für jeden eingestellten asylberechtigten drei jahre lang 1.000 euro pro monat erhalten, so leitl, der auch präsident der vereinigung der europäischen wirtschaftskammern eurochambres ist. damit sich einheimische nicht benachteiligt fühlen, sollten neben den ausländern auch langzeitarbeitslose hilfen aus den fonds für arbeit und soziales erhalten. eine solche "integrationsprämie", für die die europäische gemeinschaft einen großteil der kosten trüge, würde die aufnahmebereitschaft für asylberechtigte deutlich erhöhen, glaubt leitl. zugleich könnte es die einbeziehung in den arbeitsmarkt und in die gesellschaft der gastländer fördern. "die integrationsprämie könnte helfen, europäische solidarität nicht über eine quote zu erreichen, was immer verärgerung und bevormundungsängste auslöst, sondern über ein anreizmodell", resümiert leitl. finanzieren ließe sich das programm aus nicht abgerufenen mitteln am ende einer förderperiode. der plan stehe im einklang mit der systematik und den richtlinien der europäischen kohäsionsfonds. & 202 & medium & Medium & Socio-Economic & Power & Socio-Economic & 2018-02-23 & 2018 & 3 & ECO
Frame & low-medium & Regional & <500 & -0.7948903 & -0.1706634 & 1.0035124 & 0.5015415 & -0.3119516 & 9.0 & 0.2648514 & -0.9910023 & Payer & Domestic & Domestic & Domestic & Domestic|ECO & Positive\\
Austria & https://www.ots.at/presseaussendung/OTS\_20190327\_OTS0121/vana-kein-geld-aus-eu-fonds-fuer-klimaschaedliche-projekte & 21 & OTS.at & Private/Non-Public & Online only & National & medium = CP is important part of story & Political leverage & Negative & EU + National & No myth & NA & NA & NA & NA & NA & NA & NA & NA & Austria & vana: "kein geld aus eu-fonds für klimaschädliche projekte!" & 2019-03-27 & kohäsionsfonds & straßburg (ots) - das europäische parlament wird heute über die parlamentsposition zum europäischen fonds für regionale entwicklung (efre) und den kohäsionsfonds abstimmen. mit fast 200 milliarden euro in der derzeitigen förderperiode ist der efre mit abstand der größte der struktur- und investitionsfonds der eu. monika vana, stellvertretende fraktionsvorsitzende und regionalpolitische sprecherin der grünen im europaparlament, warnt eindringlich: "der uns vorgelegte bericht ist eine herbe enttäuschung und droht die eu-klimapolitik völlig zu untergraben. während in der dachverordnung der eu fonds eine bewertung der projekte anhand ihres beitrags zum klimaschutz vorgesehen ist, so passiert nun beim efre das gegenteil!" eine klausel im kommissionsbericht zum ausschluss von stark klimaschädlichen förderprojekten wurde vom s\&d berichterstatter und einer konservativen mehrheit gestrichen. ohne sie könnten projekte zum bau von flughäfen, für mülldeponien und für die infrastruktur von fossilen brennstoffen weiterhin mit eu-geldern finanziert werden. "wir grüne haben nun abänderungsanträge eingebracht und wir fordern die anderen fraktionen auf, sich uns anzuschließen, um diese klimapolitische katastrophe zu verhindern. wir haben nicht mehr viel zeit um unser klima in den griff zu bekommen und dieser bericht wäre ein wichtiges signal sowie auch eine hilfe für die regionen europas auf ihrem weg hin zu einer nachhaltigen entwicklung! sollte der bericht in seiner jetzigen form das plenum passieren, so ist dies nicht nur ein schlag gegen den klimaschutz, sondern auch einer ins gesicht der jungen leute, die jeden freitag für ihre zukunft demonstrieren. die eu muss hier mit einer vorbildwirkung vorgehen, alles andere wäre fatal", so vana abschließend. & 249 & medium & Medium & Power & NA & NA & 2019-03-27 & 2019 & 3 & POL
Frame & low-medium & National & <500 & -0.7948903 & -0.1706634 & 1.0035124 & 0.5015415 & -0.3119516 & 9.0 & 0.2648514 & -0.9910023 & Payer & Domestic & European & Mixed & Domestic|POL & Negative\\
Austria & http://www.vol.at/esm-chef-griechenland-rettung-kostet-weniger-als-erwartet/apa-s24\_1326860206 & 91 & vol.at & Private/Non-Public & Online and Offline & Regional/Local & very low = CP mentioned once & Institutional bargaining over funding & Balanced & EU + Other country & No myth & NA & NA & NA & NA & NA & NA & NA & NA & Austria & esm-chef: griechenland-rettung kostet weniger als erwartet & 2015-11-23 & strukturfonds & die rettung griechenlands vor dem staatsbankrott kostet aller voraussicht nach weniger als befürchtet. wie der direktor des euro-rettungsfonds esm, klaus regling, am montagabend in brüssel sagte, wird das land die in einem dritten hilfspaket veranschlagten bis zu 86 milliarden euro nicht benötigen. grund sei, dass für die rekapitalisierung der griechischen banken weniger geld nötig sei als veranschlagt. "all das bedeutet, dass die gesamtsumme von 86 milliarden nicht genutzt werden wird", sagte regling nach einem treffen der euro-finanzminister. demnach wird die bankenrekapitalisierung weniger kosten als die zehn milliarden euro, die bereits im sommer zur auszahlung vorgesehen gewesen seien. die darüber hinaus zur verfügung stehenden 15 milliarden euro für die bankenrettung würden "nicht gebraucht", sagte der esm-chef. "all dies reduziert die 86 milliarden." angesichts des drohenden staatsbankrotts hatten sich die euro-länder und die linksgeführte griechische regierung mitte august auf ein neues rettungspaket geeinigt. es soll über drei jahre laufen und sieht eine reihe von auflagen und reformen vor, die athen erfüllen muss, um die gelder zu erhalten. am montag beschloss das esm-direktorium die auszahlung von weiteren zwei milliarden euro an athen. sie sind insbesondere für die schuldentilgung und für von über eu-strukturfonds kofinanzierte projekte bestimmt. & 199 & very low & Low & Power & NA & NA & 2015-11-23 & 2015 & 1 & POL
Frame & v.low & Regional & <500 & -0.7948903 & -0.1706634 & 1.0035124 & 0.5015415 & -0.3119516 & 9.0 & 0.2648514 & -0.9910023 & Payer & European & European & European & European|POL & Neutral\\
Austria & https://orf.at/v2/stories/2348509 & 7 & newsORF.at & Public & Online and Offline & National & very low = CP mentioned once & Political leverage & Negative & EU + Other country & No myth & NA & NA & NA & NA & NA & NA & NA & NA & NA & eu bereitet sanktionen gegen spanien und portugal vor & 2016-07-07 & NA & NA & 378 & very low & Low & Power & NA & NA & 2016-07-07 & 2016 & 2 & POL
Frame & v.low & National & <500 & -0.7948903 & -0.1706634 & 1.0035124 & 0.5015415 & -0.3119516 & 9.0 & 0.2648514 & -0.9910023 & Payer & European & European & European & European|POL & Negative\\
Austria & http://kurier.at/politik/inland/mitterlehner-wir-haben-nichts-zu-verteilen/140.977.759 & 44 & Kurier & Private/Non-Public & Online and Offline & National & very low = CP mentioned once & Solidarity to poor countries/regions & Negative & EU + Other country & 8.Mismanaged & NA & NA & NA & NA & NA & NA & NA & NA & Austria & mitterlehner: "wir haben nichts zu verteilen" & 2015-07-12 & kohäsionsfonds & kurier: herr vizekanzler, über die ursachen für die griechische tragödie tobt ein glaubenskrieg. was ist ihre analyse, warum griechenland unter den krisenländern zum sonderfall wurde? reinhold mitterlehner: ursächlich ist, dass griechenland den beitritt zur eurozone nicht dazu verwendet hat, um konkurrenzfähig zu werden. mit dem günstigen euro wurden die löhne erhöht und importe finanziert, anstatt die griechische wirtschaft produktiv zu machen. ein bezeichnendes beispiel ist, dass griechenland selbst im lebensmittelbereich einen importüberhang hat, was für ein fruchtbares, mediterranes land skurril anmutet. man hat verabsäumt, strukturanpassungen in richtung eines modernen europäischen landes vorzunehmen. die versäumnisse haben nicht nur viel gekostet, sondern einen teil der griechischen bevölkerung massiv getroffen. hätte europa früher druck für strukturreformen machen müssen? der fehler ist nicht, dass helmut kohl und die anderen das nicht gesehen hätten. sie haben einen kohäsionsfonds aufgesetzt, der es den südlichen ländern ermöglichen sollte, ihre wettbewerbsfähigkeit anzupassen. da ist aber viel in falsche kanäle und projekte gelaufen. das war ein grundfehler. und der zweite fehler war, dass man die spielregeln der eurozone, die kontrolle über die budgetdisziplin, erst entwickelt hat, als die ersten probleme auftraten. müsste man nicht auch einmal etwas für jene gruppen in der europäischen bevölkerung machen, die die verlierer solcher reformprozesse sind, damit europa nicht immer nur negativ mit dem erhobenen zeigefinger daherkommt? nein, die sozialunion ist eine notmaßnahme, keine gewährleistung. sozialhilfe wäre eine folge, wenn ein staat die spielregeln der währungsunion nicht einhält. dann bliebe der union nichts anderes übrig, als jenen menschen, für die es ums überleben geht, mit einem notprogramm zu helfen. was ist mit gruppen, die innerhalb der eurozone verlierer solcher anpassungsprozesse sind? wie sollen jugendliche an ein europa glauben, von dem sie nichts als arbeitslosigkeit haben? genau aus diesem grund fördern wir die duale berufsausbildung. auch griechenland hat sich dafür interessiert. wir haben vor ort informiert über kmu-finanzierungen, über ein duales ausbildungssystem usw. das wurde aber abgebrochen, weil in athen eine andere regierung ans ruder kam. hier genau liegt das problem: man braucht einfach funktionierende staatliche strukturen. ohne uns mit griechenland zu vergleichen - gibt es eine lehre, die man aus einem extremfall wie griechenland für österreich ziehen kann? die lehre ist: länder, die sich wirtschafts- und strukturreformen gestellt haben, haben die höchsten wachstumsraten. spanien hat heuer 2,8 prozent, großbritannien hat 2,5 prozent. länder, die ihr system angepasst haben, werden von den märkten belohnt. das stellt sich für österreich als zukunftsaufgabe. wir haben strukturelle mängel und glauben, mit weniger desselben drüberzukommen. wir müssen uns genauso umstrukturieren. wo genau müssen wir uns umstrukturieren? es sind im wesentlichen das pensions-, das gesundheits- und das arbeitsmarktsystem, die effizienter arbeiten müssen. mit einem staatsanteil von 52 prozent geben wir deutlich mehr zur stabilisierung der systeme aus als zum beispiel deutschland. in österreich ist das anspruchsdenken an den staat überdimensional entwickelt. bei krankheit, im alter, bei einem unfall - wo der einzelne, auf sich allein gestellt, überfordert ist - ist die absicherung richtig und wichtig. wir haben aber auch ansprüche auf wohnungen, auf entsprechendes einkommen, im gesundheitsbereich überdimensional, nicht nur auf eine grundversorgung. der staat soll so ziemlich alles übernehmen, das ist die mentalität. wenn ich eine bestimmte mindestsicherung kassiere und dazu eine wohnungsunterstützung, ist der anreiz zum arbeiten kaum mehr da, weil der unterschied zum arbeitseinkommen kaum mehr da ist. unser problem ist, dass wir nicht mehr eigenverantwortung und leistung forcieren. die überzogene absicherung ist schon so im denken mancher leute drinnen, dass jeder reformen nur beim anderen versteht, nicht bei sich selbst. sie wollen also den sozialstaat rückbauen - klingt nicht nach wahlschlager. wir sollen nicht den armen, die es brauchen, etwas wegnehmen. aber man sieht es auch am flüchtlingsandrang: es spricht sich herum, dass österreich ein sozial toll abgesichertes land ist. wir müssen schauen, dass wir unsere systeme optimieren. wir müssen den zugang zur mindestsicherung verschärfen und eine kontrolle einführen. oder ein anderes beispiel: wir haben wahnsinnig viel geld für den letzten teil des lebens aufgebaut, für pension, gesundheit, pflege. für den ersten lebensabschnitt, wo wir zukunftsfit sein müssen, für kindergarten, bildungssystem und unis, haben wir eher zu wenig geld. wir müssen versuchen, diese kurve zu drehen. ich gebe zu, das ist alles andere als lustig und populär, aber es führt kein weg daran vorbei. erfreulich ist, dass die anzahl der leute, die das mittragen können, größer wird, weil sie im fernsehen mitverfolgen, wohin reformverweigerung führt. nach griechenland. apropos - der rechnungshof zeigt auf, dass ausgerechnet die lehrer, die nicht alters-arbeitslos werden können, massenhaft in frühpension gehen. finden sie das okay? natürlich sollte der staat ein vorbild für die privatwirtschaft sein, weil sonst keine stimmigkeit da ist. da muss der staat eine andere kultur vorleben. die regierung hat zwei mal einen angekündigten arbeitsplatzgipfel verschoben. warum? weil uns die gipfeldramaturgie nicht hilfreich erscheint: alles wartet vor der tür mit fotoapparat und kamera, dass wir eine patentlösung vorlegen, die es nicht gibt. wir werden jetzt intern das thema aufarbeiten und erst verkaufen, wenn wir ergebnisse haben. wir werden die sozialpartner einbinden, weil auch dort viel zu wenig weitergeht. unsere überdimensionierte interessensvertretungs-landschaft verbreitet nur thesen und forderungen, aber das zusammenspiel, damit lösungen herauskommen, hat potenzial nach oben. die klienteldarstellung hat nur sinn, wenn es etwas zu verteilen gibt. aber wir haben nichts zu verteilen, sondern müssen in manchen bereichen einschränken. wieso regiert die regierung so zaghaft? im wort nationalrat steckt die silbe "national" und nicht "länder" oder "sozialpartner". die steuerreform hat gezeigt, die regierung ist handlungsfähig. sie hat die steuerreform gemacht, weil sie überzeugt war, dass sie für die konjunktur richtig ist. aber dann gab es eine kakofonie über die gegenfinanzierung in zehnfacher lautstärke. das war der wirtschaftsbund, wo sie herkommen. das beispiel zeigt, dass es einer regierungspartei nicht mehr möglich ist, reine klientelpolitik zu machen, sonst bewegt sich gar nichts mehr. bei der steuerreform hoffe ich, dass wir jetzt den kaugummi der gegenfinanzierung endlich abschütteln und die positiven elemente gut verkaufen. im herbst wird die stimmung besser werden. 2016 wird auch die wirtschaft mehr wachsen. halten sie die trendumkehr im sozialsystem, die sie anstreben, mit der spö für machbar? wir sind eine zweckgemeinschaft, die keine andere wahl hat. wir haben den konflikt um die asylpolitik ausgesprochen. der haussegen hängt wieder gerade? ja, es bleibt auch nichts anderes übrig. die bevölkerung ist von streit nicht begeistert. die erwartungshaltung, dass wir sagen, es reicht, werden wir nicht erfüllen, weil es der bevölkerung nicht reicht. es würde nur ein dritter profitieren, der sich in der zuschauerrolle befindet. ein dritter profitiert aber auch dann, wenn die regierung zu wenig leistung bringt. wann ist mit den von ihnen genannten reformen zu rechnen? der 17. november für die bildungsreform pickt. der 29. februar 2016 ist der termin für eine pensionsreform. das arbeitsmarktprogramm bereiten wir gerade intern vor. bei der steuerreform haben sie gesagt: wenn sie scheitert, stellt sich die frage nach der existenzberechtigung der regierung. sehen sie den termin für die bildungsreform auch so streng? ich möchte nicht schon wieder wenn-dann-konditionen aufstellen. aber ja, die bildungsreform ist auf jeden fall ein sehr wichtiges projekt. & 1161 & very low & Low & Values & NA & NA & 2015-07-12 & 2015 & 1 & ECO
Frame & v.low & National & +1000 & -0.7948903 & -0.1706634 & 1.0035124 & 0.5015415 & -0.3119516 & 9.0 & 0.2648514 & -0.9910023 & Payer & European & European & European & European|ECO & Negative\\
\addlinespace
Austria & https://www.ots.at/presseaussendung/OTS\_20180921\_OTS0186/foederalismus-und-was-kommt-dann & 94 & APA-OTS & Private/Non-Public & Online only & National & low = CP mentioned more times but NOT important part of story (mainly about others issues) & Empowerment of institutions & Positive & EU + National & No myth & Civic participation/collaboration & Positive & EU + National & No myth & NA & NA & NA & NA & Austria & föderalismus. und was kommt dann? & 2018-09-21 & kohäsionspolitik & wien/eisenstadt (ots) - internationale expertinnen und entscheidungsträgerinnen aus österreich und der eu diskutieren die rolle von kommunen und regionen im künftigen europa. in der keynote erläuterte bundesminister josef moser die reformideen zum föderalismus und karl-heinz lambertz, präsident des ausschusses der regionen, zog den fokus auf die europäische ebene. [eisenstadt, 20-21. september 2018] die derzeitigen strukturen der staaten werden den anforderungen des 21. jahrhunderts nicht mehr ausreichend gerecht, ist der einhellige tenor bei der international besetzten konferenz "innovation und fortschritt im bundesstaat", heute in eisenstadt. dabei geht es noch vielmehr als um steuerung. in einem multilevel-governance-ansatz, der auch viele verflechtungen inne hat und sich durch weitgehende funktionale differenzierungen auszeichnet, müssen steuerung, zuständigkeiten, partizipation und finanzierung klar und transparent nachvollziehbar sein und auch erfolgreich ziele verfolgen können. diese konferenz präsentiert neue ansätze zur besseren und effektiveren zusammenarbeit von staaten und regionen im föderalen netzwerk "europäische union". die internationale expertinnen-runde ist sich einig, dass die großen herausforderungen von klimawandel über digitalisierung bis zu den bevorstehenden demographischen umbrüchen nur im zusammenwirken der unterschiedlichen ebenen im bundesstaat und der europäischen union gelöst werden können. die europäischen regionen werden dabei immer wichtiger, allerdings sind deren kompetenzen und rechte in der eu nur sehr schwach ausgeprägt. auch in österreich können die bundesländer aufgrund der unklaren kompetenzverteilungen ihre potenziale nicht ausschöpfen. die reformperspektiven der bundesregierung beim föderalismus in österreich erläuterte josef moser, bundesminister für verfassung, reformen, deregulierung und justiz, in seiner keynote zur eröffnung der konferenz. dabei verwies er darauf, dass es nicht primär um die frage der kompetenzverteilung zwischen bund und länder geht, sondern um klarere strukturen. "ein effizienter staat ist keine frage des zentralismus oder föderalismus. wir können unabhängig von dieser frage klarere strukturen schaffen und doppelgleisigkeiten abbauen, um eine höhere effizienz in österreich zu erreichen", so josef moser. karl-heinz lambertz, präsident des europäischen ausschusses der regionen (adr) hob die wichtigkeit der eu-kohäsionspolitik für regionen der eu hervor und meinte, dass europa verstärkt von unten gebaut und gelebt werden müsse. die eu kohäsionspolitik muss zu einer europäisierung kommunaler und regionaler politik führen. "die kohäsionspolitik sorgt für den zusammenhalt in der union, indem sie das leben der menschen vor ort konkret verbessert. sie schafft jobs, ausbildungsplätze und notwendige infrastrukturen. jeder lokale, regionale politiker ist ebenso oder mindestens ebenso ein europapolitiker, wie jeder vertreter im europäischen parlament oder jeder eu-kommissar oder der präsident des europäischen rates", appellierte lambertz. & 398 & low & Low & Power & Socio-Economic & NA & 2018-09-21 & 2018 & 3 & POL
Frame & low-medium & National & <500 & -0.7948903 & -0.1706634 & 1.0035124 & 0.5015415 & -0.3119516 & 9.0 & 0.2648514 & -0.9910023 & Payer & Domestic & European & Mixed & Domestic|POL & Positive\\
Austria & https://burgenland.orf.at/v2/news/stories/2928631/ & 3 & burgenland.orf.at & Public & Online and Offline & National & low = CP mentioned more times but NOT important part of story (mainly about others issues) & Economic development & Positive & Subnational & No myth & Ineffective goal achievement & Balanced & Subnational & No myth & NA & NA & NA & NA & NA & eu: övp für stärkung der regionen & 2018-08-06 & NA & NA & 262 & low & Low & Socio-Economic & Socio-Economic & NA & 2018-08-06 & 2018 & 3 & ECO
Frame & low-medium & National & <500 & -0.7948903 & -0.1706634 & 1.0035124 & 0.5015415 & -0.3119516 & 9.0 & 0.2648514 & -0.9910023 & Payer & Domestic & Domestic & Domestic & Domestic|ECO & Positive\\
Austria & http://www.oe-journal.at/Aktuelles/\%212015/1115/W3/31911ApkEU.htm & 74 & oe-journal.at & Private/Non-Public & Online only & National & high = CP is most important issue in story (can also cover other issues) & Mismanagement & Negative & EU + National & No myth & Bureaucracy and/or delays & Negative & EU + National & No myth & NA & NA & NA & NA & Austria & eu assistance: error rate is too high & 2015-11-20 & europäischer fonds für regionale entwicklung & bundesrat diskutiert jahresbericht 2014 des europäischen rechnungshofs brüssel/wien (pk) - die eu hat probleme mit der rechtmäßigen auszahlung ihrer mittel, und davon ist auch österreich nicht ausgenommen. den meisten fehlern liegen aber keine betrugsabsichten zugrunde. so könnte man kurz den jahresbericht des europäischen rechnungshofs zusammenfassen, den oskar herics, österreichisches mitglied im eu-rechnungshof, und seine mitarbeiterinnen margit spindelegger und thomas obermayr am 18.11. im eu-ausschuss des bundesrats den ländervertreterinnen präsentierten. herics forderte in diesem zusammenhang eine grundlegende veränderung im finanzmanagement der eu und ihrer mitgliedstaaten ein. wie der europäische rechnungshof feststellt, konnte er zwar die zuverlässigkeit der rechnungsführung der eu für 2014 bestätigen, die zahlungen sind jedoch in wesentlichem ausmaß mit fehlern behaftet. deshalb gab der rechnungshof auch ein negatives prüfungsurteil zu ihrer rechtmäßigkeit und ordnungsmäßigkeit ab. wie herics berichtete, beträgt die geschätzte fehlerquote für das jahr 2014 4,4\% und liegt damit konstant über der wesentlichkeitsschwelle von 2\%, stellte herics fest. die höchsten fehlerquoten seien bei den regional- und sozialfonds (5,7 \%) und im bereich wachstum und beschäftigung (5,6 \%) festzustellen, woraus man schließen könne, dass es einen zusammenhang zwischen aufgabengebiet und fehlerrisiko gibt. auch österreich schneidet nicht immer gut ab auch österreich schneidet bei der diesbezüglichen bewertung nicht gut ab. so lagen beispielsweise die fehler bei kohäsionszahlungen bei 64\%, und damit belegt österreich die drittletzte stelle in der eu und hat sich sogar noch verschlechtert. die durchschnittliche fehlerquote in der eu beläuft sich in diesem bereich auf 44\%. wesentlich besser stellt sich die lage im bereich landwirtschaft dar, hier lag die heimische fehlerquote bei 39\%, im eu-durchschnitt bei 47\%. der europäische rechnungshof hat im vorjahr 18 transaktionen nach österreich geprüft, davon betrafen 10 eler (europäischer landwirtschaftsfonds), 7 esf (europäischer sozialfonds) und 1 efre (europäischer fonds für regionale entwicklung), wobei sich 9 transaktionen als fehlerhaft erwiesen. österreich musste insgesamt im bereich struktur- und kohäsionsfonds 2007-2013 18 mio. € bzw. 1,5 \% der erhaltenen eu-zahlungen rücküberweisen und liegt damit an fünftletzter stelle. besonders schwere mängel stellte der europäische rechnungshof bei der verwaltung und kontrolle des programms "beschäftigung österreich 2007-2013" fest. staaten ziehen keine konsequenzen aus prüfergebnissen herics informierte zudem, dass es einen erheblichen rückstand bei der ausschöpfung der mittel aus dem mehrjährigen europäischen investitions- und strukturfonds gibt. so seien aus dem jahr 2014 erst 77 \% von 403 mrd. € ausbezahlt worden. die ausschöpfungsrate liegt im einzelnen zwischen 50 \% bis 92 \%, in österreich bei 84 \%. herics kritisierte auch, dass die europa-2020-strategie nicht mit den für die eu-haushaltzyklen maßgeblichen zeiträumen abgestimmt ist. die mitgliedstaaten würden die verwirklichung der ziele nicht ausreichend berücksichtigen, so der europäische rechnungsprüfer. herics drängte nachdrücklich darauf, die prüfungsergebnisse des rechnungshofs ernst zu nehmen und stellte mit bedauern fest, dass etwa die probleme mit der kohäsion zu keinen konsequenzen in den mitgliedstaaten führen. es würden weder die gründe für die fehler analysiert, noch der fokus auf wirtschaftlichkeit und wirksamkeit gelegt. in den einzelnen ländern konzentriere man sich eher auf die mittelabschöpfung und vernachlässige in den berichten den aspekt, ob die ziele auch tatsächlich erreicht wurden. die risiken einer unrechtmäßigen mittelvergabe zögen unangenehme folgen nach sich, so herics, etwa eine finanzkorrektur, vertragsverletzungsverfahren oder die tatsache, dass eu-pläne nicht eingehalten werden. ruf nach bürokratieabbau in der diskussion beklagten vor allem edgar mayer (v/v) und sonja zwazl (v/n) die überbordende bürokratie bei der antragstellung. zwazl fehlte insbesondere eine kompetente stelle bei der kommission, die bei anfallenden fragen auskunft geben könnte. sie unterstrich zudem mit nachdruck, dass bei aller notwendigkeit öffentlicher europaweiter ausschreibungen möglichst viele heimische betriebe zum zug kommen, damit die wirtschaftskraft in der region bleibt. auf die besonderen bürokratischen probleme in der landwirtschaft machten martin preineder (v/n) und ferdinand tiefnig (v/o) aufmerksam. so gebe es beispielsweise bei der flächenidentifizierung unterschiedliche messsysteme, wodurch die fehlerquote systemimmanent sei, sagte preineder. wenn grundstücke von firmen in anspruch genommen werden, so sei auf einmal der landwirt dafür verantwortlich, dass die fläche herausgenommen wird, brachte tiefnig ein weiteres beispiel. er wies auch darauf hin, dass nicht immer die wirtschaftlichkeit im vordergrund stehen dürfe, sondern man auch auf soziale aspekte achten müsse. eduard köck (v/n) informierte, dass die bauernvertretung der kommission 67 vorschläge zur vereinfachung der antragstellung gemacht habe, man werde diese auch dem eu-rechnungshof zur verfügung stellen. der rechnungshof sei bemüht, die vorschriften zu vereinfachen, es gebe auch bereits pauschalierungsmöglichkeiten, reagierte herics auf die anregungen. zugleich gab er aber zu bedenken, dass vereinfachungen nicht zu einer verfehlung der wirkungen und zielsetzungen führen dürfe. auch gebe es bereits toleranzen bei der flächenvermessung, stellte er fest. jedenfalls wäre es notwendig, durch schulungsmaßnahmen die fehlerquote zu senken, meinte er in richtung monika mühlwerth (f/w), die sich darüber beklagt hatte, dass es viel unkenntnis und zu wenig kooperation gebe. ihr zufolge müssten die kontrollsysteme verbessert und die vorschriften vereinfacht werden. herics zufolge wäre es auch aufgabe der kommission, den einzelnen ländern stärker zur hand zu gehen, bekräftigte er gegenüber stefan schennach (s/w). der europäische rechnungshof ist eine unabhängige externe prüfstelle, die 1977 eingerichtet wurde und ihren sitz in luxemburg hat. als kollegialorgan gehören ihm 28 mitglieder - jeweils ein mitglied aus jedem eu-land - an. er prüft die rechnung über alle einnahmen und ausgaben der eu sowie ihrer agenturen und dezentralen einrichtungen nach den maßstäben der wirtschaftlichkeit. in diesem zusammenhang hält es herics für erforderlich, mehr flexibilität im eu-haushalt walten zu lassen, um auf aktuelle ereignisse, wie etwa die flüchtlingskrise, besser reagieren zu können. der europäische rechnungshof kann auch die europäische zentralbank prüfen, jedoch nicht deren kerntätigkeit. jedenfalls könne er aber ein genaues auge auf das neue bankenaufsichtssystem der eu im rahmen der bankenunion werfen. er prüft auch die europäische investitionsbank, wenn sie in umsetzung von eu-programmen eingebunden ist. & 946 & high & High & Governance & Governance & NA & 2015-11-20 & 2015 & 1 & POL
Frame & high-very high & National & 500-1000 & -0.7948903 & -0.1706634 & 1.0035124 & 0.5015415 & -0.3119516 & 9.0 & 0.2648514 & -0.9910023 & Payer & Domestic & European & Mixed & Domestic|POL & Negative\\
Austria & https://www.ots.at/presseaussendung/OTS\_20180321\_OTS0186/strasser-budget-garantiert-baeuerlichen-familienbetrieben-eine-sichere-zukunft & 47 & APA-OTS & Private/Non-Public & Online only & National & very low = CP mentioned once & Solidarity to poor countries/regions & Positive & EU + National & No myth & NA & NA & NA & NA & NA & NA & NA & NA & Austria & strasser: budget garantiert bäuerlichen familienbetrieben eine sichere zukunft & 2018-03-21 & europäischer fonds für regionale entwicklung & wien (ots/övp-pk) - sparen im system und nicht bei den menschen: unter diesem motto stand das heute im nationalrat vorgestellte doppelbudget für 2018/19. finanzminister hartwig löger präsentierte erstmals seit 64 jahren einen budgetüberschuss, seit 1954 nimmt der staat mehr geld ein, als er ausgibt. "dieses doppelbudget ist ein budget der veränderung. der staat spart bei sich selbst und nicht bei den menschen. der rotstift wird hauptsächlich bei der verwaltung angesetzt und ineffiziente maßnahmen werden gestrichen. erstmals seit jahren werden die österreicherinnen und österreicher nicht durch neue steuern belastet, sondern durch konkrete maßnahmen wie dem familienbonus entlastet", erklärt övp-landwirtschaftssprecher bauernbund-präsident abg. di georg strasser anlässlich der heutigen budgetrede. doppelbudget sichert kleinstrukturierte landwirtschaft "es ist wichtig, dass vor allem unsere kleinstrukturierten bäuerlichen familienbetriebe eine sichere zukunft haben und das ist mit dem budget, das finanzminister hartwig löger heute vorgestellt hat, gesichert. unsere bäuerlichen familienbetriebe sind im vergleich zur internationalen konkurrenz klein, produzieren aber in höchster qualität. sie brauchen und verdienen unsere volle unterstützung", so strasser. im bereich nachhaltigkeit und tourismus erhöht sich die auszahlungsobergrenze von 2,1 milliarden euro im jahr 2017 auf 2,2 milliarden euro im jahr 2018. 60 prozent der auszahlungen werden auf rechnung der eu in variabler gebarung (direktzahlungen, gemeinsame marktorganisation, eu-mittel für die ländliche entwicklung, europäischer fonds für regionale entwicklung) getätigt. die inhaltlichen schwerpunkte umfassen unter anderem die förderung der landbewirtschaftung sowie einer nachhaltigen ländlichen entwicklung einschließlich umweltgerechter produktionsverfahren, die landwirtschaft in benachteiligten gebieten und berggebieten sowie auch den schutz vor naturgefahren. "der ländliche raum ist in österreich ein wichtiger lebensraum. wir werden immer für die gleichwertigkeit der lebensbedingungen auch in ländlichen gebieten eintreten", sagt strasser. "jeder cent für die landwirtschaft ist zugleich eine konsumentenstützung für leistbare lebensmittelpreise. jeder konsument profitiert von den öffentlichen mitteln für die landwirtschaft." (schluss) & 297 & very low & Low & Values & NA & NA & 2018-03-21 & 2018 & 3 & ECO
Frame & v.low & National & <500 & -0.7948903 & -0.1706634 & 1.0035124 & 0.5015415 & -0.3119516 & 9.0 & 0.2648514 & -0.9910023 & Payer & Domestic & European & Mixed & Domestic|ECO & Positive\\
Austria & http://aktien-portal.at/shownews.html?id=44127 & 11 & aktien-portal.at & Private/Non-Public & Online only & National & very low = CP mentioned once & Infrastructure & Positive & National & No myth & NA & NA & NA & NA & NA & NA & NA & NA & Austria & strabag & 2016-08-09 & kohäsionsfonds & 46 kilometer lange bahnstrecke im süden des landes wird erneuert - auftragswert liegt bei etwa 34 mio. euro der österreichische bauriese strabag hat sich einen bahn-auftrag in tschechien gesichert. der konzern erneuert die 46 kilometer lange bahnstrecke inklusive brücken und bahnübergängen zwischen okrisky und zastavka u brna im süden des landes. die streckengeschwindigkeit soll erhöht werden. der auftragswert betrage rund 34 mio. euro, teilte das unternehmen heute, dienstag, mit. "der eisenbahnbau ist ein zukunftsträchtiger infrastrukturbereich. in diesem geschäftsfeld setzen wir jährlich etwa € 550 mio. um", so strabag-chef thomas birtel. das infrastrukturprojekt wird den angaben zufolge von der eu im rahmen des kohäsionsfonds kofinanziert. die bauarbeiten starten noch diesen sommer; im juli 2017 soll der auftrag abgewickelt sein. auftraggeber ist die tschechische eisenbahn-infrastrukturbehörde (sprava zeleznicni dopravni cesty). & 128 & very low & Low & Socio-Economic & NA & NA & 2016-08-09 & 2016 & 2 & ECO
Frame & v.low & National & <500 & -0.7948903 & -0.1706634 & 1.0035124 & 0.5015415 & -0.3119516 & 9.0 & 0.2648514 & -0.9910023 & Payer & Domestic & Domestic & Domestic & Domestic|ECO & Positive\\
\addlinespace
Austria & https://www.sn.at/politik/weltpolitik/kommission-will-mehr-eu-geld-in-krisenlaender-leiten-28540915 & 51 & sn.at & Private/Non-Public & Online and Offline & Regional/Local & medium = CP is important part of story & Solidarity to poor countries/regions & Positive & EU & No myth & NA & NA & NA & NA & NA & NA & NA & NA & Austria & kommission will mehr eu-geld in krisenländer leiten & 2018-05-29 & kohäsionsfonds & die eu-kommission hat die im nächsten langfrist-budget geplante umschichtung von hilfen für strukturschwache regionen in richtung südeuropa verteidigt. "etwa die slowakei, das baltikum oder polen bekommen in unserem vorschlag weniger geld, weil sie wettbewerbsstärker geworden sind", sagte haushaltskommissar günther oettinger. schon länger stagnierende länder wie italien, bekämen dann mehr geld. die bisherigen hauptempfänger in osteuropa hätten die unterstützungen richtig eingesetzt, sagte oettinger am dienstag im europaparlament in straßburg. er geht daher fest davon aus, dass einige der osteuropäischen eu-länder im nächsten jahrzehnt von der wirtschaftskraft her zum europäischen durchschnitt aufholen und erstmals auch geld in das budget einzahlen müssten. "die kohäsionspolitik war erfolgreich", sagte oettinger. die eu-kommission beschließt am dienstag ihre pläne für die zukunft der regional- und kohäsionsfonds. sie sollen eine angleichung der lebensverhältnisse in der eu fördern und sind nach den agrarausgaben der größte posten im eu-budget. die eu-kommission hat vorgeschlagen, dafür im nächsten eu-finanzrahmen von 2021 bis 2027 rund 373 milliarden euro zur verfügung zu stellen. die eu-kommission unterstützt dabei auch den deutschen vorschlag, gebiete mit einer hohen zahl von flüchtlingen künftig stärker zu berücksichtigen. dies könnte zu lasten osteuropäischer staaten gehen, welche die flüchtlingsaufnahme verweigern. hauptankunftsländer für flüchtlinge wie italien oder griechenland, aber auch deutsche regionen könnten davon profitieren. allerdings wird deutschland wegen seiner großen wirtschaftskraft nach dem brüsseler vorschlag insgesamt weniger bekommen. denn wegen des eu-austritts großbritanniens und neuer aufgaben bei verteidigung, grenzschutz und forschung sollen bei der kohäsionspolitik einsparungen erfolgen. den plänen müssen das europaparlament und die mitgliedstaaten noch zustimmen. & 255 & medium & Medium & Values & NA & NA & 2018-05-29 & 2018 & 3 & ECO
Frame & low-medium & Regional & <500 & -0.7948903 & -0.1706634 & 1.0035124 & 0.5015415 & -0.3119516 & 9.0 & 0.2648514 & -0.9910023 & Payer & European & European & European & European|ECO & Positive\\
Austria & http://burgenland.orf.at/news/stories/2911408/ & 75 & burgenland.orf.at & Public & Online and Offline & National & very high = CP is most important issue + CP is mentioned in title/headline & Institutional bargaining over funding & Negative & EU + National + Subnational & No myth & Economic development & Positive & EU + National + Subnational & No myth & NA & NA & NA & NA & Austria & burgenland will weitere eu-förderungen & 2018-05-09 & kohäsionsfonds & am mittwoch wird der europatag der europäischen union gefeiert. landtagspräsident christian illedits (spö) nimmt den tag zum anlass, um zu bekräftigen, warum das burgenland auch künftig fördergeld aus brüssel braucht. von 1995 bis heute war das burgenland zwei mal ziel-1-gebiet, danach in der sogenannten phasing-out-phase. aktuell hat das burgenland bis 2020 den status einer übergangsregion. in den vergangenen jahren vollzog das burgenland einen außergewöhnlichen wirtschaftlichen aufholprozess. dieser aufschwung führe aus derzeitiger sicht dazu, dass das burgenland nach 2020 gemeinsam mit sechs anderen übergangsregionen keine entsprechende strukturförderungen bekommen würde, sagt landtagspräsident illedits. "wir sind der meinung, dass ohne kohäsionsfonds zukünftig die weiterentwicklung schwierig möglich ist. es würde ein stopp in diesen regionen bei der weiterentwicklung passieren. das ist sicher nicht im europäischen interesse, nämlich viel geld zu investieren und dann danach die entwicklung zu stoppen", sagt illedits. dass das burgenland, obwohl es bereits 90 prozent des europäischen bruttoinlandsprodukts erreicht hat, weiterhin förderungen benötigt, begründet illedits neuerlich mit der besonderen geographischen lage des landes. man wolle nun österreichs eu-ratsvorsitz ab 1. juli und die damit verbundenen besuche von eu-kommissionspräsident jean claude juncker und eu-haushaltskommissar günther oettinger in österreich dazu nutzen, um für die anliegen des burgenlands lobbying zu betreiben, so illedits. auch die övp meldete sich zum europatag zu wort. probleme wie friedenssicherung, bewältigen der flüchtlingsströme sowie wirtschaftliches fortkommen könne man nur gemeinsam angehen. "nationalstaaten alleine sind dazu nicht in der lage. es bedarf akkordierter aktionen", erklärten landesparteiobmann thomas steiner und der zweite landtagspräsident rudolf strommer in einer gemeinsamen aussendung. & 255 & very high & High & Power & Socio-Economic & NA & 2018-05-09 & 2018 & 3 & POL
Frame & high-very high & National & <500 & -0.7948903 & -0.1706634 & 1.0035124 & 0.5015415 & -0.3119516 & 9.0 & 0.2648514 & -0.9910023 & Payer & Domestic & European & Mixed & Domestic|POL & Negative\\
Austria & https://www.derstandard.at/story/2000078493159/eu-will-in-osteuropa-milliarden-einsparen & 6 & der Standard & Private/Non-Public & Online and Offline & National & low = CP mentioned more times but NOT important part of story (mainly about others issues) & Institutional bargaining over funding & Negative & EU + Other country & No myth & NA & NA & NA & NA & NA & NA & NA & NA & NA & eu will in osteuropa milliarden einsparen & 2018-04-23 & NA & NA & 473 & low & Low & Power & NA & NA & 2018-04-23 & 2018 & 3 & POL
Frame & low-medium & National & <500 & -0.7948903 & -0.1706634 & 1.0035124 & 0.5015415 & -0.3119516 & 9.0 & 0.2648514 & -0.9910023 & Payer & European & European & European & European|POL & Negative\\
Austria & https://www.ots.at/presseaussendung/OTS\_20181102\_OTS0085/kaernten-eu-landesregierung-beschliesst-1-eu-bericht-fuer-1-europapolitische-stunde-im-landtag & 79 & APA-OTS & Private/Non-Public & Online only & National & high = CP is most important issue in story (can also cover other issues) & Economic development & Positive & EU + Subnational & No myth & NA & NA & NA & NA & NA & NA & NA & NA & Austria & kärnten-eu: landesregierung beschließt 1. eu-bericht für 1. europapolitische stunde im landtag & 2018-11-02 & regionalpolitik & klagenfurt (ots/lpd) - ein absolutes novum wird es in der sitzung des kärntner landtages am 22. november geben: zum allerersten mal wird es dann eine in der neuen landesverfassung festgeschriebene "europapolitische stunde" geben. diskutieren wird man darin den eu-bericht, der am kommenden dienstag in der regierungssitzung von landeshauptmann peter kaiser zur beschlussfassung vorgelegt wird. "wir unterstreichen damit einmal mehr die bedeutung der europäischen union für kärnten. es ist es entscheidend, europa aktiv mitzugestalten, gerade in zeiten von globalen und auch von innereuropäischen herausforderungen wie dem brexit. wir wollen und werden kärntens stimme in europa immer lauter werden lassen. gleichzeitig ist es im sinne eines stärkeren europabewusstseins nötig, auf allen ebenen deutlich zu machen, wie wichtig und unverzichtbar die eu gerade für regionen wie kärnten ist", betont kaiser. er hebt hervor, dass kärnten in vielfacher weise von der eu profitiere. unterstützung und förderung gebe es bei der standortentwicklung, bei der grenzüberschreitenden zusammenarbeit, in vielen bereichen von infrastruktur, forschung und innovation bis hin zu bildung oder sozialem. der landeshauptmann unterlegt dies mit konkreten zahlen: so sei kärnten seit dem eu-beitritt österreich 1995 mit über einer milliarde euro netto-empfänger, und würde kärnten bis zum ende der gegenwärtigen finanzperiode 2020 insgesamt rund 420,52 mio. euro an mitteln allein aus der eu-regionalpolitik erhalten. diese summe setzt sich aus 314,22 mio. euro aus dem europäischen fonds für regionale entwicklung (efre) und 106,30 mio. euro aus dem europäischen sozialfonds (esf) zusammen. zusätzlich fließen eu-mittel aus zahlreichen weiteren eu-förderprogrammen wie erasmus+, europa für bürgerinnen und bürger, horizon 2020, usw. nach kärnten. seit 2014 bis juli 2018 wurden hier rund 94 mio. euro an kärntner projektpartner genehmigt und ausbezahlt. "damit erhielt kärnten in der aktuellen eu-förderperiode so hohe finanzielle mittel wie noch nie aus diesen eu-programmen", erklärt kaiser. der eu-bericht geht laut landeshauptmann weiters auf den ausschuss der regionen (adr) ein, in dem er, kaiser, als aktives mitglied die interessen kärntens direkt in den eu-gesetzgebungsprozess einbringen könne. "der adr gibt stellungnahmen zu allen eu-gesetzgebungsakten ab, die regionalen bezug haben - das betrifft rund zwei drittel der gesetzgebung", macht er die bedeutung deutlich. kärnten vernetze sich aber insgesamt sehr stark und aktiv in brüssel. kaiser streicht das eu-verbindungsbüro des landes (vbb) hervor sowie institutionelle europäische netzwerke wie elisan im sozialpolitischen bereich und errin im bereich forschung und innovation. unser bundesland präsentiere sich zudem bei den verschiedensten veranstaltungen in brüssel, insbesondere auf der jährlichen european week of regions and cities (ewrc) mit ca. 6.000 teilnehmenden. "wichtig ist es uns natürlich auch, das europabewusstsein in kärnten zu stärken", erklärte kaiser. dies erfolge über laufende informationen der eu-koordinationsstelle, veranstaltungen in zusammenarbeit mit institutionen wie insbesondere dem verein europahaus klagenfurt und das neue "europedirect"-informationszentrum (edic). "wir wollen natürlich auch speziell die jugend mit unseren initiativen erreichen, das geht von projekten in schulen bis hin zum schnuppern von eu-luft in brüssel im rahmen eines praktikums im verbindungsbüro", hält kaiser fest. & 493 & high & High & Socio-Economic & NA & NA & 2018-11-02 & 2018 & 3 & ECO
Frame & high-very high & National & <500 & -0.7948903 & -0.1706634 & 1.0035124 & 0.5015415 & -0.3119516 & 9.0 & 0.2648514 & -0.9910023 & Payer & Domestic & European & Mixed & Domestic|ECO & Positive\\
France & http://www.lepoint.fr/economie/budget-de-l-ue-coups-de-rabot-en-vue-apres-le-brexit-02-05-2018-2215076\_28.php & 110 & Le Point & Private/Non-Public & Online and Offline & National & medium = CP is important part of story & Financial burden & Factual & EU + National & No myth & Institutional bargaining over funding & Factual & National + Other country & No myth & NA & NA & NA & NA & France & ue: bruxelles veut un budget post-brexit en hausse, mais avec des coupes & 2018-05-02 & politique de cohésion & la commission européenne a plaidé mercredi pour un budget de l'ue en hausse après le brexit, mais avec des coupes promettant de vives controverses dans deux secteurs emblématiques, l'agriculture et la cohésion en faveur des régions les plus modestes. après des mois de préparation, l'exécutif européen a également mis sur la table la création d'un lien inédit entre le versement de fonds européens et le respect de l'etat de droit, qui ne devrait pas manquer de braquer des pays se sentant visés, comme la pologne et la hongrie. la proposition, qui devra être négociée entre etats membres et le parlement européen, fixe à 1.279 milliards d'euros le budget pour la période 2021-2027 (contre 1.087 mds pour 2014-2020 en prix courants), en hausse malgré la perte prévue de l'importante contribution britannique. "c'est un budget ambitieux mais équilibré, juste pour tous", a défendu le président de la commission européeenne, jean-claude juncker, qui a présenté sa proposition de "cadre financier pluriannuel" devant les eurodéputés réunis à bruxelles. le cocktail d'économies et de nouvelles ressources demandées vise selon bruxelles à donner à l'union les moyens des ambitions affichées pour sa nouvelle vie à 27, sans le royaume-uni, dont le départ prévu fin mars 2019 rend l'équation budgétaire plus complexe que jamais. selon m. juncker, le départ britannique va laisser un "trou de 15 milliards d'euros" par an dans les finances européennes après 2020 - dernière année de contribution de londres malgré un brexit programmé au printemps de l'année précédente. et la rupture avec ce "contributeur net" tombe d'autant plus mal que l'union européenne cherche à financer à 27 de nouvelles politiques, en matière de défense ou de migration notamment, sans renoncer aux "anciennes". parmi les mesures les plus difficiles à faire passer dans les capitales, bruxelles réclame "une réduction modérée" du financement de la politique agricole commune (pac), chère à la france, et de la politique de cohésion, dont les pays de l'est sont les grands bénéficiaires, "de 5 \% environ dans les deux cas". ceux domaines politiques emblématiques représentent actuellement respectivement 37 \% et 35 \% du budget de l'ue. pour paris, "il est inacceptable que dans un budget en expansion il y ait des coupes si importantes dans les aides directes" aux agriculteurs, a réagi sans tarder une source diplomatique. parmi les pays de l'est, opposés à la baisse de la politique de cohésion, certains comme la pologne et la hongrie sont d'autant plus sur la défensive qu'ils se sentent visées par la proposition inédite de la commission de lier versement de fonds européens et respect de l'etat de droit. ce "nouveau mécanisme qui permettra de protéger le budget en fonction des risques liés aux déficiences de l'etat de droit", a expliqué jean-claude juncker, assurant qu'"il ne visait pas des etats membres en particulier". plusieurs pays le réclamaient pour tirer les leçons du bras de fer infructueux entre bruxelles et le gouvernement ultra-conservateur polonais, accusé de menacer l'indépendance de sa justice. face à la lourdeur de la procédure en cours lancée par la commission, l'idée est de pouvoir recourir à la pression financière dans des cas comparables. "nous n'accepterons pas de mécanismes arbitraires qui feront de la gestion des fonds un instrument de pression politique à la demande", avait averti récemment le vice-ministre polonais pour les affaires européennes, konrad szymanski. des pays comme l'autriche ou les pays-bas sont déjà mobilisés pour leur part contre une hausse des contributions nationales, à laquelle l'allemagne et la france sont en revanche disposées. la commission plaide pour la nécessité de fonds plus importants pour le numérique, la recherche, le programme erasmus+, la défense ou encorte la protection des frontières extérieures. elle a aussi proposé de nouvelles ressources propres pour l'ue, en demandant qu'une partie des revenus de la taxation des échanges de quotas de carbone soit à l'avenir orientée vers le budget européen. elle a aussi mis sur la table la création d'une nouvelle taxe sur les déchets plastiques non recyclés. la commission veut que les tractations entre etats membres et eurodéputés soient bouclées avant les prochaines élections européennes, soit moins de deux mois après le divorce avec les britanniques. "ce genre de négociations prend normalement deux ans", souligne une source diplomatique, perplexe face à ce calendrier. & 739 & medium & Medium & Values & Power & NA & 2018-05-02 & 2018 & 3 & ECO
Frame & low-medium & National & 500-1000 & -0.7708104 & -0.8786862 & 0.6041414 & -0.2983228 & 0.6486612 & 0.0 & -0.0275917 & -0.0818024 & Payer & Domestic & European & Mixed & Domestic|ECO & Neutral\\
\addlinespace
France & http://www.midilibre.fr/2015/10/30/avignon-a-agroparc-un-nouveau-batiment-pour-les-recherches-horticoles-de-l-inra\%2C1234484.php\#xtor\%3DRSS-5 & 120 & Lindependant.fr & Private/Non-Public & Online and Offline & Regional/Local & very low = CP mentioned once & Research \& innovation & Positive & EU + Subnational & No myth & NA & NA & NA & NA & NA & NA & NA & NA & France & avignon: to agroparc, a new building for research horticoles the inra & 2015-10-30 & fonds européen de développement régional & jean-marc roubaud, le président de l'agglomération du grand avignon a participé, ce vendredi 30 octobre, en milieu d'après-midi, aux côtés de bernard gonzalès, le préfet de vaucluse, de maurice chabert, le président du conseil départemental de vaucluse et de christine lagrange, conseillère communautaire, à l'inauguration d'un nouveau bâtiment du pôle de recherches production horticole intégrée de l' institut national de la recherche agronomique (inra) d'avignon. pour le président de l'agglo, "la création d'un pôle de recherche production horticole intégrée, au coeur du technopôle agroparc que le grand avignon aménage, et qui abrite déjà le pôle de compétitivité terralia, la faculté des sciences de l'université d'avignon et bien d'autres acteurs des filières du végétal ne peut que renforcer la cohérence du dispositif de recherche-développement autour de la filière agroalimentaire sur notre territoire, et plus généralement dans le grand sud-est". jean-marc roubaud a aussi, dans la foulée, souligné "la nécessité de soutenir ce type d'action est, pour les élus du grand avignon, l'expression d'une conviction ancienne et profonde : faire du grand avignon un véritable territoire d'innovation, et d'agroparc un "campus de recherche", un technopôle où les laboratoires et les entreprises, où tous les acteurs de la filière agroalimentaire travaillent main dans la main pour garantir l'avenir de notre industrie et de notre agriculture". et l'inauguration de ce vendredi n'est qu'une étape "puisqu'en complément, l'effort du grand avignon dans le domaine du soutien à la recherche se poursuivra en 2016 puisque la collectivité va soutenir le projet 3a (agroparc : la science au service des filières agricoles et apicoles) porté par l'inra et l'université à hauteur de 500 000 euros", a indiqué le président de l'agglo du grand avignon. cette réalisation est la concrétisation d'un projet financé dans le cadre du contrat de projet état-région 2007-2013 et du fonds européen de développement régional de l'union européenne. l'objectif du projet financé est de fédérer des recherches existantes ou émergeantes de deux unités de recherche "sécurité et qualité des produits d'origine végétale" et "plantes et systèmes de culture horticoles", en des objectifs partagés de recherche au sein du pôle production horticole intégrée, en synergie avec d'autres unités de recherche basées à avignon, sur le domaine saint-maurice, à gotheron et à sophia-antipolis. les objectifs de recherche portent sur la qualité des fruits et légumes consommés et intègrent l'ensemble des facteurs génétique, environnementaux, des pratiques culturales et des techniques de transformation qui impactent l'élaboration de la qualité des fruits et légumes et son maintien après récolte et transformation. l' inauguration a concerné un nouveau bâtiment, relié à trois anciens bâtiments qui ont été rénovés, l'aménagement de bureaux et de laboratoires ainsi que la destruction d'une ancienne halle, sur le domaine inra de saint-paul. & 488 & very low & Low & Socio-Economic & NA & NA & 2015-10-30 & 2015 & 1 & ECO
Frame & v.low & Regional & <500 & -0.7708104 & -0.8786862 & 0.6041414 & -0.2983228 & 0.6486612 & 0.0 & -0.0275917 & -0.0818024 & Payer & Domestic & European & Mixed & Domestic|ECO & Positive\\
France & http://www.ladepeche.fr/article/2017/04/20/2559524-sensibiliser-a-l-egalite-professionnelle.html & 145 & Ladepeche.fr & Private/Non-Public & Online and Offline & Regional/Local & very low = CP mentioned once & Economic development & Positive & EU & No myth & NA & NA & NA & NA & NA & NA & NA & NA & France & sensibiliser à l'égalité professionnelle & 2017-04-20 & fonds social européen & depuis 2015, le cbe du net porte un projet d'ingénierie sur l'égalité professionnelle auprès des territoires, des acteurs économiques et des collectivités locales. différents groupes de travail ont été mis en place avec les acteurs socio-économiques du territoire autour de cette thématique. ce projet est co-financé par l'europe (fonds social européen), l'etat, la sénatrice françois laborde, la région occitanie et les collectivités adhérentes au cbe du net. dernièrement s'est déroulée une réunion de culture commune autour du thème "réglementation de l'égalité professionnelle dans le processus de recrutement". cette réunion, organisée par le cbe du net, en partenariat avec la mcef de saint-jean a réuni autour de la table des professionnels de l'emploi et de la formation pour échanger sur cette thématique. mais également pour les former et leur donner des outils pour conseiller et accompagner les employeurs lors du recrutement. les différents sujets abordés ont été l'offre d'emploi, le "sourcing", l'entretien d'embauche, le "process" d'évaluation jusqu'à l'intégration du salarié. à l'issue de cette réunion, un groupe de travail coordonné par le cbe du net a émergé avec pour objectif la co-construction d'un outil pour les professionnels de l'emploi et de la formation pour accompagner les employeurs dans leur prise en compte des politiques d'égalité professionnelle dans le recrutement. il est possible de faire partie de ce groupe ou de s'informer auprès de sophie lopez ou aurélie buhagiar au 05 62 89 07 70 ou à contact@cbedunet.org. & 261 & very low & Low & Socio-Economic & NA & NA & 2017-04-20 & 2017 & 2 & ECO
Frame & v.low & Regional & <500 & -0.7708104 & -0.8786862 & 0.6041414 & -0.2983228 & 0.6486612 & 0.0 & -0.0275917 & -0.0818024 & Payer & European & European & European & European|ECO & Positive\\
France & http://www.liberation.fr/planete/2018/06/01/la-pac-perd-sa-place-de-premier-budget-de-l-union-europeenne\_1655884 & 177 & Libération & Private/Non-Public & Online and Offline & National & very low = CP mentioned once & Institutional bargaining over funding & Factual & EU & No myth & NA & NA & NA & NA & NA & NA & NA & NA & France & la pac perd sa place de premier budget de l'union européenne & 2018-06-01 & fonds structurels & si la proposition de "cadre financier pluriannuel" pour 2021-2027 est adoptée par les vingt-sept etats membres, la politique agricole commune, qui représente aujourd'hui plus de 43\% des dépenses européennes, passera à 30\%, sa première diminution dans l'histoire de l'ue. pour la première fois dans l'histoire de l'union, le budget de la politique agricole commune (pac) va diminuer. et fortement diminuer, au point, tout un symbole, de perdre sa place de premier budget de l'union au profit de la politique régionale (fonds structurels). si la proposition de "cadre financier pluriannuel" (cfp) pour la période 2021-2027 présentée le 2 mai par la commission est adoptée en l'état par les vingt-sept etats membres, la pac ne représentera plus que 30\% des dépenses européennes, contre plus de 43\% aujourd'hui. pour ne rien arranger, cette pénurie de moyens va s'accompagner d'une large renationalisation de ce qui est non seulement la plus vieille politique commune de l'union, mais jusque-là la plus intégrée. en effet, la énième réforme présentée vendredi par le commissaire à l'agriculture, l'irlandais phil hogan, prévoit que les etats seront désormais libres de gérer comme ils l'entendent les fonds qu'ils recevront du budget européen à condition qu'ils s'engagent à respecter neuf objectifs définis au niveau européen. autrement dit, on passe d'une politique commune fondée sur une vision européenne du modèle agricole à des programmes nationaux de gestion de fonds européens contrôlés (tant bien que mal) par bruxelles... autant dire que cette réforme marque la fin d'une époque, celle où la vie des agriculteurs était rythmée par les décisions prises à bruxelles. a lire aussi union européenne : un budget sans ambition et la pac aux tisons ironiquement, la commission présidée par jean-claude juncker profite donc du brexit et du trou budgétaire laissé par son départ pour commencer à réaliser le rêve britannique d'un démantèlement de la pac. ce n'est pas un hasard si elle a tenté de brouiller les pistes sur l'ampleur exacte des coupes : en jouant sur les euros constants (hors inflation) et courants (inflation comprise), elle a annoncé le 2 mai une diminution du budget de la pac d'environ 5\%. le parlement européen, dans une résolution adoptée mercredi, juge, lui, que le chiffre réel est de 15\%. de fait, la commission, qui a renoncé à proposer une augmentation du budget européen pour combler le trou de plus de 10 milliards d'euros annuel laissé par le brexit, a choisi de faire principalement porter l'effort sur la pac, mais aussi sur les aides régionales (-10\%) afin de pouvoir financer - un peu - de nouvelles politiques (contrôle des frontières, défense, etc.). résultat : en france, la baisse des aides au revenu sera de l'ordre de 3,9\% (hors inflation, soit 9,5\% en terme réel sur la période), alors qu'elle atteindra plus de 25\% au danemark ou 13\% en république tchèque, des pays où la part des paiements directs est plus importante. globalement, sur la période 2021-2027, le budget de la pac en euros constants passera à 365 milliards d'euros à vingt-sept contre 408 milliards à vingt-huit entre 2014 et 2020. soit 43 milliards de moins, ce qui représente une année d'aide aux revenus, ce qui donne une idée de l'ampleur de la saignée (dont une partie est imputable au brexit). point particulièrement positif, phil hogan propose d'introduire plus d'équité dans la répartition des aides, à la fois en réduisant les écarts qui existent encore entre les pays de l'est et ceux de l'ouest, et surtout en plafonnant le montant annuel des aides à 100 000 euros (soit pour les exploitations de plus de 200 à 250 hectares) avec une dégressivité à partir de 60 000 euros. les plus touchés seront l'allemagne, la tchéquie, la hongrie, la roumanie et la bulgarie pays des exploitations géantes héritées des sovkhozes soviétiques. l'argent ainsi économisé ne retournera pas au niveau communautaire : il restera dans l'enveloppe nationale. car c'est là l'essentiel, chaque etat aura droit à une enveloppe globale intouchable qu'il gérera dans le cadre de neuf objectifs plutôt généraux assortis d'indicateurs plus précis : garantir un revenu agricole décent tout en accroissant la sécurité alimentaire, assurer une meilleure compétitivité, assurer la place de l'agriculture dans la chaîne alimentaire, etc. même si les programmes nationaux devront être approuvés par la commission, la marge de manœuvre laissée aux etats est particulièrement large. "cette nouvelle gouvernance est rendue nécessaire par l'élargissement : à vingt-sept, on ne peut plus définir à 100\% la politique agricole de bruxelles", se justifie-t-on à la commission. reste qu'on se demande comment la commission vérifiera que les pays respectent bien leur programme, vu leur propension à mentir, sauf à engager une armée d'auditeurs. en clair, la pac risque bien de se réduire rapidement à la signature de chèques aux etats, ce qui posera, à terme, la question du maintien de ce qui ne sera plus qu'une gigantesque caisse de compensation. & 856 & very low & Low & Power & NA & NA & 2018-06-01 & 2018 & 3 & POL
Frame & v.low & National & 500-1000 & -0.7708104 & -0.8786862 & 0.6041414 & -0.2983228 & 0.6486612 & 0.0 & -0.0275917 & -0.0818024 & Payer & European & European & European & European|POL & Neutral\\
France & http://www.france24.com/fr/20170825-macron-acheve-tournee-a-lest-renforcant-pression-pologne & 157 & France 24 & Public & Online only & National & very low = CP mentioned once & Solidarity to poor countries/regions & Factual & Other country & No myth & NA & NA & NA & NA & NA & NA & NA & NA & France & macron achève sa tournée à l'est en renforçant la pression sur la pologne & 2017-08-25 & fonds structurels & le président français emmanuel macron a conclu vendredi par une vive attaque contre la pologne une mini-tournée en europe de l'est au cours de laquelle il a convaincu plusieurs pays d'accepter le principe d'une réforme de la directive sur le travail détaché. varsovie se met "en marge de l'ue", a taclé le chef d'etat français depuis la bulgarie, après que la première ministre souverainiste polonaise beata szydlo a rappelé que son pays, large exportateur de travailleurs, défendrait le statu quo "jusqu'au bout". "le peuple polonais mérite mieux que cela et la première ministre aura beaucoup de mal à expliquer qu'il est bon de mal payer les polonais", a estimé m. macron à varna, dernière étape d'une tournée de trois jours. varsovie "décide d'aller à l'encontre des intérêts européens sur de nombreux sujets". or "l'europe s'est construite pour créer de la convergence, c'est le sens même des fonds structurels que touche la pologne", a-t-il souligné. mme szydlo a en retour taxé le jeune président français d'"arrongance", évoquant son "manque d'expérience". paris souhaite limiter à un an le travail détaché, qui permet aux entreprises des pays à coût moindre d'envoyer temporairement des travailleurs dans les pays riches, au risque de favoriser le dumping social. la france entend aussi renforcer les contrôles. soutenu par berlin et vienne, le président français a rencontré les dirigeants tchèques, slovaques, roumains et bulgares, qui tous se sont dit d'accord pour discuter d'une réforme de la directive européenne régissant cette pratique. "tous les pays savent qu?il y a un problème et qu?il faut changer les choses", a résumé vendredi le premier ministre bulgare, boïko borissov, à varna. même sans engagement chiffré, cette ouverture constitue une avancée diplomatique puisque ces etats avaient au printemps dernier refusé tout amendement du texte. - 'accord possible' - m. borissov, dont le pays assurera la présidence tournante de l'ue en janvier et qui doit recevoir la dirigeante polonaise mi-septembre, a cependant mis en garde contre toute tentation de passage en force face aux réticences de varsovie et budapest. "la pologne et la hongrie sont nos amis, une part importante de l'ue. il serait fatal que les relations entre les pays de l'ue passent dans une phase de confrontation ouverte", a-t-il déclaré. la france, qui avait fait bloquer un premier accord jugé trop laxiste en mai, en espère un dès octobre lors d'un sommet social à bruxelles. "je retire de ce déplacement la conviction que d?ici la fin de l?année un accord à majorité qualifiée est possible, selon les termes ambitieux mis sur la table par la france, ce sera incomparable avec situation actuelle et bien meilleur que l'accord initialement prévu", a confié m. macron vendredi. le commissaire européen aux affaires économiques, pierre moscovici, a jugé nécessaire de trouver "un compromis" dans ce dossier, rappelant que "la liberté de circulation, qui est une des grandes libertés de l'europe, doit aller de pair avec le combat contre le +dumping social+". en échange, emmanuel macron a accepté de séparer la discussion sur les transporteurs routiers et assuré la bulgarie et la roumanie de son soutien pour une entrée à terme dans l'espace schengen, mais une fois celui-ci réformé. il a aussi promis de soutenir l'entrée de la bulgarie dans "l'antichambre de la zone euro". il a par ailleurs plaidé devant tous les dirigeants en faveur de son projet de "refonder l'europe" autour d'un coeur de pays les plus décidés à accepter un convergence dans tous les domaines, et expliqué qu'il proposerait en fin d?année une "feuille de route" pour l'ue dans les cinq à dix ans. "si nous ne proposons pas une perspective européenne aux balkans occidentaux, ce sont eux qui proposeront une perspective balkanique à l'europe", a pour sa part prévenu le président bulgare, roumen radev. "ce n'est ni le président de la france ni aucun autre dirigeant qui décidera personnellement de l'avenir de l'europe mais l'ensemble des membres de la communauté", a souligné de son côté mme szydlo. "l'important est de ne pas créer de clivage en europe et d'avancer", a convenu m. macron. & 718 & very low & Low & Values & NA & NA & 2017-08-25 & 2017 & 2 & ECO
Frame & v.low & National & 500-1000 & -0.7708104 & -0.8786862 & 0.6041414 & -0.2983228 & 0.6486612 & 0.0 & -0.0275917 & -0.0818024 & Payer & European & European & European & European|ECO & Neutral\\
France & http://www.lavoixdunord.fr/501218/article/2018-12-05/certains-eleves-reviennent-de-loin & 187 & La Voix du Nord & Private/Non-Public & Online and Offline & Regional/Local & low = CP mentioned more times but NOT important part of story (mainly about others issues) & Social awareness/inclusion & Positive & Subnational & No myth & NA & NA & NA & NA & NA & NA & NA & NA & France & certains élèves (re)viennent de loin & 2018-12-05 & fonds social européen & les bâtiments d'apprentis d'auteuil abritent le siège régional de la fondation et le collège. et, au sein du collège, en plus des sections d'enseignement général, un dispositif de raccrochage et de préparation à l'apprentissage (drpa). cette unité spécialisée, cofinancée par l'union européenne avec le fonds social européen (fse) et dans le cadre de l'initiative emploi jeunes, accueille les grands adolescents et jeunes adultes (plus de 15 ans) sortis du système éducatif sans qualification, ni stage... & 81 & low & Low & Socio-Economic & NA & NA & 2018-12-05 & 2018 & 3 & ECO
Frame & low-medium & Regional & <500 & -0.7708104 & -0.8786862 & 0.6041414 & -0.2983228 & 0.6486612 & 0.0 & -0.0275917 & -0.0818024 & Payer & Domestic & Domestic & Domestic & Domestic|ECO & Positive\\
\addlinespace
France & http://www.lemonde.fr/en-bref/article/2018/05/02/budget-europeen-eta-pollution-de-l-air-le-resume-de-l-actualite-a-19-heures\_5293511\_4597295.html & 112 & Le Monde.fr & Private/Non-Public & Online and Offline & National & low = CP mentioned more times but NOT important part of story (mainly about others issues) & Financial burden & Factual & EU + National & No myth & Institutional bargaining over funding & Factual & EU + National & No myth & NA & NA & NA & NA & France & budget européen, eta, pollution de l'air..., le résumé de l'actualité à 19 heures & 2018-05-02 & fonds social européen & le projet de budget européen, la dissolution d'eta, les décès liés à la pollution de l'air... le point sur l'actualité du mercredi 2 mai en fin de journée. ce qu'il faut retenir des propositions explosives de la commission européenne sur le budget la commission propose de conditionner tous les fonds du budget 2021-2027 au respect par les gouvernements de l'etat de droit. les fonds allant à un pays pourraient être suspendus si son système judiciaire national s'avère défaillant. dans le viseur de bruxelles : la pologne, la hongrie, malte ou la roumanie.autre coup porté par bruxelles aux gouvernements s'éloignant des valeurs communes : une partie du fonds social européen (fse), composante des fonds structurels (environ 100 milliards pour le budget 2014-2020), ira à l'intégration de migrants.la commission européenne a également plaidé pour une baisse " d'environ 5 \% " des fonds alloués à la politique agricole commune et à la politique de cohésion de l'ue, consacrée aux régions les plus pauvres. la proposition sur la pac a été jugée " inacceptable " par le gouvernement français. dans le reste de l'actualité terrorisme. eta annonce avoir " complètement dissous toutes ses structures ". après des décennies d'attentats meurtriers, l'organisation séparatiste basque dit " mettre fin à son cycle historique et à sa fonction ", dans une lettre publiée par un journal espagnol. grève. la direction de la sncf a annoncé qu'un tgv et un transilien sur deux, et deux ter sur cinq, circuleraient jeudi, au septième épisode de grève. il y aura également un train intercités sur trois, tandis qu'à l'international trois trains sur cinq sont prévus. de son côté, air france prévoit 85 \% des vols. santé. la pollution de l'air tue 7 millions de personnes par an dans le monde, alerte l'organisation mondiale de la santé. neuf habitants sur dix respirent quotidiennement un air trop chargé en particules fines. " il y a deux cents millions de personnes sur facebook qui se définissent comme célibataires, donc, à l'évidence, il y a quelque chose à faire. " mark zuckerberg, le patron de facebook, a annoncé mardi l'intégration prochaine d'une plate-forme de rencontres dans son réseau social pour bâtir des " relations durables, pas seulement des plans d'un soir ". ce qu'il ne fallait pas manquer les déconvenues s'accumulent pour vincent bolloré. jugé arrogant et critiqué pour son mode de gouvernance, l'industriel, à la tête d'un empire dans les médias et les transports, est également mis en examen pour corruption en afrique. le 4 mai, l'homme d'affaires breton devra en outre faire face à l'offensive du fonds vautour elliott pour prendre le contrôle de telecom italia. le monde explique pourquoi le système bolloré a fini par se gripper. & 465 & low & Low & Values & Power & NA & 2018-05-02 & 2018 & 3 & ECO
Frame & low-medium & National & <500 & -0.7708104 & -0.8786862 & 0.6041414 & -0.2983228 & 0.6486612 & 0.0 & -0.0275917 & -0.0818024 & Payer & Domestic & European & Mixed & Domestic|ECO & Neutral\\
France & http://www.lefigaro.fr/flash-eco/2018/02/15/97002-20180215FILWWW00199-ouragans-bruxelles-propose-49-m-eur-pour-saint-martin-et-la-guadeloupe.php & 162 & Le Figaro.fr & Private/Non-Public & Online and Offline & National & high = CP is most important issue in story (can also cover other issues) & Solidarity to poor countries/regions & Positive & EU & No myth & NA & NA & NA & NA & NA & NA & NA & NA & France & ouragans : bruxelles propose 49 milliards pour saint-martin et la guadeloupe & 2018-02-15 & politique régionale & la commission européenne a proposé jeudi de débloquer 49 millions d'euros pour saint-martin et la guadeloupe, deux territoires français durement touchés par les ouragans irma et maria en septembre 2017, au titre du fonds de solidarité de l'ue. "les fonds contribueront au relèvement de ces deux régions", frappées "par des ouragans dévastateurs", affirme bruxelles dans un communiqué. le versement de ces 49 millions d'euros, qui prennent en compte une avance de 5 millions d'euros déjà versée en décembre, reste conditionné à l'approbation du parlement européen et du conseil, qui représente les etats membres. la commissaire européenne à la politique régionale, la roumaine corina cretu, espère que le paiement pourra être "effectif en mai". les dégâts provoqués en septembre par les ouragans irma et maria à saint-martin et en guadeloupe, îles franco-néerlandaise et française des antilles, ont été estimés à près de deux milliards d'euros par le gouvernement français d'edouard philippe. fin novembre, la ministre des outre-mer annick girardin s'était rendue à bruxelles pour demander à la commission une demande d'activation du fonds de solidarité de l'union européenne (fsue). ce fonds a pour but de faire face aux grandes catastrophes naturelles et d'exprimer la solidarité de l'union européenne à l'égard des régions sinistrées. la commission a également proposé jeudi de verser 50,6 millions d'euros au portugal, dont le nord et le centre ont été dévastés en 2017 par de violents incendies, "coûtant la vie à de nombreuses personnes et submergeant par leur ampleur les capacités des services de sauvetage". egalement touchée, l'espagne devrait bénéficier de 3,2 millions d'euros au titre de ce fonds. enfin, bruxelles propose de débloquer 1,3 million d'euros pour la grèce, dont l'île de lesbos a été touchée par un tremblement de terre en juin dernier. & 313 & high & High & Values & NA & NA & 2018-02-15 & 2018 & 3 & ECO
Frame & high-very high & National & <500 & -0.7708104 & -0.8786862 & 0.6041414 & -0.2983228 & 0.6486612 & 0.0 & -0.0275917 & -0.0818024 & Payer & European & European & European & European|ECO & Positive\\
France & http://www.lemonde.fr/europe/article/2017/09/26/un-budget-de-l-eurozone-pour-quoi-faire\_5191374\_3214.html & 115 & Le Monde.fr & Private/Non-Public & Online and Offline & National & very low = CP mentioned once & Solidarity to poor countries/regions & Factual & EU + National & No myth & NA & NA & NA & NA & NA & NA & NA & NA & France & un budget de l'eurozone : pour quoi faire ? & 2017-09-26 & fonds de cohésion & klaus regling est un homme discret, inconnu du public français. a bruxelles, néanmoins, la voix de cet allemand compte. depuis quelques mois déjà, le directeur général du mécanisme européen de stabilité (mes) développe publiquement des idées plutôt novatrices concernant un éventuel budget de la zone euro. que préconise cet ancien patron de la très puissante direction générale à l'économie de la commission ? comme jean-claude juncker, le président de l'institution communautaire, il n'est pas en faveur d'un budget spécifique, très élevé, pour la zone euro. " le budget existant de l'union européenne permet déjà suffisamment de transferts financiers des etats les plus riches vers les plus pauvres [dans le cadre des fonds structurels, dont les fonds de cohésion] ", a t-il déclaré fin septembre devant les étudiants de sciences po paris, citant aussi le fonds d'investissement juncker, qui, " au besoin ", pourrait être augmenté. pourtant, m. regling fait aussi des ouvertures substantielles : " il y a une discussion sur un budget limité pour la zone euro. nous avons vraiment besoin d'un mécanisme permettant de faire face aux chocs asymétriques. un pays victime d'un choc asymétrique [tremblement de terre, incendies massifs, etc.] recevrait de l'argent pendant une crise, mais aurait l'obligation de rembourser une fois le choc passé. " un fonds pour les mauvais jours s'inspirant de dispositifs existant aux etats-unis, m. regling promeut une sorte de " rainy day fund " (fonds pour les mauvais jours), qui pourrait peser de " 1 \% à 2 \% de l'économie de l'eurozone, soit entre 100 et 200 milliards d'euros ". il va plus loin, suggérant d'y adjoindre une assurance-chômage " complémentaire " européenne. l'idée est de soutenir temporairement le système d'assurance-chômage d'un pays, afin d'éviter que celui-ci, du fait de raisons exogènes (comme une crise migratoire), soit contraint de prendre... & 309 & very low & Low & Values & NA & NA & 2017-09-26 & 2017 & 2 & ECO
Frame & v.low & National & <500 & -0.7708104 & -0.8786862 & 0.6041414 & -0.2983228 & 0.6486612 & 0.0 & -0.0275917 & -0.0818024 & Payer & Domestic & European & Mixed & Domestic|ECO & Neutral\\
France & http://www.lepoint.fr/culture/chris-un-createur-de-mode-qui-sublime-les-differences-30-03-2015-1917085\_3.php & 125 & Le Point & Private/Non-Public & Online and Offline & National & very low = CP mentioned once & Social justice & Positive & EU & No myth & NA & NA & NA & NA & NA & NA & NA & NA & France & chris, un créateur de mode qui sublime les différences & 2015-03-30 & fonds social européen & pour le jeune styliste chris ambraisse boston, tout a commencé dans le métro, quand il esquissait un modèle: "c'est classe ! nous aussi, on aimerait porter de beaux vêtements", lui a lancé une passagère handicapée. son idée d'une mode pour tous était née. "cette remarque m'a fait cogiter pendant des mois. j'ai rencontré des médecins, des associations. je ne voulais pas faire des +vêtements pour handicapés+, ne pas les stigmatiser encore", explique à l'afp cet élégant jeune homme de 32 ans d'origine antillaise, voix douce mais décidée, dans son atelier et show room à la devanture rose de la rue de l'ourcq à paris. "alors, j'ai décidé d'inventer des vêtements esthétiques, innovants et fonctionnels qui puissent être portés par les personnes valides comme par celles en situation de handicap". le pari était osé: "j'étais très jeune. je suis valide. je suis black. certains m'agressaient: +tu veux te faire de l'argent sur le dos des handicapés+". le parcours du combattant a débuté pour convaincre de la pertinence du concept, trouver des aides.... "après beaucoup de refus, le fonds social européen a été mon premier financeur", raconte chris ambraisse boston, également président de l'association mode et handicap, qui a créé sa première collection en 2009 sous sa marque de prêt-à-porter a\&k classics. si chris ne savait rien au départ du handicap, le rejet et la souffrance, il connaît. placé à la ddass, victime, avec son frère, de maltraitance de la part de ses parents, il en a bavé. "après une thérapie, la seule séquelle qui me reste, c'est la dyslexie ! mais je n'ai jamais voulu pleurer sur mon sort et je me suis battu pour faire financer mes études de stylisme". enfiler une veste, une jupe ou un pantalon, c'est pour la majorité d'entre nous banal. pour d'autres, s'habiller seul peut s'avérer très difficile, voire impossible. -une mode "aimantée"- pour les personnes en fauteuil ou souffrant d'autres handicaps, chris crée des modèles transformables, à la fois originaux, fonctionnels et beaux, en jouant sur des ouvertures placées à des endroits stratégiques. "zips, velcros, aimants... on fait presque de la haute-couture sans couture !", sourit-il. ici une longue cape imperméable, qui protège les jambes quand on est dans un fauteuil (plus courte, grâce à un zip, elle sert aussi aux cyclistes). là, une chemise avec pressions et zips sur toute la longueur des manches, un pull pour homme avec fermetures croisées aimantées ou un blouson dont tout le dos s'enlève pour les personnes portant des coques dorsales... les tissus sont de qualité, les modèles fabriqués à l'atelier. chris a notamment un partenariat avec lvmh qui lui fournit la matière première. "la france est très en retard dans ce domaine", déplore ce pionnier. il se réjouit en revanche que la haute-couture commence à ouvrir la mode aux handicapés, comme à la récente fashion week de tokyo. "je crée au moins une collection par an, en innovant à chaque fois, et organise de nombreux défilés, avec des valides et des personnes handicapées", précise le créateur dont les modèles sont visibles sur le site www.aandkclassics.fr. un défilé a\&k classics se déroule ainsi vendredi, à la mairie du 16e, dans le cadre des journées européennes des métiers d'art. "la vente des modèles se fait rue de l'ourcq, mais aussi au canada, grâce à une récompense qui m'y a fait connaître. bientôt, on pourra acheter mes vêtements sur internet", ajoute-t-il, "très fier" d'avoir déjà reçu douze prix pour son travail. ses créations coûtent de 30 à 300 euros en moyenne. "on arrive à s'en sortir grâce à des aides". ce "fashion angel" emploie et forme aussi des jeunes en insertion sociale. ainsi, son modéliste est un réfugié politique afghan, arrivé en france à 16 ans. "lui, dit chris, il a connu la guerre, des atrocités, c'est pire que ce que j'ai vécu". 30/03/2015 16:34:48 - paris (afp) - par myriam chaplain riou - © 2015 afp & 686 & very low & Low & Socio-Economic & NA & NA & 2015-03-30 & 2015 & 1 & ECO
Frame & v.low & National & 500-1000 & -0.7708104 & -0.8786862 & 0.6041414 & -0.2983228 & 0.6486612 & 0.0 & -0.0275917 & -0.0818024 & Payer & European & European & European & European|ECO & Positive\\
France & http://fr.euronews.com/2018/02/14/propositions-de-reformes-institutionnelles-de-la-commission-europeenne?utm\_source=feedburner\&utm\_medium=feed\&utm\_campaign=Feed\%253A\%2Beuronews\%252Ffr\%252Fhome\%2B\%2528euronews\%2B-\%2Bhome\%2B-\%2Bfr\%2529\&utm\_content=FeedBurner & 163 & euronewsfr & Private/Non-Public & Online only & National & very low = CP mentioned once & Financial burden & Factual & EU & No myth & NA & NA & NA & NA & NA & NA & NA & NA & France & propositions de réformes institutionnelles de la commission européenne & 2018-02-16 & politique de cohésion & le président de la commission européenne a présenté ses propositions pour renforcer les liens entre les citoyens et les institutions. jean-claude juncker est "en faveur du système des têtes de liste pour les élections de 2019". ce principe, surnommé aussi "spitzenkandidat" prévoit que le candidat dont la liste aura recueilli le plus de voix devienne président de la commission. cette méthode avait entrainé en 2014 la désignation de jean-claude juncker à la tête de l'institution. pour le prochain scrutin, le luxembourgeois propose d'ailleurs de présenter les têtes de liste dans les prochains mois pour être identifiés au plus tôt par les électeurs. pour améliorer l'efficacité politique des institutions, jean-claude juncker propose aussi de fusionner le poste de président du conseil européen, qui représente les chefs d'etats et de gouvernements, et avec celui de président la commission européenne. cette démarche permettrait d'empêcher les risques de conflit entre les deux responsables. mais jean-claude juncker reconnaît que cette idée ne pourra être applicable dans le cadre du scrutin en 2019. questionné en fin de conférence de presse sur les négociations concernant le prochain cadre budgétaire européen, le président de la commission reconnaît que des efforts seront nécessaires. "il faudra faire des coupes dans la politique agricole commune et dans la politique de cohésion si nous voulons atteindre nos nouvelles priorités. mais je ne suis pas en faveur de coupes brutales", précise-t-il. & 239 & very low & Low & Values & NA & NA & 2018-02-16 & 2018 & 3 & ECO
Frame & v.low & National & <500 & -0.7708104 & -0.8786862 & 0.6041414 & -0.2983228 & 0.6486612 & 0.0 & -0.0275917 & -0.0818024 & Payer & European & European & European & European|ECO & Neutral\\
\addlinespace
France & http://www.lepoint.fr/europe/le-brexit-met-le-budget-europeen-sens-dessus-dessous-29-06-2017-2139115\_2626.php & 148 & Le Point & Private/Non-Public & Online and Offline & National & very low = CP mentioned once & Financial burden & Factual & EU & No myth & NA & NA & NA & NA & NA & NA & NA & NA & France & le brexit met le budget européen sens dessus dessous & 2017-06-29 & politique de cohésion & le budget européen pourrait crouler sous les impayés après la sortie de la grande-bretagne. c'est ce que la commission européenne a déclaré, alarmiste, ce mercredi, en publiant un document de " réflexion " sur l'avenir du budget européen. elle évalue à près de 20 milliards d'euros par an le trou à combler dans les finances européennes, entre le brexit et les nouvelles dépenses demandées au budget européen. impossible donc de garder le statu quo et de " faire la politique de l'autruche ", a ainsi affirmé le commissaire européen au budget, günther oettinger. avec le divorce de la grande-bretagne, contributeur net de l'ue, ce serait déjà près de dix milliards d'euros par an qui partent en fumée. et ce, alors qu'il faudrait compter dix milliards supplémentaires pour les nouvelles dépenses demandées au budget, à commencer par le fonds européen sur la défense poussé notamment par la france et l'allemagne. le message de la commission est clair. sans réforme du budget européen, il sera impossible d'honorer ces engagements. " l'union européenne ne peut pas s'endetter ", a répété günther oettinger. le commissaire prévoit ainsi des propositions de réformes pour le nouveau cadre pluriannuel européen après 2020, qui doit fixer la contribution de chaque état pour les sept années suivantes... en effet, la commission, qui doit batailler ferme chaque année pour que les états membres paient leur contribution au budget européen, compte bien saisir l'opportunité pour réformer le budget. c'est l'occasion pour l'union de se doter de ressources propres, et de ne plus dépendre de la bonne volonté de ses pays membres. pour l'instant, les fonds au budget européen proviennent en majeure partie de contributions extérieures : des paiements directs des états membres ou encore des recettes liées à la tva. la commission cherche ainsi à remettre au goût du jour les résultats du rapport monti publié en janvier 2017 qui proposait notamment de mettre en place des taxes communes sur l'énergie et l'environnement, de lever des recettes basées sur l'impôt des sociétés, ou encore de créer une tva européenne. autant de propositions reprises par la commission dans son document de réflexion publié ce mercredi. l'institution européenne a ses préférences. mais en dessinant cinq scénarios, du moins révolutionnaire - rien ne change - au plus révolutionnaire - on fait beaucoup plus ensemble -, elle se garde bien de choisir entre tous, et renvoie la balle aux états membres. dans les trois premiers scénarios, qui ne planifient aucune réforme profonde du budget, la commission prévoit des coupes budgétaires drastiques sur les deux grandes politiques communes : la politique agricole commune (la pac), dont bénéficient en grande partie les agriculteurs français, ou encore la politique de cohésion, auxquels les bénéficiaires nets au budget européen, les pays d'europe de l'est en tête, sont particulièrement attachés. la négociation sur ce budget prévue pour l'année prochaine s'annonce rude. en règle générale, si les pays veulent réformer le budget européen jugé trop complexe et trop dispendieux, aucun n'est d'accord sur la manière dont il faut s'y prendre. " la pologne, qui a une économie liée au charbon, est très réticente à l'introduction d'une taxation sur le carbone ", explique ainsi la chercheuse à l'institut jacques delors, eulalia rubio. " la france s'est montrée, elle, très favorable à la mise en place de ressources propres pour le budget européen, mais elle pourrait privilégier une option plutôt qu'une autre ", ajoute la chercheuse. problème : de telles réformes doivent être acceptées à l'unanimité par les états membres. et une fois l'accord trouvé, il faudrait qu'il soit ratifié ensuite par les parlements nationaux des 28. " c'est une procédure lourde et qui pourrait être bloquée au moindre veto d'un état membre ", indique eulalia rubio. des négociations d'autant plus sensibles qu'elles se dérouleront en même temps que celles sur la sortie de la grande-bretagne. un autre scénario pourrait ainsi voir le jour, celui que la commission veut éviter à tout prix autant pour le brexit que pour le budget européen : celui du " no deal ". & 686 & very low & Low & Values & NA & NA & 2017-06-29 & 2017 & 2 & ECO
Frame & v.low & National & 500-1000 & -0.7708104 & -0.8786862 & 0.6041414 & -0.2983228 & 0.6486612 & 0.0 & -0.0275917 & -0.0818024 & Payer & European & European & European & European|ECO & Neutral\\
France & http://www.laprovence.com/article/sorties-loisirs/4436502/marseille-quand-un-parc-simagine-au-pied-de-grand-littoral.html & 146 & LaProvence.com & Private/Non-Public & Online and Offline & Regional/Local & very low = CP mentioned once & Environment/green/low-carbon & Positive & EU + Subnational & No myth & NA & NA & NA & NA & NA & NA & NA & NA & France & marseille : quand un parc s'imagine au pied de grand littoral & 2017-05-06 & fonds européen de développement régional & avec "foresta", yes we camp voit grand. 16 hectares en gestation pour 2018 là, au pied de grand littoral, près des immenses lettres marseille, ils imaginent depuis l'automne dernier un hameau vivant, une ferme de maraîchage. d'ateliers en balades, le projet de foresta (du nom du marquis qui fut autrefois propriétaire des lieux) continue de se chercher et trouve toujours plus d'idées, un peu folles comme les herbes qui poussent ici. ce parc métropolitain esquissé par yes we camp (avec la complicité d'hôtel du nord et le bureau des guides du gr2013 qui y passe, notamment) se dessine tout doucement et attire des curieux toujours plus nombreux - les classes de la ligue de l'enseignement sont passées par là pour des olympiades, le collectif safi y prépare ses sirops et confitures. pour l'instant, les chemins d'argile, arides, creusés par les motos ne sont guère arpentés que par des retraitées-marcheuses qui leurs bâtons en mains font de la gymnastique dans cette ancienne coulée verte abandonnée, ou par guy qui chaque jour promène son chien devant la vue de la rade marseillaise à couper le souffle. aujourd'hui, en dehors de billy, l'âne qui broute goulûment la verdure de foresta parsemée de coquelicots et de quelques affichettes pour répertorier les végétaux et délimiter des parcelles, rien ne laisse présager des futurs usages de ce site sauvage de 16 hectares, confié à ces rêveurs par son propriétaire résiliance (qui construit non loin son marché marseille international fashion 68). léa ortelli et gauthier oddo qui chapeautent ce projet participatif et "captivant" pour yes we camp voient pourtant les choses avancer : "on a besoin de ce temps long pour lancer un parc pérenne, qui va employer 25 personnes. on se crée un cadre et des outils", glissent-ils, occupés à cette lente digestion d'un paysage qu'ils doivent apprivoiser avant de lui donner forme. les travaux n'ont pas commencé sur le terrain car les recherches de financement sont toujours en cours (auprès du fonds européen de développement régional pour un co-financement à hauteur d'1,9 million d'euros) et parce qu'un projet de permis de construire doit aussi être terminé pour les installations légères (en conteneur et en terre) du hameau. ce coeur sera constitué "d'ateliers de savoir-faire (une base pour accueillir la créativité céramique, mécanique, textile ou boulangère), d'une placette autour d'un four à pain et aussi d'un studio radio, d'une buvette, plus loin d'un camping", précise gauthier oddo, son architecte. "on a quatre ans pour habiter des cheminements autour de quatre axes : le sport, l'histoire des tuiles, l'eau et la biodiversité", dit-il encore, qui s'enthousiasme à imaginer des joggeurs le soir dans des allées éclairées, accédant aux vestiaires qu'il a conçus, près d'une salle commune où leurs enfants envahiront une bibliothèque. "on avance en terrain sensible, c'est un défi de le faire le plus interactif possible, il faut amener la vie d'abord", poursuit léa ortelli. c'est aussi l'objectif de la journée du 4 juin. un dimanche à foresta, léger et animé, ouvert à la nature et ludique, qui promet d'apporter de nouvelles clés pour les aménagements futurs. yeswecamp.org gwenola gabellec & 549 & very low & Low & Socio-Economic & NA & NA & 2017-05-06 & 2017 & 2 & ECO
Frame & v.low & Regional & 500-1000 & -0.7708104 & -0.8786862 & 0.6041414 & -0.2983228 & 0.6486612 & 0.0 & -0.0275917 & -0.0818024 & Payer & Domestic & European & Mixed & Domestic|ECO & Positive\\
France & http://www.lefigaro.fr/flash-eco/2018/05/29/97002-20180529FILWWW00115-budget-de-l-ue-moins-d-argent-pour-l-europe-de-l-est.php & 113 & Le Figaro.fr & Private/Non-Public & Online and Offline & National & medium = CP is important part of story & Economic development & Positive & Other country & No myth & Solidarity to poor countries/regions & Positive & EU + Other country & No myth & NA & NA & NA & NA & France & budget de l'ue: moins d'argent pour l'europe de l'est & 2018-05-29 & politique de cohésion & les pays de l'est de l'europe toucheront moins d'argent de bruxelles au titre de sa politique de cohésion dans le prochain budget de l'ue pour la période 2021-2027, car ils sont devenus "compétitifs", a affirmé mardi le commissaire européen au budget günther oettinger. "les pays comme la slovaquie, les pays baltes ou la pologne reçoivent moins d'argent dans notre proposition de budget pour la politique de cohésion, parce qu'ils sont devenus plus compétitifs, qu'ils ont grandi économiquement", a expliqué le commissaire allemand devant les eurodéputés lors d'un débat à strasbourg sur le prochain cadre financier pluriannuel (cfp) de l'ue. "d'autres, parce qu'ils sont restés un peu plus en stagnation ces dernières années, comme les italiens, reçoivent plus d'argent", a-t-il ajouté. m. oettinger se dit "convaincu" que certains des etats membres qui bénéficiaient de cette politique européenne de soutien pourraient à l'avenir devenir des contributeurs nets au budget de l'ue grâce à leurs bonnes performances économiques. & 172 & medium & Medium & Socio-Economic & Values & NA & 2018-05-29 & 2018 & 3 & ECO
Frame & low-medium & National & <500 & -0.7708104 & -0.8786862 & 0.6041414 & -0.2983228 & 0.6486612 & 0.0 & -0.0275917 & -0.0818024 & Payer & European & European & European & European|ECO & Positive\\
France & https://www.20minutes.fr/monde/2226267-20180223-union-europeenne-sommet-27-debattre-budget-post-brexit & 167 & 20minutes & Private/Non-Public & Online and Offline & National & very low = CP mentioned once & Financial burden & Balanced & EU + Other country & 1.Poor regions funded only & NA & NA & NA & NA & NA & NA & NA & NA & France & union européenne: un sommet à 27 pour débattre du budget post-brexit & 2018-02-23 & politique de cohésion & réunion importante ce vendredi où les dirigeants européens vont débattre pour la première fois, à 27, du budget de l'ue après 2020 et le départ du royaume-uni, et se pencher sur le mode de désignation du successeur de jean-claude juncker à la tête de la commission, un enjeu qui divise. >> a lire aussi : royaume-uni: des britanniques lancent "renew", l'alter-ego de la république en marche les grands projets de l'ue pour se réinventer après le retrait britannique, dont le président français emmanuel macron s'est fait l'un des hérauts, vont être confrontés aux dures questions d'argent et de souveraineté. après avoir fait face à des défis inédits ces dernières années, en matière de sécurité intérieure ou d'accueil de migrants, l'ue est prête à se doter des moyens financiers pour y répondre à l'occasion de son prochain " cadre financier pluriannuel " (cfp). " le budget de l'ue a toujours été un sujet qui crée des divisions. cela le restera avec le départ du royaume-uni, il y aura peut-être un peu moins de divisions ", a ironiquement noté un haut responsable européen sous couvert de l'anonymat. >> a lire aussi : "renew britain": le nouveau parti britannique est-il vraiment le jumeau d'en marche? mais se pose la question du financement de ces nouvelles mesures alors que le budget de l'ue va perdre l'un des ses principaux contributeurs, soit environ 10 milliards d'euros par an selon le conseil européen, l'instance qui regroupe les etats membres. selon une autre source européenne, il s'agit d'un " exercice inédit " pour le conseil puisque la commission européenne ne mettra ses propositions sur la table que début mai. sous la forme d'une " discussion politique " qui doit fournir quelques pistes à l'exécutif européen. >> a lire aussi : brexit: déjà 10 entreprises étrangères ont quitté londres pour paris, une soixantaine pourraient suivre pour ce budget post-2020 - le cfp actuel court sur 7 ans : de 2014 à 2020 -, l'idée de la commission est de décider des priorités, puis d'ajuster le budget en conséquence. il faudra faire des choix, a d'ores et déjà prévenu bruxelles. l'essentiel du budget de l'ue, environ 70 \%, est pour l'instant consacré aux piliers historiques de l'union : la politique de cohésion, visant à permettre aux régions les plus pauvres de rattraper leur retard, et la politique agricole commune (pac). " il y a les pays qui ne veulent pas payer plus et ceux qui ne veulent pas recevoir moins ", a résumé le président de la commission européenne jean-claude juncker. selon le haut responsable européen, la plupart des etats membres sont d'accord pour soutenir l'idée d'une augmentation de leur participation, mais " ceux qui sont sceptiques ou contre sont très déterminés ". & 470 & very low & Low & Values & NA & NA & 2018-02-23 & 2018 & 3 & ECO
Frame & v.low & National & <500 & -0.7708104 & -0.8786862 & 0.6041414 & -0.2983228 & 0.6486612 & 0.0 & -0.0275917 & -0.0818024 & Payer & European & European & European & European|ECO & Neutral\\
France & http://www.lemonde.fr/idees/article/2017/03/15/l-europe-doit-apporter-un-soutien-concret-aux-citoyens-les-plus-durement-frappes-par-la-crise\_5094803\_3232.html & 143 & Le Monde.fr & Private/Non-Public & Online and Offline & National & low = CP mentioned more times but NOT important part of story (mainly about others issues) & Economic development & Balanced & EU + National & No myth & NA & NA & NA & NA & NA & NA & NA & NA & France & " l'europe doit apporter un soutien concret aux citoyens les plus durement frappés par la crise " & 2017-03-15 & fonds social européen & tribune. face aux défis immenses qu'il affronte, le projet européen ne survivra que s'il est capable de faire la preuve de son utilité dans nos vies quotidiennes. mais le projet européen et, singulièrement, la monnaie unique requièrent aussi, pour leur survie, des réformes de fond, moins visibles quoique tout aussi indispensables. pour agir sur ces deux dimensions, c'est-à-dire viser des résultats à court terme tout en ayant une action structurante, il faut instituer un budget de la zone euro qui puisse venir en aide à un ou plusieurs pays en crise économique afin de contrer celle-ci et d'éviter la contagion. les propositions en ce sens se ­concentrent à ce jour sur l'instauration d'une assurance-chômage européenne qui aurait le mérite d'offrir une solidarité européenne dans un domaine très ­concret pour les citoyens. il est un autre domaine où ces deux propriétés pourraient être alliées : celui de la formation. les pays de la zone euro devraient lancer un dispositif simple, un fonds d'investissement à même d'être mobilisé rapidement en cas de choc économique. avec la garantie conjointe des etats, ce fonds emprunterait sur les ­marchés les sommes nécessaires pour ­financer une année d'étude ou d'apprentissage pour les jeunes, les chômeurs, les salariés des entreprises confrontées à un surplus temporaire de main-d'œuvre afin d'éviter leur licenciement, ainsi que pour les réfugiés. un diplôme valide dans toute l'europe les bénéficiaires eux-mêmes rembourseraient le fonds par le biais d'une taxe spécifique modeste sur leurs revenus futurs, à la condition que ceux-ci dépassent le salaire minimum. nous suggérons de le dénommer " fonds spinelli ", du nom d'un des pères fondateurs de l'europe. france stratégie a proposé ce projet à la commission européenne dans le cadre de la réflexion ouverte par le livre blanc sur l'avenir de l'europe, et notamment son pilier social. le dispositif peut être conçu sans induire de transferts financiers entre pays, car chacun d'entre eux resterait redevable des sommes engagées par le fonds sur son territoire. si, à terme, trop de ­bénéficiaires d'un pays ne remboursaient pas, c'est le budget national qui devrait compenser, mais cela n'interviendrait qu'une fois le pays sorti de la période économique difficile. il est aussi possible d'envisager des subventions du fonds social européen (fse). les formations seraient dispensées par des institutions éducatives et de... & 407 & low & Low & Socio-Economic & NA & NA & 2017-03-15 & 2017 & 2 & ECO
Frame & low-medium & National & <500 & -0.7708104 & -0.8786862 & 0.6041414 & -0.2983228 & 0.6486612 & 0.0 & -0.0275917 & -0.0818024 & Payer & Domestic & European & Mixed & Domestic|ECO & Neutral\\
\addlinespace
France & http://www.ouest-france.fr/europe/ue/derapage-budgetaire-lespagne-et-le-portugal-echappent-des-amendes-4389775/ & 135 & Ouest France & Private/Non-Public & Online and Offline & Regional/Local & low = CP mentioned more times but NOT important part of story (mainly about others issues) & Political leverage & Factual & EU + Other country & No myth & NA & NA & NA & NA & NA & NA & NA & NA & France & dérapage budgétaire. l'espagne et le portugal échappent à des amendes & 2016-07-27 & fonds structurels & la commission européenne, sous pression après le brexit, a renoncé ce mercredi à infliger des amendes pour dérapage budgétaire à l'espagne et au portugal, des sanctions qui auraient été sans précédent dans l'histoire de l'ue. le collège des 28 commissaires européens s'est mis d'accord " pour annuler les amendes des deux pays ", a déclaré à la presse le vice-président de la commission européenne chargé de l'euro, valdis dombrovskis. " des sanctions, même symboliques, n'auraient pas permis de corriger le passé et auraient été contre-productives à un moment où les peuples doutent de l'europe ", a expliqué le commissaire européen aux affaires économiques, le français pierre moscovici. les deux pays s'en sont aussitôt félicités. " une très bonne nouvelle pour le portugal et pour l'europe ", a réagi le ministre portugais des affaires étrangères, augusto santos silva. des amendes jusqu'à 0,2 \% du pib " satisfait de la décision ", le chef du gouvernement espagnol sortant mariano rajoy a promis que " la consolidation budgétaire, la croissance et l'emploi étaient des priorités pour l'espagne ". les amendes auraient pu atteindre jusqu'à 0,2 \% du produit intérieur brut (pib) de chacun de ces deux pays. si cette proposition de la commission ne convient pas aux ministres de la zone euro, ils ont dix jours pour la rejeter à la majorité qualifiée. en 2015, le déficit public espagnol avait atteint 5,1 \% du pib, un chiffre bien au-dessus du plafond de 3 \% fixé par le pacte de stabilité et des objectifs de la commission de 4,2 \%. quant au portugal, il avait affiché un déficit public de 4,4 \% du pib l'an passé alors que l'objectif fixé était de repasser sous les 3 \%. délai supplémentaire outre sa clémence concernant les amendes, l'exécutif européen a accordé deux années supplémentaires à l'espagne pour faire tomber son déficit sous les 3 \%, soit jusqu'en 2018. pour ce pays - qui n'a toujours pas formé de nouveau gouvernement depuis les législatives du 26 juin -, la commission européenne recommande la trajectoire budgétaire suivante : un déficit de 4,6 \% du pib en 2016, 3,1 \% en 2017 et 2,2 \% en 2018. concernant le portugal, la commission européenne table sur un déficit de 2,5 \% du pib en 2016, se montrant ainsi moins optimiste que le gouvernement socialiste, au pouvoir depuis novembre, qui promet de le ramener à 2,2 \%. m. dombrovskis a pointé les " dures crises économiques et financières " par lesquelles sont passés les deux pays. des efforts " qui ne doivent pas être sous-estimés " " ils ont réussi à rétablir la stabilité financière grâce à des ajustements budgétaires majeurs. et ils sont passés par des réformes structurelles pour gagner en compétitivité. ces efforts ne doivent pas être sous-estimés ", a-t-il commenté. la commission européenne est aussi dans l'obligation de proposer une suspension totale ou partielle des engagements des fonds structurels, qui peut aller jusqu'à 0,5 \% du pib ou jusqu'à 50 \% des engagements pour 2017 dans ces deux pays. elle compte entamer un " dialogue " après les vacances d'été avec le parlement européen sur cette question qui concerne douze fonds au portugal et une soixantaine en espagne. leur but principal est de réduire les écarts de développement en europe en aidant les régions en difficulté. les règles budgétaires européennes n'ont cessé d'être bafouées depuis leur création, et notamment par les deux poids lourds de la zone euro, la france mais aussi dans une moindre mesure l'allemagne, qui prône désormais davantage de sévérité. en 2017, année électorale, paris, qui a déjà obtenu plusieurs délais, est en tout cas attendu au tournant : son déficit public doit passer en dessous des 3 \% du pib. & 623 & low & Low & Power & NA & NA & 2016-07-27 & 2016 & 2 & POL
Frame & low-medium & Regional & 500-1000 & -0.7708104 & -0.8786862 & 0.6041414 & -0.2983228 & 0.6486612 & 0.0 & -0.0275917 & -0.0818024 & Payer & European & European & European & European|POL & Neutral\\
France & https://www.huffingtonpost.fr/michel-levyprovencal/laide-durgence-dont-leurope-a-besoin-pour-eviter-son-declin\_a\_23635835/ & 190 & Le Huffington Post & Private/Non-Public & Online and Offline & National & medium = CP is important part of story & Institutional bargaining over funding & Negative & EU & No myth & NA & NA & NA & NA & NA & NA & NA & NA & France & blog - l'aide d'urgence dont l'europe a besoin pour éviter son déclin & 2019-01-07 & fonds de cohésion & si nous voulons construire une europe ambitieuse, souveraine, protectrice et progressiste nous devrons dépasser les clivages qui s'annoncent entre populistes et mondialistes. l'année 2019 sera décisive pour l'avenir de l'europe. son rayonnement, son économie, son modèle de société seront durablement affectés par les choix qui seront faits cette année: le vote au parlement, la réalité du brexit et par conséquent le contenu des discussions qui entoureront ces moments clés. il faut espérer que le prochain débat européen soit l'occasion de nous projeter dans les deux prochaines décennies et non les seules deux prochaines années. si nous ne le faisons pas, si nous restons obsédés par les petites ambitions politiques, si notre horizon reste bloqué aux prochaines élections, nous perdrons sur tous les plans: économique, politique, social, environnemental, militaire, culturel, technologique... car pendant que nous débattrons, 2019 sera l'année où nous vivrons de véritables tournants. comme le souligne très justement la dernière édition de the economist (en anglais et sur abonnement, ndlr), l'inde va dépasser simultanément la grande-bretagne et la france pour devenir la cinquième économie mondiale. la confrontation entre l'amérique et la chine, conjuguée à l'influence croissante de la chine en europe, pourrait nous contraindre à envisager des compromis délicats avec pékin. le retrait des troupes américaines du moyen-orient nous mettra au défi de renforcer nos propres forces sur le terrain. même dilemme sur le front entre l'ukraine et la russie, nous serons peut être amené à nous positionner de manière ferme. idem dans les balkans occidentaux, où une révolte pourrait augurer d'un retour à l'instabilité et à la violence. en europe centrale, l'hystérie liée aux migrants passe sous silence une menace bien plus grave: la mort lente causée par l'émigration et le faible taux de natalité. pour faire face au déclin des nations européennes, il nous faut une vision claire pour les vingts prochaines années, il nous faut une stratégie et une cohérence politique et économique. aujourd'hui nous sommes en retard sur l'amérique et la chine dans la course aux nouvelles technologies. notre démographie, nos insuffisances industrielles, notre incapacité à nous projeter nous exposent aux grandes tendances qui émergent et que nous sous-estimons. si nous voulons construire une europe ambitieuse, souveraine, protectrice et progressiste nous devrons dépasser les clivages qui s'annoncent entre populistes et mondialistes. chaque bloc devra entendre l'autre et faire des concessions. comment ? probablement en s'attaquant aux deux points les plus durs: la question de la souveraineté d'une part et les contraintes budgétaires imposées par bruxelles d'autre part. il nous faudra passer par un nouveau compromis sur ces deux points. la bonne nouvelle est que les deux puissances mondiales ont largement confirmé leur politique en matière de souveraineté et de budget. le protectionnisme américain et chinois ne sont plus à démontrer. pourquoi l'europe ne pourrait-elle pas instaurer une politique protectionniste du même type ? cloud act, taxes douanières, protections de ses brevets, l'europe doit se protéger comme le font chinois et américains. sur le front de la politique budgétaire toutes les grandes puissances laissent filer leur dette et leur déficit (usa 3.9\% de déficit et 110\% de dette, chine 3\% de déficit en 2017 et 250\% de dette). notre dette à l'échelle européenne n'est que de 80\% avec un déficit que l'on s'impose autour de 3\%. notre marge de manœuvre est donc immense. aujourd'hui les fonds européens tentent de maintenir en place un système bâti au xxeme siècle mais qui peine à s'adapter aux enjeux de demain. il existe cinq fonds structurels: le fonds européen de développement régional (feder), le fonds social européen (fse), le fonds de cohésion, le fonds européen agricole pour le développement rural (feader) et le fonds européen pour les affaires maritimes et la pêche (feamp). ces cinq fonds pèsent autour de 600 milliards d'euros sur la période 2014 - 2020. nous devrons les réadapter et les redéployer pour la période 2020-2025 sous peine de détruire les équilibres existants. mais il faudra un nouveau souffle, une nouvelle manne pour préparer demain. il faut doter l'europe d'un nouveau fond stratégique pour demain, dédié au développement de l'industrie technologique du xxieme siècle. ce fond pourrait atteindre 150 milliards d'euros sur 5 ans et être financé par un grand emprunt européen. notre dette resterait alors très largement en dessous de celle des autres puissances mondiales en avoisinant les 100\% du pib à l'échelle des 27. il nous permettrait de co-construire les pôles industriels de la défense, de la santé, de l'énergie, de l'éducation, de la mobilité et de l'habitat de demain. comme le font américains et chinois, nous investirions massivement dans la création de programmes publics / privés puissants à l'échelle du continent pour le développement de l'ia, la robotique, l'iot, le spatial, les énergies nouvelles, la blockchain, le quantique et la biotech. cette arrivée massive de financement, comparable au plan marshall pendant les années 50 donnerait un nouveau souffle à nos économies, au marché de l'emploi et au secteur de la recherche. elle permettrait de bâtir en 5 ans le socle de ce que serait l'industrie européenne de demain. evidemment, il faudrait être sourd pour ne pas entendre ricaner les cyniques et être naïf pour penser qu'un tel plan serait simple à mettre en œuvre à bruxelles. sauf qu'il faudrait aussi être aveugle pour ne pas voir que nous sommes au bord du gouffre. sans vision, sans débat constructif et sans plan nous sommes condamnés à vivre collectivement ce que la grèce a vécu ces dernières années. nous avons des options qui nous permettent de rester optimistes et ambitieux pour demain. investir à l'échelle des 27 en est une. l'europe c'est 6\% du pib bloqué en liquidité dans les grandes entreprises non financières et 13\% de taux d'épargne des populations. cette épargne ne vaudra rien si la crise éclate. décider de lancer un grand emprunt européen pour construire l'industrie européenne du xxieme siècle n'est plus un risque c'est un gilet de sauvetage ! un acte qui pourrait dépasser les clivages actuels, en suivant la ligne populiste sur la question de la souveraineté européenne et la fin du carcan budgétaire des technocrates de bruxelles. ce serait en même temps conserver le cap voulu par les mondialistes vers une europe ambitieuse, avant-gardiste et ouverte sur le monde. seuls les citoyens européens peuvent prendre conscience de la situation et faire basculer le débat vers un vrai projet de société à long terme. c'est notre responsabilité de faire de cette année une opportunité et arrêter de nous écharper sur le sexe des anges pendant que le monde avance. & 1135 & medium & Medium & Power & NA & NA & 2019-01-07 & 2019 & 3 & POL
Frame & low-medium & National & +1000 & -0.7708104 & -0.8786862 & 0.6041414 & -0.2983228 & 0.6486612 & 0.0 & -0.0275917 & -0.0818024 & Payer & European & European & European & European|POL & Negative\\
France & https://www.lesechos.fr/idees-debats/cercle/cercle-180566-europe-lappel-des-territoires-2163315.php\#Xtor=AD-6000 & 101 & LesEchos.fr & Private/Non-Public & Online and Offline & National & very high = CP is most important issue + CP is mentioned in title/headline & Institutional bargaining over funding & Positive & EU + National + Subnational & No myth & Economic development & Positive & EU + National + Subnational & No myth & NA & NA & NA & NA & France & europe : l'appel des territoires & 2018-03-22 & politique de cohésion & nous lançons ici l'appel des territoires, par lequel nous rappelons avec force notre soutien à une politique de cohésion ambitieuse, pour l'avenir de l'espace européen. pour la première fois de notre histoire, un état membre s'apprête à quitter l'europe. la vague de fond qui a conduit le royaume-uni à renoncer à l'horizon européen a été alimentée des courants du populisme, du repli sur soi, des fractures territoriales et de la défiance envers des institutions communautaires jugées trop distantes, complexes, désincarnées. ne nous y trompons pas : ces courants ne s'arrêtent pas aux rivages britanniques, mais traversent l'ensemble de nos sociétés. nous sommes tous garants de la préservation d'un idéal européen longuement construit et chèrement payé. souvent critiquée, l'europe reste pourtant une construction démocratique et politique inédite, un destin partagé dont nous sommes tous les garants. nous pourrions être fatalistes et considérer qu'aucune digue ne sera assez puissante pour arrêter ce mouvement. nous pouvons aussi être optimistes, et voir dans la période actuelle une opportunité historique de renforcer l'europe en faisant de ces courants une énergie puissante au service d'une europe refondée. comme l'a rappelé le président de la république française dans son ambitieux discours de la sorbonne, "les plus belles idées, celles qui nous font avancer, qui améliorent le sort des hommes sont toujours fragiles". nous le savons pour être des élus de terrain, confrontés chaque jour à la responsabilité de faire advenir une société plus durable, plus juste, plus inclusive. nous le savons pour travailler chaque jour à la réduction des fractures territoriales. nous le savons pour organiser chaque jour l'alliance des territoires qui permettra de répondre aux grands enjeux de demain. car si le président de la république a rappelé qu'une "nation qui se rétrécit sur elle-même ne peut faire qu'à peu près et peu de choses" face aux enjeux du réchauffement climatique, de la sécurité, de la transition numérique, de la crise migratoire, nous expérimentons au quotidien l'impossibilité de répondre à ces défis avec des territoires repliés sur eux-mêmes. renforcer l'europe, c'est d'abord préserver la politique de cohésion, principale politique d'investissement de l'union européenne mise en place pour harmoniser le développement de l'ensemble des territoires et qui, depuis trente ans, a largement démontré sa plus-value. renforcer l'europe, c'est la rendre toujours plus visible aux yeux des citoyens. renforcer l'europe, c'est la rendre plus efficace, plus lisible, plus évidente. la cohésion est une réalité : nous sommes interdépendants, notre avenir est lié, avec ou sans l'europe. la cohésion est surtout une ambition et une responsabilité : elle se construit jour après jour, et constitue une dynamique qui doit toujours être alimentée. nous pourrions bien sûr rappeler que la politique européenne de cohésion représente, pour la france, près de 16 milliards d'euros sur sept ans. nous pourrions faire la liste des nombreux projets qui n'auraient pas vu le jour sans son soutien, et qui changent la vie de nos habitants. nous pourrions rappeler notre mobilisation quotidienne pour construire un monde durable et inclusif, qui ne laisse personne sur le bord du chemin, et la nécessité de porter cette ambition collective ensemble, pour aujourd'hui et les générations futures. pourtant, dans un espace globalisé en pleine reconfiguration et théâtre de nombreuses crises, nous mettons plus que des enjeux financiers en balance. nous jouons notre capacité à créer du commun, à renforcer la démocratie européenne en l'incarnant sur nos territoires, à placer la solidarité et la réciprocité au coeur du projet communautaire. nous sommes donc face à une alternative qui ressemble à un défi de civilisation : céder à la tentation du repli sur soi en faisant de la politique de cohésion une variable d'ajustement budgétaire. ou au contraire, renforcer cet outil au service d'un projet partagé, en déployant toute son efficacité sur les territoires. l'ouverture des négociations sur l'évolution de la politique de cohésion est un enjeu crucial, un choix d'avenir, voire historique. au regard des grands équilibres mondiaux et de leur évolution, l'union européenne doit être un phare et porter un projet de société plus solidaire, plus respectueux de l'environnement, plus humain. les territoires, notamment urbains, sont dépositaires de cet humanisme européen. pour l'europe, pour notre pays et pour l'ensemble de ses habitants, nous appelons solennellement à préserver la politique européenne de cohésion comme l'une des fondations de notre projet européen. face aux défis qui sont les nôtres, les territoires sont en première ligne. maintenir la cohésion au coeur de notre projet européen est un devoir, la construire avec les territoires est une nécessité. jean-luc moudenc est président de france urbaine, maire de toulouse et président de toulouse métropole. jean-luc rigaut est président l'assemblée des communautés de france, maire d'annecy et président de la communauté d'agglomération du grand annecy & 827 & very high & High & Power & Socio-Economic & NA & 2018-03-22 & 2018 & 3 & POL
Frame & high-very high & National & 500-1000 & -0.7708104 & -0.8786862 & 0.6041414 & -0.2983228 & 0.6486612 & 0.0 & -0.0275917 & -0.0818024 & Payer & Domestic & European & Mixed & Domestic|POL & Positive\\
France & http://www.lexpress.fr/actualite/monde/europe/migrants-berlin-suggere-de-sanctionner-financierement-les-pays-opposes-aux-quotas\_1715785.html & 154 & LExpress.fr & Private/Non-Public & Online and Offline & National & very low = CP mentioned once & Solidarity to poor countries/regions & Factual & EU + Other country & No myth & NA & NA & NA & NA & NA & NA & NA & NA & France & migrants: berlin menace de sanctions financières les pays opposés aux quotas & 2015-09-15 & fonds structurels & nouveau coup de pression de berlin sur la question des migrants. après avoir annoncé le rétablissement provisoire de ses frontières avec l'autriche, le gouvernement continue de militer pour une prise de responsabilité européenne. le ministre de l'intérieur allemand, thomas de maizière, a évoqué mardi matin la possibilité de réduire les fonds structurels versés par l'union européenne aux pays qui rejettent l'idée de quotas de répartition des réfugiés, après l'échec lundi d'une réunion européenne. "nous devons parler de moyens de pression", a-t-il dit à la chaîne allemande zdf. les pays qui refusent la répartition par quotas "sont des pays qui reçoivent beaucoup de fonds structurels" européens, a-t-il justifié, trouvant "juste qu'ils reçoivent moins de moyens", et disant reprendre une proposition du président de la commission européenne, jean-claude juncker. & 139 & very low & Low & Values & NA & NA & 2015-09-15 & 2015 & 1 & ECO
Frame & v.low & National & <500 & -0.7708104 & -0.8786862 & 0.6041414 & -0.2983228 & 0.6486612 & 0.0 & -0.0275917 & -0.0818024 & Payer & European & European & European & European|ECO & Neutral\\
France & http://france3-regions.francetvinfo.fr/normandie/aide-europeenne-peut-toucher-petites-entreprises-1353263.html & 132 & France 3 Picardie & Public & Online only & Regional/Local & high = CP is most important issue in story (can also cover other issues) & Economic development & Positive & EU + Subnational & No myth & NA & NA & NA & NA & NA & NA & NA & NA & France & " l'aide européenne peut toucher les petites entreprises ! " - - france 3 normandie & 2017-10-23 & fonds européen de développement régional & pierre moscovici, est en déplacement en normandie ce lundi. cet après-midi, il se rendra à caen pour débattre de l'action de l'europe dans le développement de l'économie locale. .@pierremoscovici est à l' @emnormandie pour dialoguer sur l'action de l'\#europe dans le développement et l' \#economie locale\#normandie pic.twitter.com/ki37cmnb0e -- préfet du calvados (@prefet14) 23 octobre 2017 car la question nous taraude, et vous vous la posez peut-être vous aussi : que fait l'europe au niveau local ? qui bénéficie ou peut bénéficier d'aides financières ? comment s'y prendre ? pour nous aider à y voir clair, bénédicte feuger, responsable de l'association carrefour rural européen des acteurs normands (crean), a répondu à nos questions. avez-vous des exemples concrets de projets normands qui ont bénéficié de l'aide européenne ? en 2016, les maîtres laitiers du cotentin ont eu le projet de construire une nouvelle usine pour produire du lait infantile à destination du marché chinois. l'europe les a financés à hauteur de 50\% ce nouveau projet. l'usine a été érigée à méautis, près de carentan dans la manche. depuis juin dernier, elle a commencé ses exportations vers la chine. la ferme coutançaise, qui travaille avec des producteurs locaux autour de coutance, a bénéficié d'un coup de pouce de l'europe pour ouvrir son magasin en 2013, car elle promeut le principe du circuit court. a flers dans l'orne, la commune a voulu rouvrir les bains-douches pour en faire un nouvel espace de co-working. ils ont bénéficié de deux aides, l'une pour les travaux de réhabilitation, l'autre pour la transformation en espace numérique. quels sont les fonds européens dont bénéficie la normandie ? il y a trois types de fonds. les fonds structurels, les fonds thématiques et, depuis 2015, le plan d'investissement pour l'europe. les fonds structurels sont gérés en collaboration entre l'etat et la région. il en existe 4 types :le fonds européen de développement régional (feder) qui vise principalement à améliorer la compétitivité des entreprises et l'attractivité du territoirele fonds social européen et l'initiative pour l'emploi des jeunes (fse-iej) qui finance des actions d'insertion, de formation et d'emploile fonds européen agricole pour le développement rural (feader) qui finance la mise en place de la politique agricole commune (pac). c'est cette aide dont a bénéficié la ferme de coutances.le fonds européen pour les affaires maritimes et la pêche pour l'application de la politique commune de la pêche (pcp)pour la période 2014-2020, la normandie a reçue près d'un milliard d'euros d'aide de l'ue, pour financer ces 4 types de fonds structurel. ils se fondent sur les particularités de la région pour financer des projets concrets dans les domaines de l'emploi, la formation, l'environnement, l'agriculture, etc. les fonds européens structurels les fonds thématiques, eux, sont gérés par l'union européenne elle-même et ne tiennent pas en compte des particularités de la normandie. mais les acteurs de la région peuvent en bénéficier. ils sont très nombreux et agissent dans des domaines très variés. le programme horizon 2020 finance la recherche et l'innovation dans les petites et moyennes entreprises. easi, qui a financé une partie des bains-douches numériques à flers, est un programme pour l'emploi et l'innovation sociale... enfin, le plan d'investissement pour l'europe doit faciliter les prêts et les garanties. quand un projet a besoin de débloquer des fonds rapidement, il envoie une demande au plan d'investissements. c'est lui qui a notamment financé les maîtres laitiers du cotentin. l'entreprise devait débourser 110 millions d'euros et le plan d'investissement lui a prêté 50 millions... comment fait-on pour bénéficier de ces fonds ? les fonds fonctionnent comme des appels à projet. la commission européenne définit au préalable les critères à remplir, le type d'acteur qui peut en bénéficier et l'échéance à respecter. ensuite, les acteurs (entreprises, particuliers, institutions,...) font leur demande. comment financer votre projet grâce aux fonds européens ? parfois, les 3 types d'aides (structurelle, thématique et plan d'investissement) peuvent se combiner pour un même projet. mais les potentiels bénéficiaires ne sont pas forcément au courant de toutes ces aides. il y a un problème de transmission de l'information. et faire une demande de financement, ça demande beaucoup d'investissement. sans aide, les porteurs de projet risquent d'aller dans le mur ! c'est pour ça qu'il y a des accompagnateurs en régions comme nous, les centres d'information europe direct (cied). en normandie il y en a trois : un à caen, un à evreux et le nôtre est à vire. nous sommes allées aux quatre coins de la normandie pour faire passer ce message : l'aide européenne peut aussi toucher les petites entreprises ! lors d'une rencontre organisée, un maçon autoentrepreneur a réalisé qu'il travaillait sur un chantier qui bénéficiait de fonds européens. il s'est rendu compte que lui aussi, indirectement, il était aidé par l'europe. nous-même, quand nous avons eu le projet de créer ce centre d'information à vire, en mettant l'accent sur la ruralité, nous avons fait une demande de financement par l'europe qui nous a aidés. découvrez d'autres projets financés par l'europe sur la chaîne youtube de l'europe s'engage en france. & 905 & high & High & Socio-Economic & NA & NA & 2017-10-23 & 2017 & 2 & ECO
Frame & high-very high & Regional & 500-1000 & -0.7708104 & -0.8786862 & 0.6041414 & -0.2983228 & 0.6486612 & 0.0 & -0.0275917 & -0.0818024 & Payer & Domestic & European & Mixed & Domestic|ECO & Positive\\
\addlinespace
France & http://www.francetvinfo.fr/politique/francois-hollande/quand-la-premiere-ministre-polonaise-se-moque-de-francois-hollande\_2091899.html & 141 & francetv info & Public & Online only & National & very low = CP mentioned once & Solidarity to poor countries/regions & Factual & Other country & No myth & NA & NA & NA & NA & NA & NA & NA & NA & France & quand la première ministre polonaise se moque de françois hollande & 2017-03-11 & fonds structurels & "je suis supposée prendre sérieusement le chantage d'un président dont le taux de popularité est à 4\% et qui ne sera bientôt plus président ?" la question, concernant françois hollande, est cinglante. rapportée par le guardian (en anglais), elle a été prononcée par beata szydlo, première ministre polonaise, après un échange tendu lors d'un sommet des leaders européens à bruxelles le 10 mars. ce qui a déclenché la colère de l'élue polonaise ? la proposition d'angela merkel et françois hollande d'une europe "à plusieurs vitesses". l'idée présentée par paris et berlin était d'être "capable d'avancer plus vite à quelques-uns", a présenté françois hollande, car l'europe a "montré qu'elle n'était pas capable de prendre des décisions au bon moment". "nous n'accepterons jamais de parler d'une europe à plusieurs vitesses", a alors prévenu la première ministre polonaise, estimant que cela "compromettrait l'intégrité" de l'ue et refusant ainsi de signer les conclusions de cette réunion. françois hollande a réagi après ce refus. "m. hollande a souligné que la pologne était l'un des plus grands bénéficiaires des fonds structurels européens", explique le finantial times (en anglais), provoquant ainsi les moqueries de beata szydlo. & 204 & very low & Low & Values & NA & NA & 2017-03-11 & 2017 & 2 & ECO
Frame & v.low & National & <500 & -0.7708104 & -0.8786862 & 0.6041414 & -0.2983228 & 0.6486612 & 0.0 & -0.0275917 & -0.0818024 & Payer & European & European & European & European|ECO & Neutral\\
France & http://www.sudouest.fr/2016/06/29/l-europe-vitale-pour-les-regions-francaises-2417308-6204.php & 140 & sudouest.fr & Private/Non-Public & Online and Offline & Regional/Local & medium = CP is important part of story & Economic development & Positive & EU + Subnational & No myth & NA & NA & NA & NA & NA & NA & NA & NA & France & l'europe, vitale pour les régions françaises & 2016-06-29 & fonds structurels & chargée de l'europe pour la nouvelle-aquitaine, la socialiste souligne l'importance des fonds européens. propos recueillis par benoît lasserre publicité b.lasserre@sudouest.fr ancienne vice-présidente déléguée aux finances de l'ex-aquitaine, la socialiste isabelle boudineau est désormais vice-présidente du conseil régional de la nouvelle-aquitaine, chargée de l'europe et de l'international. " sud ouest " vous siégez depuis la mi-juin au comité des régions de bruxelles. de quoi s'agit-il ? isabelle boudineau le comité des régions (cdr) a été créé en 1994. c'est un organe consultatif de l'union européenne, composé de 350 délégués des pays membres représentant les collectivités territoriales de leur pays. le cdr a pour mission de donner un avis sur les décisions et les directives des instances européennes, à savoir la commission et le parlement. j'y siège depuis la dernière session plénière, qui s'est déroulée les 15 et 16 juin. la représentation française a-t-elle changé avec le nouveau découpage des régions ? non, la france compte toujours 12 représentants des régions mais, comme on ne fait jamais rien comme les autres, le cdr compte aussi des représentants des départements et des communes, alors que les autres pays se limitent aux régions. la nouveauté la plus importante entre l'europe et les régions est que, désormais, grâce à alain rousset lorsqu'il présidait l'association des régions de france, les fonds européens sont gérés par les régions et elles seules, sans double commande avec l'état. à combien se montent les fonds européens pour la nouvelle-aquitaine ? les fonds européens sont votés pour une mandature, en l'occurrence 2014-2020. à l'époque, ils avaient été votés pour les trois régions distinctes mais le total s'élève à 2,5 milliards d'euros. soit le budget annuel de la nouvelle région. ces fonds vont en direction de l'environnement, de l'aide à la compétitivité des entreprises et de la cohésion sociale, à laquelle, malgré ce qu'on peut entendre ici ou là, l'europe s'intéresse de près. il faut ajouter les fonds structurels, qui vont précisément aux infrastructures. les fonds européens sont considérables pour les régions mais ça ne se sait pas car il est plus facile et plus porteur de la considérer juste comme un tiroir-caisse, ou bien sûr de la critiquer. deux exemples seulement : la cité du vin de bordeaux et le projet lascaux 4 ont bénéficié de 12 millions d'euros chacun de la part de l'europe. quelles conséquences aura le brexit pour la région ? le 16 juin, au cdr de bruxelles, la délégation socialiste avait brandi des panneaux " remain " à l'intention des délégués britanniques. les dégâts du brexit seront d'abord pour les ressortissants britanniques installés dans la région. au-delà de la nouvelle-aquitaine, le brexit va hélas amplifier le mouvement populiste antieuropéen, en france et ailleurs. & 482 & medium & Medium & Socio-Economic & NA & NA & 2016-06-29 & 2016 & 2 & ECO
Frame & low-medium & Regional & <500 & -0.7708104 & -0.8786862 & 0.6041414 & -0.2983228 & 0.6486612 & 0.0 & -0.0275917 & -0.0818024 & Payer & Domestic & European & Mixed & Domestic|ECO & Positive\\
France & http://www.liberation.fr/planete/2017/09/26/cinq-axes-cles-pour-regonfler-une-ue-essoufflee\_1599179 & 127 & Libération & Private/Non-Public & Online and Offline & National & very low = CP mentioned once & Political leverage & Factual & EU + National & No myth & NA & NA & NA & NA & NA & NA & NA & NA & France & cinq axes clés pour regonfler une ue essoufflée & 2017-09-26 & fonds de cohésion & europe différenciée, zone euro, sécurité... emmanuel macron a mis sur la table des propositions précises. emmanuel macron, dans un discours ambitieux et d'une rare densité, qui rompt avec les cinq ans de silence obstiné observé sur le sujet par son prédécesseur françois hollande, a plaidé pour une "refondation" de l'union européenne d'ici à 2024 car "nous ne pouvons pas nous permettre de garder les mêmes politiques, les mêmes habitudes, les mêmes procédures, les mêmes budgets". il n'a pas hésité à se placer dans les pas des "pères fondateurs" : "je pense à cet instant à robert schuman, le 9 mai 1950, osant proposer de construire l'europe. je pense à ces mots saisissants : "l'europe n'a pas été faite, nous avons eu la guerre."" décryptage des cinq points clefs de son discours. pour macron, unité ne rime pas avec paralysie. s'il n'exclut pas qu'à terme tous les pays de l'union participent à toutes les politiques européennes, y compris à l'euro, il refuse que les plus réticents bloquent ceux qui veulent aller de l'avant, d'autant que "l'europe est déjà à plusieurs vitesses, n'ayons pas peur de le dire. allons vers ces différenciations" qui impliquent la mise en place d'une "avant-garde". pour rassurer les pays d'europe de l'est, qui sont visés, le chef de l'etat affirme qu'il ne veut pas bâtir de nouveaux murs mais "assurer l'unité sans chercher l'uniformité" et sortir de la recherche actuelle du "plus petit dénominateur commun". assumer que l'europe n'ira pas du même pas avant longtemps ne veut pas dire que ceux qui n'appartiendront pas au cœur nucléaire de l'union pourront faire ce qu'ils veulent. outre les règles du marché unique, qui continueront à se décider à vingt-sept et à s'appliquer aux vingt-sept, "sur les valeurs de la démocratie et de l'etat de droit, [...] il ne peut y avoir d'europe à deux vitesses". la pologne et la hongrie sont prévenues. pour macron, "le cœur d'une europe intégrée", c'est l'euro. en préalable, il écarte toute mutualisation "des dettes du passé", un épouvantail outre-rhin, et il se garde bien d'évoquer de futurs "eurobonds" ou emprunts européens, meilleurs moyens de bloquer toute discussion. il estime cependant nécessaire un budget de la zone euro non pas pour voler au secours des déficits publics, mais pour investir et disposer de "moyens face aux chocs économiques" : "un etat ne peut, seul, faire face à une crise lorsqu'il ne décide pas de sa politique monétaire." ce budget serait abondé par les taxes européennes sur les géants du numérique qu'il veut mettre en place, la taxe carbone, voire une partie de l'impôt sur les sociétés lorsque les taux auront été harmonisés. toujours afin de rassurer l'allemagne, il réaffirme son attachement au pacte de stabilité et à la coordination des politiques économiques dont le pilotage serait assuré par un ministre des finances européen (fusion du poste de commissaire européen aux affaires économiques et de celui de président de l'eurogroupe, l'enceinte réunissant les ministres des finances de la zone euro). enfin, il veut qu'un contrôle parlementaire sur la zone euro soit institué, sans dire s'il sera assuré par l'actuel parlement européen ou par un parlement ad hoc. le chef de l'etat a particulièrement insisté sur la transition énergétique rendue nécessaire par l'accord de paris sur le climat. il veut notamment instituer un prix plancher de la tonne de carbone, et, afin d'assurer une juste concurrence, instituer une taxe aux frontières européennes sur les produits provenant de pays ne respectant pas les standards européens. il propose aussi de créer une véritable communauté européenne de la voiture électrique afin "de traverser l'europe sans l'abîmer". après les scandales alimentaires qui ont ébranlé l'union, comme celui des œufs contaminés, macron souhaite accélérer la transition vers une agriculture "écologique et responsable", notamment en assurant à tous les agriculteurs des revenus décents, ce qui passe par une réforme en profondeur de la politique agricole commune, "un tabou français", celle-ci réservant 80 \% de ses aides à 20 \% des agriculteurs, généralement les plus intensifs. il veut aussi la création d'une "force européenne d'enquête et de contrôle" pour lutter contre les fraudes alimentaires, un domaine jusque-là souverain, et réformer en profondeur les évaluations scientifiques européennes. le chef de l'etat milite pour la mise en place d'une europe fiscale afin de mettre fin à la concurrence de tous contre tous, en adoptant d'ici à 2020 une fourchette de taux d'impôts sur les sociétés. si un etat ne la respecte pas, il serait privé d'une partie des aides régionales (fonds de cohésion). de même, un salaire minimum européen, variable selon les pays, serait créé. enfin, macron demande une politique commerciale plus transparente, qui tienne compte des exigences sociales et environnementales ainsi que la création d'un "procureur commercial européen" chargé de vérifier le respect des règles par les pays tiers et doté d'un pouvoir de sanction. face à "l'inéluctable" retrait américain, l'union n'a d'autre choix que d'assumer sa propre défense "en complément de l'otan", affirme macron. il propose en préalable de créer "une culture stratégique commune" en accueillant au sein de l'armée française des militaires européens afin de participer "à nos travaux d'anticipation, de renseignement, de planification et de soutien aux opérations", une initiative qu'il aimerait voir reprise par toutes les armées de l'union. ensuite, dès les années 2020, "l'europe devra être dotée d'une force commune d'intervention", une proposition qui remonte à 1999 mais n'a jamais vu le jour, "d'un budget de défense commun et d'une doctrine commune pour agir". pour renforcer la coordination des services de renseignements européens dans la lutte contre le terrorisme, le chef de l'etat souhaite la création d'une "académie européenne du renseignement" et que le parquet européen en cours de création ait compétence dans les domaines de la criminalité organisée et du terrorisme, alors que pour l'instant, elle est limitée à la lutte contre les fraudes au budget communautaire. enfin, macron veut rompre avec la politique du chacun pour soi en matière d'immigration et d'asile, les pays de la "ligne de front" étant laissés à eux-mêmes. il propose la création d'un office européen de l'asile, destiné à harmoniser les procédures, une proposition de la commission qui remonte au début du siècle, l'interconnexion des fichiers, le renforcement de la "police européenne des frontières" actuellement dotée d'effectifs symboliques, une gestion commune des reconduites à la frontière et le financement européen de l'intégration des réfugiés. cette politique d'immigration passera aussi par un "partenariat avec l'afrique", qui implique une augmentation de l'aide au développement, qui pourrait à terme être financée par une taxe sur les transactions financières européennes, un serpent de mer, qui prendrait comme modèle la taxe française ou britannique. macron ne veut pas se lancer d'emblée dans une réforme des traités européens. il évoque seulement deux réformes : une réduction de la taille de la commission de vingt-huit à quinze membres (comme cela était prévu à l'origine par le traité de lisbonne) et la création de listes transnationales pour les élections européennes afin de créer un espace public européen. dans un premier temps, seuls les 73 postes libérés par les britanniques seraient concernés, mais il souhaite que la moitié du parlement soit élu à terme dans le cadre d'une circonscription européenne. en clair, les partis européens désigneraient ces candidats et non plus les partis nationaux. pour le reste, il se montre ouvert : soit des coopérations renforcées, soit un accord ad hoc, soit une nouvelle législation communautaire. mais il estime que tous les pays devraient lancer des "conventions démocratiques" afin que les citoyens soient consultés sur l'europe qu'ils souhaitent : elles se réuniraient pendant six mois en 2018 pour débattre et permettre aux partis de s'emparer de leurs conclusions. & 1359 & very low & Low & Power & NA & NA & 2017-09-26 & 2017 & 2 & POL
Frame & v.low & National & +1000 & -0.7708104 & -0.8786862 & 0.6041414 & -0.2983228 & 0.6486612 & 0.0 & -0.0275917 & -0.0818024 & Payer & Domestic & European & Mixed & Domestic|POL & Neutral\\
France & http://www.lefigaro.fr/vox/monde/2017/03/14/31002-20170314ARTFIG00180-brexit-et-referendum-ecossais-l-unite-du-royaume-uni-est-elle-menacee.php & 142 & Le Figaro.fr & Private/Non-Public & Online and Offline & National & very low = CP mentioned once & Economic development & Factual & National & No myth & NA & NA & NA & NA & NA & NA & NA & NA & France & brexit et référendum écossais : l'unité du royaume-uni est-elle menacée ? & 2017-03-14 & fonds de cohésion & pierre-alain coffinier est conseiller des affaires etrangères, ancien consul général à edimbourg, et chercheur associé à l'institut thomas more. pierre-alain coffinier donne une conférence demain, mercredi 15 mars, à 17h30 sur "l'écosse dans le brexit" à l'association franco-écossaise (présidée par l'ancien ambassadeur de france à londres, jean guéguinou). lieu: l'ancien collège des écossais, 65 rue du cardinal lemoine, paris 5e (métro cardinal lemoine) alors qu'on s'attendait à ce que theresa may notifie ce mardi 14 mars la sortie du royaume-uni de l'union européenne, le premier ministre écossais, mme nicola sturgeon, a annoncé la tenue d'un nouveau référendum d'indépendance. il serait de bon sens, a-t-elle déclaré à la bbc dimanche, qu'il ait lieu quand l'accord de sortie sera connu, entre l'automne 2018 et le printemps 2019. prise de court, theresa may se donne deux semaines de plus, avant la fin mars, pour la notification de sortie, alors que westminster vient d'approuver, sans amendement, son projet de loi. pour sturgeon, le brexit change tout. lors du débat sur l'indépendance de l'écosse en 2014, londres expliquait que pour se maintenir dans l'europe, l'écosse devait rester dans le royaume-uni... elle a alors rejeté l'indépendance à 54\%. mais les termes du marché sont aujourd'hui inversés. le royaume-uni dans lequel les écossais ont choisi de se maintenir n'est plus le même. moins eurosceptiques que les anglais, les écossais se sont prononcés, le 23 juin dernier, à 62\% pour l'europe - contre 48\%. toutes les circonscriptions écossaises ont voté dans le même sens. fervente nationaliste, nicola sturgeon a une belle occasion d'approfondir le fossé politique entre les deux "nations". dès juillet dernier, une commission a été créée pour étudier la possibilité d'un "brexit partiel", où seule l'angleterre quitterait l'europe. irréaliste. nicola sturgeon prône maintenant le maintien du pays dans l'espace économique européen, à l'instar de la norvège. mais theresa may a rejeté nettement cette formule en janvier. pire, elle balaie le rapatriement à edimbourg lors du brexit de celles des compétences exclusives de bruxelles (politique agricole commune, politique commune de pêche, innovation, environnement) qui sont au royaume-uni "dévolues" aux régions. dans cette hypothèse, tout reviendra à londres. le long débat sur l'indépendance, entre 2012 et 2014, a bouleversé le soutien à l'indépendance dans la population écossaise, passé d'un étiage traditionnel de 25 à 30\% depuis plusieurs décennies aux 45\% obtenus le jour du scrutin. un exploit. le choc du vote du 23 juin a provoqué un sursaut pour l'indépendance, désormais autour de 54\%. depuis, les écossais, comme leurs voisins, sont restés circonspects: qui pouvait prédire à quoi ressemblerait ce fameux brexit que la discipline démocratique invitait à respecter? l'économie déjouait les prédictions alarmistes. pendant six mois, les sondages ont indiqué un soutien à l'indépendance équivalent à ce qu'il était au premier référendum: 45\%. mais depuis janvier, le choix d'un hard brexit, hors de l'espace économique européen, inquiète. l'écosse recueille une part considérable des aides européennes à l'agriculture britannique, à la recherche et à l'innovation, notamment dans le domaine des énergies éoliennes et marines où elle est pionnière. avec la baisse des cours des hydrocarbures (16\% de son pib), l'économie écossaise est vulnérable. d'autant plus que son secteur financier - glasgow et edimbourg sont les deuxième et troisième places financières du royaume-uni, comparables à copenhague, rome ou madrid - est aussi touché (la gestion de fortune, dont une partie est automatisée, est l'une de ses spécialités). pas question de perdre le "passeport financier européen" ... surtout si la city le perd de son côté! edimbourg serait alors en excellente position pour récupérer une partie des transfuges, avant paris, francfort, amsterdam ou dublin. et les trois derniers sondages mettent en écosse l'indépendance à 49 ou 50\%. si edimbourg ne peut tenir un référendum contraignant sans l'aval de londres, il peut tenir un référendum "consultatif" qui constituerait un fait politique majeur, notamment compte tenu du précédent de 2014: l'union au sein du royaume a toujours été considérée comme volontaire, à l'inverse du royaume d'espagne, constitutionnellement indivisible. mais rien ne serait plus hasardeux que de pronostiquer la voie que suivra le royaume-uni dans les deux années de négociations qui s'annoncent. le premier accord que le royaume-uni devra négocier en deux ans à partir de sa notification de sortie en vertu de l'article 50 du traité de l'union européenne, pour solder les comptes, présente déjà toutes sortes de difficultés. michel barnier demande une facture de l'ordre de soixante milliards d'euros. il s'agit d'abord des fonds de cohésion et investissements structurels pour soutenir le rattrapage des économies d'europe centrale et orientales qui doivent être versés après le départ envisagé du royaume-uni (2019). le deuxième poste est celui des pensions des fonctionnaires européens. comment vendre une telle facture à un électorat qui a protesté contre bruxelles? au-delà, une incertitude totale entoure toujours l'accord de libre-échange "hardi et ambitieux" que theresa may demande avec l'union européenne. le brexit menace aussi de rallumer les tensions en irlande du nord où, pour la première fois depuis l'indépendance en 1921, les partisans de la réunification de l'île avec la république d'irlande viennent de devenir majoritaires au parlement régional. a cette crise multiple pourrait s'ajouter la fronde des milieux économiques et de la city susceptible d'entraîner une dépression qui toucherait nos voisins au portefeuille. les signes avant-coureurs se précisent. ces risques apparaissent soudain à theresa may suffisamment sérieux pour qu'elle reporte à la fin du mois une annonce de la sortie du royaume-uni que l'on attendait pour ce mardi 14 mars. & 980 & very low & Low & Socio-Economic & NA & NA & 2017-03-14 & 2017 & 2 & ECO
Frame & v.low & National & 500-1000 & -0.7708104 & -0.8786862 & 0.6041414 & -0.2983228 & 0.6486612 & 0.0 & -0.0275917 & -0.0818024 & Payer & Domestic & Domestic & Domestic & Domestic|ECO & Neutral\\
France & http://www.laprovence.com/article/actualites/3797265/le-rendez-vous-a-deux-cents-millions-deuros-destrosi.html & 104 & LaProvence.com & Private/Non-Public & Online and Offline & Regional/Local & medium = CP is important part of story & Bureaucracy and/or delays & Negative & EU & No myth & Economic development & Positive & EU & No myth & Poor communication of funding/rules & Negative & EU & NA & France & le rendez-vous à deux cents millions d'euros d'estrosi & 2016-02-11 & fonds structurels & le président de la région paca est à bruxelles pour débloquer des fonds les habitués du dédale bruxellois sont formels. pour croquer une part du pantagruélique gâteau financier européen, il faut demander aux bonnes personnes. simple ? "cela n'a rien d'évident, c'est même un métier, l'art de la politique ici étant d'identifier le bon contact", sourit l'allemand michael hinterdobler, représentant d'une région, la bavière, championne de l'utilisation des crédits proposés par la tortueuse union. arrivé hier sous la pluie belge avec son premier vice-président, l'eurodéputé renaud muselier, christian estrosi a d'abord mesuré l'ampleur de la tâche. essuyé quelques fraîches réflexions. "cela fait plaisir de vous rencontrer, on ne voit pas si souvent les français", glisse ainsi la pétillante luxembourgeoise martine reicherts, directrice générale chargée de la délégation éducation et culture. elle fera découvrir au nouveau président de provence-alpes-côte d'azur les moyens de répondre à des appels d'offres européens sur des projets méconnus d'insertion par le sport. "les français croient qu'il suffit de s'inscrire pour obtenir des fonds, s'amuse un peu plus tard l'espagnol daniel callaja-crespo, directeur général à l'environnement. mais ils arrivent trop tard. si vous avez un projet à nous transmettre d'ici le mois de mai touchant à la croissance verte ou à l'économie circulaire, nous vous aiderons, car nous n'avons pour l'heure rien reçu de la france, alors que nous avons les financements". passé la stupéfaction, christian estrosi approuve et prend rendez-vous pour mars : "avec tout ce que j'ai entendu aujourd'hui, j'ai compris les mécanismes permettant de monter des dossiers sérieux et éligibles, j'espère récupérer 400 à 500 millions de fonds européens dès cette année." le président de la région en a pêché une bonne partie lors d'un seul rendez-vous. en fin d'après-midi, la délégation s'est assise à la même table que walter deffa et ses acolytes de la direction générale à la politique régionale. en clair, ce sont eux qui, de leurs bureaux de verre situés en banlieue bruxelloise, distribuent les fonds structurels aux régions. les fameux feder, sfe ou feader, distribués en fonction de la richesse d'un territoire. très riche, la bavière a droit à 495 millions d'euros entre 2014 et 2020. pour la région paca, c'est 950 millions d'euros. pour y avoir droit, il faut toutefois le justifier à travers des projets et les détailler dans des dossiers qui, pour le coup, relèvent parfois d'orwell ou d'ubu roi. à lire aussi : estrosi va fermer l'antenne de la région à avignonla commission européenne sera à marseille en mars l'administration européenne n'est néanmoins pas seule responsable des manquements français. lors de l'exercice 2007-2013, la région paca a rendu à l'europe, faute d'avoir su les dépenser, quelque 100 millions d'euros de fonds sociaux (fse). depuis 2014, elle n'a engagé que 15 \% des crédits potentiels... "il fallait mettre le paquet et imprimer un nouveau rythme", soufflait hier christian estrosi. en sortant de la réunion avec walter deffa, il a pu ajouter 218 millions d'euros à ses futures lignes budgétaires, dont les premières seront votées le 8 avril. "nous avons indiqué des dossiers qui nous permettent de débloquer dès les prochains mois 60 millions d'euros de fonds feder pour les entreprises, les filières d'excellence et l'accès aux transports, énumère-t-il. nous aurons également 26 m€ pour le retour à l'emploi au titre du fse et 7 m€ pour le massif alpin via les fonds poia. enfin, sur les fonds feader pour l'agriculture, nous avons décroché 125 m€." ravie, la commission européenne sera à marseille à la mi-mars pour aider à ficeler les dossiers et un comité de suivi se réunira fin mai. aujourd'hui, il sera plus question de politique à bruxelles, où christian estrosi rencontrera jean-claude juncker, le président de la commission. juste avant un ultime rendez-vous avec le commissaire européen à l'économie numérique. qui a sans doute des niches financières à dévoiler. françois tonneau & 698 & medium & Medium & Governance & Socio-Economic & Governance & 2016-02-11 & 2016 & 2 & POL
Frame & low-medium & Regional & 500-1000 & -0.7708104 & -0.8786862 & 0.6041414 & -0.2983228 & 0.6486612 & 0.0 & -0.0275917 & -0.0818024 & Payer & European & European & European & European|POL & Negative\\
\addlinespace
France & http://www.ouest-france.fr/bretagne/quimperle-29300/quimperle-l-espace-associatif-le-couteau-suisse-des-assos-5229955 & 153 & Ouest France & Private/Non-Public & Online and Offline & Regional/Local & very low = CP mentioned once & Social awareness/inclusion & Positive & EU + Subnational & No myth & NA & NA & NA & NA & NA & NA & NA & NA & France & quimperlé. l'espace associatif, le couteau suisse des assos & 2017-09-07 & fonds social européen & guillaume hardy animera un atelier pour renforcer la participation des bénévoles lors du forum associatif, samedi. il est le coordinateur ressources de l'espace associatif de quimper. 893 associations gravitent autour de l'espace associatif de quimper. une véritable boîte à outils montée sous statut associatif, qui repose sur 17 salariés prescripteurs de savoir-faire et compétences tous azimuts. l'espace associatif tient une place prépondérante dans le paysage local. financé pour un tiers par quimperlé communauté, le fonds social européen (fse) et le département, un autre tiers par la ville de quimper, il tire la dernière part de ses revenus, de ses prestations. si les adhérents ont la possibilité de s'y domicilier, ils peuvent recourir à un grand nombre de conseils et services mutualisés : sous-traitance des fiches de paye, service d'emploi partagé. l'espace associatif propose ainsi des solutions de partage de compétences, de groupements d'employeurs, et joue sur de nombreux autres tableaux. les associations disposent d'un service de reproduction (photocopieuses, massicoteuse, agrafeuse, etc.) et de conception graphique : création de sites internet, communication papier (billetterie, rapports d'activités, etc.).un banc de montage audiovisuel est également à disposition : le public vient avec ses rushs et peut se faire accompagner dans la scénarisation, la postproduction de son outil de communication. location et soutien technique son et lumière sont proposés pour l'éclairage, le mixage, et l'enregistrement d'événements sportifs et culturels. une formation à la prise de vue a aussi vu le jour (écriture d'un scénario, montage, diffusion). dans le domaine de l'informatique, une formation à un logiciel de comptabilité viendra bientôt compléter les offres pack office et logiciels de traitement d'images. mais le bon fonctionnement associatif ne va pas sans ressources humaines. c'est sur ce volet-là que guillaume hardy, coordinateur ressources de l'espace associatif, intervient. ll va tenir un rôle de médiateur pour réinstaurer de bonnes relations et véhiculer " le rapport un homme = une voix ". et de préciser : " si la participation aux actions est généralement de l'ordre de 10 \% des effectifs adhérents, c'est souvent parce que ces 10 \% le veulent bien. pour inverser la tendance, ce sont souvent les décideurs qui doivent lâcher du pouvoir et apprendre à déléguer ". c'est dans ce cadre que des journées de formation sont proposées : autodéfense verbale, prise de parole en public, animation d'une réunion participative, statut collégial. " l'atelier " susciter et renforcer la participation des bénévoles ", ce samedi, s'adressera aussi bien à ceux qui voudraient davantage prendre part aux actions et décisions, qu'aux personnes pour lesquelles il est nécessaire de passer le relais. " samedi 9, forum des associations, de 9 h à 17 h, à l'espace benoîte-groult. entrée gratuite. adhésion : 37 € pour quimper et quimperlé communauté ; 47 € hors-zones. tél. 02 98 52 33 00. atelier " susciter et renforcer la participation des bénévoles " à 11 h, salle isole.guillaume hardy est coordinateur ressources à l'espace associatif de quimper. il accompagne les associations vers des progrès humains et donnera une conférence samedi, à 11 h. & 512 & very low & Low & Socio-Economic & NA & NA & 2017-09-07 & 2017 & 2 & ECO
Frame & v.low & Regional & 500-1000 & -0.7708104 & -0.8786862 & 0.6041414 & -0.2983228 & 0.6486612 & 0.0 & -0.0275917 & -0.0818024 & Payer & Domestic & European & Mixed & Domestic|ECO & Positive\\
France & http://tempsreel.nouvelobs.com/monde/20160505.AFP4324/ue-refugies-la-bulgarie-exhorte-l-europe-de-l-est-a-plus-de-solidarite.html & 176 & L'Obs & Private/Non-Public & Online and Offline & National & medium = CP is important part of story & Solidarity to poor countries/regions & Positive & EU + Other country & No myth & NA & NA & NA & NA & NA & NA & NA & NA & France & ue/réfugiés: la bulgarie exhorte l'europe de l'est à plus de solidarité & 2016-05-05 & fonds structurels & sofia (afp) - la bulgarie a appelé jeudi les autres pays d'europe de l'est qui refusent le plan européen de partage de réfugiés à plus de "solidarité" tout en réaffirmant son engagement à accueillir le quota de migrants prévu par ce système de répartition. dans un entretien à l'afp, le premier ministre bulgare boïko borissov a souligné que la solidarité ne pouvait pas être à sens unique pour "les pays qui, comme nous, comptent sur la solidarité européenne" en matière de fonds structurels. "nous sommes heureux quand de l'argent est versé" pour le développement des etats les plus pauvres de l'ue mais en retour, quand les pays les plus riches "ont un problème parce que le flux migratoire se dirige chez eux, nous aussi devons faire preuve de solidarité et d'entraide", a insisté le dirigeant bulgare, à la tête depuis 2014 d'une coalition de partis conservateurs. la crise des réfugiés a fait apparaître une fracture entre les pays d'europe de l'ouest promoteurs d'une répartition des demandeurs d'asile au sein de l'ue et plusieurs pays d'europe centrale, comme la hongrie, la slovaquie la pologne, qui rejettent ce mécanisme adopté à l'automne 2015. la bulgarie se distingue de ces anciens pays du bloc communiste en affirmant sa volonté de remplir sa part du contrat. un quota d'accueil de 1.200 migrants lui a été attribué. "qu'ils soient 1.200 ou même 2.000, nous avons pris l'engagement de les accueillir", a assuré le premier ministre qui anticipe cependant une difficulté de taille: les réfugiés "veulent aller en autriche ou dans d'autres pays, ils ne veulent pas rester en bulgarie", pays le plus pauvre de l'union européenne. "faut-il faire une prison pour les retenir ? deux irakiens sont venus (dans le cadre de la répartition, ndlr) et l'un s'est enfui", a affirmé m. borissov. le premier ministre ne s'est toutefois pas montré très enthousiaste sur l'amende de 250.000 euros, proposée par la commission européenne pour chaque demandeur d'asile refusé par un etat membre dans le cadre d'un futur système automatique de répartition des réfugiés. "au lieu de lancer des anathèmes pour dire qu'il y aura des amendes ou autre chose pour ceux qui n'acceptent pas de migrants, il est grand temps que nous trouvions des mécanismes pratiques de travail", a plaidé m. borissov. la hongrie avait qualifié mercredi ce projet de sanction financière de "chantage" et d'"impasse". alors que la bulgarie craint de devenir un axe majeur de transit des migrants depuis la fermeture de la route des balkans de l'ouest, rien n'indique à ce stade, selon le premier ministre, une hausse de la pression migratoire dans ce pays qui a 269 kilomètres de frontière avec la turquie et 460 km avec la grèce. "aujourd'hui la bulgarie protège très bien les frontières de l'espace schengen", dont elle ne fait pourtant pas partie, a souligné le premier ministre regrettant cette situation. membres de l'ue depuis 2007, la roumanie et la bulgarie se sont vu barrer en 2011 l'accès à la zone schengen par certains etats, notamment les pays-bas. & 539 & medium & Medium & Values & NA & NA & 2016-05-05 & 2016 & 2 & ECO
Frame & low-medium & National & 500-1000 & -0.7708104 & -0.8786862 & 0.6041414 & -0.2983228 & 0.6486612 & 0.0 & -0.0275917 & -0.0818024 & Payer & European & European & European & European|ECO & Positive\\
France & http://www.lavoixdunord.fr/region/calais-2-5-millions-d-euros-de-l-europe-pour-soutenir-ia33b48581n3529468 & 116 & La Voix du Nord & Private/Non-Public & Online and Offline & Regional/Local & high = CP is most important issue in story (can also cover other issues) & Social awareness/inclusion & Positive & EU + National + Subnational & No myth & NA & NA & NA & NA & NA & NA & NA & NA & France & calais : 2,5 millions d'euros de l'europe pour soutenir les quartiers prioritaires & 2016-05-25 & fonds structurels & 2,5 millions d'euros pour le renouvellement urbain " l'europe va venir compléter le besoin de financement de la communauté d'agglomération cap calaisis de 2,5 millions supplémentaires pour la rénovation urbaine et pour l'accompagnement des populations des quartiers prioritaires ", c'est ce qu'a indiqué hier valérie létard*, première vice-présidente à la région et chargée des fonds structurels européens, et de l'aménagement du territoire. elle était en déplacement hier à calais pour signer la convention avec la présidente d'agglomération natacha bouchart. ce fonds vient renforcer le contrat de ville signé en juin. cette enveloppe émane d'un fonds européen intitulé investissement territorial intégré (iti). 81 millions d'euros sont consacrés pour la période 2014-2020 à la région des hauts-de-france. treize collectivités ont été retenues dont cap calaisis. pour quels projets ? l'agglomération avait déposé un dossier de candidature en septembre autour de trois axes. sa candidature a été retenue en novembre.le premier axe concerne l'aide à la création d'entreprise dans les quartiers prioritaires du beau-marais et du fort-nieulay avec une subvention européenne de 243 000 € sur un budget global de 410 000 €. le second point concerne le développement numérique (création d'une maison numérique en centre-ville) avec une participation européenne de 472 000 € sur un budget de 790 000 €. enfin,le dernier volet est celui du développement urbain avec une aide de 1,75 million pour la construction (en cours) du centre de loisirs coluche (3,5 millions), avenue courbertin. l'hôpital de calais : un exemple de développement grâce au fonds européen valérie létard a terminé sa visite par une rencontre avec le service informatique du centre hospitalier de calais. l'hôpital a bénéficié depuis 2006 d'1,2 million d'euros de la part de l'europe. cette manne a notamment permis au centre hospitalier d'améliorer la prise en charge des patients par une meilleure coordination des praticiens avec la numérisation des dossiers. *la venue de valérie létard s'inscrivait dans l'opération le joli mois de l'europe. cette action vise à communiquer sur les projets locaux que peut financer l'europe. & 361 & high & High & Socio-Economic & NA & NA & 2016-05-25 & 2016 & 2 & ECO
Frame & high-very high & Regional & <500 & -0.7708104 & -0.8786862 & 0.6041414 & -0.2983228 & 0.6486612 & 0.0 & -0.0275917 & -0.0818024 & Payer & Domestic & European & Mixed & Domestic|ECO & Positive\\
France & http://www.laprovence.com/actu/en-direct/4590046/macron-le-dumping-social-risque-de-provoquer-un-demantelement-de-lue.html & 151 & LaProvence.com & Private/Non-Public & Online and Offline & Regional/Local & very low = CP mentioned once & Political leverage & Negative & EU + Other country & No myth & NA & NA & NA & NA & NA & NA & NA & NA & France & macron : le dumping social risque de provoquer un "démantèlement de l'ue" & 2017-08-24 & fonds structurels & le président français emmanuel macron, en déplacement en roumanie, a estimé jeudi que faute d'amendement de la directive européenne sur le travail détaché, le "dumping fiscal" pratiqué dans certains pays de l'est risquait de mener à un "démantèlement de l'ue". maquillage d'emmanuel macron : une facture de 26 000 € en trois mois "certains politiques ou milieux d'affaires (...) disent on va toucher les fonds structurels et développer un modèle qui est le dumping fiscal et social", a-t-il déclaré lors d'une conférence de presse à bucarest. si rien n'est fait, "ça éclatera", "c'est un démantèlement de l'union européenne", a ajouté le chef de l'etat français au deuxième jour d'une mini-tournée européenne en vue d'un amendement de ce texte controversé. afp & 132 & very low & Low & Power & NA & NA & 2017-08-24 & 2017 & 2 & POL
Frame & v.low & Regional & <500 & -0.7708104 & -0.8786862 & 0.6041414 & -0.2983228 & 0.6486612 & 0.0 & -0.0275917 & -0.0818024 & Payer & European & European & European & European|POL & Negative\\
France & http://www.lepoint.fr/politique/sommet-de-l-ue-l-avenir-de-tusk-en-hors-d-oeuvre-empoisonne-09-03-2017-2110427\_20.php & 123 & Le Point & Private/Non-Public & Online and Offline & National & very low = CP mentioned once & Financial burden & Balanced & National & No myth & NA & NA & NA & NA & NA & NA & NA & NA & France & paris et berlin poussent vers une europe à "plusieurs vitesses" & 2017-03-10 & fonds structurels & l'allemagne et la france ont poussé vendredi vers le scénario d'une europe à "plusieurs vitesses" pour surmonter l'épreuve du brexit, se heurtant au refus des pays de l'est, pologne en tête. les dirigeants nationaux ont débattu à bruxelles de leur avenir à 27, après une première réunion jeudi marquée par une confrontation avec la pologne, qui a tenté en vain d'empêcher la reconduction du polonais donald tusk à la tête du conseil européen. "la devise est que nous sommes unis, mais unis dans la diversité", a déclaré la chancelière allemande angela merkel, évoquant l'objectif d'un texte solennel préparé par les 27 dans l'optique du sommet de rome, prévu le 25 mars. la délicate préparation de cette "déclaration de rome" a occupé pendant plusieurs heures les dirigeants, réunis sans la première ministre britannique theresa may, comme c'est devenu l'usage pour ces réunions sur l'avenir post-brexit. "il faut que l'on soit capable d'avancer plus vite à quelques-uns", a appuyé le chef de l'état français françois hollande, car l'europe a "montré qu'elle n'était pas capable de prendre des décisions au bon moment". il a cité la défense, la zone euro, l'harmonisation fiscale et sociale comme autant de sujets sur lesquels des groupes de pays doivent être "capables d'aller plus vite, plus loin, sans fermer la porte à qui que ce soit", comme c'est déjà le cas aujourd'hui dans d'autres domaines. d'autres pays, comme la belgique, le luxembourg ou l'espagne ont apporté leur soutien à cette idée. ce scénario n'instaure pas "un nouveau rideau de fer entre l'est et l'ouest", a assuré le président de la commission européenne, jean-claude juncker. "ce n'est pas l'intention", a-t-il assuré à l'adresse des pays de l'est. ces derniers, notamment ceux du groupe de visegrad - hongrie, république tchèque, slovaquie et pologne - s'inquiètent d'être déclassés comme des membres de seconde zone de l'union. "nous n'accepterons jamais de parler d'une europe à plusieurs vitesses", a prévenu la première ministre polonaise beata szydlo, estimant que cela "compromettrait l'intégrité" de l'ue. mme szydlo s'était déjà illustrée la veille en s'opposant à la reconduction dans ses fonctions du président du conseil européen, le polonais donald tusk, considéré comme un ennemi politique par le gouvernement nationaliste et conservateur au pouvoir à varsovie. mais m. tusk a été réélu jeudi à une écrasante majorité, à 27 voix contre une. en représailles, varsovie a refusé d'accepter les conclusions du sommet, qui devaient ponctuer la journée. ces conclusions sur des thèmes variés (immigration, économie, défense ou encore la situation dans les balkans occidentaux), avaient donc été publiées jeudi au nom du président du conseil européen, avec le "soutien de 27 états membres", sans incidence sur la reconduction de tusk jusqu'en novembre 2019. -"chantage"- revenant sur les tensions de la veille, mme szydlo a affirmé vendredi qu'elle avait eu une escarmouche verbale avec le président français françois hollande, qu'elle a accusé de "chantage". "devrais-je prendre au sérieux le chantage d'un président qui a 4 \% de soutien dans les sondages et qui bientôt ne sera plus président ? je ne sais pas", a-t-elle lancé devant la presse, sans préciser la teneur de propos qu'aurait tenus m. hollande. selon des médias polonais, il aurait évoqué la question des fonds structurels massivement octroyés à la pologne par l'ue. mais ni une source européenne, ni l'entourage du président, interrogés par l'afp, n'ont confirmé qu'il y avait eu des mots assimilables à une menace portant sur ces fonds. françois hollande a réaffirmé que l'ue n'était pas "un jeu à somme nulle où l'on calculait ce qu'on obtenait pour examiner ce que l'on consentait à mettre en oeuvre", a affirmé la source européenne. m. tusk, au coeur de la confrontation de jeudi, a tenté vendredi de concilier toutes les sensibilités concernant l'avenir de l'ue: "notre principal objectif devrait être de renforcer notre confiance mutuelle et l'unité à 27", a-t-il dit. le chef de l'exécutif européen, m. juncker, a de son côté adressé un message à londres, qui a promis de notifier son départ de l'ue d'ici la fin du mois: "j'espère qu'un jour viendra où les britanniques remonteront dans le bateau" européen. & 750 & very low & Low & Values & NA & NA & 2017-03-10 & 2017 & 2 & ECO
Frame & v.low & National & 500-1000 & -0.7708104 & -0.8786862 & 0.6041414 & -0.2983228 & 0.6486612 & 0.0 & -0.0275917 & -0.0818024 & Payer & Domestic & Domestic & Domestic & Domestic|ECO & Neutral\\
\addlinespace
France & http://www.lexpress.fr/actualites/1/politique/refugies-le-ps-denonce-l-egoisme-des-pays-refusant-les-quotas-pose-la-question-de-l-aide\_1715988.html & 128 & LExpress.fr & Private/Non-Public & Online and Offline & National & very low = CP mentioned once & Political leverage & Negative & EU + Other country & No myth & NA & NA & NA & NA & NA & NA & NA & NA & France & réfugiés: le ps dénonce "l'égoïsme" des pays refusant les quotas, pose la question de l'aide & 2015-09-15 & fonds structurels & paris - les députés ps jugent "légitime" de poser la question, comme l'a fait l'allemagne d'une réduction de l'aide financière européenne aux pays d'europe centrale qui refusent les quotas de répartition des réfugiés, dont ils dénoncent "l'égoïsme et la cécité". "je veux dénoncer l'égoïsme, la cécité de la hongrie, la république tchèque et la slovaquie", a déclaré mardi lors d'un point presse le porte-parole du groupe ps hugues fourage "ils ont été bien heureux que l'ue leur ouvre ses bras après la chute du mur et je trouve légitime que l'allemagne pose la question de l'aide européenne", a-t-il ajouté. les pays récalcitrants "sont souvent des pays qui reçoivent beaucoup de fonds structurels" européens, a déclaré mardi matin le ministre allemand de l'intérieur, thomas de maizière, trouvant "juste (...) qu'ils reçoivent moins de moyens" financiers de la part de l'europe. les etats membres de l'ue n'ont pas réussi à s'accorder de manière unanime lundi à bruxelles sur un mécanisme de répartition contraignant de 120.000 réfugiés du fait de l'opposition des pays du "groupe de visegrad" (hongrie, pologne, république tchèque et slovaquie). le droit d'asile est "consubstantiel aux valeurs de l'europe" et "la seule réponse possible" à la crise actuelle est "la solidarité", a plaidé m. fourage. "ce n'est pas la solidarité qui provoque l'appel d'air, c'est la guerre", a-t-il insisté. & 247 & very low & Low & Power & NA & NA & 2015-09-15 & 2015 & 1 & POL
Frame & v.low & National & <500 & -0.7708104 & -0.8786862 & 0.6041414 & -0.2983228 & 0.6486612 & 0.0 & -0.0275917 & -0.0818024 & Payer & European & European & European & European|POL & Negative\\
France & http://www.ouest-france.fr/bretagne/quimperle-29300/quimperle-le-rendez-vous-des-femmes-chefs-d-entreprises-4611119/ & 139 & Ouest France & Private/Non-Public & Online and Offline & Regional/Local & very low = CP mentioned once & Social awareness/inclusion & Positive & EU + National + Subnational & No myth & NA & NA & NA & NA & NA & NA & NA & NA & France & quimperlé. le rendez-vous des femmes chefs d'entreprises & 2016-11-14 & fonds social européen & entreprendre au féminin se pose à quimperlé communauté le jeudi 17 novembre pour parler du patrimoine du dirigeant. l'occasion d'un point avec marie boutron, animatrice du réseau. c'est une structure d'accompagnement à la création d'entreprise pour les femmes. nous sommes une association régionale qui agit sur les quatre départements bretons. notre mission est de permettre aux femmes de choisir leur parcours professionnel. elle suit trois axes : la formation à l'émergence de projet, les rencontres réseau qui sont des ateliers de l'entreprenariat et des actions de sensibilisation à l'entreprenariat des femmes. au départ, entre 1999 et 2006, il y a eu une étude du conseil général du finistère sur la création d'entreprises. elle a permis de se rendre compte qu'il y avait autant d'hommes que de femmes qui désiraient créer une entreprise. mais, il se trouvait que sur 10 créateurs, 3 seulement étaient des femmes. en fait, les hommes décrochaient les aides. on s'est rendu compte qu'il y avait des freins à la création d'entreprises par des femmes. le programme européen equal a étudié ces freins et mis en place des actions pour les lever. dix femmes chefs d'entreprises reprennent le flambeau et créent entreprendre au féminin en 2008, pour poursuivre les actions du programme equal. forte de ses bons résultats dans le finistère, elle s'est étendue aux côtes-d'armor, au morbihan et à l'ille-et-vilaine. en 2015, l'association reçoit le trophée national de l'apec, association pour l'emploi des cadres. en 2016, l'association se rapproche des territoires plus ruraux en ouvrant deux antennes supplémentaires et des comités d'animations voient le jour. l'association est née pour dire : oui, les femmes veulent entreprendre. elles ont un projet de vie plutôt qu'un projet de carrière, elles ont une autre relation à l'argent. elles se lancent peut-être vers l'âge de 35 ans dans la création d'entreprise. même en 2016, il est toujours nécessaire d'aider les femmes à entreprendre. on se rend compte que s'il y a autant d'hommes que de femmes à vouloir créer, 30 \% seulement des créations sont le fait des femmes. il faudrait monter à 40 \%. les freins existent toujours, le poids des regards. il y a encore des choses à travailler, donner confiance aux femmes. entreprendre au féminin est encore dans l'air du temps. le réseau tend à perdurer. depuis sa création, entreprendre au féminin a permis la création par des femmes, de 415 entreprises soit plus de 500 emplois. nous avons formé 960 femmes à l'émergence de projet et 2 500 ont été reçues en entretien individuel. aujourd'hui, nous comptons 450 adhérents sur le territoire. nous agissons avec 30 partenaires, acteurs économiques ou collectivités. pour toutes nos actions, nous bénéficions de financements du fonds social européen, de l'état, de collectivités. l'organisation patrimoniale du dirigeant d'entreprise. c'est un angle juridique. en fait, il s'agit de voir comment assurer la pérennité de l'entreprise, comment le conjoint est pris en compte. ce sont des questions qu'il faut se poser. cet atelier sera animé par alain nicolas, consultant diplômé en gestion du patrimoine. notre présidente, christine maurin sera présente aussi. nous accueillerons une douzaine de personnes, pour que le groupe créé des liens et pour plus d'interactivité. jeudi 17 novembre, de 9 h à 12 h, atelier organisation patrimoniale du dirigeant d'entreprise présenté par entreprendre au féminin, à quimperlé communauté. places limitées à 12 personnes. inscription sur le site http://www.entreprendre-au-feminin.net/ eafb-29-atelier-l-organisation & 610 & very low & Low & Socio-Economic & NA & NA & 2016-11-14 & 2016 & 2 & ECO
Frame & v.low & Regional & 500-1000 & -0.7708104 & -0.8786862 & 0.6041414 & -0.2983228 & 0.6486612 & 0.0 & -0.0275917 & -0.0818024 & Payer & Domestic & European & Mixed & Domestic|ECO & Positive\\
France & https://www.la-croix.com/Monde/Europe/LUnion-europeenne-defi-moyens-Brexit-2018-12-13-1200989546 & 133 & La Croix & Private/Non-Public & Online and Offline & National & very low = CP mentioned once & Political leverage & Balanced & EU & No myth & NA & NA & NA & NA & NA & NA & NA & NA & France & l'union européenne au défi de ses moyens, après le brexit & 2018-12-13 & politique de cohésion & le président du conseil européen, donald tusk, lors d'une conférence de presse à bruxelles, en novembre 2018. / john thys/afp rien ne change au budget 2019, avant le grand saut dans le vide du brexit. mercredi 12 décembre, les eurodéputés ont voté la dernière année du cadre financier 2014-2020 avec la contribution britannique. après quoi l'ue devra continuer à fonctionner avec 12 milliards d'euros de moins. se donner les moyens c'est l'un des enjeux du sommet européen des 13 et 14 décembre, où la négociation budgétaire à 27 doit être officiellement lancée pour 2021-2027. les discussions, en réalité, ont déjà lieu depuis un bon moment, avec d'importants désaccords à surmonter dans le club européen. le président du conseil donald tusk dans sa lettre d'invitation des chefs d'état, affiche sa confiance sur le respect du calendrier : " ces derniers mois, le travail a bien avancé, et je propose que nous visions un accord pour l'automne prochain ", a-t-il déclaré à ce sommet de décembre. l'europe cherche à boucler son prochain budget tout le monde ne partage pas son optimisme. mardi 11 décembre, en conseil des affaires générales, le commissaire au budget, günther oettinger n'y est pas allé par quatre chemins " on n'avance pas beaucoup, cela freine des quatre fers ! ". ce dernier condamne depuis l'automne les pays qui ne veulent pas mettre la main à la poche pour compenser le brexit. moins de budget, plus de projets la commission européenne a proposé un budget de 1 279 milliards d'euros sur la période 2021-2027, alors qu'elle avait recommandé 1 033 milliards d'euros pour la période 2014-2020. c'est le prix à payer pour développer le numérique, la recherche, erasmus, la protection des frontières extérieures de l'ue, et surtout la défense, qui à elle seule coûterait 20 milliards d'euros. mais cela implique une augmentation des contributions respectives des états membres (1,1 \% à 1,2 \% du revenu national brut contre 1 \% aujourd'hui). les pays-bas, le danemark, la finlande et l'autriche ont fait part de leurs réticences à cette hausse, tandis que les pays de l'est, tout comme la france et l'allemagne, consentent à faire un effort. vers une pac verte ? l'un des sujets ultra-sensibles du futur budget est le devenir de la politique agricole commune (pac), dont la france est la première bénéficiaire. bruxelles préconise une baisse " d'environ 5 \% " des fonds alloués à la pac (aujourd'hui 37 \% du budget). même chose pour la politique de cohésion de l'ue (35 \% du budget), qui de surcroît sera conditionnée au respect de l'état de droit. paris a d'ores et déjà mis en place une " stratégie d'influence ". le gouvernement français sait qu'elle peut compter sur un socle solide d'alliés pour négocier : l'allemagne, l'irlande, la finlande, l'autriche, mais aussi le portugal, la grèce, la slovénie, ou la république tchèque. la france voudrait mettre en place une agriculture moins dépendante des aides, avec un système de bonus-malus donnant la prime aux exploitations les plus innovantes, ou dont l'effort porte sur une alimentation saine. " mais on est assez esseulés là-dessus ", explique-t-on au ministère de l'agriculture. emmanuel macron peine à rassurer le monde agricole solidarité et état de droit une autre idée explosive, portée par paris et berlin, fait son chemin : couper les fonds de solidarité aux pays qui ne respectent pas les grands principes démocratiques. " le respect de l'état de droit est une condition préalable indispensable à une saine gestion financière et à une mise en œuvre efficace du budget ", avait estimé le président de la commission, jean-claude juncker, lorsqu'il a mis sur la table sa proposition de cadre financier. un mécanisme serait mis en place pour " suspendre, réduire ou restreindre l'accès aux fonds de l'ue d'une manière proportionnée à la nature, à la gravité et à l'étendue des défaillances généralisées de l'état de droit ". il pourrait être déclenché à la majorité de 55 \% des états membres représentant 65 \% de la population de l'ue. bruxelles promet qu'il n'est pas conçu pour cibler tel ou tel pays. mais la pologne et la hongrie se sentent déjà visés. le vice-ministre polonais pour les affaires européennes, konrad szymanski, a prévenu : " nous n'accepterons pas de mécanismes arbitraires qui feront de la gestion des fonds un instrument de pression politique à la demande. " & 756 & very low & Low & Power & NA & NA & 2018-12-13 & 2018 & 3 & POL
Frame & v.low & National & 500-1000 & -0.7708104 & -0.8786862 & 0.6041414 & -0.2983228 & 0.6486612 & 0.0 & -0.0275917 & -0.0818024 & Payer & European & European & European & European|POL & Neutral\\
France & https://www.lepoint.fr/politique/emmanuel-berretta/l-eurogroupe-de-la-derniere-chance-03-12-2018-2276228\_1897.php & 186 & Le Point & Private/Non-Public & Online and Offline & National & very low = CP mentioned once & Institutional bargaining over funding & Factual & Other country & NA & NA & NA & NA & NA & NA & NA & NA & NA & France & l'eurogroupe de la dernière chance & 2018-12-03 & fonds structurels & qui sait vraiment ce qu'olaf scholz a en tête en entrant dans ce dernier eurogroupe utile avant l'ultime conseil européen de l'année ? le ministre des finances allemand a fait tourner les français en bourriques depuis des mois, se montrant tantôt ouvert à une réforme profonde de la zone euro, tantôt plus circonspect, pour ne rien dire de son inventivité pour contourner d'une manière ou d'une autre la création d'une taxe sur le chiffre d'affaires des géants du numérique... c'est un peu perdu par les mille et une circonvolutions de son homologue allemand que bruno le maire entame cette réunion de l'eurogroupe. du côté de la commission, le sentiment est plus net : " les allemands ne veulent rien. " du côté des français, c'est l'autonomie d'olaf scholz vis-à-vis de son administration qui pose question... les chantiers sur la table sont pourtant immenses. il s'agit de savoir si l'union économique et monétaire est capable, par temps calme, de renforcer ses mécanismes de protection de manière à aborder avec plus d'outils entre ses mains la prochaine crise. les propositions de la commission sont sur la table depuis des mois. elles sont censées répondre à quatre faiblesses identifiées. d'abord, une réforme du mécanisme européen de stabilité (mes) pour aider, à titre préventif, un pays en difficulté avant qu'il ne soit coupé des marchés. avec cet outil, la grèce aurait pu être secourue plus tôt en 2008. l'allemagne entend édicter des critères d'éligibilité plus stricts que ceux prônés par la france... ensuite, il s'agit de créer un filet de sécurité (" backstop ") public pour les banques déjà protégées entre elles par le fonds de résolution unique (fru). il s'agirait d'adosser ce fru au mes, ce qui serait une façon de doubler l'enveloppe initiale (60 milliards d'euros). encore faut-il que les prêts non performants qui lestent encore certaines banques européennes soient purgés plus efficacement... enfin, il s'agit de créer un budget de la zone euro (proposition française) et un système de réassurance de l'assurance chômage (proposition d'olaf scholz). sur les deux derniers points, les divergences des pays du nord, pays-bas en tête, sont encore fortes. la philosophie générale des pays scandinaves repose sur un précepte simple : la zone euro sera forte quand tous les états membres qui la composent sauront gérer leurs finances sainement. l'attitude frondeuse de l'italie vis-à-vis des traités budgétaires ne les encourage pas à mettre au pot commun. pas question, donc, de " transferts d'argent " du nord vers le sud... emmanuel macron répond qu'il n'existe pas de zone monétaire qui puisse se passer d'une certaine solidarité entre ses membres. encore faut-il que tous jouent avec la même règle du jeu et coordonnent un tant soit peu leur politique économique... l'italie, en laissant volontairement filer son déficit (132 \% du pib), entrave une intégration plus forte de la zone euro. si l'allemagne s'est ralliée à la création d'un budget de la zone euro à travers la déclaration de meseberg, elle a posé ses conditions : ne seraient aidés que les pays qui auront fait des réformes ; les investissements soutenus doivent être consacrés à la recherche, aux industries d'avenir et ne pas être redondants avec les aides déjà perçues par ailleurs au titre des fonds structurels. par ailleurs, l'allemagne considère que ce budget de la zone euro doit être adopté dans le cadre du budget général de l'union, donc lors du prochain cadre financier multiannuel 2021-2027. berlin plaide pour une adoption rapide, au cours de cette législature finissante, de crainte que les prochaines européennes de mai aboutissent à l'élection d'un parlement improbable... la france, pour l'heure, s'y refuse et repousse l'adoption de ce budget multiannuel à la prochaine législature. & 651 & very low & Low & Power & NA & NA & 2018-12-03 & 2018 & 3 & POL
Frame & v.low & National & 500-1000 & -0.7708104 & -0.8786862 & 0.6041414 & -0.2983228 & 0.6486612 & 0.0 & -0.0275917 & -0.0818024 & Payer & European & European & European & European|POL & Neutral\\
France & http://www.lepoint.fr/economie/zone-euro-l-espagne-et-le-portugal-bientot-en-derapage-budgetaire-11-07-2016-2053798\_28.php & 129 & Le Point & Private/Non-Public & Online and Offline & National & low = CP mentioned more times but NOT important part of story (mainly about others issues) & Political leverage & Factual & EU + National & No myth & NA & NA & NA & NA & NA & NA & NA & NA & France & zone euro : l'espagne et le portugal bientôt en dérapage budgétaire & 2016-07-11 & fonds structurels & cette annonce ouvrira la voie à une procédure inédite de sanctions dans l'histoire de la monnaie unique. les pays ont dix jours pour affirmer leur position. les budgets du portugal et de l'espagne sont au centre de toutes les attentions. mardi 12 juillet, les ministres des finances de la zone euro vont déclarer ces deux pays en dérapage budgétaire. une mesure qui ouvre la voie à une procédure inédite de sanctions dans l'histoire de la monnaie unique. "les membres de la zone euro vont soutenir demain la recommandation de la commission européenne" (qui avait déclaré jeudi l'espagne et le portugal en dérapage budgétaire, ndlr), a annoncé jeroen dijsselbloem, le président de l'eurogroupe, à l'issue d'une réunion des grands argentiers de la monnaie unique, à bruxelles. selon les règles de procédure, chacun des deux pays se prononcera sur l'autre, mais pas sur lui-même. a partir du moment où les ministres de la zone euro, dans le cadre d'une réunion des grands argentiers de l'ue prévue mardi dans la capitale belge, "vont dans notre sens, s'ouvre alors une période de 20 jours" pendant laquelle la commission européenne évaluera les sanctions possibles à l'encontre des deux pays, a précisé le commissaire européen aux affaires économiques, pierre moscovici, lors d'une conférence de presse avec jeroen dijsselbloem. pierre moscovici a rappelé que les amendes possibles était "au maximum de 0,2\%" du produit intérieur brut (pib) et au "minimum zéro". dans le cadre de cette échéance de vingt jours, l'exécutif européen "doit aussi proposer la suspension d'une partie des engagements de versements de fonds structurels européens" à partir de 2017. selon une source européenne, les fonds concernés pour l'an prochain s'élèvent pour l'espagne à 1,3 milliard d'euros et pour le portugal à 500 millions d'euros. "nous allons engager un processus de dialogue avec les ministres" de l'espagne et du portugal, a expliqué pierre moscovici, les incitant à faire valoir "le plus vite possible" leurs arguments pour expliquer pourquoi ils ont dérapé et ce qu'ils veulent faire pour améliorer leur situation budgétaire. les deux pays disposent au maximum de dix jours pour faire parvenir leur position, à compter du constat de dérapage par leurs pairs de la zone euro. en arrivant à bruxelles, le ministre espagnol de l'economie, luis de guindos, s'était montré confiant : "la raison pour laquelle je suis optimiste, c'est le non-sens que supposerait une sanction (imposée) à l'espagne", avait-il expliqué. luis de guindos avait de nouveau dit espérer que son pays - qui a divisé par deux son déficit public entre 2012 et 2015 grâce à d'immenses efforts budgétaires - passerait sous la barre des 3\% du produit intérieur brut (pib) en 2017, rentrant ainsi dans les clous des règles européennes. une fois que la commission européenne aura fait ses propositions de sanctions - probablement décidées lors la réunion de tous ses commissaires le 27 juillet -, les ministres des finances de la zone euro devront à nouveau donner leur feu vert. selon une source européenne, ces derniers pourraient opter pour un aval par procédure écrite, sans avoir besoin d'une nouvelle réunion imprévue, en plein été. en 2015, le déficit public espagnol a atteint 5\% du produit intérieur brut (pib), bien au-delà du seuil du pacte de stabilité (3\% du pib) et des objectifs que lui avait fixés la commission, à 4,2\%. il devrait aussi déraper en 2016, alors que l'espagne n'est pas encore parvenue à former un gouvernement après les élections législatives du 26 juin, précédées de six mois de blocage politique. quant au portugal, il avait affiché un déficit public de 4,4\% du pib l'an passé alors que l'objectif fixé était de repasser sous les 3\%. en 2016, il devrait toutefois rentrer dans les clous. & 649 & low & Low & Power & NA & NA & 2016-07-11 & 2016 & 2 & POL
Frame & low-medium & National & 500-1000 & -0.7708104 & -0.8786862 & 0.6041414 & -0.2983228 & 0.6486612 & 0.0 & -0.0275917 & -0.0818024 & Payer & Domestic & European & Mixed & Domestic|POL & Neutral\\
\addlinespace
France & http://www.lepoint.fr/politique/refugies-le-ps-denonce-l-egoisme-des-pays-refusant-les-quotas-pose-la-question-de-l-aide-15-09-2015-1964971\_20.php\#xtor\%3DRSS-221 & 173 & Le Point & Private/Non-Public & Online and Offline & National & very low = CP mentioned once & Political leverage & Negative & Other country & No myth & NA & NA & NA & NA & NA & NA & NA & NA & France & réfugiés: le ps dénonce "l'égoïsme" des pays refusant les quotas, pose la question de l'aide & 2015-09-15 & fonds structurels & les députés ps jugent "légitime" de poser la question, comme l'a fait l'allemagne d'une réduction de l'aide financière européenne aux pays d'europe centrale qui refusent les quotas de répartition des réfugiés, dont ils dénoncent "l'égoïsme et la cécité". "je veux dénoncer l'égoïsme, la cécité de la hongrie, la république tchèque et la slovaquie", a déclaré mardi lors d'un point presse le porte-parole du groupe ps hugues fourage "ils ont été bien heureux que l'ue leur ouvre ses bras après la chute du mur et je trouve légitime que l'allemagne pose la question de l'aide européenne", a-t-il ajouté. les pays récalcitrants "sont souvent des pays qui reçoivent beaucoup de fonds structurels" européens, a déclaré mardi matin le ministre allemand de l'intérieur, thomas de maizière, trouvant "juste (...) qu'ils reçoivent moins de moyens" financiers de la part de l'europe. les etats membres de l'ue n'ont pas réussi à s'accorder de manière unanime lundi à bruxelles sur un mécanisme de répartition contraignant de 120.000 réfugiés du fait de l'opposition des pays du "groupe de visegrad" (hongrie, pologne, république tchèque et slovaquie). le droit d'asile est "consubstantiel aux valeurs de l'europe" et "la seule réponse possible" à la crise actuelle est "la solidarité", a plaidé m. fourage. "ce n'est pas la solidarité qui provoque l'appel d'air, c'est la guerre", a-t-il insisté. le porte-parole juge que "le repli sur soi n'est pas une solution". "ce n'est pas schengen qui est responsable, c'est de ne pas avoir fini schengen, il faut une politique solidaire qui s'applique à tous" 15/09/2015 14:54:15 - paris (afp) - © 2015 afp cet article vous a plu ? accédez à l'intégralité des contenus du point à partir de 1€ seulement >> & 314 & very low & Low & Power & NA & NA & 2015-09-15 & 2015 & 1 & POL
Frame & v.low & National & <500 & -0.7708104 & -0.8786862 & 0.6041414 & -0.2983228 & 0.6486612 & 0.0 & -0.0275917 & -0.0818024 & Payer & European & European & European & European|POL & Negative\\
France & http://la1ere.francetvinfo.fr/maripasoula-kourou-rup-quels-seront-temps-forts-visite-emmanuel-macron-guyane-524267.html & 158 & guadeloupe 1ère & Public & Online only & Regional/Local & very low = CP mentioned once & Economic development & Factual & EU + National + Subnational & No myth & NA & NA & NA & NA & NA & NA & NA & NA & France & maripasoula, kourou, les rup : quels seront les temps forts de la visite d'emmanuel macron en guyane ? - outre-mer 1ère & 2017-10-26 & politique régionale & pour son premier déplacement officiel outre-mer en tant que chef de l'etat, emmanuel macron se rend deux jours en guyane à partir de ce jeudi 26 octobre. une visite organisée à l'occasion de la 22ème édition de la conférence des présidents des régions ultrapériphériques qui dure toute la semaine et dont la collectivité territoriale de guyane assure la présidence. le dernier passage d'emmanuel macron en guyane remonte à décembre 2016, il était alors candidat "en marche" à l'élection présidentielle. pour cette visite officielle, le chef de l'etat sera accompagné du président de la commission européenne, jean-claude juncker, de la commissaire européenne à la politique régionale, corinne cretu, des ministres des outre-mer, annick girardin, de l'education nationale, jean-michel blanquer (également ancien recteur de l'académie de guyane 2004 à 2006, ndlr). il y aura également la ministre de l'enseignement supérieur, frédérique vidal et le secrétaire d'etat auprès du ministre de la transition écologique, sébastien lecornu. emmanuel macron assistera à la clôture de ce sommet européen, vendredi 27 octobre, en présence des présidents des rup. on compte neuf régions ultrapériphériques : la guyane, la guadeloupe, la martinique, la réunion, saint-martin, et mayotte pour la france, ainsi que les açores et madère pour le portugal, et les îles canaries pour l'espagne. avant cela, emmanuel macron a prévu de se rendre dès son arrivée, jeudi 26 octobre, dans l'ouest de la guyane, à maripasoula. a peine posé à l'aéroport félix eboué, il devrait redécoller immédiatement pour l'ouest de la guyane, jeudi, en fin de matinée (heure locale). maripasoula, plus grande commune de france, se situe sur le fleuve maroni qui sépare la guyane du suriname. aucune route n'existe pour s'y rendre. emmanuel macron survolera la forêt amazonienne, et devrait également observer du ciel les sites d'orpaillages très présents dans cette partie du territoire guyanais. lors d'une précédente visite dans le département en août 2015, emmanuel macron alors ministre de l'economie, s'était rendu sur le site minier de la montagne d'or, où il avait plaidé en faveur d'une augmentation de l'activité aurifère. cette visite du chef de l'etat est un événement pour maripasoula qui avait également accueilli jacques chirac, en décembre 1975. alors premier ministre, jacques chirac venait passer noël à maripasoula. en ce temps-là, valéry giscard d'estaing, président de la république, annonçait que la guyane, "longtemps à la recherche de sa vérité", allait enfin pouvoir "sortir de la somnolence qui l'étreignait". jacques chirac s'était rendu à maripasoula en pirogue. plusieurs clichés de cette visite sont restés célèbres. après jacques chirac, en 2012, nicolas sarkozy, alors président de la république, s'était lui rendu dans les villages de twenké et taluen, sur la commune de maripasoula. accompagné de ses ministres de l'époque, il avait été reçu par les chefs coutumiers. lors de cette visite à maripasoula, emmanuel macron sera notamment accompagné du député guyanais lrem, lénaïck adam, bushinengué, originaire de saint-laurent du maroni et âgé de 25 ans. quarante-deux ans après jacques chirac, cinq ans après nicolas sarkozy et alors que les populations du fleuve se sont faites entendre lors de la mobilisation de mars et avril dernier en guyane, le chef de l'etat devra répondre aux nombreuses attentes des habitants de cette commune enclavée de guyane. manque d'infrastructures sanitaires, de transports, de moyens pour l'éducation : dans cette commune du fleuve riche d'or, il devrait être question d'enjeux économiques, mais aussi environnementaux. a maripasoula, emmanuel macron doit visiter le chantier du nouvel internat avant de rencontrer les élus et de s'exprimer face à la population à la mairie, jeudi en fin de journée. autre étape importante de la visite officielle du président de la république : le centre spatial guyanais. emmanuel macron se rendra à kourou, vendredi 27 octobre, vers 9h (heure locale). le chef de l'etat va découvrir le chantier d'ariane 6. regardez ci-dessous le reportage de guyane 1ère sur le chantier d'ariane 6 : aucun passage n'est en revanche prévu pour emmanuel macron dans la salle "jupiter" d'où sont lancées les fusées ariane et soyouz qui quittent chaque mois la guyane direction l'espace avec à bord plusieurs satellites du monde entier. une salle qu'emmanuel macron, encore ministre de l'economie, avait déjà eu l'occasion de visiter en août 2015. après la clôture du sommet des rup, vendredi après-midi, le chef de l'etat est attendu en fin de journée au commissariat de cayenne. dernière étape de sa visite, il assistera samedi matin aux assises des outre-mer qui se tiennent actuellement dans le département. emmanuel macron devrait repartir pour paris samedi en fin de matinée (heure locale). & 801 & very low & Low & Socio-Economic & NA & NA & 2017-10-26 & 2017 & 2 & ECO
Frame & v.low & Regional & 500-1000 & -0.7708104 & -0.8786862 & 0.6041414 & -0.2983228 & 0.6486612 & 0.0 & -0.0275917 & -0.0818024 & Payer & Domestic & European & Mixed & Domestic|ECO & Neutral\\
France & http://www.lemonde.fr/europe/article/2017/09/29/a-tallinn-l-accueil-poli-des-europeens-aux-idees-de-macron\_5193356\_3214.html & 155 & Le Monde.fr & Private/Non-Public & Online and Offline & National & very low = CP mentioned once & Political leverage & Factual & EU + Other country & 1.Poor regions funded only & NA & NA & NA & NA & NA & NA & NA & NA & France & a tallinn, l'accueil poli des européens aux idées de macron & 2017-09-29 & fonds de cohésion & ils se tiennent par les bras, se regardent dans les yeux en souriant. derrière emmanuel macron et angela merkel, un copieux buffet, probablement celui du swiss hotel de tallinn où les deux dirigeants se sont rencontrés, jeudi 28 septembre au soir, pour un entretien d'une grosse demi-heure. la photographie a été postée sur le compte instagram de la chancelière allemande quelques minutes plus tard, avec un objectif transparent : mettre en scène, de nouveau, leur complicité. le président français s'était déplacé en estonie à un dîner des chefs d'etat et de gouvernement de l'union (suivi, vendredi, d'un sommet consacré au numérique), afin de tester la réaction de ses pairs à son discours fleuve de la sorbonne. sont-ils prêts à embrasser sa vision " à dix ans ", pour refonder une union " souveraine, unie, et démocratique " ? l'accueil qu'ils lui ont réservé était plutôt bon, intéressé. mais pour autant pas enthousiaste. très attendue, la réponse de la chancelière merkel a été une des plus positives. " france is back on stage [la france est de retour sur le devant de la scène] ", lui a-t-elle confié en anglais lors d'un tête-à-tête d'une demi-heure, peu avant le dîner. le discours volontariste du chef de l'etat constitue " une nouvelle impulsion ", " une bonne base de discussion " pour l'avenir de europe, a-t-elle par ailleurs estimé devant la presse. " je suis convaincue que l'europe ne peut pas rester immobile ", a ajouté la chancelière. " eviter le piège de la division " " l'ambiance [de leur rendez-vous] était très bonne. la chancelière avait étudié de près le discours, et a estimé que sur les sujets défense et migration, des avancées rapides pourraient être constatées ", se félicitait une source élyséenne, jeudi soir. " le président et la chancelière ont évoqué les sujets liés à l'intégration plus poussée de la zone euro, en reconnaissant leurs différences. ce ne sera pas un sujet facile [...] mais ils ont voulu ce soir se concentrer sur leurs convergences ", ajoutait-on dans l'entourage du chef de l'etat. pour autant, rien de concret ne pourra avancer sur le front franco-allemand tant qu'à berlin la chancelière ne sera pas parvenue à bâtir une acrobatique coalition gouvernementale avec les verts et les libéraux du fdp. cela pourrait prendre deux mois, a-t-elle confié au président français. les dirigeants de l'union ont eux aussi, pour la plupart, apprécié que paris prenne à nouveau ses responsabilités européennes, après la longue éclipse du quinquennat hollande. lire aussi : le discours de macron sur l'europe diversement apprécié à berlin " tout le monde écoute macron avec intérêt, trouve qu'il fait du bon travail en france ", confie un diplomate bruxellois. pour autant, ses pairs restent prudents. parmi les multiples réformes proposées par le président français à la sorbonne, certaines divisent. xavier bettel, le premier ministre luxembourgeois, s'inquiète de cette suggestion de conditionner l'accès aux fonds de cohésion, destinés aux régions les plus pauvres, à la convergence fiscale. l'idée d'introduire des listes transnationales aux élections européennes n'emballe pas non plus, les petits pays craignant d'y perdre en nombre d'élus à envoyer à strasbourg. seules l'italie et la grèce auraient marqué leur intérêt jusqu'à présent. la méthode recommandée par m. macron, la convocation d'un " groupe de pays de la refondation ", mérite d'être précisée. " il faudrait éviter de tomber dans le piège de la division ", estimait un dirigeant en " off " jeudi soir. " il y a des dirigeants qui sont frileux à l'idée de lancer des débats nationaux sur l'europe en ce moment ", estime un diplomate bruxellois " le risque, avec cette vision à dix ans, c'est d'être trop ambitieux, de casser la dynamique à l'œuvre depuis le sommet de bratislava organisé dans la foulée du brexit. par exemple, on est tout près d'un accord sur le fonds de secours à constituer dans le cadre de l'union bancaire ", estimait ces derniers jours un officiel européen. " attention à éviter les mirages dans le désert ", a twitté la présidente lituanienne, dalia grybauskaité, juste avant le dîner de tallinn. adopter une perspective de long terme, consulter les citoyens dans le cadre de conventions démocratiques comme le préconise m. macron ? " il y a des dirigeants qui sont frileux à l'idée de lancer des débats nationaux sur l'europe en ce moment ", souligne un diplomate bruxellois. ainsi du néerlandais mark rutte, qui n'est toujours pas parvenu à former un gouvernement six mois après les législatives. mais aussi des italiens, déjà en campagne pour leurs élections générales de 2018, ou des autrichiens, qui votent pour leurs députés le 15 octobre. sans parler du gouvernement rajoy, en espagne, piégé par un bras de fer à haut risque avec les indépendantistes catalans. preuve qu'il n'a pas la tête aux réformes européennes : le premier ministre espagnol a préféré rester à madrid jeudi, trois jours avant le référendum organisé par barcelone. a la fin du dîner, donald tusk, le président du conseil européen, s'est quand même engagé à " consulter les dirigeants dans les deux prochaines semaines " sur la méthode à choisir pour poursuivre la réflexion sur l'avenir de l'union. " une dynamique est enclenchée ", s'est-on félicité côté français. & 887 & very low & Low & Power & NA & NA & 2017-09-29 & 2017 & 2 & POL
Frame & v.low & National & 500-1000 & -0.7708104 & -0.8786862 & 0.6041414 & -0.2983228 & 0.6486612 & 0.0 & -0.0275917 & -0.0818024 & Payer & European & European & European & European|POL & Neutral\\
France & http://www.ladepeche.fr/article/2017/10/03/2657608-une-26e-edition-particulierement-passionnante.html & 156 & Ladepeche.fr & Private/Non-Public & Online and Offline & Regional/Local & very low = CP mentioned once & Research \& innovation & Positive & EU + Subnational & No myth & NA & NA & NA & NA & NA & NA & NA & NA & France & une 26e édition particulièrement passionnante & 2017-10-03 & fonds européen de développement régional & les sciences s'exposent, s'animent, s'expliquent, en aveyron comme partout en france, à l'occasion de la 26e édition de la fête de la science, organisée par le ministère de l'éducation nationale, de l'enseignement supérieur et de la recherche et cofinancée par le fonds européen de développement régional. une édition particulièrement riche qui se tiendra cette année du 7 au 15 octobre et dont danièle souyri, présidente de l'association science en aveyron, entourée de nombreux... & 80 & very low & Low & Socio-Economic & NA & NA & 2017-10-03 & 2017 & 2 & ECO
Frame & v.low & Regional & <500 & -0.7708104 & -0.8786862 & 0.6041414 & -0.2983228 & 0.6486612 & 0.0 & -0.0275917 & -0.0818024 & Payer & Domestic & European & Mixed & Domestic|ECO & Positive\\
France & https://www.ladepeche.fr/article/2018/07/13/2835553-politique-chaise-vide-conference-territoires.html & 180 & Ladepeche.fr & Private/Non-Public & Online and Offline & National & very low = CP mentioned once & Institutional bargaining over funding & Factual & National + Subnational & No myth & NA & NA & NA & NA & NA & NA & NA & NA & France & la politique de la chaise vide  à la conférence des territoires & 2018-07-13 & politique de cohésion & l'association des maires de france, celles des régions et des départements ont pratiqué la politique de la chaise vide, hier, lors de la conférence des territoires. "pour mettre d'accord françois baroin, dominique bussereau et hervé morin, il fallait faire fort", sourit un élu lrem. ce député, pourtant proche du chef del'etat s'inquiétait, il y a quelques jours, de la fronde lancée par les présidents des associations des maires de france, des régions de france et des départements qui a aboutie, hier, au boycott de la conférence des territoires, ce rendez-vous semestriel, instauré par emmanuel macron il y a un an. les trois associations déplorent notamment une recentralisation latente et contestent le jeu de dominos provoqué par la refonte de la fiscalité locale. les deux députés représentant les républicains à la conférence - guillaume peltier et annie genevard - ont également boudé la rencontre. en préambule, le premier ministre a donc tenté de tendre la main aux élus locaux en mettant en avant le "lien partenarial" entre l'état et les collectivités. revenant sur les contrats financiers qui encadrent la hausse de la dépense des grandes collectivités - une des pommes de discorde - le locataire de matignon a expliqué : "je suis intimement convaincu que" ces contrats sont "infiniment plus féconds et plus respectueux qu'une diminution brutale et non discutée des dotations". il a cependant reconnu que leur "principe même" avait pu susciter "un peu d'agacement et de désaccord". "au fur et à mesure, nous nous améliorerons collectivement pour, peut-être, dans les années qui viennent, prendre mieux en compte telle ou telle spécificité de telle ou telle collectivité", a-t-il insisté. carte de l'apaisement "nous disons à celles et ceux qui n'ont pas voulu venir, à ceux qui ont pris cette décision pour des positions que l'on doit respecter, mais qui sont parfois de nature très politique, que la porte est ouverte", a ajouté le secrétaire d'état olivier dussopt. le président lr du sénat gérard larcher, a lui aussi joué la carte de l'apaisement en lançant "un appel pour que le dialogue soit restauré entre l'état et les collectivités territoriales". la conférence nationale des territoires d'hier, dont le thème était "europe, cohésion et territoires", évitait pourtant les sujets qui fâchent. elle s'était fixée comme objectif de dégager une position commune sur la politique de cohésion européenne. il s'agit de "peser dans la négociation européenne sur le montant" des aides, qui devrait avoisiner 18 milliards d'euros, "peser sur la doctrine européenne" et "définir ensemble les bons tuyaux pour distribuer ces fonds", a résumé edouard philippe. mais l'absence des régions, qui sont les principales délégataires de la gestion de ces fonds, est préjudiciable. "il n'est jamais bon que des messages divergents émanent de notre pays" même si "ça n'exclut pas de discuter", a fait valoir le ministre de la cohésion des territoires jacques mézard. "je n'ai pas de doute sur le fait que les régions viendront discuter, ce n'est pas possible autrement. les uns et les autres sont liés", a pour sa part assuré la ministre jacqueline gourault. les raisons de la colère depuis l'arrivée d'emmanuel macron au pouvoir, les sujets de tension avec les élus locaux se sont multipliés. la baisse du nombre de contrats aidés a été un élément déclencheur, tout comme la suppression de la taxe d'habitation. la demande du gouvernement de limiter la hausse des dépenses de fonctionnement des collectivités à 1,2 \% par an a donné, aux élus, le sentiment d'être mis sous tutelle. enfin, la mise en place des 80 km/heures sur lla plupart des routes et les fermetures de classes en milieu rural - 207 à la rentrée - sont venues envenimer les choses. & 631 & very low & Low & Power & NA & NA & 2018-07-13 & 2018 & 3 & POL
Frame & v.low & National & 500-1000 & -0.7708104 & -0.8786862 & 0.6041414 & -0.2983228 & 0.6486612 & 0.0 & -0.0275917 & -0.0818024 & Payer & Domestic & Domestic & Domestic & Domestic|POL & Neutral\\
\addlinespace
France & http://www.lepoint.fr/bourse/emploi-la-garantie-jeunes-etendue-a-61-nouveaux-territoires-en-2015-01-12-2014-1885874\_81.php\#xtor\%3DRSS-221 & 119 & Le Point & Private/Non-Public & Online and Offline & National & low = CP mentioned more times but NOT important part of story (mainly about others issues) & Jobs & Factual & EU + National & No myth & NA & NA & NA & NA & NA & NA & NA & NA & France & emploi: la garantie jeunes étendue à 61 nouveaux territoires en 2015 & 2014-12-01 & fonds social européen & la garantie jeunes, dispositif d'accompagnement des jeunes décrocheurs vers l'emploi déjà expérimenté dans 10 territoires, sera étendue à 61 nouveaux territoires courant 2015, a annoncé lundi le ministre du travail françois rebsamen. dix nouveaux territoires (départements, agglomérations, villes...) entreront dans le dispositif au 1er janvier, puis 51 supplémentaires "à partir de la fin du premier trimestre 2015", a détaillé le ministre lors d'une conférence de presse. objectif de m. rebsamen: "que la garantie jeunes concerne, fin 2015, 50.000 jeunes sans emploi, sans formation et sans stage". manuel valls avait fixé lors de la conférence sociale début juillet l'objectif de 100.000 jeunes en 2017. le dispositif est déjà expérimenté dans 10 territoires depuis fin 2013, notamment à marseille, en seine-saint-denis, à la réunion ou dans les vosges. il offre, pour une durée d'un an, un accompagnement renforcé vers l'emploi, des périodes en entreprise et une allocation mensuelle de 450 euros correspondant au rsa sans le forfait logement. la durée peut être prolongée de 6 mois, au cas par cas. le projet de budget pour 2015 prévoit une enveloppe de 164 millions d'euros, dont 31 millions d'euros de fonds européens, destinée au dispositif. a terme, l'europe devrait prendre en charge 50\% du dispositif. le cofinancement européen provient de la déclinaison française de l'initiative pour l'emploi des jeunes (iej), dotée d'une enveloppe de 310 millions d'euros (2014-2015) et abondée par le fonds social européen (fse) à hauteur de 310 millions supplémentaires. outre de la garantie jeunes, ces 620 millions d'euros cofinancent le service civique, des accompagnements vers la création d'entreprise, une plateforme de lutte contre le décrochage et le déploiement de 700 conseillers pôle emploi spécialisés dans l'accompagnement renforcé des jeunes. 01/12/2014 14:14:36 - paris, 1 déc 2014 (afp) - © 2014 afp & 312 & low & Low & Socio-Economic & NA & NA & 2014-12-01 & 2014 & 1 & ECO
Frame & low-medium & National & <500 & -0.7708104 & -0.8786862 & 0.6041414 & -0.2983228 & 0.6486612 & 0.0 & -0.0275917 & -0.0818024 & Payer & Domestic & European & Mixed & Domestic|ECO & Neutral\\
France & http://www.ouest-france.fr/grece-leurope-met-disposition-deux-milliards-daide-3269004 & 124 & Ouest France & Private/Non-Public & Online and Offline & Regional/Local & high = CP is most important issue in story (can also cover other issues) & Solidarity to poor countries/regions & Positive & EU & No myth & NA & NA & NA & NA & NA & NA & NA & NA & France & grèce. l'europe met à disposition deux milliards d'aide & 2015-03-20 & fonds structurels & ces fonds " ne viendront pas renflouer les caisses de l'etat grec ", a précisé le président de la commission européenne, jean-claude juncker, alors qu'athènes a un urgent besoin de liquidités. en revanche, cette aide pourra être utilisée " pour renforcer les efforts en faveur de la croissance et de la cohésion sociale ", notamment " en réponse au problème massif du chômage des jeunes ", a-t-il expliqué au cours d'un point de presse à l'issue du sommet européen à bruxelles. " crise humanitaire " reconnaissant que la grèce souffrait d'une " crise humanitaire ", il a expliqué avoir présenté au premier ministre grec alexis tsipras la semaine dernière un document sur les possibilités pour que le pays absorbe plus efficacement les fonds structurels européens. " nous avons mis en place une équipe technique à bruxelles pour aider les autorités grecques à pouvoir absorber ces fonds disponibles et notre "task force" à athènes a été instruite de travailler en bonne intelligence avec les autorités grecques ", a poursuivi m. juncker. il a rappelé que la grèce bénéficiait déjà " d'un traitement privilégié " car elle peut bénéficier de taux de cofinancement de 5 \% des projets soutenus par ces fonds européens, notamment sociaux, contre 15 \% en moyenne dans l'ue. " je voudrais que les autorités grecques, avec l'appui des services de la commission, puissent investir cet argent dans des secteurs propices en matière de perspectives de croissance ", a insisté jean-claude juncker, citant notamment les pme. & 241 & high & High & Values & NA & NA & 2015-03-20 & 2015 & 1 & ECO
Frame & high-very high & Regional & <500 & -0.7708104 & -0.8786862 & 0.6041414 & -0.2983228 & 0.6486612 & 0.0 & -0.0275917 & -0.0818024 & Payer & European & European & European & European|ECO & Positive\\
France & https://www.ladepeche.fr/article/2018/02/23/2748100-ue-les-27-divises-sur-leur-futur-budget-post-brexit.html & 168 & Ladepeche.fr & Private/Non-Public & Online and Offline & Regional/Local & very low = CP mentioned once & Financial burden & Factual & EU & No myth & NA & NA & NA & NA & NA & NA & NA & NA & France & ue: les 27 divisés sur leur futur budget post-brexit & 2018-02-23 & politique de cohésion & les dirigeants européens ont fait le constat vendredi de leurs divisions sur les choix budgétaires auxquels ils seront confrontés après 2020, quand les recettes seront grevées par le départ du royaume-uni. cette "première discussion politique", en amont des propositions officielles que doit publier la commission européenne début mai, a été "moins conflictuelle que je n'avais pensé", a toutefois estimé jean-claude juncker, le chef de l'exécutif européen, à l'issue de la rencontre. l'ue est confrontée à une double difficulté. elle doit faire face à des défis inédits et coûteux en matière de protection des frontières, de défense ou de migration. et doit les financer alors que le budget de l'ue va perdre avec le brexit l'un de ses principaux contributeurs, à hauteur de plus de 10 milliards d'euros par an. "s'il n'y a pas d'argent dans la maison, l'amour s'envole par la porte", a lancé malicieusement m. juncker. "tous les dirigeants sont prêts à travailler à la modernisation du budget et de ses programmes. et beaucoup sont prêts à contribuer plus au budget après 2020", a assuré donald tusk, le président du conseil européen, instance qui regroupe les dirigeants de l'ue. l'allemagne d'angela merkel s'y engage dans l'accord de coalition du futur gouvernement. le français emmanuel macron a de son côté assuré que la france était "prête à ce que nous ayons un budget européen en expansion". a l'inverse, plusieurs pays, tous des contributeurs nets (donnant plus qu'ils ne reçoivent) ont prévenu qu'ils refuseraient de payer plus, comme les pays-bas, la suède, le danemark ou l'autriche. le budget pluriannuel de l'ue fixe des plafonds de dépenses sur plusieurs années: le cadre actuel, qui court de 2014 à 2020, prévoit ainsi des engagements autour de 1.000 milliards d'euros. des choix à faire il faudra faire des choix, a prévenu bruxelles. l'essentiel du budget de l'ue -- environ 70\% -- est pour l'instant consacré aux piliers historiques que sont la politique de cohésion, pour permettre aux régions les plus pauvres de rattraper leur retard, et la politique agricole commune (pac). le brexit et l'émergence de nouvelles priorités "ne doivent en aucun cas conduire à un écrasement des politiques historiques" de l'ue, a fait valoir emmanuel macron. a la surprise de donald tusk, le débat sur la conditionnalité, un mécanisme qui lierait par exemple le respect des valeurs démocratiques de l'ue ou l'accueil de réfugiés au versement de fonds européens, n'a pas donné lieu à une forte controverse. "la seule chose, c'est que cela doit être construit sur des critères objectifs", a-t-il souligné. ce qui promet d'être difficile à traduire dans un langage juridique. offensif sur ce sujet, m. macron a estimé que le budget européen ne devait "plus servir à financer des gouvernements qui ne respectent pas les droits fondamentaux tels qu'ils figurent dans nos traités", ni "à financer des politiques dont la stratégie est d'organiser du dumping fiscal ou social". "cela n'est pas forcément à voir de facon négative, cela peut être vu de facon positive", a noté de son côté angela merkel, qui a défendu l'idée que les pays faisant davantage pour accueillir des réfugiés devaient recevoir des fonds européens en conséquence. 'spitzenkandidat' le départ du royaume-uni soulève plus généralement un débat institutionnel au sein de l'ue, à l'approche des élections européennes de 2019, qui seront suivies de la formation d'une nouvelle commission européenne. les 27 ont apporté leur soutien à une nouvelle répartition au sein du parlement européen, après le départ des 73 eurodéputés britanniques, qui réduirait le nombre total de sièges de 751 à 705. la recomposition de l'hémicycle sera officiellement adoptée en juin. ils ont par ailleurs refusé de se restreindre dans leur choix du futur candidat à la présidence de la commission européenne, qu'ils soumettront au vote des eurodéputés pour succéder à jean-claude juncker à l'automne 2019. "le traité est très clair sur la compétence autonome du conseil européen", a asséné donald tusk, réponse directe à la récente mise en garde du parlement européen. ce dernier exige que le candidat que les dirigeants choisiront soit une "tête de liste" sélectionnée par les partis politiques européens (ou "spitzenkandidat" selon le terme allemand qui s'est imposé dans les institutions). le candidat sélectionné par le conseil doit ensuite recueillir une majorité de suffrage de la part des eurodéputés, a rappelé m. juncker. "ce qui en termes démocratiques traduit tout de même une hiérarchie qu'il ne faudrait pas que nous oublions trop rapidement", a-t-il ajouté, estimant que la procédure avait bien fonctionné pour lui en 2014. & 798 & very low & Low & Values & NA & NA & 2018-02-23 & 2018 & 3 & ECO
Frame & v.low & Regional & 500-1000 & -0.7708104 & -0.8786862 & 0.6041414 & -0.2983228 & 0.6486612 & 0.0 & -0.0275917 & -0.0818024 & Payer & European & European & European & European|ECO & Neutral\\
France & http://www.ladepeche.fr/article/2014/12/29/2020020-140-millions-en-un-an-pour-le-gers.html & 102 & Ladepeche.fr & NA & Online and Offline & Regional/Local & high = CP is most important issue in story (can also cover other issues) & Economic development & Positive & EU + Subnational & No myth & Social awareness/inclusion & Positive & EU + Subnational & No myth & Bureaucracy & Balanced & EU & NA & France & 140 millions en un an pour le gers & 2014-12-29 & fonds européen de développement régional & il y a du changement du côté des financements européens. ce qui met en lumière, si c'était nécessaire, que les aides de l'europe sont omniprésentes dans le gers. en 2013, plus de 140 millions y ont été versés par l'union européenne. qui a dit que l'europe était lointaine ? ses subsides, en tout cas, sont palpables au quotidien... en 2013 (derniers chiffres connus), ce ne sont pas moins de 141 millions d'euros qui ont été versés par l'union européenne dans le département. une manne qui se divise comme suit : plus de 139 millions d'euros pour la seule politique agricole commune, 377 789 euros pour le fonds européen de développement régional et 1 205 029 euros pour le fonds social européen (pour l'emploi et la réinsertion). (1) alors, soulignent les services de la préfecture, "les fonds européens sont devenus incontournables, en complément des fonds publics nationaux. sans ces derniers, le gers souffrirait davantage de la crise économique et financière, dans un contexte où l'état et les collectivités ont des marges financières d'intervention particulièrement étroites." lucie da rocha, directrice des plates-formes wimoov' du gers et des hautes-pyrénées, ne dit pas le contraire : "nous avons le fonds social européen depuis 2011, parce qu'on amène des solutions très concrètes en termes de mobilité, ce qui favorise l'accès à l'emploi. sans ce fonds, on ne se serait jamais autant développé, on n'aurait jamais pu accompagner autant de personnes." 233 en 2014 dans le département. reste que des changements se font jour, la région est devenue l'intermédiaire majeur depuis le 1er décembre pour le feader (2), qui finance la politique agricole commune. et les critères d'obtention des fonds évoluent avec la mise en place concrète de la stratégie européenne 2014-2020. ce qui a notamment des conséquences sur les agriculteurs (lire ci-contre). "c'est clair qu'il faut aussi s'adapter aux règles de financement, yves barbaste, directeur de l'association départementale pour l'aménagement des structures et exploitations agricoles (adasea). sur un budget de 550 000 €, l'europe en finance 150 à 200 000 €. mais l'idée, c'est de construire une cohérence entre les critères des différents financeurs." car, souligne mme da rocha, "l'europe finance en complément des autres financeurs publics, et maximum autant que les autres cumulés." mais les dossiers de demande et surtout de justification des activités nécessitent un temps impressionnant", assure la directrice. "et on est payé après justificatif de nos actions, pour des actions que nous avons donc déjà réalisées". ce qui implique d'avoir les reins très solides, avant même de connaître la réponse aux demandes d'aides européennes... 1. source préfecture du gers et http ://cartobenef.asp-public.fr/cartobenef/liste\_benef.php?nivgeo=reg\&codgeo=73 2 deux fonds existent en plus du feader. le feder, fonds européen de développement régional, finance des investissements d'entreprises créatrices d'emplois, des infrastructures. dans le gers, il a notamment financé l'unité de méthanisation du grand auch. le fse, ou fonds social européen, vise à favoriser la création et le retour vers l'emploi, et à combattre la pauvreté. dans le département, il a notamment soutenu les activités de l'atelier gersois d'innovation. pac : la perte le ministre de l'agriculture l'affiche haut : la nouvelle politique agricole commune (pac) serait "plus favorable à l'élevage". pourtant, relève bernard malabirade, président départemental du syndicat agricole fdsea, "nous avons de nombreux éleveurs dans le gers. mais, avec la nouvelle pac, nous perdrons entre 14 et 15 millions d'euros par an." illogique, donc ? la question a été posée aux différentes institutions distribuant les aides européennes dans le gers. sans réponse... "il faudra bien que le ministère assume !" s'exclame m.malabirade. & 630 & high & High & Socio-Economic & Socio-Economic & Governance & 2014-12-29 & 2014 & 1 & ECO
Frame & high-very high & Regional & 500-1000 & -0.7708104 & -0.8786862 & 0.6041414 & -0.2983228 & 0.6486612 & 0.0 & -0.0275917 & -0.0818024 & Payer & Domestic & European & Mixed & Domestic|ECO & Positive\\
France & http://www.lexpress.fr/actualite/monde/europe/les-reformes-reclamees-par-cameron-pour-eviter-un-no-britannique-a-l-ue\_1683996.html & 121 & LExpress.fr & Private/Non-Public & Online and Offline & National & very low = CP mentioned once & Financial burden & Factual & Other country & No myth & NA & NA & NA & NA & NA & NA & NA & NA & France & les réformes réclamées par cameron pour éviter un "no" britannique à l'ue & 2015-05-28 & fonds structurels & david cameron entame une tournée des capitales européenne ce jeudi pour réclamer des réformes, dans la perspective du référendum prévu d'ici 2017 sur la sortie -ou non- de son pays de l'union européenne. voici la liste des ses principales attentes. david cameron met les bouchées doubles sur l'union européenne. son gouvernement a officialisé, mercredi, l'organisation d'ici à la fin 2017 d'un référendum sur "le maintien ou pas" du royaume-uni dans l'ue, à l'occasion du discours de la reine présentant les grandes lignes de son programme législatif. a peine ce discours prononcé, le chef du gouvernement a entamé une série de visites de ses partenaires européens. objectif, les convaincre de la nécessité de réformes, pour empêcher ses compatriotes de dire "non". le dirigeant britannique a laissé à dessein planer le flou sur ses exigences. "david cameron a tout intérêt à ne pas formuler d'exigences trop précises dont le rejet risquerait de décevoir la frange eurosceptique de son opinion", explique pieter cleppe, du think tank libéral open europe. mieux vaudrait pour lui présenter un compromis final de mesures comme un succès. david cameron devrait formuler des exigences susceptibles de trouver un écho favorable dans plusieurs autres pays européens. il a d'ailleurs commencé sa tournée par le danemark et les pays-bas supposés sensibles à des réformes libérales. puis il dîne ce jeudi à l'élysée avec le président françois hollande avant de rencontrer la chancelière allemande angela merkel, vendredi, après un détour par la pologne. david cameron souhaite limiter les droits sociaux accordés aux ressortissants des 28 membres de l'ue installés dans le royaume. il s'agirait notamment de restreindre les allocations familiales ou le droit au chômage pour les nouveaux venus, dans un délai de deux à quatre ans après leur arrivée dans le pays. le premier ministre britannique a martelé pendant la campagne des législatives son rejet du "tourisme social". les étrangers ne viennent pas, prétend-il, pour chercher du travail mais pour bénéficier des prestations sociales du royaume. peu importe que cette notion soit hautement controversée, les migrants contribuant, selon plusieurs études, plus à la croissance qu'ils ne tirent profit des aides sociales. david cameron pourrait également demander de durcir les procédures de regroupement familial. les conservateurs britanniques espèrent flatter les opinions publiques du continent dont une partie est sensibles aux discours anti-immigration des mouvements populistes. ces exigences posent toutefois un problème: elles portent atteinte au principe d'égalité des citoyens de l'ue dans tous les pays membres. le premier ministre entend réformer les institutions de l'ue afin de restituer plus de pouvoirs aux pays membres. londres souhaiterait supprimer le préambule du traité de rome qui prévoit "une union toujours plus étroite entre les peuples européens", ou pour le moins pouvoir s'en exempter. il voudrait aussi renforcer le pouvoir des communes face à bruxelles. "là où les parlements ont obtenu un "carton jaune", cameron souhaiterait qu'ils disposent du droit de délivrer un "carton rouge", explique pieter cleppe. le traité sur l'union européenne prévoit déjà qu'un parlement national puisse demander des éclaircissements lorsqu'une proposition ne respecte pas le principe de subsidiarité qui permet de retarder la mise en oeuvre d'une loi. les conservateurs britanniques voudraient aller plus loin, en accordant aux parlements nationaux un véritable droit de veto vis-à-vis des nouvelles lois européennes. londres souhaiterait la mise en place d'un mécanisme accordant un droit de regard sur les décisions monétaires et fiscales de l'eurogroupe qui touchent au marché unique. auto-exclus de la zone euro, les britanniques veulent en effet préserver les intérêts des pays extérieurs à la zone euro, en particulier en matière de régulation financière. le premier ministre devrait insister sur l'allègement des régulations, dans le but de "doper la croissance", croit savoir, pieter cleppe: étendre la libéralisation des services, du secteur de l'e-commerce et du marché de l'énergie. compte-tenu de l'opposition de certains pays -avec en mémoire la "directive bolkestein" accusée de favoriser le dumping social, une telle évolution pourrait se limiter aux seuls pays volontaires. david cameron devrait dans le même temps défendre une réduction des dépenses budgétaires de l'ue. conscient qu'il est très difficile de toucher à la politique agricole commune (pac), il pourrait, en revanche, estime pieter cleppe, proposer de réduire les fonds structurels, les subventions attribuées aux régions pauvres des pays riches. il espère pouvoir convaincre plusieurs de ses partenaires à l'heure où la rigueur s'impose. & 759 & very low & Low & Values & NA & NA & 2015-05-28 & 2015 & 1 & ECO
Frame & v.low & National & 500-1000 & -0.7708104 & -0.8786862 & 0.6041414 & -0.2983228 & 0.6486612 & 0.0 & -0.0275917 & -0.0818024 & Payer & European & European & European & European|ECO & Neutral\\
\addlinespace
France & http://www.ledauphine.com/politique/2018/03/01/mobilisation-regionale-pour-les-aides-europeennes & 106 & ledauphine.com & Private/Non-Public & Online and Offline & Regional/Local & very high = CP is most important issue + CP is mentioned in title/headline & Institutional bargaining over funding & Positive & EU + National + Subnational & No myth & Economic development & Positive & EU + National + Subnational & No myth & NA & NA & NA & NA & France & mobilisation régionale pour les aides européennes & 2018-03-02 & politique de cohésion & oui, " je suis optimiste ", a confié hier à notre journal la commissaire européenne corina cretu, qui faisait étape à paris dans un tour d'europe de défense de son budget des aides régionales. optimiste, alors que le prochain budget de l'union (sur la période 2021-2027) devra à la fois absorber le choc financier du départ du royaume-uni, et financer de nouvelles politiques comme la défense ou le numérique. or les fonds de la politique de cohésion, qui incluent les aides régionales, représentent le deuxième budget de l'union (35 \% du total), juste derrière la politique agricole commune (37 \%), également menacée. sur le budget en cours (2014-2020), ces subventions totalisent 15 milliards d'euros pour la france. & 120 & very high & High & Power & Socio-Economic & NA & 2018-03-02 & 2018 & 3 & POL
Frame & high-very high & Regional & <500 & -0.7708104 & -0.8786862 & 0.6041414 & -0.2983228 & 0.6486612 & 0.0 & -0.0275917 & -0.0818024 & Payer & Domestic & European & Mixed & Domestic|POL & Positive\\
France & http://www.lepoint.fr/economie/derapage-budgetaire-pas-de-sanctions-pour-l-espagne-et-le-portugal-09-08-2016-2059987\_28.php & 136 & Le Point & Private/Non-Public & Online and Offline & National & low = CP mentioned more times but NOT important part of story (mainly about others issues) & Political leverage & Factual & EU + Other country & No myth & NA & NA & NA & NA & NA & NA & NA & NA & France & dérapage budgétaire : pas de sanctions pour l'espagne et le portugal & 2016-08-09 & fonds structurels & si elles avaient été validées, les amendes auraient pu atteindre jusqu'à 0,2 \% du produit intérieur brut (pib) de chacun de ces deux pays. " le conseil accepte de ne pas imposer d'amendes à l'espagne et au portugal pour avoir échoué à mettre en place des mesures efficaces afin de corriger leurs déficits excessifs ", annonce un communiqué du conseil de l'union européenne. la proposition de la commission européenne de ne pas sanctionner l'espagne et le portugal malgré leurs dérapages budgétaires a été approuvée par les ministres des finances de l'union européenne, a annoncé mardi 9 août le conseil de l'ue. le conseil avait dix jours, soit jusqu'à lundi minuit, pour s'opposer à la proposition de la commission européenne, formulée le 27 juillet, sans quoi elle était formellement adoptée. la commission avait proposé d'annuler les amendes des deux pays, des mesures qui " auraient été contre-productives à un moment où les peuples doutent de l'europe " après le brexit, avait expliqué le commissaire européen aux affaires économiques, le français pierre moscovici. les amendes auraient pu atteindre jusqu'à 0,2 \% du produit intérieur brut (pib) de chacun de ces deux pays. en 2015, le déficit public espagnol avait atteint 5,1 \% du pib, un chiffre bien au-dessus du plafond de 3 \% fixé par le pacte de stabilité et des objectifs de la commission de 4,2 \%. quant au portugal, il avait affiché un déficit public de 4,4 \% du pib l'an passé, alors que l'objectif fixé était de repasser sous les 3 \%. outre sa clémence concernant les amendes, l'exécutif européen avait accordé deux années supplémentaires à l'espagne pour faire tomber son déficit sous les 3 \%, soit jusqu'en 2018. pour ce pays, qui n'a toujours pas formé de nouveau gouvernement depuis les législatives du 26 juin, la commission européenne avait recommandé la trajectoire budgétaire suivante : un déficit de 4,6 \% du pib en 2016, 3,1 \% en 2017 et 2,2 \% en 2018. concernant le portugal, elle avait tablé sur un déficit de 2,5 \% du pib en 2016. deux objectifs que le conseil de l'ue, instance qui représente les gouvernements des états membres, a qualifiés mardi de " crédibles ". " des mesures efficaces doivent être prises d'ici le 15 octobre ", date à laquelle les deux pays devront aussi présenter un projet de plan budgétaire, a précisé le conseil. la commission européenne s'est réjouie mardi de cette décision, qui témoigne d'une " application intelligente du pacte de stabilité ", selon pierre moscovici. outre ces amendes, la commission européenne est dans l'obligation de proposer une suspension totale ou partielle des engagements des fonds structurels pour l'espagne et le portugal. elle compte entamer " un dialogue structuré " après les vacances d'été avec le parlement européen sur cette question. & 472 & low & Low & Power & NA & NA & 2016-08-09 & 2016 & 2 & POL
Frame & low-medium & National & <500 & -0.7708104 & -0.8786862 & 0.6041414 & -0.2983228 & 0.6486612 & 0.0 & -0.0275917 & -0.0818024 & Payer & European & European & European & European|POL & Neutral\\
France & https://www.lesechos.fr/monde/europe/0301627872287-etat-de-droit-bruxelles-veut-toucher-varsovie-et-budapest-au-portefeuille-2173025.php\#Xtor=AD-6000 & 117 & LesEchos.fr & Private/Non-Public & Online and Offline & National & medium = CP is important part of story & Political leverage & Factual & EU + Other country & No myth & NA & NA & NA & NA & NA & NA & NA & NA & France & etat de droit : bruxelles veut toucher varsovie et budapest au portefeuille & 2018-05-01 & fonds de cohésion & les négociations sur le budget européen offrent aux pays contributeurs nets un moyen de pression sans précédent sur la pologne et la hongrie. puisque les traités ne disposent pas d'instrument permettant de forcer un etat à rentrer dans le droit chemin en matière d'etat de droit, les européens s'apprêtent à utiliser une arme plus informelle, et potentiellement plus efficace, dans leur bataille avec la pologne et la hongrie : le chantage au porte-monnaie. alors que varsovie avait passé des mois à moquer les mises en garde de bruxelles au sujet de ses réformes judiciaires, le ton a subitement changé au cours des dernières semaines. frans timmermans, le premier vice-président de la commission européenne qui s'est rendu à varsovie début avril, a même fait état d'un esprit constructif sans précédent. " pression gigantesque " a en croire plusieurs diplomates, cet apparent revirement découle pour partie de la négociation du prochain budget pluriannuel. " les négociations financières ont mis une pression gigantesque et toujours croissante sur ces pays ", résume une source européenne de premier plan. cette pression prend plusieurs formes. la plus visible, et la plus humiliante potentiellement pour varsovie ou budapest, concerne l'idée de conditionner les aides financières au respect de l'etat de droit. ce principe n'est pas simple à concrétiser, et plusieurs idées ont été évoquées à ce sujet. günther oettinger, le commissaire en charge du budget, a récemment évoqué la nécessité de pouvoir " récupérer les fonds via un tribunal " en cas de corruption avérée ou de détournement. majorité qualifiée le " financial times " croit savoir qu'à cette fin, bruxelles va proposer un mécanisme permettant de couper les financements en cas de risque pour l'etat de droit. la commission pourrait même proposer un mécanisme par lequel il faudrait une majorité qualifiée d'etats-membres pour empêcher la décision de bloquer les fonds... le soutien indéfectible des deux enfants terribles de l'union l'un pour l'autre deviendrait alors inopérant. fonds de cohésion mais en plus de cela, les fonds de cohésion , une manne pour ces etats, vont très certainement être revus à la baisse en volume. surtout, les règles présidant à leur allocation vont être affinées. en particulier, il est question de ne plus se contenter du pib par habitant pour décider des régions à soutenir financièrement. pourraient également intervenir des critères relatifs à la diversité socio-économique dans une région ou encore au coût d'accueil des migrants - ce qui peut également être interprété comme une sanction à l'égard de ces etats. avant même le début des hostilités, varsovie et budapest savent que l'ère de la grande générosité européenne touche à sa fin. & 444 & medium & Medium & Power & NA & NA & 2018-05-01 & 2018 & 3 & POL
Frame & low-medium & National & <500 & -0.7708104 & -0.8786862 & 0.6041414 & -0.2983228 & 0.6486612 & 0.0 & -0.0275917 & -0.0818024 & Payer & European & European & European & European|POL & Neutral\\
France & https://actu.fr/hauts-de-france/gamaches\_80373/des-travaux-le-respect-la-biodiversite-letang-lepinoy-gamaches\_20435851.html & 188 & actu.fr & Private/Non-Public & Online and Offline & Regional/Local & very low = CP mentioned once & Environment/green/low-carbon & Positive & Subnational & No myth & NA & NA & NA & NA & NA & NA & NA & NA & France & des travaux pour le respect de la biodiversité à l'étang de l'épinoy, à gamaches & 2018-12-26 & fonds européen de développement régional & avec sept chemins de randonnées et des dizaines d'hectares d'étangs, la nature est prépondérante à gamaches (somme). elle fait la joie des plus de deux cents pêcheurs, mais aussi des promeneurs, et plus largement des habitants du secteur. sa préservation est donc l'affaire de tous. consciente des enjeux, la municipalité y travaille depuis de nombreuses années. dernier dossier en date : celui de l'étang de l'épinoy. ce mercredi 19 décembre 2018, les différents acteurs du projet se sont réunis sur place pour procéder à l'inauguration des réalisations. samuel biau, de la fédération de la somme pour la pêche et la protection du milieu aquatique, en a rappelé les différentes étapes : au départ, il y avait ici une peupleraie, dont les feuilles sont nocives pour l'eau de l'étang, et qui a donc été abattue par la commune. cette dernière nous a ensuite sollicitée, pour savoir quelles plantations mettre au bord de l'eau. quoi, et comment, tout en respectant l'activité pêche. avec le centre régional de la propriété forestière hauts-de-france, différents financements ont été recherchés, et obtenus de la part du fonds européen de développement régional, ou encore de la commune bien sûr. pour la création d'une ripisylve, ou ensemble de formations boisées, buissonnantes et herbacées présentes sur les rives d'un cours d'eau, les différents acteurs du projet ont décidé de faire appel aux élèves de l'école d'horticulture de la maison familiale et rurale d'yzengremer. nicolas, leur encadrant, explique : " pour les jeunes, cela est très intéressant car c'est pédagogique et concret aussi ". les plantations ont été réalisées sur une journée, en février 2017. depuis, samuel biau et bruno balligand, ingénieur forestier chargé de mission ripisylve au cnpf, en suivent l'évolution. " une dizaine d'espèces différentes a été plantée, dont une espèce test : des peupliers noirs de la loire " indique bruno balligand. " c'est l'occasion de montrer plus encore que des espèces naturelles plutôt issues du sud de la france peuvent reconquérir le nord ". en ce mois de décembre, un panneau explicatif a été posé, marquant en quelque sorte l'aboutissement du projet. mais il reste encore à faire des aménagements pour faciliter l'accès au site, avec du tout-venant pour la création d'un chemin. des travaux prévus par la ville de gamaches au cours du premier semestre 2019. quant aux plantations, elles vont encore garder leurs protections autour du tronc pendant trois ou quatre ans, et atteindront leur taille adulte d'ici une dizaine d'années. l'ingénieur forestier du cnpf annonce : l'été, on aura alors une belle et grande barrière végétale, et non linéaire, offrant la possibilité de circuler entre les éléments plantés. enfin, le maire daniel destruel " espère que la population de gamaches et des environs viendra là se promener et constater que la ville agit pour la biodiversité ". des propos appuyés par michel blanchard, président de la fédération départementale de la pêche, très attaché aux étangs de la commune. & 503 & very low & Low & Socio-Economic & NA & NA & 2018-12-26 & 2018 & 3 & ECO
Frame & v.low & Regional & 500-1000 & -0.7708104 & -0.8786862 & 0.6041414 & -0.2983228 & 0.6486612 & 0.0 & -0.0275917 & -0.0818024 & Payer & Domestic & Domestic & Domestic & Domestic|ECO & Positive\\
France & https://fr.euronews.com/2019/02/21/corina-cretu?utm\_source=feedburner\&utm\_medium=feed\&utm\_campaign=Feed\%253a\%2Beuronews\%252ffr\%252fnews\%2B\%28euronews\%2B-\%2Bnews\%2B-\%2Bfr\%29\&utm\_content=FeedBurner & 114 & euronews & Private/Non-Public & Online only & National & high = CP is most important issue in story (can also cover other issues) & Solidarity to poor countries/regions & Positive & EU & No myth & Institutional bargaining over funding & Factual & EU & No myth & Fraud/Corruption & Positive & EU & No myth & France & corina cretu : "les prochaines élections sont le plus grand test pour l'europe" & 2019-02-24 & fonds de cohésion & environ un tiers du budget de l'union européenne est consacré à la politique régionale. l'objectif : réduire les disparités de revenus, de ressources et d'opportunités entre les pays membres. une grande partie des fonds de cohésion sert à financer des projets dans les régions les plus pauvres d'europe. the global conversation vous emmène à bruxelles à la rencontre de corina cretu, commissaire européenne à la politique régionale. pour euronews, elle évoque les accomplissements et les défis à venir. sandor zsiros, euronews : madame la commissaire, vous êtes à bruxelles depuis plus de 4 ans maintenant. etes-vous parvenue à combler les écarts entre l'est et l'ouest, entre les régions riches et les régions plus pauvres ? corina cretu : d'après une de nos études, de nombreuses régions ont très bien réussi à revenir vers la croissance et l'emploi. tandis que certains pays sont encore bloqués dans la crise. nous en sommes à la neuvième année consécutive de croissance économique en europe, et c'est positif. mais en même temps, nous avons de nouveaux défis à relever. il faut aussi gérer la crise migratoire, également les problèmes liés au terrorisme et à la défense... des sujets qui ne concernaient pas l'ue auparavant. nous devons donc prendre en compte toute cette situation et faire tout notre possible pour combler les lacunes. après le brexit, nous aurons moins d'argent à distribuer, notamment pour les fonds de cohésion. quels conseils pourriez-vous donner aux dirigeants régionaux ? quels types de projets devraient-ils proposer afin d'obtenir des fonds européens ? au total, le budget de l'ue sera réduit de 10\%, ce qui constituera un nouveau grand défi. nous regrettons beaucoup le départ du royaume-uni, mais nous devons respecter la volonté des citoyens britanniques. malgré ces difficultés, nous avons réussi à élaborer la plus grande enveloppe jamais proposée en terme de politique de cohésion, soit 373 milliards d'euros. nous proposons aux états membres et aux régions de se concentrer sur les secteurs clés, comme l'innovation et l'efficacité énergétique pour les petites et moyennes entreprises. c'est ce qui fait la force de l'ensemble de l'économie européenne actuellement. il faut aussi faire très attention au niveau social. nous devons faire de gros efforts pour que les citoyens sachent ce que l'europe fait pour eux. on connaît tous l'idée reçue selon laquelle tout ce qui est mauvais vient de bruxelles, et tout ce qui est bien vient des maires et des gouvernements locaux, je pense que cette idée doit cesser. parce que nous serons tous perdants si nous persistons à ne pas reconnaître ce que l'union européenne fait pour les citoyens. le parlement européen a approuvé une proposition visant à lier les paiements européens aux normes de l'état de droit. qu'est-ce que cela va entraîner comme changements ? et selon vous, y a-t-il un risque que des personnes soient privées de fonds européens en raison d'erreurs commises par des hommes politiques ? \_je ne comprends pas vraiment pourquoi certains états membres considèrent cette proposition comme une ennemie. car tout le monde assure n'avoir aucune tolérance pour la fraude et la corruption. et justement, cette mesure vise à protéger l'argent des contribuables européens. par exemple, lorsque l'office européen de lutte antifraude (olaf) est sollicitée sur une affaire, nous suspendons immédiatement les fonds jusqu'à ce qu'à la résolution de cette affaire. il faut se mettre à la place des gens, on ne peut pas voir l'union européenne comme un distributeur d'argent ou un guichet de banque. il s'agit de solidarité partagée. il faut être ensemble dans les bons moments, mais aussi dans les périodes de secousses. \_ donc l'office européen de lutte antifraude ne peut émettre que des recommandations, mais pensez-vous qu'il faudrait une organisation pour superviser et contrôler ces questions ? une intervention du parquet européen par exemple ? je pense vraiment que c'est un pas en avant, la commission cherche depuis longtemps à protéger le budget de l'ue. la création d'un bureau du procureur européen d'ici à 2020 va dans ce sens et marque le début d'une nouvelle ère en terme de lutte contre la fraude. il y aura désormais un nouvel organisme qui sera chargé de contrôler la manière dont l'argent est dépensé. nous travaillons en étroite collaboration avec les états membres et les auditeurs de la commission. je pense vraiment que cette nouvelle institution renforcera notre capacité à lutter contre la fraude des fonds européens et à protéger l'argent des contribuables. nous sommes à moins de cent jours des élections européennes. a quoi vous vous attendez ? est-ce qu'il y aura une union européenne plus fragmentée, ou au contraire plus unie ? les populistes seront-ils au rendez-vous ? il est évident que nous sommes dans une phase émergente pour les mouvements anti-européens et populistes. mes services, au sein de la direction générale de la politique régionale et urbaine, ont d'ailleurs réalisé une étude à ce sujet, baptisée " la géographie du mécontentement européen ". cette analyse montre que le vote anti-européen est motivé par une combinaison de facteurs : le déclin économique et industriel, le faible niveau d'éducation - ce qui est essentiel - et enfin le chômage. a mon avis, les élections représentent le test le plus important de ces dernières années pour l'europe. il ne s'agit pas d'idéologies politiques, mais de pro et d'anti-européens. \_tous les deux, vous aujourd'hui journaliste et moi aujourd'hui commissaire, nous nous trouvions de l'autre côté du rideau de fer il y a trente ans. nous ne pouvions pas imaginer que nous nous retrouverions ici. donc je pense que c'est notre devoir de défendre le projet européen, qui est unique sur notre planète, et nous devons nous souvenir de notre passé, le transmettre à la jeune génération car leur avenir est en jeu. \_ & 999 & high & High & Values & Power & Governance & 2019-02-24 & 2019 & 3 & ECO
Frame & high-very high & National & 500-1000 & -0.7708104 & -0.8786862 & 0.6041414 & -0.2983228 & 0.6486612 & 0.0 & -0.0275917 & -0.0818024 & Payer & European & European & European & European|ECO & Positive\\
\addlinespace
France & http://bruxelles.blogs.liberation.fr/2015/09/20/Grece/ & 130 & bruxelles.blogs.liberation.fr & Private/Non-Public & Online and Offline & National & very low = CP mentioned once & Political leverage & Negative & EU + Other country & No myth & NA & NA & NA & NA & NA & NA & NA & NA & France & how some families monitor greece & 2015-09-21 & fonds structurels & vardis vardinoyannis il y a des noms que l'on prononce à voix basse en grèce. ceux des oligarques qui contrôlent à la fois l'état et l'économie du pays et qui les mettent en coupe réglée. en grèce, on les connait, on en parle en privé d'un air gourmand, mais de là à dénoncer publiquement ce système quasi mafieux, il y a un pas qui est rarement franchi, les grands médias étant sous le contrôle direct ou indirect de ces familles et la plupart des politiques leur devant leur carrière... certes, syriza a fait de la lutte contre ce système l'un des axes de son programme, mais en six mois, il n'a curieusement pas trouvé le temps de s'y attaquer. s'il est réélu ce dimanche, peut-être le fera-t-il, mais nombreux sont ceux qui en doutent. une anecdote révélatrice sur la loi du silence qui règne en grèce. récemment, j'ai eu un débat télévisé avec l'une mes consoeurs grecques au cours duquel elle a développé l'argumentaire habituel de " la grèce victime des européens et de la finance ". le débat a été vif, mon analyse étant que la grèce doit son échec à elle-même et à personne d'autre. à l'issue du plateau, celle-ci m'a dit qu'en réalité, j'avais raison et elle m'a invité à enquêter sur le système oligarchique. surpris, je lui ai demandé pourquoi elle ne le faisait pas : " parce que je tiens à garder mon travail ", m'a-t-elle répondu. dans le cadre du documentaire que je prépare (" grèce, le jour d'après " qui sera diffusé le 20 octobre sur arte), j'ai voulu interviewer des oligarques, ce qui a beaucoup faire rire sur place, ceux-ci n'ayant pas l'habitude de répondre aux questions des journalistes. j'ai alors cherché des grecs prêts à dénoncer ce système. je me suis heurté à un véritable mur. finalement, le journaliste d'investigation nikolas leontopoulos, qui fait l'objet de poursuites judiciaires de la part des oligarques n'ayant pas aimé qu'on s'intéresse à leurs affaires, a accepté de me parler. rendez-vous a été pris dans un parc public d'athènes. voici cet entretien. comment fonctionne le système oligarchique ? nous avons une expression pour le décrire en grèce : nous parlons du " triangle du péché " ou du " triangle du pouvoir ". en réalité, c'est plutôt un carré : le premier côté est l'élite entrepreneuriale, le second, les banques, le troisième, les médias et le quatrième, le monde politique. ceux qui possèdent le pouvoir entrepreneurial sont propriétaires des principaux médias et sont actionnaires des banques et en même temps entretiennent des rapports incestueux avec le pouvoir politique. qui sont ces oligarques ? il s'agit de cinq familles pour l'essentiel ou de vingt familles, si l'on veut agrandir le cercle de cette élite entrepreneuriale. plus précisément ? il vaut mieux ne pas nommer ces familles, car celles que je ne citerais pas seraient d'une certaine manière vexées de ne pas en faire partie. c'est une pirouette... bon. les deux familles les plus puissantes -je parle exclusivement de la puissance économique et non de la corruption - sont les familles vardinoyannis (qui contrôle l'industrie pétrolière, nda) et la famille latsis (transport maritime, immobilier, etc., nda). la meilleure façon de mesurer le pouvoir des oligarques, c'est d'examiner séparément les différents domaines. par exemple, dans celui de l'énergie et du pétrole, deux familles le contrôle. la construction est le domaine d'une famille tout comme l'immobilier. ou encore, deux familles détiennent une position dominante dans l'activité financière. mais, elles ne sont pas seules à exercer ce contrôle, elles le font en coopération avec des entreprises étrangères. en vérité, ce que nous appelons oligarchie en grèce ne pourrait pas exister dans la plupart des cas sans la coopération d'une entreprise le plus souvent européenne - française ou allemande. c'est-à-dire ? en fait, ces familles sont des médiateurs. le système fonctionne de la façon suivante : une grande entreprise étrangère coopère avec une une famille locale qui a des liens avec le pouvoir politique afin d'obtenir un marché public. autrement dit, ce système oligarchique est international : sans la présence de l'entreprise étrangère, ce modèle ne pourrait pas exister. un très bon exemple est celui des jeux olympiques de 2004 qui sont à l'origine de l'augmentation de la dette grecque. tous les travaux publics qui ont été faits pour les jeux, et dont plusieurs sont entachés de corruption, obéissaient au même modèle : d'un côté, une entreprise étrangère, de l'autre côté, une entreprise grecque et l'état grec. le rôle de l'entreprise grecque se résumait à jouer de son rapport privilégié avec le pouvoir politique. l'investisseur véritable, au moins pour 50 \% de chaque chantier, était une grande entreprise multinationale de france (bouygues ou vinci), d'allemagne (hochtief), d'espagne (acs), etc.. c'est de cette manière que le système fonctionne depuis les 30 dernières années. spyros latsis et le pouvoir politique ? le pouvoir politique est, dans une grande mesure, dépendant des intérêts entrepreneuriaux. il y a une véritable loi du silence autour de ces oligarques. c'est vrai et cela concerne autant les médias grecs que les médias étrangers. il a fallu la crise de la dette pour que leur rôle sorte enfin de l'ombre. pour les médias, ce système a longtemps été conçu comme un moteur de croissance et de prospérité. peut-on comparer ce système oligarchique à la mafia italienne ? non. mais il y a quelques similitudes : tout comme la mafia vend de la protection, les médias grecs, possédés par les oligarques, protègent les intérêts entrepreneuriaux. ainsi, lors du referendum du 5 juillet, alors que le peuple était vraiment divisé, tous les médias privés, sans aucune exception, ont mené une bataille à la limite du fanatisme en faveur du " oui ", car cela correspondait clairement aux intérêts des oligarques. ces familles qui contrôlent la grèce sont-elles toujours les mêmes ? c'est un système qui se renouvelle d'une période historique à une autre. les grandes familles qui contrôlent le pays remontent aux années 80'. des années 50 aux années 80, c'était d'autres familles. ce système est-il consubstantiel à la grèce ? non, c'est même le contraire. historiquement, la grèce n'a jamais eu un pouvoir central fort. cela explique l'absence de confiance que les citoyens ont vis-à-vis de l'état. la grèce est un pays décentralisé, pour des raisons historiques et géographiques, avec de petites villes, de petites communautés dans les montagnes et les îles, qui avaient une grande autonomie. le système oligarchique est un renversement complet de ce modèle. à partir du moment où un centre puissant est apparu, il a entrainé la création d'élites entrepreneuriales autour de lui qui se sont opposées à l'activité et à la créativité de ceux qui ne font pas partie de ce centre. l'union européenne a-t-elle lutté contre ce système ? au contraire. ainsi, en 2005, le gouvernement conservateur de karamanlis a voté une loi interdisant à une entreprise (y compris les membres de la famille possédant cette entreprise) susceptible de participer à un marché public de posséder en même temps une entreprise médiatique. cela était le premier véritable effort du pouvoir grec de lutter contre l'oligarchie. mais la commission a jugé que cette loi était contraire au droit européen. au lieu de demander une transformation de la loi ou d'aider d'une manière ou d'une autre le gouvernement dans sa lutte contre la corruption et l'oligarchie, la commission a menacé le gouvernement grec de ne plus verser les fonds structurels. le gouvernement a été obligé d'abroger cette loi. aujourd'hui encore, il est déplaisant de constater qu'après cinq ans de contrôle total par la troïka, aucune mesure n'a été proposée pour lutter contre ceux qui possèdent le pouvoir dans ce pays alors les retraités, les gens simples, les salariés ont souffert des réformes. syriza s'est engagé à lutter contre les oligarques, mais jusqu'à présent il n'a rien fait. une des raisons principales de la victoire de syriza, c'est la lutte contre la corruption. mais il est exact qu'il n'a pas fait grand-chose pour l'instant, en grande partie parce qu'il a été occupé par les négociations avec la zone euro sur le programme d'assistance financière. n.b.: on peut ajouter aux familles citées par nikolas leontopoulos, les alafouzos (armateurs), les melissanidis (pétrole, loterie), les makarinis (transport maritime, etc.) ou les bobolas (btp, autoroute, traitement des déchets) ou les capelouzos (énergie, gestion des aéroports). toutes ces familles possèdent les médias grecs et surtout, sont actionnaires des banques (comme la famille latsis). & 1478 & very low & Low & Power & NA & NA & 2015-09-21 & 2015 & 1 & POL
Frame & v.low & National & +1000 & -0.7708104 & -0.8786862 & 0.6041414 & -0.2983228 & 0.6486612 & 0.0 & -0.0275917 & -0.0818024 & Payer & European & European & European & European|POL & Negative\\
France & http://www.lefigaro.fr/flash-eco/2018/05/02/97002-20180502FILWWW00172-budget-de-l-ue-bruxelles-demande-des-coupes-de-5-environ-de-la-politique-agricole-commune-pac.php & 171 & Le Figaro.fr & Private/Non-Public & Online and Offline & National & medium = CP is important part of story & Financial burden & Factual & EU & No myth & NA & NA & NA & NA & NA & NA & NA & NA & France & budget de l'ue: bruxelles demande des coupes "de 5\% environ" de la politique agricole commune (pac) & 2018-05-02 & politique de cohésion & la commission européenne a plaidé mercredi pour une baisse "d'environ 5\%" des fonds alloués à la politique agricole commune (pac) et à la politique de cohésion de l'ue, consacrée aux régions les plus pauvres, dans une proposition de budget pour la période 2021-2027. il s'agit d'une "une réduction modérée du financement de la politique agricole commune et de la politique de cohésion, de 5\% environ dans les deux cas, afin de tenir compte de la nouvelle réalité d'une union à 27", a fait valoir la commission dans un document de présentation. & 97 & medium & Medium & Values & NA & NA & 2018-05-02 & 2018 & 3 & ECO
Frame & low-medium & National & <500 & -0.7708104 & -0.8786862 & 0.6041414 & -0.2983228 & 0.6486612 & 0.0 & -0.0275917 & -0.0818024 & Payer & European & European & European & European|ECO & Neutral\\
France & http://www.france24.com/fr/20180502-budget-pac-france-juge-inacceptables-propositions-bruxelles & 152 & France 24 & Public & Online only & National & very low = CP mentioned once & Solidarity to poor countries/regions & Factual & EU & 1.Poor regions funded only & NA & NA & NA & NA & NA & NA & NA & NA & France & budget de la pac: la france juge "inacceptables" les propositions de bruxelles & 2018-05-02 & politique de cohésion & le gouvernement français a jugé mercredi "inacceptables" les propositions budgétaires de bruxelles pour la politique agricole commune (pac), qui prévoient notamment une baisse "d'environ 5\%" des fonds alloués pour la période 2021-2027. "une telle baisse, drastique, massive et aveugle, est simplement inenvisageable" et "la france ne pourra accepter aucune baisse de revenu direct pour les agriculteurs", indique le ministère de l'agriculture dans un communiqué. "cette part que l'on va enlever aux agriculteurs va remettre en cause la viabilité de leurs exploitations", a souligné le ministre, stéphane travert, lors d'un point de presse. il a rappelé que cette décision arrive au moment où la france travaille "sur la question du revenu des agriculteurs et sur la question d'une alimentation plus sûre, plus saine et plus durable ", dans le cadre des etats généraux de l'alimentation. ces propositions ne constituent toutefois que le point de départ des négociations qui vont s'engager au niveau européen, et le ministre a exprimé le souhait de "retrouver une base de discussion pour porter un budget à la hauteur des ambitions de l'agriculture européenne". "je vais dès à présent prendre attache avec mes collègues européens: je rencontre ce soir le ministre irlandais, j'aurai demain matin au téléphone ma collègue espagnole, et ma collègue allemande, pour faire le point sur ces sujets". le gouvernement défend "une modernisation et une simplification de la pac pour protéger les agriculteurs face aux aléas climatiques et à la volatilité des marchés mondiaux, libérer le développement des entreprises agricoles et agroalimentaires, et accompagner la transition environnementale", selon le ministère. "l'objectif est de revenir à la table des négociations", a assuré le ministre, citant comme prochaine échéance une réunion informelle des ministres de l'agriculture à sofia début juin. la commission européenne a plaidé mercredi pour une baisse "d'environ 5\%" des fonds alloués à la politique agricole commune (pac) et à la politique de cohésion de l'ue, consacrée aux régions les plus pauvres, dans une proposition de budget pour la période 2021-2027. selon l'entourage du ministre toutefois, une baisse de 5\% en euros courants pourrait représenter une baisse en euros constants beaucoup plus importante. c'est également le constat de la présidente de la fnsea, christiane lambert, qui parle d'une "grande déception": "on dit que c'est -5\% mais c'est -10\% avec l'inflation. face aux nouvelles priorités, c'est la politique agricole qui trinque alors qu'on lui demande toujours plus: qu'elle soit plus verte, qu'elle monte en gamme et qu'elle lutte contre le changement climatique, c'est inexplicable!" si la coordination rurale partage la position du ministre de l'agriculture qualifiant les propositions de la commission d'inacceptables, "son désaccord n'en est pas moins profond avec la position globale du gouvernement français qui s'est jusqu'ici borné à réclamer le maintien du budget antérieur de la pac au lieu de dénoncer sa dérive", indique pour sa part le deuxième syndicat agricole français dans un communiqué. & 504 & very low & Low & Values & NA & NA & 2018-05-02 & 2018 & 3 & ECO
Frame & v.low & National & 500-1000 & -0.7708104 & -0.8786862 & 0.6041414 & -0.2983228 & 0.6486612 & 0.0 & -0.0275917 & -0.0818024 & Payer & European & European & European & European|ECO & Neutral\\
France & http://www.lesechos.fr/economie-france/social/0203980393035-la-garantie-jeunes-etendue-en-2015-1070091.php?xtor=RSS-71 & 122 & LesEchos.fr & Private/Non-Public & Online and Offline & National & medium = CP is important part of story & Jobs & Factual & EU + National & No myth & NA & NA & NA & NA & NA & NA & NA & NA & France & la garantie jeunes étendue en 2015 & 2014-12-01 & fonds social européen & 61 nouveaux territoires vont expérimenter l'an prochain la garantie jeunes pour les 18-25 ans en grande difficulté. entre 18 et 25 ans, ni en emploi, ni en formation, ni en études. c'est à ces jeunes en grande difficulté que s'adresse la garantie jeunes. ce dispositif allie accompagnement social, professionnel et aide financière : une allocation garantissant un revenu de quelque 450 euros pour aider à reprendre pied professionnellement. la france est en pointe sur ce programme cofinancé par la commission européenne. la mesure a commencé il y a un an à être expérimentée dans 10 territoires, notamment à marseille et en seine-saint-denis auprès de 10.000 jeunes. il était prévu qu'elle s'appliquerait dans dix nouveaux territoires au 1er janvier prochain. ce lundi, le ministre du travail, françois rebsamen, a annoncé que 51 autres vont s'y ajouter " à partir de la fin du premier trimestre 2015 ", en marge d'une rencontre avec la commissaire européenne pour l'emploi, la belge marianne thyssen. 50.000 jeunes concernés fin 2015 l'objectif, a précisé l'ancien maire de dijon, est " que la garantie jeunes concerne, fin 2015, 50.000 jeunes sans emploi, sans formation et sans stage ". s'il est atteint, la moitié du chemin aura été parcouru par rapport à la cible (100.000 jeunes en 2017) fixée par manuel valls lors de la dernière conférence sociale , cet été. une enveloppe de 164 millions d'euros est affectée au financement de la garantie jeunes dans le projet de budget pour 2015. au total, 19 \% proviendra de fonds européens dédiés à l'accompagnement de ces jeunes sans emploi, sans formation et sorti des études, soit 31 millions d'euros. a terme, le financement communautaire représentera 50 \% du coût de la garantie jeunes. il sera pris sur l'enveloppe de 310 millions d'euros affectés à l'initiative pour l'emploi des jeunes par l'europe auxquels s'ajoutera la même somme prélevée sur le fonds social européen. cette somme totale de 620 millions d'euros contribuera aussi au financement du service civique, des accompagnements vers la création d'entreprise, d'une plateforme de lutte contre le décrochage et de 700 conseillers de pôle emploi spécialisés dans l'accompagnement renforcé des jeunes en difficulté. le ministère ne dispose pour l'instant d'aucunes données d'évaluation des premières garanties jeunes mises en place cette année. il faudra attendre quelques mois pour en avoir. & 407 & medium & Medium & Socio-Economic & NA & NA & 2014-12-01 & 2014 & 1 & ECO
Frame & low-medium & National & <500 & -0.7708104 & -0.8786862 & 0.6041414 & -0.2983228 & 0.6486612 & 0.0 & -0.0275917 & -0.0818024 & Payer & Domestic & European & Mixed & Domestic|ECO & Neutral\\
France & https://www.20minutes.fr/societe/2265051-20180502-budget-pac-bruxelles-veut-baisser-aide-5-inacceptable-france & 174 & 20minutes & Private/Non-Public & Online and Offline & National & very low = CP mentioned once & Financial burden & Factual & EU & 1.Poor regions funded only & NA & NA & NA & NA & NA & NA & NA & NA & France & budget de la pac: bruxelles veut baisser l'aide de 5\%, "inacceptable" pour la france & 2018-05-02 & politique de cohésion & agriculture l'union européenne, qui est en train de revoir ses budgets, envisage de baisser l'aide aux agriculteurs de 5\%... " inacceptables ". le gouvernement français a rejeté ce mercredi les propositions budgétaires de bruxelles pour la politique agricole commune (pac), qui prévoient une baisse " d'environ 5 \% " des fonds alloués pour la période 2021-2027. "je défendrai sans relâche 1budget \#pac à la hauteur ds défis de notre agriculture" la ἞b἟7 refuse la proposition \#eu pic.twitter.com/2bk3agtnxg -- stephane travert (@sttravert) may 2, 2018 " une telle baisse, drastique, massive et aveugle, est simplement inenvisageable " et " la france ne pourra accepter aucune baisse de revenu direct pour les agriculteurs ", a réagi le ministère de l'agriculture dans un communiqué. " cette part que l'on va enlever aux agriculteurs va remettre en cause la viabilité de leurs exploitations ", a souligné le ministre, stéphane travert, lors d'un point presse. il a rappelé que cette décision arrive au moment où la france travaille " sur la question du revenu des agriculteurs et sur la question d'une alimentation plus sûre, plus saine et plus durable ", dans le cadre des etats généraux de l'alimentation. ces propositions ne constituent toutefois que le point de départ des négociations qui vont s'engager au niveau européen, et le ministre a exprimé le souhait de " retrouver une base de discussion pour porter un budget à la hauteur des ambitions de l'agriculture européenne ". nous proposons un budget européen de 1 135 mds€ d'engagements financiers sur la période 2021 - 2027, soit 1,11\% du revenu national brut de l'ue27. une fois l'inflation prise en compte, cela représenterait 1 279mds€ d'engagements \#eubudget pic.twitter.com/f4dw1hhvvs -- commission européenne ἞a἟a (@uefrance) may 2, 2018 " je vais dès à présent prendre attache avec mes collègues européens : je rencontre ce soir le ministre irlandais, j'aurai demain matin au téléphone ma collègue espagnole, et ma collègue allemande, pour faire le point sur ces sujets ". le gouvernement défend " une modernisation et une simplification de la pac pour protéger les agriculteurs face aux aléas climatiques et à la volatilité des marchés mondiaux, libérer le développement des entreprises agricoles et agroalimentaires, et accompagner la transition environnementale ", selon le ministère. " l'objectif est de revenir à la table des négociations ", a assuré le ministre, citant comme prochaine échéance une réunion informelle des ministres de l'agriculture à sofia début juin. la commission européenne a plaidé mercredi pour une baisse " d'environ 5 \% " des fonds alloués à la pac et à la politique de cohésion de l'ue, consacrée aux régions les plus pauvres, dans une proposition de budget pour la période 2021-2027. selon l'entourage du ministre toutefois, une baisse de 5 \% en euros courants pourrait représenter une baisse en euros constants beaucoup plus importante. c'est également le constat de la présidente de la fnsea, christiane lambert, qui parle d'une " grande déception " : " on dit que c'est -5 \% mais c'est -10 \% avec l'inflation. face aux nouvelles priorités, c'est la politique agricole qui trinque alors qu'on lui demande toujours plus : qu'elle soit plus verte, qu'elle monte en gamme et qu'elle lutte contre le changement climatique, c'est inexplicable ! " & 537 & very low & Low & Values & NA & NA & 2018-05-02 & 2018 & 3 & ECO
Frame & v.low & National & 500-1000 & -0.7708104 & -0.8786862 & 0.6041414 & -0.2983228 & 0.6486612 & 0.0 & -0.0275917 & -0.0818024 & Payer & European & European & European & European|ECO & Neutral\\
\addlinespace
France & http://www.france24.com/fr/20180110-ue-bruxelles-veut-une-taxe-plastiques-budgets-post-brexit & 160 & France 24 & Public & Online only & National & very low = CP mentioned once & Financial burden & Factual & EU + Other country & No myth & NA & NA & NA & NA & NA & NA & NA & NA & France & ue: bruxelles veut une taxe sur les plastiques pour ses budgets post-brexit & 2018-01-10 & politique de cohésion & la commission européenne a proposé mercredi l'instauration d'une taxe sur les plastiques pour nourrir les budgets de l'ue, qui vont pâtir de la perte de la contribution britannique après le brexit tout en faisant face à de nouvelles dépenses. l'exécutif européen "envisage une taxe sur les plastiques comme une nouvelle source de recettes pour le budget de l'ue et pour réduire les déchets", a indiqué devant la presse le commissaire chargé du budget, günther oettinger. le commissaire allemand n'a pas donné plus de détails sur les contours de cette taxe ou sur les gains attendus, rappelant que la commission publierait en mai des propositions détaillées pour le prochain "cadre financier pluriannuel" couvrant la période après 2020. le cadre actuel couvre la période 2014-2020 et, malgré son départ en 2019, le royaume-uni s'est engagé à verser sa part lors d'une période de transition jusqu'à 2020 inclus. mais l'ue devra ensuite faire sans sa contribution nette, évaluée par la commission entre 12 et 14 milliards d'euros par an. l'exécutif européen propose aux 27 pays qui resteront dans l'ue de combler ce "trou" pour moitié par la recherche de nouvelles ressources et pour moitié par des mesures d'économies, notamment sur les subventions de la politique agricole commune (pac) et de la politique de cohésion, deux domaines représentant ensemble plus de deux tiers du budget de l'ue. de nouvelles politiques européennes, comme la défense et la gestion des frontières, vont créer de nouveaux besoins "d'une dizaine de milliards d'euros" par an, a par ailleurs estimé m. oettinger, proposant de les financer à 80\% par de nouvelles ressources. outre la taxe sur le plastique, la commission souhaite que l'ue dispose d'une autre nouvelle source propre de revenus au travers de la taxation des échanges de quotas de carbone, qui bénéficie actuellement aux etats membres. côté recettes, l'exécutif européen demande surtout une hausse des contributions nationales au budget de l'ue, actuellement plafonnées à 1\% du pib des etats membres. "je dirais entre 1,1 et 1,2\%", a dit mercredi m. oettinger. les dirigeants des pays de l'ue auront une première occasion d'échanger entre eux sur les budgets post-brexit lors d'un sommet informel le 23 février à bruxelles. les décisions entre etats membres sur le budget de l'ue sont prises à l'unanimité. elles sont marqués traditionnellement par un bras de fer entre les gros contributeurs comme l'allemagne, peu enclins à donner davantage, et les principaux bénéficiaires de fonds européens, inquiets à l'idée de recevoir moins. & 440 & very low & Low & Values & NA & NA & 2018-01-10 & 2018 & 3 & ECO
Frame & v.low & National & <500 & -0.7708104 & -0.8786862 & 0.6041414 & -0.2983228 & 0.6486612 & 0.0 & -0.0275917 & -0.0818024 & Payer & European & European & European & European|ECO & Neutral\\
France & https://www.laprovence.com/article/france-monde/4952447/budget-de-lue-coups-de-rabot-en-vue-apres-le-brexit.html & 109 & LaProvence.com & Private/Non-Public & Online and Offline & Regional/Local & low = CP mentioned more times but NOT important part of story (mainly about others issues) & Financial burden & Balanced & EU + National & No myth & Political leverage & Factual & EU + Other country & No myth & NA & NA & NA & NA & France & budget de l'ue: coups de rabot en vue après le brexit & 2018-05-02 & fonds de cohésion & bruxelles - bruxelles dévoile mercredi son plan pour bâtir les budgets post-brexit de l'ue, avec des propositions détonantes comme de couper dans les politiques agricole et régionale ou de conditionner le versement de fonds européens au respect de l'etat de droit. après des mois de préparation et de ballons d'essai, la commission va mettre sur la table un cocktail d'économies et de nouvelles ressources pour que l'union ait les moyens des ambitions affichées pour sa nouvelle vie à 27, sans le royaume-uni. l'exécutif européen veut que les tractations entre etats membres et parlement européen pour la période 2021-2027 soient bouclées avant les prochaines élections européennes, soit moins de deux mois après le divorce avec les britanniques prévu le 30 mars 2019. "ce genre de négociations prend normalement deux ans", souligne une source diplomatique, perplexe face à ce calendrier, alors que l'équation du "cadre financier pluriannuel" (cfp) de l'ue n'a jamais semblé aussi complexe. selon les estimations de bruxelles, le départ du royaume-uni va laisser un "trou" annuel de 12 à 14 milliards d'euros après 2020 -- dernière année de contribution de londres malgré un brexit programmé en cours d'année précédente. la rupture avec ce "contributeur net" tombe d'autant plus mal que l'union européenne cherche à financer de nouvelles politiques, en matière de défense ou de migration notamment, sans renoncer aux "anciennes". ce qui nécessiterait un budget plus important que celui de 1.000 milliards d'euros fixé pour la période 2014-2020. -la pac visée- le commissaire au budget, l'allemand günther oettinger, veut que les 27 acceptent un budget au-delà de la limite actuelle de 1\% du revenu national brut (rnb) cumulé des etats membres. il faudrait "entre 1,1 et 1,2\%", a-t-il récemment estimé. "il va falloir faire des coupes", a prévenu m. oettinger en visant la politique agricole commune (pac) et la politique de cohésion pour les régions les plus en retard économiquement, deux domaines représentant respectivement 37\% et 35\% du budget de l'ue. la commission proposera "des réductions modérées", "en-dessous de dix pour cent", selon une source européenne. elles seront néanmoins difficiles à accepter, en particulier en france, dont les agriculteurs sont les principaux bénéficiaires des aides directes de la pac. paris est prête à défendre une "réforme assez substantielle", mais "le filet de sécurité indispensable des aides directes pour les agriculteurs ne peut pas être affecté", prévient une source diplomatique. les pays de l'est sont eux déjà vent debout face aux coupes dans les fonds de cohésion, dont ils sont les principaux destinataires, et qui pourraient par ailleurs être en partie réorientés vers d'autres pays connaissant un fort chômage des jeunes ou des "fractures territoriales". la pologne et la hongrie sont d'autant plus sur la défensive qu'elles se sentent visées par un autre projet de la commission, qui veut lier versement de fonds européens et respect de l'etat de droit. -'pression politique'- plusieurs pays réclament ce mécanisme pour tirer les leçons du bras de fer infructueux entre bruxelles et le gouvernement ultra-conservateur polonais, qui est accusé notamment de menacer l'indépendance de sa justice. face à la lourdeur de la procédure en cours lancée par la commission, l'idée est de pouvoir recourir à la pression financière dans des cas comparables. "nous n'accepterons pas de mécanismes arbitraires qui feront de la gestion des fonds un instrument de pression politique à la demande", a déjà averti le vice-ministre polonais pour les affaires européennes, konrad szymanski. des pays comme l'autriche ou les pays-bas sont déjà mobilisés pour leur part contre une hausse des contributions nationales, à laquelle l'allemagne et la france sont en revanche disposées. la commission va plaider pour la nécessité de fonds plus importants pour le numérique, la recherche, la défense ou encore la protection des frontières extérieures, avec une proposition de "plus que quintupler" les effectifs de l'agence frontex après 2020, pour les porter à près de 6.000 selon une source européenne. le débat budgétaire, qui nécessitera in fine une décision unanime des pays européens, va enfin ressusciter un serpent de mer: la création de nouvelles ressources propres pour l'ue. la commission veut notamment que la taxation des échanges de quotas de carbone soit orientée vers le budget européen et la création d'une taxe sur les plastiques est dans les tuyaux. & 743 & low & Low & Values & Power & NA & 2018-05-02 & 2018 & 3 & ECO
Frame & low-medium & Regional & 500-1000 & -0.7708104 & -0.8786862 & 0.6041414 & -0.2983228 & 0.6486612 & 0.0 & -0.0275917 & -0.0818024 & Payer & Domestic & European & Mixed & Domestic|ECO & Neutral\\
France & https://www.lepoint.fr/monde/macron-accuse-les-dirigeants-de-hongrie-et-pologne-de-mentir-a-leur-peuple-26-10-2018-2266389\_24.php & 184 & Le Point & Private/Non-Public & Online and Offline & National & low = CP mentioned more times but NOT important part of story (mainly about others issues) & Mismanagement & Negative & Other country & 8.Mismanaged & NA & NA & NA & NA & NA & NA & NA & NA & France & macron accuse les dirigeants de hongrie et pologne de "mentir à leur peuple" & 2018-10-27 & fonds structurels & lors d'une "consultation citoyenne" sur l'europe à bratislava, devant un public pro-européen, il a laissé éclater sa colère contre les gouvernements des deux pays de l'est, à sept mois des élections européennes qu'il voit comme un affrontement entre nationalistes et progressistes pro-européens. "en hongrie et en pologne, certains dirigeants ont joué avec une idée inacceptable. quand je les écoute comparer bruxelles et moscou d'avant (de l'époque soviétique, ndlr), c'est fou et inacceptable. ils mentent à leur peuple", a-t-il lancé avec force, en anglais. "quand je vois de grandes affiches disant "stop à bruxelles", que font-ils ? ils veulent stopper bruxelles et ses fonds structurels ? allez-y je vous en prie ! comment vivent-ils ? comment vit (le parti du premier ministre hongrois viktor orban) fidesz, qui les finance ? qui les a payés ? qui a fait leur carrière ? les fonds structurels européens !", a-t-il dénoncé. "que font ces dirigeants avec ces esprits fous et qui mentent à leur peuple", a-t-il encore demandé. "les nationalistes sont déjà grands, ils ont gagné dans certains pays d'europe", a-t-il ajouté devant la presse. "oui il y a ce clivage" entre progressistes et nationalistes, avait-il dit plus tôt dans la journée. "qui a gagné les élections européennes en france lors des dernières échéances ? qui ? le front national !" de marine le pen, rebaptisé depuis rassemblement national. "ca n'a choqué personne ! qui était au deuxième tour de l'élection présidentielle en france ? le front national, face à moi. donc ils sont là, les nationalistes", a-t-il souligné, lors d'un point de presse commun avec le premier ministre slovaque peter pellegrini. "ils ont grandi (...) parce qu'on ne les condamnait que moralement, sans vouloir les combattre sur le terrain des réponses(...) parce qu'ils sont d'accord pour démanteler l'europe, mais allez demander aux nationalistes la solution qu'ils proposent ensemble", s'est-il insurgé. arrivé vendredi matin à bratislava, emmanuel macron a passé la journée avec le président slovaque andrej kiska et peter pellegrini, puis est arrivé à prague dans la soirée pour y rencontrer les dirigeants tchèques, le président milos zeman et le premier ministre andrej babis. il a martelé dans les deux capitales qu'il ne "croyait pas au clivage est-ouest" et plaidé pour un europe renforcée, tant en matière de défense que pour la zone euro. la slovaquie et la république tchèque font partie, avec la hongrie et la pologne, du groupe de visegrad, qui partage une position très hostiles aux migrants. vendredi, emmanuel macron n'a quasiment pas évoqué publiquement ce sujet clivant, préférant insister sur les sujets européens qui rapprochent paris de prague et bratislava. il avait commencé vendredi par des critiques mesurées contre les positions nationalistes et antimigrants de la hongrie et de la pologne, déclarant vouloir "ne rien céder sur les valeurs", mais soulignant aussi qu'il ne fallait "pas diviser l'europe". dans une interview, il avait également nuancé ses critiques contre viktor orban, qui l'a pourtant désigné comme son principal adversaire pour les élections européennes. lors du sommet européen de salzbourg en septembre, le président français avait pris soin de serrer la main de viktor orban devant les journalistes. & 543 & low & Low & Governance & NA & NA & 2018-10-27 & 2018 & 3 & POL
Frame & low-medium & National & 500-1000 & -0.7708104 & -0.8786862 & 0.6041414 & -0.2983228 & 0.6486612 & 0.0 & -0.0275917 & -0.0818024 & Payer & European & European & European & European|POL & Negative\\
France & https://www.la-croix.com/France/Politique/Macron-accuse-dirigeants-Hongrie-Pologne-mentir-leur-peuple-2018-10-27-1300978989 & 183 & La Croix & Private/Non-Public & Online and Offline & National & low = CP mentioned more times but NOT important part of story (mainly about others issues) & Mismanagement & Negative & Other country & 8.Mismanaged & NA & NA & NA & NA & NA & NA & NA & NA & France & macron accuse les dirigeants de hongrie et pologne de "mentir à leur peuple" & 2018-10-27 & fonds structurels & le président français emmanuel macron à bratislava en slovaquie, le 26 octobre 2018 / afp en visite en slovaquie, emmanuel macron est passé de la critique mesurée à la colère contre les dirigeants de hongrie et de pologne, des "esprits fous" qui selon lui "mentent à leur peuple" par leurs positions anti-européennes. lors d'une "consultation citoyenne" sur l'europe à bratislava, devant un public pro-européen, il a laissé éclater sa colère contre les gouvernements des deux pays de l'est, à sept mois des élections européennes qu'il voit comme un affrontement entre nationalistes et progressistes pro-européens. "en hongrie et en pologne, certains dirigeants ont joué avec une idée inacceptable. quand je les écoute comparer bruxelles et moscou d'avant (de l'époque soviétique, ndlr), c'est fou et inacceptable. ils mentent à leur peuple", a-t-il lancé avec force, en anglais. "quand je vois de grandes affiches disant "stop à bruxelles", que font-ils ? ils veulent stopper bruxelles et ses fonds structurels ? allez-y je vous en prie ! comment vivent-ils ? comment vit (le parti du premier ministre hongrois viktor orban) fidesz, qui les finance ? qui les a payés? qui a fait leur carrière ? les fonds structurels européens !", a-t-il dénoncé. "que font ces dirigeants avec ces esprits fous et qui mentent à leur peuple", a-t-il encore demandé. "les nationalistes sont déjà grands, ils ont gagné dans certains pays d'europe", a-t-il ajouté devant la presse. le premier ministre hongrois viktor orban à milan en italie, le 28 août 2018 / afp/archives "oui il y a ce clivage" entre progressistes et nationalistes, avait-il dit plus tôt dans la journée. "qui a gagné les élections européennes en france lors des dernières échéances ? qui ? le front national !" de marine le pen, rebaptisé depuis rassemblement national. "ca n'a choqué personne ! qui était au deuxième tour de l'élection présidentielle en france ? le front national, face à moi. donc ils sont là, les nationalistes", a-t-il souligné, lors d'un point de presse commun avec le premier ministre slovaque peter pellegrini. "ils ont grandi (...) parce qu'on ne les condamnait que moralement, sans vouloir les combattre sur le terrain des réponses(...) parce qu'ils sont d'accord pour démanteler l'europe, mais allez demander aux nationalistes la solution qu'ils proposent ensemble", s'est-il insurgé. arrivé vendredi matin à bratislava, emmanuel macron a passé la journée avec le président slovaque andrej kiska et peter pellegrini, puis est arrivé à prague dans la soirée pour y rencontrer les dirigeants tchèques, le président milos zeman et le premier ministre andrej babis. il a martelé dans les deux capitales qu'il ne "croyait pas au clivage est-ouest" et plaidé pour un europe renforcée, tant en matière de défense que pour la zone euro. la slovaquie et la république tchèque font partie, avec la hongrie et la pologne, du groupe de visegrad, qui partage une position très hostiles aux migrants. vendredi, emmanuel macron n'a quasiment pas évoqué publiquement ce sujet clivant, préférant insister sur les sujets européens qui rapprochent paris de prague et bratislava. il avait commencé vendredi par des critiques mesurées contre les positions nationalistes et antimigrants de la hongrie et de la pologne, déclarant vouloir "ne rien céder sur les valeurs", mais soulignant aussi qu'il ne fallait "pas diviser l'europe". dans une interview, il avait également nuancé ses critiques contre viktor orban, qui l'a pourtant désigné comme son principal adversaire pour les élections européennes. lors du sommet européen de salzbourg en septembre, le président français avait pris soin de serrer la main de viktor orban devant les journalistes. & 611 & low & Low & Governance & NA & NA & 2018-10-27 & 2018 & 3 & POL
Frame & low-medium & National & 500-1000 & -0.7708104 & -0.8786862 & 0.6041414 & -0.2983228 & 0.6486612 & 0.0 & -0.0275917 & -0.0818024 & Payer & European & European & European & European|POL & Negative\\
France & https://www.ouest-france.fr/pays-de-la-loire/saint-nazaire-44600/saint-nazaire-comment-faire-subventionner-son-projet-par-l-europe-5577594 & 164 & Ouest France & Private/Non-Public & Online and Offline & Regional/Local & very high = CP is most important issue + CP is mentioned in title/headline & Improve governance & Positive & Subnational & No myth & NA & NA & NA & NA & NA & NA & NA & NA & France & saint-nazaire. comment faire subventionner son projet par l'europe ?. info & 2018-02-20 & fonds structurels & vous avez un projet à faire financer ? pensez à l'europe annonce en quelque sorte la région en mettant en place sa permanence de l'information européenne. c'est à saint-nazaire dans ses locaux de l'espace régional, près de la gare, qu'est organisé ce premier rendez-vous adressé à un large public. " ce dispositif innovant vise à renforcer l'accompagnement régional des porteurs de projets sur les territoires, explique vanessa charbonneau, vice-présidente en charge des affaires européennes. l'europe de demain se construira non par le verbe mais par la preuve en démontrant aux européens qu'elle peut répondre concrètement à leurs attentes. " les porteurs de projets visés sont particuliers, entreprises, associations, écoles intéressées pour décrocher aussi bien les fonds structurels (feder, fse, feader, feamp) gérés par la région que les programmes gérés par la commission européenne (interreg, erasmus +, life, horizon 2020, cosme, europe creative, europe pour les citoyens). & 153 & very high & High & Governance & NA & NA & 2018-02-20 & 2018 & 3 & POL
Frame & high-very high & Regional & <500 & -0.7708104 & -0.8786862 & 0.6041414 & -0.2983228 & 0.6486612 & 0.0 & -0.0275917 & -0.0818024 & Payer & Domestic & Domestic & Domestic & Domestic|POL & Positive\\
\addlinespace
France & https://www.laprovence.com/article/france-monde/4952705/ue-bruxelles-veut-un-budget-post-brexit-en-hausse-mais-avec-des-coupes.html & 111 & LaProvence.com & Private/Non-Public & Online and Offline & Regional/Local & medium = CP is important part of story & Financial burden & Factual & EU + National & No myth & Political leverage & Factual & EU + Other country & No myth & NA & NA & NA & NA & France & ue: bruxelles veut un budget post-brexit en hausse, mais avec des coupes & 2018-05-02 & politique de cohésion & bruxelles - la commission européenne a plaidé mercredi pour un budget de l'ue en hausse après le brexit, mais avec des coupes promettant de vives controverses dans deux secteurs emblématiques, l'agriculture et la cohésion en faveur des régions les plus modestes. après des mois de préparation, l'exécutif européen a également mis sur la table la création d'un lien inédit entre le versement de fonds européens et le respect de l'etat de droit, qui ne devrait pas manquer de braquer des pays se sentant visés, comme la pologne et la hongrie. la proposition, qui devra être négociée entre etats membres et le parlement européen, fixe à 1.279 milliards d'euros le budget pour la période 2021-2027 (contre 1.087 mds pour 2014-2020 en prix courants), en hausse malgré la perte prévue de l'importante contribution britannique. "c'est un budget ambitieux mais équilibré, juste pour tous", a défendu le président de la commission européeenne, jean-claude juncker, qui a présenté sa proposition de "cadre financier pluriannuel" devant les eurodéputés réunis à bruxelles. - le "trou" du brexit - le cocktail d'économies et de nouvelles ressources demandées vise selon bruxelles à donner à l'union les moyens des ambitions affichées pour sa nouvelle vie à 27, sans le royaume-uni, dont le départ prévu fin mars 2019 rend l'équation budgétaire plus complexe que jamais. selon m. juncker, le départ britannique va laisser un "trou de 15 milliards d'euros" par an dans les finances européennes après 2020 - dernière année de contribution de londres malgré un brexit programmé au printemps de l'année précédente. et la rupture avec ce "contributeur net" tombe d'autant plus mal que l'union européenne cherche à financer à 27 de nouvelles politiques, en matière de défense ou de migration notamment, sans renoncer aux "anciennes". parmi les mesures les plus difficiles à faire passer dans les capitales, bruxelles réclame "une réduction modérée" du financement de la politique agricole commune (pac), chère à la france, et de la politique de cohésion, dont les pays de l'est sont les grands bénéficiaires, "de 5\% environ dans les deux cas". ceux domaines politiques emblématiques représentent actuellement respectivement 37\% et 35\% du budget de l'ue. pour paris, "il est inacceptable que dans un budget en expansion il y ait des coupes si importantes dans les aides directes" aux agriculteurs, a réagi sans tarder une source diplomatique. - "etat de droit" - parmi les pays de l'est, opposés à la baisse de la politique de cohésion, certains comme la pologne et la hongrie sont d'autant plus sur la défensive qu'ils se sentent visées par la proposition inédite de la commission de lier versement de fonds européens et respect de l'etat de droit. ce "nouveau mécanisme qui permettra de protéger le budget en fonction des risques liés aux déficiences de l'etat de droit", a expliqué jean-claude juncker, assurant qu'"il ne visait pas des etats membres en particulier". plusieurs pays le réclamaient pour tirer les leçons du bras de fer infructueux entre bruxelles et le gouvernement ultra-conservateur polonais, accusé de menacer l'indépendance de sa justice. face à la lourdeur de la procédure en cours lancée par la commission, l'idée est de pouvoir recourir à la pression financière dans des cas comparables. "nous n'accepterons pas de mécanismes arbitraires qui feront de la gestion des fonds un instrument de pression politique à la demande", avait averti récemment le vice-ministre polonais pour les affaires européennes, konrad szymanski. des pays comme l'autriche ou les pays-bas sont déjà mobilisés pour leur part contre une hausse des contributions nationales, à laquelle l'allemagne et la france sont en revanche disposées. la commission plaide pour la nécessité de fonds plus importants pour le numérique, la recherche, le programme erasmus+, la défense ou encorte la protection des frontières extérieures. elle a aussi proposé de nouvelles ressources propres pour l'ue, en demandant qu'une partie des revenus de la taxation des échanges de quotas de carbone soit à l'avenir orientée vers le budget européen. elle a aussi mis sur la table la création d'une nouvelle taxe sur les déchets plastiques non recyclés. la commission veut que les tractations entre etats membres et eurodéputés soient bouclées avant les prochaines élections européennes, soit moins de deux mois après le divorce avec les britanniques. "ce genre de négociations prend normalement deux ans", souligne une source diplomatique, perplexe face à ce calendrier. & 747 & medium & Medium & Values & Power & NA & 2018-05-02 & 2018 & 3 & ECO
Frame & low-medium & Regional & 500-1000 & -0.7708104 & -0.8786862 & 0.6041414 & -0.2983228 & 0.6486612 & 0.0 & -0.0275917 & -0.0818024 & Payer & Domestic & European & Mixed & Domestic|ECO & Neutral\\
France & http://www.lesechos.fr/journal20151012/lec1\_france/021394913174-valerie-pecresse-je-veux-tout-chambouler-a-la-region-ile-de-france-1164542.php\#Xtor=AD-6000 & 103 & LesEchos.fr & Private/Non-Public & Online and Offline & National & low = CP mentioned more times but NOT important part of story (mainly about others issues) & Economic development & Positive & EU + Subnational & No myth & Research \& innovation & Positive & EU + Subnational & No myth & Infrastructure & Positive & EU + Subnational & NA & France & valã©rie pã©cresse : â«â je veux tout chambouler ã â laâ rã©gion ile-de-franceâ â» & 2015-10-11 & fonds structurels & londres, berlin, barcelone ou munich ont une dynamique très forte et nous, qui avons des atouts incomparables avec une des plus belles capitales au monde, une concentration unique en europe de recherche et d'innovation avec saclay et paris, une grande qualité de vie, nous avons un chômage qui augmente plus vite que dans d'autres régions. c'est juste sidérant ! l'ile-de-france a toujours été la région la plus dynamique de france, mais elle est en perte de vitesse. aujourd'hui, le président de la région ne se conduit pas en entrepreneur mais en rentier. il récolte l'argent des impôts puis le distribue en faisant des chèques. je veux faire de la politique différemment : être un président entrepreneur dans une région qui sera pro-business. autrement dit, créer de la richesse et de l'emploi en entrant dans un cercle vertueux d'attractivité. en un mot, un président qui va chercher de l'argent partout ailleurs que dans la poche du contribuable et des entreprises. la première, je veux m'entourer d'un conseil de chefs d'entreprises de toutes tailles, françaises et étrangères, implantées en ile-de-france, comme le font les maires de londres et de shanghai. il m'aidera à orienter la politique économique et de formation de la région vers les secteurs d'avenir créateurs d'emplois et à construire un environnement plus favorable à la création d'entreprises. deuxième mesure, dès janvier, je veux organiser une grande conférence avec les entreprises, les syndicats, les chercheurs et les banquiers sur la croissance, l'emploi, les conditions de vie au travail et la formation. notre problème, en ile-de- france, est que nous avons des mondes qui ne se rencontrent pas et que la mayonnaise ne prend pas. il faut que nous choisissions les filières dans lesquelles nous voulons être les premiers au monde. j'en ai identifié un certain nombre : le luxe, les industries culturelles, l'aéronautique, le numérique, les véhicules du futur, les biotechnologies, la ville du futur, les services financiers, l'assurance. dans tous ces domaines, nous pouvons être demain la première place européenne. enfin, je veux, et c'est ma troisième mesure, tout chambouler à la région ile-de-france. il faut en refaire une collectivité d'investissement. en 1998, quand la droite a perdu la région, deux tiers de son budget étaient consacrés à l'investissement et un tiers aux subventions et au fonctionnement. aujourd'hui, c'est l'inverse. or l'investissement, c'est ce qui va changer le visage de la région. ce sont les transports, le logement, le développement économique, la recherche, l'université, les lycées. nous devons engager la révolution des transports, qui sont aujourd'hui le point faible de l'attractivité et de la qualité de vie dans notre région. le tourisme en fait aussi partie. actuellement, il concerne essentiellement paris, disneyland et versailles. personne n'a jamais pris le touriste par la main pour le faire sortir de paris. si les touristes dépensaient autant dans notre région que ce qu'ils dépensent ailleurs, cela ferait 1 milliard d'euros de plus pour l'économie francilienne chaque année. il y a là une mine d'emplois. cela suppose de lever certains blocages sur le travail du dimanche ou l'absence de liaison ferroviaire directe entre paris et l'aéroport de roissy. nous ferons des économies et irons chercher les crédits européens. je commencerai par déménager le siège de la région en banlieue. nous ferons ainsi 25 millions d'euros d'économies. je suis en train de finaliser un grand plan de lutte contre le gaspillage et de bonne gestion de la région, qui se chiffrera en centaines de millions d'euros que je redéploierai sur les transports et la formation, qui est la clef de l'emploi. je me concentrerai sur ces deux priorités plutôt que de saupoudrer les crédits. oui, je veux en finir avec ce scandale. en juin dernier, bruxelles a annulé 50 millions d'euros de crédits européens prévus pour l'emploi en ile-de-france sur les 117 millions qui lui étaient destinés. la région n'avait pas déposé suffisamment de projets ! il y aussi les fonds structurels, qui permettraient de payer des parkings gratuits dans les gares, les fonds de recherche horizon 2020, où la france a un taux de retour de 12 \% alors qu'elle en finance 16,4 \%. il y a aussi les milliards d'euros du plan d'investissements d'avenir européen, le plan juncker. ses promoteurs m'ont dit être intéressés par saclay, mais aussi le cancer campus de villejuif, un territoire des industries culturelles en seine-saint-denis, le projet du grand roissy, mais n'avoir vu personne de la région. je veux passer d'une région guichet à une région projets. je suis pour une région du très grand paris. il faut des projets qui permettent d'identifier des territoires et de les rendre visibles au niveau mondial. ce sont la vallée des biotechs en essonne et dans le val-de-marne, saclay sur les systèmes, les nanotechnologies et l'énergie, marne-la-vallée sur la ville durable, la seine-saint-denis sur les industries culturelles et sur le luxe, la vallée de la seine sur l'aéronautique et les mobilités du futur. sans oublier paris, qui est l'incubateur des start-up numériques, la ville du digital, ou la défense, qui peut retrouver une seconde jeunesse, dans l'assurance et les services financiers avec l'isolement de la place londonienne. cela passe aussi par la construction du réseau de transports du grand paris, qu'on ne peut pas opposer à l'amélioration des lignes existantes. & 944 & low & Low & Socio-Economic & Socio-Economic & Socio-Economic & 2015-10-11 & 2015 & 1 & ECO
Frame & low-medium & National & 500-1000 & -0.7708104 & -0.8786862 & 0.6041414 & -0.2983228 & 0.6486612 & 0.0 & -0.0275917 & -0.0818024 & Payer & Domestic & European & Mixed & Domestic|ECO & Positive\\
France & https://lexpansion.lexpress.fr/actualites/1/actualite-economique/ue-des-coupes-en-vue-dans-les-fonds-agricoles-et-regionaux-apres-le-brexit\_2005090.html & 172 & LExpansion.com & Private/Non-Public & Online and Offline & National & very low = CP mentioned once & Financial burden & Factual & EU & No myth & NA & NA & NA & NA & NA & NA & NA & NA & France & les pays de l'ue se braquent sur les budgets post-brexit & 2018-05-02 & politique de cohésion & bruxelles - la commission européenne a plaidé mercredi pour un budget de l'ue en hausse après le brexit, mais avec des coupes dans des secteurs emblématiques comme l'agriculture, qui ont provoqué sans tarder le courroux de plusieurs pays dont la france. les coupes dans la politique de cohésion, en faveur des régions les plus modestes, promettent elles un bras de fer avec les pays de l'est, dont certains comme la pologne se sentent aussi visés par le gel des aides aux pays violant l'etat de droit, proposé pour la première fois par bruxelles. le danemark, les pays-bas et l'autriche ont de leur côté refusé de mettre davantage la main à la poche à l'avenir pour compenser le divorce avec le royaume-uni, qui laissera un trou annuel de "15 milliards" dans le budget de l'ue selon le président de la commission, jean-claude juncker. ces premières réactions négatives augurent de tractations tendues autour du "cadre financier pluriannuel" (cfp) pour 2021-2027, qui doit encore être adopté à l'unanimité entre etats membres et négocié avec le parlement européen. - "pas un massacre" - l'exécutif européen veut que l'ue se dote d'un budget à 27 de 1.279 milliards d'euros pour la période 2021-2027 (contre 1.087 mds pour 2014-2020 à 28, en prix courants), en hausse malgré la perte de l'importante contribution britannique. "c'est un budget ambitieux mais équilibré, juste pour tous", a défendu jean-claude juncker. "ce n'est pas un massacre", a-t-il ajouté concernant les coupes proposées, faisant valoir la nécessité d'augmenter les financements dans des domaines comme la défense, la recherche ou encore la protection des frontières extérieures. mais les "réductions modérées" proposées de 5\% pour la politique agricole commune (pac) et de 7\% pour la politique de cohésion, les deux principaux postes budgétaires de l'ue, ont déjà du mal à passer. la france "ne pourra accepter aucune baisse de revenu direct pour les agriculteurs", a rapidement réagi son ministère de l'agriculture dans un communiqué, déplorant une "baisse, drastique, massive et aveugle" et des mesures "inacceptables". la politique de cohésion, dont les pays de l'est sont les principaux bénéficiaires, "sert l'ue toute entière, nous n'allons certainement pas accepter des coupes disproportionnées dans ce domaine", a réagi de son côté le vice-ministre polonais pour les affaires européennes, konrad szymanski. varsovie est également méfiante concernant la proposition inédite de la commission de pouvoir suspendre des versements de fonds européens en cas de violation de l'etat de droit par un etats membre. "il ne faut aucun espace pour de l'arbitraire", a mis en garde le responsable polonais, dont le gouvernement est accusé par bruxelles de menacer l'indépendance de sa justice. la commission a déclenché à l'encontre de la pologne une procédure inédite pour qu'elle amende ces réformes, mais qui n'a pour l'heure porté aucun fruit. d'où l'idée, défendue par paris et berlin et désormais endossée par la commission, de pouvoir recourir à l'avenir à la pression financière dans de pareils cas. - "supprimer les rabais" - "une union plus petite, ça signifie aussi un budget plus petit", ont de leur côté réagi en choeur les premiers ministres libéraux danois et néerlandais, déclinant comme l'autriche l'appel lancé par la commission à relever les contributions nationales. pour bruxelles, il s'agit de faire passer le budget de l'ue au niveau de 1,1\% du total cumulé du revenu national brut (rnb) des etats membres, au-delà de la limite de 1\% fixée pour 2014-2020. en complément, la commission a aussi relancé la recherche, maintes fois avortée, de nouvelles ressources propres pour l'ue, souhaitant qu'elles puissent représenter jusqu'à 12\% du budget. elle demande notamment que 20\% des revenus de la taxation des échanges de quotas de carbone soient à l'avenir orientés vers l'ue. et propose une "contribution nationale" basée sur le volume de déchets d'emballage en plastique non recyclés. le brexit va aussi permettre, selon la commission, de "supprimer tous les rabais", après que celui arraché par londres en a justifié d'autres par ricochet. mais cette suppression "se fera progressivement sur cinq ans", plaide l'exécutif, alors que la france souhaitait une mesure plus radicale. jean-claude juncker a demandé mercredi que les tractations entre etats membres et eurodéputés soient bouclées avant les prochaines élections européennes de fin mai 2019, soit moins de deux mois après le divorce avec les britanniques. "ce genre de négociations prend normalement deux ans", souligne une source diplomatique, perplexe face à ce calendrier. & 776 & very low & Low & Values & NA & NA & 2018-05-02 & 2018 & 3 & ECO
Frame & v.low & National & 500-1000 & -0.7708104 & -0.8786862 & 0.6041414 & -0.2983228 & 0.6486612 & 0.0 & -0.0275917 & -0.0818024 & Payer & European & European & European & European|ECO & Neutral\\
France & https://www.lesechos.fr/journal20180412/lec1\_france/0301553272986-la-france-veut-des-fonds-europeens-pour-toutes-ses-regions-2168571.php\#Xtor=AD-6000 & 107 & LesEchos.fr & Private/Non-Public & Online and Offline & National & very high = CP is most important issue + CP is mentioned in title/headline & Institutional bargaining over funding & Positive & EU + National + Subnational & No myth & Economic development & Positive & EU + National + Subnational & No myth & NA & NA & NA & NA & France & la france veut des fonds européens pour toutes ses régions & 2018-04-11 & fonds social européen & le ministre de la cohésion des territoires, jacques mézard, va défendre ce jeudi à luxembourg la poursuite d'une " ambitieuse " politique européenne en faveur des régions. pas question de baisser pavillon. le ministre de la cohésion des territoires, jacques mézard, apportera le soutien de la france à une politique régionale européenne " ambitieuse " ce jeudi à luxembourg. les représentants des vingt-huit se retrouvent une dernière fois avant la proposition d'un budget européen pour l'après 2020. or cette politique de cohésion qui dispose du deuxième budget européen après la politique agricole, se voit menacée par la disparition des contributions britanniques (brexit) et l'émergence de nouvelles missions européennes (immigration, sécurité). les deux fonds concernés, le fonds européen de développement économique régional (feder) et le fonds social européen (fse) apportent 15,5 milliards d'euros aux régions françaises sur les 350 milliards de cette politique européenne. des aides mieux ciblées en février, la commission européenne avait agité plusieurs scénarios de réforme qui auraient pu aboutir à limiter l'aide aux seules régions les plus pauvres. le ministère de jacques mézard assure avoir convaincu matignon de ne sacrifier aucune région, se justifiant par le contexte de montée des fractures territoriales partout dans l'union. en clair, il y a des territoires en déclin dans toutes les régions. comme tout le monde s'attend à une baisse des crédits, le gouvernement devrait défendre l'idée qu'il faudra un ciblage plus fort des territoires aidés dans chaque région. reste à trouver les critères de sélection pertinents. la france va aussi plaider pour un rééquilibrage vers l'ouest de ces fonds qui profitent très fortement aux pays de l'est dans le programme actuel. ce sujet est d'autant plus sensible politiquement que la france comme l'allemagne veulent introduire de nouvelles conditionnalités fiscales et sociales aux aides, histoire de lutter contre le dumping social ou les atteintes aux libertés. l'etat soutiendra par ailleurs la nécessité de simplifier les procédures administratives, alourdies par exemple par sept niveaux de contrôle. a la direction du commissariat général à l'égalité des territoires, on conteste toutefois l'inefficacité du financement : l'engagement des fonds pour la période 2014-2020 a déjà atteint 51,4 \%, selon ses données. l'association régions de france, qui revendique deux ans d'intense lobbying, salue le ralliement du gouvernement à son point de vue, mais le juge tardif et dans le sillage des autres pays européens. tout en ironisant sur le fait qu'une autre position aurait été incompréhensible à un an des élections européennes. " les fonds européens, c'est la seule chose visible pour les citoyens ", rappelle-t-elle. & 440 & very high & High & Power & Socio-Economic & NA & 2018-04-11 & 2018 & 3 & POL
Frame & high-very high & National & <500 & -0.7708104 & -0.8786862 & 0.6041414 & -0.2983228 & 0.6486612 & 0.0 & -0.0275917 & -0.0818024 & Payer & Domestic & European & Mixed & Domestic|POL & Positive\\
France & http://bfmbusiness.bfmtv.com/monde/macron-le-dumping-social-risque-de-mener-a-un-demantelement-de-l-ue-1241842.html & 150 & BFM BUSINESS & Private/Non-Public & Online only & National & very low = CP mentioned once & Political leverage & Negative & EU + Other country & No myth & NA & NA & NA & NA & NA & NA & NA & NA & France & macron: le dumping social risque de mener à un " démantèlement de l'ue " & 2017-08-23 & fonds structurels & en tournée en europe de l'est, le président de la république souhaite obtenir un accord d'ici fin octobre pour raccourcir la durée du travail détaché et renforcer les contrôles contre les fraudes. emmanuel macron, en déplacement en roumanie, a estimé ce jeudi que, faute d'amendement de la directive européenne sur le travail détaché, le "dumping fiscal" pratiqué dans certains pays de l'est risquait de mener à un "démantèlement de l'ue". "certains politiques ou milieux d'affaires" en europe veulent "toucher les fonds structurels et développer un modèle qui est le dumping fiscal et social", a-t-il déclaré lors d'une conférence de presse à bucarest. si rien n'est fait, "ça éclatera", ce sera "un démantèlement de l'union européenne", car "aucune opinion publique des pays fortement développés n'acceptera le système tel qu'il fonctionne", a estimé le chef de l'état français. emmanuel macron a rencontré ce jeudi à bucarest le président roumain klaus iohannis, au deuxième jour d'une tournée en europe de l'est pour promouvoir un durcissement de la directive sur le travail détaché. accord attendu pour la fin octobre comme promis durant sa campagne, le président français espère obtenir fin octobre à bruxelles un accord majoritaire pour raccourcir la durée du travail détaché et renforcer les contrôles contre les fraudes. la veille, il a rencontré les dirigeants autrichien, tchèque et slovaque. l'autriche a appuyé la position française, mais la république tchèque et la slovaquie, plus prudents, ont accepté de rechercher rapidement un compromis, sans s'engager sur le fond. également prudent, le président roumain a reconnu le "fort mécontentement" en europe de l'ouest sur ce dossier et estimé qu'il fallait "améliorer cette directive", mais sans "éliminer la concurrence ou le marché libre". il a aussi souhaité séparer la question des transporteurs routiers qui circulent en europe, sujet très sensible en roumanie, ce que paris a accepté. "nous aurons ensuite une autre négociation sur le transport routier", a ainsi confirmé emmanuel macron, un geste envers la roumanie mais aussi la bulgarie ou l'espagne. autre attention de la france, emmanuel macron s'est dit prêt à soutenir la demande de la roumanie d'entrer dans l'espace schengen, mais une fois ce dernier réformé. "vous êtes en droit de demander votre intégration dans schengen. j'y suis totalement ouvert" a-t-il lancé. & 397 & very low & Low & Power & NA & NA & 2017-08-23 & 2017 & 2 & POL
Frame & v.low & National & <500 & -0.7708104 & -0.8786862 & 0.6041414 & -0.2983228 & 0.6486612 & 0.0 & -0.0275917 & -0.0818024 & Payer & European & European & European & European|POL & Negative\\
\addlinespace
France & https://www.nouvelobs.com/topnews/20180621.AFP2486/les-roms-d-italie-veulent-montrer-a-salvini-qui-ils-sont-vraiment.html & 179 & L'Obs & Private/Non-Public & Online and Offline & National & very low = CP mentioned once & Fraud/Corruption & Factual & EU + Other country & No myth & NA & NA & NA & NA & NA & NA & NA & NA & France & les roms d'italie veulent montrer à salvini qui ils sont vraiment & 2018-06-21 & fonds structurels & rome (afp) - dans la poussière d'un camp de roms à la périphérie de la capitale italienne, la vie continue en dépit des menaces du nouveau ministre de l'intérieur, et patron de la ligue (extrême droite), matteo salvini. déjà sous le feu des critiques pour avoir refusé d'ouvrir les ports italiens à un navire humanitaire, l'aquarius, avec à son bord 630 migrants, m. salvini a suscité une nouvelle vague d'indignation en annonçant un prochain recensement des roms vivant en italie. ces propos ont soulevé un tollé dans les rangs de l'opposition de gauche et un certain malaise au sein du nouveau gouvernement populiste italien. un "fichage" dénoncé comme tel par la gauche et les organisations de défense des droits de l'homme, que certains ont comparé aux lois raciales votées en 1938 en plein fascisme. mais au "river village", nom de ce camp de roms, situé loin des routes principales et des palais du centre historique de la ville éternelle, on préfère se tenir loin de cette polémique. "s'il (salvini) veut venir ici, nous l'accueillerons avec courtoisie. il faut qu'il voit comment nous vivons", assure habibi mehmedi, 18 ans. - 'mauvaises choses' - bien sûr, "il y en a certains chez nous qui font de mauvaises choses, mais la plupart ici sont italiens et n'ont jamais commis aucun délit", ajoute-t-il, interrogé par l'afp. marcello zuinisi, fondateur de nazione rom (nation rom), une association qui se bat pour les droits des roms et sintis (tziganes) en italie, considère qu'il ne faut pas trop avoir peur des menaces du nouveau ministre de l'intérieur et homme fort du gouvernement italien. "je connais matteo salvini, nous avons eu plusieurs conversations", a-t-il assuré. et, "je ne crois pas qu'il fasse grand chose, ce sont juste des mots", a-t-il ajouté. et pour le prouver il montre à l'afp un message qu'il a envoyé au ministre de l'intérieur pour l'avertir que l'institut de la statistique en italie avait déjà mené une longue enquête en 2017 sur les roms d'italie. et "plus tard dans la journée, il a changé son discours et a commencé à évoquer la façon dont les fonds européens (à destination des roms) ont été utilisés. il m'a écouté", a-t-il affirmé. m. zuinisi a ainsi expliqué que son association s'était rendue à bruxelles récemment pour réclamer de la commission européenne une enquête sur l'utilisation des sept milliards d'euros de fonds structurels accordés à l'italie pour favoriser le logement, la santé ou l'éducation des roms et des sans-abri. - 'cachet d'aspirine' - "mais dans les camps, on a rien reçu, même pas un cachet d'aspirine", a-t-il affirmé. selon l'association 21 juillet, il y a en italie entre 120 et 180.000 roms et sintis, dont environ 16.400 vivent dans des camps comme celui de river village, disséminés dans 87 villes différentes. dans ces camps, 43\% des résidents sont italiens, le reste provenant essentiellement de pays de l'ex-yougoslavie. rien qu'à rome, il y a 17 camps, et bien que les "zingari", les tziganes ou roms ne représentent que 0,3\% de la population en italie, ils sont sujets à toute sorte de discrimination. souvent accusées de vols et trafics en tout genre, les roms ont du mal à s'insérer, surtout dans la capitale italienne où le crime organisé est pour partie entre les mains de clans d'origine sinti. en avril dernier, la police a arrêté à river village une dizaine de personnes après un vol de voitures. et pour accéder à ce camp, il faut franchir un "check-point" de la police au bout d'une petite route qui longe le tibre, le fleuve qui traverse rome. quelque 127 familles, et environ 180 enfants, vivent dans ce camp, dans des bungalows, des caravanes ou des structures préfabriquées. c'est l'un des meilleurs camps de la capitale, affirme m. zuinisi, qui souligne par exemple que tous les enfants vont à l'école. pourtant, river village n'a pas l'eau courante, et l'eau potable est livrée par une compagnie privée. - absence de travail - "et heureusement, vous imaginez ce que ce serait de vivre ici avec cette chaleur !", lance albert hasimi, un kosovar âgé de 18 ans, qui a vécu presque toute sa vie en ialie. mais la principale préoccupation est l'absence de travail, pour la simple raison, explique-t-il, que les employeurs ne veulent pas embaucher des roms. "tout le monde dit que les tziganes volent, mais la camorra (la mafia napolitaine) ? ils pensent que leur race vaut mieux que la notre. c'est vraiment aussi simple que ça ?", lance avec amertume habibi mehmedi. & 801 & very low & Low & Governance & NA & NA & 2018-06-21 & 2018 & 3 & POL
Frame & v.low & National & 500-1000 & -0.7708104 & -0.8786862 & 0.6041414 & -0.2983228 & 0.6486612 & 0.0 & -0.0275917 & -0.0818024 & Payer & European & European & European & European|POL & Neutral\\
France & http://www.france24.com/fr/20180621-roms-ditalie-veulent-montrer-a-salvini-ils-sont-vraiment & 170 & France 24 & Public & Online only & National & very low = CP mentioned once & Mismanagement & Negative & Other country & 7.Fraud & NA & NA & NA & NA & NA & NA & NA & NA & France & les roms d'italie veulent montrer à salvini qui ils sont vraiment & 2018-06-21 & fonds structurels & dans la poussière d'un camp de roms à la périphérie de la capitale italienne, la vie continue en dépit des menaces du nouveau ministre de l'intérieur, et patron de la ligue (extrême droite), matteo salvini. déjà sous le feu des critiques pour avoir refusé d'ouvrir les ports italiens à un navire humanitaire, l'aquarius, avec à son bord 630 migrants, m. salvini a suscité une nouvelle vague d'indignation en annonçant un prochain recensement des roms vivant en italie. ces propos ont soulevé un tollé dans les rangs de l'opposition de gauche et un certain malaise au sein du nouveau gouvernement populiste italien. un "fichage" dénoncé comme tel par la gauche et les organisations de défense des droits de l'homme, que certains ont comparé aux lois raciales votées en 1938 en plein fascisme. mais au "river village", nom de ce camp de roms, situé loin des routes principales et des palais du centre historique de la ville éternelle, on préfère se tenir loin de cette polémique. "s'il (salvini) veut venir ici, nous l'accueillerons avec courtoisie. il faut qu'il voit comment nous vivons", assure habibi mehmedi, 18 ans. - 'mauvaises choses' - bien sûr, "il y en a certains chez nous qui font de mauvaises choses, mais la plupart ici sont italiens et n'ont jamais commis aucun délit", ajoute-t-il, interrogé par l'afp. marcello zuinisi, fondateur de nazione rom (nation rom), une association qui se bat pour les droits des roms et sintis (tziganes) en italie, considère qu'il ne faut pas trop avoir peur des menaces du nouveau ministre de l'intérieur et homme fort du gouvernement italien. "je connais matteo salvini, nous avons eu plusieurs conversations", a-t-il assuré. et, "je ne crois pas qu'il fasse grand chose, ce sont juste des mots", a-t-il ajouté. et pour le prouver il montre à l'afp un message qu'il a envoyé au ministre de l'intérieur pour l'avertir que l'institut de la statistique en italie avait déjà mené une longue enquête en 2017 sur les roms d'italie. et "plus tard dans la journée, il a changé son discours et a commencé à évoquer la façon dont les fonds européens (à destination des roms) ont été utilisés. il m'a écouté", a-t-il affirmé. m. zuinisi a ainsi expliqué que son association s'était rendue à bruxelles récemment pour réclamer de la commission européenne une enquête sur l'utilisation des sept milliards d'euros de fonds structurels accordés à l'italie pour favoriser le logement, la santé ou l'éducation des roms et des sans-abri. - 'cachet d'aspirine' - "mais dans les camps, on a rien reçu, même pas un cachet d'aspirine", a-t-il affirmé. selon l'association 21 juillet, il y a en italie entre 120 et 180.000 roms et sintis, dont environ 16.400 vivent dans des camps comme celui de river village, disséminés dans 87 villes différentes. dans ces camps, 43\% des résidents sont italiens, le reste provenant essentiellement de pays de l'ex-yougoslavie. rien qu'à rome, il y a 17 camps, et bien que les "zingari", les tziganes ou roms ne représentent que 0,3\% de la population en italie, ils sont sujets à toute sorte de discrimination. souvent accusées de vols et trafics en tout genre, les roms ont du mal à s'insérer, surtout dans la capitale italienne où le crime organisé est pour partie entre les mains de clans d'origine sinti. en avril dernier, la police a arrêté à river village une dizaine de personnes après un vol de voitures. et pour accéder à ce camp, il faut franchir un "check-point" de la police au bout d'une petite route qui longe le tibre, le fleuve qui traverse rome. quelque 127 familles, et environ 180 enfants, vivent dans ce camp, dans des bungalows, des caravanes ou des structures préfabriquées. c'est l'un des meilleurs camps de la capitale, affirme m. zuinisi, qui souligne par exemple que tous les enfants vont à l'école. pourtant, river village n'a pas l'eau courante, et l'eau potable est livrée par une compagnie privée. - absence de travail - "et heureusement, vous imaginez ce que ce serait de vivre ici avec cette chaleur !", lance albert hasimi, un kosovar âgé de 18 ans, qui a vécu presque toute sa vie en ialie. mais la principale préoccupation est l'absence de travail, pour la simple raison, explique-t-il, que les employeurs ne veulent pas embaucher des roms. "tout le monde dit que les tziganes volent, mais la camorra (la mafia napolitaine) ? ils pensent que leur race vaut mieux que la notre. c'est vraiment aussi simple que ça ?", lance avec amertume habibi mehmedi. & 799 & very low & Low & Governance & NA & NA & 2018-06-21 & 2018 & 3 & POL
Frame & v.low & National & 500-1000 & -0.7708104 & -0.8786862 & 0.6041414 & -0.2983228 & 0.6486612 & 0.0 & -0.0275917 & -0.0818024 & Payer & European & European & European & European|POL & Negative\\
France & http://www.europe1.fr/international/ue-un-sommet-a-27-pour-debattre-du-budget-post-brexit-3582176 & 165 & Europe 1 & Public & Online only & National & very low = CP mentioned once & Financial burden & Factual & EU & No myth & NA & NA & NA & NA & NA & NA & NA & NA & France & ue : un sommet à 27 pour débattre du budget post-brexit & 2018-02-23 & politique de cohésion & les dirigeants européens vont débattre vendredi pour la première fois, à 27, du budget de l'ue après 2020 et le départ du royaume-uni. ils vont également se pencher sur le mode de désignation du successeur de jean-claude juncker à la tête de la commission, un enjeu qui divise. trouver un nouvel équilibre budgétaire sans le royaume-uni un budget sans l'un des principaux contributeurs. les grands projets de l'ue pour se réinventer après le retrait britannique, dont le président français emmanuel macron s'est fait l'un des hérauts, vont être confrontés aux dures questions d'argent et de souveraineté. après avoir fait face à des défis inédits ces dernières années, en matière de sécurité intérieure ou d'accueil de migrants, l'ue est prête à se doter des moyens financiers pour y répondre à l'occasion de son prochain "cadre financier pluriannuel" (cfp). se pose la question du financement de ces nouvelles mesures alors que le budget de l'ue va perdre l'un des ses principaux contributeurs, soit environ 10 milliards d'euros par an selon le conseil européen. un "exercice inédit". selon une autre source européenne, il s'agit d'un "exercice inédit" pour le conseil puisque la commission européenne ne mettra ses propositions sur la table que début mai, sous la forme d'une "discussion politique" qui doit fournir quelques pistes à l'exécutif européen. définir des priorités. pour ce budget post-2020 - le cfp actuel court sur 7 ans, de 2014 à 2020 -, l'idée de la commission est de décider des priorités, puis d'ajuster le budget en conséquence. il faudra faire des choix, a d'ores et déjà prévenu bruxelles. l'essentiel du budget de l'ue, environ 70\%, est pour l'instant consacré aux piliers historiques de l'union : la politique de cohésion, visant à permettre aux régions les plus pauvres de rattraper leur retard, et la politique agricole commune (pac). "il y a les pays qui ne veulent pas payer plus et ceux qui ne veulent pas recevoir moins", a résumé le président de la commission européenne jean-claude juncker. selon le haut responsable européen, la plupart des états membres sont d'accord pour soutenir l'idée d'une augmentation de leur participation, mais "ceux qui sont sceptiques ou contre sont très déterminés". nommer un remplaçant pour jean-claude juncker le successeur de juncker nommé par le conseil. le départ du royaume-uni soulève un débat au sein de l'ue sur son avenir, et les 27 ont d'ailleurs déjà prévu de se retrouver en mai 2019 à sibiu (roumanie), pour préparer leur futur "agenda stratégique" pour la période 2019-2024. ce sommet, sous présidence roumaine, se tiendra juste avant l'élection d'un nouveau parlement européen et la formation d'une nouvelle commission, dont le président, successeur de l'emblématique jean-claude juncker, sera nommé par le conseil avec l'aval des eurodéputés. un candidat sélectionné par les partis politiques. or le parlement actuel vient de lancer un message très clair aux 27 états membres, exigeant que le candidat qu'ils choisiront en 2019 pour diriger l'exécutif européen soit l'une des "têtes de liste" (ou "spitzenkandidat" selon le terme allemand qui s'est imposé dans les institutions) sélectionnées par les partis politiques européens. & 551 & very low & Low & Values & NA & NA & 2018-02-23 & 2018 & 3 & ECO
Frame & v.low & National & 500-1000 & -0.7708104 & -0.8786862 & 0.6041414 & -0.2983228 & 0.6486612 & 0.0 & -0.0275917 & -0.0818024 & Payer & European & European & European & European|ECO & Neutral\\
France & http://www.ladepeche.fr/article/2017/03/24/2542684-l-europe-ce-grand-marche-economique.html & 144 & Ladepeche.fr & Private/Non-Public & Online and Offline & Regional/Local & medium = CP is important part of story & Economic development & Positive & EU & No myth & NA & NA & NA & NA & NA & NA & NA & NA & France & l'europe, ce grand marché économique & 2017-03-24 & fonds européen de développement régional & depuis 60 ans, l'économie reste au cœur des préoccupations de l'union européenne. et pour cause puisque le traité de rome instituait bien une communauté économique européenne (cee) dont l'objectif était de conduire une intégration économique plus poussée entre les pays membres, six en 1957. au lendemain de la seconde guerre mondiale, les pères fondateurs partaient du principe que les pays liés par des échanges commerciaux deviennent économiquement interdépendants et sont donc moins enclins à entrer en conflit. très vite la cee a étendu ses activités après la création de l'union douanière. depuis, un immense marché unique a vu le jour, qui continue à croître et à développer tout son potentiel. pour créer ce marché unifié, des centaines d'obstacles techniques, juridiques et bureaucratiques qui entravaient le libre-échange et la libre circulation des personnes, des biens, des services et des capitaux entre les pays membres de l'ue ont été supprimés. cette concurrence libre et non faussée a eu le mérite de faire baisser les prix et d'offrir un choix plus large aux consommateurs, par exemple pour les communications téléphoniques, les billets d'avion, ou les fournisseurs d'électricité et de gaz. les entreprises européennes qui vendent leurs produits dans l'ue bénéficient ainsi d'un accès direct à près de 500 millions de consommateurs. des écueils persistent bien sûr comme la diversité des systèmes fiscaux nationaux qui provoque la détestable optimisation fiscale ou encore l'absence d'harmonisation sociale que souligne encore aujourd'hui le brûlant sujet des travailleurs détachés. de la politique économique de l'union, la monnaie unique reste, d'évidence, la réalisation la plus concrète de l'intégration européenne (lire encadré). mais la politique économique de l'union se déploie sur plusieurs champs de compétences qui concernent au plus près les territoires et les régions comme l'occitanie. destinée à stimuler la croissance économique et à améliorer la qualité de vie au moyen d'investissements stratégiques, la politique régionale - mise en œuvre notamment par l'intermédiaire du fonds européen de développement régional (feder) ou le fonds de cohésion - finance plusieurs domaines prioritaires (recherche et innovation, technologies de l'information et de la communication, compétitivité des pme, transition vers une économie sobre en carbone, etc.) et donne des résultats tangibles. entre 2007 et 2012, elle a aidé les pays de l'ue à créer 769 000 emplois. en occitanie, la région gère 80 \% des fonds européens 2014-2020, grâce auxquels 3 milliards d'euros vont appuyer la stratégie régionale pour l'emploi, l'innovation et la transition énergétique. fin 2016, plus de 31 \% des programmes européens feder-fse-iej et 25 \% du feader étaient déjà programmés et bénéficiaient à 10 900 projets. en 2017, le budget régional intègre 560 millions d'euros de fonds européens. enfin, si la part des dépenses agricoles dans le budget de l'ue a fortement baissé, passant d'un montant record de près de 70 \% dans les années 1970 à environ 38 \% aujourd'hui, la politique agricole commune (pac) est à la fois un pilier et un symbole de l'action de l'europe. le budget de la pac, fixé pour la période 2014-2020, prévoit une enveloppe totale de 408,31 milliards d'euros. la stabilité grâce à l'euro adoptée par 19 des 28 pays de l'union européenne, la monnaie unique est utilisée chaque jour par 338,6 millions de personnes et apparaît, notamment depuis la crise de 2008, comme facteur de stabilité. et plus de 175 millions de personnes dans le monde utilisent d'ailleurs des monnaies qui sont à parité fixe avec l'euro, deuxième devise la plus importante dans le monde après le dollar américain. reste que l'orthodoxie budgétaire et la règle de 3 \% de déficit budgétaire qui prévalent depuis l'avènement de l'euro sont source de tensions et d'interprétations divergentes entre pays du nord et du sud. & 650 & medium & Medium & Socio-Economic & NA & NA & 2017-03-24 & 2017 & 2 & ECO
Frame & low-medium & Regional & 500-1000 & -0.7708104 & -0.8786862 & 0.6041414 & -0.2983228 & 0.6486612 & 0.0 & -0.0275917 & -0.0818024 & Payer & European & European & European & European|ECO & Positive\\
France & https://la1ere.francetvinfo.fr/polynesie/tahiti/polynesie-francaise/journal-europeennes-du-15-mai-2019-711167.html & 193 & Wallis et Futuna la 1ère & Public & Online only & Regional/Local & high = CP is most important issue in story (can also cover other issues) & Solidarity to poor countries/regions & Positive & EU & No myth & NA & NA & NA & NA & NA & NA & NA & NA & France & le journal des européennes du 15 mai 2019 - polynésie la 1ère & 2019-05-16 & fonds européen de développement régional & l'union européenne et ses etats membres sont les seconds bailleurs de fonds dans le pacifique sud, après l'australie. pour les 4 ptom associés, l'union européenne dispose de son propre instrument financier : le fed régional. le fonds européen de développement régional est ainsi réparti entre la nouvelle-calédonie, wallis et futuna, la polynésie française et pitcairn, seul territoire britannique dans l'océan pacifique. sa dotation pèse à ce jour 8 milliards de francs pacifique. a dix jours du scrutin organisé samedi 25 mai en polynésie, les opérations de mise sous plis du matériel électoral bat son plein. des dizaines de milliers d'enveloppes vont parvenir dans les boites postales des personnes inscrites sur les listes électorales. ce travail de fourmis a été confié trois jours durant à près de deux cents personnes sans emploi inscrites au sefi. & 139 & high & High & Values & NA & NA & 2019-05-16 & 2019 & 3 & ECO
Frame & high-very high & Regional & <500 & -0.7708104 & -0.8786862 & 0.6041414 & -0.2983228 & 0.6486612 & 0.0 & -0.0275917 & -0.0818024 & Payer & European & European & European & European|ECO & Positive\\
\addlinespace
France & https://www.bfmtv.com/politique/l-avertissement-de-macron-a-ceux-qui-veulent-une-europe-a-la-carte-1527450.html & 182 & BFMTV & Private/Non-Public & Online only & National & very low = CP mentioned once & Political leverage & Factual & EU + Other country & No myth & NA & NA & NA & NA & NA & NA & NA & NA & France & l'avertissement de macron à ceux qui veulent une europe " à la carte " & 2018-09-20 & fonds structurels & lors d'une conférence de presse en marge du sommet européen de salzbourg, ce jeudi en autriche, le chef de l'etat a ciblé les pays qui manqueraient à leur devoir de solidarité sur les questions migratoires, et en particulier la hongrie. ils pourraient sortir de l'espace schengen ou être privés des fonds européens. "l'europe n'est pas un menu à la carte, c'est un projet politique." depuis le sommet informel européen de salzbourg, en autriche, emmanuel macron a lancé ce jeudi un avertissement aux membres de l'ue qui manqueraient à leur devoir de solidarité. cette rencontre entre les 28 membres de l'union européenne était consacrée au brexit mais aussi à la crise migratoire. si le ton est monté concernant le premier point, d'après le chef de l'etat, les discussions sur les migrants ont été marquées par une meilleure entente. ce qui ne l'a pas empêché d'opter pour la fermeté lors de sa conférence de presse donnée dans l'après-midi. "il y a une crise et des tensions, mais qui les génère?", a-t-il interrogé. "ceux qui disent 'je ne veux pas respecter le droit humanitaire et le droit maritime international', quand un bateau arrive, que je suis le port le plus sûr et le plus proche je ne le prends pas', quand bien même nous avons divisé par dix les flux qui arrivent", a déploré emmanuel macron, en référence au refus du ministre italien de l'intérieur, matteo salvini, d'accueillir dans les ports italiens des navires humanitaires voguant en méditerranée, dont l'aquarius. la hongrie particulièrement visée "qui génère la crise? ce sont ceux qui expliquent 'moi je suis dans schengen, j'aime l'europe quand il faut toucher les fonds structurels, j'aime l'europe quand elle me donne de l'argent, quand elle permet la prospérité à mon peuple, quand elle permet à mes travailleurs d'aller mieux gagner leur vie dans des pays voisins, mais chez moi, pas un seul migrant ne sera pris, pas un seul réfugié qui a le droit à l'asile', ne respectant aucune règle", a-t-il poursuivi, en référence à certains pays de l'est. parmi eux, la hongrie, gouvernée par le premier ministre viktor orban. au plus fort de la vague migratoire de 2015, elle avait érigé une clôture à ses frontières avec la croatie et la serbie, ce dernier pays n'étant pas membre de l'ue. les relations orageuses du dirigeant hongrois avec ses partenaires européens se sont alourdies d'un contentieux supplémentaire depuis que, mercredi, les eurodéputés ont tiré la sonnette d'alarme sur le non-respect de l'état de droit et des valeurs européennes en hongrie. dans un vote inédit, le parlement européen a activé une procédure exceptionnelle pouvant conduire à des sanctions contre ce pays membre. "ils sortiront de schengen" c'est dans ce contexte de tensions qu'emmanuel macron, à la fin de la conférence de presse qu'il donnait jeudi après-midi, s'est montré fataliste, voire menaçant. "ceux qui créent la crise, ce sont ceux qui ensuite vous expliquent que l'europe ne sait pas se mettre d'accord, et à un moment donné, le règlement se fera simplement: comme ils ont poussé pour certains le brexit, les pays qui ne veulent pas davantage de frontex ou de solidarité sortiront de schengen. les pays qui ne veulent pas davantage d'europe, ils ne toucheront plus les fonds structurels", a lancé le chef de l'etat, ajoutant que les pays membres devraient avoir au printemps cette discussion "en vérité". lundi, viktor orban s'était opposé à la proposition de ses partenaires de faire de l'agence européenne de garde-frontières, surnommée frontex, une force susceptible de déployer ses effectifs aux frontières de l'ue. pour le premier ministre hongrois, cela constituerait une atteinte à la souveraineté du pays. & 649 & very low & Low & Power & NA & NA & 2018-09-20 & 2018 & 3 & POL
Frame & v.low & National & 500-1000 & -0.7708104 & -0.8786862 & 0.6041414 & -0.2983228 & 0.6486612 & 0.0 & -0.0275917 & -0.0818024 & Payer & European & European & European & European|POL & Neutral\\
France & https://www.lesechos.fr/idees-debats/editos-analyses/0301335067719-cette-nouvelle-guerre-est-ouest-qui-dechire-leurope-2156093.php\#Xtor=AD-6000 & 166 & LesEchos.fr & Private/Non-Public & Online and Offline & National & low = CP mentioned more times but NOT important part of story (mainly about others issues) & Political leverage & Factual & EU + Other country & No myth & NA & NA & NA & NA & NA & NA & NA & NA & France & cette nouvelle guerre est-ouest qui déchire l'europe & 2018-02-23 & fonds de cohésion & sur quinze pays d'europe de l'est, sept sont gouvernés par des populistes. leur projet de " contre-révolution culturelle ", essentiellement identitaire, vient heurter de front les valeurs défendues par l'europe de l'ouest depuis soixante ans. au risque de déstabiliser l'union européenne. konrad szymanski, le ministre polonais des affaires européennes, a menacé bruxelles cette semaine de " représailles énormes " si la commission persistait à vouloir priver de fonds de cohésion les pays qui ne respectent pas l'etat de droit. une procédure a été ouverte en décembre contre la pologne qui est accusée par les capitales de l'ouest de s'en prendre à l'indépendance de la justice et à la presse. la pologne pourrait bloquer les discussions sur vote du budget européen 2021-2028 post-brexit qui ont commencé et, plus largement, paralyser toutes les tentatives de " relance européenne ". populisme triomphant la pologne, où gouverne le parti populiste droit et justice, a plein d'amis. en république tchèque, le président, milos zeman, prorusse, a été réélu fin janvier face au " pro-européen " jiri drahos. en hongrie, le parti de viktor orban devrait remporter haut la main les législatives du 8 avril. sur quinze pays d'europe de l'est, les populistes détiennent le pouvoir dans sept, appartiennent à une coalition dirigeante dans deux de plus et sont la principale force d'opposition dans trois (1). le populisme, inexistant il y a deux décennies, triomphe, il s'affirme et compte faire entendre ce que le hongrois orban et le polonais kaczynski nomment la " contre-révolution culturelle ". " en 1989 [chute du mur], ici, en europe centrale, nous pensions que l'europe était notre avenir ; aujourd'hui, nous avons le sentiment d'être l'avenir de l'europe ", dit orban. illibéralisme économique le problème, pour les capitales de l'ouest, est que cet avenir-là s'écrit comme l'antithèse de leur projet depuis soixante ans et de leurs valeurs : une société fermée plutôt qu'ouverte, le nationalisme plutôt que la " souveraineté européenne ", comme la prône emmanuel macron, un illibéralisme économique revendiqué, et une culture traditionnelle de la famille et de l'eglise qui fait dénoncer la permissivité multiculturelle " de gauche ", à commencer par l'immigration. donald trump ne s'y est pas trompé, qui, à varsovie, a fait l'éloge de la pologne comme rempart de la civilisation occidentale " combattant pour la famille, la liberté, la patrie et dieu ". que peuvent faire paris, berlin, rome devant ce bloc des pays européens de l'est, aujourd'hui d'autant plus fermement soudés entre eux que la grande-bretagne n'est plus là ? la cassure est-elle inévitable ? dans une europe à deux vitesses, parce qu'en réalité deux projets, faut-il les retenir quand même ? comment les empêcher de se rapprocher de moscou avec qui ils ont tant d'aspects illibéraux en apparence communs ? capacité de blocage la priorité immédiate est d'inventer des mécanismes qui leur ôtent toute capacité de blocage d'une relance comme en menace varsovie. ce ne sera pas simple avec les traités actuels. mais, ensuite, il faut pouvoir faire d'un mal un bien. les pays de l'ouest ont aussi leurs populistes et pour que ces derniers ne soient pas " l'avenir " ici aussi, les partisans de l'ouverture démocratique et libérale feraient bien d'y regarder à deux fois. il est une différence majeure entre les populistes de l'est et ceux de l'ouest : l'économie. le taux de chômage n'est que de 2,9 \% en république tchèque, de 5 \% en hongrie, de 4,9 \% en pologne. les électeurs populistes n'y sont pas les " perdants " de la mondialisation comme en italie du sud ou au nord de la france. la pologne n'est pas non plus une " victime " de bruxelles, elle est la première bénéficiaire des fonds structurels, avec 80 milliards d'euros d'aides sur le budget actuel 2014-2020. c'est dire, sans doute, que le substrat profond du populisme n'est pas dans l'économie. les partis populistes de l'ouest qui en font leur ligne de bataille (l'ukip, le fn, la france insoumise) se trompent. l'autre différence est-ouest tient au passé : ces pays n'ont, hélas pour eux, que rarement connu la démocratie et leurs institutions sont fragiles. un trump ne s'attaque pas à la justice, kaczynski et orban n'hésitent pas. identité menacée ce qui unit les populistes est leur combat pour l'identité culturelle. la résistance contre les élites n'est pas économique mais identitaire, ces élites sont responsables d'une dilution des valeurs traditionnelles du " peuple ", elles sont au service des étrangers. la nation doit être défendue parce qu'elle est le lieu de cette identité menacée. la frontière n'est pas tant l'arme de défense de l'emploi et des intérêts matériels que la barrière " entre eux et nous ". ce besoin identitaire est plus fort à l'est, parce que ces pays ne se sont pas constitués par l'etat, comme ce fut le cas à l'ouest, mais justement comme communautés culturelles, souligne justement le politologue zaki laïdi. et cette identité s'est trempée dans l'histoire : elle a été capable de résister à soixante-dix ans de domination soviétique. dérive nationaliste pourquoi la gauche et les libéraux n'ont-ils pas été capables de s'opposer dans ces pays à la dérive nationaliste et conservatrice ? demande le politologue jacques rupnik (2). c'est la bonne question qui éclaire ce que doivent faire les capitales ouest-européennes. il ne sert à rien de dénoncer le " projet " d'orban et de kaczynski comme raciste, antidémocratique, superficiel et illusoire. il plaît parce qu'il a une consistance. en face, l'ouverture des frontières et le multiculturalisme ne peuvent pas emporter l'adhésion contre la société fermée. l'objectif est trop vague, le cadre sans limite, le but trop imprécis, le tout laisse grande ouverte la porte aux peurs. il est urgent de constituer autrement solidement le projet d'une europe ouverte, humaniste, multiculturelle. (1) " how eastern european populism is different ", slawomir sierakowski. project syndicate, 31 janvier. (2) revue " commentaire ", hiver 2017, no 160. & 1026 & low & Low & Power & NA & NA & 2018-02-23 & 2018 & 3 & POL
Frame & low-medium & National & +1000 & -0.7708104 & -0.8786862 & 0.6041414 & -0.2983228 & 0.6486612 & 0.0 & -0.0275917 & -0.0818024 & Payer & European & European & European & European|POL & Neutral\\
France & http://www.lepoint.fr/politique/emmanuel-berretta/macron-juge-le-budget-juncker-10-05-2018-2217375\_1897.php & 175 & Le Point & Private/Non-Public & Online and Offline & National & low = CP mentioned more times but NOT important part of story (mainly about others issues) & Ineffective goal achievement & Balanced & EU & No myth & NA & NA & NA & NA & NA & NA & NA & NA & France & macron juge le budget juncker & 2018-05-10 & fonds de cohésion & le président français et ses conseillers ont épluché la proposition budgétaire de la commission juncker. bilan : des plus et des moins... contrairement à d'autres leaders européens, emmanuel macron n'a pas rejeté en bloc les propositions budgétaires de la commission juncker. si le ministère de l'agriculture a été le plus prompt a dénoncé la réduction proposée des paiements directs aux agriculteurs, l'élysée a une vision plus nuancée du cadre financier multiannuel - 2021/2027 - mis sur la table par jean-claude juncker en amorce d'une longue négociation intérétatique. il se réserve de plus amples commentaires quand le détail des financements des politiques sera dévoilé fin mai par la commission européenne. à ce stade, le président français a pris acte des "points positifs". à commencer par une "architecture plus claire" qui rend "plus lisible" les politiques européennes suivies, notamment les nouvelles missions sécurité, gestion des frontières, innovation... la france salue le doublement des fonds nécessaire au déploiement du programme erasmus + qui permet les échanges entre jeunes éuropéens. conformément à ce que l'élysée avait suggéré, la commission propose une fin progressive - étalée sur 5 ans - de la politique des rabais qui permettent à l'heure actuelle à l'allemagne, à l'autriche, au danemark, aux pay-bas et à la suède d'obtenir divers ristournes sur le montant réel de leur contribution. toutefois, pour macron, la commission n'est pas allée assez loin. "les rabais doivent s'éteindre plus rapidement", insiste l'élysée qui considère ces ristournes comme "une anomalie historique" générée par le royaume-uni. lire aussi budget européen : le grand marchandage commence de la même façon, paris considère que conditionner le versement des fonds de cohésion (aux pays les plus pauvres) au "respect des valeurs européennes" n'accomplit qu'une partie du chemin. "la commission a ouvert le débat mais là encore, la conditionnalité du versement doit prendre en compte le comportement pratique, économique, fiscal et social des états-membres", selon l'élysée. autrement dit, la pologne et la hongrie - mises en cause pour leur atteinte à l'état de droit - ne doivent pas être les seuls à être rappelées à l'ordre. dans l'esprit du président macron, les états membres, qui permettent, par exemple, aux géants du numérique d'opérer une optimisation fiscale agressive, doivent aussi se voir conditionner les subsides européens. du reste, la politique de cohésion mériterait, selon l'élysée, une réorientation plus nette. l'objectif du "rattrapage économique" des pays de l'est ne suffit plus. emmanuel macron souhaiterait que d'autres critères soient pris en compte afin de soutenir les régions les plus atteintes "par le chômage des jeunes" ou "les plus impactées par le brexit" (l'irlande, le port de rotterdam...). ce faisant, on comprend bien qu'une partie plus importante des fonds européens reviendrait à l'ouest... lire aussi chirac, sarkozy, hollande, macron : quatre présidents à l'est sur la pac, emmanuel macron avait admis qu'une "modernisation" de celle-ci était souhaitable, notamment si elle engendrait des économies. "mais il ne peut y avoir de baisses des revenus des agriculteurs ", souligne l'elysée qui considère, en outre, que les réductions budgétaires ne sauraient être un "sacrifice de la souveraineté alimentaire de l'europe". lire aussi europe : la métamorphose de la politique agricole commune à la question cruciale, "la france est-elle prête à augmenter sa contribution financière ?", l'élysée répond clairement : "oui, si nos préoccupations sont prises en compte, si les rabais cessent et si la conditionnalité d'un bon comportement devient effective". rappel : le budget européen doit être adoptée à l'unanimité par le conseil européen des chefs d'état et de gouvernement. ce qui laisse entrevoir une négociation très tendue... & 615 & low & Low & Socio-Economic & NA & NA & 2018-05-10 & 2018 & 3 & ECO
Frame & low-medium & National & 500-1000 & -0.7708104 & -0.8786862 & 0.6041414 & -0.2983228 & 0.6486612 & 0.0 & -0.0275917 & -0.0818024 & Payer & European & European & European & European|ECO & Neutral\\
France & http://www.lexpress.fr/actualite/monde/europe/pourquoi-la-hongrie-et-la-pologne-se-permettent-de-defier-les-regles-de-l-ue\_1917833.html & 105 & LExpress.fr & Private/Non-Public & Online and Offline & National & medium = CP is important part of story & Political leverage & Factual & EU + Other country & No myth & Solidarity to poor countries/regions & Factual & EU + Other country & No myth & NA & NA & NA & NA & France & pourquoi la hongrie et la pologne se permettent de défier les règles de l'ue & 2017-06-14 & fonds structurels & l'ue vient de lancer une procédure contre budapest, varsovie et prague pour leur refus d'accueillir des demandeurs d'asile. jusqu'où peuvent aller ces pays dans leur défiance? "l'europe, ce n'est pas seulement pour demander des financements". dimitris avramopoulos, commissaire européen aux migrations lançait un avertissement, ce mardi, en annonçant le lancement d'une procédure d'infraction contre trois pays qui refusent la réinstallation de demandeurs d'asile au sein de l'ue: la hongrie, la pologne et la république tchèque. la commission fait pression depuis des mois sur ces pays pour qu'ils fassent preuve de solidarité dans le partage du fardeau migratoire. en vain. et le niet de budapest et varsovie n'est que l'un des nombreux sujets de contentieux avec bruxelles. lire aussi >> migrants à calais: "des conditions de vie inhumaines", dénonce jacques toubon la hongrie fait par ailleurs l'objet, depuis la fin du mois d'avril, d'une procédure d'infraction pour sa récente loi sur les universités, qui menace de fermeture un établissement financé par le milliardaire américain libéral george soros, bête noire du premier ministre viktor orban. migrants, universités, ong, constitution, les multiples griefs de bruxelles après les coups portés par budapest et varsovie contre les médias, l'ue ferraille aussi contre la loi adoptée mardi par le parlement hongrois qui renforce le contrôle gouvernemental sur les ong bénéficiant de fonds étrangers. elle les oblige à se présenter explicitement comme "organisation bénéficiant de financements étrangers". "la hongrie rejoint ainsi une série de pays comme la russie, la chine, israël qui considèrent que le financement étranger d'ong [...] est un acte hostile ou du moins inamical", s'est ému le porte-parole du ministère allemand des affaires étrangères, ce mercredi. de son côté, la pologne a rejeté toutes les demandes de la commission de revenir sur ses réformes controversées de la justice constitutionnelle. "la perte d'autonomie du tribunal constitutionnel polonais met pourtant en cause l'équilibre des pouvoirs propre à une démocratie", explique à l'express marie-laure basilien gainche, professeur de droit public à l'université lyon 3. quels moyens de rétorsion? en théorie, les traités européens prévoient des mesures contre les pays qui ne respecteraient pas ses valeurs. dans la pratique, c'est une autre histoire. en cas de violation grave et persistante des principes fondamentaux de l'ue, la sanction peut aller jusqu'à la suspension du droit de vote. mais "comme on l'a vu en 2000 lors de l'arrivée au pouvoir de l'extrême droite en autriche, elle est quasiment inapplicable", constate marie-laure basilien. qu'il s'agisse de manquements liés au fonctionnement démocratique des pays concernés ou du non-respect d'une décision prise par les 28, comme dans le cas de la relocalisation des demandeurs d'asile, les instruments à disposition sont limités. complexe, la procédure d'infraction, entamée ce mercredi, passe par une mise en demeure suivie d'une injonction de se conformer au droit européen. en cas de refus d'obtempérer, la commission peut saisir la cour de justice. la plupart des cas sont réglés avant d'en arriver-là. en cas de blocage, la cour la commission peut demander des sanctions. une fois les différents recours activés par la commission, les mesures de rétorsion relèvent du conseil européen. "parvenir à des sanctions relève avant tout d'une décision politique, observe christian lequesne, chercheur au ceri-sciences-po. plusieurs pays européens ne sont pas favorables à l'adoption d'une ligne dure. parce qu'ils estiment que cela ne fait que renforcer le discours des nationalistes". l'impuissance de bruxelles il est peu probable que les chefs d'état européens parviennent à l'unanimité requise pour sanctionner les récalcitrants. "et si cela devait arriver, cela n'interviendrait pas sur la question de l'accueil des réfugiés, qui divise beaucoup", ajoute susi dennison, spécialiste de l'union européenne au conseil européen des relations internationales (ecfr). "bien d'autres des agissements du gouvernement orban auraient pu justifier de déclencher des procédures d'infraction, complète marie-laure basilien. mais les rapports de force politiques en vigueur dans l'ue compliquent les choses: le parti populaire européen (ppe, droite) a besoin des voix du fidesz de viktor orban au sein du parlement européen, ce qui peut expliquer la lenteur de la réaction à ses agissements." le mode de fonctionnement de l'ue privilégie la négociation, poursuit la juriste: "l'idée est d'accompagner les états en infraction pour qu'ils reviennent sur leurs décisions sans perdre la face. " lier le respect des engagements aux fonds structurels? des députés européens sont plus virulents. certains évoquent la possibilité de corréler la solidarité avec l'accès aux fonds structurels et d'investissement. la pologne et la hongrie bénéficient beaucoup plus des fonds européens qu'ils n'y contribuent: 13,3 milliards d'euros de crédits pour la première, contre 3,7 milliards de contribution en 2015, 5,6 milliards de crédits pour la hongrie, contre 945 millions de contribution. ce qui fait dire à jean-claude juncker, président de la commission, que "les décisions prises sont applicables, même si vous avez voté contre. c'est l'essence de la solidarité européenne, qui ne peut fonctionner à sens unique." "l'allemagne elle-même, au plus fort de la crise des migrants, avait évoqué la possibilité de lier l'accueil de migrants à l'accès aux fonds européens", rappelle susi dennison. mais il est peu probable que les choses bougent vraiment, en particulier dans le contexte du brexit. "alors que vont démarrer de complexes négociations sur la sortie du royaume-uni de l'ue, note christian lequesne, les 27 ne vont pas avoir envie d'ajouter un autre front de crise." les démocraties 'illibérales' -des gouvernement issus d'élections démocratiques qui s'affranchissent des règles démocratiques- ont bien compris cette faiblesse intrinsèque du club européen: on peut en être membre sans en respecter les règles. et ce n'est pas prêt de changer, même si cela en sape peu à peu les fondements. & 1007 & medium & Medium & Power & Values & NA & 2017-06-14 & 2017 & 2 & POL
Frame & low-medium & National & +1000 & -0.7708104 & -0.8786862 & 0.6041414 & -0.2983228 & 0.6486612 & 0.0 & -0.0275917 & -0.0818024 & Payer & European & European & European & European|POL & Neutral\\
France & http://www.linternaute.com/actualite/depeches/1451336-budget-de-l-ue-coups-de-rabot-en-vue-apres-le-brexit/ & 108 & L'Internaute & Private/Non-Public & Online only & National & medium = CP is important part of story & Financial burden & Balanced & EU + National + Subnational & 1.Poor regions funded only & Political leverage & Factual & EU + Other country & No myth & NA & NA & NA & NA & France & budget de l'ue: coups de rabot en vue après le brexit & 2018-05-02 & fonds de cohésion & bruxelles dévoile mercredi son plan pour bâtir les budgets post-brexit de l'ue, avec des propositions détonantes comme de couper dans les politiques agricole et régionale ou de conditionner le versement de fonds européens au respect de l'etat de droit. après des mois de préparation et de ballons d'essai, la commission va mettre sur la table un cocktail d'économies et de nouvelles ressources pour que l'union ait les moyens des ambitions affichées pour sa nouvelle vie à 27, sans le royaume-uni. l'exécutif européen veut que les tractations entre etats membres et parlement européen pour la période 2021-2027 soient bouclées avant les prochaines élections européennes, soit moins de deux mois après le divorce avec les britanniques prévu le 30 mars 2019. "ce genre de négociations prend normalement deux ans", souligne une source diplomatique, perplexe face à ce calendrier, alors que l'équation du "cadre financier pluriannuel" (cfp) de l'ue n'a jamais semblé aussi complexe. selon les estimations de bruxelles, le départ du royaume-uni va laisser un "trou" annuel de 12 à 14 milliards d'euros après 2020 -- dernière année de contribution de londres malgré un brexit programmé en cours d'année précédente. la rupture avec ce "contributeur net" tombe d'autant plus mal que l'union européenne cherche à financer de nouvelles politiques, en matière de défense ou de migration notamment, sans renoncer aux "anciennes". ce qui nécessiterait un budget plus important que celui de 1.000 milliards d'euros fixé pour la période 2014-2020. -la pac visée- le commissaire au budget, l'allemand günther oettinger, veut que les 27 acceptent un budget au-delà de la limite actuelle de 1\% du revenu national brut (rnb) cumulé des etats membres. il faudrait "entre 1,1 et 1,2\%", a-t-il récemment estimé. "il va falloir faire des coupes", a prévenu m. oettinger en visant la politique agricole commune (pac) et la politique de cohésion pour les régions les plus en retard économiquement, deux domaines représentant respectivement 37\% et 35\% du budget de l'ue. la commission proposera "des réductions modérées", "en-dessous de dix pour cent", selon une source européenne. elles seront néanmoins difficiles à accepter, en particulier en france, dont les agriculteurs sont les principaux bénéficiaires des aides directes de la pac. paris est prête à défendre une "réforme assez substantielle", mais "le filet de sécurité indispensable des aides directes pour les agriculteurs ne peut pas être affecté", prévient une source diplomatique. les pays de l'est sont eux déjà vent debout face aux coupes dans les fonds de cohésion, dont ils sont les principaux destinataires, et qui pourraient par ailleurs être en partie réorientés vers d'autres pays connaissant un fort chômage des jeunes ou des "fractures territoriales". la pologne et la hongrie sont d'autant plus sur la défensive qu'elles se sentent visées par un autre projet de la commission, qui veut lier versement de fonds européens et respect de l'etat de droit. -'pression politique'- plusieurs pays réclament ce mécanisme pour tirer les leçons du bras de fer infructueux entre bruxelles et le gouvernement ultra-conservateur polonais, qui est accusé notamment de menacer l'indépendance de sa justice. face à la lourdeur de la procédure en cours lancée par la commission, l'idée est de pouvoir recourir à la pression financière dans des cas comparables. "nous n'accepterons pas de mécanismes arbitraires qui feront de la gestion des fonds un instrument de pression politique à la demande", a déjà averti le vice-ministre polonais pour les affaires européennes, konrad szymanski. des pays comme l'autriche ou les pays-bas sont déjà mobilisés pour leur part contre une hausse des contributions nationales, à laquelle l'allemagne et la france sont en revanche disposées. la commission va plaider pour la nécessité de fonds plus importants pour le numérique, la recherche, la défense ou encore la protection des frontières extérieures, avec une proposition de "plus que quintupler" les effectifs de l'agence frontex après 2020, pour les porter à près de 6.000 selon une source européenne. le débat budgétaire, qui nécessitera in fine une décision unanime des pays européens, va enfin ressusciter un serpent de mer: la création de nouvelles ressources propres pour l'ue. la commission veut notamment que la taxation des échanges de quotas de carbone soit orientée vers le budget européen et la création d'une taxe sur les plastiques est dans les tuyaux. & 742 & medium & Medium & Values & Power & NA & 2018-05-02 & 2018 & 3 & ECO
Frame & low-medium & National & 500-1000 & -0.7708104 & -0.8786862 & 0.6041414 & -0.2983228 & 0.6486612 & 0.0 & -0.0275917 & -0.0818024 & Payer & Domestic & European & Mixed & Domestic|ECO & Neutral\\
\addlinespace
France & http://www.lemonde.fr/europe/article/2018/01/16/budget-europeen-feroce-bataille-en-vue-a-bruxelles\_5242457\_3214.html & 161 & Le Monde.fr & Private/Non-Public & Online and Offline & National & low = CP mentioned more times but NOT important part of story (mainly about others issues) & Financial burden & Factual & EU + Other country & No myth & NA & NA & NA & NA & NA & NA & NA & NA & France & budget européen : féroce bataille en vue à bruxelles & 2018-01-16 & fonds de cohésion & la commission européenne ne présentera qu'en mai ses propositions pour le " cadre financier pluriannuel " (cfp), ce budget commun censé financer des politiques entre 2021 et 2027. mais la bataille a déjà commencé, et elle promet d'être féroce. boïko borissov, le premier ministre bulgare, dont le pays récupère la présidence tournante de l'ue jusqu'au 1er juillet, s'est fait le porte-voix des pays ayant largement profité des fonds de cohésion et qui espèrent bien préserver ces milliards nécessaires pour rattraper leur retard sur les économies de l'ouest. " je ferai de mon mieux pour qu'il n'y ait pas de réduction drastique de la politique de cohésion. sinon, nous nous sentirons sanctionnés ", a prévenu, jeudi 11 janvier, le conservateur, dont le pays est le plus pauvre de l'ue. renégocié tous les cinq à sept ans, le cfp donne toujours lieu à d'intenses batailles : pour parvenir au cfp 2014-2020, deux années pleines furent nécessaires. mais la discussion qui vient s'annonce plus brutale que tout ce qu'ont connu les experts bruxellois. deux nouvelles contraintes entrent en jeu, au-delà du fait qu'un cfp s'adopte à l'unanimité des etats membres. le brexit, d'abord : avec le départ britannique, le budget de l'union (surtout composé des contributions directes des etats) sera amputé d'au moins 12 milliards d'euros annuels (8 \%). dans le même temps, le cfp actuel s'est révélé trop rigide pour s'adapter aux nouvelles priorités apparues avec la crise des réfugiés et la montée des risques géopolitiques. il ne finance pas assez la défense, la sécurité, la migration, la jeunesse et le numérique. dilemme il repose encore sur deux grosses " enveloppes " : la politique agricole commune (pac, 39 \% des montants pour 2014-2020) et les fonds de cohésion (26 \%). les dirigeants européens vont devoir faire face à un dilemme : s'ils décident de préserver ces deux " piliers " tout en consacrant plus d'argent à leurs nouvelles... & 329 & low & Low & Values & NA & NA & 2018-01-16 & 2018 & 3 & ECO
Frame & low-medium & National & <500 & -0.7708104 & -0.8786862 & 0.6041414 & -0.2983228 & 0.6486612 & 0.0 & -0.0275917 & -0.0818024 & Payer & European & European & European & European|ECO & Neutral\\
France & http://www.rtl.fr/actu/societe-faits-divers/le-nationalisme-industriel-gagne-partout-en-europe-7785259713 & 137 & RTL.fr & Private/Non-Public & Online only & National & low = CP mentioned more times but NOT important part of story (mainly about others issues) & Solidarity to poor countries/regions & Negative & Other country & No myth & NA & NA & NA & NA & NA & NA & NA & NA & France & le nationalisme industriel gagne partout en europe & 2016-10-13 & fonds structurels & édito - c'est la crise entre paris et varsovie, à cause d'un contrat d'armement malheureux. le groupe airbus était en lice pour fournir à varsovie cinquante hélicoptères militaires, pour une valeur de plus de 3 milliards d'euros. le contrat était quasiment signé. et voilà que les élections législatives amènent un nouveau gouvernement au pouvoir, issu du parti droit et justice, le parti nationaliste. celui-ci a purement et simplement annulé le contrat, mettant en fureur et airbus et le gouvernement français. françois hollande a même annulé sa visite officielle en pologne. quel est le fond de l'affaire ? à en croire les polonais, c'est la part du contrat fabriquée en pologne. les hélicoptères seront attribués à l'américain lockheed, qui a des usines en pologne, comme par hasard dans des circonscriptions du parti au pouvoir. des manifestations avaient même été organisées pour refuser l'attribution du contrat au franco-allemand. airbus rétorque toutefois qu'il avait fait des concessions très importantes, en matière de fabrication locale et de transfert de technologie. l'européen s'indigne de la méthode des polonais pour signifier le refus. "jamais nous n'avons été traités ainsi", a dit tom enders, le patron de l'avionneur. on a le droit en europe d'exiger de telles contreparties pour un contrat. pour l'armement, c'est l'usage. notez bien que le gouvernement français ne fait pas très différent, sur le fond, quand il parle de réserver les commandes de train aux industriels français. cette affaire est le dernier épisode en date du nationalisme industriel qui gagne partout, jusqu'au sein du marché unique. cela témoigne aussi de la dérive de la pologne. plus le temps passe, plus varsovie s'éloigne de l'europe au plan politique. le nouveau gouvernement est férocement anti-européen, tout comme l'opinion publique. cette affaire montre que l'idée d'une défense européenne intégrée est une chimère c'est un paradoxe, car la pologne est le premier bénéficiaire des fonds structurels communautaires. il y en a pour plusieurs dizaines de milliards d'euros sur la période 2014-2020. il a bien sûr accès au grand marché unique. varsovie a encore refusé de prendre son quota de réfugiés syriens, en ne voulant pas suivre les directives communes. un nouveau paradoxe, alors que le pays expédie des bataillons de travailleurs détachés dans les pays de l'ouest. "les immigrés, c'est bien quand on les envoie chez les autres, mais pas quand ils arrivent chez soi" : c'est ce que semblent dire les polonais. vingt ans après la chute du mur, la greffe n'a pas pris. à l'époque, l'europe était la bouée de sauvetage après la chute des communistes. c'est aujourd'hui un repoussoir pour les polonais, mais aussi pour les hongrois, également gouvernés par des nationalistes. l'affaire des hélicoptères montre également que l'idée d'une défense européenne intégrée, assise sur les industriels du continent, est une chimère, au moins pour ces nouveaux membres. elle n'a pas résisté, ni aux poussées nationalistes, ni à la pression politique des américains. c'était pourtant l'un des axes de renouveau du projet européen après le brexit, choisi par les chefs d'état lors d'un sommet récent. cet échec tombe mal pour les usines airbus de marignane, dans le sud de la france, dont le plan de charge n'est pas si rempli. à moins que les polonais ne se ravisent. ils ont annoncé mercredi 12 octobre un "lot de consolation" pour airbus. tout à fait indéterminé. & 593 & low & Low & Values & NA & NA & 2016-10-13 & 2016 & 2 & ECO
Frame & low-medium & National & 500-1000 & -0.7708104 & -0.8786862 & 0.6041414 & -0.2983228 & 0.6486612 & 0.0 & -0.0275917 & -0.0818024 & Payer & European & European & European & European|ECO & Negative\\
France & http://www.europe1.fr/international/refugies-berlin-menace-de-sanctions-juridiques-des-pays-de-lunion-europeenne-2638721 & 131 & Europe 1 & Public & Online only & National & very low = CP mentioned once & Solidarity to poor countries/regions & Negative & Other country & No myth & NA & NA & NA & NA & NA & NA & NA & NA & France & réfugiés : berlin menace de sanctions juridiques des pays de l'union européenne & 2015-12-19 & fonds structurels & le ministre allemand des affaires étrangères, frank-walter steinmeier, vise en particulier la slovaquie et la hongrie. le ministre allemand des affaires étrangères, frank-walter steinmeier, a menacé samedi de sanctions juridiques les pays de l'union européenne qui refusent d'accueillir des réfugiés dans le cadre d'un programme de quotas de répartition. adopté en septembre, ce projet veut répartir 120.000 réfugiés dans l'union européenne, mais plusieurs pays d'europe centrale y sont cependant toujours opposés. la slovaquie et la hongrie dans le viseur. "l'europe est une communauté de droit", a avancé frank-walter steinmeier, dans un entretien à l'hebdomadaire allemand der spiegel et "la parole donnée compte". "si on ne peut pas faire autrement, les choses seront réglées par les voies juridiques prévues à cet effet", a-t-il menacé. le chef de la diplomatie vise notamment la slovaquie et la hongrie, qui ont porté plainte auprès de la justice européenne contre les quotas de répartition. le ministre allemand rappelle à ces pays que "la solidarité européenne n'est pas une voie à sens unique" : "ceux qui refusent (d'accueillir ces réfugiés) doivent savoir ce qui est également en jeu pour eux : des frontières ouvertes en europe", a-t-il souligné. frank-walter steinmeier n'est pas le premier à proférer des menaces contre les pays récalcitrants. en septembre, le ministre de l'intérieur allemand, thomas de maizière, évoquait déjà la possibilité de réduire les fonds structurels versés à ces pays par l'union européenne. & 251 & very low & Low & Values & NA & NA & 2015-12-19 & 2015 & 1 & ECO
Frame & v.low & National & <500 & -0.7708104 & -0.8786862 & 0.6041414 & -0.2983228 & 0.6486612 & 0.0 & -0.0275917 & -0.0818024 & Payer & European & European & European & European|ECO & Negative\\
France & https://www.la-croix.com/JournalV2/Roumanie-mauvais-eleve-commandes-lUE-2018-12-31-1100992407 & 189 & La Croix & Private/Non-Public & Online and Offline & National & very low = CP mentioned once & Political capital/interests & Factual & Other country & No myth & NA & NA & NA & NA & NA & NA & NA & NA & France & la roumanie, un mauvais élève aux commandes de l'ue & 2018-12-30 & fonds structurels & manifestation contre la réforme de la justice engagée par le psd de liviu dragnea (sur la pancarte), à bucarest, en mai 2018. / vadim ghirda/ap un semestre tumultueux attend l'union européenne. à cause du brexit, le 29 mars, et des prochaines élections européennes, fin mai, bien sûr. mais aussi parce que l'ue sera pilotée dès le mois de janvier 2019 par la roumanie. or ce pays d'europe centrale, alors qu'il arrive à la présidence du conseil pour la première fois, enchaîne les provocations contre bruxelles, qui n'a pas besoin de cela dans cette période où les eurosceptiques multiplient déjà leurs attaques. jean-claude ­jun­cker a exprimé ses doutes le 29 décembre, affirmant que " le gouvernement de bucarest n'a pas encore compris ce que signifie présider les pays de l'ue ". depuis quelques semaines, le parti social-démocrate (psd) au pouvoir en roumanie, fermement dirigé par liviu dragnea, tente d'imposer une vaste réforme du système judiciaire qui est au cœur du mécontentement de bruxelles. s'il est adopté, ce projet permettrait l'amnistie et la grâce d'individus accusés de corruption. une mesure susceptible de bénéficier à des membres éminents du psd ayant un casier judiciaire, dont dragnea lui-même, et que dénonce fermement l'opposition. " l'adoption de cette réforme, dont le but est de blanchir liviu dragnea, remettrait complètement en cause l'indépendance de la justice roumaine ", constate traian sandu, professeur à la sorbonne et spécialiste de la roumanie. face à cette menace, bruxelles tente depuis plusieurs semaines d'obtenir l'abandon de la réforme. en novembre 2018, le parlement européen a ainsi voté une résolution dans laquelle il se dit " très inquiet face à la refonte de la législation régissant le système judiciaire et le système pénal roumains ". en réaction, le psd a fait des institutions européennes ses cibles privilégiées. pour certains de ses membres, la résolution constitue " un document injuste, montrant que l'union européenne utilise deux poids, deux mesures ". bucarest est même allé jusqu'à évoquer à demi-mot la possibilité d'un recours devant la cour de justice de l'ue. si les relations entre bucarest et bruxelles sont actuellement des plus tendues, elles pourraient toutefois s'apaiser avec les débuts de la présidence roumaine du conseil. " la roumanie n'a pas intérêt à se brouiller avec l'ue dans les prochains mois, analyse christian lequesne, professeur à sciences-po et spécialiste des questions européennes. tout d'abord, le gouvernement tient un minimum à son image de formation sociale-démocrate. de plus, les autorités roumaines ont besoin des fonds structurels alloués par l'union. sur la période allant de 2014 à 2020, l'enveloppe est tout de même de 30,9 milliards d'euros. " de son côté, bruxelles serait également dans une position inconfortable si la discorde avec la roumanie s'intensifiait. " la roumanie assurant la présidence du conseil, il serait très compliqué d'engager de véritables sanctions si elle allait au bout de ses réformes, pour des questions de formalités ", explique encore christian lequesne. un avis partagé par william thay, président du laboratoire d'idées le millénaire, spécialisé en politiques publiques et dans la refondation idéologique de la droite. " la roumanie va sûrement continuer à tenter d'affirmer sa souveraineté sans pour autant provoquer bruxelles de manière trop frontale ", prévoit-il. actuellement, la roumanie, qui célébrera à la fin 2019 le 30e anniversaire de la chute du régime de ceausescu, est un pays extrêmement divisé. en témoigne l'affrontement politique entre klaus iohannis, le président de centre droit, et son gouvernement emmené par le psd. " le débat autour de la réforme de la justice a un caractère presque symbolique, détaille jean-arnault dérens, rédacteur en chef du média en ligne le courrier des balkans. il met en avant les fractures de la société roumaine, l'une des plus polarisées d'europe. " et ces fractures sont autant idéologiques qu'économiques. en effet, si le pib par habitant représente environ 140 \% de la moyenne de l'ue dans la région de bucarest, il atteint à peine 40 \% dans les régions rurales les plus démunies du pays. & 688 & very low & Low & Power & NA & NA & 2018-12-30 & 2018 & 3 & POL
Frame & v.low & National & 500-1000 & -0.7708104 & -0.8786862 & 0.6041414 & -0.2983228 & 0.6486612 & 0.0 & -0.0275917 & -0.0818024 & Payer & European & European & European & European|POL & Neutral\\
France & http://www.liberation.fr/debats/2016/10/17/espagne-et-portugal-l-europe-de-la-sanction\_1522522 & 138 & Libération & Private/Non-Public & Online and Offline & National & medium = CP is important part of story & Economic development & Positive & EU + Other country & No myth & NA & NA & NA & NA & NA & NA & NA & NA & France & espagne et portugal : l'europe de la sanction & 2016-10-17 & fonds structurels & les deux etats de la péninsule ibérique risquent de se voir couper l'accès aux fonds structurels européens. un signe de plus que l'ue s'oublie, obnubilée par des règles budgétaires que nul ne parvient plus à respecter. et qui risquent de sacrifier les populations du sud. tandis que la croissance et l'investissement sont en berne en europe, l'espagne et le portugal s'apprêtent à se voir couper l'accès aux fonds structurels européens, ces fonds dédiés au financement des projets partout dans les territoires des pays de l'union. l'europe s'oublie. dans son fonctionnement nombriliste, éprise de règles budgétaires que personne n'est en mesure de justifier, elle oublie les raisons mêmes pour lesquelles une "gouvernance" économique européenne devrait être mise en œuvre. la prospérité économique des etats et le bien-être de leurs populations semblent avoir complètement disparu de son radar. dernier exemple en date, cette punition de l'espagne et du portugal. les deux etats n'ont pas suffisamment réduit leur déficit public comme le préconisent les règles européennes. au regard de l'application toujours incompréhensible de celles-ci, la commission et l'eurogroupe de jeroen dijsselbloem ont apparemment décidé qu'il fallait faire un exemple. le critère pour déterminer la bonne volonté des deux etats est une notion floue qui ne devrait pas avoir sa place dans du droit écrit : c'est celle d'"effective action". en français, on la traduit par "action suivie d'effet". dans la novlangue bruxelloise, cela signifie peu ou prou que vous n'avez pas pris les "mesures nécessaires" pour réduire votre déficit. peu importe alors à la commission que le portugal sorte à peine d'un plan d'austérité de plusieurs années qui a ravagé les conditions de vie des portugais - dont le revenu moyen par tête est maintenant inférieur de 23 \% à la moyenne des 28 pays de l'union ! que l'espagne traverse une crise politique et institutionnelle sans précédent, sans gouvernement légitime depuis près d'un an, ne semble pas retenir non plus la commission et l'eurogroupe dans leurs décisions. voilà deux pays qui ont un besoin vital de marges de manœuvre et que l'on devrait mettre dans les meilleures conditions pour relancer leurs investissements et leur consommation. mais à la raison des gens, les institutions européennes préfèrent leur propre raison d'être : le droit qui les a créées. même quand celui-ci est manifestement contraire à l'intérêt des citoyens européens. car l'europe dans son ensemble ne peut pas se permettre d'ajouter à la situation de la grèce la perspective d'une austérité renouvelée en espagne et au portugal. a tout point de vue, économique, financier, social et politique, supprimer les fonds européens dans une logique de sanction aux deux pays de la péninsule ibérique constituerait une erreur grave. d'abord, cela inverserait complètement la logique des fonds en question : les "fonds structurels" sont le bras armé de la politique commune de cohésion censée aider à la convergence entre les économies. ensuite on voit mal comment supprimer cet outil de financement capital pour les projets des entreprises, des communes et des régions, aiderait à relancer l'investissement... enfin on assisterait encore à ce spectacle affligeant où les institutions européennes enverraient d'elles-mêmes un message négatif aux marchés financiers sur l'un de leurs membres. en pareil cas, comme d'habitude, la logique délétère des problèmes de financement s'enclenche et le portugal, à qui l'on reproche presque ouvertement d'avoir élu un gouvernement de gauche, devra demander un nouveau prêt contre un plan d'austérité piloté par la troïka ; tandis que dans les deux pays la consommation et l'investissement seront anesthésiés et les inégalités renforcées. voilà le résultat, déjà observable en grèce, de ce que le sabir technocratique européen appelle la "macro-conditionnalité". la traduction en langage commun de ce principe, c'est le proverbe "on ne prête qu'aux riches". comme pour les entreprises : les pme et les entreprises de taille intermédiaire nationales qui auraient le plus besoin d'accès au crédit bancaire sont celles à qui les banques prêtent le moins. l'ue n'est pas loin de se comporter comme une banque puisque ce sont les etats qui ont le plus besoin de dispositifs européens qui ont l'accès le plus restreint à ces outils. et les commissaires jyrki katainen et corina cretu, devant le parlement européen, ont le culot d'appeler cela une "incitation". on voit pourtant mal la différence avec un employeur qui pourrait décider d'une retenue sur le salaire d'un employé - par mesure d'incitation ! la bataille pour l'espagne et le portugal est donc cruciale. et contrairement aux préjugés faciles, ce n'est pas par réflexe méditerranéen. la guerre économique froide qui s'installe en europe et la divise - très médiatiquement - en europe du sud versus europe du nord et mitteleuropa, ces regroupements ne sont pas la cause ou l'origine des divisions politiques observables, mais certainement un symptôme, une conséquence du fonctionnement de l'ue et de ses institutions. c'est parce que l'europe fonctionne autour de son plus petit dénominateur commun, ses règles financières (d'ailleurs décidées hors des traités fondateurs), qu'elle se divise en deux, entre ceux qui accumulent l'épargne et sont prêts à sacrifier leurs infrastructures, le pouvoir d'achat de leurs citoyens, et ceux pour qui épargner davantage serait un suicide économique, parce que la consommation soutient leur économie et les exportations des voisins. il y a bel et bien un déni de réalité dans ces règles qu'on regroupe sous le nom de "semestre européen". un déni de ces millions de jeunes que leurs si longues périodes d'inactivité risquent de rendre inemployables. un déni de la déflation latente, qui renchérit le coût du désendettement lui-même. un déni des inégalités, qui fait que même la croissance espagnole ne débouche sur aucune amélioration de la situation des ménages et des pme. alors soutenons le portugal, soutenons l'espagne. la morale d'un droit inventé de toutes pièces et qui a perdu sa légitimité démocratique n'a pas sa place lorsque l'on considère l'avenir d'un pays. & 1032 & medium & Medium & Socio-Economic & NA & NA & 2016-10-17 & 2016 & 2 & ECO
Frame & low-medium & National & +1000 & -0.7708104 & -0.8786862 & 0.6041414 & -0.2983228 & 0.6486612 & 0.0 & -0.0275917 & -0.0818024 & Payer & European & European & European & European|ECO & Positive\\
\addlinespace
France & http://www.leparisien.fr/economie/business/commerces-une-aide-gratuite-pour-conquerir-des-clients-sur-internet-14-11-2018-7942482.php & 185 & Le Parisien & Private/Non-Public & Online and Offline & Regional/Local & medium = CP is important part of story & Economic development & Positive & EU + Subnational & No myth & NA & NA & NA & NA & NA & NA & NA & NA & France & commerces : une aide gratuite pour conquérir des clients sur internet & 2018-11-14 & fonds social européen & plus de 500 boutiques franciliennes ont fait le grand saut numérique grâce aux experts de la chambre de commerce et d'industrie paris ile-de-france. epiciers, coiffeurs, fleuristes... plus de 500 commerces franciliens de proximité ont bénéficié d'un accompagnement sur-mesure de la cci paris ile-de-france pour doper leur activité grâce au numérique, avec le soutien du conseil régional et de l'union européenne (via le fonds social européen). comment ? en créant une page professionnelle sur facebook ou, pour les plus aguerris, en se lançant dans la vente en ligne via leur site internet ou une plate-forme comme amazon... ce programme gratuit, qui mobilise 50 conseillers de la cci, s'étale sur plusieurs semaines. un diagnostic : " pas besoin de se déplacer, un expert vient à vous ", explique gérald barbier, président de la commission commerce de la cci paris idf. le premier rendez-vous, d'une demi-journée, " permet de connaître l'activité du commerçant, de faire le point sur ses souhaits et les limites imposées par son activité, ses capacités et besoins ", illustre-t-il. une deuxième entrevue vise ensuite à décider, conjointement, des actions à mettre en place : créer un compte instagram (très prisé des coiffeurs) ou pinterest (idéal pour la décoration) par exemple, améliorer la visibilité de son commerce avec google my business... " certains commerçants sont réfractaires au changement, constate gérald barbier. avec d'autres, il n'y a pas de débat, car ils voient le numérique comme une opportunité de développement. " du coaching : " ce qui les intéresse le plus, c'est la notoriété sur internet pour séduire de potentiels clients et les fidéliser ", précise-t-il. ainsi, lors du troisième rendez-vous, toujours dans les locaux du commerçant, l'expert de la cci l'accompagne et le conseille dans la mise en place des actions choisies : construire sa page facebook, améliorer sa visibilité sur les moteurs de recherche... ce coaching se déroule sur une journée. des ateliers en mairie : en parallèle, le commerçant peut participer à des ateliers numériques qui réunissent pendant deux heures jusqu'à 15 chefs d'entreprise. ils sont organisés dans la cci de son département ou à la mairie de sa commune. " au regard des problèmes de vacance des commerces de centres-villes, les municipalités sont de plus en plus sensibles au numérique. elles ont compris que le digital était un moyen de développer le commerce local ", conclut gérald barbier. mamar cherif, gérant de l'épicerie o'bio, à sucy-en-brie (val-de-marne) a participé à l'événement connect street by les digiteurs, organisée dans sa commune le 23 octobre par la cci 94. pourquoi avez-vous participé à l'opération connect street by les digiteurs? mamar cherif. j'ai 40 ans et je ne suis pas du tout un geek... donc l'événement est bien tombé ! depuis, j'ai eu plusieurs rendez-vous avec une conseillère de la cci. elle m'a aidé à créer une page facebook et appris à utiliser google business, qui permet de rendre visible son magasin. chaque semaine, je publie des photos de l'intérieur et des produits pour être bien référencé. et ça marche ? j'ai eu trois clients grâce à mon compte professionnel instagram ! je poste des photos environ tous les trois jours. j'utilise le hashtag (ndlr : mot-clé) de la ville de sucy car les habitants aiment savoir ce qu'il se passe près de chez eux. en quatre mois, j'ai bâti une communauté de 91 personnes. qu'on parle de mon épicerie sur les réseaux, c'est pas mal. quels sont les points forts de cet accompagnement ? il est gratuit. vous êtes suivi par un seul conseiller. et c'est lui qui vient dans votre commerce. en deux semaines, je me sens plus à l'aise sur le portable et les réseaux sociaux. le numérique, c'est un petit plus que je n'imaginais pas. surtout quand on vient d'ouvrir un commerce comme moi. & 661 & medium & Medium & Socio-Economic & NA & NA & 2018-11-14 & 2018 & 3 & ECO
Frame & low-medium & Regional & 500-1000 & -0.7708104 & -0.8786862 & 0.6041414 & -0.2983228 & 0.6486612 & 0.0 & -0.0275917 & -0.0818024 & Payer & Domestic & European & Mixed & Domestic|ECO & Positive\\
France & http://www.lemonde.fr/europe/article/2017/11/24/paris-adoucit-le-ton-avec-la-pologne-malgre-les-divergences\_5219731\_3214.html & 159 & Le Monde.fr & Private/Non-Public & Online and Offline & National & very low = CP mentioned once & Political leverage & Factual & Other country & No myth & NA & NA & NA & NA & NA & NA & NA & NA & France & paris adoucit le ton avec la pologne malgré les divergences & 2017-11-24 & fonds de cohésion & emmanuel macron encourage la première ministre polonaise, beata szydlo, à respecter les traités européens au moment où la réforme de la justice revient au parlement. emmanuel macron et la première ministre polonaise, beata szydlo, ont tenté d'aplanir leurs différends, lors de leur rencontre à paris, jeudi 23 novembre. fini les propos aigres-doux à distance, les deux dirigeants ont chanté ensemble les louanges du dialogue. pour le président français, qui se rendra en pologne en 2018, il ne s'agit pas de nier les désaccords, mais d'essayer de trouver aussi des compromis (sur la réforme du transport routier en europe) et de poursuivre une relation bilatérale normale, dans l'énergie ou la défense (négociations pour l'achat de sous-marins). emmanuel macron a veillé à ne pas utiliser de mots blessants pour parler de la réforme de la justice en cours qui suscite une vive inquiétude à bruxelles, même si la divergence est totale entre paris et varsovie. " la france aura une position très simple, elle suivra les travaux de la commission européenne ", a dit le président français. depuis le retour au pouvoir en 2015 du parti ultraconservateur droit et justice (pis), la commission s'inquiète des risques sur l'indépendance de la justice, qui a commencé par la reprise en main du tribunal constitutionnel. " s'il est avéré que ce qui est fait n'est pas conforme aux textes européens, tout le monde en tirera les conséquences ", a expliqué m. macron, en précisant qu'il n'y aurait " plus de problèmes " si les modifications de la réforme judiciaire sont " conformes " aux traités européens. il n'a toutefois pas réitéré la menace de lier les fonds de cohésion européens au respect de l'etat de droit. atmosphère houleuse beata szydlo jure que " tout ce qui est fait en pologne se passe dans le respect des règles, valeurs et droits " de l'europe. mais le compromis trouvé entre le président, andrzej duda, et le président du parti droit et justice, jaroslaw kaczynski, pour une réforme de la justice inquiète les milieux judiciaires en pologne. la veille de la visite de mme... & 352 & very low & Low & Power & NA & NA & 2017-11-24 & 2017 & 2 & POL
Frame & v.low & National & <500 & -0.7708104 & -0.8786862 & 0.6041414 & -0.2983228 & 0.6486612 & 0.0 & -0.0275917 & -0.0818024 & Payer & European & European & European & European|POL & Neutral\\
France & https://www.lemonde.fr/economie/article/2018/05/25/wopke-hoekstra-il-n-y-pas-d-analogie-entre-la-grece-et-l-italie\_5304749\_3234.html & 134 & Le Monde.fr & Private/Non-Public & Online and Offline & National & very low = CP mentioned once & Political leverage & Negative & Other country & No myth & NA & NA & NA & NA & NA & NA & NA & NA & France & wopke hoekstra : " il n'y pas d'analogie entre la grèce et l'italie " & 2018-05-25 & fonds de cohésion & ministre des finances des pays-bas depuis octobre 2017, wopke hoekstra, 42 ans, membre du parti chrétien-démocrate cda, a été cadre dans le secteur privé (shell et mac kinsey) avant de devenir l'une des étoiles montantes de la politique néerlandaise. il se dit " admirateur " d'emmanuel macron mais n'approuve pas - loin de là - tous ses projets pour l'europe et sa relance. comment jugez-vous le programme du nouveau gouvernement italien, particulièrement et singulièrement en ce qui concerne l'euro ? j'espère que, comme je l'ai toujours fait, les dirigeants prennent des décisions dans l'intérêt de leurs citoyens, se concentrent sur l'équilibre budgétaire, sur la croissance. surtout en ce moment où l'on assiste au retour de cette croissance. c'est d'ailleurs ce que j'admire le plus chez emmanuel macron, qui profite de la bonne conjoncture pour effectuer des réformes. l'arrivée de ce gouvernement " anti système " ne vous fait-elle pas craindre une nouvelle crise " à la grecque " ? la crise grecque était très spécifique et est survenue en pleine crise de l'eurozone. le budget et les banques grecques étaient aussi dans une situation très différente de celle de l'italie. donc, non, pas d'analogie. la commission européenne devrait-elle envisager d'assouplir les règles du pacte de stabilité, comme l'y invitent certains ? il est complètement évident que les pays de l'eurozone se sont engagés à respecter une série de règles et de pratiques, non pas parce que c'était une demande de bruxelles mais parce qu'ils comprenaient tous que c'était dans leur intérêt. la commission doit rester un gardien juste, strict, cohérent de ces règles. emmanuel macron a proposé des réformes pour avancer dans l'intégration de la zone euro, vous les soutenez ? j'admire vraiment les réformes qu'a engagées le gouvernement français et sa volonté de rééquilibrer le budget. j'admire aussi la clarté et la ténacité de bruno le maire, avec qui je partage beaucoup de points de vue sur l'achèvement de l'union bancaire notamment. la grande question, c'est l'ordre des priorités. nous pouvons envisager qu'une assurance commune des dépôts soit un jour constituée. mais il est crucial que nous soyons très stricts sur les conditions de participation des banques à ce mécanisme. je distingue parfaitement la logique de cette assurance, mais nous devons d'abord nous attaquer à la réduction des risques dans les bilans des banques et ne commencer à partager la solidarité qu'une fois ces réductions des risques achevées. il ne peut être question de mener les deux en parallèle. comme la france, nous sommes aussi d'avis que l'union doit disposer d'un budget commun adapté au xxie siècle. plus restreint et modernisé. et donc bien plus modeste que celui que la commission a proposé au début du mois de mai ? les britanniques contribuent à hauteur de 15 \% du budget commun actuel. et, pour nous les néerlandais, le brexit est un sujet très sensible : la grande-bretagne est l'un de nos trois principaux partenaires. nous serons donc parmi les plus affectés par leur départ, avec l'irlande. et la logique de la commission serait de compenser simplement ce trou budgétaire et même d'aller plus loin, obligeant notamment les néerlandais à contribuer beaucoup plus ? il ne serait ni juste ni raisonnable de nous demander davantage alors que nous sommes l'un des premiers des contributeurs nets au budget actuel. ce qu'il y a actuellement sur la table n'est donc tout simplement pas acceptable. etes-vous prêt, avec la france à vous battre pour éviter des coupes dans la politique agricole commune ? en france, on comprend, je pense, qu'un budget modernisé signifie plus d'argent pour des enjeux sensibles, comme la migration et la protection des frontières, la défense, l'environnement, l'innovation. et que, pour tous ces sujets prioritaires, on doit ajuster le budget. la commission propose de lier l'accès aux fonds de cohésion au respect de l'etat de droit, ce qui cible évidemment des pays de l'est. vous approuvez ? l'europe est un projet économique et un projet de valeurs partagées. il est difficile d'expliquer à nos concitoyens que, d'un côté, nous payons et nous montrons solidaires tandis que, de l'autre, des pays refusent de respecter certaines valeurs fondamentales et d'accepter la réciprocité. je suis indisposé par la discussion sur la migration : des pays qui reçoivent de l'europe l'équivalent de plusieurs pourcents de leur produit intérieur refusent de recevoir des réfugiés. si nous ne rétablissons pas l'équilibre entre solidarité et réciprocité, nous allons éroder les bases, la logique, la compréhension de l'europe dans un pays comme le mien. que pensez-vous du budget de la zone euro évoqué par le président français ? nous sommes à l'unisson sur de nombreux sujets mais je ne vois pas comment nous pourrions accepter un budget de la zone euro. je ne vois pas à quelle logique convaincante il répondrait et comment il pourrait être budgétairement neutre. ni comment on pourrait éviter d'entrer alors dans une logique de transferts financiers. & 862 & very low & Low & Power & NA & NA & 2018-05-25 & 2018 & 3 & POL
Frame & v.low & National & 500-1000 & -0.7708104 & -0.8786862 & 0.6041414 & -0.2983228 & 0.6486612 & 0.0 & -0.0275917 & -0.0818024 & Payer & European & European & European & European|POL & Negative\\
France & https://www.lesechos.fr/monde/europe/0301244370793-avec-le-brexit-la-politique-de-cohesion-europeenne-va-devoir-se-reinventer-2150621.php\#Xtor=AD-6000 & 118 & LesEchos.fr & Private/Non-Public & Online and Offline & National & very high = CP is most important issue + CP is mentioned in title/headline & Financial burden & Factual & EU + Other country & No myth & NA & NA & NA & NA & NA & NA & NA & NA & France & avec le brexit, la politique de cohésion européenne va devoir se réinventer & 2018-02-04 & fonds structurels & la marge budgétaire de l'union va se réduire. l'occasion de repenser des mécanismes essentiels mais devenus trop rigides. menace ou opportunité ? l'avenir de la politique de cohésion, qui représente un tiers du budget de l'union européenne, est en question. alors qu'ils ont à définir un nouveau cadre budgétaire pour la période 2021-2027, les européens savent qu'ils vont devoir s'attaquer à de nouveaux impératifs (migrations, sécurité, défense) tout en se passant de la douzaine de milliards d'euros que leur apportait, annuellement, le royaume-uni. même si l'idée est d'augmenter légèrement la contribution des etats-membres, le commissaire européen en charge de ce sujet, günther oettinger, a été clair : il n'y aura pas de vache sacrée. seules les politiques européennes démontrant leur valeur ajoutée seront préservées. l'occasion, pour l'union, de s'interroger notamment sur ce pilier de la politique européenne que sont les financements aux régions. un mécanisme essentiel les fonds structurels européens ont une efficacité incontestable. selon grégory claeys, du centre bruegel, la recherche prouve que " leur effet sur la croissance des régions est clair, dans la limite des capacités d'absorption des etats ". mais compte tenu de la complexité des mécanismes de financement, cette limite n'est pas négligeable. en france, seuls 16\% des fonds prévus pour la période 2014-2020 avaient été dépensés à la fin 2017. en italie, ce chiffre tombe à 8\%. certes, la mise en oeuvre de ces financements prend toujours du temps et le même grégory claeys note que, " au final, les fonds sont généralement utilisés ". complexité il n'empêche, sur le terrain, ces financements suscitent parfois des exaspérations. frédéric cuvillier, le maire de boulogne-sur-mer, a bien cru que les fonds européens attendus dans le cadre de l'extension du centre de la mer nausicaa ne viendraient jamais. " les règles européennes ont failli changer en cours de route ! ", s'étonne-t-il avant de préciser que l'ensemble de la très longue procédure est " infiniment compliqué, nécessitant l'intervention de cabinets d'experts sur le montage de dossiers. c'est le fond qui devrait l'emporter au lieu de tout faire reposer sur des procédures pointilleuses ", tranche-t-il. a paris, le problème est identifié. une réunion interministérielle est prévue, ce mardi, qui pourrait déboucher sur une saisine du conseil économique, social et environnemental (cese) sur ce sujet afin d'élaborer des pistes permettant d'améliorer le système. dans une interview aux " echos ", le président du cese, patrick bernasconi , plaidait récemment pour un sérieux toilettage des mécanismes de financements. outre les travers de la " suradministration ", il pointait la nécessité d'alléger les règles strictes relatives au cofinancement qui obligent les etats à co-investir. " il faudrait aussi regarder de près les sujets d'entretiens des infrastructures existantes ", ajoute-t-il. la commission planche même à la commission européenne, on reconnaît volontiers que la complexité des mécanismes - conçue pour éviter les utilisations frauduleuses - est excessive. plusieurs idées sont sur la table, au-delà de la simplification. il pourrait être question de rendre ces financements plus flexibles, afin de pouvoir les adapter aux crises. une autre piste serait de renforcer leur lien avec les réformes structurelles des etats-membres. s'y ajoute enfin l'idée, sulfureuse à l'est de l'union, de conditionner ces financements au respect de quelques principes, notamment relatifs au respect de l'etat de droit. & 569 & very high & High & Values & NA & NA & 2018-02-04 & 2018 & 3 & ECO
Frame & high-very high & National & 500-1000 & -0.7708104 & -0.8786862 & 0.6041414 & -0.2983228 & 0.6486612 & 0.0 & -0.0275917 & -0.0818024 & Payer & European & European & European & European|ECO & Neutral\\
France & http://fr.euronews.com/2018/08/16/pourquoi-salvini-a-tort-d-accuser-bruxelles-du-drame-de-genes?utm\_source=feedburner\&utm\_medium=feed\&utm\_campaign=Feed\%253A\%2Beuronews\%252Ffr\%252Fhome\%2B\%2528euronews\%2B-\%2Bhome\%2B-\%2Bfr\%2529\&utm\_content=FeedBurner & 181 & euronewsfr & Private/Non-Public & Online only & National & very low = CP mentioned once & Infrastructure & Positive & EU + Other country & NA & NA & NA & NA & NA & NA & NA & NA & NA & France & pourquoi salvini a tort d'accuser bruxelles du drame de gênes & 2018-08-17 & fonds structurels & il y a quelques mois, emmanuel macron dénonçait les dirigeants européens qui accusaient systématiquement bruxelles quand tout allait mal chez eux. "accuser bruxelles ou strasbourg de tous les maux. continuer à faire cela, c'est décider d'avoir un jeu de dupes qui sera peut-être plus confortable pour chacun d'entre nous mais qui nous conduira à ne résoudre aucun problème", disait-il devant le parlement européen. quelques heures après l'effondrement mortel d'un pont à gênes, le vice-premier ministre italien matteo salvini semblait choisir cette option, en déclarant : "alors que les contraintes européennes nous empêchent de dépenser de l'argent pour sécuriser les écoles de nos enfants et les routes qu'empruntent nos travailleurs, nous ferons tout pour la sécurité des italiens." salvini, le chef du parti d'extrême droite la ligue, se réfère au pacte budgétaire de l'ue qui impose des restrictions sur les dépenses pour éviter que la dette ne devienne incontrôlable. l'italie est l'un des pays les plus endettés de l'ue. alors, a-t-il raison ? pas vraiment, répond à euronews veronica vecchi, professeur de gestion publique à l'université bocconi de milan. elle indique que si les restrictions de dépenses sont une réalité en italie, rome avait choisi de les faire peser sur les régions en réduisant leurs budgets. selon le professeur vecchi, cela a conduit les autorités locales à chercher à tout prix des liquidités, en rendant par exemple les partenariats public-privé plus attractifs. ainsi, c'est un accord conclu avec le secteur privé qui devrait être examiné à la lumière de la catastrophe de gênes, a-t-elle ajouté. autostrade per l'italia, entité du groupe d'infrastructure atlantia, était responsable du contrat d'entretien du pont, vieux de 50 ans. "dans ce cas spécifique, il s'agit d'un contrat relevant du ministère de l'infrastructure", a-t-elle déclaré. "le vrai problème avec de tels contrats est la négociation de plans d'investissement avec les entrepreneurs. le tarif fixé par autostrade per l'italia doit être suffisant pour couvrir la maintenance." autostrade affirme avoir effectué des contrôles réguliers et poussés de la structure et qu'elle n'avait décelé aucun problème. cependant, le ministre italien des transports a appelé à la démission de la direction de l'entreprise. en réponses aux déclarations de m. salvini, la commission européenne a déclaré que l'italie avait été l'un des principaux bénéficiaires de la flexibilité fiscale de bruxelles, qui lui avait permise "d'investir et de dépenser beaucoup plus ces dernières années". "sur la période 2014-2020, l'italie devrait recevoir environ 2,5 milliards d'euros au titre des fonds structurels et d'investissement européens pour des investissements dans des infrastructures de réseau, telles que les routes ou le rail", a poursuivi le communiqué. "en avril 2018, la commission a également approuvé un plan d'investissement pour les autoroutes italiennes, qui permettra d'investir environ 8,5 milliards d'euros, y compris dans la région de gênes". & 503 & very low & Low & Socio-Economic & NA & NA & 2018-08-17 & 2018 & 3 & ECO
Frame & v.low & National & 500-1000 & -0.7708104 & -0.8786862 & 0.6041414 & -0.2983228 & 0.6486612 & 0.0 & -0.0275917 & -0.0818024 & Payer & European & European & European & European|ECO & Positive\\
\addlinespace
France & http://lexpansion.lexpress.fr/actualites/1/actualite-economique/l-ue-leve-la-menace-de-suspension-des-fonds-structurels-pour-l-espagne-et-le-portugal\_1851020.html & 178 & LExpansion.com & Private/Non-Public & Online and Offline & National & high = CP is most important issue in story (can also cover other issues) & Political leverage & Factual & EU & 1.Poor regions funded only & NA & NA & NA & NA & NA & NA & NA & NA & France & l'ue lève la menace de suspension des fonds structurels pour l'espagne et le portugal & 2016-11-16 & fonds structurels & bruxelles - la commission européenne a levé mercredi la menace qui pesait sur l'espagne et le portugal concernant la suspension des fonds structurels européens pour 2017. "nous n'allons pas proposer de suspendre ces fonds. je sais que c'était attendu et c'est évidemment une bonne nouvelle pour les deux pays", a déclaré le commissaire européen aux affaires économiques pierre moscovici. l'espagne et le portugal avaient échappé en juillet à des amendes (qui auraient pu atteindre jusqu'à 0,2\% du pib) pour dérapage budgétaire en 2015, des sanctions qui auraient été sans précédent dans l'histoire de l'ue, mais ils étaient toujours sous la menace de suspension des fonds structurels européens pour 2017. "la commission européenne est arrivée à la conclusion que la procédure engagée pour déficit excessif à l'encontre de ces deux pays devait être suspendue. par conséquent, il n'y a plus de faits qui susciteraient une proposition de la commission de suspendre une partie des fonds structurels européens et il n'y aura pas de telle proposition", écrit l'exécutif européen, dans un communiqué. les fonds structurels européens visent en particulier à réduire les écarts de développement en europe en aidant les régions en difficulté. en 2015, le déficit public espagnol avait atteint 5,1\% du pib, un chiffre bien au-dessus du plafond de 3\% fixé par le pacte de stabilité et des objectifs de la commission de 4,2\%. quant au portugal, il a affiché un déficit public de 4,4\% du pib l'an passé alors que l'objectif fixé était de repasser sous les 3\%. & 267 & high & High & Power & NA & NA & 2016-11-16 & 2016 & 2 & POL
Frame & high-very high & National & <500 & -0.7708104 & -0.8786862 & 0.6041414 & -0.2983228 & 0.6486612 & 0.0 & -0.0275917 & -0.0818024 & Payer & European & European & European & European|POL & Neutral\\
France & http://www.lefigaro.fr/flash-actu/2015/09/15/97001-20150915FILWWW00059-migrants-berlin-suggere-de-reduire-les-fonds-de-l-ue-aux-pays-opposes-aux-quotas.php & 126 & Le Figaro.fr & Private/Non-Public & Online and Offline & National & very high = CP is most important issue + CP is mentioned in title/headline & Political leverage & Factual & Other country & No myth & NA & NA & NA & NA & NA & NA & NA & NA & France & rã©fugiã©s : l'allemagne suggã¨re de couper les fonds de l'ue aux pays opposã©s aux quotas & 2015-09-15 & fonds structurels & le ministre de l'intérieur allemand, thomas de maizière, a évoqué ce matin la possibilité de réduire les fonds structurels versés par l'union européenne aux pays qui rejettent l'idée de quotas de répartition des réfugiés, après l'échec lundi d'une réunion européenne. "nous devons parler de moyens de pression", a-t-il dit à la chaîne allemande zdf. les pays qui refusent la répartition par quotas "sont des pays qui reçoivent beaucoup de fonds structurels" européens, a-t-il justifié, trouvant "juste qu'ils reçoivent moins de moyens", et disant reprendre une proposition du président de la commission européenne, jean-claude juncker. hier soir, les 28 etats membres de l'ue, réunis en urgence à bruxelles, ne sont pas parvenus à un accord sur la répartition contraignante de 120.000 réfugiés. selon le ministre français de l'intérieur, bernard cazeneuve, "un certain nombre de pays ne veulent pas adhérer à ce processus de solidarité", parmi lesquels la hongrie, la pologne, la république tchèque et la slovaquie. l'allemagne, qui s'attend désormais à recevoir entre 800.000 et un million de demandeurs d'asile selon les sources, a réintroduit dimanche des contrôles à ses frontières, suspendant de facto l'accord de libre-circulation européen. "nous ne voulons pas que (ces contrôles) soient une solution durable, mais ils ne sont pas non plus passagers", a prévenu thomas de maizière. lire aussi : " crise migratoire : ce que prévoit le plan de quotas européen & 243 & very high & High & Power & NA & NA & 2015-09-15 & 2015 & 1 & POL
Frame & high-very high & National & <500 & -0.7708104 & -0.8786862 & 0.6041414 & -0.2983228 & 0.6486612 & 0.0 & -0.0275917 & -0.0818024 & Payer & European & European & European & European|POL & Neutral\\
France & http://reunion.orange.fr//actu/reunion/mise-en-place-d-un-plan-anglais-dans-les-ecoles.html & 149 & Orange.re & Private/Non-Public & Online only & Regional/Local & very low = CP mentioned once & Social awareness/inclusion & Positive & EU + Subnational & No myth & NA & NA & NA & NA & NA & NA & NA & NA & France & mise en place d'un "plan anglais" dans les écoles & 2017-07-12 & fonds social européen & dans le but de renforcer l'apprentissage de la langue de shakespeare, la ville de saint-paul va mettre en place un " plan anglais " de 2017 à 2020. l'anglais est aujourd'hui la principale langue de communication internationale. mais force est de constater que la population réunionnaise est peu confrontée à cette langue alors que nos voisins de la zone baignent dans cette culture. dans le cadre du " plan anglais " développé par l'état, la ville de saint-paul souhaite s'appuyer sur le dispositif pour renforcer l'apprentissage de l'anglais dans les écoles de la commune, en période scolaire et périscolaire. le projet qui s'étalera sur la période 2017/2020 prendra forme au travers de plusieurs objectifs : étendre d'abord le " plan anglais " depuis la grande section, accompagner ensuite le programme ti guide mafate et le décliner sur les écoles de saint-paul. le but est de permettre aux jeunes enfants de s'exprimer en anglais sur l'histoire, la vie, la présentation des lieux. il est envisagé aussi, toujours dans le cadre du projet, de créer une journée " english day ", de réaliser des projections de films en anglais dans les écoles en temps scolaire, de mettre en place des ateliers en anglais, et enfin, d'accompagner le développement de projets collaboratifs etwinning en anglais. le coût global de l'opération est évalué à un peu plus de 2 millions d'euros dont 1,6 millions financés par le fse (fonds social européen). par ailleurs, conformément aux engagements pour la jeunesse, le conseil municipal a voté la mise en place d'un conseil communal des enfants et des jeunes. véritable instance citoyenne, ce conseil permettra au " conseiller jeune " de découvrir le fonctionnement démocratique des institutions, de pratiquer le civisme et la citoyenneté, de participer à la vie locale par l'élaboration de projets collectifs, de préparer et de réaliser des actions concrètes. et pour la collectivité, cette action permettra la mise en œuvre de projets cohérents en direction de la jeunesse. le conseil sera composé des élèves de cm1 et cm2 des écoles de la commune. ils seront 49 " conseillers jeunes " désignés par élection au scrutin majoritaire à un tour, à bulletins secrets. un temps d'élection qui représentera une belle opportunité d'échanges avec les plus jeunes et la transmission de l'un des fondements de notre système démocratique : l'élection de représentants librement choisis. & 400 & very low & Low & Socio-Economic & NA & NA & 2017-07-12 & 2017 & 2 & ECO
Frame & v.low & Regional & <500 & -0.7708104 & -0.8786862 & 0.6041414 & -0.2983228 & 0.6486612 & 0.0 & -0.0275917 & -0.0818024 & Payer & Domestic & European & Mixed & Domestic|ECO & Positive\\
\bottomrule
\end{tabular}


\end{document}
